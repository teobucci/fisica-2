%!TEX root = ../main.tex
\chapter{Costanti ed unità di misura}

\section*{Costanti fisiche}

\begin{equation*}
	\begin{array}{ l l }
		\text{Numero di Avogadro} & N_A =6.022\times 10^{23} \ mol^{-1}\\
		\text{Costante dei gas perfetti} & R=8.314\ J/( K\ mol)\\
		\text{Costante di Boltzmann} & k_B =R/N_A =1.381\times 10^{-23} \ J/K\\
		\text{Costante gravitazionale} & G=6.672\times 10^{-11} \ N\ m^2 /kg^2
	\end{array}
\end{equation*}

% FIS2
\begin{equation*}
\begin{array}{ l l }
\text{Carica elementare} & e=1.602\times 10^{-19} \ C\\
\text{Massa dell'elettrone} & m_e =9.109\times 10^{-31} \ kg\\
\text{Costante dielettrica del vuoto} & \varepsilon_0 =8.854\times 10^{-12} \ C^2 /\left( N\ m^2\right)\\
\text{Permeabilita magnetica del vuoto} & \mu_0 =4\pi \times 10^{-7} \ N/A^2\\
\text{Velocita della luce nel vuoto} & c=2.998\times 10^8 \ m/s\\
\text{Raggio di Bohr} & a_0 =5.292\times 10^{-11} \ m\\
\text{Magnetone di Bohr} & \mu_B =9.274\times 10^{-24} \ J/T
\end{array}
\end{equation*}

\section*{Prefissi per le potenze di dieci}

\begin{center}

	\begin{table}[!h]
	\centering
	\begin{tabular}{ccc|ccc}
		\textbf{Potenza} & \textbf{Prefisso} & \textbf{Abbreviazione} & \textbf{Potenza} & \textbf{Prefisso} & \textbf{Abbreviazione} \\
		\hline
		$\displaystyle 10^{-18}$ & atto & a & $\displaystyle 10^1$ & deca & da \\
		$\displaystyle 10^{-15}$ & femto & f & $\displaystyle 10^2$ & etto & h \\
		$\displaystyle 10^{-12}$ & pico & p & $\displaystyle 10^3$ & chilo & k \\
		$\displaystyle 10^{-9}$ & nano & n & $\displaystyle 10^6$ & mega & M \\
		$\displaystyle 10^{-6}$ & micro & $\displaystyle \mu $ & $\displaystyle 10^9$ & giga & G \\
		$\displaystyle 10^{-3}$ & milli & m & $\displaystyle 10^{12}$ & tera & T \\
		$\displaystyle 10^{-2}$ & centi & c & $\displaystyle 10^{15}$ & peta & P \\
		$\displaystyle 10^{-1}$ & deci & d & $\displaystyle 10^{18}$ & exa & E \\
	\end{tabular}
	\end{table}
\end{center}

\newpage\section*{Grandezze ed unità di misura impiegate nel testo}

\begin{center}
	\textbf{Grandezze fondamentali nel Sistema Internazionale}
\end{center}

\begin{table}[!h]
	\centering
	\begin{tabular}{l|l|l}
		\hline
		\textbf{Grandezza} & \textbf{Unità di misura} & \textbf{Simbolo} \\
		\hline
		Lunghezza & metro & $m$ \\
		Massa & chilogrammo & $kg$ \\
		Intervalli di tempo & secondo & $s$ \\
		Temperatura assoluta & kelvin & $K$ \\
		Quantità di sostanza & mole & $mol$ \\
		Angolo & radiante & $rad$ \\
	\end{tabular}
\end{table}

\begin{center}
	\textbf{Grandezze derivate}
\end{center}

\begin{table}[!h]
	\centering
	\begin{tabular}{l|c|c|c}
		\hline
		\textbf{Grandezza} & \textbf{Unità} & \textbf{Simbolo} & \textbf{Espressioni equivalenti} \\
		\hline
		Velocità $\displaystyle \vec{v}$ & - & $\displaystyle m/s$ & - \\
		Velocità angolare $\displaystyle \vec{\omega }$ & - & $\displaystyle rad/s$ & - \\
		Frequenza $\displaystyle \nu $ & hertz & $\displaystyle Hz$ & $\displaystyle 1/s$ \\
		Accelerazione $\displaystyle \vec{a}$ & - & $\displaystyle m/s^2$ & - \\
		Accelerazione angolare $\displaystyle \vec{\alpha }$ & - & $\displaystyle rad/s^2$ & - \\
		Forza $\displaystyle \vec{F}$ & newton & $\displaystyle N$ & $\displaystyle kg\ m/s^2 ;\ J/m$ \\
		Momento meccanico $\displaystyle \vec{M}$ & - & $\displaystyle N\ m$ & $\displaystyle kg\ m^2 /s^2$ \\
		Momento angolare $\displaystyle \vec{L}$ & - & $\displaystyle kg\ m^2 /s$ & - \\
		Quantità di moto $\displaystyle \vec{p}$ & - & $\displaystyle kg\ m /s$ & - \\
		Momento d'inerzia $\displaystyle I$ & - & $\displaystyle kg\ m^2$ & - \\
		Energia $\displaystyle E,U$; lavoro $\displaystyle L$; calore $\displaystyle Q$ & joule & $\displaystyle J$ & $\displaystyle N\ m;\ kg\ m^2 /s^2$ \\
		Potenza $\displaystyle P$ & watt & $\displaystyle W$ & $\displaystyle J/s;\ kg\ m^2 /s^3$ \\
		Pressione $\displaystyle p$ & pascal & $\displaystyle Pa$ & $\displaystyle N/m^2 ;\ kg/m\ s^2$ \\
		Densità per unità di volume $\displaystyle \rho $ & - & $\displaystyle kg/m^3$ & - \\
		Entropia $\displaystyle S$ & - & $\displaystyle J/K$ & - \\
	\end{tabular}
\end{table}

\newpage

\begin{center}
	\textbf{Altre unità di misura}
\end{center}

\begin{table}[!h]
	\centering
	\begin{tabular}{l|l|l}
		\hline
		\textbf{Grandezza} & \textbf{Unità di misura} & \textbf{Equivalenza nel S.I.} \\
		\hline
		Velocità & chilometro-ora ($\displaystyle km/h$) & $\displaystyle 0.2778\ m/s$ \\
		Forza & chilogrammo-forza ($\displaystyle kg_f$) & $\displaystyle 9.81\ N$ \\
		Energia & chilowatt-ora ($\displaystyle kWh$) & $\displaystyle 3.6\times 10^6 \ J$ \\
		Volume & litro ($\displaystyle L$) & $\displaystyle 10^{-3} \ m^3$ \\
		Pressione & $\displaystyle bar$ & $\displaystyle 10^5 \ Pa$ \\
		'' & $\displaystyle mm\ Hg$ & $\displaystyle 133.3\ Pa$ \\
		'' & atmosfera ($\displaystyle atm$) & $\displaystyle 1.013\times 10^5 \ Pa$ \\
		Calore & caloria ($\displaystyle cal$) & $\displaystyle 4.18\ J$ \\
	\end{tabular}
\end{table}

\begin{center}
	\textbf{Equivalenze utili}
\end{center}

Detta $\displaystyle t$ la temperatura di un oggetto misurata nella scala Celsius e $\displaystyle T$ la corrispondente temperatura assoluta, vale la relazione:

\begin{equation*}
	t( \degree \text{C}) =T(\text{K}) +273.15
\end{equation*}