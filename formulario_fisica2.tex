\documentclass[10pt,a4paper]{article}

\usepackage{amsfonts,amsmath}

\usepackage{multicol}
\setlength{\columnsep}{1cm}

\usepackage[T1]{fontenc}
\usepackage[utf8]{inputenc}
\usepackage[italian]{babel}
\usepackage{siunitx}
\usepackage[none]{hyphenat}

% PARAGRAFI, INTERLINEA E MARGINE
\emergencystretch 3em % PER EVITARE CHE IL TESTO VADA OLTRE I MARGINI
\parindent 0ex % TOGLIE INTENDAMENTO PARAGRAFI
% \setlength{\parindent}{4em} % CAMBIA INDENTAMENTO PARAGRAFI
\setlength{\parskip}{0.2em} % CAMBIA SPAZIO TRA PARAGRAFI (POSSO METTERE ANCHE 1em)
% \renewcommand{\baselinestretch}{1.5} % CAMBIA INTERLINEA

\usepackage[
	left=1.2cm,
	right=1.2cm,
	top=2cm,
	bottom=1cm
]{geometry} % CAMBIA MARGINE DEL DOCUMENTO

% HEADER
\usepackage{fancyhdr}
\fancyhead{} % PULISCI HEADER
\fancyfoot{} % PULISCI FOOTER
\fancyhead[L]{\small{\textsc{Formulario di Fisica Sperimentale 2}}}
\fancyhead[R]{\small{\textsc{Copyright Teo Bucci}}}

\allowdisplaybreaks

\begin{document}

\pagestyle{fancy}

\begin{multicols}{2}

\section*{Campo Elettrico}

Legge di Coulomb $\vec{F}_{21} =k\frac{q_{1} q_{2}}{r^{2}}\vec{u}_{r}$

Campo elettrico $\vec{E} (P)=\sum _{i}\frac{Q\vec{r}_{i}}{4\pi \varepsilon _{0} r^{3}_{i}}$

Campo elettrico di un filo rettilineo infinito $\vec{E} =\frac{\lambda }{2\pi \varepsilon _{0} a}\vec{u}_{n}$

Componenti del potenziale $\frac{dx}{V_{x}} =\frac{dy}{V_{y}} =\frac{dz}{V_{z}}$

Teorema di Gauss $\Phi _{\Sigma } (\vec{E} )=\frac{Q^{\text{interna a} \Sigma }_{\text{tot}}}{\varepsilon _{0}}$

I Equazione di Maxwell (anche in regime tempovariante) $\text{div}\vec{E} =\vec{\nabla } \cdot \vec{E} =\frac{\rho (x,y,z)}{\varepsilon _{0}}$

Divergenza di un campo $\text{div}\vec{v} =\frac{\partial v_{x}}{\partial x} +\frac{\partial v_{y}}{\partial y} +\frac{\partial v_{z}}{\partial z}$

Energia potenziale elettrostatica $U_{\text{elettrostatica}} =\frac{Qq}{4\pi \varepsilon _{0} r}$

Potenziale elettrostatico $V(P)=\frac{U(P)}{q} =\frac{Q}{4\pi \varepsilon _{0} r}$

Conservatività del campo elettrico $\int ^{B}_{A}\vec{E} \cdot d\vec{l} =V(A)-V(B)$

Lavoro delle forze di Coulomb $\mathcal{L} =-q\Delta V$

Energia elettrostatica $U_{e} =\frac{1}{2}\sum _{i\neq j}\frac{q_{i} q_{j}}{4\pi \varepsilon _{0} r_{ij}}$

Campo elettrico come gradiente del potenziale \\$-dV=\vec{E} \cdot d\vec{l} \Rightarrow \vec{E} =-\text{grad} V=-\vec{\nabla } V$

II Equazione di Maxwell (solo in regime stazionario) $\text{rot}\vec{E} =0$

Equazione di Poisson $\nabla ^{2} V=\frac{\partial ^{2} V}{\partial x^{2}} +\frac{\partial ^{2} V}{\partial y^{2}} +\frac{\partial ^{2} V}{\partial z^{2}} =-\frac{\rho (x,y,z)}{\varepsilon _{0}}$

Condizioni al contorno di $\vec{E}$: $E_{t1} =E_{t2} ,\ \ E_{n1} -E_{n2} =\frac{\sigma }{\varepsilon _{0}}$

Momento di dipolo $\vec{p} =q\ \vec{d}$

Potenziale di dipolo a grandi distanze $V(P)=\frac{\vec{p} \cdot \vec{u}_{r}}{4\pi \varepsilon _{0} r^{2}}$

Interazione fra un dipolo e un campo elettrico
\begin{itemize}
\item Campo elettrico uniforme $\vec{R} =0,\vec{M} =\vec{p} \times \vec{E}$
\item Campo elettrico non uniforme $\vec{R} =(\vec{p} \cdot \vec{\nabla }) E_{x}\vec{u}_{x} +(\vec{p} \cdot \vec{\nabla }) E_{y}\vec{u}_{y} +(\vec{p} \cdot \vec{\nabla }) E_{z}\vec{u}_{z} =(\vec{p} \cdot \vec{\nabla })\vec{E}$
\end{itemize}

Energia elettrostatica di un dipolo immerso in un campo elettrico: \\$\vec{R} =\vec{\nabla } (\vec{p} \cdot \vec{E} )$ per campi conservativi, \\$\vec{R} =(\vec{p} \cdot \vec{\nabla } )\vec{E}$ per tutti i campi.

Teorema di Coulomb $\vec{E}_{1} =\frac{\sigma }{\varepsilon _{0}}\vec{n}$

Potere delle punte $\frac{\sigma _{1}}{\sigma _{2}} =\frac{R_{2}}{R_{1}}$

Capacità del conduttore $\frac{Q}{V_{0}} =C$

Capacità del condensatore $C=\frac{Q}{\Delta V}$. Se piano $C=\frac{Q}{\Delta V} =\frac{Q}{\frac{Qd}{\varepsilon _{0} A}} =\frac{\varepsilon _{0} A}{d}$
\begin{itemize}
\item Condensatori in parallelo \\$C_{eq} =\frac{Q_{\text{tot}}}{\Delta V} =\frac{\Delta V\sum ^{n}_{i=1} C_{i}}{\Delta V} =\sum ^{n}_{i=1} C_{i}$
\item Condensatore in serie \\$\frac{1}{C_{eq}} =\sum ^{n}_{i=1}\frac{1}{C_{i}}$
\end{itemize}

Energia elettrostatica di un condensatore carico $U_{e} =\frac{1}{2} Q\Delta V=\frac{1}{2} C\Delta V^{2} =\frac{1}{2}\frac{Q^{2}}{C}$

Legge di Felici $Q(t)=\frac{1}{\mathrm{R}} [\Phi (0)-\Phi (t)]$

Densità di energia elettrostatica \\$U_{e} =\int _{\text{tutto lo spazio}}\frac{1}{2} \varepsilon _{0} E^{2} d\tau $, $u_{e} =\frac{1}{2} \varepsilon _{0} E^{2}$

Forza tra le armature di un condensatore: pressione elettrostatica
\begin{itemize}
\item Carica costante $\vec{F} =-\vec{\nabla } U_{e}$

Pressione elettrostatica $p_{e} =\frac{\sigma ^{2}}{2\varepsilon _{0}}$
\item Potenziale costante $\vec{F} =+\vec{\nabla } U_{e}$
\end{itemize}

Costante dielettrica relativa $\varepsilon _{r} =\frac{C_{d}}{C_{0}}$

Suscettività elettrica $\chi =\varepsilon _{r} -1$

Densità di carica di polarizzazione $\sigma _{p} =\vec{P} \cdot \vec{n}$, $\rho _{p} =-\text{div}\vec{P}$

Equazioni generali dell'elettrostatica $\vec{P} =\varepsilon _{0} (\varepsilon _{r} -1)\vec{E}_{d} =\varepsilon _{0} \chi \vec{E}_{d}$, $\vec{D} =\varepsilon _{0}\vec{E} +\vec{P} =\varepsilon _{0} \varepsilon _{r}\vec{E} =\varepsilon \vec{E}$

Teorema di Gauss per $D$ $\Phi _{\Sigma } (\vec{D} )=Q^{\text{int. a} \Sigma }_{\text{libere}}$

I Equazione di Maxwell in presenza di dielettrici $\vec{\nabla } \cdot \vec{E} =\frac{\rho _{l}}{\varepsilon _{0} \varepsilon _{r}} =\frac{\rho _{l}}{\varepsilon }$

Discontinuità dei campi sulla superficie di separazione tra due dielettrici $D_{n1} -D_{n2} =\sigma _{l}$, $E_{n1} -E_{n2} =\frac{\sigma _{l} +\sigma _{2} -\sigma _{1}}{\varepsilon _{0}}$

Energia elettrostatica nei dielettrici \\$U_{e} =\frac{1}{2}\int _{\text{tutto lo spazio}}\vec{D} \cdot \vec{E} \ d\tau $, $u_{e} =\frac{1}{2}\vec{D} \cdot \vec{E}$
\section*{Corrente elettrica nei conduttori}

Intensità di corrente elettrica $dI=\frac{dQ}{dt} =nq\vec{v}_{d} \cdot \vec{n} \ dS$

Densità di corrente elettrica \\$\vec{J} =nq\vec{v}_{d} \Longrightarrow dI=\vec{J} \cdot \vec{n} \ dS=d\Phi (\vec{J} )\Rightarrow I=\Phi (\vec{J} )$

Legge di continuità della corrente elettrica $\text{div}\vec{J} =-\frac{\partial \rho _{l}}{\partial t}$

Regime stazionario $\text{div}\vec{J} =0$

Leggi di Kirchhoff $\sum _{i} I_{i} =0,\ \ \sum _{i} \Delta V_{i} =0$

Legge di Ohm $\Delta V=RI$

Legge di Ohm in forma locale $\vec{E} =\rho \vec{J} \ \ \ \ \vec{J} =\sigma \vec{E}$

Resistenza di un conduttore ohmico di forma generica $R=\int ^{B}_{A}\frac{\rho dl}{S}$
\begin{itemize}
\item Resistori in serie $R_{eq} =\sum ^{N}_{i} R_{i}$
\item Resistori in parallelo $\frac{1}{R_{eq}} =\sum ^{N}_{i}\frac{1}{R_{i}}$
\end{itemize}

Potenza dissipata $W=\frac{d\mathcal{L}}{dt} =\frac{dQ\ \Delta V}{dt} =I\cdot \Delta V$, nel caso di conduttori ohmici $W=\frac{\Delta V^{2}}{R} =RI^{2}$ (Legge di Joule)

Potenza dissipata per unità di volume $w=\frac{dW^{(d\tau )}}{d\tau } =\vec{E} \cdot \vec{J}$

Generatore a circuito aperto $V_{A} -V_{B} =fem$

Generatore a circuito chiuso $\Delta V=fem-rI$
\section*{Campo Magnetostatico}

III Equazione di Maxwell (anche in regime tempovariante) \\$\Phi _{\Sigma } (\vec{B} )=\int _{\Sigma }\vec{B} \cdot \vec{n} dS=\int _{\tau }\text{div}\vec{B} \ d\tau =0,\ \text{div}\vec{B} =0$

Forza di Lorentz $\vec{F} =q\vec{v} \times \vec{B}$

Moto di una carica in un campo magnetico uniforme $R=\frac{mv}{qB}$, $\vec{\omega } =-\frac{q\vec{B}}{m}$

Forza magnetica agente su conduttori percorsi da corrente (II legge di Laplace) $d\vec{F} =I\ d\vec{l} \times \vec{B}$

Momento magnetico $\vec{M} =IS\vec{n} \times \vec{B} =\vec{m} \times \vec{B}$

Energia potenziale magnetostatica $U_{m} =-\vec{m} \cdot \vec{B}$

Espressioni di forza, momento e lavoro tramite flusso magnetico $U_{m} =-I\cdot \Phi _{\Sigma } (\vec{B} )$, $d\mathcal{L} =I\ d\Phi $, $\vec{R} =\vec{\nabla } (I\ \Phi (\vec{B} ))$

Campo magnetico prodotto da un circuito percorso da corrente (I legge di Laplace) $d\vec{B} (P)=\frac{\mu _{0}}{4\pi }\frac{I\ d\vec{l} \times \vec{u}_{r}}{r^{2}}$

Legge di Biot-Savart (campo magnetico prodotto da un filo rettilineo infinito) $\vec{B} (P)=\frac{\mu _{0} I}{2\pi R}\vec{u}_{\varphi }$

Legge di Ampére $\oint _{\gamma }\vec{B} \cdot d\vec{l} =\mu _{0} I^{\text{concatenata a} \gamma }_{\text{tot}}$

IV Equazione di Maxwell (solo in regime stazionario) $\text{rot}\vec{B} =\mu _{0}\vec{J}$

Densità di corrente superficiale $j_{s} =\frac{di}{dl}$

Condizioni al contorno di B: $B_{1n} =B_{2n}$, $B_{1t} -B_{2t} =\mu _{0} \ j_{s}$

Potenziale magnetico vettore $\vec{B} =\text{rot}\vec{A} ,\ \nabla ^{2}\vec{A} =-\mu _{0}\vec{J}$

Permeabilità magnetica relativa del materiale $\frac{B}{B_{0}} =\mu _{r}$

Suscettività magnetica relativa del materiale $\chi _{m} =\mu _{r} -1$

Vettore magnetizzazione $\vec{M} (P)=\lim _{\Delta \tau \rightarrow 0}\frac{\Delta \vec{m}}{\Delta \tau }$

Densità di corrente di magnetizzazione \\$\vec{j}_{sm} =\vec{M} \times \vec{n}$, $\vec{J}_{m} =\text{rot}\vec{M}$

Vettore campo magnetico H \\$\vec{H} =\frac{\vec{B}}{\mu _{0}} -\vec{M}$, \ $\vec{M} =\chi _{m}\vec{H} =(\mu _{r} -1)\vec{H}$, \ $\vec{B} =\mu _{0} \mu _{r}\vec{H}$

Condizioni al contorno per B e H: \\$B_{1n} =B_{2n} ,\ \ \mu _{r1} H_{1n} =\mu _{r2} H_{2n} ,\\ \ H_{1t} -H_{2t} =j^{\text{cond.}} ,\ \ \frac{B_{1t}}{\mu _{r1}} -\frac{B_{2t}}{\mu _{r2}} =\mu _{0} \ j^{\text{cond.}}$
\section*{Campi elettrici e magnetici lentamente variabili}

Legge di Faraday-Neumann $f_{i} =-\frac{d\Phi _{\Sigma } (\vec{B} )}{dt}$

II Equazione di Maxwell (in regime tempovariante) $\text{rot}\vec{E} =-\frac{\partial \vec{B}}{\partial t}$

Energia magnetica \\$U_{m} =\frac{1}{2} L\ I^{2} =\frac{1}{2} (\mu _{0} n^{2}_{s} S\ l)\ I^{2}$, $u_{m} =\frac{1}{2}\frac{B^{2}}{\mu _{0}}$

Densità di corrente di spostamento $\vec{J}_{s} =\frac{\partial \vec{D}}{\partial t}$

IV Equazione di Maxwell (in regime tempovariante) $\text{rot}\vec{B} =\mu _{0}\left(\vec{J}_{c} +\frac{\partial \vec{D}}{\partial t}\right)$

In presenza di dielettrici $\text{rot}\vec{H} =\vec{J}_{c} +\frac{\partial \vec{D}}{\partial t}$
\section*{Onde elettromagnetiche}

Equazione delle onde $\frac{\partial ^{2} \xi }{\partial x^{2}} -\frac{1}{v^{2}}\frac{\partial ^{2} \xi }{\partial t^{2}} =0$

Funzione d'onda $\xi (x,t)=\xi _{0}\sin [k(x-vt)]=\xi _{0}\sin [kx-\omega t]$

Lunghezza d'onda $\lambda =\frac{2\pi }{k}$

Periodo $T=\frac{2\pi }{\omega }$

Frequenza $\nu =\frac{1}{T} =\frac{\omega }{2\pi } =\frac{kv}{2\pi } =\frac{v}{\lambda } \Longrightarrow v=\lambda \nu ,\lambda =vT$

Onde elettromagnetiche \\$\nabla ^{2}\vec{E} =\mu _{0} \varepsilon _{0}\frac{\partial ^{2}\vec{E}}{\partial t^{2}} ,\ \ \nabla ^{2}\vec{B} =\mu _{0} \varepsilon _{0}\frac{\partial ^{2}\vec{B}}{\partial t^{2}}$

Operatore d’alembertiano \\$\Box =\nabla ^{2} -\frac{1}{c^{2}}\frac{\partial ^{2}}{\partial t^{2}} \Longrightarrow \Box \vec{E} =0,\Box \vec{B} =0$

$\vec{E} =\vec{B} \times \vec{v}$

Densità di energia $u=u_{e} +u_{m} =2u_{m} =2u_{e}$

Energia elettromagnetica \\$U=\int _{\tau } (u_{e} +u_{m} )d\tau =\int _{\tau }\left[\frac{1}{2}\vec{D} \cdot \vec{E} +\frac{1}{2}\vec{B} \cdot \vec{H}\right] d\tau $

Vettore di Poynting $\vec{S} =\vec{E} \times \vec{H}$

Scambio di energia \\$\frac{dU}{dt} =-\int _{\Sigma }\vec{S} \cdot \vec{n} \ dS-\int _{\tau } \rho J^{2} \ d\tau -\int _{\tau }\vec{J} \cdot \vec{E}_{m} \ d\tau $

Vettore di Poynting di un'onda piana armonica polarizzata rettilinearmente $\vec{S} =u\ \vec{v} =\frac{B^{2}}{\mu _{0} \mu _{r}}\vec{v} =\varepsilon _{0} \varepsilon _{r} E^{2}\vec{v}$

Propagazione di onde nei mezzi $\frac{1}{v^{2}} =\mu _{0} \varepsilon _{0} \mu _{r} \varepsilon _{r} =\frac{\mu _{r} \varepsilon _{r}}{c^{2}} \Rightarrow v=\frac{c}{\sqrt{\mu _{r} \varepsilon _{r}}} ,k=\frac{n\omega }{c} =\frac{\omega }{c}\sqrt{\mu _{r} \varepsilon _{r}}$

Equazioni di Maxwell in presenza di sorgenti \\$\vec{B} =\vec{\nabla } \times \vec{A} ,\ \ \vec{E} =-\vec{\nabla } V-\frac{\partial \vec{A}}{\partial t} ,\\\ \ \nabla ^{2}\vec{A} -\frac{1}{c^{2}}\frac{\partial ^{2}\vec{A}}{\partial t^{2}} =-\mu _{0}\vec{J} (t),\ \ \nabla ^{2} V-\frac{1}{c^{2}}\frac{\partial ^{2} V}{\partial t^{2}} =-\frac{\rho (t)}{\varepsilon _{0}}$
\section*{Ottica}

Legge di Snell $\vartheta _{i} =\vartheta _{r} ,\ \ n_{1}\sin \vartheta _{i} =n_{2}\sin \vartheta _{r}$

Angolo limite $\vartheta _{i} =\arcsin\left(\frac{n_{2}}{n_{1}}\right)$

Formule di Fresnel
\begin{itemize}
\item Caso TM. \\$\frac{E_{r}}{E_{i}} =\frac{n_{2}\cos \vartheta _{i} -n_{1}\cos \vartheta _{t}}{n_{2}\cos \vartheta _{i} +n_{1}\cos \vartheta _{t}} \ \ \ \ \frac{E_{t}}{E_{i}} =\frac{2n_{1}\cos \vartheta _{i}}{n_{1}\cos \vartheta _{t} +n_{2}\cos \vartheta _{i}}$
\item Caso TE. \\$\frac{E_{r}}{E_{i}} =\frac{n_{1}\cos \vartheta _{i} -n_{2}\cos \vartheta _{t}}{n_{1}\cos \vartheta _{i} +n_{2}\cos \vartheta _{t}} \ \ \ \ \frac{E_{t}}{E_{i}} =\frac{2n_{1}\cos \vartheta _{i}}{n_{1}\cos \vartheta _{i} +n_{2}\cos \vartheta _{t}}$
\end{itemize}

Incidenza normale alla superficie di separazione \\$r=\frac{E_{r}}{E_{i}} =\frac{n_{1} -n_{2}}{n_{1} +n_{2}} \ \ \ \ t=\frac{E_{t}}{E_{i}} =\frac{2n_{1}}{n_{1} +n_{2}}$

Interferenza, intensità media \\$I_{m} =I_{m1} +I_{m2} +2\sqrt{I_{m1} I_{m2}}\cos (\Delta \Phi )$

Sorgenti uguali \\$I=I_{0} +I_{0} +2\sqrt{I^{2}_{0}}\cos[ \Delta \Phi (\vec{r} )] =2I_{0}[ 1+\cos \Delta \Phi (\vec{r} )]$
\begin{itemize}
\item interferenza costruttiva \\$\Delta \Phi =2\pi m\ \ \ \ m=0,\pm 1,\pm 2\dotsc $

caso di sorgenti puntiformi $r_{1} -r_{2} =m\lambda $

massimi di interferenza $\sin \vartheta =m\frac{\lambda }{D}$
\item interferenza distruttiva \\$\Delta \Phi =(2m+1)\pi \ \ \ \ m=0,\pm 1,\pm 2\dotsc $
\end{itemize}

Inferenza in lamine sottili di materiale dielettrico, interferenza costruttiva $\lambda =\frac{4nd}{2m+1}$

Diffrazione ad una fenditura rettilinea $I_{m} (P)=I_{0}\frac{\sin^{2} \alpha }{\alpha ^{2}}$

Diffrazione ad un foro circolare, angolo minimo di risoluzione $\alpha _{\text{min}} =1.22\ \frac{\lambda }{d}$ (criterio di Rayleigh)

Reticolo di diffrazione \\$I=4I_{0}\frac{\sin^{2} \alpha }{\alpha ^{2}}\cos^{2} \beta ,\ \ \alpha =\frac{kd\sin \vartheta }{2} \ \ \ \ \beta =\frac{kD\sin \vartheta }{2}$
\section*{Richiami}

Teorema della divergenza \\$\Phi _{\Sigma } (\vec{v} )=\int _{\Sigma }\vec{v} \cdot \vec{n} dS=\int _{\tau } (\text{div}\vec{v} )d\tau $

Teorema di Stokes $\oint _{\gamma }\vec{v} \cdot d\vec{l} =\int _{\Sigma } (\text{rot}\vec{v} )\cdot \vec{n} dS$

Nabla $\vec{\nabla } =\vec{u}_{x}\frac{\partial }{\partial x} +\vec{u}_{y}\frac{\partial }{\partial y} +\vec{u}_{z}\frac{\partial }{\partial z}$

Divergenza $\text{div}\vec{v} =\vec{\nabla } \cdot \vec{v} =\frac{\partial v_{x}}{\partial x} +\frac{\partial v_{y}}{\partial y} +\frac{\partial v_{z}}{\partial z}$

Rotore $\text{rot}\vec{v} =\vec{\nabla } \times \vec{v} =\left(\frac{\partial v_{z}}{\partial y} -\frac{\partial v_{y}}{\partial z} ,\frac{\partial v_{x}}{\partial z} -\frac{\partial v_{z}}{\partial x} ,\frac{\partial v_{y}}{\partial x} -\frac{\partial v_{x}}{\partial y}\right)$

Flusso di un campo vettoriale $d\Phi (\vec{v} )=\vec{v} (P)\cdot \vec{n} \ dS$

Angolo nello spazio $\Omega =\frac{A}{r^{2}}$

\end{multicols}

\end{document}
