\documentclass[10pt,a4paper]{book}
\usepackage{amsfonts}
\usepackage[none]{hyphenat} % PER NON FAR ANDARE A CAPO LE PAROLE CON IL TRATTINO
\usepackage[nottoc,notlot,notlof]{tocbibind} % boh
\usepackage{graphicx}
\usepackage{float}
\usepackage{centernot} % serve per il 'A NON implica B' nel comando \centernot
\usepackage{wrapfig}
\pdfsuppresswarningpagegroup=1



% FIGURE MATHCHA.IO
\usepackage{amsmath}
\usepackage{tikz}
\usepackage{mathdots}
\usepackage{yhmath}
\usepackage{cancel}
\usepackage{color}
\usepackage{siunitx}
\usepackage{array}
\usepackage{multirow}
\usepackage{amssymb}
\usepackage{gensymb}
\usepackage{tabularx}
\usepackage{booktabs}
\usepackage{caption}\captionsetup{belowskip=12pt,aboveskip=4pt}
\usetikzlibrary{fadings}
\usetikzlibrary{patterns}
\usetikzlibrary{shadows.blur}
\usepackage{placeins} % The placeins package gives the command \FloatBarrier, which will make sure any floats will be put in before this point
\usepackage{flafter}  % The flafter package ensures that floats don't appear until after they appear in the code.



% PER INCLUDERE FIGURE
\usepackage{import}
\usepackage{pdfpages}
\usepackage{transparent}
\usepackage{xcolor}

% QUESTO NUOVO COMANDO SERVE PER INCLDERE LE FIGURE FATTE IN INKSCAPE, CHE SI DEVONO TROVARE NELLA STESSA DIRECTORY DENTRO LA CARTELLA figures
\newcommand{\incfig}[2][1]{%
	\def\svgwidth{#1\columnwidth}
	\import{./figures/}{#2.pdf_tex}
}



% RISCRITTURA DI COMANDI
%\renewcommand{\epsilon}{\varepsilon} % NON USATO
%\renewcommand{\theta}{\vartheta} % NON USATO
%\renewcommand{\rho}{\varrho}
%\renewcommand{\phi}{\varphi} % NON USATO
\renewcommand{\degree}{^\circ\text{C}} % SIMBOLO GRADI
\newcommand{\notimplies}{\mathrel{{\ooalign{\hidewidth$\not\phantom{=}$\hidewidth\cr$\implies$}}}}

\usepackage{mathtools} % SERVE PER I DUE COMANDI DOPO
\DeclarePairedDelimiter{\abs}{\lvert}{\rvert} % CREA UN COMANDO abs()
\DeclarePairedDelimiter{\norma}{\lVert}{\rVert} % CREA UN COMANDO norma()



% INDICE
\setcounter{secnumdepth}{3} % DI DEFAULT LE SUBSUBSECTION NON SONO NUMERATE, COSÌ SÌ
\setcounter{tocdepth}{1} % FISSA LA PROFONDITÀ DELLE COSE MOSTRATE NELL'INDICE
\usepackage{tocstyle}
\usetocstyle{standard}
\usepackage[hidelinks]{hyperref} % RENDE L'INDICE INTERATTIVO E hidelinks NASCONDE IL BORDO ROSSO DAI RIFERIMENTI


% APPENDICI
\usepackage[toc,page]{appendix}
\newcommand{\nocontentsline}[3]{} % QUESTO COMANDO E QUELLO DOPO SERVONO PER AVERE IL COMANDO \tocless DA METTERE PRIMA DI UNA SEZIONE CHE NON VOGLIO FAR APPARIRE NELL'INDICE
\newcommand{\tocless}[2]{\bgroup\let\addcontentsline=\nocontentsline#1{#2}\egroup}



% FONT
\usepackage[T1]{fontenc}
\usepackage[utf8]{inputenc}
\usepackage[italian]{babel}
\usepackage{latexsym}
\usepackage{textcomp}
\usepackage{siunitx}



% SCRITTA LaTeX
\newcommand{\latex}{\LaTeX\xspace}
\newcommand{\tex}{\TeX\xspace}



% PARAGRAFI, INTERLINEA E MARGINE
\emergencystretch 3em % PER EVITARE CHE IL TESTO VADA OLTRE I MARGINI
\parindent 0ex % TOGLIE INTENDAMENTO PARAGRAFI
% \setlength{\parindent}{4em} % CAMBIA INDENTAMENTO PARAGRAFI
\setlength{\parskip}{\baselineskip} % CAMBIA SPAZIO TRA PARAGRAFI (POSSO METTERE ANCHE 1em)
% \renewcommand{\baselinestretch}{1.5} % CAMBIA INTERLINEA
% \usepackage[margin=1in]{geometry} % CAMBIA MARGINE DEL DOCUMENTO



% HEADER
\usepackage{fancyhdr}
\pagestyle{fancy}
\fancyhead{} % PULISCI HEADER
\fancyfoot{} % PULISCI FOOTER

% TOGLI I PROSISMI DUE COMMENTI PER CAMBIARE DA 'Capitolo X. Blabla' A 'X. Blabla'
%\renewcommand{\chaptermark}[1]{\markboth{#1}{}}
\fancyhead[RE]{\nouppercase{\leftmark}} %\fancyhead[RE]{\thechapter.  \leftmark}
\fancyhead[LO]{\nouppercase{\rightmark}}
\fancyhead[LE,RO]{\thepage}

%\fancyhead[L]{\slshape \MakeUppercase{FISICA SPERIMENTALE 2}}
%\fancyhead[R]{\slshape A cura di Teo Bucci e Giulia Di Giusto}
%\fancyfoot[C]{\thepage}
%\renewcommand{\headrulewidth}{0pt} % CANCELLA LINEA ORIZZONTALE HEAD



% CENTRARE VERTICALMENTE IL TITOLO DELLA PAGINA PRINCIPALE
\usepackage{titling}
\renewcommand\maketitlehooka{\null\mbox{}\vfill}
\renewcommand\maketitlehookd{\vfill\null}



% BOX DI VARIO TIPO

% TEOREMA TIPO 1
\usepackage{mdframed}
\newmdtheoremenv{mdtheo}{Teorema}

% BOX DEFINIZIONI/FORMULE IMPORANTI/TEOREMA TIPO 2
\usepackage{tcolorbox}
\usepackage{blindtext}
\usepackage{tikz,tkz-tab,amsmath}
\tcbuselibrary{theorems}

% FORMULE
%\newtcolorbox{formula}{
%	colback=red!5!white,
%	colframe=red!75!black
%	}
\newtcolorbox{formula}{
	colback=black!5!white,
	colframe=black
	}

% TEOREMA TIPO 2
\newtcbtheorem
	[number within=section]
	{theo}
	{Teorema}
	{
		colback=green!5,
		colframe=green!35!black,
		fonttitle=\bfseries,
		separator sign none % toglie i :
	}
	{th}

% DEFINIZIONI
\newtcbtheorem
	[number within=section] % init options
	{definition} % nome da scrivere in LaTeX
	{Definizione}% Titolo che verrà visualizzazione
	{
		colback=blue!5,
		colframe=blue!45!black,
		fonttitle=\bfseries,
	} % options
	{def} % PREFISSO/LABEL PER LE REF



% CHECKLIST
%% perchèé èé ' "
% $1$ $ 1$ $1 $ $ 1 $
%% pedici fin, in con \text{} tot int a Sigma sup lat
%% \[ \]
%% spazio prima di ,:;
%% spazio equation spazio
%$$ spazio section spazio
%% spazio a inizio riga
%% spazio a fine riga
%% spazio }
%% itemize
%%_{}^{} togliere graffe se possibile
%% spazio prima di |
%% i-esima $i$-esima
%% \vec{ spazio}
%%^\circ\text{C} gradi \degree o^o
%% minuscole dopo il .
%% Terra/terra
%% ne né nè
% n,. ?????
%% . dopo paragraph
%% f.e.m.
%% finora, NON fin ora, NON fin'ora

\numberwithin{equation}{subsection}
\counterwithout{equation}{section}


%%%%%%%%%%%%%%%%%%%%%%%%%%%%%%%%%%%%%%%%%%%%%%%
%%%%%%%%%%%%%%%%%%%%%%%%%%%%%%%%%%%%%%%%%%%%%%%



\begin{document}




%%%%%%%%%%%%%%%%%%%%%%%%%%%%%%%%%%%%%%%%%%%%%%%
%%%%%%%%%%%%%%%%%%%%%%%%%%%%%%%%%%%%%%%%%%%%%%%


% INFORMAZIONI PER LA PAGINA INIZIALE
%\title{Fisica Sperimentale 2}
%\author{Teo Bucci, Giulia Di Giusto}
%\date{\today}

% PAGINA INIZIALE
%\begin{titlingpage} % PER AVERLO CENTRATO
%	\maketitle
%	\thispagestyle{empty} % TOGLI STILE HEADER E FOOTER
%	\clearpage % PASSA ALLA PAGINA DOPO
%\end{titlingpage}


%%%%%%%%%%%%%%%%%%%%%%%%%%%%%%%%%%%%%%%%%%%%%%%
%%%%%%%%%%%%%%%%%%%%%%%%%%%%%%%%%%%%%%%%%%%%%%%


% COPERTINA

\pagestyle{empty} % SWITCHA PER NON AVERE NUMERO PAGINA
\vspace*{\fill}
Teo Bucci \\
Giulia Di Giusto \\
\begin{center}
	{\large \textsc{Appunti di}}\\
	\vspace*{0.4cm}
	{\Huge \textsc{Fisica Sperimentale 2}}


	%{\Huge \textsc{Appunti di Fisica 2}} \\
	%\vspace*{0.4cm}
	%{\large \textsc{Dalle lezioni del prof. Salvatore Stagira}}
\end{center}
\vspace*{\fill}
\newpage


%%%%%%%%%%%%%%%%%%%%%%%%%%%%%%%%%%%%%%%%%%%%%%%
%%%%%%%%%%%%%%%%%%%%%%%%%%%%%%%%%%%%%%%%%%%%%%%


% SECONDA PAGINA

% {\Large \textit{Appunti di Fisica Sperimentale 1}}

\vspace*{\fill}

\textcopyright \ Gli autori, tutti i diritti riservati

Sono proibite tutte le riproduzioni senza autorizzazione scritta degli autori.

Revisione del 20 marzo 2020%\today

Developed by\\
Teo Bucci - \texttt{teobucci8@gmail.com}\\
Giulia Di Giusto - \texttt{digiusto99@gmail.com}

Compiled with \ensuremath\heartsuit \\
%Foto di copertina by Matt Pett (\emph{@mattpunsplash, unsplash.com})

Per segnalare eventuali errori o suggerimenti potete contattare gli autori.

\newpage


%%%%%%%%%%%%%%%%%%%%%%%%%%%%%%%%%%%%%%%%%%%%%%%
%%%%%%%%%%%%%%%%%%%%%%%%%%%%%%%%%%%%%%%%%%%%%%%


% PREFAZIONE

{\Huge \textbf{Prefazione}}

Questo libro raccoglie gli appunti del corso di \emph{Fisica Sperimentale 2} per alunni ingegneri matematici, tenuto al Politecnico di Milano nell'anno accademico 2019-2020.% dal prof. Salvatore Stagira.

Seguendo l'esperienza di altri ragazzi abbiamo deciso di realizzare questo manuale per venire incontro alle esigenze degli studenti, che talvolta non riescono a seguire perfettamente il docente e non trovano una corrispondenza appropriata sul libro di testo, il quale è spesso sovrabbondante di argomenti.

Scrivendo questo libro abbiamo avuto modo di capire meglio alcuni aspetti della materia, cercando di renderli più chiari possibile con gli strumenti che l'ambiente \latex ci ha messo a disposizione. Speriamo che questo stimoli lo studio e l'interesse nei lettori.

Il manuale si articola in vari capitoli che seguono il programma del corso e si conclude con una serie di appendici con alcuni riferimenti utili durante lo studio dei vari argomenti (derivazione, integrazione, costanti fisiche, trigonometria, ecc.).

Ringraziamo tutti coloro che ci hanno supportato.\\
Ringraziamo in special modo tutti i forum e gli utenti che inizialmente ci hanno aiutato a farci strada nei meandri di questo nuovo ambiente di scrittura. È stato faticoso e abbiamo sbattuto la testa davanti a molti ostacoli, ma è stato anche estremamente istruttivo e formativo.\\
Ringraziamo infine il docente per i suoi insegnamenti e la chiarezza con cui ha portato avanti uno dei corsi più interessanti e fondamentali per ogni ingegnere.

\begin{flushright}
\emph{Gli autori} \hspace*{2cm}
\end{flushright}

\newpage


%%%%%%%%%%%%%%%%%%%%%%%%%%%%%%%%%%%%%%%%%%%%%%%
%%%%%%%%%%%%%%%%%%%%%%%%%%%%%%%%%%%%%%%%%%%%%%%


% INDICE

\addtocontents{toc}{\protect\thispagestyle{empty}} % MI ASSICURA CHE NON CI SIA STILE NELL'INDICE, COME ULTERIORE METODO DI SICUREZZA
\tableofcontents % PRODUCE L'INDICE
\newpage % PER TERMINARE SUBITO LA PAGINA DOPO L'INDICE

% FACOLTATIVA PAGINA VUOTA
$ $\\
\newpage

\pagestyle{fancy} % RISWITCHA PER RIAVERE IL NUMERO PAGINA
\setcounter{page}{1} % FA RIPARTIRE IL CONTATORE PAGINA DA 1


%%%%%%%%%%%%%%%%%%%%%%%%%%%%%%%%%%%%%%%%%%%%%%%
%%%%%%%%%%%%%%%%%%%%%%%%%%%%%%%%%%%%%%%%%%%%%%%
%%%%%%%%%%%%%%%%%%%%%%%%%%%%%%%%%%%%%%%%%%%%%%%
%%%%%%%%%%%%%%%%%%%%%%%%%%%%%%%%%%%%%%%%%%%%%%%


\chapter{Campo Elettrostatico}

In questa sezione assumeremo che gli oggetti interagenti si muovano seguendo le leggi della meccanica classica. I fenomeni elettrostatici sono quelli di natura elettrica in cui gli oggetti sono in quiete in un sistema di riferimento inerziale. Essi sono noti fin dall'antichità. In greco antico l'ambra si chiama elecrton, da qui deriva il termine elettricità. I fenomeni elettrici sono rimasti per molti secoli una curiosità fino al 1600 in cui vi è la corsa allo studio dei pianeti. Cominciano ad esserci su di essi tutta una serie di teorie fra cui quelle sull'esistenza delle forze di tipo elettrico o magnetico. La teoria cadde in disuso dopo le scoperte di Newton sulla gravitazione universale. Nel corso del 1700 tuttavia incominciano tutta una serie di esperienze. Nel 1740 si scopre che gli oggetti si dividono in isolanti e conduttori. Dieci anni dopo si scopre il concetto di carica elettrica. Nel 1785 Coulomb si accorge che c'è una legge generale che regola l'interazione fra cariche elettriche. Nel 1800 volta inventa la pila elettrica. Grazie a tale oggetto si cominciano a studiare i fenomeni della conduzione elettrica. Faraday studia anche le leggi dell'elettrolisi; in particolare, in una cella elettrolitica la massa depositata fra gli elettrodi è proporzionale alla quantità di carica che ha attraversato la cella. Nel 1917 ha luogo l'esperimento di Millikan, sulla quantizzazione carica elettrica.







































\section{Struttura della materia}

La materia così come la conosciamo si compone di tre particelle: \textbf{protoni neutroni ed elettroni}. Per quanto riguarda protoni e neutroni, la loro massa è $m_p \sim m_n \sim  1.67 \times 10^{-27} \text{kg}$.
La massa dell'elettrone è quella del protone divisa per 1840: $m_e= \frac{m_p}{1840}=9,1\times 10^{31} \text{kg}$.
Si pensa che le dimensioni del protone e neutrone siano dell'ordine del pentometro, $10^{-15}\,m$.
Protoni e neutroni sono particelle passive a loro volta composte da sub-particelle. I protoni sono dotati anche di carica elettrica. In particolare tale carica è positiva e viene indicata con la lettera $e=1.6\times 10^{-19} C$.
La carica elettrica dell'elettrone invece è pari a $-e$. Si tratta di cariche di segno opposto. La cosa da notare è che siccome queste sono le particelle più piccole che caratterizzano la materia, le cariche estratte non possono che essere multipli fondamentali di $e$. La carica elettrica per tale motivo pare essere quantizzata. Cariche più piccole non sono mai state osservate in natura, anche se sappiamo che protoni e neutroni sono composti da particelle ancora più piccole, i quark, aventi carica elettrica stazionaria. Il neutrone è privo di carica.
L'elettromagnetismo governa le interazioni fra elettrone e nucleo. La struttura dell'atomo è quella di un nucleo positivo in cui abbiamo protoni e neutroni legati fra loro da interazioni nucleari forti, attorno al quale orbitano elettroni di carica negativa. Nello stato fondamentale un atomo è sempre elettricamente neutro. Ciò significa che la somma algebrica delle cariche contenute al suo interno fa $0$. Il numero di elettroni e di protoni è lo stesso. Il numero atomico, ossia il numero di protoni, è quello che ne stabilisce le proprietà chimiche.
Il numero di massa è la somma del numero di protoni più quello di neutroni.
Possiamo avere atomi con stesso numero atomico ma diverso numero di massa, si parla di isotopi. La cosa fondamentale è che l'atomo è sempre neutro. Le dimensioni dell'atomo sono dell'ordine di $10^{-10}\,m = 1 $ Angstrom e sono 100mila volte maggiori di quelle del nucleo. Esso è piccolissimo ed è circondato da una nube di elettroni che si muovono in uno spazio sostanzialmente vuoto.




















\subsection{Principio di conservazione della carica elettrica}

Afferma che considerato un sistema isolato la \emph{somma algebrica di tutte le cariche} in esso presenti rimane costante nel tempo, ovvero \emph{si conserva}. Tale principio lo si può comprendere in base a come è strutturata la materia. Se abbiamo un sistema isolato in cui non c'è scambio di massa al suo interno il numero di elettroni e protoni rimane costante nel tempo. Dal punto di vista della fisica classica corrisponde al principio di conservazione della massa.




















\subsection{Fenomeni triboelettrici}

I fenomeni triboelettrici sono quei fenomeni indotti per strofinio. Se prendiamo due oggetti e li strofiniamo per poi metterli in prossimità, essi cominciano a interagire tramite un interazione che ha le seguenti proprietà:
\begin{itemize}
	\item Le forze sono di tipo repulsivo se gli oggetti sono dello stesso materiale.
	\item Altrimenti, a seconda delle sostanze con cui sono fatti gli oggetti osserveremo forze di tipo attrattivo o repulsivo. Se considero un panno di lana e un altro oggetto vi è sempre interazione di tipo attrattivo.
	\item Se consideriamo tre oggetti noteremo che se $A$ e $C$ si attraggono e $C$ e $B$ si attraggono, allora $A$ e $B$ si respingono. Viceversa se $A$ e $C$ ed $B$ e $C$ si respingono, $A$ e $B$ si attraggono.
\end{itemize}
Osserviamo cariche elettriche positive e negative. Quello che deduciamo sperimentalmente è che cariche dello stesso segno si respingono e cariche di segno opposto si attraggono. Mentre le forze gravitazionali sono solo attrattive, quelle elettrostatiche sono sia attrattive che repulsive.
Dopo lo strofinio ci troveremo in uno stato di elettrizzazione. Le cariche nucleari non sono più completamente compensate. Consideriamo un materiale composto da plastica. Avremo un trasferimento di elettroni dal panno alla plastica, essi di attraggono. In generale questi fenomeni sono dovuti al trasferimento di carica elettrica. Tutti i materiali citati sono detti isolanti o dielettrici. Quando strofino un materiale isolante otterrò in quel punto un eccesso di carica ma essa non si distribuisce, rimane localizzata in quel punto. In tali materiali c'è una scarsa mobilità delle cariche elettriche.
Esiste anche una grande categoria di materiali detti conduttori. Tipicamente essi sono materiali metallici. Ciò che accade è che alcuni elettroni dei vari atomi sono messi in compartecipazione e liberi di muoversi. Possiamo immaginare che ci sia un gas di elettroni liberi. Se depongo della carica negativa in un certo punto essa su sposta immediatamente e si ridistribuisce su tutto il materiale.

È sperimentalmente dimostrato che se consideriamo un conduttore inizialmente neutro e vi avviciniamo una carica elettrica, proprio perché al suo interno le cariche sono libere di muoversi fra loro, gli elettroni si affacceranno su un lato. Le cariche sono sempre le stesse ma sono separate. Questo fenomeno prende il nome di induzione elettrica. Se colleghiamo il conduttore con un filo ad un conduttore molto più grande, ad esempio il pianeta terra, là cariche positive, siccome sono libere di muoversi e si respingono fra loro, vogliono disporsi in modo più lontano possibile l'una dall'altra e quindi fuggono verso terra.







































\section{Legge di Coulomb}

È stato Coulomb a iniziare a studiare in maniera sistematica le cariche elettriche e le interazioni fra oggetti carichi elettricamente di dimensioni trascurabili rispetto alla distanza fra di loro. Egli è arrivato a formulare una legge che vale solo quindi per quanto riguarda le cariche puntiformi.
\begin{figure}[htpb]
	\centering
	

	\tikzset{every picture/.style={line width=0.75pt}} %set default line width to 0.75pt        

	\begin{tikzpicture}[x=0.75pt,y=0.75pt,yscale=-1,xscale=1]
	%uncomment if require: \path (0,300); %set diagram left start at 0, and has height of 300

	%Shape: Circle [id:dp6759441700488122] 
	\draw  [fill={rgb, 255:red, 0; green, 0; blue, 0 }  ,fill opacity=1 ] (120,123.75) .. controls (120,122.23) and (121.23,121) .. (122.75,121) .. controls (124.27,121) and (125.5,122.23) .. (125.5,123.75) .. controls (125.5,125.27) and (124.27,126.5) .. (122.75,126.5) .. controls (121.23,126.5) and (120,125.27) .. (120,123.75) -- cycle ;
	%Straight Lines [id:da770733030940054] 
	\draw [line width=1.5]    (122.75,123.75) -- (67.5,123.75) ;
	\draw [shift={(63.5,123.75)}, rotate = 360] [fill={rgb, 255:red, 0; green, 0; blue, 0 }  ][line width=0.08]  [draw opacity=0] (13.4,-6.43) -- (0,0) -- (13.4,6.44) -- (8.9,0) -- cycle    ;
	%Shape: Circle [id:dp8565834305540732] 
	\draw  [fill={rgb, 255:red, 0; green, 0; blue, 0 }  ,fill opacity=1 ] (220,123.75) .. controls (220,122.23) and (221.23,121) .. (222.75,121) .. controls (224.27,121) and (225.5,122.23) .. (225.5,123.75) .. controls (225.5,125.27) and (224.27,126.5) .. (222.75,126.5) .. controls (221.23,126.5) and (220,125.27) .. (220,123.75) -- cycle ;
	%Straight Lines [id:da7726897544624338] 
	\draw [line width=1.5]    (278,123.75) -- (222.75,123.75) ;
	\draw [shift={(282,123.75)}, rotate = 180] [fill={rgb, 255:red, 0; green, 0; blue, 0 }  ][line width=0.08]  [draw opacity=0] (13.4,-6.43) -- (0,0) -- (13.4,6.44) -- (8.9,0) -- cycle    ;
	%Straight Lines [id:da33333178128062513] 
	\draw [line width=1.5]    (157.5,123.75) -- (122.75,123.75) ;
	\draw [shift={(161.5,123.75)}, rotate = 180] [fill={rgb, 255:red, 0; green, 0; blue, 0 }  ][line width=0.08]  [draw opacity=0] (13.4,-6.43) -- (0,0) -- (13.4,6.44) -- (8.9,0) -- cycle    ;
	%Straight Lines [id:da5435846648051312] 
	\draw [line width=0.75]  [dash pattern={on 0.84pt off 2.51pt}]  (222.75,123.75) -- (122.75,123.75) ;
	%Shape: Brace [id:dp15876881438202872] 
	\draw   (123,134) .. controls (123,138.67) and (125.33,141) .. (130,141) -- (163,141) .. controls (169.67,141) and (173,143.33) .. (173,148) .. controls (173,143.33) and (176.33,141) .. (183,141)(180,141) -- (216,141) .. controls (220.67,141) and (223,138.67) .. (223,134) ;

	% Text Node
	\draw (124,102.5) node    {$q_{1}$};
	% Text Node
	\draw (223,103.5) node    {$q_{2}$};
	% Text Node
	\draw (251,139) node    {$\vec{F}_{21}$};
	% Text Node
	\draw (96.5,139.5) node    {$\vec{F}_{12}$};
	% Text Node
	\draw (151.5,105.5) node    {$\vec{u}_{r}$};
	% Text Node
	\draw (174.5,155.5) node    {$r$};


	\end{tikzpicture}
\end{figure}
\FloatBarrier
Siano due cariche in quiete in un sistema di riferimento inerziale. Sia $r$ la distanza fra le cariche. $\vec{F}_{21}$ può essere scritta come:
\[
	\boxed{\vec{F}_{21}=k \frac{q_1q_2}{r^2} \vec{u}_r}  \qquad \text{con} \quad k=8.98\times 10^9 \, Nm^2 /C^2
\]
Con $\vec{u}_r$ versore radiale. Tale legge ricorda quella di gravitazione universale di Newton.
\[
	\vec{F}_{12}=-\vec{F}_{21}
\]
Guardiamo $\vec{F}_{21}$ come funzione vettoriale della posizione, che associa ad un punto dello spazio il vettore forza. Abbiamo così identificato un campo vettoriale. La Nostra forza $\vec{F}_{21}$ è un campo vettoriale il cui modulo dipende solo dalla distanza $r$ e la cui direzione è quella radiale. Si parla di \textbf{campo di forza centrale}. Si può dimostrare che un campo di forza centrale è \emph{conservativo}. Per le proprietà introdotte, il verso della forza è opposto a $\vec{u}_r$ se le cariche sono opposte. La legge di Coulomb contiene in questo senso già tutti i casi possibili.

Vediamo una seconda versione. Immaginiamo di identificare la posizione di $q_2$ rispetto a $q_1$ con un vettore posizione $r$. Il versore $\vec{u}_r$ si può anche scrivere come il rapporto:
\[
	\frac{\vec{r}}{|\vec{r} |} = \frac{\vec{r}}{r}
\]
Poniamoci in un sistema di coordinate cartesiane.
\begin{figure}[htpb]
	\centering
	

	\tikzset{every picture/.style={line width=0.75pt}} %set default line width to 0.75pt        

	\begin{tikzpicture}[x=0.75pt,y=0.75pt,yscale=-1,xscale=1]
	%uncomment if require: \path (0,300); %set diagram left start at 0, and has height of 300

	%Straight Lines [id:da6964491635777041] 
	\draw    (179.5,140) -- (179.5,60) ;
	\draw [shift={(179.5,57)}, rotate = 450] [fill={rgb, 255:red, 0; green, 0; blue, 0 }  ][line width=0.08]  [draw opacity=0] (10.72,-5.15) -- (0,0) -- (10.72,5.15) -- (7.12,0) -- cycle    ;
	%Straight Lines [id:da7452549937508859] 
	\draw    (179.5,140) -- (239.93,176.45) ;
	\draw [shift={(242.5,178)}, rotate = 211.1] [fill={rgb, 255:red, 0; green, 0; blue, 0 }  ][line width=0.08]  [draw opacity=0] (10.72,-5.15) -- (0,0) -- (10.72,5.15) -- (7.12,0) -- cycle    ;
	%Straight Lines [id:da5024739469391739] 
	\draw    (179.5,140) -- (120.06,176.43) ;
	\draw [shift={(117.5,178)}, rotate = 328.5] [fill={rgb, 255:red, 0; green, 0; blue, 0 }  ][line width=0.08]  [draw opacity=0] (10.72,-5.15) -- (0,0) -- (10.72,5.15) -- (7.12,0) -- cycle    ;
	%Shape: Circle [id:dp27933265319891487] 
	\draw  [fill={rgb, 255:red, 0; green, 0; blue, 0 }  ,fill opacity=1 ] (240.5,67.75) .. controls (240.5,65.13) and (242.63,63) .. (245.25,63) .. controls (247.87,63) and (250,65.13) .. (250,67.75) .. controls (250,70.37) and (247.87,72.5) .. (245.25,72.5) .. controls (242.63,72.5) and (240.5,70.37) .. (240.5,67.75) -- cycle ;
	%Shape: Circle [id:dp5790263105950524] 
	\draw  [fill={rgb, 255:red, 0; green, 0; blue, 0 }  ,fill opacity=1 ] (281,120.75) .. controls (281,118.13) and (283.13,116) .. (285.75,116) .. controls (288.37,116) and (290.5,118.13) .. (290.5,120.75) .. controls (290.5,123.37) and (288.37,125.5) .. (285.75,125.5) .. controls (283.13,125.5) and (281,123.37) .. (281,120.75) -- cycle ;
	%Straight Lines [id:da3756037875902982] 
	\draw    (179.5,140) -- (238.98,74.47) ;
	\draw [shift={(241,72.25)}, rotate = 492.23] [fill={rgb, 255:red, 0; green, 0; blue, 0 }  ][line width=0.08]  [draw opacity=0] (10.72,-5.15) -- (0,0) -- (10.72,5.15) -- (7.12,0) -- cycle    ;
	%Straight Lines [id:da6433495966928404] 
	\draw    (179.5,140) -- (278.05,121.31) ;
	\draw [shift={(281,120.75)}, rotate = 529.26] [fill={rgb, 255:red, 0; green, 0; blue, 0 }  ][line width=0.08]  [draw opacity=0] (10.72,-5.15) -- (0,0) -- (10.72,5.15) -- (7.12,0) -- cycle    ;
	%Straight Lines [id:da4508868053914086] 
	\draw    (250,72.25) -- (280.24,113.82) ;
	\draw [shift={(282,116.25)}, rotate = 233.97] [fill={rgb, 255:red, 0; green, 0; blue, 0 }  ][line width=0.08]  [draw opacity=0] (10.72,-5.15) -- (0,0) -- (10.72,5.15) -- (7.12,0) -- cycle    ;

	% Text Node
	\draw (114,188) node    {$x$};
	% Text Node
	\draw (249.5,186.5) node    {$y$};
	% Text Node
	\draw (166,54) node    {$z$};
	% Text Node
	\draw (296,77.5) node    {$\vec{r}_{2} -\vec{r}_{1}$};
	% Text Node
	\draw (258,142.5) node    {$\vec{r}_{2}$};
	% Text Node
	\draw (213,76.5) node    {$\vec{r}_{1}$};
	% Text Node
	\draw (287.5,44) node    {$q_{1}( x_{1} ,y_{1} ,z_{1})$};
	% Text Node
	\draw (337.5,128.5) node    {$q_{2}( x_{2} ,y_{2} ,z_{2})$};


	\end{tikzpicture}
\end{figure}
\FloatBarrier
\begin{gather*}
	\vec{r}_1 = x_1\vec{u}_x+y_1\vec{u}_y+z_1\vec{u}_z \\
	\vec{r}_2 = x_2\vec{u}_x+y_2\vec{u}_y+z_2\vec{u}_z \\
	\vec{F}_{21} = k q_1q_2 \frac{\vec{r}_1-\vec{r}_2}{|r_1-r_2  |^3}
\end{gather*}
Fino al 2018 la grandezza fondamentale era l'intensità di corrente. Formalmente nel sistema internazionale le dimensioni della carica sono quelle della corrente per un tempo.
\[
	[Q] = [I][T] = (A\cdot s) = C \quad \text{Coulomb}
\]
Dal 2019 si assume che la carica dell'elettrone sia esattamente pari a $1.60 \times 10^{-19} C$ e da qui si definisce il Coulomb. Le forze elettriche come vediamo dal valore di k sono forze estremamente intense. In realtà k spesso viene descritta come:
\[
	k=\frac{1}{4\pi \varepsilon_0}
\]
$\varepsilon_0$ è la \textbf{costante elettrica assoluta del vuoto}. Il suo valore è $\varepsilon_0 = 8.85 \times 10^{-12} C^2 / N m^2$.
A livello microscopico le forze gravitazionali non contano nulla. Se infatti proviamo a mettere a confronto l'intensità della forza di Coulomb e della forza gravitazionale calcolate relativamente all'atomo di idrogeno, si ha:
\[
	F_e = \frac{e^2}{4\pi \varepsilon_0 r^2} = 9\times 10^8N \qquad F_G = \frac{G m_p m_e}{r^2}   =4\times 10^{-47} N
\]
\begin{figure}[htpb]
	\centering
	

	\tikzset{every picture/.style={line width=0.75pt}} %set default line width to 0.75pt        

	\begin{tikzpicture}[x=0.75pt,y=0.75pt,yscale=-1,xscale=1]
	%uncomment if require: \path (0,300); %set diagram left start at 0, and has height of 300

	%Shape: Circle [id:dp9391076515585639] 
	\draw  [dash pattern={on 0.84pt off 2.51pt}] (182.5,133) .. controls (182.5,91.85) and (215.85,58.5) .. (257,58.5) .. controls (298.15,58.5) and (331.5,91.85) .. (331.5,133) .. controls (331.5,174.15) and (298.15,207.5) .. (257,207.5) .. controls (215.85,207.5) and (182.5,174.15) .. (182.5,133) -- cycle ;
	%Straight Lines [id:da7327286186495761] 
	\draw    (295.04,106.28) -- (318.2,90) ;
	\draw [shift={(292.58,108)}, rotate = 324.91] [fill={rgb, 255:red, 0; green, 0; blue, 0 }  ][line width=0.08]  [draw opacity=0] (10.72,-5.15) -- (0,0) -- (10.72,5.15) -- (7.12,0) -- cycle    ;
	%Straight Lines [id:da28216679196328576] 
	\draw    (257,133) -- (282.55,115.05) ;
	\draw [shift={(285,113.33)}, rotate = 504.91] [fill={rgb, 255:red, 0; green, 0; blue, 0 }  ][line width=0.08]  [draw opacity=0] (10.72,-5.15) -- (0,0) -- (10.72,5.15) -- (7.12,0) -- cycle    ;

	% Text Node
	\draw  [fill={rgb, 255:red, 222; green, 222; blue, 222 }  ,fill opacity=1 ]  (257, 133) circle [x radius= 13.9, y radius= 13.9]   ;
	\draw (257,133) node    {$+$};
	% Text Node
	\draw  [fill={rgb, 255:red, 222; green, 222; blue, 222 }  ,fill opacity=1 ]  (317, 91) circle [x radius= 13.6, y radius= 13.6]   ;
	\draw (317,91) node    {$-$};
	% Text Node
	\draw (260.4,154.2) node    {$m_{p}$};
	% Text Node
	\draw (348,79.8) node    {$m_{e}$};
	% Text Node
	\draw (277.2,92.2) node    {$\vec{F}_{e}$};


	\end{tikzpicture}
\end{figure}
\FloatBarrier







































\section{Principio di sovrapposizione}

Fondamentale in questo campo è il principio di sovrapposizione degli effetti. In matematica e in fisica, il \textbf{principio di sovrapposizione} stabilisce che per un sistema dinamico lineare l'effetto di una somma di perturbazioni in ingresso è uguale alla somma degli effetti prodotti da ogni singola perturbazione. Il principio di sovrapposizione esprime la possibilità di scomporre un problema lineare.
\begin{figure}[htpb]
	\centering
	

	\tikzset{every picture/.style={line width=0.75pt}} %set default line width to 0.75pt        

	\begin{tikzpicture}[x=0.75pt,y=0.75pt,yscale=-1,xscale=1]
	%uncomment if require: \path (0,300); %set diagram left start at 0, and has height of 300

	%Shape: Circle [id:dp9532587092519822] 
	\draw  [fill={rgb, 255:red, 0; green, 0; blue, 0 }  ,fill opacity=1 ] (206.6,229.15) .. controls (206.6,226.53) and (208.73,224.4) .. (211.35,224.4) .. controls (213.97,224.4) and (216.1,226.53) .. (216.1,229.15) .. controls (216.1,231.77) and (213.97,233.9) .. (211.35,233.9) .. controls (208.73,233.9) and (206.6,231.77) .. (206.6,229.15) -- cycle ;
	%Straight Lines [id:da6293871325411147] 
	\draw    (269.72,158.83) -- (304.6,116.8) ;
	\draw [shift={(267.8,161.14)}, rotate = 309.69] [fill={rgb, 255:red, 0; green, 0; blue, 0 }  ][line width=0.08]  [draw opacity=0] (10.72,-5.15) -- (0,0) -- (10.72,5.15) -- (7.12,0) -- cycle    ;
	%Shape: Boxed Line [id:dp10731073492947596] 
	\draw    (147.82,180.5) -- (110.6,152) ;
	\draw [shift={(150.2,182.32)}, rotate = 217.44] [fill={rgb, 255:red, 0; green, 0; blue, 0 }  ][line width=0.08]  [draw opacity=0] (10.72,-5.15) -- (0,0) -- (10.72,5.15) -- (7.12,0) -- cycle    ;
	%Shape: Boxed Line [id:dp7675257187319859] 
	\draw    (205.99,151.68) -- (203,108.4) ;
	\draw [shift={(206.2,154.68)}, rotate = 266.04] [fill={rgb, 255:red, 0; green, 0; blue, 0 }  ][line width=0.08]  [draw opacity=0] (10.72,-5.15) -- (0,0) -- (10.72,5.15) -- (7.12,0) -- cycle    ;
	%Shape: Circle [id:dp9186357505606382] 
	\draw  [fill={rgb, 255:red, 0; green, 0; blue, 0 }  ,fill opacity=1 ] (299.85,116.8) .. controls (299.85,114.18) and (301.98,112.05) .. (304.6,112.05) .. controls (307.22,112.05) and (309.35,114.18) .. (309.35,116.8) .. controls (309.35,119.42) and (307.22,121.55) .. (304.6,121.55) .. controls (301.98,121.55) and (299.85,119.42) .. (299.85,116.8) -- cycle ;
	%Shape: Circle [id:dp6102120191986977] 
	\draw  [fill={rgb, 255:red, 0; green, 0; blue, 0 }  ,fill opacity=1 ] (198.25,108.4) .. controls (198.25,105.78) and (200.38,103.65) .. (203,103.65) .. controls (205.62,103.65) and (207.75,105.78) .. (207.75,108.4) .. controls (207.75,111.02) and (205.62,113.15) .. (203,113.15) .. controls (200.38,113.15) and (198.25,111.02) .. (198.25,108.4) -- cycle ;
	%Shape: Circle [id:dp9249207905796555] 
	\draw  [fill={rgb, 255:red, 0; green, 0; blue, 0 }  ,fill opacity=1 ] (105.85,152) .. controls (105.85,149.38) and (107.98,147.25) .. (110.6,147.25) .. controls (113.22,147.25) and (115.35,149.38) .. (115.35,152) .. controls (115.35,154.62) and (113.22,156.75) .. (110.6,156.75) .. controls (107.98,156.75) and (105.85,154.62) .. (105.85,152) -- cycle ;
	%Shape: Boxed Line [id:dp3327031648527261] 
	\draw    (211.35,229.15) -- (169.38,197.01) ;
	\draw [shift={(167,195.19)}, rotate = 397.44] [fill={rgb, 255:red, 0; green, 0; blue, 0 }  ][line width=0.08]  [draw opacity=0] (10.72,-5.15) -- (0,0) -- (10.72,5.15) -- (7.12,0) -- cycle    ;
	%Shape: Boxed Line [id:dp8962126358308287] 
	\draw    (211.35,229.15) -- (207.61,175.02) ;
	\draw [shift={(207.4,172.03)}, rotate = 446.04] [fill={rgb, 255:red, 0; green, 0; blue, 0 }  ][line width=0.08]  [draw opacity=0] (10.72,-5.15) -- (0,0) -- (10.72,5.15) -- (7.12,0) -- cycle    ;
	%Shape: Boxed Line [id:dp2779968769324759] 
	\draw    (211.35,229.15) -- (250.28,182.24) ;
	\draw [shift={(252.2,179.93)}, rotate = 489.69] [fill={rgb, 255:red, 0; green, 0; blue, 0 }  ][line width=0.08]  [draw opacity=0] (10.72,-5.15) -- (0,0) -- (10.72,5.15) -- (7.12,0) -- cycle    ;

	% Text Node
	\draw (221.5,103.7) node    {$q_{2}$};
	% Text Node
	\draw (212.7,244.1) node    {$q_{0}$};
	% Text Node
	\draw (319.9,119.3) node    {$q_{3}$};
	% Text Node
	\draw (124.3,136.5) node    {$q_{1}$};


	\end{tikzpicture}
\end{figure}
\FloatBarrier
Se si è in grado di scrivere i dati di ingresso in più componenti linearmente indipendenti (ad esempio, in un moto a due dimensioni si possono considerare la componente verticale e la componente orizzontale) allora è possibile risolvere il problema analizzando separatamente ciascuna delle componenti: si calcola ogni singola risposta e poi si sommano le singole risposte secondo la stessa proporzione (ovvero con gli stessi coefficienti).
\[
	\vec{R} = \Sigma_i\vec{F}_i  =\Sigma_i \frac{q_iq}{4\pi \varepsilon_0 r_i^2} \vec{u}_{r_i}
\]
Così arriviamo a dire che per le forze, la risultante delle forze è la somma delle varie forze che ciascuna delle cariche esercita da sola sulla carica in questione.







































\section{Campo Elettrico}

Molto spesso le interazioni di tipo elettrico non vengono trattate in termini di forza ma sono interpretate in un altro modo.
Consideriamo il caso della coppia di cariche $Q$ e $q$ e chiamiamo $P$ la posizione della carica $q$. La forza $\vec{F}$ dipende dalla posizione della carica. L'espressione rappresentata dalla legge di Coulomb associa ad ogni punto dello spazio un vettore, si parla di campo vettoriale.
Potremmo interpretare la forza nel seguente modo. Immaginiamo di avere $Q$ isolata dal resto dell'universo. Possiamo immaginare che la sua presenza alteri le proprietà dello spazio circostante. Questa alterazione non è visibile fino a che non avviciniamo un'altra carica. Chiameremo la $Q$ sorgente e la $q$ esploratrice. L'alterazione della sorgente è visibile solo in presenza di una carica esploratrice, che diventa un modo per sondare tale effetto. Invece di vedere le alterazioni come fra due oggetti che hanno la stessa dignità, ci concentriamo sugli effetti di uno sull'altro.
\begin{figure}[htpb]
	\centering
	

	\tikzset{every picture/.style={line width=0.75pt}} %set default line width to 0.75pt        

	\begin{tikzpicture}[x=0.75pt,y=0.75pt,yscale=-1,xscale=1]
	%uncomment if require: \path (0,300); %set diagram left start at 0, and has height of 300

	%Shape: Circle [id:dp1923816477548197] 
	\draw  [fill={rgb, 255:red, 0; green, 0; blue, 0 }  ,fill opacity=1 ] (140,143.75) .. controls (140,142.23) and (141.23,141) .. (142.75,141) .. controls (144.27,141) and (145.5,142.23) .. (145.5,143.75) .. controls (145.5,145.27) and (144.27,146.5) .. (142.75,146.5) .. controls (141.23,146.5) and (140,145.27) .. (140,143.75) -- cycle ;
	%Straight Lines [id:da3508314043109242] 
	\draw [line width=1.5]    (142.75,143.75) -- (87.5,143.75) ;
	\draw [shift={(83.5,143.75)}, rotate = 360] [fill={rgb, 255:red, 0; green, 0; blue, 0 }  ][line width=0.08]  [draw opacity=0] (13.4,-6.43) -- (0,0) -- (13.4,6.44) -- (8.9,0) -- cycle    ;
	%Shape: Circle [id:dp08151978077095245] 
	\draw  [fill={rgb, 255:red, 0; green, 0; blue, 0 }  ,fill opacity=1 ] (240,143.75) .. controls (240,142.23) and (241.23,141) .. (242.75,141) .. controls (244.27,141) and (245.5,142.23) .. (245.5,143.75) .. controls (245.5,145.27) and (244.27,146.5) .. (242.75,146.5) .. controls (241.23,146.5) and (240,145.27) .. (240,143.75) -- cycle ;
	%Straight Lines [id:da06965713056575118] 
	\draw [line width=1.5]    (298,143.75) -- (242.75,143.75) ;
	\draw [shift={(302,143.75)}, rotate = 180] [fill={rgb, 255:red, 0; green, 0; blue, 0 }  ][line width=0.08]  [draw opacity=0] (13.4,-6.43) -- (0,0) -- (13.4,6.44) -- (8.9,0) -- cycle    ;
	%Straight Lines [id:da728097408747798] 
	\draw [line width=1.5]    (177.5,143.75) -- (142.75,143.75) ;
	\draw [shift={(181.5,143.75)}, rotate = 180] [fill={rgb, 255:red, 0; green, 0; blue, 0 }  ][line width=0.08]  [draw opacity=0] (13.4,-6.43) -- (0,0) -- (13.4,6.44) -- (8.9,0) -- cycle    ;
	%Straight Lines [id:da2785371453500993] 
	\draw [line width=0.75]  [dash pattern={on 0.84pt off 2.51pt}]  (242.75,143.75) -- (142.75,143.75) ;

	% Text Node
	\draw (144,122.5) node    {$Q$};
	% Text Node
	\draw (243,123.5) node    {$q$};
	% Text Node
	\draw (271,159) node    {$\vec{F}( P)$};
	% Text Node
	\draw (171.5,125.5) node    {$\vec{u}_{r}$};


	\end{tikzpicture}
\end{figure}
\FloatBarrier
Per comprendere in maniera quantitativa l'alterazione provocata da $Q$, introduciamo il concetto di campo elettrico generato dalla carica sorgente.
Il campo elettrico $\vec{E} $ generato dalla carica $Q$ in posizione $P$ per definizione è il rapporto fra la forza esercitata sulla carica esploratrice e l'entità della carica esploratrice stessa.
\[
	\vec{E} (P)=\frac{\vec{F} (P)}{q}=\frac{1}{q}\cdot \frac{Qq}{4\pi \epsilon_0 r^2}\vec{u}_r = \frac{Q}{4\pi \epsilon_0 r^2}\vec{u}_r
\]
Le dimensioni del campo elettrico sono
\[
	[E]=\frac{[F]}{[Q]} = \left( \frac{N}{C} \right) = \left( \frac{V}{m} \right)
\]
Dal punto di vista della struttura, nel momento in cui vogliiamo determinare nel punto $P$ l'entità del campo elettrico, otterremo un vettore diretto come $\vec{F}$. Tale vettore avrà direzione radiale e verso rivolto all'esterno se $Q$ e $q$ hanno segno concorde, all'interno se invece lo hanno discorde.
\begin{figure}[htpb]
	\centering
	

	\tikzset{every picture/.style={line width=0.75pt}} %set default line width to 0.75pt        

	\begin{tikzpicture}[x=0.75pt,y=0.75pt,yscale=-1,xscale=1]
	%uncomment if require: \path (0,300); %set diagram left start at 0, and has height of 300

	%Straight Lines [id:da18862268387082737] 
	\draw [line width=1.5]    (174.75,50.75) -- (119.5,50.75) ;
	\draw [shift={(115.5,50.75)}, rotate = 360] [fill={rgb, 255:red, 0; green, 0; blue, 0 }  ][line width=0.08]  [draw opacity=0] (13.4,-6.43) -- (0,0) -- (13.4,6.44) -- (8.9,0) -- cycle    ;
	%Straight Lines [id:da608158656410348] 
	\draw [line width=1.5]    (330,50.75) -- (274.75,50.75) ;
	\draw [shift={(334,50.75)}, rotate = 180] [fill={rgb, 255:red, 0; green, 0; blue, 0 }  ][line width=0.08]  [draw opacity=0] (13.4,-6.43) -- (0,0) -- (13.4,6.44) -- (8.9,0) -- cycle    ;
	%Straight Lines [id:da8127149875512598] 
	\draw [line width=1.5]    (215.8,50.75) -- (174.75,50.75) ;
	\draw [shift={(219.8,50.75)}, rotate = 180] [fill={rgb, 255:red, 0; green, 0; blue, 0 }  ][line width=0.08]  [draw opacity=0] (13.4,-6.43) -- (0,0) -- (13.4,6.44) -- (8.9,0) -- cycle    ;
	%Straight Lines [id:da4900066534826375] 
	\draw [line width=0.75]  [dash pattern={on 0.84pt off 2.51pt}]  (274.75,50.75) -- (174.75,50.75) ;
	%Straight Lines [id:da43114939450056067] 
	\draw [line width=1.5]    (347.8,58.75) -- (274.75,58.75) ;
	\draw [shift={(351.8,58.75)}, rotate = 180] [fill={rgb, 255:red, 0; green, 0; blue, 0 }  ][line width=0.08]  [draw opacity=0] (13.4,-6.43) -- (0,0) -- (13.4,6.44) -- (8.9,0) -- cycle    ;
	%Straight Lines [id:da17578940519252795] 
	\draw [line width=1.5]    (304,150.75) -- (248.75,150.75) ;
	\draw [shift={(244.75,150.75)}, rotate = 360] [fill={rgb, 255:red, 0; green, 0; blue, 0 }  ][line width=0.08]  [draw opacity=0] (13.4,-6.43) -- (0,0) -- (13.4,6.44) -- (8.9,0) -- cycle    ;
	%Straight Lines [id:da8949433624753187] 
	\draw [line width=1.5]    (200,150.75) -- (144.75,150.75) ;
	\draw [shift={(204,150.75)}, rotate = 180] [fill={rgb, 255:red, 0; green, 0; blue, 0 }  ][line width=0.08]  [draw opacity=0] (13.4,-6.43) -- (0,0) -- (13.4,6.44) -- (8.9,0) -- cycle    ;
	%Straight Lines [id:da8755641170993631] 
	\draw [line width=1.5]    (185.8,145.75) -- (144.75,145.75) ;
	\draw [shift={(189.8,145.75)}, rotate = 180] [fill={rgb, 255:red, 0; green, 0; blue, 0 }  ][line width=0.08]  [draw opacity=0] (13.4,-6.43) -- (0,0) -- (13.4,6.44) -- (8.9,0) -- cycle    ;
	%Straight Lines [id:da2517549122221281] 
	\draw [line width=0.75]  [dash pattern={on 0.84pt off 2.51pt}]  (244.75,150.75) -- (144.75,150.75) ;
	%Straight Lines [id:da7204302662170563] 
	\draw [line width=1.5]    (233,158.25) -- (301.75,158.25) ;
	\draw [shift={(229,158.25)}, rotate = 0] [fill={rgb, 255:red, 0; green, 0; blue, 0 }  ][line width=0.08]  [draw opacity=0] (13.4,-6.43) -- (0,0) -- (13.4,6.44) -- (8.9,0) -- cycle    ;

	% Text Node
	\draw (174.8,21.5) node    {$Q$};
	% Text Node
	\draw (275.4,17.3) node    {$q$};
	% Text Node
	\draw (310.2,26.4) node    {$\vec{F}( P)$};
	% Text Node
	\draw (211.1,29.3) node    {$\vec{u}_{r}$};
	% Text Node
	\draw  [fill={rgb, 255:red, 222; green, 222; blue, 222 }  ,fill opacity=1 ]  (174.75, 50.75) circle [x radius= 13.9, y radius= 13.9]   ;
	\draw (174.75,50.75) node    {$+$};
	% Text Node
	\draw  [fill={rgb, 255:red, 222; green, 222; blue, 222 }  ,fill opacity=1 ]  (274.75, 50.75) circle [x radius= 13.9, y radius= 13.9]   ;
	\draw (274.75,50.75) node    {$+$};
	% Text Node
	\draw (309,74) node    {$\vec{E}( P)$};
	% Text Node
	\draw (144.8,121.5) node    {$Q$};
	% Text Node
	\draw (306.4,124.3) node    {$q$};
	% Text Node
	\draw (275.2,126.4) node    {$\vec{F}( P)$};
	% Text Node
	\draw (177.1,130.8) node    {$\vec{u}_{r}$};
	% Text Node
	\draw  [fill={rgb, 255:red, 222; green, 222; blue, 222 }  ,fill opacity=1 ]  (144.75, 150.75) circle [x radius= 13.9, y radius= 13.9]   ;
	\draw (144.75,150.75) node    {$+$};
	% Text Node
	\draw  [fill={rgb, 255:red, 222; green, 222; blue, 222 }  ,fill opacity=1 ]  (304, 150.75) circle [x radius= 13.6, y radius= 13.6]   ;
	\draw (304,150.75) node    {$-$};
	% Text Node
	\draw (274,172) node    {$\vec{E}( P)$};


	\end{tikzpicture}
\end{figure}
\FloatBarrier
In base alla definizione di campo elettrico, se lo abbiamo già calcolato, la forza agente sulla carica $q$ sarà semplicemente:
\[
	\vec{F} (P) = q\vec{E} (P)
\]
Quando consideriamo distribuzioni di carica molto complesse sarà un approccio preferibile.
Il campo elettrico in se non è misurabile direttamente, quello che possiamo fare è misurare la forza che la sorgente esercita sulla carica esploratrice.
La procedura di misura del campo elettrico può alterare il risultato che voglio ottenere perché se avvicino la carica esploratrice alla sorgente, quest'ultima si allontana a sua volta. In realtà il modo migliore di definire questo campo elettrico è di scegliere una carica esploratrice piccolissima, di modo che la sua influenza sulla sorgente sia minimale. Quindi una definizione più rigorosa di campo elettrico è la seguente:
\[
	\vec{E} (P)=\lim_{q \to 0}\frac{\vec{F} (P)}{q}
\]
Dove il limite è da intendersi non in senso analitico, ma in senso fisico, cioé prendere $q$ più piccola possibile.

Immaginiamo di avere un insieme di tante cariche e di voler determinare il valore del campo elettrico complessivo che la carica sorgente genera nello spazio. Per fare questo studiamo la risultante $\vec{R}$ delle forze che le varie cariche esercitano sulla esploratrice.
\[
	\vec{E} (P) = \frac{\vec{R} (P)}{q} = \frac{ \frac{\sum_i Q_i q \vec{u}_{r_i}}{4\pi \varepsilon_0 r_i^2}}{q} =  \frac{\sum_i Q_i \vec{u}_{r_i}}{4\pi \varepsilon_0 r_i^2}
\]
Il campo elettrico complessivo prodotto da questa distribuzione è semplicemente la somma dei campi elettrici che ciascuna carica produce. In questo senso il principio di sovrapposizione vale anche per i singoli campi elettrici. $\vec{u}_r$ si può scrivere anche come il vettore $\vec{r}_i$ diviso il suo modulo.
\[
	\vec{u}_{r_i} = \frac{\vec{r}_i}{r_i}
\]
Con tal scrittura il campo elettrico nel punto $P$ sarà:
\[
	\vec{E} (P) = \sum_i \frac{Q\vec{r}_i}{4\pi \varepsilon_0 r_i^3}
\]




















\subsection{Rappresentazione cartesiana del campo elettrico}

Ci sono alcuni casi in cui il versore $\vec{u}_r$ e di conseguenza la distanza fra le cariche sono più complicati da calcolare. In questi casi ci si può anche cimentare su un calcolo basato sulla scomposizione cartesiana.
Vedendo le cose in questo modo potremmo anche pensare di introdurre i versori $\vec{u}_x$, $\vec{u}_y$, $\vec{u}_z$ e individuare la posizione della carica $Q$ e del punto $P$ con opportuni vettori come in figura.
\begin{figure}[htpb]
	\centering
	

	\tikzset{every picture/.style={line width=0.75pt}} %set default line width to 0.75pt        

	\begin{tikzpicture}[x=0.75pt,y=0.75pt,yscale=-1,xscale=1]
	%uncomment if require: \path (0,300); %set diagram left start at 0, and has height of 300

	%Straight Lines [id:da3359763454506417] 
	\draw    (199.5,160) -- (199.5,80) ;
	\draw [shift={(199.5,77)}, rotate = 450] [fill={rgb, 255:red, 0; green, 0; blue, 0 }  ][line width=0.08]  [draw opacity=0] (10.72,-5.15) -- (0,0) -- (10.72,5.15) -- (7.12,0) -- cycle    ;
	%Straight Lines [id:da41620874068539027] 
	\draw    (199.5,160) -- (259.93,196.45) ;
	\draw [shift={(262.5,198)}, rotate = 211.1] [fill={rgb, 255:red, 0; green, 0; blue, 0 }  ][line width=0.08]  [draw opacity=0] (10.72,-5.15) -- (0,0) -- (10.72,5.15) -- (7.12,0) -- cycle    ;
	%Straight Lines [id:da45968237944825696] 
	\draw    (199.5,160) -- (140.06,196.43) ;
	\draw [shift={(137.5,198)}, rotate = 328.5] [fill={rgb, 255:red, 0; green, 0; blue, 0 }  ][line width=0.08]  [draw opacity=0] (10.72,-5.15) -- (0,0) -- (10.72,5.15) -- (7.12,0) -- cycle    ;
	%Shape: Circle [id:dp9255160036546715] 
	\draw  [fill={rgb, 255:red, 0; green, 0; blue, 0 }  ,fill opacity=1 ] (260.5,87.75) .. controls (260.5,85.13) and (262.63,83) .. (265.25,83) .. controls (267.87,83) and (270,85.13) .. (270,87.75) .. controls (270,90.37) and (267.87,92.5) .. (265.25,92.5) .. controls (262.63,92.5) and (260.5,90.37) .. (260.5,87.75) -- cycle ;
	%Shape: Circle [id:dp6015853897023355] 
	\draw  [fill={rgb, 255:red, 0; green, 0; blue, 0 }  ,fill opacity=1 ] (303.5,140.75) .. controls (303.5,139.51) and (304.51,138.5) .. (305.75,138.5) .. controls (306.99,138.5) and (308,139.51) .. (308,140.75) .. controls (308,141.99) and (306.99,143) .. (305.75,143) .. controls (304.51,143) and (303.5,141.99) .. (303.5,140.75) -- cycle ;
	%Straight Lines [id:da3732851936378614] 
	\draw    (199.5,160) -- (258.98,94.47) ;
	\draw [shift={(261,92.25)}, rotate = 492.23] [fill={rgb, 255:red, 0; green, 0; blue, 0 }  ][line width=0.08]  [draw opacity=0] (10.72,-5.15) -- (0,0) -- (10.72,5.15) -- (7.12,0) -- cycle    ;
	%Straight Lines [id:da4541507595227321] 
	\draw    (199.5,160) -- (298.05,141.31) ;
	\draw [shift={(301,140.75)}, rotate = 529.26] [fill={rgb, 255:red, 0; green, 0; blue, 0 }  ][line width=0.08]  [draw opacity=0] (10.72,-5.15) -- (0,0) -- (10.72,5.15) -- (7.12,0) -- cycle    ;
	%Straight Lines [id:da09502568264571676] 
	\draw    (270,92.25) -- (300.24,133.82) ;
	\draw [shift={(302,136.25)}, rotate = 233.97] [fill={rgb, 255:red, 0; green, 0; blue, 0 }  ][line width=0.08]  [draw opacity=0] (10.72,-5.15) -- (0,0) -- (10.72,5.15) -- (7.12,0) -- cycle    ;

	% Text Node
	\draw (134,208) node    {$x$};
	% Text Node
	\draw (269.5,206.5) node    {$y$};
	% Text Node
	\draw (186,74) node    {$z$};
	% Text Node
	\draw (312,100) node    {$\vec{r} -\vec{r}_{i}$};
	% Text Node
	\draw (278,162.5) node    {$\vec{r}$};
	% Text Node
	\draw (233,96.5) node    {$\vec{r}_{i}$};
	% Text Node
	\draw (307.5,64) node    {$Q_{i}( x_{i} ,y_{i} ,z_{i})$};
	% Text Node
	\draw (347.5,149) node    {$P( x ,y,z)$};


	\end{tikzpicture}
\end{figure}
\FloatBarrier
Sia $Q_i(x_i,y_i,z_i)$ una generica carica al variare di $i$ e $P(x,y,z)$ un punto nello spazio. Consideriamo la distanza, vettoriale, fra questi due punti espressa da
\[
	\vec{r} - \vec{r}_i = (x-x_i )\vec{u}_x + (y-y_i )\vec{u}_y + (z-z_i)\vec{u}_z
\]
Allora il campo elettrico elettrico generato dalle cariche $Q_i$ sul punto $P$ vale
\begin{align*}
	\vec{E} (P) = E(x,y,z) &= \sum_i \frac{Q_i(\vec{r} - \vec{r}_i)}{4\pi \varepsilon_0 |\vec{r} -\vec{r}_i |^3} \\
	&= \sum_i \frac{Q_i [(x-x_i )\vec{u}_x + (y-y_i )\vec{u}_y + (z-z_i)\vec{u}_z]}{4\pi \varepsilon_0 [(x-x_i)^2 + (y-y_i)^2 + (z-z_i)^2]^{\frac{3}{2}}}
\end{align*}
A questo punto supponiamo di voler calcolare il campo elettrico generato da un oggetto in cui siamo riusciti a inserire un numero elevatissimo di cariche. Invece di vedere queste cariche come tantissime cariche puntiformi, possiamo fare l'ipotesi che la carica sia distribuita all'intero dell'oggetto con una certa continuità. Cercheremo di rappresentare cariche all'interno dell'oggetto con delle funzioni di distribuzione.
\begin{itemize}
	\item \emph{Densità di carica di volume} $\rho (x',y',z')$ \\
	Consideriamo ora un cubetto di volume $d\tau$ e carica $dq$ centrato in $P'(x',y',z')$. Sia $\vec{r}'$ variabile e $\vec{r}$ costante. Esprimiamo la carica $ dq = \rho (x',y',z') d\tau$ e il campo elettrico infinitesimo $ d\vec{E} = \frac{dq (\vec{r} -\vec{r}_i )}{4\pi \varepsilon_0 |\vec{r} -\vec{r}_i |^3} $ generato dalla distribuzione di carica nel punto $P$. Allora
	\begin{align*}
		\vec{E} (P) = \int_{\text{oggetto}} d\vec{E} &= \int_{\text{ogg.}} \frac{dq (\vec{r} -\vec{r}_i )}{4\pi \varepsilon_0 |\vec{r} -\vec{r}_i |^3}\\
		\vec{E} (P) &= \int_{\tau} \frac{\rho (x',y',z') d\tau (\vec{r} -\vec{r}_i )}{4\pi \varepsilon_0 |\vec{r} -\vec{r}_i |^3} \\
		\vec{E} (P) &= \int_{\tau} \frac{\rho (x',y',z') (\vec{r} -\vec{r}_i )}{4\pi \varepsilon_0 |\vec{r} -\vec{r}_i |^3}d\tau
	\end{align*}
	\begin{figure}[htpb]
		\centering
		

		\tikzset{every picture/.style={line width=0.75pt}} %set default line width to 0.75pt        

		\begin{tikzpicture}[x=0.75pt,y=0.75pt,yscale=-1,xscale=1]
		%uncomment if require: \path (0,300); %set diagram left start at 0, and has height of 300

		%Straight Lines [id:da8420671197768959] 
		\draw    (217.5,208.67) -- (217.5,128.67) ;
		\draw [shift={(217.5,125.67)}, rotate = 450] [fill={rgb, 255:red, 0; green, 0; blue, 0 }  ][line width=0.08]  [draw opacity=0] (10.72,-5.15) -- (0,0) -- (10.72,5.15) -- (7.12,0) -- cycle    ;
		%Straight Lines [id:da5234979003637599] 
		\draw    (217.5,208.67) -- (277.93,245.12) ;
		\draw [shift={(280.5,246.67)}, rotate = 211.1] [fill={rgb, 255:red, 0; green, 0; blue, 0 }  ][line width=0.08]  [draw opacity=0] (10.72,-5.15) -- (0,0) -- (10.72,5.15) -- (7.12,0) -- cycle    ;
		%Straight Lines [id:da5970517142159921] 
		\draw    (217.5,208.67) -- (158.06,245.1) ;
		\draw [shift={(155.5,246.67)}, rotate = 328.5] [fill={rgb, 255:red, 0; green, 0; blue, 0 }  ][line width=0.08]  [draw opacity=0] (10.72,-5.15) -- (0,0) -- (10.72,5.15) -- (7.12,0) -- cycle    ;
		%Shape: Circle [id:dp8904353238466189] 
		\draw  [fill={rgb, 255:red, 0; green, 0; blue, 0 }  ,fill opacity=1 ] (402.17,247.42) .. controls (402.17,246.17) and (403.17,245.17) .. (404.42,245.17) .. controls (405.66,245.17) and (406.67,246.17) .. (406.67,247.42) .. controls (406.67,248.66) and (405.66,249.67) .. (404.42,249.67) .. controls (403.17,249.67) and (402.17,248.66) .. (402.17,247.42) -- cycle ;
		%Straight Lines [id:da20882025926140635] 
		\draw    (217.5,208.67) -- (421.51,135.02) ;
		\draw [shift={(424.33,134)}, rotate = 520.15] [fill={rgb, 255:red, 0; green, 0; blue, 0 }  ][line width=0.08]  [draw opacity=0] (10.72,-5.15) -- (0,0) -- (10.72,5.15) -- (7.12,0) -- cycle    ;
		%Straight Lines [id:da8503611158657807] 
		\draw    (217.5,208.67) -- (399.23,246.8) ;
		\draw [shift={(402.17,247.42)}, rotate = 191.85] [fill={rgb, 255:red, 0; green, 0; blue, 0 }  ][line width=0.08]  [draw opacity=0] (10.72,-5.15) -- (0,0) -- (10.72,5.15) -- (7.12,0) -- cycle    ;
		%Straight Lines [id:da9131411337431448] 
		\draw    (437,145.33) -- (405.35,242.31) ;
		\draw [shift={(404.42,245.17)}, rotate = 288.08] [fill={rgb, 255:red, 0; green, 0; blue, 0 }  ][line width=0.08]  [draw opacity=0] (10.72,-5.15) -- (0,0) -- (10.72,5.15) -- (7.12,0) -- cycle    ;
		%Shape: Circle [id:dp1522505573812425] 
		\draw   (377.33,142.5) .. controls (377.33,102.83) and (409.49,70.67) .. (449.17,70.67) .. controls (488.84,70.67) and (521,102.83) .. (521,142.5) .. controls (521,182.17) and (488.84,214.33) .. (449.17,214.33) .. controls (409.49,214.33) and (377.33,182.17) .. (377.33,142.5) -- cycle ;
		%Shape: Cube [id:dp20550893096799427] 
		\draw   (429.17,125.83) -- (434.17,120.83) -- (454.17,120.83) -- (454.17,137.5) -- (449.17,142.5) -- (429.17,142.5) -- cycle ; \draw   (454.17,120.83) -- (449.17,125.83) -- (429.17,125.83) ; \draw   (449.17,125.83) -- (449.17,142.5) ;

		% Text Node
		\draw (155.33,256.67) node    {$x$};
		% Text Node
		\draw (287.5,255.17) node    {$y$};
		% Text Node
		\draw (204,122.67) node    {$z$};
		% Text Node
		\draw (436.67,228) node    {$\vec{r} -\vec{r}_{i}$};
		% Text Node
		\draw (340.67,215.83) node    {$\vec{r}$};
		% Text Node
		\draw (311,155.17) node    {$\vec{r}_{i}$};
		% Text Node
		\draw (442.17,106) node    {$dq$};
		% Text Node
		\draw (420.83,264.33) node    {$P( x ,y,z)$};
		% Text Node
		\draw (446.67,83.67) node    {$+$};
		% Text Node
		\draw (481.33,101.67) node    {$+$};
		% Text Node
		\draw (494.67,134.33) node    {$+$};
		% Text Node
		\draw (486.67,167.67) node    {$+$};
		% Text Node
		\draw (460.67,188.33) node    {$+$};
		% Text Node
		\draw (434.67,189.67) node    {$+$};
		% Text Node
		\draw (414,163.67) node    {$+$};
		% Text Node
		\draw (416,105.67) node    {$+$};
		% Text Node
		\draw (459.33,161) node    {$+$};
		% Text Node
		\draw (439.5,133.33) node    {$P_{i}$};
		% Text Node
		\draw (468.17,130.67) node    {$d\tau $};


		\end{tikzpicture}
	\end{figure}
	\FloatBarrier

	\item \emph{Densità di carica superficiale} $\sigma (x',y',z') $ \\
	Consideriamo ora una superficie infinitesima $dS$ e carica $dq$ in $P'(x',y',z')$. Sia $\vec{r}'$ variabile e $\vec{r}$ costante. Esprimiamo la carica $dq=\sigma(x',y',z') dS$ e il campo elettrico infinitesimo $ d\vec{E} = \frac{dq (\vec{r} -\vec{r}_i )}{4\pi \varepsilon_0 |\vec{r} -\vec{r}_i |^3} $ generato dalla distribuzione di carica nel punto $P$. Allora
	\begin{align*}
		\vec{E} (P) &= \int_{\Sigma} \frac{\sigma  (x',y',z') (\vec{r} -\vec{r}_i )}{4\pi \varepsilon_0 |\vec{r} -\vec{r}_i |^3}dS
	\end{align*}
	\begin{figure}[htpb]
		\centering
		

		\tikzset{every picture/.style={line width=0.75pt}} %set default line width to 0.75pt        

		\begin{tikzpicture}[x=0.75pt,y=0.75pt,yscale=-1,xscale=1]
		%uncomment if require: \path (0,300); %set diagram left start at 0, and has height of 300

		%Straight Lines [id:da06313925185671954] 
		\draw    (177.5,189.67) -- (177.5,109.67) ;
		\draw [shift={(177.5,106.67)}, rotate = 450] [fill={rgb, 255:red, 0; green, 0; blue, 0 }  ][line width=0.08]  [draw opacity=0] (10.72,-5.15) -- (0,0) -- (10.72,5.15) -- (7.12,0) -- cycle    ;
		%Straight Lines [id:da7777160466511888] 
		\draw    (177.5,189.67) -- (237.93,226.12) ;
		\draw [shift={(240.5,227.67)}, rotate = 211.1] [fill={rgb, 255:red, 0; green, 0; blue, 0 }  ][line width=0.08]  [draw opacity=0] (10.72,-5.15) -- (0,0) -- (10.72,5.15) -- (7.12,0) -- cycle    ;
		%Straight Lines [id:da5510445060802585] 
		\draw    (177.5,189.67) -- (118.06,226.1) ;
		\draw [shift={(115.5,227.67)}, rotate = 328.5] [fill={rgb, 255:red, 0; green, 0; blue, 0 }  ][line width=0.08]  [draw opacity=0] (10.72,-5.15) -- (0,0) -- (10.72,5.15) -- (7.12,0) -- cycle    ;
		%Shape: Circle [id:dp5550712782342335] 
		\draw  [fill={rgb, 255:red, 0; green, 0; blue, 0 }  ,fill opacity=1 ] (362.17,228.42) .. controls (362.17,227.17) and (363.17,226.17) .. (364.42,226.17) .. controls (365.66,226.17) and (366.67,227.17) .. (366.67,228.42) .. controls (366.67,229.66) and (365.66,230.67) .. (364.42,230.67) .. controls (363.17,230.67) and (362.17,229.66) .. (362.17,228.42) -- cycle ;
		%Straight Lines [id:da39423175034912705] 
		\draw    (177.5,189.67) -- (381.51,116.02) ;
		\draw [shift={(384.33,115)}, rotate = 520.15] [fill={rgb, 255:red, 0; green, 0; blue, 0 }  ][line width=0.08]  [draw opacity=0] (10.72,-5.15) -- (0,0) -- (10.72,5.15) -- (7.12,0) -- cycle    ;
		%Straight Lines [id:da7865572141870176] 
		\draw    (177.5,189.67) -- (359.23,227.8) ;
		\draw [shift={(362.17,228.42)}, rotate = 191.85] [fill={rgb, 255:red, 0; green, 0; blue, 0 }  ][line width=0.08]  [draw opacity=0] (10.72,-5.15) -- (0,0) -- (10.72,5.15) -- (7.12,0) -- cycle    ;
		%Straight Lines [id:da8495210953077466] 
		\draw    (397,126.33) -- (365.35,223.31) ;
		\draw [shift={(364.42,226.17)}, rotate = 288.08] [fill={rgb, 255:red, 0; green, 0; blue, 0 }  ][line width=0.08]  [draw opacity=0] (10.72,-5.15) -- (0,0) -- (10.72,5.15) -- (7.12,0) -- cycle    ;
		%Flowchart: Data [id:dp2491781918344611] 
		\draw   (358.6,48) -- (488.33,48) -- (458.4,174.67) -- (328.67,174.67) -- cycle ;
		%Shape: Square [id:dp7066276931584026] 
		\draw   (385.17,99.5) -- (409.17,99.5) -- (409.17,123.5) -- (385.17,123.5) -- cycle ;

		% Text Node
		\draw (115.33,237.67) node    {$x$};
		% Text Node
		\draw (247.5,236.17) node    {$y$};
		% Text Node
		\draw (164,103.67) node    {$z$};
		% Text Node
		\draw (408,199.67) node    {$\vec{r} -\vec{r}_{i}$};
		% Text Node
		\draw (300.67,196.83) node    {$\vec{r}$};
		% Text Node
		\draw (271,136.17) node    {$\vec{r}_{i}$};
		% Text Node
		\draw (398.17,84.33) node    {$dq$};
		% Text Node
		\draw (380.83,245.33) node    {$P( x ,y,z)$};
		% Text Node
		\draw (406.67,64.67) node    {$+$};
		% Text Node
		\draw (441.33,82.67) node    {$+$};
		% Text Node
		\draw (454.67,115.33) node    {$+$};
		% Text Node
		\draw (446.67,148.67) node    {$+$};
		% Text Node
		\draw (460.67,64.67) node    {$+$};
		% Text Node
		\draw (364.67,148) node    {$+$};
		% Text Node
		\draw (366,92.67) node    {$+$};
		% Text Node
		\draw (416,152) node    {$+$};
		% Text Node
		\draw (397.17,111.5) node    {$P_{i}$};
		% Text Node
		\draw (425.5,111.67) node    {$dS$};


		\end{tikzpicture}
	\end{figure}
	\FloatBarrier

	\item \emph{Densità di carica lineare} $\lambda (x',y',z') $ \\
	Consideriamo ora un segmento infinitesimo $dl$ e carica $dq$ in $P'(x',y',z')$. Sia $\vec{r}'$ variabile e $\vec{r}$ costante. Esprimiamo la carica $ dq = \lambda(x',y',z') dl $ e il campo elettrico infinitesimo $d\vec{E} = \frac{dq (\vec{r} -\vec{r}_i )}{4\pi \varepsilon_0 |\vec{r} -\vec{r}_i |^3}$ generato dalla distribuzione di carica nel punto $P$. Allora
	\begin{align*}
		\vec{E} (P) &= \int_{\Lambda} \frac{\lambda  (x',y',z') (\vec{r} -\vec{r}_i )}{4\pi \varepsilon_0 |\vec{r} -\vec{r}_i |^3}dl
	\end{align*}
	\begin{figure}[htpb]
		\centering
		

		\tikzset{every picture/.style={line width=0.75pt}} %set default line width to 0.75pt        

		\begin{tikzpicture}[x=0.75pt,y=0.75pt,yscale=-1,xscale=1]
		%uncomment if require: \path (0,300); %set diagram left start at 0, and has height of 300

		%Straight Lines [id:da2742946758986795] 
		\draw    (197.5,209.67) -- (197.5,129.67) ;
		\draw [shift={(197.5,126.67)}, rotate = 450] [fill={rgb, 255:red, 0; green, 0; blue, 0 }  ][line width=0.08]  [draw opacity=0] (10.72,-5.15) -- (0,0) -- (10.72,5.15) -- (7.12,0) -- cycle    ;
		%Straight Lines [id:da1582112085793861] 
		\draw    (197.5,209.67) -- (257.93,246.12) ;
		\draw [shift={(260.5,247.67)}, rotate = 211.1] [fill={rgb, 255:red, 0; green, 0; blue, 0 }  ][line width=0.08]  [draw opacity=0] (10.72,-5.15) -- (0,0) -- (10.72,5.15) -- (7.12,0) -- cycle    ;
		%Straight Lines [id:da5400686138190309] 
		\draw    (197.5,209.67) -- (138.06,246.1) ;
		\draw [shift={(135.5,247.67)}, rotate = 328.5] [fill={rgb, 255:red, 0; green, 0; blue, 0 }  ][line width=0.08]  [draw opacity=0] (10.72,-5.15) -- (0,0) -- (10.72,5.15) -- (7.12,0) -- cycle    ;
		%Shape: Circle [id:dp6260213863639099] 
		\draw  [fill={rgb, 255:red, 0; green, 0; blue, 0 }  ,fill opacity=1 ] (382.17,248.42) .. controls (382.17,247.17) and (383.17,246.17) .. (384.42,246.17) .. controls (385.66,246.17) and (386.67,247.17) .. (386.67,248.42) .. controls (386.67,249.66) and (385.66,250.67) .. (384.42,250.67) .. controls (383.17,250.67) and (382.17,249.66) .. (382.17,248.42) -- cycle ;
		%Straight Lines [id:da4980329673200876] 
		\draw    (197.5,209.67) -- (424.15,134.28) ;
		\draw [shift={(427,133.33)}, rotate = 521.6] [fill={rgb, 255:red, 0; green, 0; blue, 0 }  ][line width=0.08]  [draw opacity=0] (10.72,-5.15) -- (0,0) -- (10.72,5.15) -- (7.12,0) -- cycle    ;
		%Straight Lines [id:da318682975442671] 
		\draw    (197.5,209.67) -- (379.23,247.8) ;
		\draw [shift={(382.17,248.42)}, rotate = 191.85] [fill={rgb, 255:red, 0; green, 0; blue, 0 }  ][line width=0.08]  [draw opacity=0] (10.72,-5.15) -- (0,0) -- (10.72,5.15) -- (7.12,0) -- cycle    ;
		%Straight Lines [id:da939671023931534] 
		\draw    (426.33,138) -- (385.5,243.37) ;
		\draw [shift={(384.42,246.17)}, rotate = 291.18] [fill={rgb, 255:red, 0; green, 0; blue, 0 }  ][line width=0.08]  [draw opacity=0] (10.72,-5.15) -- (0,0) -- (10.72,5.15) -- (7.12,0) -- cycle    ;
		%Curve Lines [id:da9500837709709089] 
		\draw    (320,84.33) .. controls (452.33,60) and (436.33,234) .. (559,184.67) ;
		%Shape: Rectangle [id:dp5919414273162227] 
		\draw   (426.4,118.73) -- (443.71,138.57) -- (437.94,143.61) -- (420.62,123.77) -- cycle ;

		% Text Node
		\draw (135.33,257.67) node    {$x$};
		% Text Node
		\draw (267.5,256.17) node    {$y$};
		% Text Node
		\draw (184,123.67) node    {$z$};
		% Text Node
		\draw (425.33,205) node    {$\vec{r} -\vec{r}_{i}$};
		% Text Node
		\draw (320.67,216.83) node    {$\vec{r}$};
		% Text Node
		\draw (313,154.17) node    {$\vec{r}_{i}$};
		% Text Node
		\draw (431.5,147.67) node    {$dq$};
		% Text Node
		\draw (400.83,265.33) node    {$P( x ,y,z)$};
		% Text Node
		\draw (334,70) node    {$+$};
		% Text Node
		\draw (409.33,88.67) node    {$+$};
		% Text Node
		\draw (485.33,170.67) node    {$+$};
		% Text Node
		\draw (515.33,182.67) node    {$+$};
		% Text Node
		\draw (373.33,72.67) node    {$+$};
		% Text Node
		\draw (548,176.67) node    {$+$};
		% Text Node
		\draw (453.17,124.17) node    {$P_{i}$};
		% Text Node
		\draw (434.17,107.67) node    {$dl$};
		% Text Node
		\draw (466.67,152.67) node    {$+$};


		\end{tikzpicture}
	\end{figure}
	\FloatBarrier
\end{itemize}







































\section{Campo elettrico di un filo rettilineo infinito}

La dimostrazione suppone $\lambda$ costante.
\begin{figure}[htpb]
	\centering
	

	\tikzset{every picture/.style={line width=0.75pt}} %set default line width to 0.75pt        

	\begin{tikzpicture}[x=0.75pt,y=0.75pt,yscale=-0.8,xscale=0.8]
	%uncomment if require: \path (0,393); %set diagram left start at 0, and has height of 393

	%Shape: Rectangle [id:dp44310977854933165] 
	\draw   (140,81) -- (153,81) -- (153,345) -- (140,345) -- cycle ;
	%Straight Lines [id:da3097942485491145] 
	\draw  [dash pattern={on 0.84pt off 2.51pt}]  (132.67,210) -- (540.33,210) ;
	%Straight Lines [id:da7374934618400164] 
	\draw    (170,90) -- (497.4,278.5) ;
	\draw [shift={(500,280)}, rotate = 209.93] [fill={rgb, 255:red, 0; green, 0; blue, 0 }  ][line width=0.08]  [draw opacity=0] (10.72,-5.15) -- (0,0) -- (10.72,5.15) -- (7.12,0) -- cycle    ;
	%Straight Lines [id:da8123119214956673] 
	\draw    (170,330) -- (497.4,141.5) ;
	\draw [shift={(500,140)}, rotate = 510.07] [fill={rgb, 255:red, 0; green, 0; blue, 0 }  ][line width=0.08]  [draw opacity=0] (10.72,-5.15) -- (0,0) -- (10.72,5.15) -- (7.12,0) -- cycle    ;
	%Shape: Arc [id:dp1380381757942739] 
	\draw  [draw opacity=0] (327.56,239.18) .. controls (322.74,230.54) and (320,220.59) .. (320,210) .. controls (320,199.22) and (322.85,189.1) .. (327.83,180.35) -- (380,210) -- cycle ; \draw   (327.56,239.18) .. controls (322.74,230.54) and (320,220.59) .. (320,210) .. controls (320,199.22) and (322.85,189.1) .. (327.83,180.35) ;
	%Shape: Arc [id:dp4414106025648661] 
	\draw  [draw opacity=0] (431.46,179.14) .. controls (436.88,188.16) and (440,198.71) .. (440,210) .. controls (440,221.11) and (436.98,231.52) .. (431.71,240.45) -- (380,210) -- cycle ; \draw   (431.46,179.14) .. controls (436.88,188.16) and (440,198.71) .. (440,210) .. controls (440,221.11) and (436.98,231.52) .. (431.71,240.45) ;
	%Shape: Rectangle [id:dp8309920268613575] 
	\draw   (134,79) -- (160,79) -- (160,103) -- (134,103) -- cycle ;
	%Shape: Rectangle [id:dp19702818666456712] 
	\draw   (134,323) -- (160,323) -- (160,347) -- (134,347) -- cycle ;
	%Straight Lines [id:da37058240455784275] 
	\draw    (160,360) -- (380,360) ;
	\draw [shift={(380,360)}, rotate = 180] [color={rgb, 255:red, 0; green, 0; blue, 0 }  ][line width=0.75]    (0,5.59) -- (0,-5.59)   ;
	\draw [shift={(160,360)}, rotate = 180] [color={rgb, 255:red, 0; green, 0; blue, 0 }  ][line width=0.75]    (0,5.59) -- (0,-5.59)   ;
	%Shape: Circle [id:dp07344677800481558] 
	\draw  [fill={rgb, 255:red, 0; green, 0; blue, 0 }  ,fill opacity=1 ] (375,210) .. controls (375,208.34) and (376.34,207) .. (378,207) .. controls (379.66,207) and (381,208.34) .. (381,210) .. controls (381,211.66) and (379.66,213) .. (378,213) .. controls (376.34,213) and (375,211.66) .. (375,210) -- cycle ;
	%Straight Lines [id:da17272746827516716] 
	\draw    (82.67,210) -- (82.67,89.83) ;
	\draw [shift={(82.67,89.83)}, rotate = 450] [color={rgb, 255:red, 0; green, 0; blue, 0 }  ][line width=0.75]    (0,5.59) -- (0,-5.59)   ;
	\draw [shift={(82.67,210)}, rotate = 450] [color={rgb, 255:red, 0; green, 0; blue, 0 }  ][line width=0.75]    (0,5.59) -- (0,-5.59)   ;
	%Straight Lines [id:da06771818672423668] 
	\draw    (124,103) -- (124,79) ;
	\draw [shift={(124,79)}, rotate = 450] [color={rgb, 255:red, 0; green, 0; blue, 0 }  ][line width=0.75]    (0,5.59) -- (0,-5.59)   ;
	\draw [shift={(124,103)}, rotate = 450] [color={rgb, 255:red, 0; green, 0; blue, 0 }  ][line width=0.75]    (0,5.59) -- (0,-5.59)   ;
	%Shape: Rectangle [id:dp19188739834462787] 
	\draw  [dash pattern={on 0.84pt off 2.51pt}] (378,140) -- (500,140) -- (500,210) -- (378,210) -- cycle ;
	%Straight Lines [id:da2401580892794155] 
	\draw    (378,210) -- (378,143) ;
	\draw [shift={(378,140)}, rotate = 450] [fill={rgb, 255:red, 0; green, 0; blue, 0 }  ][line width=0.08]  [draw opacity=0] (10.72,-5.15) -- (0,0) -- (10.72,5.15) -- (7.12,0) -- cycle    ;
	%Straight Lines [id:da7211816669864399] 
	\draw    (378,210) -- (378,277) ;
	\draw [shift={(378,280)}, rotate = 270] [fill={rgb, 255:red, 0; green, 0; blue, 0 }  ][line width=0.08]  [draw opacity=0] (10.72,-5.15) -- (0,0) -- (10.72,5.15) -- (7.12,0) -- cycle    ;
	%Shape: Rectangle [id:dp6422672802074842] 
	\draw  [dash pattern={on 0.84pt off 2.51pt}] (378,210) -- (500,210) -- (500,280) -- (378,280) -- cycle ;
	%Straight Lines [id:da4577026773595012] 
	\draw    (378,210) -- (497,210) ;
	\draw [shift={(500,210)}, rotate = 180] [fill={rgb, 255:red, 0; green, 0; blue, 0 }  ][line width=0.08]  [draw opacity=0] (10.72,-5.15) -- (0,0) -- (10.72,5.15) -- (7.12,0) -- cycle    ;
	%Straight Lines [id:da21003998909834887] 
	\draw    (286,115) -- (337.5,115) ;
	\draw [shift={(340.5,115)}, rotate = 180] [fill={rgb, 255:red, 0; green, 0; blue, 0 }  ][line width=0.08]  [draw opacity=0] (10.72,-5.15) -- (0,0) -- (10.72,5.15) -- (7.12,0) -- cycle    ;

	% Text Node
	\draw (146.4,93.4) node    {$+$};
	% Text Node
	\draw (146.4,113.4) node    {$+$};
	% Text Node
	\draw (146.4,133.4) node    {$+$};
	% Text Node
	\draw (146.4,153.4) node    {$+$};
	% Text Node
	\draw (146.4,173.4) node    {$+$};
	% Text Node
	\draw (146.4,193.4) node    {$+$};
	% Text Node
	\draw (146.4,213.4) node    {$+$};
	% Text Node
	\draw (146.4,233.4) node    {$+$};
	% Text Node
	\draw (146.4,253.4) node    {$+$};
	% Text Node
	\draw (146.4,273.4) node    {$+$};
	% Text Node
	\draw (146.4,293.4) node    {$+$};
	% Text Node
	\draw (146.4,313.4) node    {$+$};
	% Text Node
	\draw (146.4,333.4) node    {$+$};
	% Text Node
	\draw (118.5,206) node    {$O$};
	% Text Node
	\draw (302.5,190) node    {$\vartheta $};
	% Text Node
	\draw (302.5,225) node    {$\vartheta $};
	% Text Node
	\draw (457.5,190) node    {$\vartheta $};
	% Text Node
	\draw (457.5,225) node    {$\vartheta $};
	% Text Node
	\draw (388.5,189.33) node    {$P$};
	% Text Node
	\draw (271.83,345.33) node    {$a$};
	% Text Node
	\draw (73.83,144.67) node    {$l$};
	% Text Node
	\draw (109.17,89.33) node    {$dl$};
	% Text Node
	\draw (520.17,134) node    {$d\vec{E}_{1}$};
	% Text Node
	\draw (518.83,281.33) node    {$d\vec{E}_{2}$};
	% Text Node
	\draw (517.5,194) node    {$d\vec{E}$};
	% Text Node
	\draw (310.17,100) node    {$\vec{u}_{n}$};


	\end{tikzpicture}
\end{figure}
\FloatBarrier
L'idea per fare questo tipo di dimostrazioni è ricordarsi quale sia l'angolo ``migliore'' da prendere come parametro, poi dare dei nomi appropriati a tutte le altre variabili e infine cercare una relazione tra le variabili e l'angolo, il quale infine andrà integrato.
\begin{gather*}
	|d\vec{E}_1|=\frac{dq}{4\pi \varepsilon_0 r^2}=\frac{\lambda dl}{4\pi \varepsilon_0 r^2} \\
	d\vec{E} = d\vec{E}_1+d\vec{E}_2 = 2|d\vec{E}_1 |\cos \vartheta \:\vec{u}_n = \frac{\lambda \, dl \, \cos \vartheta  \, \vec{u}_n}{2\pi \varepsilon_0 r^2}
\end{gather*}
Ora esplicitiamo gli altri termini in funzione dell'angolo $\vartheta$:
\begin{gather*}
	a=r\cos \vartheta  \; \implies  \; r=\frac{a}{\cos  \vartheta} \\
	\frac{l}{a}=\tan \vartheta \implies l=a \tan \vartheta  \implies dl=\frac{dl}{d\vartheta}d\vartheta =\frac{a}{\cos^2 \vartheta}d\vartheta
\end{gather*}
Sostituendo in $d\vec{E}$,
\begin{align*}
	d\vec{E} &= \frac{\lambda\vec{u}_n}{2\pi \varepsilon_0} \frac{ad\vartheta}{\cos^2 \vartheta} \frac{\cos \vartheta \vec{u}_n}{\left( \frac{a}{\cos \vartheta} \right)^2} = \frac{\lambda\cos \vartheta d\vartheta}{2\pi \varepsilon_0 a}\vec{u}_n \\
	\vec{E} &= \int_0^{\pi /2} d\vec{E} = \frac{\lambda \vec{u}_n}{2\pi \varepsilon_0 a} \underbrace{\int_0^{\pi /2} \cos \vartheta d\vartheta}_1 \tag*{integrando} \\
	\Aboxed{\vec{E} &= \frac{\lambda}{2\pi \varepsilon_0 a}\vec{u}_n}
\end{align*}







































\section{Linee di forza del campo elettrostatico}

L'introduzione del concetto di campo elettrostatico mette in evidenza che la presenza di un sistema di cariche modifica lo spazio circostante, nel senso che una carica di prova posta in un qualsiasi punto risente della forza. Partendo da una generica posizione e muovendosi per tratti infinitesimi successivi, ciascuno parallelo e concorde al campo elettrostatico in quel dato punto, si ottiene una linea che è detta \textbf{linea di forza} o \textbf{linea di campo}. Essa è orientata e in ogni suo punto risulta tangente e di verso concorde al campo $\vec{v}(P)$ valutato nello stesso punto.
\begin{figure}[htpb]
	\centering
	

	\tikzset{every picture/.style={line width=0.75pt}} %set default line width to 0.75pt        

	\begin{tikzpicture}[x=0.75pt,y=0.75pt,yscale=-1,xscale=1]
	%uncomment if require: \path (0,300); %set diagram left start at 0, and has height of 300

	%Curve Lines [id:da886214435685617] 
	\draw    (105.5,166) .. controls (163.5,122) and (249.5,74) .. (397.5,82) ;
	\draw [shift={(244.38,96.27)}, rotate = 524.4] [fill={rgb, 255:red, 0; green, 0; blue, 0 }  ][line width=0.08]  [draw opacity=0] (10.72,-5.15) -- (0,0) -- (10.72,5.15) -- (7.12,0) -- cycle    ;
	%Straight Lines [id:da4705565705020258] 
	\draw    (142.75,139.75) -- (237.49,77.89) ;
	\draw [shift={(240,76.25)}, rotate = 506.86] [fill={rgb, 255:red, 0; green, 0; blue, 0 }  ][line width=0.08]  [draw opacity=0] (10.72,-5.15) -- (0,0) -- (10.72,5.15) -- (7.12,0) -- cycle    ;
	%Shape: Circle [id:dp5625214438520421] 
	\draw  [fill={rgb, 255:red, 0; green, 0; blue, 0 }  ,fill opacity=1 ] (141,139.75) .. controls (141,138.78) and (141.78,138) .. (142.75,138) .. controls (143.72,138) and (144.5,138.78) .. (144.5,139.75) .. controls (144.5,140.72) and (143.72,141.5) .. (142.75,141.5) .. controls (141.78,141.5) and (141,140.72) .. (141,139.75) -- cycle ;

	% Text Node
	\draw (203,74.5) node    {$\vec{v}( P)$};
	% Text Node
	\draw (151.5,148.5) node    {$P$};


	\end{tikzpicture}
\end{figure}
\FloatBarrier
Se vogliamo rappresentare un campo vettoriale graficamente avremo un insieme di linee di flusso orientate che rappresentano l'andamento del campo.
Evidenziamo le proprietà principali delle linee di forza:
\begin{itemize}
	\item Per ogni punto $P$ in cui il campo $\vec{v}(P)$ definito è non nullo, passa una ed una sola linea di flusso (il campo in un punto è univocamente determinato).
	\item Nei casi particolari in cui più linee di flusso convergono verso un punto $P$, $P$ è sorgente negativa o pozzo delle linee di flusso.
	\item Nei casi particolari in cui più linee di flusso si diramano a partire da un punto $P$, diremo che quel punto è una sorgente positiva delle linee di flusso.
	\item Per campi vettoriali uniformi, in cui modulo direzione e verso sono costanti nello spazio, le linee di flusso sono linee rette parallele.
	\item Le linee di forza si addensano dove l'intensità del campo è maggiore.
	\item Una linea di forza in ogni suo punto è tangente e concorde al campo in quel punto.
	\item Le linee di forza hanno origine dalle cariche positive e terminano sulle cariche negative. Qualora ci siano solo cariche di uno stesso segno le linee di forza si chiudono all'infinito.
	\item Nel caso di cariche di segno opposto, ma eguali in modulo, tutte le linee che partono dalle cariche positive si chiudono sulle cariche negative, alcune passando eventualmente per l'infinito. Se invece le cariche non sono eguali in modulo, alcune linee terminano o provengono dall'infinito.
\end{itemize}
\begin{figure}[htpb]
	\centering
	

	\tikzset{every picture/.style={line width=0.75pt}} %set default line width to 0.75pt        

	\begin{tikzpicture}[x=0.75pt,y=0.75pt,yscale=-1,xscale=1]
	%uncomment if require: \path (0,300); %set diagram left start at 0, and has height of 300

	%Straight Lines [id:da01701154185838294] 
	\draw    (175.42,150.42) -- (235.7,90.14) ;
	\draw [shift={(237.82,88.01)}, rotate = 495] [fill={rgb, 255:red, 0; green, 0; blue, 0 }  ][line width=0.08]  [draw opacity=0] (10.72,-5.15) -- (0,0) -- (10.72,5.15) -- (7.12,0) -- cycle    ;
	%Straight Lines [id:da12523204391835363] 
	\draw    (115.14,210.7) -- (175.42,150.42) ;
	\draw [shift={(113.01,212.82)}, rotate = 315] [fill={rgb, 255:red, 0; green, 0; blue, 0 }  ][line width=0.08]  [draw opacity=0] (10.72,-5.15) -- (0,0) -- (10.72,5.15) -- (7.12,0) -- cycle    ;
	%Straight Lines [id:da48631543354363305] 
	\draw    (175.42,150.42) -- (235.7,210.7) ;
	\draw [shift={(237.82,212.82)}, rotate = 225] [fill={rgb, 255:red, 0; green, 0; blue, 0 }  ][line width=0.08]  [draw opacity=0] (10.72,-5.15) -- (0,0) -- (10.72,5.15) -- (7.12,0) -- cycle    ;
	%Straight Lines [id:da7261478359039628] 
	\draw    (115.14,90.14) -- (175.42,150.42) ;
	\draw [shift={(113.01,88.01)}, rotate = 45] [fill={rgb, 255:red, 0; green, 0; blue, 0 }  ][line width=0.08]  [draw opacity=0] (10.72,-5.15) -- (0,0) -- (10.72,5.15) -- (7.12,0) -- cycle    ;
	%Straight Lines [id:da7554190837838399] 
	\draw    (175.42,150.42) -- (260.67,150.42) ;
	\draw [shift={(263.67,150.42)}, rotate = 180] [fill={rgb, 255:red, 0; green, 0; blue, 0 }  ][line width=0.08]  [draw opacity=0] (10.72,-5.15) -- (0,0) -- (10.72,5.15) -- (7.12,0) -- cycle    ;
	%Straight Lines [id:da014485419811356648] 
	\draw    (90.17,150.42) -- (175.42,150.42) ;
	\draw [shift={(87.17,150.42)}, rotate = 360] [fill={rgb, 255:red, 0; green, 0; blue, 0 }  ][line width=0.08]  [draw opacity=0] (10.72,-5.15) -- (0,0) -- (10.72,5.15) -- (7.12,0) -- cycle    ;
	%Straight Lines [id:da9699072893325926] 
	\draw    (175.42,150.42) -- (175.42,235.67) ;
	\draw [shift={(175.42,238.67)}, rotate = 270] [fill={rgb, 255:red, 0; green, 0; blue, 0 }  ][line width=0.08]  [draw opacity=0] (10.72,-5.15) -- (0,0) -- (10.72,5.15) -- (7.12,0) -- cycle    ;
	%Straight Lines [id:da5695531836183898] 
	\draw    (175.42,65.17) -- (175.42,150.42) ;
	\draw [shift={(175.42,62.17)}, rotate = 90] [fill={rgb, 255:red, 0; green, 0; blue, 0 }  ][line width=0.08]  [draw opacity=0] (10.72,-5.15) -- (0,0) -- (10.72,5.15) -- (7.12,0) -- cycle    ;
	%Straight Lines [id:da1847540335451905] 
	\draw    (435.34,138.56) -- (486.72,87.17) ;
	\draw [shift={(433.21,140.68)}, rotate = 315] [fill={rgb, 255:red, 0; green, 0; blue, 0 }  ][line width=0.08]  [draw opacity=0] (10.72,-5.15) -- (0,0) -- (10.72,5.15) -- (7.12,0) -- cycle    ;
	%Straight Lines [id:da5233688237452658] 
	\draw    (360.24,213.66) -- (411.62,162.27) ;
	\draw [shift={(413.75,160.15)}, rotate = 495] [fill={rgb, 255:red, 0; green, 0; blue, 0 }  ][line width=0.08]  [draw opacity=0] (10.72,-5.15) -- (0,0) -- (10.72,5.15) -- (7.12,0) -- cycle    ;
	%Straight Lines [id:da16846293285064373] 
	\draw    (435.34,162.27) -- (486.72,213.66) ;
	\draw [shift={(433.21,160.15)}, rotate = 45] [fill={rgb, 255:red, 0; green, 0; blue, 0 }  ][line width=0.08]  [draw opacity=0] (10.72,-5.15) -- (0,0) -- (10.72,5.15) -- (7.12,0) -- cycle    ;
	%Straight Lines [id:da05635078284520678] 
	\draw    (360.24,87.17) -- (411.62,138.56) ;
	\draw [shift={(413.75,140.68)}, rotate = 225] [fill={rgb, 255:red, 0; green, 0; blue, 0 }  ][line width=0.08]  [draw opacity=0] (10.72,-5.15) -- (0,0) -- (10.72,5.15) -- (7.12,0) -- cycle    ;
	%Straight Lines [id:da1612606104555494] 
	\draw    (440.58,150.42) -- (513.67,150.42) ;
	\draw [shift={(437.58,150.42)}, rotate = 0] [fill={rgb, 255:red, 0; green, 0; blue, 0 }  ][line width=0.08]  [draw opacity=0] (10.72,-5.15) -- (0,0) -- (10.72,5.15) -- (7.12,0) -- cycle    ;
	%Straight Lines [id:da4849446243609239] 
	\draw    (334.17,150.42) -- (407.25,150.42) ;
	\draw [shift={(410.25,150.42)}, rotate = 540] [fill={rgb, 255:red, 0; green, 0; blue, 0 }  ][line width=0.08]  [draw opacity=0] (10.72,-5.15) -- (0,0) -- (10.72,5.15) -- (7.12,0) -- cycle    ;
	%Straight Lines [id:da6572115408819028] 
	\draw    (423.48,167.65) -- (423.48,239.73) ;
	\draw [shift={(423.48,164.65)}, rotate = 90] [fill={rgb, 255:red, 0; green, 0; blue, 0 }  ][line width=0.08]  [draw opacity=0] (10.72,-5.15) -- (0,0) -- (10.72,5.15) -- (7.12,0) -- cycle    ;
	%Straight Lines [id:da9281080036930767] 
	\draw    (423.48,61.1) -- (423.48,134.19) ;
	\draw [shift={(423.48,137.19)}, rotate = 270] [fill={rgb, 255:red, 0; green, 0; blue, 0 }  ][line width=0.08]  [draw opacity=0] (10.72,-5.15) -- (0,0) -- (10.72,5.15) -- (7.12,0) -- cycle    ;

	% Text Node
	\draw  [fill={rgb, 255:red, 222; green, 222; blue, 222 }  ,fill opacity=1 ]  (174.75, 150.75) circle [x radius= 13.9, y radius= 13.9]   ;
	\draw (174.75,150.75) node    {$+$};
	% Text Node
	\draw  [fill={rgb, 255:red, 222; green, 222; blue, 222 }  ,fill opacity=1 ]  (424, 150.75) circle [x radius= 13.6, y radius= 13.6]   ;
	\draw (424,150.75) node    {$-$};


	\end{tikzpicture}
\end{figure}
\FloatBarrier
Il metodo delle linee di flusso consente anche di capire come è fatto il campo elettrico nel caso di sistemi di cariche complesse. Consideriamo il caso di due cariche uguali di segno opposto. Possiamo immaginare di congiungere le linee come in figura per visualizzare il campo elettrico generato da un bipolo elettrico.
\begin{figure}[htpb]
	\centering
	

	\tikzset{every picture/.style={line width=0.75pt}} %set default line width to 0.75pt        

	\begin{tikzpicture}[x=0.75pt,y=0.75pt,yscale=-1,xscale=1]
	%uncomment if require: \path (0,300); %set diagram left start at 0, and has height of 300

	%Straight Lines [id:da09226733946766874] 
	\draw    (168.42,159.42) -- (228.7,99.14) ;
	\draw [shift={(230.82,97.01)}, rotate = 495] [fill={rgb, 255:red, 0; green, 0; blue, 0 }  ][line width=0.08]  [draw opacity=0] (10.72,-5.15) -- (0,0) -- (10.72,5.15) -- (7.12,0) -- cycle    ;
	%Straight Lines [id:da7989832300210724] 
	\draw    (108.14,219.7) -- (168.42,159.42) ;
	\draw [shift={(106.01,221.82)}, rotate = 315] [fill={rgb, 255:red, 0; green, 0; blue, 0 }  ][line width=0.08]  [draw opacity=0] (10.72,-5.15) -- (0,0) -- (10.72,5.15) -- (7.12,0) -- cycle    ;
	%Straight Lines [id:da40069576982828825] 
	\draw    (168.42,159.42) -- (228.7,219.7) ;
	\draw [shift={(230.82,221.82)}, rotate = 225] [fill={rgb, 255:red, 0; green, 0; blue, 0 }  ][line width=0.08]  [draw opacity=0] (10.72,-5.15) -- (0,0) -- (10.72,5.15) -- (7.12,0) -- cycle    ;
	%Straight Lines [id:da5830973681811298] 
	\draw    (108.14,99.14) -- (168.42,159.42) ;
	\draw [shift={(106.01,97.01)}, rotate = 45] [fill={rgb, 255:red, 0; green, 0; blue, 0 }  ][line width=0.08]  [draw opacity=0] (10.72,-5.15) -- (0,0) -- (10.72,5.15) -- (7.12,0) -- cycle    ;
	%Straight Lines [id:da9186612483958567] 
	\draw    (168.42,159.42) -- (253.67,159.42) ;
	\draw [shift={(256.67,159.42)}, rotate = 180] [fill={rgb, 255:red, 0; green, 0; blue, 0 }  ][line width=0.08]  [draw opacity=0] (10.72,-5.15) -- (0,0) -- (10.72,5.15) -- (7.12,0) -- cycle    ;
	%Straight Lines [id:da6761948854902533] 
	\draw    (83.17,159.42) -- (168.42,159.42) ;
	\draw [shift={(80.17,159.42)}, rotate = 360] [fill={rgb, 255:red, 0; green, 0; blue, 0 }  ][line width=0.08]  [draw opacity=0] (10.72,-5.15) -- (0,0) -- (10.72,5.15) -- (7.12,0) -- cycle    ;
	%Straight Lines [id:da30542672054099773] 
	\draw    (168.42,159.42) -- (168.42,244.67) ;
	\draw [shift={(168.42,247.67)}, rotate = 270] [fill={rgb, 255:red, 0; green, 0; blue, 0 }  ][line width=0.08]  [draw opacity=0] (10.72,-5.15) -- (0,0) -- (10.72,5.15) -- (7.12,0) -- cycle    ;
	%Straight Lines [id:da9339624700136409] 
	\draw    (168.42,74.17) -- (168.42,159.42) ;
	\draw [shift={(168.42,71.17)}, rotate = 90] [fill={rgb, 255:red, 0; green, 0; blue, 0 }  ][line width=0.08]  [draw opacity=0] (10.72,-5.15) -- (0,0) -- (10.72,5.15) -- (7.12,0) -- cycle    ;
	%Straight Lines [id:da9122656922012751] 
	\draw    (428.34,147.56) -- (479.72,96.17) ;
	\draw [shift={(426.21,149.68)}, rotate = 315] [fill={rgb, 255:red, 0; green, 0; blue, 0 }  ][line width=0.08]  [draw opacity=0] (10.72,-5.15) -- (0,0) -- (10.72,5.15) -- (7.12,0) -- cycle    ;
	%Straight Lines [id:da33197275662358083] 
	\draw    (353.24,222.66) -- (404.62,171.27) ;
	\draw [shift={(406.75,169.15)}, rotate = 495] [fill={rgb, 255:red, 0; green, 0; blue, 0 }  ][line width=0.08]  [draw opacity=0] (10.72,-5.15) -- (0,0) -- (10.72,5.15) -- (7.12,0) -- cycle    ;
	%Straight Lines [id:da6193410325711912] 
	\draw    (428.34,171.27) -- (479.72,222.66) ;
	\draw [shift={(426.21,169.15)}, rotate = 45] [fill={rgb, 255:red, 0; green, 0; blue, 0 }  ][line width=0.08]  [draw opacity=0] (10.72,-5.15) -- (0,0) -- (10.72,5.15) -- (7.12,0) -- cycle    ;
	%Straight Lines [id:da3379499451012773] 
	\draw    (353.24,96.17) -- (404.62,147.56) ;
	\draw [shift={(406.75,149.68)}, rotate = 225] [fill={rgb, 255:red, 0; green, 0; blue, 0 }  ][line width=0.08]  [draw opacity=0] (10.72,-5.15) -- (0,0) -- (10.72,5.15) -- (7.12,0) -- cycle    ;
	%Straight Lines [id:da8633902170868257] 
	\draw    (433.58,159.42) -- (506.67,159.42) ;
	\draw [shift={(430.58,159.42)}, rotate = 0] [fill={rgb, 255:red, 0; green, 0; blue, 0 }  ][line width=0.08]  [draw opacity=0] (10.72,-5.15) -- (0,0) -- (10.72,5.15) -- (7.12,0) -- cycle    ;
	%Straight Lines [id:da6449899890477775] 
	\draw    (327.17,159.42) -- (400.25,159.42) ;
	\draw [shift={(403.25,159.42)}, rotate = 540] [fill={rgb, 255:red, 0; green, 0; blue, 0 }  ][line width=0.08]  [draw opacity=0] (10.72,-5.15) -- (0,0) -- (10.72,5.15) -- (7.12,0) -- cycle    ;
	%Straight Lines [id:da9631546207388431] 
	\draw    (416.48,176.65) -- (416.48,248.73) ;
	\draw [shift={(416.48,173.65)}, rotate = 90] [fill={rgb, 255:red, 0; green, 0; blue, 0 }  ][line width=0.08]  [draw opacity=0] (10.72,-5.15) -- (0,0) -- (10.72,5.15) -- (7.12,0) -- cycle    ;
	%Straight Lines [id:da3834974996926863] 
	\draw    (416.48,70.1) -- (416.48,143.19) ;
	\draw [shift={(416.48,146.19)}, rotate = 270] [fill={rgb, 255:red, 0; green, 0; blue, 0 }  ][line width=0.08]  [draw opacity=0] (10.72,-5.15) -- (0,0) -- (10.72,5.15) -- (7.12,0) -- cycle    ;
	%Curve Lines [id:da13623299495386587] 
	\draw    (230.82,97.01) .. controls (270.82,67.01) and (309.5,67) .. (352.24,95.17) ;
	\draw [shift={(291.1,74.27)}, rotate = 537.6700000000001] [fill={rgb, 255:red, 0; green, 0; blue, 0 }  ][line width=0.08]  [draw opacity=0] (10.72,-5.15) -- (0,0) -- (10.72,5.15) -- (7.12,0) -- cycle    ;
	%Curve Lines [id:da2452924867095878] 
	\draw    (256.67,159.42) .. controls (290.67,159.17) and (305,159.5) .. (325.17,159.42) ;
	\draw [shift={(290.99,159.34)}, rotate = 180.12] [fill={rgb, 255:red, 0; green, 0; blue, 0 }  ][line width=0.08]  [draw opacity=0] (10.72,-5.15) -- (0,0) -- (10.72,5.15) -- (7.12,0) -- cycle    ;
	%Curve Lines [id:da15092657275075516] 
	\draw    (352.24,222.98) .. controls (312.24,252.98) and (273.56,249.99) .. (230.82,221.82) ;
	\draw [shift={(291.35,244.17)}, rotate = 180.23] [fill={rgb, 255:red, 0; green, 0; blue, 0 }  ][line width=0.08]  [draw opacity=0] (10.72,-5.15) -- (0,0) -- (10.72,5.15) -- (7.12,0) -- cycle    ;

	% Text Node
	\draw  [fill={rgb, 255:red, 222; green, 222; blue, 222 }  ,fill opacity=1 ]  (167.75, 159.75) circle [x radius= 13.9, y radius= 13.9]   ;
	\draw (167.75,159.75) node    {$+$};
	% Text Node
	\draw  [fill={rgb, 255:red, 222; green, 222; blue, 222 }  ,fill opacity=1 ]  (417, 159.75) circle [x radius= 13.6, y radius= 13.6]   ;
	\draw (417,159.75) node    {$-$};


	\end{tikzpicture}
\end{figure}
\FloatBarrier
Supponendo che il campo vettoriale sia noto, possiamo procedere con questo ragionamento. Possiamo considerare un piccolo spostamento da $P$ a $P'$. Sostanzialmente $d\vec{r}$ si confonde con l'arco della linea di flusso. Mano a mano che $P'$ si avvicina $P$ diventa tangente all'andamento della linea. Questo vuol dire che possiamoc immaginare di rappresentare lo spostamento $d\vec{r}$ come:
\[
	\vec{r} = dx\vec{u}_x+dy\vec{u}_y+dz\vec{u}_z
\]
E se facciamo l'ipotesi che $V$ ed $\vec{r}$ sono paralleli, possiamo scrivere la similitudine tra le componenti di $V$ e le componenti di $\vec{r}$.
\[
	\boxed{\frac{dx}{V_x} = \frac{dy}{V_y} = \frac{dz}{V_z}}
\]







































\section{Elementi di geometria e calcolo differenziale vettoriale}

\subsection{Angolo nel piano}
L'angolo nel piano è una parte di piano, delimitata da due semirette. Si definiscono i radianti come rapporto tra l'arco del cerchio e il raggio del cerchio.
\[
	\vartheta = \frac{l}{r} \; \implies \; \vartheta_{\text{giro}}  = \frac{2\pi r}{r} = 2\pi \;\; \text{(radianti)}
\]




















\subsection{Angolo solido}
Possiamo immaginare di estendere il concetto di angolo nel piano alle tre dimensioni. Immaginiamo di tracciare un cono con vertice nel punto $O$. Tale cono, analogamente al caso del piano, delimita una porzione di spazio al suo interno. Immaginiamo di tracciare una sfera che abbia
centro coincidente in $O$. Dall'intersezione con il cono otterremo una calotta sferica di area $A$. Definiremo l'ampiezza dell'angolo solido come la quantità pari a questo rapporto:
\[
	\Omega = \frac{A}{r^2} \; \implies \; \Omega_{\text{giro}} = \frac{4\pi r^2}{r^2} =4\pi \;\; \text{(steradianti)}
\]
\begin{figure}[htpb]
	\centering
	

	\tikzset{every picture/.style={line width=0.75pt}} %set default line width to 0.75pt        

	\begin{tikzpicture}[x=0.75pt,y=0.75pt,yscale=-1,xscale=1]
	%uncomment if require: \path (0,300); %set diagram left start at 0, and has height of 300

	%Straight Lines [id:da8133407336889511] 
	\draw    (69,187) -- (238,187) ;
	%Straight Lines [id:da9927555196654625] 
	\draw    (69,187) -- (239.5,92.25) ;
	%Shape: Arc [id:dp4648347309681775] 
	\draw  [draw opacity=0] (178.08,125.91) .. controls (188.18,143.91) and (193.96,164.66) .. (194,186.76) -- (69,187) -- cycle ; \draw   (178.08,125.91) .. controls (188.18,143.91) and (193.96,164.66) .. (194,186.76) ;
	%Straight Lines [id:da09140360982082418] 
	\draw    (309,187) -- (478,187) ;
	%Straight Lines [id:da1760952495285375] 
	\draw    (309,187) -- (479.5,92.25) ;
	%Shape: Ellipse [id:dp9119470894722195] 
	\draw   (418.37,126.17) .. controls (422.16,125.1) and (429.04,137.72) .. (433.74,154.35) .. controls (438.44,170.99) and (439.18,185.34) .. (435.4,186.41) .. controls (431.61,187.48) and (424.73,174.86) .. (420.03,158.23) .. controls (415.33,141.59) and (414.59,127.24) .. (418.37,126.17) -- cycle ;

	% Text Node
	\draw (204,148.5) node    {$l$};
	% Text Node
	\draw (138.5,198.5) node    {$r$};
	% Text Node
	\draw (462.5,150) node    {$A_{\text{calotta}}$};
	% Text Node
	\draw (378.5,198.5) node    {$r$};
	% Text Node
	\draw (161,232) node   [align=left] {angolo nel piano};
	% Text Node
	\draw (391,232) node   [align=left] {angolo solido};
	% Text Node
	\draw (57,189) node    {$O$};
	% Text Node
	\draw (296,191) node    {$O$};


	\end{tikzpicture}
\end{figure}
\FloatBarrier
Consideriamo un angolo solido di ampiezza infinitesima, $d\Omega$. Se consideriamo l'area intercettata dalla superficie sferica, essa sarà pari a:
\[
	A=d\Omega \, r^2
\]
Possiamo anche individuare la normale, il versore perpendicolare a questo tratto di superficie.
\begin{figure}[htpb]
	\centering
	

	\tikzset{every picture/.style={line width=0.75pt}} %set default line width to 0.75pt        

	\begin{tikzpicture}[x=0.75pt,y=0.75pt,yscale=-1,xscale=1]
	%uncomment if require: \path (0,300); %set diagram left start at 0, and has height of 300

	%Shape: Polygon Curved [id:ds5242701370299456] 
	\draw   (216.33,82.67) .. controls (257,89.33) and (291.67,118.67) .. (323.67,147.33) .. controls (303.67,154) and (292.33,160.67) .. (274.33,174) .. controls (243.67,146) and (207.67,125.33) .. (170.33,115.33) .. controls (183.67,98) and (197.67,90.67) .. (216.33,82.67) -- cycle ;
	%Straight Lines [id:da719989530014286] 
	\draw    (170.33,115.33) -- (193,254) ;
	%Shape: Boxed Line [id:dp9909676259845537] 
	\draw    (379.67,101.62) -- (193,254) ;
	%Shape: Boxed Line [id:dp7130967066860716] 
	\draw    (336.33,113.02) -- (193,254) ;
	%Shape: Polygon Curved [id:ds5438868770758298] 
	\draw   (216.33,82.67) .. controls (259,77.33) and (345,73.33) .. (379.67,101.62) .. controls (357,103.33) and (348.33,107.33) .. (336.33,113.02) .. controls (285,98) and (206.33,105.33) .. (170.33,115.33) .. controls (183.67,98) and (197.67,90.67) .. (216.33,82.67) -- cycle ;
	%Straight Lines [id:da2179443455009622] 
	\draw  [dash pattern={on 0.84pt off 2.51pt}]  (216.33,82.67) -- (193,254) ;
	%Straight Lines [id:da03071624194168887] 
	\draw    (277.12,40.33) -- (279,89.33) ;
	\draw [shift={(277,37.33)}, rotate = 87.8] [fill={rgb, 255:red, 0; green, 0; blue, 0 }  ][line width=0.08]  [draw opacity=0] (10.72,-5.15) -- (0,0) -- (10.72,5.15) -- (7.12,0) -- cycle    ;
	%Straight Lines [id:da3105470001085451] 
	\draw    (312.62,41.13) -- (279,89.33) ;
	\draw [shift={(314.33,38.67)}, rotate = 124.89] [fill={rgb, 255:red, 0; green, 0; blue, 0 }  ][line width=0.08]  [draw opacity=0] (10.72,-5.15) -- (0,0) -- (10.72,5.15) -- (7.12,0) -- cycle    ;
	%Shape: Arc [id:dp11397532389152976] 
	\draw  [draw opacity=0] (278.09,65.35) .. controls (278.39,65.34) and (278.7,65.33) .. (279,65.33) .. controls (284.11,65.33) and (288.85,66.93) .. (292.74,69.65) -- (279,89.33) -- cycle ; \draw   (278.09,65.35) .. controls (278.39,65.34) and (278.7,65.33) .. (279,65.33) .. controls (284.11,65.33) and (288.85,66.93) .. (292.74,69.65) ;

	% Text Node
	\draw (263.2,39.8) node    {$\vec{n}$};
	% Text Node
	\draw (330,46.6) node    {$\vec{u}_{r}$};
	% Text Node
	\draw (288.8,53.4) node    {$\vartheta $};
	% Text Node
	\draw (347.2,161) node    {$dS_{\text{sfera}}$};
	% Text Node
	\draw (385.2,86.6) node    {$dS$};
	% Text Node
	\draw (189.2,265.6) node    {$O$};


	\end{tikzpicture}
\end{figure}
\FloatBarrier
Supponiamo che nella stessa zona ci sia una superficie non sferica ma di forma qualunque situata alla stessa distanza da $O$. Se proiettiamo la nuova superficie ds sulla sfera otterremo ds sfera. In formula:
\begin{align*}
	dS_{\text{sfera}} &= dS\,\cos \vartheta \\
	r^2 d\Omega &= dS\,\cos \vartheta \\
	r^2 d\Omega &= dS\,\vec{n} \cdot \vec{u}_r \\
	d\Omega &= \frac{dS\,\vec{n} \cdot \vec{u}_r}{r^2} \\
\end{align*}




















\subsection{Le derivate parziali}

Supponiamo di avere una funzione $f=f(x,y,z)$.
Derivare parzialmente vuol dire mantenere costanti le altre variabili e far variare solo quella rispetto a cui stiamo derivando.
\[
	\boxed{\frac{\partial f}{\partial x} = \lim_{\Delta x \to 0} \frac{f(x+\Delta x,y,z)-f(x,y,z)}{\Delta x}}
\]
È possibile derivare una funzione $f(x,y,z)$ rispetto a ciascuna delle tre variabili, ottenendo quindi tre derivate parziali. La situazione si complica passando alle derivate di ordine superiore. Possiamo avere ad esempio la derivata seconda una volta rispetto a $x$ e una rispetto a $y$, oppure due volte rispetto a $x$ e via dicendo.




















\subsection{Operatore Divergenza}

Il concetto di derivata parziale è necessario per introdurre l'operatore divergenza. Nel calcolo differenziale vettoriale, la divergenza è un campo scalare che misura la tendenza di un campo vettoriale a divergere o a convergere verso un punto dello spazio. Consideriamo il caso di un campo vettoriale $\vec{v} (P)$, funzione della posizione, definito in una certa regione dello spazio e in essa un punto $P$. Sia poi $\Sigma$ una superficie chiusa che lo circondi. Chiamiamo $\tau$ il volume racchiuso da essa. Definiamo la divergenza del vettore $\vec{v}$ in $P$ come:
\[
	\text{div}\vec{v} (P)=\lim_{\tau  \to 0} \frac{\Phi_{\Sigma}(\vec{v} )}{\tau}= \lim_{\tau  \to 0} \frac{\int_{\Sigma} \vec{v} \cdot \vec{n} dS}{\tau}
\]
che mappa un campo vettoriale in un campo scalare. Si dimostra inoltre che
\[
	\text{div}\vec{v} = \frac{\partial v_x}{\partial x} +\frac{\partial v_y}{\partial y} +\frac{\partial v_z}{\partial z}
\]




















\subsection{Teorema della Divergenza}

Dato un campo $\vec{v}=\vec{v}(P)$ e $\Sigma$ una superficie chiusa di volume $\tau$, allora
\[
	\boxed{\Phi_{\Sigma}(\vec{v} ) = \int_{\Sigma}\vec{v} \cdot \vec{n} dS = \int_{\tau}(\text{div}\vec{v} )d\tau}
\]
Ossia ci mette in relazione integrali di superficie con integrali di volume.

L'idea alla base della dimostrazione è quella di pensare di suddividere
il volume in tanti cubetti infinitesimi e di calcolare la somma dei flussi.
I cubetti che sono dentro il volume hanno una faccia in comune. I flussi si cancellano a due a due e rimangono solo quelli attraverso la frontiera. Ecco perché la somma dei flussi sulla superficie esterna è uguale a quella sul volume.

Se non ci sono sorgenti, accade in genere che la divergenza del campo vettoriale è uguale a zero. Abbiamo infatti visto che tale operatore misura la tendenza del campo a convergere o a divergere in/da un punto. Immaginiamo due linee chiuse $\Gamma_1$ e $\Gamma_2$ e le varie linee di flusso del campo che passano per esse. Supponiamo che la divergenza del campo $\vec{v}$ in questa regione sia nulla.
\begin{figure}[htpb]
	\centering
	

	\tikzset{every picture/.style={line width=0.75pt}} %set default line width to 0.75pt        

	\begin{tikzpicture}[x=0.75pt,y=0.75pt,yscale=-1,xscale=1]
	%uncomment if require: \path (0,300); %set diagram left start at 0, and has height of 300

	%Shape: Can [id:dp07821714761930187] 
	\draw   (345.3,235.5) -- (168.7,235.5) .. controls (157.27,235.5) and (148,204.61) .. (148,166.5) .. controls (148,128.39) and (157.27,97.5) .. (168.7,97.5) -- (345.3,97.5) .. controls (356.73,97.5) and (366,128.39) .. (366,166.5) .. controls (366,204.61) and (356.73,235.5) .. (345.3,235.5) .. controls (333.87,235.5) and (324.6,204.61) .. (324.6,166.5) .. controls (324.6,128.39) and (333.87,97.5) .. (345.3,97.5) ;
	%Straight Lines [id:da819729350698617] 
	\draw    (198.88,198) -- (311.5,198) ;
	\draw [shift={(314.5,198)}, rotate = 180] [fill={rgb, 255:red, 0; green, 0; blue, 0 }  ][line width=0.08]  [draw opacity=0] (10.72,-5.15) -- (0,0) -- (10.72,5.15) -- (7.12,0) -- cycle    ;
	%Straight Lines [id:da13329799845494272] 
	\draw    (244,98) -- (244,77) ;
	\draw [shift={(244,74)}, rotate = 450] [fill={rgb, 255:red, 0; green, 0; blue, 0 }  ][line width=0.08]  [draw opacity=0] (10.72,-5.15) -- (0,0) -- (10.72,5.15) -- (7.12,0) -- cycle    ;
	%Straight Lines [id:da6320350854720245] 
	\draw    (198.88,167) -- (311.5,167) ;
	\draw [shift={(314.5,167)}, rotate = 180] [fill={rgb, 255:red, 0; green, 0; blue, 0 }  ][line width=0.08]  [draw opacity=0] (10.72,-5.15) -- (0,0) -- (10.72,5.15) -- (7.12,0) -- cycle    ;
	%Straight Lines [id:da09893112591978959] 
	\draw    (198,136) -- (310.62,136) ;
	\draw [shift={(313.62,136)}, rotate = 180] [fill={rgb, 255:red, 0; green, 0; blue, 0 }  ][line width=0.08]  [draw opacity=0] (10.72,-5.15) -- (0,0) -- (10.72,5.15) -- (7.12,0) -- cycle    ;
	%Curve Lines [id:da7589395231546388] 
	\draw  [dash pattern={on 0.84pt off 2.51pt}]  (168.7,97.5) .. controls (191.5,97.25) and (191,235.75) .. (168.7,235.5) ;

	% Text Node
	\draw (262.5,76) node    {$\vec{n}$};
	% Text Node
	\draw (166.5,165) node    {$S_{1}$};
	% Text Node
	\draw (346.5,165.5) node    {$S_{2}$};
	% Text Node
	\draw (165.5,251) node    {$\Gamma _{1}$};
	% Text Node
	\draw (345.5,251) node    {$\Gamma _{2}$};


	\end{tikzpicture}
\end{figure}
\FloatBarrier
L'inviluppo di queste linee da vita a una superficie chiusa che chiamiamo \textbf{superficie di flusso} (parallela alla linee di flusso).
Immaginiamo di considerare il cilindro dato dal tubo di flusso e le due aree di base racchiuse dalle due linee $\Gamma_1$ e $\Gamma_2$. Se vogliamo calcolare il flusso del campo vettoriale attraverso quella superficie chiusa, questo flusso totale sarà la somma di tre contributi: il flusso sulla superficie laterale del tubo, il flusso attraverso la superficie $S_1$ e il flusso attraverso la superficie $S_2$, chiusa dall'altra linea. Se in quella regione la divergenza è uguale a zero il flusso totale è uguale a zero.
\[
	\Phi_{\text{tot}}(\vec{v} ) = \Phi_{\Sigma}(\vec{v}) + \Phi_{S_1}(\vec{v}) + \Phi_{S_2}(\vec{v})
\]
Siccome la superficie considerata è costituita dalle linee di flusso, il vettore normale è perpendicolare a $\vec{v}$ e anche il flusso attraverso la superficie laterale è uguale a zero.
Questo ci dice che il flusso attraverso la superficie $S_1$ è uguale all'opposto di quello che attraversa la sezione $S_2$.
\[
	|\Phi_{S_1}(\vec{v})| = |\Phi_{S_2}(\vec{v})|
\]
Il risultato che abbiamo ottenuto è che in una regione in cui la divergenza è uguale a zero, il flusso del il campo vettoriale attraverso il tubo di flusso rimane costante.




















\subsection{Operatore Rotore}

Sia $ \vec{v} =\vec{v} (P) $ un campo vettoriale definito in un dominio semplicemente connesso. Denotiamo $\vec{w} = \text{rot}\vec{v} (P)$.
Sia $\gamma$ una linea chiusa orientata che circonda $P$, sia $\Sigma$ una superficie di area $S$ avente $\gamma$ come contorno e passi per il punto $P$. Sia $\vec{n}$ il versore normale alla superficie, con verso scelto con la regola del cavatappi.
\begin{figure}[htpb]
	\centering
	

	\tikzset{every picture/.style={line width=0.75pt}} %set default line width to 0.75pt        

	\begin{tikzpicture}[x=0.75pt,y=0.75pt,yscale=-1,xscale=1]
	%uncomment if require: \path (0,300); %set diagram left start at 0, and has height of 300

	%Curve Lines [id:da2720989042066049] 
	\draw    (154.5,220) .. controls (155.5,272) and (367.5,272) .. (365.5,219) ;
	\draw [shift={(261.13,258.88)}, rotate = 180.43] [fill={rgb, 255:red, 0; green, 0; blue, 0 }  ][line width=0.08]  [draw opacity=0] (10.72,-5.15) -- (0,0) -- (10.72,5.15) -- (7.12,0) -- cycle    ;
	%Curve Lines [id:da9383416644738221] 
	\draw  [dash pattern={on 0.84pt off 2.51pt}]  (365.5,219) .. controls (364.5,167) and (152.5,167) .. (154.5,220) ;
	\draw [shift={(258.87,180.13)}, rotate = 360.43] [fill={rgb, 255:red, 0; green, 0; blue, 0 }  ][line width=0.08]  [draw opacity=0] (10.72,-5.15) -- (0,0) -- (10.72,5.15) -- (7.12,0) -- cycle    ;
	%Shape: Circle [id:dp49398570007450715] 
	\draw  [fill={rgb, 255:red, 0; green, 0; blue, 0 }  ,fill opacity=1 ] (230.66,125.77) .. controls (230.66,124.12) and (232,122.77) .. (233.66,122.77) .. controls (235.32,122.77) and (236.66,124.12) .. (236.66,125.77) .. controls (236.66,127.43) and (235.32,128.77) .. (233.66,128.77) .. controls (232,128.77) and (230.66,127.43) .. (230.66,125.77) -- cycle ;
	%Straight Lines [id:da032901312924457526] 
	\draw    (263.66,29) -- (263.66,57.77) ;
	\draw [shift={(263.66,26)}, rotate = 90] [fill={rgb, 255:red, 0; green, 0; blue, 0 }  ][line width=0.08]  [draw opacity=0] (10.72,-5.15) -- (0,0) -- (10.72,5.15) -- (7.12,0) -- cycle    ;
	%Curve Lines [id:da9435042530698341] 
	\draw    (154.5,220) .. controls (168.5,87) and (231.82,58.55) .. (263.66,57.77) .. controls (295.5,57) and (343.5,76) .. (365.5,219) ;

	% Text Node
	\draw (174,107) node    {$S$};
	% Text Node
	\draw (249,124) node    {$P$};
	% Text Node
	\draw (279,41) node    {$\vec{n}$};
	% Text Node
	\draw (168,257) node    {$\gamma $};


	\end{tikzpicture}
\end{figure}
\FloatBarrier
Si definisce
\[
	(\text{rot}\vec{v} )\cdot (\vec{n} )= \lim_{S \to 0} \frac{\oint_{\gamma} \vec{v} \cdot d\vec{l}}{S}
\]
Essendo $ \vec{v} = v_x\vec{u}_x +v_y\vec{u}_y + v_z\vec{u}_z    $ si dimostra che che
\[
	\vec{w} = \text{rot}\vec{v} = \begin{vmatrix}
		\vec{u}_x & \vec{u}_y & \vec{u}_z \\
		\frac{\partial}{\partial x} & \frac{\partial}{\partial y} & \frac{\partial}{\partial z}  \\
		v_x & v_y & v_z
	\end{vmatrix}
\]
\[
	\vec{w} = \text{rot}\vec{v} = \vec{u}_x\left( \frac{\partial v_z}{\partial y} -\frac{\partial v_y}{\partial z}  \right)  +
	\vec{u}_y\left( \frac{\partial v_x}{\partial z} -\frac{\partial v_z}{\partial x}  \right)  +
	\vec{u}_z\left( \frac{\partial v_y}{\partial x} -\frac{\partial v_x}{\partial y}  \right)
\]




















\subsection{Teorema di Stokes}

Il passaggio ad una linea chiusa finita è garantito dal teorema di Stokes. Supponiamo di aver definito un campo vettoriale $\vec{v}(P)$ funzione della posizione in un certo dominio semplicemente connesso (data una linea chiusa è sempre possibile trovare almeno una superficie contenuta interamente nel dominio e avente la linea come contorno). Immaginiamo di definire in questa regione una linea chiusa $\gamma$ orientata che abbia una superficie $\Sigma$ aperta di cui costituisce il contorno.
\begin{figure}[htpb]
	\centering
	

	\tikzset{every picture/.style={line width=0.75pt}} %set default line width to 0.75pt        

	\begin{tikzpicture}[x=0.75pt,y=0.75pt,yscale=-1,xscale=1]
	%uncomment if require: \path (0,300); %set diagram left start at 0, and has height of 300

	%Curve Lines [id:da15583153412401485] 
	\draw    (164.5,218) .. controls (165.5,270) and (377.5,270) .. (375.5,217) ;
	\draw [shift={(271.13,256.88)}, rotate = 180.43] [fill={rgb, 255:red, 0; green, 0; blue, 0 }  ][line width=0.08]  [draw opacity=0] (10.72,-5.15) -- (0,0) -- (10.72,5.15) -- (7.12,0) -- cycle    ;
	%Curve Lines [id:da5672800465241312] 
	\draw  [dash pattern={on 0.84pt off 2.51pt}]  (375.5,217) .. controls (374.5,165) and (162.5,165) .. (164.5,218) ;
	\draw [shift={(268.87,178.13)}, rotate = 360.43] [fill={rgb, 255:red, 0; green, 0; blue, 0 }  ][line width=0.08]  [draw opacity=0] (10.72,-5.15) -- (0,0) -- (10.72,5.15) -- (7.12,0) -- cycle    ;
	%Shape: Circle [id:dp2951783555532528] 
	\draw  [fill={rgb, 255:red, 0; green, 0; blue, 0 }  ,fill opacity=1 ] (240.66,123.77) .. controls (240.66,122.12) and (242,120.77) .. (243.66,120.77) .. controls (245.32,120.77) and (246.66,122.12) .. (246.66,123.77) .. controls (246.66,125.43) and (245.32,126.77) .. (243.66,126.77) .. controls (242,126.77) and (240.66,125.43) .. (240.66,123.77) -- cycle ;
	%Straight Lines [id:da6906371612066755] 
	\draw    (273.66,27) -- (273.66,55.77) ;
	\draw [shift={(273.66,24)}, rotate = 90] [fill={rgb, 255:red, 0; green, 0; blue, 0 }  ][line width=0.08]  [draw opacity=0] (10.72,-5.15) -- (0,0) -- (10.72,5.15) -- (7.12,0) -- cycle    ;
	%Curve Lines [id:da12239522891457866] 
	\draw    (164.5,218) .. controls (178.5,85) and (241.82,56.55) .. (273.66,55.77) .. controls (305.5,55) and (353.5,74) .. (375.5,217) ;

	% Text Node
	\draw (180,105) node    {$\Sigma $};
	% Text Node
	\draw (259,122) node    {$P$};
	% Text Node
	\draw (289,39) node    {$\vec{n}$};
	% Text Node
	\draw (178,255) node    {$\gamma $};


	\end{tikzpicture}
\end{figure}
\FloatBarrier
Scegliamo sulla superficie il verso della normale $\vec{n}$,
diretta verso l'alto, in accordo con la regola del cavatappi. Il teorema afferma che la circuitazione di un campo vettoriale lungo una linea chiusa $\gamma$ è uguale al flusso del rotore del campo attraverso una \emph{qualunque} superficie $\Sigma$ avente come contorno $\gamma$.
\[
	\boxed{\oint_{\gamma} \vec{v} \cdot d\vec{l} = \int_{\Sigma}(\text{rot}\vec{v} )\cdot \vec{n} dS}
\]
Mette in relazione un integrale di linea, con un integrale di superficie.




















\subsection{Operatore Nabla}

Si può introdurre il vettore simbolico $ \vec{\nabla}  $ definito come:
\[
	\vec{\nabla} = \vec{u}_x \frac{\partial}{\partial x} + \vec{u}_y \frac{\partial}{\partial y} + \vec{u}_z \frac{\partial}{\partial z}
\]
per scrivere la divergenza come
\[
	\text{div}\vec{v} = \vec{\nabla} \cdot \vec{v}  = \frac{\partial v_x}{\partial x} + \frac{\partial v_y}{\partial y} + \frac{\partial v_z}{\partial z}
\]
e analogamente il rotore come
\[
	\text{rot}\vec{v} = \vec{\nabla} \times  \vec{v}  = \left(
	\frac{\partial v_z}{\partial y} -\frac{\partial v_y}{\partial z},
	\frac{\partial v_x}{\partial z} -\frac{\partial v_z}{\partial x},
	\frac{\partial v_y}{\partial x} -\frac{\partial v_x}{\partial y} \right)
\]




















\subsection{Flusso di un campo vettoriale}

Supponiamo che in una regione di spazio sia definito un campo vettoriale e individuata una superficie $\Sigma$. Nel punto $P$ individuiamo un quadrettino di area infinitesima $dS$. Sia $\vec{n}$ la normale alla superficie. Definiremo flusso elementare di $\vec{v}(P)$ attraverso l'area infinitesima $dS$ la quantità $d\Phi$.
\[
	\boxed{d\Phi (\vec{v}) = \vec{v} (P)\cdot \vec{n} \,dS}
\]
Possiamo estendere la definizione di flusso all'intera superficie $\Sigma$. Infatti posso immaginarla come divisa da quadretti di area infinitesima e calcolare il flusso in ognuno di essi, andando a sommare poi queste quantità.
\begin{figure}[htpb]
	\centering
	

	\tikzset{every picture/.style={line width=0.75pt}} %set default line width to 0.75pt        

	\begin{tikzpicture}[x=0.75pt,y=0.75pt,yscale=-1,xscale=1]
	%uncomment if require: \path (0,300); %set diagram left start at 0, and has height of 300

	%Shape: Polygon Curved [id:ds8769549835569137] 
	\draw   (228.33,69.67) .. controls (279,65.67) and (362.33,80.67) .. (415,110) .. controls (391,134) and (376.33,163.33) .. (362.33,193.33) .. controls (289,164) and (236.33,147.33) .. (162.33,145.33) .. controls (177.67,113.33) and (208.33,79.67) .. (228.33,69.67) -- cycle ;
	%Curve Lines [id:da563805179515227] 
	\draw    (209,149) .. controls (223,116) and (241,92.67) .. (269.67,70) ;
	%Curve Lines [id:da05137663634575418] 
	\draw    (258,158) .. controls (272,125) and (292,99.67) .. (320.67,77) ;
	%Curve Lines [id:da44819172214290304] 
	\draw    (309,173) .. controls (323,140) and (341.67,112.67) .. (368.67,90) ;
	%Curve Lines [id:da47225998773408695] 
	\draw    (202,90) .. controls (266,76.67) and (363,113) .. (395.33,133) ;
	%Curve Lines [id:da6171027402669229] 
	\draw    (180,116) .. controls (244,102.67) and (346,142) .. (378.33,162) ;
	%Straight Lines [id:da9267137414829982] 
	\draw    (310,116) -- (310,31.67) ;
	\draw [shift={(310,28.67)}, rotate = 450] [fill={rgb, 255:red, 0; green, 0; blue, 0 }  ][line width=0.08]  [draw opacity=0] (10.72,-5.15) -- (0,0) -- (10.72,5.15) -- (7.12,0) -- cycle    ;
	%Straight Lines [id:da934350194093134] 
	\draw    (310,116) -- (357.22,51.75) ;
	\draw [shift={(359,49.33)}, rotate = 486.32] [fill={rgb, 255:red, 0; green, 0; blue, 0 }  ][line width=0.08]  [draw opacity=0] (10.72,-5.15) -- (0,0) -- (10.72,5.15) -- (7.12,0) -- cycle    ;

	% Text Node
	\draw (381.33,52.33) node    {$\vec{v}( P)$};
	% Text Node
	\draw (295.33,37.67) node    {$\vec{n}$};
	% Text Node
	\draw (321.33,122.67) node    {$dS$};
	% Text Node
	\draw (299.33,113.33) node    {$P$};
	% Text Node
	\draw (408.67,138) node    {$\Sigma $};


	\end{tikzpicture}
\end{figure}
\FloatBarrier
Il flusso del campo vettoriale $\vec{v}(P)$ attraverso la superficie $\Sigma$ sarà:
\[
	\Phi_{\Sigma}(\vec{v} ) = \int_{\tau}d\Phi (\vec{v} )=\int_{\Sigma}\vec{v} \cdot \vec{n} \,dS
\]
Il nome di flusso deriva dalle applicazioni in idrodinamica, ma deve essere chiaro che il flusso di un campo vettoriale è un concetto matematico e non si accompagna necessariamente al passaggio attraverso $\Sigma$ di materia o di energia.

Nel caso di una superficie chiusa si assume che la normale sia diretta sempre verso l'esterno della superficie stessa. Supponiamo che in tal caso il campo vettoriale di cui ci stiamo occupando sia diretto anche lui verso l'esterno. Quello che risulta è che l'angolo che $\vec{v}$ ed $\vec{n}$ formano in genere è un angolo acuto. Il prodotto scalare fra $\vec{n}$ e $\vec{v}$ sarà positivo. Ecco perché allora il flusso positivo è un \textbf{flusso uscente}. Viceversa se il nostro campo vettoriale è diretto verso l'interno della superficie, allora in questo caso l'angolo che $\vec{v}$ ed $\vec{n}$ formano è ottuso e il flusso sarà negativo, diremo che sarà un \textbf{flusso entrante}.
\begin{figure}[htpb]
	\centering
	

	\tikzset{every picture/.style={line width=0.75pt}} %set default line width to 0.75pt        

	\begin{tikzpicture}[x=0.75pt,y=0.75pt,yscale=-1,xscale=1]
	%uncomment if require: \path (0,300); %set diagram left start at 0, and has height of 300

	%Shape: Ellipse [id:dp8770968669765249] 
	\draw   (152.31,148.29) .. controls (152.31,102.89) and (189.11,66.09) .. (234.51,66.09) .. controls (279.91,66.09) and (316.71,102.89) .. (316.71,148.29) .. controls (316.71,193.69) and (279.91,230.5) .. (234.51,230.5) .. controls (189.11,230.5) and (152.31,193.69) .. (152.31,148.29) -- cycle ;
	%Straight Lines [id:da7209345608057556] 
	\draw    (309.86,115.8) -- (341.69,50.36) ;
	\draw [shift={(343,47.66)}, rotate = 475.94] [fill={rgb, 255:red, 0; green, 0; blue, 0 }  ][line width=0.08]  [draw opacity=0] (10.72,-5.15) -- (0,0) -- (10.72,5.15) -- (7.12,0) -- cycle    ;
	%Straight Lines [id:da7804379502263452] 
	\draw    (309.86,115.8) -- (370.73,83.78) ;
	\draw [shift={(373.38,82.38)}, rotate = 512.26] [fill={rgb, 255:red, 0; green, 0; blue, 0 }  ][line width=0.08]  [draw opacity=0] (10.72,-5.15) -- (0,0) -- (10.72,5.15) -- (7.12,0) -- cycle    ;
	%Straight Lines [id:da6934703276913838] 
	\draw    (156.12,122.81) -- (86.22,143.04) ;
	\draw [shift={(83.34,143.88)}, rotate = 343.86] [fill={rgb, 255:red, 0; green, 0; blue, 0 }  ][line width=0.08]  [draw opacity=0] (10.72,-5.15) -- (0,0) -- (10.72,5.15) -- (7.12,0) -- cycle    ;
	%Straight Lines [id:da13641523911734055] 
	\draw    (156.12,122.81) -- (91.56,99.09) ;
	\draw [shift={(88.74,98.06)}, rotate = 380.18] [fill={rgb, 255:red, 0; green, 0; blue, 0 }  ][line width=0.08]  [draw opacity=0] (10.72,-5.15) -- (0,0) -- (10.72,5.15) -- (7.12,0) -- cycle    ;
	%Straight Lines [id:da8265055914518438] 
	\draw    (302.28,195.52) -- (354.49,192.74) ;
	\draw [shift={(357.49,192.58)}, rotate = 536.96] [fill={rgb, 255:red, 0; green, 0; blue, 0 }  ][line width=0.08]  [draw opacity=0] (10.72,-5.15) -- (0,0) -- (10.72,5.15) -- (7.12,0) -- cycle    ;
	%Straight Lines [id:da12763025741024703] 
	\draw    (302.28,195.52) -- (343.56,222.61) ;
	\draw [shift={(346.07,224.25)}, rotate = 213.28] [fill={rgb, 255:red, 0; green, 0; blue, 0 }  ][line width=0.08]  [draw opacity=0] (10.72,-5.15) -- (0,0) -- (10.72,5.15) -- (7.12,0) -- cycle    ;

	% Text Node
	\draw (323.51,55.97) node    {$\vec{v}$};
	% Text Node
	\draw (376.43,96.6) node    {$\vec{n}$};
	% Text Node
	\draw (320.47,221.53) node    {$\vec{n}$};
	% Text Node
	\draw (116.16,94.87) node    {$\vec{n}$};
	% Text Node
	\draw (342.16,180.03) node    {$\vec{v}$};
	% Text Node
	\draw (103.14,152.7) node    {$\vec{v}$};


	\end{tikzpicture}
\end{figure}
\FloatBarrier
Supponiamo ancora una volta di avere un campo vettoriale in una regione di spazio e di avere una linea chiusa $\Gamma$. Disegniamo soltanto quelle particolari linee di flusso che passano per la linea $\Gamma$. Otterremo una superficie tubolare individuata da tutte le linee che passano per tale linea.
\begin{figure}[htpb]
	\centering
	

	\tikzset{every picture/.style={line width=0.75pt}} %set default line width to 0.75pt        

	\begin{tikzpicture}[x=0.75pt,y=0.75pt,yscale=-1,xscale=1]
	%uncomment if require: \path (0,300); %set diagram left start at 0, and has height of 300

	%Shape: Ellipse [id:dp021868135678277723] 
	\draw   (108,138) .. controls (108,102.1) and (115.95,73) .. (125.75,73) .. controls (135.55,73) and (143.5,102.1) .. (143.5,138) .. controls (143.5,173.9) and (135.55,203) .. (125.75,203) .. controls (115.95,203) and (108,173.9) .. (108,138) -- cycle ;
	%Straight Lines [id:da0013832645143680988] 
	\draw    (125.75,73) -- (322.5,73) ;
	\draw [shift={(224.13,73)}, rotate = 180] [fill={rgb, 255:red, 0; green, 0; blue, 0 }  ][line width=0.08]  [draw opacity=0] (10.72,-5.15) -- (0,0) -- (10.72,5.15) -- (7.12,0) -- cycle    ;
	%Straight Lines [id:da09557240360638342] 
	\draw    (125.75,203) -- (322.5,203) ;
	\draw [shift={(224.13,203)}, rotate = 180] [fill={rgb, 255:red, 0; green, 0; blue, 0 }  ][line width=0.08]  [draw opacity=0] (10.72,-5.15) -- (0,0) -- (10.72,5.15) -- (7.12,0) -- cycle    ;
	%Straight Lines [id:da3950416933849661] 
	\draw    (142.5,162) -- (326.5,162) ;
	\draw [shift={(234.5,162)}, rotate = 180] [fill={rgb, 255:red, 0; green, 0; blue, 0 }  ][line width=0.08]  [draw opacity=0] (10.72,-5.15) -- (0,0) -- (10.72,5.15) -- (7.12,0) -- cycle    ;
	%Straight Lines [id:da7389004927289446] 
	\draw    (142.5,112) -- (326.5,112) ;
	\draw [shift={(234.5,112)}, rotate = 180] [fill={rgb, 255:red, 0; green, 0; blue, 0 }  ][line width=0.08]  [draw opacity=0] (10.72,-5.15) -- (0,0) -- (10.72,5.15) -- (7.12,0) -- cycle    ;
	%Straight Lines [id:da14730215659138457] 
	\draw    (125.75,73) -- (125.75,47.75) ;
	\draw [shift={(125.75,44.75)}, rotate = 450] [fill={rgb, 255:red, 0; green, 0; blue, 0 }  ][line width=0.08]  [draw opacity=0] (10.72,-5.15) -- (0,0) -- (10.72,5.15) -- (7.12,0) -- cycle    ;

	% Text Node
	\draw (101,97.5) node    {$\Gamma $};
	% Text Node
	\draw (140.5,47) node    {$\vec{n}$};


	\end{tikzpicture}
\end{figure}
\FloatBarrier
Questo tubo che abbiamo individuato prende il nome di \textbf{tubo di flusso}. La normale alla superficie sarà perpendicolare a $\vec{v}$ e il prodotto scalare fra $\vec{v}$ ed $\vec{n}$ è nullo.







































\section{Teorema di Gauss per il campo elettrico}

Possiamo ora enunciare e dimostrare il teorema di Gauss per il campo elettrico.
\begin{figure}[htpb]
	\centering
	

	\tikzset{every picture/.style={line width=0.75pt}} %set default line width to 0.75pt        

	\begin{tikzpicture}[x=0.75pt,y=0.75pt,yscale=-1,xscale=1]
	%uncomment if require: \path (0,300); %set diagram left start at 0, and has height of 300

	%Shape: Polygon Curved [id:ds8897561476684546] 
	\draw   (396.59,96.87) .. controls (444.84,157.59) and (305.5,236) .. (265.5,207) .. controls (225.5,178) and (187.41,218.67) .. (152.86,186.81) .. controls (118.32,154.95) and (199.95,54.69) .. (234.5,86.55) .. controls (269.04,118.41) and (348.34,36.15) .. (396.59,96.87) -- cycle ;
	%Straight Lines [id:da27415252737009266] 
	\draw [line width=2.25]    (266,94) -- (287.5,87.75) ;
	%Straight Lines [id:da5060852752682012] 
	\draw    (245.26,67.72) -- (344.5,154.5) ;
	\draw [shift={(243,65.75)}, rotate = 41.17] [fill={rgb, 255:red, 0; green, 0; blue, 0 }  ][line width=0.08]  [draw opacity=0] (10.72,-5.15) -- (0,0) -- (10.72,5.15) -- (7.12,0) -- cycle    ;
	%Shape: Circle [id:dp03455482329752391] 
	\draw  [fill={rgb, 255:red, 0; green, 0; blue, 0 }  ,fill opacity=1 ] (341.5,154.5) .. controls (341.5,152.84) and (342.84,151.5) .. (344.5,151.5) .. controls (346.16,151.5) and (347.5,152.84) .. (347.5,154.5) .. controls (347.5,156.16) and (346.16,157.5) .. (344.5,157.5) .. controls (342.84,157.5) and (341.5,156.16) .. (341.5,154.5) -- cycle ;
	%Straight Lines [id:da5400783698484286] 
	\draw    (324.26,133.93) -- (345.5,152.5) ;
	\draw [shift={(322,131.95)}, rotate = 41.17] [fill={rgb, 255:red, 0; green, 0; blue, 0 }  ][line width=0.08]  [draw opacity=0] (10.72,-5.15) -- (0,0) -- (10.72,5.15) -- (7.12,0) -- cycle    ;
	%Straight Lines [id:da5326366958895112] 
	\draw    (274.26,95.83) -- (342.5,155.5) ;
	\draw [shift={(272,93.86)}, rotate = 41.17] [fill={rgb, 255:red, 0; green, 0; blue, 0 }  ][line width=0.08]  [draw opacity=0] (10.72,-5.15) -- (0,0) -- (10.72,5.15) -- (7.12,0) -- cycle    ;
	%Straight Lines [id:da6906937215964353] 
	\draw    (266.36,56.12) -- (276.75,90.88) ;
	\draw [shift={(265.5,53.25)}, rotate = 73.35] [fill={rgb, 255:red, 0; green, 0; blue, 0 }  ][line width=0.08]  [draw opacity=0] (10.72,-5.15) -- (0,0) -- (10.72,5.15) -- (7.12,0) -- cycle    ;

	% Text Node
	\draw (299.5,68) node    {$dS$};
	% Text Node
	\draw (360.5,154.5) node    {$Q$};
	% Text Node
	\draw (342.5,128) node    {$\vec{u}_{r}$};
	% Text Node
	\draw (278.5,121) node    {$\vec{r}$};
	% Text Node
	\draw (231.5,65.5) node    {$\vec{v}$};
	% Text Node
	\draw (277.5,46) node    {$\vec{n}$};
	% Text Node
	\draw (308.5,153.5) node    {$d\Omega $};
	% Text Node
	\draw (405.5,214) node    {$\Sigma _{\text{chiusa}}$};


	\end{tikzpicture}
\end{figure}
\FloatBarrier
Siano:
\begin{itemize}
	\item $\Sigma$ una superficie chiusa di forma qualunque
	\item La carica $Q$ dentro $\Sigma$
	\item $d\Omega$ l'angolo solido infinitesimo che individua $dS$
	\item $\vec{u}_r$ il versore radiale
\end{itemize}
Sostituendo l'espressione $ \vec{E} =\frac{Q\vec{u}_r}{4\pi \varepsilon_0r^2} $ nel flusso infinitesimo $ d\Phi(\vec{E} ) =\vec{E} \cdot \vec{n} dS$ si ottiene
\[
	d\Phi(\vec{E} )=\frac{Q}{4\pi \varepsilon_0}\left( \frac{\vec{u}_r\vec{n} dS}{r^2} \right)  = \frac{Q}{4\pi \varepsilon_0} d\Omega
\]
Integrando
\[
	\Phi_{\Sigma}(\vec{E} )=\int_{\Sigma}d\Phi=\int_{\Sigma} \frac{Q}{4\pi\varepsilon_0} d\Omega = \frac{Q}{4\pi \varepsilon_0} \int_{\Sigma} d\Omega = \frac{Q}{4\pi \varepsilon_0} \cdot 4\pi = \frac{Q}{\varepsilon_0}
\]
Quindi il flusso non dipende da $\Sigma$.

Siamo riusciti a fare questa semplificazione perché il campo elettrico decresce con il quadrato della distanza ed è diretto proprio in direzione radiale. Per tutti i campi che hanno queste proprietà si può definire un teorema di Gauss analogo. Viceversa, per un campo radiale qualunque del tipo $ v = \frac{1}{r^n} $, con $n\neq 2$ anche di una quantità molto piccola, non vale la legge di Gauss: questa può quindi essere considerata come una formulazione alternativa della legge di Coulomb, basata sul concetto di campo.

Considerata una superficie chiusa che racchiude una carica puntiforme, il flusso attraverso essa del campo elettrico prodotto è pari a $ \frac{Q}{\varepsilon_0} $ qualunque sia la forma della superficie, non è mai nullo perché essendo la carica interna, tutti i contributi di flusso elementare hanno lo stesso segno.

Se la carica è negativa il flusso infatti è negativo perché il campo elettrico è entrante e viceversa se la carica è positiva il flusso sarà uscente.
\begin{figure}[htpb]
	\centering
	

	% Pattern Info
	 
	\tikzset{
	pattern size/.store in=\mcSize, 
	pattern size = 5pt,
	pattern thickness/.store in=\mcThickness, 
	pattern thickness = 0.3pt,
	pattern radius/.store in=\mcRadius, 
	pattern radius = 1pt}
	\makeatletter
	\pgfutil@ifundefined{pgf@pattern@name@_95exuwzao}{
	\pgfdeclarepatternformonly[\mcThickness,\mcSize]{_95exuwzao}
	{\pgfqpoint{0pt}{0pt}}
	{\pgfpoint{\mcSize+\mcThickness}{\mcSize+\mcThickness}}
	{\pgfpoint{\mcSize}{\mcSize}}
	{
	\pgfsetcolor{\tikz@pattern@color}
	\pgfsetlinewidth{\mcThickness}
	\pgfpathmoveto{\pgfqpoint{0pt}{0pt}}
	\pgfpathlineto{\pgfpoint{\mcSize+\mcThickness}{\mcSize+\mcThickness}}
	\pgfusepath{stroke}
	}}
	\makeatother

	% Pattern Info
	 
	\tikzset{
	pattern size/.store in=\mcSize, 
	pattern size = 5pt,
	pattern thickness/.store in=\mcThickness, 
	pattern thickness = 0.3pt,
	pattern radius/.store in=\mcRadius, 
	pattern radius = 1pt}
	\makeatletter
	\pgfutil@ifundefined{pgf@pattern@name@_ahy7x98kw}{
	\pgfdeclarepatternformonly[\mcThickness,\mcSize]{_ahy7x98kw}
	{\pgfqpoint{0pt}{0pt}}
	{\pgfpoint{\mcSize+\mcThickness}{\mcSize+\mcThickness}}
	{\pgfpoint{\mcSize}{\mcSize}}
	{
	\pgfsetcolor{\tikz@pattern@color}
	\pgfsetlinewidth{\mcThickness}
	\pgfpathmoveto{\pgfqpoint{0pt}{0pt}}
	\pgfpathlineto{\pgfpoint{\mcSize+\mcThickness}{\mcSize+\mcThickness}}
	\pgfusepath{stroke}
	}}
	\makeatother
	\tikzset{every picture/.style={line width=0.75pt}} %set default line width to 0.75pt        

	\begin{tikzpicture}[x=0.75pt,y=0.75pt,yscale=-0.8,xscale=0.8]
	%uncomment if require: \path (0,300); %set diagram left start at 0, and has height of 300

	%Shape: Ellipse [id:dp7149901964828156] 
	\draw   (185.67,37.85) .. controls (222.73,3.91) and (289.75,16.75) .. (335.35,66.53) .. controls (380.96,116.32) and (387.89,184.2) .. (350.83,218.15) .. controls (313.77,252.09) and (246.75,239.25) .. (201.15,189.47) .. controls (155.54,139.68) and (148.61,71.8) .. (185.67,37.85) -- cycle ;
	%Shape: Circle [id:dp5251217430686774] 
	\draw  [fill={rgb, 255:red, 0; green, 0; blue, 0 }  ,fill opacity=1 ] (111.5,234.5) .. controls (111.5,232.84) and (112.84,231.5) .. (114.5,231.5) .. controls (116.16,231.5) and (117.5,232.84) .. (117.5,234.5) .. controls (117.5,236.16) and (116.16,237.5) .. (114.5,237.5) .. controls (112.84,237.5) and (111.5,236.16) .. (111.5,234.5) -- cycle ;
	%Shape: Ellipse [id:dp5296689649528463] 
	\draw  [pattern=_95exuwzao,pattern size=6pt,pattern thickness=0.75pt,pattern radius=0pt, pattern color={rgb, 255:red, 0; green, 0; blue, 0}] (181.83,162.13) .. controls (185.64,159.25) and (194.68,164.77) .. (202.03,174.46) .. controls (209.37,184.16) and (212.23,194.36) .. (208.42,197.24) .. controls (204.62,200.13) and (195.57,194.61) .. (188.23,184.91) .. controls (180.89,175.22) and (178.02,165.02) .. (181.83,162.13) -- cycle ;
	%Straight Lines [id:da5765623187007569] 
	\draw    (314.5,17) -- (114.5,234.5) ;
	%Shape: Boxed Line [id:dp48950804424816186] 
	\draw    (390.15,128.08) -- (114.5,234.5) ;
	%Shape: Ellipse [id:dp9367970487909316] 
	\draw  [pattern=_ahy7x98kw,pattern size=6pt,pattern thickness=0.75pt,pattern radius=0pt, pattern color={rgb, 255:red, 0; green, 0; blue, 0}] (299.3,35) .. controls (310.06,26.84) and (335.62,42.45) .. (356.37,69.85) .. controls (377.13,97.25) and (385.23,126.08) .. (374.46,134.23) .. controls (363.69,142.39) and (338.14,126.79) .. (317.38,99.38) .. controls (296.63,71.98) and (288.53,43.15) .. (299.3,35) -- cycle ;
	%Straight Lines [id:da6178033105848244] 
	\draw [line width=1.5]    (233.24,152.57) -- (195.13,179.69) ;
	\draw [shift={(236.5,150.25)}, rotate = 144.56] [fill={rgb, 255:red, 0; green, 0; blue, 0 }  ][line width=0.08]  [draw opacity=0] (13.4,-6.43) -- (0,0) -- (13.4,6.44) -- (8.9,0) -- cycle    ;
	%Straight Lines [id:da16899429666892196] 
	\draw [line width=1.5]    (374.99,57.5) -- (336.88,84.62) ;
	\draw [shift={(378.25,55.18)}, rotate = 144.56] [fill={rgb, 255:red, 0; green, 0; blue, 0 }  ][line width=0.08]  [draw opacity=0] (13.4,-6.43) -- (0,0) -- (13.4,6.44) -- (8.9,0) -- cycle    ;
	%Straight Lines [id:da3380206965238828] 
	\draw [line width=1.5]    (388.21,67.51) -- (336.88,84.62) ;
	\draw [shift={(392,66.25)}, rotate = 161.57] [fill={rgb, 255:red, 0; green, 0; blue, 0 }  ][line width=0.08]  [draw opacity=0] (13.4,-6.43) -- (0,0) -- (13.4,6.44) -- (8.9,0) -- cycle    ;
	%Straight Lines [id:da8743877238750009] 
	\draw [line width=1.5]    (135.92,191.47) -- (195.13,179.69) ;
	\draw [shift={(132,192.25)}, rotate = 348.75] [fill={rgb, 255:red, 0; green, 0; blue, 0 }  ][line width=0.08]  [draw opacity=0] (13.4,-6.43) -- (0,0) -- (13.4,6.44) -- (8.9,0) -- cycle    ;

	% Text Node
	\draw (97,237.5) node    {$Q$};
	% Text Node
	\draw (206.5,214.5) node    {$dS_{1}$};
	% Text Node
	\draw (393.5,150) node    {$dS_{2}$};
	% Text Node
	\draw (249,147.5) node    {$\vec{E}_{1}$};
	% Text Node
	\draw (381,36.5) node    {$\vec{E}_{2}$};
	% Text Node
	\draw (120,187) node    {$\vec{n}_{1}$};
	% Text Node
	\draw (405,74) node    {$\vec{n}_{2}$};


	\end{tikzpicture}
\end{figure}
\FloatBarrier
Se $Q$ fosse fuori da $\Sigma$ possiamo considerare che sia $\vec{E}$ che l'area decrescono come $ \frac{1}{r^2} $, pertanto i contributi si annullano allo stesso modo e il flusso complessivo sarebbe nullo. Formalmente:
\begin{gather*}
	d\Phi_1 = \vec{E}_1\cdot \vec{n}_1dS_1 < 0 \\
	d\Phi_2 = \vec{E}_2\cdot \vec{n}_2dS_2 > 0 \\
	d\Phi = d\Phi_1 + d\Phi_2 = 0 \\
	\Phi_{\Sigma}(\vec{E} ) = \int_{\Sigma}d\Phi = 0
\end{gather*}
Qualunque sia la forma della superficie chiusa, se $Q$ è esterna il numero di intersezioni dell'angolo solido con essa è sempre un numero pari ed essendo $\vec{n}$ sempre diretto all'esterno i flussi si cancellano sempre a coppie. Se $Q$ è interna il numero di intersezioni è sempre dispari e la compensazione è solo parziale.
\begin{figure}[htpb]
	\centering
	

	% Pattern Info
	 
	\tikzset{
	pattern size/.store in=\mcSize, 
	pattern size = 5pt,
	pattern thickness/.store in=\mcThickness, 
	pattern thickness = 0.3pt,
	pattern radius/.store in=\mcRadius, 
	pattern radius = 1pt}
	\makeatletter
	\pgfutil@ifundefined{pgf@pattern@name@_e8o2hz1iq}{
	\pgfdeclarepatternformonly[\mcThickness,\mcSize]{_e8o2hz1iq}
	{\pgfqpoint{0pt}{0pt}}
	{\pgfpoint{\mcSize+\mcThickness}{\mcSize+\mcThickness}}
	{\pgfpoint{\mcSize}{\mcSize}}
	{
	\pgfsetcolor{\tikz@pattern@color}
	\pgfsetlinewidth{\mcThickness}
	\pgfpathmoveto{\pgfqpoint{0pt}{0pt}}
	\pgfpathlineto{\pgfpoint{\mcSize+\mcThickness}{\mcSize+\mcThickness}}
	\pgfusepath{stroke}
	}}
	\makeatother

	% Pattern Info
	 
	\tikzset{
	pattern size/.store in=\mcSize, 
	pattern size = 5pt,
	pattern thickness/.store in=\mcThickness, 
	pattern thickness = 0.3pt,
	pattern radius/.store in=\mcRadius, 
	pattern radius = 1pt}
	\makeatletter
	\pgfutil@ifundefined{pgf@pattern@name@_x6slgilg2}{
	\pgfdeclarepatternformonly[\mcThickness,\mcSize]{_x6slgilg2}
	{\pgfqpoint{0pt}{0pt}}
	{\pgfpoint{\mcSize+\mcThickness}{\mcSize+\mcThickness}}
	{\pgfpoint{\mcSize}{\mcSize}}
	{
	\pgfsetcolor{\tikz@pattern@color}
	\pgfsetlinewidth{\mcThickness}
	\pgfpathmoveto{\pgfqpoint{0pt}{0pt}}
	\pgfpathlineto{\pgfpoint{\mcSize+\mcThickness}{\mcSize+\mcThickness}}
	\pgfusepath{stroke}
	}}
	\makeatother

	% Pattern Info
	 
	\tikzset{
	pattern size/.store in=\mcSize, 
	pattern size = 5pt,
	pattern thickness/.store in=\mcThickness, 
	pattern thickness = 0.3pt,
	pattern radius/.store in=\mcRadius, 
	pattern radius = 1pt}
	\makeatletter
	\pgfutil@ifundefined{pgf@pattern@name@_e08cmt92t}{
	\pgfdeclarepatternformonly[\mcThickness,\mcSize]{_e08cmt92t}
	{\pgfqpoint{0pt}{0pt}}
	{\pgfpoint{\mcSize+\mcThickness}{\mcSize+\mcThickness}}
	{\pgfpoint{\mcSize}{\mcSize}}
	{
	\pgfsetcolor{\tikz@pattern@color}
	\pgfsetlinewidth{\mcThickness}
	\pgfpathmoveto{\pgfqpoint{0pt}{0pt}}
	\pgfpathlineto{\pgfpoint{\mcSize+\mcThickness}{\mcSize+\mcThickness}}
	\pgfusepath{stroke}
	}}
	\makeatother

	% Pattern Info
	 
	\tikzset{
	pattern size/.store in=\mcSize, 
	pattern size = 5pt,
	pattern thickness/.store in=\mcThickness, 
	pattern thickness = 0.3pt,
	pattern radius/.store in=\mcRadius, 
	pattern radius = 1pt}
	\makeatletter
	\pgfutil@ifundefined{pgf@pattern@name@_kgwtsdvho}{
	\pgfdeclarepatternformonly[\mcThickness,\mcSize]{_kgwtsdvho}
	{\pgfqpoint{0pt}{0pt}}
	{\pgfpoint{\mcSize+\mcThickness}{\mcSize+\mcThickness}}
	{\pgfpoint{\mcSize}{\mcSize}}
	{
	\pgfsetcolor{\tikz@pattern@color}
	\pgfsetlinewidth{\mcThickness}
	\pgfpathmoveto{\pgfqpoint{0pt}{0pt}}
	\pgfpathlineto{\pgfpoint{\mcSize+\mcThickness}{\mcSize+\mcThickness}}
	\pgfusepath{stroke}
	}}
	\makeatother

	% Pattern Info
	 
	\tikzset{
	pattern size/.store in=\mcSize, 
	pattern size = 5pt,
	pattern thickness/.store in=\mcThickness, 
	pattern thickness = 0.3pt,
	pattern radius/.store in=\mcRadius, 
	pattern radius = 1pt}
	\makeatletter
	\pgfutil@ifundefined{pgf@pattern@name@_yx0ef09wr}{
	\pgfdeclarepatternformonly[\mcThickness,\mcSize]{_yx0ef09wr}
	{\pgfqpoint{0pt}{0pt}}
	{\pgfpoint{\mcSize+\mcThickness}{\mcSize+\mcThickness}}
	{\pgfpoint{\mcSize}{\mcSize}}
	{
	\pgfsetcolor{\tikz@pattern@color}
	\pgfsetlinewidth{\mcThickness}
	\pgfpathmoveto{\pgfqpoint{0pt}{0pt}}
	\pgfpathlineto{\pgfpoint{\mcSize+\mcThickness}{\mcSize+\mcThickness}}
	\pgfusepath{stroke}
	}}
	\makeatother

	% Pattern Info
	 
	\tikzset{
	pattern size/.store in=\mcSize, 
	pattern size = 5pt,
	pattern thickness/.store in=\mcThickness, 
	pattern thickness = 0.3pt,
	pattern radius/.store in=\mcRadius, 
	pattern radius = 1pt}
	\makeatletter
	\pgfutil@ifundefined{pgf@pattern@name@_5u080t6v1}{
	\pgfdeclarepatternformonly[\mcThickness,\mcSize]{_5u080t6v1}
	{\pgfqpoint{0pt}{0pt}}
	{\pgfpoint{\mcSize+\mcThickness}{\mcSize+\mcThickness}}
	{\pgfpoint{\mcSize}{\mcSize}}
	{
	\pgfsetcolor{\tikz@pattern@color}
	\pgfsetlinewidth{\mcThickness}
	\pgfpathmoveto{\pgfqpoint{0pt}{0pt}}
	\pgfpathlineto{\pgfpoint{\mcSize+\mcThickness}{\mcSize+\mcThickness}}
	\pgfusepath{stroke}
	}}
	\makeatother

	% Pattern Info
	 
	\tikzset{
	pattern size/.store in=\mcSize, 
	pattern size = 5pt,
	pattern thickness/.store in=\mcThickness, 
	pattern thickness = 0.3pt,
	pattern radius/.store in=\mcRadius, 
	pattern radius = 1pt}
	\makeatletter
	\pgfutil@ifundefined{pgf@pattern@name@_4zbbp5c0r}{
	\pgfdeclarepatternformonly[\mcThickness,\mcSize]{_4zbbp5c0r}
	{\pgfqpoint{0pt}{0pt}}
	{\pgfpoint{\mcSize+\mcThickness}{\mcSize+\mcThickness}}
	{\pgfpoint{\mcSize}{\mcSize}}
	{
	\pgfsetcolor{\tikz@pattern@color}
	\pgfsetlinewidth{\mcThickness}
	\pgfpathmoveto{\pgfqpoint{0pt}{0pt}}
	\pgfpathlineto{\pgfpoint{\mcSize+\mcThickness}{\mcSize+\mcThickness}}
	\pgfusepath{stroke}
	}}
	\makeatother
	\tikzset{every picture/.style={line width=0.75pt}} %set default line width to 0.75pt        

	\begin{tikzpicture}[x=0.75pt,y=0.75pt,yscale=-0.9,xscale=0.9]
	%uncomment if require: \path (0,300); %set diagram left start at 0, and has height of 300

	%Shape: Polygon Curved [id:ds9978296891861007] 
	\draw   (408.96,127.2) .. controls (422.42,210.4) and (415.46,238.59) .. (400.8,250.75) .. controls (386.15,262.9) and (365.69,245.35) .. (357.8,196.08) .. controls (349.91,146.82) and (339.62,124.64) .. (319.5,121) .. controls (299.38,117.36) and (273.39,156.89) .. (250.75,220.84) .. controls (228.1,284.79) and (215.57,264.99) .. (213.63,222.1) .. controls (211.68,179.21) and (220.33,113.23) .. (240.06,84.82) .. controls (259.78,56.41) and (298.19,24.2) .. (330.69,26.52) .. controls (363.18,28.85) and (395.5,44) .. (408.96,127.2) -- cycle ;
	%Shape: Circle [id:dp29298068233057384] 
	\draw  [fill={rgb, 255:red, 0; green, 0; blue, 0 }  ,fill opacity=1 ] (244.81,148.82) .. controls (245.7,147.42) and (247.56,147.02) .. (248.95,147.92) .. controls (250.35,148.82) and (250.75,150.68) .. (249.85,152.07) .. controls (248.95,153.46) and (247.09,153.86) .. (245.7,152.96) .. controls (244.31,152.06) and (243.91,150.21) .. (244.81,148.82) -- cycle ;
	%Straight Lines [id:da008781013221186518] 
	\draw    (533.3,128.17) -- (247.33,150.44) ;
	%Straight Lines [id:da6613236754288319] 
	\draw    (536.67,168.44) -- (247.3,150.42) ;
	%Shape: Circle [id:dp12111545429115034] 
	\draw  [fill={rgb, 255:red, 0; green, 0; blue, 0 }  ,fill opacity=1 ] (160.14,214.15) .. controls (161.04,212.76) and (162.89,212.36) .. (164.29,213.25) .. controls (165.68,214.15) and (166.08,216.01) .. (165.18,217.4) .. controls (164.28,218.79) and (162.43,219.19) .. (161.03,218.3) .. controls (159.64,217.4) and (159.24,215.54) .. (160.14,214.15) -- cycle ;
	%Straight Lines [id:da04391863334528656] 
	\draw    (448.63,193.51) -- (162.66,215.77) ;
	%Shape: Boxed Line [id:dp8183184588605728] 
	\draw    (452.01,233.77) -- (162.63,215.76) ;
	%Shape: Ellipse [id:dp020877861992421032] 
	\draw  [pattern=_e8o2hz1iq,pattern size=6pt,pattern thickness=0.75pt,pattern radius=0pt, pattern color={rgb, 255:red, 0; green, 0; blue, 0}] (210.67,215.17) .. controls (210.67,213.14) and (211.86,211.5) .. (213.33,211.5) .. controls (214.81,211.5) and (216,213.14) .. (216,215.17) .. controls (216,217.19) and (214.81,218.83) .. (213.33,218.83) .. controls (211.86,218.83) and (210.67,217.19) .. (210.67,215.17) -- cycle ;
	%Shape: Ellipse [id:dp2046530810571039] 
	\draw  [pattern=_x6slgilg2,pattern size=6pt,pattern thickness=0.75pt,pattern radius=0pt, pattern color={rgb, 255:red, 0; green, 0; blue, 0}] (249.81,214.34) .. controls (250.46,211.18) and (252.43,208.92) .. (254.21,209.29) .. controls (256,209.65) and (256.91,212.51) .. (256.27,215.67) .. controls (255.62,218.82) and (253.65,221.09) .. (251.87,220.72) .. controls (250.08,220.36) and (249.17,217.5) .. (249.81,214.34) -- cycle ;
	%Shape: Ellipse [id:dp2655841079256809] 
	\draw  [pattern=_e08cmt92t,pattern size=6pt,pattern thickness=0.75pt,pattern radius=0pt, pattern color={rgb, 255:red, 0; green, 0; blue, 0}] (359.13,215.02) .. controls (357.4,207.37) and (357.39,200.85) .. (359.12,200.46) .. controls (360.84,200.07) and (363.63,205.96) .. (365.36,213.61) .. controls (367.09,221.26) and (367.09,227.77) .. (365.37,228.16) .. controls (363.65,228.55) and (360.86,222.67) .. (359.13,215.02) -- cycle ;
	%Shape: Ellipse [id:dp14554770343196943] 
	\draw  [pattern=_kgwtsdvho,pattern size=6pt,pattern thickness=0.75pt,pattern radius=0pt, pattern color={rgb, 255:red, 0; green, 0; blue, 0}] (411.03,213.3) .. controls (412.34,203.61) and (415.17,196) .. (417.36,196.3) .. controls (419.56,196.59) and (420.27,204.68) .. (418.97,214.37) .. controls (417.66,224.05) and (414.83,231.66) .. (412.64,231.37) .. controls (410.44,231.07) and (409.73,222.98) .. (411.03,213.3) -- cycle ;
	%Shape: Ellipse [id:dp0007511490678715482] 
	\draw  [pattern=_yx0ef09wr,pattern size=6pt,pattern thickness=0.75pt,pattern radius=0pt, pattern color={rgb, 255:red, 0; green, 0; blue, 0}] (281.99,149.75) .. controls (282.64,148.43) and (283.76,147.66) .. (284.48,148.02) .. controls (285.21,148.38) and (285.27,149.74) .. (284.61,151.05) .. controls (283.96,152.37) and (282.84,153.14) .. (282.12,152.78) .. controls (281.39,152.42) and (281.33,151.06) .. (281.99,149.75) -- cycle ;
	%Shape: Ellipse [id:dp3456018412872477] 
	\draw  [pattern=_5u080t6v1,pattern size=6pt,pattern thickness=0.75pt,pattern radius=0pt, pattern color={rgb, 255:red, 0; green, 0; blue, 0}] (343.18,150.83) .. controls (341.67,147.12) and (341.58,143.66) .. (342.98,143.1) .. controls (344.37,142.53) and (346.72,145.07) .. (348.22,148.77) .. controls (349.73,152.48) and (349.82,155.94) .. (348.42,156.5) .. controls (347.03,157.07) and (344.68,154.53) .. (343.18,150.83) -- cycle ;
	%Shape: Ellipse [id:dp4638117508530233] 
	\draw  [pattern=_4zbbp5c0r,pattern size=6pt,pattern thickness=0.75pt,pattern radius=0pt, pattern color={rgb, 255:red, 0; green, 0; blue, 0}] (409.93,149.37) .. controls (408.9,143) and (409.44,137.62) .. (411.14,137.34) .. controls (412.85,137.07) and (415.07,142) .. (416.1,148.37) .. controls (417.13,154.74) and (416.59,160.12) .. (414.89,160.4) .. controls (413.18,160.68) and (410.96,155.74) .. (409.93,149.37) -- cycle ;

	% Text Node
	\draw (231.67,147.42) node    {$Q$};
	% Text Node
	\draw (148.33,213.42) node    {$Q$};


	\end{tikzpicture}
\end{figure}
\FloatBarrier
Invece in caso di più cariche si avrebbe che i flussi si sommano:
\begin{align*}
	\Phi_{\Sigma}(\vec{E}) &=\int_{\Sigma}\vec{E}\cdot \vec{n} dS= \int_{\Sigma}(\Sigma_i\vec{E}_i)\cdot \vec{n} dS  \\
	&= \int_{\Sigma} \Sigma_i(\vec{E}_i\cdot \vec{n}  )dS= \Sigma_i \int_{\Sigma} \vec{E}_i\cdot \vec{n} dS \\
	&= \Sigma_i\Phi_{\Sigma}^{(i)}(\vec{E} )  \\
\end{align*}
\[
	\boxed{\Phi_{\Sigma}(\vec{E} )=\frac{Q_{\text{tot}}^{\text{interna a} \Sigma}}{\varepsilon_0}}
\]
\begin{figure}[htpb]
	\centering
	

	\tikzset{every picture/.style={line width=0.75pt}} %set default line width to 0.75pt        

	\begin{tikzpicture}[x=0.75pt,y=0.75pt,yscale=-1,xscale=1]
	%uncomment if require: \path (0,300); %set diagram left start at 0, and has height of 300

	%Shape: Ellipse [id:dp6166042155684706] 
	\draw   (100,109.5) .. controls (100,67.8) and (166.15,34) .. (247.75,34) .. controls (329.35,34) and (395.5,67.8) .. (395.5,109.5) .. controls (395.5,151.2) and (329.35,185) .. (247.75,185) .. controls (166.15,185) and (100,151.2) .. (100,109.5) -- cycle ;
	%Shape: Circle [id:dp4500030224947258] 
	\draw  [fill={rgb, 255:red, 0; green, 0; blue, 0 }  ,fill opacity=1 ] (187.14,63.15) .. controls (188.04,61.76) and (189.89,61.36) .. (191.29,62.25) .. controls (192.68,63.15) and (193.08,65.01) .. (192.18,66.4) .. controls (191.28,67.79) and (189.43,68.19) .. (188.03,67.3) .. controls (186.64,66.4) and (186.24,64.54) .. (187.14,63.15) -- cycle ;
	%Shape: Circle [id:dp9796248018777871] 
	\draw  [fill={rgb, 255:red, 0; green, 0; blue, 0 }  ,fill opacity=1 ] (282.14,60.15) .. controls (283.04,58.76) and (284.89,58.36) .. (286.29,59.25) .. controls (287.68,60.15) and (288.08,62.01) .. (287.18,63.4) .. controls (286.28,64.79) and (284.43,65.19) .. (283.03,64.3) .. controls (281.64,63.4) and (281.24,61.54) .. (282.14,60.15) -- cycle ;
	%Shape: Circle [id:dp09463099152199295] 
	\draw  [fill={rgb, 255:red, 0; green, 0; blue, 0 }  ,fill opacity=1 ] (213.14,122.15) .. controls (214.04,120.76) and (215.89,120.36) .. (217.29,121.25) .. controls (218.68,122.15) and (219.08,124.01) .. (218.18,125.4) .. controls (217.28,126.79) and (215.43,127.19) .. (214.03,126.3) .. controls (212.64,125.4) and (212.24,123.54) .. (213.14,122.15) -- cycle ;
	%Shape: Circle [id:dp33538641669305247] 
	\draw  [fill={rgb, 255:red, 0; green, 0; blue, 0 }  ,fill opacity=1 ] (316.14,125.15) .. controls (317.04,123.76) and (318.89,123.36) .. (320.29,124.25) .. controls (321.68,125.15) and (322.08,127.01) .. (321.18,128.4) .. controls (320.28,129.79) and (318.43,130.19) .. (317.03,129.3) .. controls (315.64,128.4) and (315.24,126.54) .. (316.14,125.15) -- cycle ;
	%Shape: Circle [id:dp8337517217710169] 
	\draw  [fill={rgb, 255:red, 0; green, 0; blue, 0 }  ,fill opacity=1 ] (478.14,47.15) .. controls (479.04,45.76) and (480.89,45.36) .. (482.29,46.25) .. controls (483.68,47.15) and (484.08,49.01) .. (483.18,50.4) .. controls (482.28,51.79) and (480.43,52.19) .. (479.03,51.3) .. controls (477.64,50.4) and (477.24,48.54) .. (478.14,47.15) -- cycle ;
	%Shape: Circle [id:dp4238901690042187] 
	\draw  [fill={rgb, 255:red, 0; green, 0; blue, 0 }  ,fill opacity=1 ] (526.14,118.15) .. controls (527.04,116.76) and (528.89,116.36) .. (530.29,117.25) .. controls (531.68,118.15) and (532.08,120.01) .. (531.18,121.4) .. controls (530.28,122.79) and (528.43,123.19) .. (527.03,122.3) .. controls (525.64,121.4) and (525.24,119.54) .. (526.14,118.15) -- cycle ;

	% Text Node
	\draw (191.33,79.42) node    {$Q_{1}$};
	% Text Node
	\draw (286.33,76.42) node    {$Q_{3}$};
	% Text Node
	\draw (217.33,138.42) node    {$Q_{2}$};
	% Text Node
	\draw (320.33,141.42) node    {$Q_{4}$};
	% Text Node
	\draw (482.33,63.42) node    {$Q_{5}$};
	% Text Node
	\draw (530.33,134.42) node    {$Q_{6}$};
	% Text Node
	\draw (410.5,144) node    {$\Sigma _{\text{chiusa}}$};


	\end{tikzpicture}
\end{figure}
\FloatBarrier
Se abbiamo un vero e proprio oggetto all'interno di $\Sigma$, dobbiamo andare a considerare la densità di carica di volume. E quindi:
\[
	Q_{\text{tot}}^{\text{interna a } \Sigma} = \int_{\tau}\rho d\tau
\]




















\subsection{Applicazione del teorema di Gauss}

La legge di Gauss diventa uno strumento molto potente per determinare
il campo $\vec{E}$ nei casi in cui la distribuzione di carica che genera il campo presenti un elevato grado di \emph{simmetria} (sferica, cilindrica, piana). In queste condizioni di norma è facile individuare a priori l'andamento delle linee di forza e trovare di conseguenza delle superfici chiuse nei cui punti il campo è parallelo o ortogonale alla superficie stessa, per cui i contributi $\vec{E} \cdot \vec{n} \,dS$ o sono nulli o si scrivono facilmente.
\begin{figure}[htpb]
	\centering
	

	\tikzset{every picture/.style={line width=0.75pt}} %set default line width to 0.75pt        

	\begin{tikzpicture}[x=0.75pt,y=0.75pt,yscale=-1,xscale=1]
	%uncomment if require: \path (0,300); %set diagram left start at 0, and has height of 300

	%Shape: Rectangle [id:dp6271259438150636] 
	\draw   (230,11) -- (243,11) -- (243,275) -- (230,275) -- cycle ;
	%Straight Lines [id:da30089470945639274] 
	\draw    (243,176) -- (357,176) ;
	\draw [shift={(360,176)}, rotate = 180] [fill={rgb, 255:red, 0; green, 0; blue, 0 }  ][line width=0.08]  [draw opacity=0] (10.72,-5.15) -- (0,0) -- (10.72,5.15) -- (7.12,0) -- cycle    ;
	%Straight Lines [id:da9835619887853664] 
	\draw    (245,223) -- (301,223) ;
	\draw [shift={(301,223)}, rotate = 180] [color={rgb, 255:red, 0; green, 0; blue, 0 }  ][line width=0.75]    (0,5.59) -- (0,-5.59)   ;
	\draw [shift={(245,223)}, rotate = 180] [color={rgb, 255:red, 0; green, 0; blue, 0 }  ][line width=0.75]    (0,5.59) -- (0,-5.59)   ;
	%Shape: Circle [id:dp6971224069947288] 
	\draw  [fill={rgb, 255:red, 0; green, 0; blue, 0 }  ,fill opacity=1 ] (303,131) .. controls (303,129.34) and (304.34,128) .. (306,128) .. controls (307.66,128) and (309,129.34) .. (309,131) .. controls (309,132.66) and (307.66,134) .. (306,134) .. controls (304.34,134) and (303,132.66) .. (303,131) -- cycle ;
	%Straight Lines [id:da7901589292957081] 
	\draw    (153.67,236.5) -- (153.67,58.83) ;
	\draw [shift={(153.67,58.83)}, rotate = 450] [color={rgb, 255:red, 0; green, 0; blue, 0 }  ][line width=0.75]    (0,5.59) -- (0,-5.59)   ;
	\draw [shift={(153.67,236.5)}, rotate = 450] [color={rgb, 255:red, 0; green, 0; blue, 0 }  ][line width=0.75]    (0,5.59) -- (0,-5.59)   ;
	%Shape: Can [id:dp5914560994290878] 
	\draw   (306,58.2) -- (306,234.8) .. controls (306,246.23) and (275.11,255.5) .. (237,255.5) .. controls (198.89,255.5) and (168,246.23) .. (168,234.8) -- (168,58.2) .. controls (168,46.77) and (198.89,37.5) .. (237,37.5) .. controls (275.11,37.5) and (306,46.77) .. (306,58.2) .. controls (306,69.63) and (275.11,78.9) .. (237,78.9) .. controls (198.89,78.9) and (168,69.63) .. (168,58.2) ;
	%Straight Lines [id:da18093708742362602] 
	\draw    (306,100) -- (335,100) ;
	\draw [shift={(338,100)}, rotate = 180] [fill={rgb, 255:red, 0; green, 0; blue, 0 }  ][line width=0.08]  [draw opacity=0] (10.72,-5.15) -- (0,0) -- (10.72,5.15) -- (7.12,0) -- cycle    ;

	% Text Node
	\draw (236.4,23.4) node    {$+$};
	% Text Node
	\draw (236.4,43.4) node    {$+$};
	% Text Node
	\draw (236.4,63.4) node    {$+$};
	% Text Node
	\draw (236.4,83.4) node    {$+$};
	% Text Node
	\draw (236.4,103.4) node    {$+$};
	% Text Node
	\draw (236.4,123.4) node    {$+$};
	% Text Node
	\draw (236.4,143.4) node    {$+$};
	% Text Node
	\draw (236.4,163.4) node    {$+$};
	% Text Node
	\draw (236.4,183.4) node    {$+$};
	% Text Node
	\draw (236.4,203.4) node    {$+$};
	% Text Node
	\draw (236.4,223.4) node    {$+$};
	% Text Node
	\draw (236.4,243.4) node    {$+$};
	% Text Node
	\draw (236.4,263.4) node    {$+$};
	% Text Node
	\draw (323.5,129) node    {$P$};
	% Text Node
	\draw (274.83,207.33) node    {$a$};
	% Text Node
	\draw (144.83,146.67) node    {$l$};
	% Text Node
	\draw (371.5,172) node    {$\vec{E}$};
	% Text Node
	\draw (351.5,99) node    {$\vec{n}$};
	% Text Node
	\draw (329.5,40) node    {$\Sigma _{\text{chiusa}}$};


	\end{tikzpicture}
\end{figure}
\FloatBarrier
Consideriamo un filo rettilineo infinito carico con densità di carica lineare costante $\lambda$. Vogliamo determinare l'entità del campo elettrico in un certo punto $P$ ad una certa distanza. Il campo deve essere dotato di simmetria cilindrica. Significa che possiamo in generale scriverlo in tal modo:
\[
	\vec{E} (P) = E(a) \vec{n}
\]
Per applicare il teorema di Gauss dovremo immaginare una superficie astratta, che prende il nome di \textbf{superficie gaussiana}.
Consideriamo una superficie gaussiana che abbia la stessa simmetria cilindrica. Le due basi tuttavia non contribuiscono al flusso totale perché il prodotto scalare fra $\vec{E}$ e $\vec{n}$ sarà zero.
\[
	\Phi (\vec{E}) = \int_{\Sigma} \vec{E} \cdot \vec{n} dS = \Phi_{\text{base 1}} (\vec{E}) + \Phi_{\text{base 2}} (\vec{E}) + \Phi_{\text{sup. laterale}} (\vec{E})
\]
Il versore normale $\vec{n}$ coincide con $\vec{u}_r$ per tutti i punti della superficie laterale. Il primo risultato dell'aver scelto una superficie con la stessa simmetria del campo è che abbiamo semplificato la parte vettoriale. Stiamo calcolando questo integrale su una superficie cilindrica laterale che per definizione è il luogo dei punti equidistanti dal centro dall'asse, tutti alla distanza $a$. $\vec{E}(a)$ è costante, essendo funzione della posizione. Possiamo portare quindi il termine fuori dall'integrale.
\begin{align*}
	\Phi (\vec{E}) &= \int_{\text{sup. laterale}} E(a)\vec{u}_r \cdot \vec{n} dS \\
	\frac{Q_{\text{tot}}^{\Sigma}}{\varepsilon_0} &= E(a) \int_{\text{sup. laterale}} dS \tag*{T. di Gauss a primo membro} \\
	\frac{\int\lambda\,dl}{\varepsilon_0} &= E(a) \; 2\pi a\;l \\
	\frac{\lambda\,l}{\varepsilon_0} &= E(a) \; 2\pi a\;l \tag*{$\lambda$ è costante} \\
	E(a) &= \frac{\lambda}{2\pi a\varepsilon_0} \tag*{Campo elettrico generato da un filo} \\
\end{align*}







































\section{I Equazione di Maxwell (anche in regime tempovariante)}

Si ricava a partire dal teorema di Gauss per il campo elettrico e dal teorema della Divergenza.
Data una superficie $\Sigma$ chiusa, di volume $\tau$ e densità di carica volumetrica $\rho (x,y,z)$ si ha
\[
	\Phi_{\Sigma}(\vec{E}) = \int_{\Sigma}\vec{E} \cdot \vec{n} dS = \frac{Q_{\text{tot}}^{\text{int}}}{\varepsilon_0}
\]
L'espressione della carica interna risulta essere
\[
	Q_{\text{tot}}^{\text{int}} = \int_{\tau} \rho (x,y,z)d\tau
\]
Sostituendola nell'espressione del flusso e applicando il teorema della divergenza si ottiene
\[
	\int_{\tau} \frac{\rho (x,y,z)}{\varepsilon_0}d\tau = \int_{\Sigma}\vec{E} \cdot \vec{n} dS = \int_{\tau} \text{div}\vec{E} \;d\tau
\]
Dal momento che gli integrali non dipendono né da $ \rho (x,y,z) $ né dalla forma della superficie, le funzioni integrande del primo e del terzo integrale devono essere uguali. Per cui
\[
	\boxed{\text{div}\vec{E} = \vec{\nabla} \cdot \vec{E} = \frac{\rho (x,y,z)}{\varepsilon_0}}
\]
Questa equazione ottenuta è detta anche \textbf{prima equazione di Maxwell} ed è in pratica la trasposizione in forma locale del teorema di Gauss. Essa ci dice che se in una regione dello spazio ho una densità di carica maggiore di zero, allora la divergenza è maggiore di zero. Significa che in quel punto c'è la sorgente positiva delle linee di flusso. Se abbiamo cariche negative varrà esattamente il contrario. Se $\rho$ è uguale a zero vale la proprietà sopra citata per il tubo di flusso.
Da questa equazione ricaviamo che:
\[
	\text{div}\vec{E} d\tau =\frac{\rho d\tau}{\varepsilon_0} = \frac{dq}{\varepsilon_0} = d\Phi (\vec{E}) \implies \text{div}\vec{E} = \frac{d\Phi (\vec{E})}{d\tau}
\]




















\subsection{Calcolo della divergenza}

Esiste una formula pratica alternativa per calcolare la divergenza. Scegliamo un sistema di coordinate cartesiane per rappresentare il campo vettoriale.
\[
	\vec{v}(P) = v_x(x,y,z)\vec{u}_x + v_y(x,y,z)\vec{u}_y + v_z(x,y,z)\vec{u}_z
\]
Si può dimostrare che la divergenza del campo $\vec{v}$ è:
\[
	\text{div}\vec{v} = \frac{\partial v_x}{\partial x} + \frac{\partial v_y}{\partial y} + \frac{\partial v_z}{\partial z}
\]
Nel caso di altri sistemi di riferimento la formula può apparire molto più complicata. Ciascuno dei tre componenti dipende da tutte e tre le coordinate dello spazio.
\begin{figure}[htpb]
	\centering
	

	\tikzset{every picture/.style={line width=0.75pt}} %set default line width to 0.75pt        

	\begin{tikzpicture}[x=0.75pt,y=0.75pt,yscale=-1,xscale=1]
	%uncomment if require: \path (0,300); %set diagram left start at 0, and has height of 300

	%Straight Lines [id:da5013518463598057] 
	\draw    (130.88,228) -- (382.67,228) ;
	\draw [shift={(385.67,228)}, rotate = 180] [fill={rgb, 255:red, 0; green, 0; blue, 0 }  ][line width=0.08]  [draw opacity=0] (10.72,-5.15) -- (0,0) -- (10.72,5.15) -- (7.12,0) -- cycle    ;
	%Straight Lines [id:da6443750190887165] 
	\draw    (142.88,243) -- (142.88,49) ;
	\draw [shift={(142.88,46)}, rotate = 450] [fill={rgb, 255:red, 0; green, 0; blue, 0 }  ][line width=0.08]  [draw opacity=0] (10.72,-5.15) -- (0,0) -- (10.72,5.15) -- (7.12,0) -- cycle    ;
	%Straight Lines [id:da8305005539386254] 
	\draw    (130.88,240) -- (190.76,180.12) ;
	\draw [shift={(192.88,178)}, rotate = 495] [fill={rgb, 255:red, 0; green, 0; blue, 0 }  ][line width=0.08]  [draw opacity=0] (10.72,-5.15) -- (0,0) -- (10.72,5.15) -- (7.12,0) -- cycle    ;
	%Shape: Cube [id:dp6333624936608848] 
	\draw   (225,107.45) -- (264.45,68) -- (356.5,68) -- (356.5,162.55) -- (317.05,202) -- (225,202) -- cycle ; \draw   (356.5,68) -- (317.05,107.45) -- (225,107.45) ; \draw   (317.05,107.45) -- (317.05,202) ;
	%Straight Lines [id:da0547620432904663] 
	\draw  [dash pattern={on 0.84pt off 2.51pt}]  (264.45,162.55) -- (264.45,68) ;
	%Straight Lines [id:da7246817021460845] 
	\draw  [dash pattern={on 0.84pt off 2.51pt}]  (356.5,162.55) -- (264,162.55) ;
	%Straight Lines [id:da9803429259301326] 
	\draw  [dash pattern={on 0.84pt off 2.51pt}]  (264.45,162.55) -- (225.33,201.67) ;
	%Straight Lines [id:da5389134330495531] 
	\draw    (338.21,140) -- (414,140) ;
	\draw [shift={(417,140)}, rotate = 180] [fill={rgb, 255:red, 0; green, 0; blue, 0 }  ][line width=0.08]  [draw opacity=0] (10.72,-5.15) -- (0,0) -- (10.72,5.15) -- (7.12,0) -- cycle    ;
	%Straight Lines [id:da1347325077764896] 
	\draw    (338.21,140) -- (397.01,109.06) ;
	\draw [shift={(399.67,107.67)}, rotate = 512.25] [fill={rgb, 255:red, 0; green, 0; blue, 0 }  ][line width=0.08]  [draw opacity=0] (10.72,-5.15) -- (0,0) -- (10.72,5.15) -- (7.12,0) -- cycle    ;
	%Straight Lines [id:da48588431974198354] 
	\draw    (243.55,138.67) -- (182.67,138.67) ;
	\draw [shift={(179.67,138.67)}, rotate = 360] [fill={rgb, 255:red, 0; green, 0; blue, 0 }  ][line width=0.08]  [draw opacity=0] (10.72,-5.15) -- (0,0) -- (10.72,5.15) -- (7.12,0) -- cycle    ;
	%Straight Lines [id:da7358212844953658] 
	\draw    (243.55,138.67) -- (274.46,119.26) ;
	\draw [shift={(277,117.67)}, rotate = 507.88] [fill={rgb, 255:red, 0; green, 0; blue, 0 }  ][line width=0.08]  [draw opacity=0] (10.72,-5.15) -- (0,0) -- (10.72,5.15) -- (7.12,0) -- cycle    ;

	% Text Node
	\draw (396.67,226.67) node    {$x$};
	% Text Node
	\draw (129.33,48.67) node    {$z$};
	% Text Node
	\draw (173.33,176) node    {$y$};
	% Text Node
	\draw (218,212.67) node    {$A$};
	% Text Node
	\draw (318.67,213.33) node    {$A'$};
	% Text Node
	\draw (370,160.67) node    {$B'$};
	% Text Node
	\draw (267.33,172.33) node    {$B$};
	% Text Node
	\draw (214,103.33) node    {$D$};
	% Text Node
	\draw (257.33,57.33) node    {$C$};
	% Text Node
	\draw (368.67,58.67) node    {$C'$};
	% Text Node
	\draw (329.67,114) node    {$D'$};
	% Text Node
	\draw (434.67,140.67) node    {$\vec{u}_{x}$};
	% Text Node
	\draw (411.33,99.33) node    {$\vec{E} '$};
	% Text Node
	\draw (286.67,124.67) node    {$\vec{E}$};
	% Text Node
	\draw (165.33,136) node    {$-\vec{u}_{x}$};


	\end{tikzpicture}
\end{figure}
\FloatBarrier
\emph{Dimostrazione.} Consideriamo un parallelepipedo infinitesimo con gli spigoli $dx$, $dy$, $dz$ paralleli agli assi che che contiene la carica $dq=\rho(x,y,z) d\tau $, essendo $d\tau = dx\ dy\ dz$ il volume del parallelepipedo. Il flusso attraverso la superficie $A' B' C' D'$, perpendicolare all'asse $x$, è dato da:
\[
	\vec{E}' \vec{u}_x dydz = E_x'dydz
\]
Se indichiamo con $E_x'$ la componente di $\vec{E}'$ parallela all'asse $x$. Analogamente, il flusso attraverso la faccia $ABCD$ è:
\[
	\vec{E} (-\vec{u}_x)dydz = -E_x dydz
\]
Infatti la normale alla superficie orientata verso l'esterno è $-\vec{u}_x$ per la faccia $ABCD$. Complessivamente
\[
	(E_x'-E_x)dydz = \frac{\partial E_x}{\partial x} dx\,dy\,dz
\]
È il flusso attraverso le due superfici considerate. Espressioni simili si hanno per le altre coppie di superficie e quindi il flusso attraverso la superficie del parallelepipedo è:
\begin{align*}
	d\Phi &= \left( \frac{\partial E_x}{\partial x} +\frac{\partial E_y}{\partial y} + \frac{\partial E_z}{\partial z}  \right) dx\,dy\,dz \\
	&= \left( \frac{\partial E_x}{\partial x} +\frac{\partial E_y}{\partial y} + \frac{\partial E_z}{\partial z}  \right) d\tau
\end{align*}
Si ha che la divergenza del campo $\vec{E}$ è:
\[
	\boxed{\text{div}\vec{E} = \vec{\nabla} \cdot \vec{E} = \frac{d\Phi}{d\tau} = \frac{\partial E_x}{\partial x} + \frac{\partial E_y}{\partial y} + \frac{\partial E_z}{\partial z}}
\]
La divergenza del campo nel punto $P$ è data dal rapporto tra il flusso attraverso la superficie di un parallelepipedo infinitesimo centrato su $P$ e il volume del parallelepipedo.







































\section{Conservatività del Campo Elettrico}

Come già affrontato in Fisica I, si dimostra che il lavoro compiuto dalla risultante delle forze agenti sull'oggetto, $\vec{R}$, è uguale alla variazione di energia cinetica dell'oggetto. In genere il lavoro dipende anche dal percorso scelto per andare da $A$ a $B$. Nel caso del teorema dell'energia cinetica, a parità di punti $A$ e $B$ e di velocità iniziale, a seconda del percorso che scelgo potrei avere la velocità finale diversa. Tuttavia talvolta possiamo lavorare con campi di forza di particolari proprietà.

Nel caso di forze conservative, qualunque sia la traiettoria seguita, la somma dei lavori compiuti è sempre la stessa, basta che siano fissati $A$ e $B$. Accade che allora possiamo esprimere il lavoro con un campo scalare che chiamiamo energia potenziale. Il lavoro di $\vec{R}$ lungo una traiettoria si può scrivere come la differenza fra l'energia potenziale nel punto di partenza e quella nel punto di arrivo. Lavoro ed energia si misurano in Joule.
Nel caso più generale in cui il nostro oggetto si muova in presenza sia di forze conservative che di forze non conservative, si dimostra un bilancio energetico molto importante. Definita energia meccanica dell'oggetto come la somma dell'energia cinetica $K$ e dell'energia potenziale $U$ del nostro oggetto, si dimostra che da $A$ a $B$ la variazione dell'energia meccanica è uguale al lavoro delle forze non conservative
\[
	\Delta E_{\text{mecc}} = E(B) - E(A) = \mathcal{L}_{NC}
\]
Se le forze non conservative sono nulle, allora vale il principio di conservazione dell'energia meccanica. Utilizzeremo questa proprietà quando ci occuperemo del moto di cariche elettriche.
Durante il moto delle particelle l'energia totale, somma dell'energia cinetica e dell'energia potenziale, rimane costante:
\[
	\Delta E_K + \Delta E_p = 0
\]
Dalla formula si vede chiaramente che scegliendo opportunamente il segno della differenza di potenziale è possibile accelerare la particella, trasformando l'energia potenziale in energia cinetica. Una carica positiva è accelerata se $V(A)>V(B)$ mentre una carica negativa è accelerata se $V(B)>V(A)$. Se $V(B)=V(A)$ non c'è alcun effetto complessivo; ciò non vuol dire che tra $A$ e $B$ non c'è campo, ma semplicemente che nel percorso $A\to B$ ci sono zone in cui l'effetto è accelerante e altre in cui è decelerante. In particolare $V(A)=V(B)$ se $A$ e $B$ coincidono. Alla fine di un percorso chiuso l'energia cinetica è la stessa che all'inizio, può essere cambiata in direzione, ma non di modulo.
Per le forze conservative vale anche un corollario. Se consideriamo un percorso $\Gamma$ in cui l'oggetto parte da $A$ e ritorna in $A$, allora in questo caso il lavoro compiuto dalla forza è zero.
\[
	\mathcal{L}_{AB}^{(FC)} = \oint_{\gamma} \vec{F}_cd\vec{l} = 0 = U(A) - U(A)
\]
Chiameremo l'integrale di linea su una linea chiusa \textbf{circuitazione}.

Sia una carica $Q$ fissa nello spazio ed una seconda carica esploratrice $q$ che si muove da un punto $A$ ad un punto $B$ sotto l'influenza di $Q$.
\begin{figure}[htpb]
	\centering
	

	\tikzset{every picture/.style={line width=0.75pt}} %set default line width to 0.75pt        

	\begin{tikzpicture}[x=0.75pt,y=0.75pt,yscale=-1,xscale=1]
	%uncomment if require: \path (0,300); %set diagram left start at 0, and has height of 300

	%Curve Lines [id:da26092883666409095] 
	\draw    (80,106) .. controls (147.5,75) and (284.5,68) .. (408.5,161) ;
	\draw [shift={(251.49,91.11)}, rotate = 190.1] [fill={rgb, 255:red, 0; green, 0; blue, 0 }  ][line width=0.08]  [draw opacity=0] (10.72,-5.15) -- (0,0) -- (10.72,5.15) -- (7.12,0) -- cycle    ;
	%Shape: Circle [id:dp8590828188159274] 
	\draw  [fill={rgb, 255:red, 0; green, 0; blue, 0 }  ,fill opacity=1 ] (235.14,205.15) .. controls (236.04,203.76) and (237.89,203.36) .. (239.29,204.25) .. controls (240.68,205.15) and (241.08,207.01) .. (240.18,208.4) .. controls (239.28,209.79) and (237.43,210.19) .. (236.03,209.3) .. controls (234.64,208.4) and (234.24,206.54) .. (235.14,205.15) -- cycle ;
	%Straight Lines [id:da42137116431156296] 
	\draw    (237.66,206.77) -- (315.62,56) ;
	\draw [shift={(317,53.33)}, rotate = 477.34] [fill={rgb, 255:red, 0; green, 0; blue, 0 }  ][line width=0.08]  [draw opacity=0] (10.72,-5.15) -- (0,0) -- (10.72,5.15) -- (7.12,0) -- cycle    ;
	%Straight Lines [id:da9044377847250542] 
	\draw    (237.66,206.77) -- (290.29,104.99) ;
	\draw [shift={(291.67,102.33)}, rotate = 477.34] [fill={rgb, 255:red, 0; green, 0; blue, 0 }  ][line width=0.08]  [draw opacity=0] (10.72,-5.15) -- (0,0) -- (10.72,5.15) -- (7.12,0) -- cycle    ;
	%Straight Lines [id:da7481439852023828] 
	\draw    (237.66,206.77) -- (392.17,152.99) ;
	\draw [shift={(395,152)}, rotate = 520.81] [fill={rgb, 255:red, 0; green, 0; blue, 0 }  ][line width=0.08]  [draw opacity=0] (10.72,-5.15) -- (0,0) -- (10.72,5.15) -- (7.12,0) -- cycle    ;
	%Straight Lines [id:da34856183856557954] 
	\draw    (292.67,101.33) -- (402.24,131.6) ;
	\draw [shift={(405.13,132.4)}, rotate = 195.44] [fill={rgb, 255:red, 0; green, 0; blue, 0 }  ][line width=0.08]  [draw opacity=0] (10.72,-5.15) -- (0,0) -- (10.72,5.15) -- (7.12,0) -- cycle    ;
	%Shape: Boxed Line [id:dp6249319096386896] 
	\draw    (395,152) -- (403.76,135.06) ;
	\draw [shift={(405.13,132.4)}, rotate = 477.34] [fill={rgb, 255:red, 0; green, 0; blue, 0 }  ][line width=0.08]  [draw opacity=0] (10.72,-5.15) -- (0,0) -- (10.72,5.15) -- (7.12,0) -- cycle    ;
	%Straight Lines [id:da14116411953143926] 
	\draw  [dash pattern={on 0.84pt off 2.51pt}]  (302.8,81.73) -- (405.13,132.4) ;
	%Shape: Arc [id:dp31567400253761524] 
	\draw  [draw opacity=0] (296.84,91.94) .. controls (300.44,93.54) and (302.94,97.14) .. (302.94,101.33) .. controls (302.94,102.26) and (302.82,103.16) .. (302.58,104.01) -- (292.67,101.33) -- cycle ; \draw   (296.84,91.94) .. controls (300.44,93.54) and (302.94,97.14) .. (302.94,101.33) .. controls (302.94,102.26) and (302.82,103.16) .. (302.58,104.01) ;
	%Shape: Circle [id:dp4555289917735088] 
	\draw  [fill={rgb, 255:red, 0; green, 0; blue, 0 }  ,fill opacity=1 ] (77,106) .. controls (77,104.34) and (78.34,103) .. (80,103) .. controls (81.66,103) and (83,104.34) .. (83,106) .. controls (83,107.66) and (81.66,109) .. (80,109) .. controls (78.34,109) and (77,107.66) .. (77,106) -- cycle ;
	%Shape: Circle [id:dp09958237772920753] 
	\draw  [fill={rgb, 255:red, 0; green, 0; blue, 0 }  ,fill opacity=1 ] (405.5,161) .. controls (405.5,159.34) and (406.84,158) .. (408.5,158) .. controls (410.16,158) and (411.5,159.34) .. (411.5,161) .. controls (411.5,162.66) and (410.16,164) .. (408.5,164) .. controls (406.84,164) and (405.5,162.66) .. (405.5,161) -- cycle ;

	% Text Node
	\draw (223.33,204.42) node    {$Q$};
	% Text Node
	\draw (329.33,53.33) node    {$\vec{F}$};
	% Text Node
	\draw (292,142.67) node    {$\vec{r}_{1}$};
	% Text Node
	\draw (309.2,94.87) node    {$\vartheta $};
	% Text Node
	\draw (147.5,72.07) node    {$\gamma $};
	% Text Node
	\draw (323.6,190.67) node    {$\vec{r}_{2}$};
	% Text Node
	\draw (414.4,145.07) node    {$d\vec{r}$};
	% Text Node
	\draw (385.2,103.87) node    {$d\vec{l}$};
	% Text Node
	\draw (237.2,58.67) node    {$dr=dl\cos \vartheta $};
	% Text Node
	\draw (71.33,116.42) node    {$A$};
	% Text Node
	\draw (419.83,169.42) node    {$B$};


	\end{tikzpicture}
\end{figure}
\FloatBarrier
La forza agente sulla carica sarà repulsiva e sappiamo calcolarla grazie alla legge di Coulomb.
Calcoliamo il lavoro che $\vec{F}$ compie nello spostare $q$ da $A$ a $B$.
\begin{align*}
	\mathcal{L}_{AB,\gamma} = \int_{A,\gamma}^B \vec{F} \cdot d\vec{l} &= \int_A^B \frac{Qq\vec{u}_r\cdot d\vec{l}}{4\pi \varepsilon_0 r^2}   \tag*{$\vec{u}_r\cdot d\vec{l} = dl\,\cos \vartheta $} \\
	&= \int_A^B \frac{Qq\,dl\,\cos \vartheta}{4\pi \varepsilon_0 r^2} \tag*{$dl\,\cos \vartheta = dr $}\\
	&= \int_A^B \frac{Qq\,dr}{4\pi \varepsilon_0 r^2} \\
	&= \frac{Qq}{4\pi \varepsilon_0}\int_{r_A}^{r_B} \frac{dr}{r^2} \\
	&= \frac{Qq}{4\pi \varepsilon_0} \left[ -\frac{1}{r} \right]_{r_A}^{r_B} = \underbrace{\frac{Qq}{4\pi \varepsilon_0 r_A}}_{U(A)} - \underbrace{\frac{Qq}{4\pi \varepsilon_0 r_B}}_{U(B)}
\end{align*}
Vediamo che $\vec{F}$ è conservativa perché dipende solo dal punto di partenza e da quello di arrivo. L'energia potenziale elettrostatica sarà quindi data da:
\[
	\boxed{U_{\text{elettrostatica}} = \frac{Qq}{4\pi \varepsilon_0 r}}
\]
La convenzione è che l'energia potenziale sia nulla all'infinito ($U(\infty)=0$), tuttavia è pur sempre possibile definire l'energia potenziale elettrostatica a meno di una costante.
La forza coulombiana è una forza conservativa. Sappiamo che è possibile esprimere $\vec{F}$ se conosciamo il valore del campo elettrico nel punto occupato dalla carica esploratrice $ \vec{F} (P) = q\vec{E} (P) $. Possiamo sostituire questa uguaglianza nell'espressione del lavoro e otterremo che:
\begin{align*}
	\mathcal{L}_{AB} &= \int_A^B  \vec{F} \cdot d\vec{l}  = \int_A^B  q\vec{E} (P) \cdot d\vec{l} = q \int_A^B \vec{E} (P) \cdot d\vec{l} \\
	\frac{\mathcal{L}_{AB}}{q} &= \int_A^B \vec{E} (P) \cdot d\vec{l} \\
	\frac{U(A) - U(B)}{q}&= \int_A^B \vec{E} (P) \cdot d\vec{l} \\
	\underbrace{\frac{U(A)}{q}}_{V(A)} - \underbrace{\frac{U(B)}{q}}_{V(B)} &= \int_A^B \vec{E} (P) \cdot d\vec{l} \\
	V(A) - V(B) &= \int_A^B \vec{E} (P) \cdot d\vec{l} \\
	- \Delta V &= \int_A^B \vec{E} (P) \cdot d\vec{l}
\end{align*}
Chiamiamo l'espressione
\[
	\boxed{V(P) = \frac{U(P)}{q}}
\]
\textbf{potenziale elettrostatico} generato nel punto $P$ dalla carica sorgente $Q$. Avendo già definito l'energia potenziale elettrostatica, risulta che:
\[
	V(P) = \frac{U(P)}{q} = \frac{\frac{Qq}{4\pi \varepsilon_0 r}}{q} = \frac{Q}{4\pi \varepsilon_0 r}
\]
La dipendenza dalla carica esploratrice scompare. Anche l'energia potenziale è definita a meno di una costante ma ciò non ha rilevanza perché noi avremo sempre a che fare con delle differenze. Le dimensioni di questa nuova funzione introdotta sono quelle di una energia su una carica e dunque l'unità di misura sarebbe:
\[
	\frac{[E]}{[C]} = \left( \frac{J}{C} \right) = V \quad \text{Volt}
\]
Il potenziale ha una importanza notevole nella teoria dell'elettromagnetismo, quindi ribattezziamo questa unità di misura con un nome proprio, il Volt. Fissato il potenziale all'infinito pari a $0$, possiamo immaginare il potenziale in un punto $P$ come il lavoro che la forza elettrica compie per spostare una carica positiva unitaria da quel punto all'infinito.
L'integrale trovato
\[
	\boxed{\int_A^B \vec{E} \cdot d\vec{l} = V(A) - V(B)}
\]
dipende solo dai valori che la funzione $V(P)$ assume nel punto di partenza e di arrivo. Questa proprietà ci ricorda quella delle forze conservative. Tale considerazione ci permette di estendere il concetto di conservatività a qualunque campo vettoriale. Diciamo che \emph{un campo vettoriale è conservativo se l'integrale di linea in $d\vec{l}$ dipende solo dagli estremi e non dal percorso}. In questa accezione il campo elettrostatico è un campo conservativo. I campi elettrici non statici ma variabili nel tempo \emph{possono non essere conservativi}.

Tornando a considerare il lavoro compiuto dalla forza $\vec{F}$ per spostare $q$ da $A$ a $B$ una volta introdotto il potenziale elettrostatico, vediamo che esso si può anche riscrivere come:
\begin{gather*}
	\mathcal{L} = U(A) - U(B) = qV(A) - qV(B) = q[V(A)-V(B)] = -q\Delta V \\
	\boxed{\mathcal{L} = - q\Delta V}
\end{gather*}
Se vogliamo sapere il lavoro che le forze di Coulomb compiono per spostare la carica da $A$ a $B$, basta conoscere il potenziale elettrostatico in $A$ e in $B$. Ecco perché una delle unità di misura che ricorrono per il lavoro è l'\textbf{elettronvolt}, $eV$. Esso rappresenta l'energia acquisita da un elettrone quando si sposta di una differenza di potenziale di un Volt. Un elettronvolt è $1.6\times 10^{-19}$ Joule.
Un caso particolare è quello in cui eseguiamo una traiettoria chiusa in cui partendo dal punto $A$ torniamo ad $A$:
\[
	\oint_A \vec{E} \cdot d\vec{l} = V(A) - V(B) = 0 \quad \text{dato che} \quad V(A)=V(B)
\]
\begin{figure}[htpb]
	\centering
	

	\tikzset{every picture/.style={line width=0.75pt}} %set default line width to 0.75pt        

	\begin{tikzpicture}[x=0.75pt,y=0.75pt,yscale=-1,xscale=1]
	%uncomment if require: \path (0,300); %set diagram left start at 0, and has height of 300

	%Shape: Circle [id:dp5572441369467569] 
	\draw   (207,110.75) .. controls (207,78.86) and (232.86,53) .. (264.75,53) .. controls (296.64,53) and (322.5,78.86) .. (322.5,110.75) .. controls (322.5,142.64) and (296.64,168.5) .. (264.75,168.5) .. controls (232.86,168.5) and (207,142.64) .. (207,110.75) -- cycle ;
	%Shape: Circle [id:dp7421877036925102] 
	\draw  [fill={rgb, 255:red, 0; green, 0; blue, 0 }  ,fill opacity=1 ] (204,110.75) .. controls (204,109.09) and (205.34,107.75) .. (207,107.75) .. controls (208.66,107.75) and (210,109.09) .. (210,110.75) .. controls (210,112.41) and (208.66,113.75) .. (207,113.75) .. controls (205.34,113.75) and (204,112.41) .. (204,110.75) -- cycle ;

	% Text Node
	\draw (175.2,109.4) node    {$A=B$};


	\end{tikzpicture}
\end{figure}
\FloatBarrier
Qualunque sia la linea chiusa, la circuitazione del campo elettrostatico è nulla. Ciò vale in generale nel caso di un campo conservativo. Se consideriamo una distribuzione di carica che genera questo campo elettrostatico, possiamo sempre immaginare che essa si possa scomporre in tante cariche puntiformi che eserciteranno sulla carica esploratrice una forza conservativa e quindi, ricordando che per le forze vale il principio di sovrapposizione degli effetti, \emph{la somma delle forze agenti sarà anch'essa conservativa}.
Ci sono diverse modalità per calcolare il potenziale elettrico in un punto.
Supponiamo di voler calcolare il potenziale di un punto particolare $P$ supponendo di conoscere il campo elettrico di qualsiasi punto dello spazio. Posso fissare il potenziale zero in un punto $P_0$, dal momento che è importante solo la differenza di potenziale. Avremo:
\[
	V(P) - \underbrace{V(P_0)}_{=0} = \int_{P_0}^P \vec{V} (P)d\vec{l} = V(P)
\]
Possiamo considerare anche più cariche puntiformi e sotto la loro influenza studiare il moto di una carica esploratrice $q$ da un punto $A$ a $B$. Per il principio di sovrapposizione degli effetti il campo elettrico totale in $P$ è la sommatoria del campi elettrici generati da ciascuna carica in $P$.
\begin{align*}
	V_{\text{tot}} (A) - V_{\text{tot}} (B) &= \int_A^B \vec{E}_{\text{tot}}d\vec{l} \\
	&= \int_A^B \left( \sum\nolimits_i \vec{E}_i   \right) \cdot  d\vec{l} \\
	&= \sum\nolimits_i \int_A^B \vec{E}_i\cdot d\vec{l} \\
	&= \sum\nolimits_i \left[ V(A)_i - V(B)_i   \right]
\end{align*}
Il potenziale totale generato da questa distribuzione di carica è la somma dei potenziali generati da ciascuna delle cariche presenti. Questo comporta che \emph{anche per il potenziale elettrostatico vale il principio della sovrapposizione degli effetti}. Possiamo allora estendere il ragionamento a una distribuzione di carica qualsiasi.
\begin{itemize}
	\item \emph{Potenziale generato da un volume} $\rho (x',y',z') $ \\
	Consideriamo ora un cubetto di volume $d\tau$ e carica $dq$ centrato in $P'(x',y',z')$. Sia $\vec{r}'$ variabile e $\vec{r}$ costante. Esprimiamo la carica $ dq = \rho (x',y',z') d\tau  $ e il potenziale infinitesimo $ dV = \frac{dq (\vec{r} -\vec{r}_i )}{4\pi \varepsilon_0 |\vec{r} -\vec{r}_i |^2} $ generato dalla distribuzione di carica nel punto $P$. Allora
	\begin{align*}
		V (P) = \int_{\text{oggetto}} dV &= \int_{\text{ogg.}} \frac{dq (\vec{r} -\vec{r}_i )}{4\pi \varepsilon_0 |\vec{r} -\vec{r}_i |^2}\\
		V (P) &= \int_{\tau} \frac{\rho (x',y',z') d\tau (\vec{r} -\vec{r}_i )}{4\pi \varepsilon_0 |\vec{r} -\vec{r}_i |^2} \\
		V (P) &= \int_{\tau} \frac{\rho (x',y',z') (\vec{r} -\vec{r}_i )}{4\pi \varepsilon_0 |\vec{r} -\vec{r}_i |^2}d\tau
	\end{align*}
	\begin{figure}[htpb]
		\centering
		

		\tikzset{every picture/.style={line width=0.75pt}} %set default line width to 0.75pt        

		\begin{tikzpicture}[x=0.75pt,y=0.75pt,yscale=-1,xscale=1]
		%uncomment if require: \path (0,300); %set diagram left start at 0, and has height of 300

		%Straight Lines [id:da1780203758056127] 
		\draw    (185.5,165.67) -- (185.5,85.67) ;
		\draw [shift={(185.5,82.67)}, rotate = 450] [fill={rgb, 255:red, 0; green, 0; blue, 0 }  ][line width=0.08]  [draw opacity=0] (10.72,-5.15) -- (0,0) -- (10.72,5.15) -- (7.12,0) -- cycle    ;
		%Straight Lines [id:da633348072374383] 
		\draw    (185.5,165.67) -- (245.93,202.12) ;
		\draw [shift={(248.5,203.67)}, rotate = 211.1] [fill={rgb, 255:red, 0; green, 0; blue, 0 }  ][line width=0.08]  [draw opacity=0] (10.72,-5.15) -- (0,0) -- (10.72,5.15) -- (7.12,0) -- cycle    ;
		%Straight Lines [id:da5798881183189684] 
		\draw    (185.5,165.67) -- (126.06,202.1) ;
		\draw [shift={(123.5,203.67)}, rotate = 328.5] [fill={rgb, 255:red, 0; green, 0; blue, 0 }  ][line width=0.08]  [draw opacity=0] (10.72,-5.15) -- (0,0) -- (10.72,5.15) -- (7.12,0) -- cycle    ;
		%Shape: Circle [id:dp3554852517874745] 
		\draw  [fill={rgb, 255:red, 0; green, 0; blue, 0 }  ,fill opacity=1 ] (370.17,204.42) .. controls (370.17,203.17) and (371.17,202.17) .. (372.42,202.17) .. controls (373.66,202.17) and (374.67,203.17) .. (374.67,204.42) .. controls (374.67,205.66) and (373.66,206.67) .. (372.42,206.67) .. controls (371.17,206.67) and (370.17,205.66) .. (370.17,204.42) -- cycle ;
		%Straight Lines [id:da32003343045753363] 
		\draw    (185.5,165.67) -- (389.51,92.02) ;
		\draw [shift={(392.33,91)}, rotate = 520.15] [fill={rgb, 255:red, 0; green, 0; blue, 0 }  ][line width=0.08]  [draw opacity=0] (10.72,-5.15) -- (0,0) -- (10.72,5.15) -- (7.12,0) -- cycle    ;
		%Straight Lines [id:da11503891049993031] 
		\draw    (185.5,165.67) -- (367.23,203.8) ;
		\draw [shift={(370.17,204.42)}, rotate = 191.85] [fill={rgb, 255:red, 0; green, 0; blue, 0 }  ][line width=0.08]  [draw opacity=0] (10.72,-5.15) -- (0,0) -- (10.72,5.15) -- (7.12,0) -- cycle    ;
		%Straight Lines [id:da044847521882755315] 
		\draw    (405,102.33) -- (373.35,199.31) ;
		\draw [shift={(372.42,202.17)}, rotate = 288.08] [fill={rgb, 255:red, 0; green, 0; blue, 0 }  ][line width=0.08]  [draw opacity=0] (10.72,-5.15) -- (0,0) -- (10.72,5.15) -- (7.12,0) -- cycle    ;
		%Shape: Circle [id:dp39528053403956975] 
		\draw   (345.33,99.5) .. controls (345.33,59.83) and (377.49,27.67) .. (417.17,27.67) .. controls (456.84,27.67) and (489,59.83) .. (489,99.5) .. controls (489,139.17) and (456.84,171.33) .. (417.17,171.33) .. controls (377.49,171.33) and (345.33,139.17) .. (345.33,99.5) -- cycle ;
		%Shape: Cube [id:dp9932780936605101] 
		\draw   (397.17,82.83) -- (402.17,77.83) -- (422.17,77.83) -- (422.17,94.5) -- (417.17,99.5) -- (397.17,99.5) -- cycle ; \draw   (422.17,77.83) -- (417.17,82.83) -- (397.17,82.83) ; \draw   (417.17,82.83) -- (417.17,99.5) ;

		% Text Node
		\draw (123.33,213.67) node    {$x$};
		% Text Node
		\draw (255.5,212.17) node    {$y$};
		% Text Node
		\draw (172,79.67) node    {$z$};
		% Text Node
		\draw (404.67,185) node    {$\vec{r} -\vec{r}_{i}$};
		% Text Node
		\draw (308.67,172.83) node    {$\vec{r}$};
		% Text Node
		\draw (279,112.17) node    {$\vec{r}_{i}$};
		% Text Node
		\draw (410.17,63) node    {$dq$};
		% Text Node
		\draw (388.83,221.33) node    {$P( x ,y,z)$};
		% Text Node
		\draw (414.67,40.67) node    {$+$};
		% Text Node
		\draw (449.33,58.67) node    {$+$};
		% Text Node
		\draw (462.67,91.33) node    {$+$};
		% Text Node
		\draw (454.67,124.67) node    {$+$};
		% Text Node
		\draw (428.67,145.33) node    {$+$};
		% Text Node
		\draw (402.67,146.67) node    {$+$};
		% Text Node
		\draw (382,120.67) node    {$+$};
		% Text Node
		\draw (384,62.67) node    {$+$};
		% Text Node
		\draw (427.33,118) node    {$+$};
		% Text Node
		\draw (407.5,90.33) node    {$P_{i}$};
		% Text Node
		\draw (436.17,87.67) node    {$d\tau $};


		\end{tikzpicture}
	\end{figure}
	\FloatBarrier
	\item \emph{Potenziale generato da una superfie} $\sigma (x',y',z') $ \\
	Consideriamo ora una superficie infinitesima $dS$ e carica $dq$ in $P'(x',y',z')$. Sia $\vec{r}'$ variabile e $\vec{r}$ costante. Esprimiamo la carica $ dq = \sigma(x',y',z') dS $ e il potenziale infinitesimo $ dV = \frac{dq (\vec{r} -\vec{r}_i )}{4\pi \varepsilon_0 |\vec{r} -\vec{r}_i |^2} $ generato dalla distribuzione di carica nel punto $P$. Allora
	\begin{align*}
		V (P) &= \int_{\Sigma} \frac{\sigma  (x',y',z') (\vec{r} -\vec{r}_i )}{4\pi \varepsilon_0 |\vec{r} -\vec{r}_i |^2}dS
	\end{align*}
	\begin{figure}[htpb]
		\centering
		

		\tikzset{every picture/.style={line width=0.75pt}} %set default line width to 0.75pt        

		\begin{tikzpicture}[x=0.75pt,y=0.75pt,yscale=-1,xscale=1]
		%uncomment if require: \path (0,300); %set diagram left start at 0, and has height of 300

		%Straight Lines [id:da8132437964757075] 
		\draw    (173.5,185.67) -- (173.5,105.67) ;
		\draw [shift={(173.5,102.67)}, rotate = 450] [fill={rgb, 255:red, 0; green, 0; blue, 0 }  ][line width=0.08]  [draw opacity=0] (10.72,-5.15) -- (0,0) -- (10.72,5.15) -- (7.12,0) -- cycle    ;
		%Straight Lines [id:da3778404924799825] 
		\draw    (173.5,185.67) -- (233.93,222.12) ;
		\draw [shift={(236.5,223.67)}, rotate = 211.1] [fill={rgb, 255:red, 0; green, 0; blue, 0 }  ][line width=0.08]  [draw opacity=0] (10.72,-5.15) -- (0,0) -- (10.72,5.15) -- (7.12,0) -- cycle    ;
		%Straight Lines [id:da6935075526188754] 
		\draw    (173.5,185.67) -- (114.06,222.1) ;
		\draw [shift={(111.5,223.67)}, rotate = 328.5] [fill={rgb, 255:red, 0; green, 0; blue, 0 }  ][line width=0.08]  [draw opacity=0] (10.72,-5.15) -- (0,0) -- (10.72,5.15) -- (7.12,0) -- cycle    ;
		%Shape: Circle [id:dp9243675757036502] 
		\draw  [fill={rgb, 255:red, 0; green, 0; blue, 0 }  ,fill opacity=1 ] (358.17,224.42) .. controls (358.17,223.17) and (359.17,222.17) .. (360.42,222.17) .. controls (361.66,222.17) and (362.67,223.17) .. (362.67,224.42) .. controls (362.67,225.66) and (361.66,226.67) .. (360.42,226.67) .. controls (359.17,226.67) and (358.17,225.66) .. (358.17,224.42) -- cycle ;
		%Straight Lines [id:da37428712983736157] 
		\draw    (173.5,185.67) -- (377.51,112.02) ;
		\draw [shift={(380.33,111)}, rotate = 520.15] [fill={rgb, 255:red, 0; green, 0; blue, 0 }  ][line width=0.08]  [draw opacity=0] (10.72,-5.15) -- (0,0) -- (10.72,5.15) -- (7.12,0) -- cycle    ;
		%Straight Lines [id:da0777129682659603] 
		\draw    (173.5,185.67) -- (355.23,223.8) ;
		\draw [shift={(358.17,224.42)}, rotate = 191.85] [fill={rgb, 255:red, 0; green, 0; blue, 0 }  ][line width=0.08]  [draw opacity=0] (10.72,-5.15) -- (0,0) -- (10.72,5.15) -- (7.12,0) -- cycle    ;
		%Straight Lines [id:da4139198240649695] 
		\draw    (393,122.33) -- (361.35,219.31) ;
		\draw [shift={(360.42,222.17)}, rotate = 288.08] [fill={rgb, 255:red, 0; green, 0; blue, 0 }  ][line width=0.08]  [draw opacity=0] (10.72,-5.15) -- (0,0) -- (10.72,5.15) -- (7.12,0) -- cycle    ;
		%Flowchart: Data [id:dp9333451488015583] 
		\draw   (354.6,44) -- (484.33,44) -- (454.4,170.67) -- (324.67,170.67) -- cycle ;
		%Shape: Square [id:dp008818926916588365] 
		\draw   (381.17,95.5) -- (405.17,95.5) -- (405.17,119.5) -- (381.17,119.5) -- cycle ;

		% Text Node
		\draw (111.33,233.67) node    {$x$};
		% Text Node
		\draw (243.5,232.17) node    {$y$};
		% Text Node
		\draw (160,99.67) node    {$z$};
		% Text Node
		\draw (404,195.67) node    {$\vec{r} -\vec{r}_{i}$};
		% Text Node
		\draw (296.67,192.83) node    {$\vec{r}$};
		% Text Node
		\draw (267,132.17) node    {$\vec{r}_{i}$};
		% Text Node
		\draw (394.17,80.33) node    {$dq$};
		% Text Node
		\draw (376.83,241.33) node    {$P( x ,y,z)$};
		% Text Node
		\draw (402.67,60.67) node    {$+$};
		% Text Node
		\draw (437.33,78.67) node    {$+$};
		% Text Node
		\draw (450.67,111.33) node    {$+$};
		% Text Node
		\draw (442.67,144.67) node    {$+$};
		% Text Node
		\draw (456.67,60.67) node    {$+$};
		% Text Node
		\draw (360.67,144) node    {$+$};
		% Text Node
		\draw (362,88.67) node    {$+$};
		% Text Node
		\draw (412,148) node    {$+$};
		% Text Node
		\draw (393.17,107.5) node    {$P_{i}$};
		% Text Node
		\draw (421.5,107.67) node    {$dS$};


		\end{tikzpicture}
	\end{figure}
	\FloatBarrier
	\item \emph{Potenziale generato da una linea} $\lambda (x',y',z') $ \\
	Consideriamo ora un segmento infinitesimo $ dl   $ e carica $ dq $ in $ P'(x',y',z') $. Sia $ \vec{r}' $ variabile e $ \vec{r}  $ costante. Esprimiamo la carica $ dq = \lambda(x',y',z') dl $ e il potenziale infinitesimo $ dV = \frac{dq (\vec{r} -\vec{r}_i )}{4\pi \varepsilon_0 |\vec{r} -\vec{r}_i |^2} $ generato dalla distribuzione di carica nel punto $P$. Allora
	\begin{align*}
		V (P) &= \int_{\Lambda} \frac{\lambda  (x',y',z') (\vec{r} -\vec{r}_i )}{4\pi \varepsilon_0 |\vec{r} -\vec{r}_i |^2}dl
	\end{align*}
	\begin{figure}[htpb]
		\centering
		

		\tikzset{every picture/.style={line width=0.75pt}} %set default line width to 0.75pt        

		\begin{tikzpicture}[x=0.75pt,y=0.75pt,yscale=-1,xscale=1]
		%uncomment if require: \path (0,300); %set diagram left start at 0, and has height of 300

		%Straight Lines [id:da3248192246494668] 
		\draw    (168.5,188.67) -- (168.5,108.67) ;
		\draw [shift={(168.5,105.67)}, rotate = 450] [fill={rgb, 255:red, 0; green, 0; blue, 0 }  ][line width=0.08]  [draw opacity=0] (10.72,-5.15) -- (0,0) -- (10.72,5.15) -- (7.12,0) -- cycle    ;
		%Straight Lines [id:da20698176796558965] 
		\draw    (168.5,188.67) -- (228.93,225.12) ;
		\draw [shift={(231.5,226.67)}, rotate = 211.1] [fill={rgb, 255:red, 0; green, 0; blue, 0 }  ][line width=0.08]  [draw opacity=0] (10.72,-5.15) -- (0,0) -- (10.72,5.15) -- (7.12,0) -- cycle    ;
		%Straight Lines [id:da24095408360246595] 
		\draw    (168.5,188.67) -- (109.06,225.1) ;
		\draw [shift={(106.5,226.67)}, rotate = 328.5] [fill={rgb, 255:red, 0; green, 0; blue, 0 }  ][line width=0.08]  [draw opacity=0] (10.72,-5.15) -- (0,0) -- (10.72,5.15) -- (7.12,0) -- cycle    ;
		%Shape: Circle [id:dp8211498090043992] 
		\draw  [fill={rgb, 255:red, 0; green, 0; blue, 0 }  ,fill opacity=1 ] (353.17,227.42) .. controls (353.17,226.17) and (354.17,225.17) .. (355.42,225.17) .. controls (356.66,225.17) and (357.67,226.17) .. (357.67,227.42) .. controls (357.67,228.66) and (356.66,229.67) .. (355.42,229.67) .. controls (354.17,229.67) and (353.17,228.66) .. (353.17,227.42) -- cycle ;
		%Straight Lines [id:da4729626879808948] 
		\draw    (168.5,188.67) -- (395.15,113.28) ;
		\draw [shift={(398,112.33)}, rotate = 521.6] [fill={rgb, 255:red, 0; green, 0; blue, 0 }  ][line width=0.08]  [draw opacity=0] (10.72,-5.15) -- (0,0) -- (10.72,5.15) -- (7.12,0) -- cycle    ;
		%Straight Lines [id:da23097223788931975] 
		\draw    (168.5,188.67) -- (350.23,226.8) ;
		\draw [shift={(353.17,227.42)}, rotate = 191.85] [fill={rgb, 255:red, 0; green, 0; blue, 0 }  ][line width=0.08]  [draw opacity=0] (10.72,-5.15) -- (0,0) -- (10.72,5.15) -- (7.12,0) -- cycle    ;
		%Straight Lines [id:da9822371868780124] 
		\draw    (397.33,117) -- (356.5,222.37) ;
		\draw [shift={(355.42,225.17)}, rotate = 291.18] [fill={rgb, 255:red, 0; green, 0; blue, 0 }  ][line width=0.08]  [draw opacity=0] (10.72,-5.15) -- (0,0) -- (10.72,5.15) -- (7.12,0) -- cycle    ;
		%Curve Lines [id:da15270005504887885] 
		\draw    (291,63.33) .. controls (423.33,39) and (407.33,213) .. (530,163.67) ;
		%Shape: Rectangle [id:dp02403770120059212] 
		\draw   (397.4,97.73) -- (414.71,117.57) -- (408.94,122.61) -- (391.62,102.77) -- cycle ;

		% Text Node
		\draw (106.33,236.67) node    {$x$};
		% Text Node
		\draw (238.5,235.17) node    {$y$};
		% Text Node
		\draw (155,102.67) node    {$z$};
		% Text Node
		\draw (396.33,184) node    {$\vec{r} -\vec{r}_{i}$};
		% Text Node
		\draw (291.67,195.83) node    {$\vec{r}$};
		% Text Node
		\draw (284,133.17) node    {$\vec{r}_{i}$};
		% Text Node
		\draw (402.5,126.67) node    {$dq$};
		% Text Node
		\draw (371.83,244.33) node    {$P( x ,y,z)$};
		% Text Node
		\draw (305,49) node    {$+$};
		% Text Node
		\draw (380.33,67.67) node    {$+$};
		% Text Node
		\draw (456.33,149.67) node    {$+$};
		% Text Node
		\draw (486.33,161.67) node    {$+$};
		% Text Node
		\draw (344.33,51.67) node    {$+$};
		% Text Node
		\draw (519,155.67) node    {$+$};
		% Text Node
		\draw (424.17,103.17) node    {$P_{i}$};
		% Text Node
		\draw (405.17,86.67) node    {$dl$};
		% Text Node
		\draw (437.67,131.67) node    {$+$};


		\end{tikzpicture}
	\end{figure}
	\FloatBarrier
\end{itemize}
Consideriamo il caso di una carica che si sposta da $A$ a $B$ in un campo di forze conservative. L'energia meccanica totale sarà costante lungo la traiettoria. Avendo introdotto il potenziale, l'energia potenziale si può scrivere come il prodotto della carica $q$ per il potenziale generato da $Q$ in $P$. $ \Delta E = 0 \implies E(A)=E(B) $.
Le cariche elettriche se si muovono di moto accelerato emettono onde. Dunque il principio di conservazione dell'energia meccanica va preso con le pinze. Se la carica si muove lentamente la perdita di energia per irraggiamento è trascurabile. Se invece la particella accelera considerevolmente, è come se si muovesse in un fluido viscoso in cui viene rallentata perdendo energia sottoforma di onde elettromagnetiche. Nella meccanica relativistica questo fenomeno accade. Pertanto il principio di conservazione dell'energia meccanica ha dei limiti. Un fenomeno di questo tipo è osservabile anche nel mondo macroscopico: infatti le grandi masse quando accelerano emettono onde gravitazionali che portano via energia. Tuttavia è un energia molto più piccola rispetto a quella che porta via l'irraggiamento di onde elettromagnetiche.







































\section{Energia elettrostatica}

L'energia elettrostatica di una distribuzione di carica è il lavoro che qualche forza esterna ha dovuto spendere per creare tale distribuzione. Cominciamo con il considerare una prima carica puntiforme $q_1$ dello spazio. Poniamo di voler portare in un punto $B$ una certa carica $q_2$ lungo un qualunque percorso. Chiamiamo la distanza fra le cariche $r_{12} $. Mentre portiamo $q_2$ dall'infinito a $B$, le cariche si respingono. Avremo una forza di repulsione da vincere per avvicinarle. Dovremo opporre alla forza di repulsione una forza esterna che la bilanci esattamente punto per punto.
\begin{figure}[htpb]
	\centering
	

	\tikzset{every picture/.style={line width=0.75pt}} %set default line width to 0.75pt        

	\begin{tikzpicture}[x=0.75pt,y=0.75pt,yscale=-1,xscale=1]
	%uncomment if require: \path (0,300); %set diagram left start at 0, and has height of 300

	%Curve Lines [id:da2598890702736305] 
	\draw    (270,123) .. controls (372.5,37) and (410.5,213) .. (515.5,134) ;
	\draw [shift={(392.88,126.58)}, rotate = 33.35] [fill={rgb, 255:red, 0; green, 0; blue, 0 }  ][line width=0.08]  [draw opacity=0] (10.72,-5.15) -- (0,0) -- (10.72,5.15) -- (7.12,0) -- cycle    ;
	%Shape: Circle [id:dp5297886834493957] 
	\draw  [fill={rgb, 255:red, 0; green, 0; blue, 0 }  ,fill opacity=1 ] (218.14,80.15) .. controls (219.04,78.76) and (220.89,78.36) .. (222.29,79.25) .. controls (223.68,80.15) and (224.08,82.01) .. (223.18,83.4) .. controls (222.28,84.79) and (220.43,85.19) .. (219.03,84.3) .. controls (217.64,83.4) and (217.24,81.54) .. (218.14,80.15) -- cycle ;
	%Shape: Circle [id:dp5663783446802324] 
	\draw  [fill={rgb, 255:red, 0; green, 0; blue, 0 }  ,fill opacity=1 ] (371.98,114.12) .. controls (372.88,112.73) and (374.73,112.33) .. (376.13,113.23) .. controls (377.52,114.13) and (377.92,115.98) .. (377.02,117.38) .. controls (376.12,118.77) and (374.27,119.17) .. (372.87,118.27) .. controls (371.48,117.37) and (371.08,115.52) .. (371.98,114.12) -- cycle ;
	%Straight Lines [id:da4768868780179971] 
	\draw    (374.5,115.75) -- (368.83,63.98) ;
	\draw [shift={(368.5,61)}, rotate = 443.75] [fill={rgb, 255:red, 0; green, 0; blue, 0 }  ][line width=0.08]  [draw opacity=0] (10.72,-5.15) -- (0,0) -- (10.72,5.15) -- (7.12,0) -- cycle    ;
	%Shape: Circle [id:dp8854072211352453] 
	\draw  [fill={rgb, 255:red, 0; green, 0; blue, 0 }  ,fill opacity=1 ] (199.14,176.15) .. controls (200.04,174.76) and (201.89,174.36) .. (203.29,175.25) .. controls (204.68,176.15) and (205.08,178.01) .. (204.18,179.4) .. controls (203.28,180.79) and (201.43,181.19) .. (200.03,180.3) .. controls (198.64,179.4) and (198.24,177.54) .. (199.14,176.15) -- cycle ;
	%Straight Lines [id:da10607471226633058] 
	\draw    (380.17,167.52) -- (374.5,115.75) ;
	\draw [shift={(380.5,170.5)}, rotate = 263.75] [fill={rgb, 255:red, 0; green, 0; blue, 0 }  ][line width=0.08]  [draw opacity=0] (10.72,-5.15) -- (0,0) -- (10.72,5.15) -- (7.12,0) -- cycle    ;

	% Text Node
	\draw (206.33,79.42) node    {$q_{2}$};
	% Text Node
	\draw (389.33,102.92) node    {$q_{3}$};
	% Text Node
	\draw (209.83,193.92) node    {$B$};
	% Text Node
	\draw (539.33,118.42) node    {$A( \infty )$};
	% Text Node
	\draw (187.33,175.42) node    {$q_{1}$};
	% Text Node
	\draw (276.83,135.92) node    {$C$};
	% Text Node
	\draw (389.33,62.92) node    {$\vec{F}_{c}$};
	% Text Node
	\draw (401.33,174.92) node    {$\vec{F}_{est}$};


	\end{tikzpicture}
\end{figure}
\FloatBarrier
Cerchiamo di capire qual è il lavoro che la forza esterna deve compiere per portare $q_2$ in $B$. Per calcolarlo sappiamo che:
\begin{align*}
	\mathcal{L} = \int_A^B \vec{F}_{\text{esterna}} \cdot d\vec{l} &= - \int_A^B \vec{F}_{\text{Coulomb}} \cdot d\vec{l} \\
	&= - [ q_2 \underbrace{V_1(A)}_{=0} - q_2V_1(B) ] \\
	&= q_2 V_1(B) = q_2 \, \frac{q_1}{4\pi \varepsilon_0 r_{12}} = U_e
\end{align*}
Chiameremo questa quantità \textbf{energia elettrostatica della distribuzione di cariche}. Se metto insieme delle cariche immagazzino in esse dell'energia e quando queste vengono liberate essa viene restituita. La distribuzione di carica è come un accumulatore di energia.
Se consideriamo la possibilità di avvicinare al sistema una terza carica $q_3$, quello che succederà è che il lavoro delle forze esterne per portare un'altra carica da $A$ a $C$ sarà uguale a:
\[
	\mathcal{L}_{AC}^{F_{\text{est}}}  = q_3 V_1(C) + q_3V_2(C) = \frac{q_3q_1}{4\pi \varepsilon_0 r_{31}} + \frac{q_3q_2}{4\pi \varepsilon_0 r_{32}}
\]
Questo è un lavoro aggiuntivo che si somma a quello compiuto prima per avvicinare le cariche. L'energia elettrostatica sarà la somma del lavoro per avvicinare tutte e tre le cariche.
In generale si ha:
\[
	\boxed{U_e = \frac{1}{2} \sum_{i\neq j} \frac{q_iq_j}{4\pi \varepsilon_0 r_{ij}}}
\]
Siccome $r_{ij} = r_{ji}$, risulta evidente che ogni contributo compare due volte e pertanto il risultato è quello sopra.







































\section{Il campo come gradiente del potenziale}

Immaginiamo due punti $A$ e $B$ molto vicini fra loro, separati da uno spostamento infinitesimo $d\vec{l}$. Supponiamo che in tale regione ci sia un campo elettrico.
\begin{figure}[htpb]
	\centering
	

	\tikzset{every picture/.style={line width=0.75pt}} %set default line width to 0.75pt        

	\begin{tikzpicture}[x=0.75pt,y=0.75pt,yscale=-1,xscale=1]
	%uncomment if require: \path (0,300); %set diagram left start at 0, and has height of 300

	%Straight Lines [id:da3608378869469364] 
	\draw    (395.72,126.13) -- (223.66,195.77) ;
	\draw [shift={(398.5,125)}, rotate = 157.96] [fill={rgb, 255:red, 0; green, 0; blue, 0 }  ][line width=0.08]  [draw opacity=0] (10.72,-5.15) -- (0,0) -- (10.72,5.15) -- (7.12,0) -- cycle    ;
	%Shape: Circle [id:dp385616927332582] 
	\draw  [fill={rgb, 255:red, 0; green, 0; blue, 0 }  ,fill opacity=1 ] (220.66,195.77) .. controls (220.66,194.12) and (222,192.77) .. (223.66,192.77) .. controls (225.32,192.77) and (226.66,194.12) .. (226.66,195.77) .. controls (226.66,197.43) and (225.32,198.77) .. (223.66,198.77) .. controls (222,198.77) and (220.66,197.43) .. (220.66,195.77) -- cycle ;
	%Straight Lines [id:da7031984861555005] 
	\draw    (254.84,57.93) -- (223.66,195.77) ;
	\draw [shift={(255.5,55)}, rotate = 102.74] [fill={rgb, 255:red, 0; green, 0; blue, 0 }  ][line width=0.08]  [draw opacity=0] (10.72,-5.15) -- (0,0) -- (10.72,5.15) -- (7.12,0) -- cycle    ;
	%Shape: Circle [id:dp6419156930896248] 
	\draw  [fill={rgb, 255:red, 0; green, 0; blue, 0 }  ,fill opacity=1 ] (399.5,123) .. controls (399.5,121.34) and (400.84,120) .. (402.5,120) .. controls (404.16,120) and (405.5,121.34) .. (405.5,123) .. controls (405.5,124.66) and (404.16,126) .. (402.5,126) .. controls (400.84,126) and (399.5,124.66) .. (399.5,123) -- cycle ;

	% Text Node
	\draw (210.33,204.42) node    {$A$};
	% Text Node
	\draw (419.33,127.42) node    {$B$};
	% Text Node
	\draw (318.33,172.42) node    {$d\vec{l}$};
	% Text Node
	\draw (229.33,113.42) node    {$\vec{E}$};


	\end{tikzpicture}
\end{figure}
\FloatBarrier
Si ha che la variazione infinitesima di potenziale da $A$ a $B$ è uguale a:
\begin{gather*}
	V(A)-V(B) = \vec{E} \cdot d\vec{l} \\
	\boxed{-dV = \vec{E} \cdot d\vec{l}}
\end{gather*}
Possiamo anche scrivere
\[
	\left. \begin{array}{r}
	 	d\vec{l} =dx\vec{u}_x+dy\vec{u}_y+dz\vec{u}_z \\
		\vec{E} = E_x\vec{u}_x + E_y\vec{u}_y + E_z\vec{u}_z
	\end{array} \right\} \implies dV = - [E_xdx + E_ydy + E_zdz]
\]
D'altra parte, per il \textbf{teorema del differenziale totale} la variazione infinitesima $dV$ può essere vista come la somma di tre contributi di variazione nelle tre direzioni
\begin{gather*}
	\underbrace{dV=\frac{\partial V}{\partial x} dx + \frac{\partial V}{\partial y} dy + \frac{\partial V}{\partial z} dz}_{\text{teorema del differenziale totale}} = - [E_xdx + E_ydy + E_zdz] \\
	\Downarrow \\
	E_x = -\frac{\partial V}{\partial x} \qquad E_y = -\frac{\partial V}{\partial y} \qquad E_z = -\frac{\partial V}{\partial z} \\
	\boxed{\vec{E} = - \text{grad}V = - \vec{\nabla} V}
\end{gather*}
Il simbolo $ \text{grad}V $ indica il gradiente della funzione $V$.
Il gradiente di $V$ è quel vettore tale che il suo prodotto scalare per il vettore spostamento infinitesimo $ dl $, dà la variazione di $V$ in corrispondenza di quello spostamento.
Se abbiamo già calcolato in precedenza il potenziale $V$ in tutti i punti dello spazio, con una semplice operazione di derivazione parziale possiamo ottenere le componenti del campo elettrico.







































\section{Superficie equipotenziali}

Abbiamo visto che in presenza di un campo vettoriale possiamo utilizzare metodo di rappresentazione grafica delle linee di flusso. Anche l'andamento del potenziale è visualizzatile ricorrendo al metodo di \textbf{rappresentazione delle superficie di livello}.
Supponiamo di aver definito in campo scalare $f$. Chiamiamo superficie di livello una superficie dello spazio tridimensionale nei cui punti la funzione $f$ ha lo stesso valore:
\[
	f(x,y,z) = \text{costante} = f_0
\]
Nel caso in cui $f$ sia la funzione potenziale, le chiameremo superfici di livello equipotenziali. Al variare del valore della costante, si ha tutta una famiglia di superficie equipotenziali. È chiaro che queste non si intersecano: in un punto passa una ed una sola superficie equipotenziale, essendo il potenziale una funzione univoca.
Immaginiamo che ci sia in una regione dello spazio un campo elettrico, e poniamo che in sua corrispondenza ci sia una certa funzione potenziale. In particolare consideriamo una superficie equipotenziale e prendiamo un punto $A$ su tale superficie. Consideriamo lo spostamento infinitesimo che ci porti da un punto $A$ ad un $B$ della superficie.
\begin{figure}[htpb]
	\centering
	

	\tikzset{every picture/.style={line width=0.75pt}} %set default line width to 0.75pt        

	\begin{tikzpicture}[x=0.75pt,y=0.75pt,yscale=-1,xscale=1]
	%uncomment if require: \path (0,300); %set diagram left start at 0, and has height of 300

	%Shape: Parallelogram [id:dp65389199804943] 
	\draw   (203.5,153) -- (484.5,153) -- (457,187) -- (176,187) -- cycle ;
	%Straight Lines [id:da3668762040266511] 
	\draw    (397.5,171.77) -- (255.66,171.77) ;
	\draw [shift={(400.5,171.77)}, rotate = 180] [fill={rgb, 255:red, 0; green, 0; blue, 0 }  ][line width=0.08]  [draw opacity=0] (10.72,-5.15) -- (0,0) -- (10.72,5.15) -- (7.12,0) -- cycle    ;
	%Straight Lines [id:da939309199569287] 
	\draw    (255.66,73) -- (255.66,171.77) ;
	\draw [shift={(255.66,70)}, rotate = 90] [fill={rgb, 255:red, 0; green, 0; blue, 0 }  ][line width=0.08]  [draw opacity=0] (10.72,-5.15) -- (0,0) -- (10.72,5.15) -- (7.12,0) -- cycle    ;

	% Text Node
	\draw (235.33,169.42) node    {$A$};
	% Text Node
	\draw (415.33,169.42) node    {$B$};
	% Text Node
	\draw (271.33,96.42) node    {$\vec{E}$};
	% Text Node
	\draw (124.33,172.42) node    {$V=\text{costante}$};


	\end{tikzpicture}
\end{figure}
\FloatBarrier
\begin{align*}
	dV = \underbrace{V(B)-V(A)}_{=0} = \vec{\nabla} V\,d\vec{l} = -\vec{E} \cdot d\vec{l}
\end{align*}
Se ci spostiamo lungo la superficie equipotenziale, il potenziale non cambia e quindi il gradiente è ortogonale in ogni punto ad essa. Questa caratteristica si può generalizzare nel caso di una qualsiasi funzione. Il vettore gradiente di $f$ è sempre perpendicolare alle superfici $f_0$ di $f$ costante.
Nel caso di un campo elettrico generato da una carica, le superficie equipotenziali sono delle sfere centrate sulla carica stessa.
Se invece considero uno spostamento nella direzione del gradiente $d\vec{l}$ si ha:
\[
	dV = -\vec{E} \cdot d\vec{l} = |\vec{E}|\,|d\vec{l}|\cos \alpha
\]
Il caso di $ \alpha =0 $ è quello in cui otteniamo il massimo prodotto scalare. Ciò significa che se ci muoviamo nella direzione del gradiente di $f$, troviamo il massimo aumento di funzione $f$. Detto in altri termini, il vettore $ \vec{\nabla} f $ \emph{punta nella direzione di massima crescita della funzione}.

Concludiamo che il campo elettrostatico in ogni punto è perpendicolare alla superficie equipotenziale che passa per quel punto e il suo verso indica il verso di diminuzione del potenziale, visto il segno negativo. Dunque il campo elettrico è sempre diretto dal potenziale più alto a quello più basso. Tali superficie sono dunque ortogonali alle linee di forza.

Notiamo che se rappresentiamo le superfici equipotenziali con un certo passo fisso $ \Delta V $, le superficie equipotenziali si infittiscono nelle zone in cui il campo è maggiore; in un campo uniforme esse sono equispaziate.







































\section{II Equazione di Maxwell (solo in regime stazionario)}

Il concetto di conservatività introdotto per i campi di forza si può estendere a un qualunque campo vettoriale. Abbiamo tuttavia studiato che se un campo elettrico è conservativo la sua circuitazione lungo un qualsiasi percorso chiuso è sempre zero.
\[
	\oint_{\gamma} \vec{E} \cdot d\vec{l} = 0 \qquad \forall \gamma \quad \text{chiusa}
\]
Per il teorema di Stokes:
\[
	0 = \oint_{\gamma} \vec{E} \cdot d\vec{l} =\int_{\Sigma}(\text{rot}\vec{E} )\cdot \vec{n} dS \implies \boxed{\text{rot}\vec{E} = 0}
\]
L'equazione ottenuta è la \textbf{II Equazione di Maxwell}. Mentre la prima vale in qualsiasi regime, \emph{la seconda vale solo in campi elettrostatici}. È possibile estenderla per i campi variabili nel tempo, non avremo più $0$ ma un altro termine. L'annullamento del rotore di un campo vettoriale in un qualunque punto dello spazio è una proprietà molto particolare. Se ciò accade diciamo che è un \textbf{campo vettoriale irrotazionale}.
$ \text{rot}\vec{v} =0 $ ovunque equivale a dire che il campo vettoriale è conservativo. Questo perché se il rotore è zero per il teorema di Stokes anche la circuitazione lo sarà.

Abbiamo visto che il legame fra $\vec{E}$ e il potenziale $V$ è pari a $ \vec{E} = - \text{grad}V $.
Anche questo è un modo per esprimere la conservatività di un campo. Quando si può scrivere un'espressione di questo tipo, il campo è automaticamente conservativo.
Abbiamo visto che:
\begin{gather*}
	\text{rot}\vec{v} = \vec{u}_x\left( \frac{\partial v_z}{\partial y} -\frac{\partial v_y}{\partial z}  \right)  +
	\vec{u}_y\left( \frac{\partial v_x}{\partial z} -\frac{\partial v_z}{\partial x}  \right)  +
	\vec{u}_z\left( \frac{\partial v_y}{\partial x} -\frac{\partial v_x}{\partial y}  \right) \\
	\text{rot}\vec{E} = \text{rot}(-\text{grad}V) = - \text{rot}(\text{grad}V) \\
	\text{grad}V = \frac{\partial V}{\partial x} \vec{u}_x + \frac{\partial V}{\partial y} \vec{u}_y + \frac{\partial V}{\partial z} \vec{u}_z \\
	\Downarrow \\
	\text{rot}\vec{E} = \vec{u}_x \left( \frac{\partial^2 V}{\partial z \partial y} - \frac{\partial^2 V}{\partial y \partial z}\right) + \vec{u}_y \left( \frac{\partial^2 V}{\partial z \partial x} - \frac{\partial^2 V}{\partial x \partial z}\right) + \vec{u}_z \left( \frac{\partial^2 V}{\partial y \partial x} - \frac{\partial^2 V}{\partial x \partial y}\right)
\end{gather*}
I termini tra parentesi sono tutti nulli per la proprietà delle derivate seconde miste di essere indipendenti dall'ordine di derivazione (\textbf{teorema di Schwarz}).
Quindi anche così si trova che il rotore è nullo e che \emph{un campo conservativo è irrotazionale}.
\[
	\vec{\nabla} \times (\vec{\nabla} f)=(\vec{\nabla} \times \vec{\nabla} ) f = 0
\]
Se moltiplico un vettore per se stesso il risultato è $0$, perché il prodotto vettoriale fra i vettori contiene il seno di $0$. In altre parole, l'applicazione successiva delle due operazioni di gradiente e di rotore dà risultato nullo.

Il fatto che $ \text{rot}\vec{E} =0 $ implica che abbia una struttura particolare, ossia, imponendo le componenti del rotore uguali a zero:
\begin{gather*}
	\frac{\partial E_z}{\partial y} = \frac{\partial E_y}{\partial z} \;\;\;\;\;\;\;
	\frac{\partial E_x}{\partial z} = \frac{\partial E_z}{\partial x} \;\;\;\;\;\;\;
	\frac{\partial E_y}{\partial x} = \frac{\partial E_x}{\partial y}
\end{gather*}
Un'altra considerazione importante è la seguente: se un campo $\vec{v}$ ha delle linee chiuse, allora non può essere conservativo, perché la circuitazione lungo quelle linee chiuse non sarebbe nullo.
\begin{figure}[htpb]
	\centering
	

	\tikzset{every picture/.style={line width=0.75pt}} %set default line width to 0.75pt        

	\begin{tikzpicture}[x=0.75pt,y=0.75pt,yscale=-1,xscale=1]
	%uncomment if require: \path (0,300); %set diagram left start at 0, and has height of 300

	%Shape: Circle [id:dp19640274429346416] 
	\draw   (184,145.75) .. controls (184,101.71) and (219.71,66) .. (263.75,66) .. controls (307.79,66) and (343.5,101.71) .. (343.5,145.75) .. controls (343.5,189.79) and (307.79,225.5) .. (263.75,225.5) .. controls (219.71,225.5) and (184,189.79) .. (184,145.75) -- cycle ;
	%Straight Lines [id:da389688705422198] 
	\draw    (366.59,66) -- (263.75,66) ;
	\draw [shift={(369.59,66)}, rotate = 180] [fill={rgb, 255:red, 0; green, 0; blue, 0 }  ][line width=0.08]  [draw opacity=0] (10.72,-5.15) -- (0,0) -- (10.72,5.15) -- (7.12,0) -- cycle    ;
	%Straight Lines [id:da45319967518017834] 
	\draw    (263.75,225.5) -- (160.91,225.5) ;
	\draw [shift={(157.91,225.5)}, rotate = 360] [fill={rgb, 255:red, 0; green, 0; blue, 0 }  ][line width=0.08]  [draw opacity=0] (10.72,-5.15) -- (0,0) -- (10.72,5.15) -- (7.12,0) -- cycle    ;
	%Straight Lines [id:da4667229649898803] 
	\draw    (343.5,248.59) -- (343.5,145.75) ;
	\draw [shift={(343.5,251.59)}, rotate = 270] [fill={rgb, 255:red, 0; green, 0; blue, 0 }  ][line width=0.08]  [draw opacity=0] (10.72,-5.15) -- (0,0) -- (10.72,5.15) -- (7.12,0) -- cycle    ;
	%Straight Lines [id:da9470975002641668] 
	\draw    (184,145.75) -- (184,42.91) ;
	\draw [shift={(184,39.91)}, rotate = 450] [fill={rgb, 255:red, 0; green, 0; blue, 0 }  ][line width=0.08]  [draw opacity=0] (10.72,-5.15) -- (0,0) -- (10.72,5.15) -- (7.12,0) -- cycle    ;




	\end{tikzpicture}
\end{figure}
\FloatBarrier
Il significato dell'operatore rotore è il seguente: esso ci dice sono i vortici, ci indica il piano in cui si svolgono e, in base al segno, ci dice il verso in cui ruotano.







































\section{Equazioni di Poisson e di Laplace}

Essendo $\vec{E}$ conservativo possiamo esprimere il suo potenziale come $\vec{E} = -\vec{\nabla} V$. Inoltre, usando la I Equazione di Maxwell
\[
	\text{div}\vec{E} = \frac{\rho}{\varepsilon_0}\implies \vec{\nabla} \cdot (-\vec{\nabla} V)=\frac{\rho}{\varepsilon_0} \implies \nabla^2 V=-\frac{\rho}{\varepsilon_0}
\]
L'operatore $ \nabla^2 $ è noto come \textbf{operatore Laplaciano} o \textbf{operatore di Laplace}
\[
	\nabla^2 = \frac{\partial^2}{\partial x^2} + \frac{\partial^2}{\partial y^2} + \frac{\partial^2}{\partial z^2}
\]
Mentre l'ultima relazione è nota come \textbf{Equazione di Poisson}
\[
	\boxed{\nabla^2V = \frac{\partial^2 V}{\partial x^2} + \frac{\partial^2 V}{\partial y^2} + \frac{\partial^2 V}{\partial z^2} = - \frac{\rho(x,y,z)}{\varepsilon_0}}
\]
Se la densità di carica è nulla, ad esempio dentro un conduttore, l'equazione diventa nota come \textbf{Equazione di Laplace}:
\[
	\nabla^2 V=0
\]
L'integrazione di tale equazione con determinate condizioni al
contorno permette di determinare univocamente il potenziale $ V(x,y,z) $ e da questo il campo elettrico attraverso l'operazione di gradiente. In effetti in Analisi II si dimostra il cosiddetto \emph{teorema di unicità della soluzione dell'equazione di Poisson}, secondo il quale, se si impone al potenziale di annullarsi all'infinito insieme a tutte le sue derivate (e quindi è nullo all'infinito anche il campo) e si fissa una certa distribuzione di carica $ \rho (x,y,z)$ contenuta in una regione finita di spazio, la soluzione dell'equazione di Poisson è data da:
\[
	V(P)= \int_{\tau}\frac{\rho (x,y,z) d\tau}{4\pi \varepsilon_0 |\vec{r} -\vec{r'}|} = \int_{\tau} \frac{\rho (x,y,z) dx'dy'dz'}{4\pi \varepsilon_0 \sqrt{(x-x')^2 + (y-y')^2} +(z-z')^2}
\]
Se poniamo che il potenziale $V$ abbia un certo valore assegnato $V_s $ su $\Sigma$, condizione al contorno, allora si dimostra che la soluzione all'equazione di Poisson esiste ed è unica. Una volta che assegnamo le cariche dentro una certa superficie sigma, si dimostra che l'equazione ha soluzione unica.

Se facciamo tendere $\Sigma$ ad infinito, se fissiamo il valore del potenziale all'infinito ad un certo valore che scegliamo noi, se è diversa da zero in una regione limitata dello spazio e se imponiamo che il potenziale $V$ vada come $1/r$ quando ci allontaniamo verso l'infinito e che il modulo del campo elettrico decresce come $1/r^2$, allora anche in questo caso la soluzione dell'equazione esiste ed è unica.

In analisi matematica, quando un equazione soddisfa l'equazione di Laplace si dice \textbf{armonica}. Per le funzioni armoniche vale il teorema della media. Tale teorema dice che se prendiamo un punto $P$ e immaginiamo una sfera $ \Sigma  $ che lo circonda e la funzione è armonica, allora il valore che la funzione assume nel punto $O$ nel centro è uguale al valore medio di $V$ sulla sfera.
Se vale il teorema della media allora vuol dire che questa funzione non può assumere né minimi né massimi. In generale per le funzioni armoniche si può affermare che non hanno punti di minimo o massimo. Una volta stabilito che in una regione dello spazio il potenziale non ha minimi o massimi, nemmeno l'energia potenziale della carica ne ha. I punti di minimo di energia potenziale sono punti di equilibrio stabile. Ciò significa che in tale regione non ci sono punti di equilibrio né stabili né instabili. Se mettiamo la carica in un punto rimane ferma. Nella fisica delle particelle non è possibile usare solo campi elettrici per tenere ferme le cariche perché non si riesce mai a ottenere una zona di equilibrio. Per queste applicazioni si usano i campi magnetici. Per far sì che la materia rimanga stabile gli elettroni orbitano, questo perché le forze elettrostatiche non danno punti di equilibrio. La struttura della materia è stabile perché è una struttura dinamica.







































\section{Condizioni al contorno per il campo elettrostatico}

Consideriamo il caso generale di una superficie $\Sigma$ di separazione fra due regioni dello spazio che chiameremo $1$ e $2$. Si tratta di mezzi differenti, ad esempio la regione $2$ potrebbe essere occupata da un conduttore mentre la $1$ essere vuota. Assumiamo che sulla superficie sia presente anche della carica elettrica positiva. Poniamo di conoscere la densità di carica superficiale in ogni punto della essa. Consideriamo in particolare un punto $P$ e i campi elettrici presenti \emph{in prossimità} del punto $P$ dal lato $1$ e dal lato $2$.
\begin{figure}[htpb]
	\centering
	

	\tikzset{every picture/.style={line width=0.75pt}} %set default line width to 0.75pt        

	\begin{tikzpicture}[x=0.75pt,y=0.75pt,yscale=-1,xscale=1]
	%uncomment if require: \path (0,300); %set diagram left start at 0, and has height of 300

	%Straight Lines [id:da9167485757937464] 
	\draw    (154,165) -- (483.5,165) ;
	%Shape: Circle [id:dp7785704410574901] 
	\draw  [fill={rgb, 255:red, 0; green, 0; blue, 0 }  ,fill opacity=1 ] (296.99,165.11) .. controls (296.99,163.45) and (298.34,162.11) .. (299.99,162.11) .. controls (301.65,162.11) and (302.99,163.45) .. (302.99,165.11) .. controls (302.99,166.77) and (301.65,168.11) .. (299.99,168.11) .. controls (298.34,168.11) and (296.99,166.77) .. (296.99,165.11) -- cycle ;
	%Straight Lines [id:da49790538469185663] 
	\draw    (299.99,165.11) -- (340.44,74.74) ;
	\draw [shift={(341.67,72)}, rotate = 474.11] [fill={rgb, 255:red, 0; green, 0; blue, 0 }  ][line width=0.08]  [draw opacity=0] (10.72,-5.15) -- (0,0) -- (10.72,5.15) -- (7.12,0) -- cycle    ;
	%Straight Lines [id:da5337472774971028] 
	\draw    (234.5,230) -- (297.86,167.22) ;
	\draw [shift={(299.99,165.11)}, rotate = 495.26] [fill={rgb, 255:red, 0; green, 0; blue, 0 }  ][line width=0.08]  [draw opacity=0] (10.72,-5.15) -- (0,0) -- (10.72,5.15) -- (7.12,0) -- cycle    ;

	% Text Node
	\draw (170,156) node    {$+$};
	% Text Node
	\draw (134,135) node    {$1$};
	% Text Node
	\draw (134,196) node    {$2$};
	% Text Node
	\draw (190,156) node    {$+$};
	% Text Node
	\draw (210,156) node    {$+$};
	% Text Node
	\draw (230,156) node    {$+$};
	% Text Node
	\draw (250,156) node    {$+$};
	% Text Node
	\draw (270,156) node    {$+$};
	% Text Node
	\draw (290,156) node    {$+$};
	% Text Node
	\draw (310,156) node    {$+$};
	% Text Node
	\draw (330,156) node    {$+$};
	% Text Node
	\draw (350,156) node    {$+$};
	% Text Node
	\draw (370,156) node    {$+$};
	% Text Node
	\draw (390,156) node    {$+$};
	% Text Node
	\draw (410,156) node    {$+$};
	% Text Node
	\draw (430,156) node    {$+$};
	% Text Node
	\draw (450,156) node    {$+$};
	% Text Node
	\draw (470,156) node    {$+$};
	% Text Node
	\draw (312.33,181) node    {$P$};
	% Text Node
	\draw (354.67,78.33) node    {$\vec{E}_{1}$};
	% Text Node
	\draw (269,223) node    {$\vec{E}_{2}$};
	% Text Node
	\draw (498.67,165) node    {$\Sigma $};


	\end{tikzpicture}
\end{figure}
\FloatBarrier
Vorremmo trovare un legame fra questi due campi elettrici in funzione della densità di carica superficiale $\sigma$. Tali legami sono detti \textbf{condizioni al contorno del campo elettrostatico}. A questo fine, sfruttiamo il fatto Che il campo elettrostatico è conservativo e che gode del teorema di Gauss.

Sia $\Sigma$ la superficie di separazione tra le regioni $1$ e $2$. Siano $E_1$ e $E_2$ i campi valutati in prossimità della superficie.

In prossimità della superficie possiamo assumerla piatta anziché curva. Consideriamo un percorso $\gamma$ chiuso come in figura, supponiamo inoltre $dl\gg dh $, ossia $dl$ è un infinitesimo di ordine superiore rispetto a $dh$.
\begin{figure}[htpb]
	\centering
	

	\tikzset{every picture/.style={line width=0.75pt}} %set default line width to 0.75pt        

	\begin{tikzpicture}[x=0.75pt,y=0.75pt,yscale=-1,xscale=1]
	%uncomment if require: \path (0,300); %set diagram left start at 0, and has height of 300

	%Straight Lines [id:da07387856013103389] 
	\draw    (174,185) -- (503.5,185) ;
	%Straight Lines [id:da261458001336919] 
	\draw    (319.99,185.11) -- (360.44,94.74) ;
	\draw [shift={(361.67,92)}, rotate = 474.11] [fill={rgb, 255:red, 0; green, 0; blue, 0 }  ][line width=0.08]  [draw opacity=0] (10.72,-5.15) -- (0,0) -- (10.72,5.15) -- (7.12,0) -- cycle    ;
	%Straight Lines [id:da3092345576534641] 
	\draw    (240.67,169.33) -- (460.33,169.33) ;
	\draw [shift={(350.5,169.33)}, rotate = 180] [fill={rgb, 255:red, 0; green, 0; blue, 0 }  ][line width=0.08]  [draw opacity=0] (10.72,-5.15) -- (0,0) -- (10.72,5.15) -- (7.12,0) -- cycle    ;
	%Straight Lines [id:da2617769489963089] 
	\draw    (240.67,198) -- (460.33,198) ;
	\draw [shift={(350.5,198)}, rotate = 0] [fill={rgb, 255:red, 0; green, 0; blue, 0 }  ][line width=0.08]  [draw opacity=0] (10.72,-5.15) -- (0,0) -- (10.72,5.15) -- (7.12,0) -- cycle    ;
	%Straight Lines [id:da294234843216612] 
	\draw    (460.33,169.33) -- (460.33,198) ;
	%Straight Lines [id:da09868490023182708] 
	\draw    (240.67,169.33) -- (240.67,198) ;
	%Straight Lines [id:da6190495865255945] 
	\draw    (422.66,143.77) -- (462.67,143.77) ;
	\draw [shift={(465.67,143.77)}, rotate = 180] [fill={rgb, 255:red, 0; green, 0; blue, 0 }  ][line width=0.08]  [draw opacity=0] (10.72,-5.15) -- (0,0) -- (10.72,5.15) -- (7.12,0) -- cycle    ;
	%Straight Lines [id:da15065625656460369] 
	\draw    (254.5,250) -- (317.86,187.22) ;
	\draw [shift={(319.99,185.11)}, rotate = 495.26] [fill={rgb, 255:red, 0; green, 0; blue, 0 }  ][line width=0.08]  [draw opacity=0] (10.72,-5.15) -- (0,0) -- (10.72,5.15) -- (7.12,0) -- cycle    ;

	% Text Node
	\draw (190,176) node    {$+$};
	% Text Node
	\draw (154,155) node    {$1$};
	% Text Node
	\draw (154,216) node    {$2$};
	% Text Node
	\draw (210,176) node    {$+$};
	% Text Node
	\draw (230,176) node    {$+$};
	% Text Node
	\draw (250,176) node    {$+$};
	% Text Node
	\draw (270,176) node    {$+$};
	% Text Node
	\draw (290,176) node    {$+$};
	% Text Node
	\draw (310,176) node    {$+$};
	% Text Node
	\draw (330,176) node    {$+$};
	% Text Node
	\draw (350,176) node    {$+$};
	% Text Node
	\draw (370,176) node    {$+$};
	% Text Node
	\draw (390,176) node    {$+$};
	% Text Node
	\draw (410,176) node    {$+$};
	% Text Node
	\draw (430,176) node    {$+$};
	% Text Node
	\draw (450,176) node    {$+$};
	% Text Node
	\draw (470,176) node    {$+$};
	% Text Node
	\draw (490,176) node    {$+$};
	% Text Node
	\draw (374.67,98.33) node    {$\vec{E}_{1}$};
	% Text Node
	\draw (302,232) node    {$\vec{E}_{2}$};
	% Text Node
	\draw (518.67,185) node    {$\Sigma $};
	% Text Node
	\draw (362,151.67) node    {$d\vec{l}$};
	% Text Node
	\draw (444,129.67) node    {$\vec{u}_{t}$};
	% Text Node
	\draw (247.33,154.33) node    {$\gamma $};


	\end{tikzpicture}
\end{figure}
\FloatBarrier
Ricordiamo inoltre la conservatività del campo $\vec{E}$:
\[
	\oint_{\gamma} \vec{E} \cdot d\vec{l} = 0
\]
Allora,
\begin{gather*}
	\oint_{\gamma} \vec{E} \cdot d\vec{l} \sim \vec{E}_1 \cdot (dl\vec{u}_t)+\vec{E}_2\cdot (-dl\vec{u}_t)=0 \implies \\
	(\vec{E}_1\cdot \vec{u}_t-\vec{E}_2\cdot \vec{u}_t)dl=0
\end{gather*}
Ma chiamando
\begin{gather*}
	E_{t1}= \vec{E}_1\cdot \vec{u}_t \\
	E_{t2}= \vec{E}_2\cdot \vec{u}_t
\end{gather*}
Si ottiene la seguente \textbf{condizione al contorno di tangenza}:
\[
	\boxed{E_{t1}=E_{t2}}
\]
Per ricavare informazioni sulle altre componenti del campo elettrico utilizziamo il teorema di Gauss. Useremo una superficie chiusa gaussiana $\Sigma_g$ di forma cilindrica a cavallo della superficie di separazione $\Sigma$ fra i due mezzi $1$ e $2$. Le basi sono parallele alla superficie sigma. Assumeremo che l'altezza del cilindro sia molto più piccola rispetto al raggio delle basi in modo da poter trascurare il flusso sulla superficie laterale.
\begin{figure}[htpb]
	\centering
	

	\tikzset{every picture/.style={line width=0.75pt}} %set default line width to 0.75pt        

	\begin{tikzpicture}[x=0.75pt,y=0.75pt,yscale=-1,xscale=1]
	%uncomment if require: \path (0,300); %set diagram left start at 0, and has height of 300

	%Straight Lines [id:da7426419768038623] 
	\draw    (159.33,194.33) -- (488.83,194.33) ;
	%Straight Lines [id:da13934423773510995] 
	\draw    (325.33,194.44) -- (365.77,104.07) ;
	\draw [shift={(367,101.33)}, rotate = 474.11] [fill={rgb, 255:red, 0; green, 0; blue, 0 }  ][line width=0.08]  [draw opacity=0] (10.72,-5.15) -- (0,0) -- (10.72,5.15) -- (7.12,0) -- cycle    ;
	%Shape: Can [id:dp7867133226586538] 
	\draw   (447,166.5) -- (447,215.5) .. controls (447,221.3) and (397.15,226) .. (335.67,226) .. controls (274.18,226) and (224.33,221.3) .. (224.33,215.5) -- (224.33,166.5) .. controls (224.33,160.7) and (274.18,156) .. (335.67,156) .. controls (397.15,156) and (447,160.7) .. (447,166.5) .. controls (447,172.3) and (397.15,177) .. (335.67,177) .. controls (274.18,177) and (224.33,172.3) .. (224.33,166.5) ;
	%Straight Lines [id:da48378000565482004] 
	\draw    (206.33,215.5) -- (206.33,166.5) ;
	\draw [shift={(206.33,166.5)}, rotate = 450] [color={rgb, 255:red, 0; green, 0; blue, 0 }  ][line width=0.75]    (0,5.59) -- (0,-5.59)   ;
	\draw [shift={(206.33,215.5)}, rotate = 450] [color={rgb, 255:red, 0; green, 0; blue, 0 }  ][line width=0.75]    (0,5.59) -- (0,-5.59)   ;
	%Straight Lines [id:da8031343165266469] 
	\draw    (403.33,165.77) -- (403.33,120.33) ;
	\draw [shift={(403.33,117.33)}, rotate = 450] [fill={rgb, 255:red, 0; green, 0; blue, 0 }  ][line width=0.08]  [draw opacity=0] (10.72,-5.15) -- (0,0) -- (10.72,5.15) -- (7.12,0) -- cycle    ;
	%Straight Lines [id:da922014640142611] 
	\draw    (403.33,249.77) -- (403.33,224.33) ;
	\draw [shift={(403.33,252.77)}, rotate = 270] [fill={rgb, 255:red, 0; green, 0; blue, 0 }  ][line width=0.08]  [draw opacity=0] (10.72,-5.15) -- (0,0) -- (10.72,5.15) -- (7.12,0) -- cycle    ;
	%Straight Lines [id:da24191265700742415] 
	\draw  [dash pattern={on 0.84pt off 2.51pt}]  (403.33,224.33) -- (403.33,204.33) ;
	%Straight Lines [id:da8876126540739697] 
	\draw    (259.83,259.33) -- (323.2,196.55) ;
	\draw [shift={(325.33,194.44)}, rotate = 495.26] [fill={rgb, 255:red, 0; green, 0; blue, 0 }  ][line width=0.08]  [draw opacity=0] (10.72,-5.15) -- (0,0) -- (10.72,5.15) -- (7.12,0) -- cycle    ;

	% Text Node
	\draw (175.33,185.33) node    {$+$};
	% Text Node
	\draw (139.33,164.33) node    {$1$};
	% Text Node
	\draw (139.33,225.33) node    {$2$};
	% Text Node
	\draw (195.33,185.33) node    {$+$};
	% Text Node
	\draw (215.33,185.33) node    {$+$};
	% Text Node
	\draw (235.33,185.33) node    {$+$};
	% Text Node
	\draw (255.33,185.33) node    {$+$};
	% Text Node
	\draw (275.33,185.33) node    {$+$};
	% Text Node
	\draw (295.33,185.33) node    {$+$};
	% Text Node
	\draw (315.33,185.33) node    {$+$};
	% Text Node
	\draw (335.33,185.33) node    {$+$};
	% Text Node
	\draw (355.33,185.33) node    {$+$};
	% Text Node
	\draw (375.33,185.33) node    {$+$};
	% Text Node
	\draw (395.33,185.33) node    {$+$};
	% Text Node
	\draw (415.33,185.33) node    {$+$};
	% Text Node
	\draw (435.33,185.33) node    {$+$};
	% Text Node
	\draw (455.33,185.33) node    {$+$};
	% Text Node
	\draw (475.33,185.33) node    {$+$};
	% Text Node
	\draw (345,101.33) node    {$\vec{E}_{1}$};
	% Text Node
	\draw (293.33,261) node    {$\vec{E}_{2}$};
	% Text Node
	\draw (504,194.33) node    {$\Sigma $};
	% Text Node
	\draw (444,148.33) node    {$dS$};
	% Text Node
	\draw (205.33,225.67) node    {$dh$};
	% Text Node
	\draw (421.33,121.67) node    {$\vec{n}_{1}$};
	% Text Node
	\draw (421.33,247.33) node    {$\vec{n}_{2}$};
	% Text Node
	\draw (450.67,227.67) node    {$\Sigma _{g}$};


	\end{tikzpicture}
\end{figure}
\FloatBarrier
La carica interna a $\Sigma_g$ è semplicemente pari a:
\[
	Q_{\text{tot}}^{\text{int a } \Sigma} = \sigma \,dS
\]
Applicando il teorema di Gauss quindi si ha:
\[
	\Phi_{\Sigma_g}(\vec{E}) = \int_{\Sigma_g} \vec{E} \cdot \vec{n} \, dS = \frac{Q_{\text{tot}}^{\text{int a } \Sigma}}{\varepsilon_0} = \frac{\sigma \,dS}{\varepsilon_0}
\]
Siccome le basi sono di area infinitesima possiamo eliminare l'integrale approssimando il flusso semplicemente come:
\begin{align*}
	\Phi_{\Sigma_g} (\vec{E}) &\simeq \vec{E}_1\vec{n}_1dS + \vec{E}_2\vec{n}_2dS \\
	\Phi_{\Sigma_g} &= (\vec{E}_1 \vec{n} - \vec{E}_2\vec{n} ) dS = \frac{\sigma \, dS}{\varepsilon_0} \tag*{$\vec{n} = \vec{n}_1\quad \vec{n} = - \vec{n}_2  $} \\
	& \qquad \underbrace{\vec{E}_1\vec{n}}_{E_{n1}} - \underbrace{\vec{E}_2\vec{n}}_{E_{n2}} = \frac{\sigma}{\varepsilon_0}
\end{align*}
La componente normale del campo elettrico presenta una discontinuità quando cambio mezzo legata alla densità di carica. Se non ci sono carice in mezzo ci aspettiamo che tutte le componenti del campo elettrico infatti siano continue.
\[
	\boxed{E_{n1} - E_{n2} = \frac{\sigma}{\varepsilon_0}}
\]







































\section{Dipoli elettrici}

Chiamiamo \textbf{dipolo elettrico} una distribuzione di carica composta da una carica positiva e una carica negativa separate da una certa distanza rappresentata da un vettore d che punta sempre come in figura. 
\begin{figure}[htpb]
	\centering
	

	\tikzset{every picture/.style={line width=0.75pt}} %set default line width to 0.75pt        

	\begin{tikzpicture}[x=0.75pt,y=0.75pt,yscale=-1,xscale=1]
	%uncomment if require: \path (0,300); %set diagram left start at 0, and has height of 300

	%Straight Lines [id:da3549766989431007] 
	\draw    (208.33,171.44) -- (425.5,171.44) ;
	\draw [shift={(428.5,171.44)}, rotate = 180] [fill={rgb, 255:red, 0; green, 0; blue, 0 }  ][line width=0.08]  [draw opacity=0] (10.72,-5.15) -- (0,0) -- (10.72,5.15) -- (7.12,0) -- cycle    ;

	% Text Node
	\draw  [fill={rgb, 255:red, 222; green, 222; blue, 222 }  ,fill opacity=1 ]  (194.75, 170.75) circle [x radius= 13.6, y radius= 13.6]   ;
	\draw (194.75,170.75) node    {$-$};
	% Text Node
	\draw  [fill={rgb, 255:red, 222; green, 222; blue, 222 }  ,fill opacity=1 ]  (444, 170.75) circle [x radius= 13.9, y radius= 13.9]   ;
	\draw (444,170.75) node    {$+$};
	% Text Node
	\draw (195,195) node    {$-q$};
	% Text Node
	\draw (444,195) node    {$+q$};
	% Text Node
	\draw (324,189) node    {$\vec{d}$};


	\end{tikzpicture}
\end{figure}
\FloatBarrier
Le proprietà del dipolo possono essere riassunte da una quantità nota come \textbf{momento di dipolo}:
\[
	\vec{p} =q\,\vec{d}
\]

\paragraph{La materia e la polarizzazione.} Un atomo è costituito da un nucleo positivo attorno al quale orbita una nube di elettroni (carica negativa). Se applichiamo a un atomo un campo elettrico, il nucleo positivo vorrà andare nella sua stessa direzione, mentre l'intera nuvola negativa vorrà spostarsi verso sinistra. Il nucleo non sarà più corrispondente al baricentro nella nube negativa. Se immaginiamo di concentrare la nube nel baricentro, abbiamo due cariche separate da una certa distanza. Questo è un fenomeno che accade a tutti gli atomi, parziale separazione di carica. Tutta la materia reagisce ai campi elettrici in termine di polarizzazione.
\begin{figure}[htpb]
	\centering
	

	\tikzset{every picture/.style={line width=0.75pt}} %set default line width to 0.75pt        

	\begin{tikzpicture}[x=0.75pt,y=0.75pt,yscale=-1,xscale=1]
	%uncomment if require: \path (0,300); %set diagram left start at 0, and has height of 300

	%Shape: Circle [id:dp3548704480985345] 
	\draw   (172,160.75) .. controls (172,93.51) and (226.51,39) .. (293.75,39) .. controls (360.99,39) and (415.5,93.51) .. (415.5,160.75) .. controls (415.5,227.99) and (360.99,282.5) .. (293.75,282.5) .. controls (226.51,282.5) and (172,227.99) .. (172,160.75) -- cycle ;
	%Straight Lines [id:da6906697877581127] 
	\draw    (198.33,161.44) -- (330.5,161.44) ;
	\draw [shift={(333.5,161.44)}, rotate = 180] [fill={rgb, 255:red, 0; green, 0; blue, 0 }  ][line width=0.08]  [draw opacity=0] (10.72,-5.15) -- (0,0) -- (10.72,5.15) -- (7.12,0) -- cycle    ;

	% Text Node
	\draw  [fill={rgb, 255:red, 222; green, 222; blue, 222 }  ,fill opacity=1 ]  (368, 163.75) circle [x radius= 31.06, y radius= 31.06]   ;
	\draw (368,163.75) node  [font=\Huge]  {$+$};
	% Text Node
	\draw  [fill={rgb, 255:red, 222; green, 222; blue, 222 }  ,fill opacity=1 ]  (252.75, 83.75) circle [x radius= 13.6, y radius= 13.6]   ;
	\draw (252.75,83.75) node    {$-$};
	% Text Node
	\draw  [fill={rgb, 255:red, 222; green, 222; blue, 222 }  ,fill opacity=1 ]  (213.75, 127.75) circle [x radius= 13.6, y radius= 13.6]   ;
	\draw (213.75,127.75) node    {$-$};
	% Text Node
	\draw  [fill={rgb, 255:red, 222; green, 222; blue, 222 }  ,fill opacity=1 ]  (267.75, 118.75) circle [x radius= 13.6, y radius= 13.6]   ;
	\draw (267.75,118.75) node    {$-$};
	% Text Node
	\draw  [fill={rgb, 255:red, 222; green, 222; blue, 222 }  ,fill opacity=1 ]  (215.75, 185.75) circle [x radius= 13.6, y radius= 13.6]   ;
	\draw (215.75,185.75) node    {$-$};
	% Text Node
	\draw  [fill={rgb, 255:red, 222; green, 222; blue, 222 }  ,fill opacity=1 ]  (249.75, 232.75) circle [x radius= 13.6, y radius= 13.6]   ;
	\draw (249.75,232.75) node    {$-$};
	% Text Node
	\draw  [fill={rgb, 255:red, 222; green, 222; blue, 222 }  ,fill opacity=1 ]  (251.75, 186.75) circle [x radius= 13.6, y radius= 13.6]   ;
	\draw (251.75,186.75) node    {$-$};
	% Text Node
	\draw  [fill={rgb, 255:red, 222; green, 222; blue, 222 }  ,fill opacity=1 ]  (288.75, 246.75) circle [x radius= 13.6, y radius= 13.6]   ;
	\draw (288.75,246.75) node    {$-$};
	% Text Node
	\draw (303.34,146) node    {$\vec{p}$};


	\end{tikzpicture}
\end{figure}
\FloatBarrier

\paragraph{Potenziale elettrico generato da un bipolo.} Supponiamo di avere un bipolo fatto come in figura e di voler calcolare Il potenziale elettrostatico da esso generato in un punto $P$. Questo si può calcolare con il principio di sovrapposizione degli effetti, pur di conoscere la distanza del punto $P$ dalle cariche:
\[
	V(P)=\frac{q}{4\pi \varepsilon_0 r_1} - \frac{q}{4\pi \varepsilon_0 r_2} = \frac{q(r_2-r_1  )}{4\pi \varepsilon_0 r_1 r_2}
\]
Vogliamo tuttavia trovare un'approssimazione che ci dica come varia questo potenziale asintoticamente, allontanandoci molto.
\begin{figure}[htpb]
	\centering
	

	\tikzset{every picture/.style={line width=0.75pt}} %set default line width to 0.75pt        

	\begin{tikzpicture}[x=0.75pt,y=0.75pt,yscale=-1,xscale=1]
	%uncomment if require: \path (0,300); %set diagram left start at 0, and has height of 300

	%Straight Lines [id:da4699866327560356] 
	\draw    (232.33,235.44) -- (232.33,136.5) ;
	\draw [shift={(232.33,133.5)}, rotate = 450] [fill={rgb, 255:red, 0; green, 0; blue, 0 }  ][line width=0.08]  [draw opacity=0] (10.72,-5.15) -- (0,0) -- (10.72,5.15) -- (7.12,0) -- cycle    ;
	%Straight Lines [id:da36083572492704663] 
	\draw    (232.33,105.44) -- (232.33,74.5) ;
	\draw [shift={(232.33,71.5)}, rotate = 450] [fill={rgb, 255:red, 0; green, 0; blue, 0 }  ][line width=0.08]  [draw opacity=0] (10.72,-5.15) -- (0,0) -- (10.72,5.15) -- (7.12,0) -- cycle    ;
	%Shape: Circle [id:dp40738493154038635] 
	\draw  [fill={rgb, 255:red, 0; green, 0; blue, 0 }  ,fill opacity=1 ] (484.66,66.77) .. controls (484.66,65.12) and (486,63.77) .. (487.66,63.77) .. controls (489.32,63.77) and (490.66,65.12) .. (490.66,66.77) .. controls (490.66,68.43) and (489.32,69.77) .. (487.66,69.77) .. controls (486,69.77) and (484.66,68.43) .. (484.66,66.77) -- cycle ;
	%Straight Lines [id:da5192933348938815] 
	\draw    (247,116.5) -- (487.66,66.77) ;
	%Straight Lines [id:da6350346320756566] 
	\draw    (247,249.5) -- (487.66,66.77) ;
	%Straight Lines [id:da896914496459934] 
	\draw    (232.33,184.47) -- (487.66,66.77) ;
	%Straight Lines [id:da8701075468927895] 
	\draw    (247,116.5) -- (310.7,200.4) ;
	%Shape: Square [id:dp13013832557219307] 
	\draw   (296.72,198.51) -- (304.66,192.46) -- (310.7,200.4) -- (302.77,206.44) -- cycle ;
	%Shape: Arc [id:dp8640724180283808] 
	\draw  [draw opacity=0] (232.48,166.8) .. controls (239.49,166.86) and (245.53,171) .. (248.33,176.97) -- (232.33,184.47) -- cycle ; \draw   (232.48,166.8) .. controls (239.49,166.86) and (245.53,171) .. (248.33,176.97) ;
	%Shape: Arc [id:dp9712219715449886] 
	\draw  [draw opacity=0] (232.37,230) .. controls (242.02,230.08) and (250.26,236.07) .. (253.65,244.53) -- (232.18,253.11) -- cycle ; \draw   (232.37,230) .. controls (242.02,230.08) and (250.26,236.07) .. (253.65,244.53) ;
	%Straight Lines [id:da2281231996191373] 
	\draw    (487.93,218.7) -- (530.72,186.21) ;
	\draw [shift={(533.11,184.4)}, rotate = 502.79] [fill={rgb, 255:red, 0; green, 0; blue, 0 }  ][line width=0.08]  [draw opacity=0] (10.72,-5.15) -- (0,0) -- (10.72,5.15) -- (7.12,0) -- cycle    ;
	%Straight Lines [id:da5625694167837756] 
	\draw    (457.91,179.16) -- (487.93,218.7) ;
	\draw [shift={(456.1,176.77)}, rotate = 52.79] [fill={rgb, 255:red, 0; green, 0; blue, 0 }  ][line width=0.08]  [draw opacity=0] (10.72,-5.15) -- (0,0) -- (10.72,5.15) -- (7.12,0) -- cycle    ;
	%Shape: Square [id:dp9703624883708826] 
	\draw   (481.89,210.76) -- (489.83,204.72) -- (495.87,212.66) -- (487.93,218.7) -- cycle ;

	% Text Node
	\draw  [fill={rgb, 255:red, 222; green, 222; blue, 222 }  ,fill opacity=1 ]  (232.75, 249.75) circle [x radius= 13.6, y radius= 13.6]   ;
	\draw (232.75,249.75) node    {$-$};
	% Text Node
	\draw  [fill={rgb, 255:red, 222; green, 222; blue, 222 }  ,fill opacity=1 ]  (233, 117.75) circle [x radius= 13.9, y radius= 13.9]   ;
	\draw (233,117.75) node    {$+$};
	% Text Node
	\draw (203,248) node    {$-q$};
	% Text Node
	\draw (205,116) node    {$+q$};
	% Text Node
	\draw (221.6,207.6) node    {$\vec{p}$};
	% Text Node
	\draw (224,70) node    {$z$};
	% Text Node
	\draw (503,65) node    {$P$};
	% Text Node
	\draw (308,89) node    {$\vec{r}_{1}$};
	% Text Node
	\draw (389,167) node    {$\vec{r}_{2}$};
	% Text Node
	\draw (343,121) node    {$\vec{r}$};
	% Text Node
	\draw (248,160) node    {$\vartheta $};
	% Text Node
	\draw (249.6,224.4) node    {$\vartheta $};
	% Text Node
	\draw (220.4,181.2) node    {$M$};
	% Text Node
	\draw (318,209.2) node    {$H$};
	% Text Node
	\draw (543.4,196.6) node    {$\vec{u}_{r}$};
	% Text Node
	\draw (447,186.2) node    {$\vec{u}_{\vartheta }$};


	\end{tikzpicture}
\end{figure}
\FloatBarrier
Consideriamo il punto medio $M$ della distanza fra le cariche e supponiamo che $r = MP$ (rappresentato in figura) sia molto maggiore della separazione $d$ fra le due cariche: $ r \gg d $.
Sia $ \vartheta  $ l'angolo formato da $r$ e $d$. Se immaginiamo che il punto $P$ si trovi a distanza infinita, $r$ ed $ r_2  $ finiscono per essere paralleli ed anche l'angolo formato da $d$ ed $ r_2  $ sarà pari a $ \vartheta  $.
All'infinito possiamo dire che:
\[
	\left. \begin{array}{r}
	 	PH \simeq r_1  \\
		r_2 -r_1  \simeq + d\,\cos \vartheta \\
		r_1 r_2 \simeq r^2
	\end{array} \right\} V(P) \simeq \frac{\overbrace{qd}^p\cos \vartheta}{4\pi \varepsilon_0 r^2} = \frac{\overbrace{p\cos \vartheta}^{\vec{p} \cdot \vec{u}_r}}{4\pi \varepsilon_0 r^2} = \frac{\vec{p} \cdot \vec{u}_r}{4\pi \varepsilon_0 r^2}
\]
A grandi distanze dal dipolo quindi il potenziale da esso generato sarà:
\[
	\boxed{V(P) = \frac{\vec{p} \cdot \vec{u}_r}{4\pi \varepsilon_0 r^2}} \qquad  \propto \frac{1}{r^2}
\]
Notiamo come l'unica grandezza caratteristica del dipolo è il momento $ \vec{p}  $ e non $q$ e $d$ separatamente: ciò indica che da misure di potenziale possiamo ricavare informazioni su $\vec{p}$, ma non sulla costituzione del sistema.
Poniamoci in un sistema di coordinate in cui chiamiamo $ \vec{u}_r $ il versore diretto come $ r $ e $ \vec{u}_{\vartheta} $ quello perpendicolare a $ \vec{u}_r $. Si avrà che il campo elettrico generato dal dipolo sarà:
\begin{align*}
	\vec{E} &= - \vec{\nabla} V = -\frac{\partial V}{\partial r} \vec{u}_r - \frac{1}{r} \frac{\partial V}{\partial \vartheta} \vec{u}_{\vartheta} \tag*{regola della catena}\\
	&= \frac{2p\cos \vartheta}{4\pi \varepsilon_0 r^3} \vec{u}_r + \frac{p\sin \vartheta}{4\pi \varepsilon_0 r^3} \vec{u}_{\vartheta} \tag*{derivando} \\
	&= \frac{3p\cos \vartheta}{4\pi \varepsilon_0 r^3}\vec{u}_r - \frac{\vec{p}}{4\pi \varepsilon_0 r^3} \tag*{$\vec{p} =p\cos \vartheta \vec{u}_r - p \sin \vartheta \vec{u}_{\vartheta}$} \\
	&= \frac{1}{4\pi \varepsilon_0 r^3}[3(\vec{p} \cdot \vec{u}_r)\vec{u}_r-\vec{p}] \qquad \propto \frac{1}{r^3}
\end{align*}
\paragraph{Andamento asintotico e sviluppo in serie di multipoli.}Riassumiamo nella seguente tabella gli andamenti asintotici del potenziale e del campo elettrico a distanza infinita per generati rispettivamente da una carica puntiforme e da un bipolo.
\begin{table}[htpb]
	\centering
	\begin{tabular}{r|c|c}
		& carica puntiforme & dipolo \\
		\hline
		$V(r)$ & $ \propto 1/r $ & $ \propto 1/r^2  $ \\
		\hline
		$ |\vec{E} (r)| $ & $ \propto 1/r^2 $ & $ \propto 1/r^3  $
	\end{tabular}
\end{table}
Potremmo anche considerare un quadripolo, e in quel caso si avrebbe:
\[
	|\vec{E} (r)| \propto 1/r^4 \qquad V(r) \propto 1/r^3
\]
Quando dobbiamo calcolare il potenziale prodotto da una certa distribuzione di carica $ \rho  $ abbiamo due approcci.
\begin{itemize}
	\item Uno si basa su una formula precisa calcolata in precedenza utilizzando il principio di sovrapposizione degli effetti:
	\[
		V(P) = \int_{\tau}\frac{\rho (x',y',z') d\tau}{4\pi \varepsilon_0 |\vec{r} - \vec{r'}|}
	\]
	\item Un altro approccio si basa su una approssimazione in serie del potenziale. Abbiamo visto che il potenziale può andare come $1/r$, $1/r^2$ e via dicendo. Allora poniamo che $V(P)$ si possa vedere come una sommatoria:
	\[
		V(P) = \sum_{i=1}^n \frac{k_i}{r^i}
	\]
\end{itemize}
Il terimine $i=1$ corrisponde al caso in cui abbiamo una carica puntiforme. Facendo uno sviluppo in serie del potenziale, lo vediamo come somma dei potenziali generati dalla carica puntiforme, multiforme, quadriforme ecc... Se la distribuzione di carica ha una carica totale diversa da zero, il termine preponderante a grande distanza è quello della carica puntiforme. La funzione $1/r$ decresce più lentamente inoltre rispetto agli altri termini, che diventano insignificanti. Se la distribuzione di carica ha una carica totale pari a $0$, il termine di carica puntiforme scompare, ma abbiamo un dipolo. A questo punto il termine che diventa preponderante è quello del dipolo, $1/r^2$. Nello sviluppo in serie \emph{ci si ferma al primo termine più importante degli altri}, che decresce più lentamente. Questo sviluppo in serie prende il nome di \textbf{sviluppo in serie di multipoli}. Quindi non sempre l'andamento del campo elettrico è quello di una carica puntiforme.







































\section{Interazione fra un dipolo e un campo elettrico}

\subsection{Caso I: Campo elettrico uniforme}

I seguenti ragionamenti sono attuati basandosi sull'ipotesi che il bipolo di cui ci stiamo occupando sia \emph{rigido}, ossia che la distanza fra la carica positiva e quella negativa sia sempre costante. Consideriamo il caso di un dipolo immerso in un campo elettrico uniforme. Questo significa che il vettore $\vec{E}$ è costante in tutti i punti dello spazio.
\begin{figure}[htpb]
	\centering
	

	\tikzset{every picture/.style={line width=0.75pt}} %set default line width to 0.75pt        

	\begin{tikzpicture}[x=0.75pt,y=0.75pt,yscale=-0.8,xscale=0.8]
	%uncomment if require: \path (0,300); %set diagram left start at 0, and has height of 300

	%Straight Lines [id:da3173086254193631] 
	\draw    (245.33,223.44) -- (343.23,123.64) ;
	\draw [shift={(345.33,121.5)}, rotate = 494.45] [fill={rgb, 255:red, 0; green, 0; blue, 0 }  ][line width=0.08]  [draw opacity=0] (10.72,-5.15) -- (0,0) -- (10.72,5.15) -- (7.12,0) -- cycle    ;
	%Straight Lines [id:da3908064807839329] 
	\draw    (357.33,109.44) -- (433.5,109.44) ;
	\draw [shift={(436.5,109.44)}, rotate = 180] [fill={rgb, 255:red, 0; green, 0; blue, 0 }  ][line width=0.08]  [draw opacity=0] (10.72,-5.15) -- (0,0) -- (10.72,5.15) -- (7.12,0) -- cycle    ;
	%Straight Lines [id:da3972074201231366] 
	\draw    (160.33,234.44) -- (236.5,234.44) ;
	\draw [shift={(157.33,234.44)}, rotate = 0] [fill={rgb, 255:red, 0; green, 0; blue, 0 }  ][line width=0.08]  [draw opacity=0] (10.72,-5.15) -- (0,0) -- (10.72,5.15) -- (7.12,0) -- cycle    ;
	%Straight Lines [id:da19936120489500797] 
	\draw [color={rgb, 255:red, 155; green, 155; blue, 155 }  ,draw opacity=1 ]   (112.5,111.44) -- (319.5,111.44) ;
	\draw [shift={(216,111.44)}, rotate = 180] [fill={rgb, 255:red, 155; green, 155; blue, 155 }  ,fill opacity=1 ][line width=0.08]  [draw opacity=0] (10.72,-5.15) -- (0,0) -- (10.72,5.15) -- (7.12,0) -- cycle    ;
	%Straight Lines [id:da5411513941269097] 
	\draw [color={rgb, 255:red, 155; green, 155; blue, 155 }  ,draw opacity=1 ]   (84.5,139.44) -- (291.5,139.44) ;
	\draw [shift={(188,139.44)}, rotate = 180] [fill={rgb, 255:red, 155; green, 155; blue, 155 }  ,fill opacity=1 ][line width=0.08]  [draw opacity=0] (10.72,-5.15) -- (0,0) -- (10.72,5.15) -- (7.12,0) -- cycle    ;
	%Straight Lines [id:da47310294529851005] 
	\draw [color={rgb, 255:red, 155; green, 155; blue, 155 }  ,draw opacity=1 ]   (57.5,171.44) -- (264.5,171.44) ;
	\draw [shift={(161,171.44)}, rotate = 180] [fill={rgb, 255:red, 155; green, 155; blue, 155 }  ,fill opacity=1 ][line width=0.08]  [draw opacity=0] (10.72,-5.15) -- (0,0) -- (10.72,5.15) -- (7.12,0) -- cycle    ;
	%Straight Lines [id:da33896527031281387] 
	\draw [color={rgb, 255:red, 155; green, 155; blue, 155 }  ,draw opacity=1 ]   (37.5,197.44) -- (244.5,197.44) ;
	\draw [shift={(141,197.44)}, rotate = 180] [fill={rgb, 255:red, 155; green, 155; blue, 155 }  ,fill opacity=1 ][line width=0.08]  [draw opacity=0] (10.72,-5.15) -- (0,0) -- (10.72,5.15) -- (7.12,0) -- cycle    ;
	%Straight Lines [id:da6784200085632801] 
	\draw [color={rgb, 255:red, 155; green, 155; blue, 155 }  ,draw opacity=1 ]   (42.5,263.44) -- (464.5,263.44) ;
	\draw [shift={(253.5,263.44)}, rotate = 180] [fill={rgb, 255:red, 155; green, 155; blue, 155 }  ,fill opacity=1 ][line width=0.08]  [draw opacity=0] (10.72,-5.15) -- (0,0) -- (10.72,5.15) -- (7.12,0) -- cycle    ;
	%Straight Lines [id:da7175663365447602] 
	\draw [color={rgb, 255:red, 155; green, 155; blue, 155 }  ,draw opacity=1 ]   (284.5,234.44) -- (491.5,234.44) ;
	\draw [shift={(388,234.44)}, rotate = 180] [fill={rgb, 255:red, 155; green, 155; blue, 155 }  ,fill opacity=1 ][line width=0.08]  [draw opacity=0] (10.72,-5.15) -- (0,0) -- (10.72,5.15) -- (7.12,0) -- cycle    ;
	%Straight Lines [id:da5808798540953077] 
	\draw [color={rgb, 255:red, 155; green, 155; blue, 155 }  ,draw opacity=1 ]   (298.5,202.44) -- (505.5,202.44) ;
	\draw [shift={(402,202.44)}, rotate = 180] [fill={rgb, 255:red, 155; green, 155; blue, 155 }  ,fill opacity=1 ][line width=0.08]  [draw opacity=0] (10.72,-5.15) -- (0,0) -- (10.72,5.15) -- (7.12,0) -- cycle    ;
	%Straight Lines [id:da03928239258857458] 
	\draw [color={rgb, 255:red, 155; green, 155; blue, 155 }  ,draw opacity=1 ]   (330.5,174.44) -- (537.5,174.44) ;
	\draw [shift={(434,174.44)}, rotate = 180] [fill={rgb, 255:red, 155; green, 155; blue, 155 }  ,fill opacity=1 ][line width=0.08]  [draw opacity=0] (10.72,-5.15) -- (0,0) -- (10.72,5.15) -- (7.12,0) -- cycle    ;
	%Straight Lines [id:da7787315920071702] 
	\draw [color={rgb, 255:red, 155; green, 155; blue, 155 }  ,draw opacity=1 ]   (353.5,145.44) -- (560.5,145.44) ;
	\draw [shift={(457,145.44)}, rotate = 180] [fill={rgb, 255:red, 155; green, 155; blue, 155 }  ,fill opacity=1 ][line width=0.08]  [draw opacity=0] (10.72,-5.15) -- (0,0) -- (10.72,5.15) -- (7.12,0) -- cycle    ;
	%Straight Lines [id:da9988803146962861] 
	\draw [color={rgb, 255:red, 155; green, 155; blue, 155 }  ,draw opacity=1 ]   (149.5,62.44) -- (571.5,62.44) ;
	\draw [shift={(360.5,62.44)}, rotate = 180] [fill={rgb, 255:red, 155; green, 155; blue, 155 }  ,fill opacity=1 ][line width=0.08]  [draw opacity=0] (10.72,-5.15) -- (0,0) -- (10.72,5.15) -- (7.12,0) -- cycle    ;

	% Text Node
	\draw  [fill={rgb, 255:red, 222; green, 222; blue, 222 }  ,fill opacity=1 ]  (236.75, 233.75) circle [x radius= 13.6, y radius= 13.6]   ;
	\draw (236.75,233.75) node    {$-$};
	% Text Node
	\draw  [fill={rgb, 255:red, 222; green, 222; blue, 222 }  ,fill opacity=1 ]  (356, 110.75) circle [x radius= 13.9, y radius= 13.9]   ;
	\draw (356,110.75) node    {$+$};
	% Text Node
	\draw (286.6,164.6) node    {$\vec{d}$};
	% Text Node
	\draw (405.6,84.6) node    {$\vec{F}^{( +)}$};
	% Text Node
	\draw (200.6,216.6) node    {$\vec{F}^{( -)}$};
	% Text Node
	\draw (513.6,232.6) node    {$\vec{E}$};


	\end{tikzpicture}
\end{figure}
\FloatBarrier
\[
	\vec{F}^{(+)} = q\vec{E}_0 \qquad \vec{F}^{(-)} = -q\vec{E}_0 \qquad \implies \qquad \boxed{\vec{R} = 0}
\]
Possiamo dire che se immergiamo un dipolo in un campo elettrico uniforme, la risultante delle forze è nulla e quindi anche l'accelerazione del centro di massa. Ricordando che il momento di una forza rispetto a un polo $ o $ vale:
\[
	\vec{M}^{(o)} = \vec{r} \times \vec{R}
\]
Si dimostra che nel caso in cui un sistema di forze è costituito da una coppia di forze $-\vec{F}$ e $\vec{F}$ disposte in direzione tale per cui la risultante $\vec{R}$ sia nulla, come in questo caso, $\vec{M}$ \emph{non dipende dalla scelta del polo}. Scegliamo quindi a nostra preferenza come polo il punto $O$, coincidente con la carica negativa. In questo caso il contributo del momento delle forze in $-q$ è nullo e si ha:
\[
	\vec{M} =\vec{d} \times \vec{F}^{(+)} = \vec{d} \times q\vec{E}_0 = q\vec{d} \times \vec{E}_0 = \boxed{\vec{M} = \vec{p} \times \vec{E}}
\]
Troviamo così il \textbf{momento di dipolo elettrico}. Il verso del momento dice anche il senso di rotazione (oraria in questo caso) che le forze impongono. L'effetto delle forze è quello di portare il dipolo allineato con il campo elettrico, poiché dato che il momento delle forze si annulla proprio in queste condizioni.




















\subsection{Caso II: Campo elettrico non uniforme}

Consideriamo il caso in cui il campo elettrico non è uniforme.
In tali condizioni la risultante delle forze non è zero. Assumiamo che la distanza tra le due cariche sia piccola, sostanzialmente infinitesima.
\[
	\vec{d} =dx\vec{u}_x + dy\vec{u}_y+dz\vec{u}_z
\]
Il campo elettrico generato su $+q$ e $-q$ sarà rispettivamente:
\begin{align*}
	\vec{E}^{(-)} &= \vec{E} (x,y,z) \\
	\vec{E}^{(+)} &= \vec{E} (x+dx,y+dy,z+dz)
\end{align*}
\begin{figure}[htpb]
	\centering
	

	\tikzset{every picture/.style={line width=0.75pt}} %set default line width to 0.75pt        

	\begin{tikzpicture}[x=0.75pt,y=0.75pt,yscale=-1,xscale=1]
	%uncomment if require: \path (0,300); %set diagram left start at 0, and has height of 300

	%Straight Lines [id:da6757843094824549] 
	\draw    (261.33,198.44) -- (359.23,98.64) ;
	\draw [shift={(361.33,96.5)}, rotate = 494.45] [fill={rgb, 255:red, 0; green, 0; blue, 0 }  ][line width=0.08]  [draw opacity=0] (10.72,-5.15) -- (0,0) -- (10.72,5.15) -- (7.12,0) -- cycle    ;
	%Straight Lines [id:da6715254387985528] 
	\draw    (373.33,84.44) -- (455.55,69.54) ;
	\draw [shift={(458.5,69)}, rotate = 529.72] [fill={rgb, 255:red, 0; green, 0; blue, 0 }  ][line width=0.08]  [draw opacity=0] (10.72,-5.15) -- (0,0) -- (10.72,5.15) -- (7.12,0) -- cycle    ;
	%Straight Lines [id:da506216469916128] 
	\draw    (322.31,274.95) -- (252.5,209.44) ;
	\draw [shift={(324.5,277)}, rotate = 223.18] [fill={rgb, 255:red, 0; green, 0; blue, 0 }  ][line width=0.08]  [draw opacity=0] (10.72,-5.15) -- (0,0) -- (10.72,5.15) -- (7.12,0) -- cycle    ;

	% Text Node
	\draw  [fill={rgb, 255:red, 222; green, 222; blue, 222 }  ,fill opacity=1 ]  (252.75, 208.75) circle [x radius= 13.6, y radius= 13.6]   ;
	\draw (252.75,208.75) node    {$-$};
	% Text Node
	\draw  [fill={rgb, 255:red, 222; green, 222; blue, 222 }  ,fill opacity=1 ]  (372, 85.75) circle [x radius= 13.9, y radius= 13.9]   ;
	\draw (372,85.75) node    {$+$};
	% Text Node
	\draw (302.6,139.6) node    {$\vec{d}$};
	% Text Node
	\draw (421.6,59.6) node    {$\vec{E}^{( +)}$};
	% Text Node
	\draw (321.6,236.6) node    {$\vec{E}^{( -)}$};


	\end{tikzpicture}
\end{figure}
\FloatBarrier
Otteniamo allora che la risultante delle forze è pari a:
\[
	\vec{R} = q\left( \vec{E}^{(+)} - \vec{E}^{(-)}\right) = q\left[ \vec{E} (x+dx,y+dy,z+dz) - \vec{E} (x,y,z) \right]
\]
Procediamo con il ragionamento concentrandoci sulla componente $R_x$ della risultante, con la consapevolezza che esso è analogo per le altri componenti e che le conclusioni a cui arriveremo potranno essere estese a $R_y$ e $R_z$.
\begin{align*}
	R_x &=  q [ \underbrace{E_x  (x+dx,y+dy,z+dz)}_{E_x(x,y,z) +dE_x} - E_x  (x,y,z) ]  \tag*{teo. differenziale totale} \\
	&= q [E_x  (x,y,z) + dE_x - E_x  (x,y,z)] = q[dE_x ] \\
	&= q\left[ \frac{\partial E_x}{\partial x} dx + \frac{\partial E_x}{\partial y} dy + \frac{\partial E_x}{\partial z} dz \right] \\
	&= q \left[ \vec{\nabla} E_x \cdot \vec{d} \right] = \vec{p} \cdot \vec{\nabla} E_x
\end{align*}
Possiamo vedere $\vec{p} \cdot \vec{\nabla}$ come un unico operatore e scrivere:
\begin{gather*}
	R_x  = \left( \vec{p} \cdot \vec{\nabla}  \right)  E_x \qquad R_y  = \left( \vec{p} \cdot \vec{\nabla}  \right)  E_y \qquad R_z  = \left( \vec{p} \cdot \vec{\nabla}  \right)  E_z \\
	\boxed{\vec{R} = \left( \vec{p} \cdot \vec{\nabla}  \right)E_x\vec{u}_x + \left( \vec{p} \cdot \vec{\nabla}  \right) E_y\vec{u}_y + \left( \vec{p} \cdot \vec{\nabla}  \right) E_z\vec{u}_z = \left( \vec{p} \cdot \vec{\nabla}  \right)\vec{E}} \\
	\text{dove} \quad \vec{p} \cdot \vec{\nabla} = p_x \frac{\partial}{\partial x} + p_y \frac{\partial}{\partial y} + p_z \frac{\partial}{\partial z}
\end{gather*}
Se il campo elettrico fosse uniforme le derivate sarebbero tutte zero e ci ricondurremmo al caso precedente.







































\section{Energia elettrostatica di un dipolo immerso in un campo elettrico}

Immaginiamo un caso in cui il dipolo è immerso in un campo elettrostatico non uniforme, ma conservativo.
\begin{figure}[htpb]
	\centering
	

	\tikzset{every picture/.style={line width=0.75pt}} %set default line width to 0.75pt        

	\begin{tikzpicture}[x=0.75pt,y=0.75pt,yscale=-1,xscale=1]
	%uncomment if require: \path (0,300); %set diagram left start at 0, and has height of 300

	%Straight Lines [id:da8155935001916037] 
	\draw    (253.33,181.44) -- (351.23,81.64) ;
	\draw [shift={(353.33,79.5)}, rotate = 494.45] [fill={rgb, 255:red, 0; green, 0; blue, 0 }  ][line width=0.08]  [draw opacity=0] (10.72,-5.15) -- (0,0) -- (10.72,5.15) -- (7.12,0) -- cycle    ;

	% Text Node
	\draw  [fill={rgb, 255:red, 222; green, 222; blue, 222 }  ,fill opacity=1 ]  (244.75, 191.75) circle [x radius= 13.6, y radius= 13.6]   ;
	\draw (244.75,191.75) node    {$-$};
	% Text Node
	\draw  [fill={rgb, 255:red, 222; green, 222; blue, 222 }  ,fill opacity=1 ]  (364, 68.75) circle [x radius= 13.9, y radius= 13.9]   ;
	\draw (364,68.75) node    {$+$};
	% Text Node
	\draw (294.6,122.6) node    {$\vec{d}$};
	% Text Node
	\draw (475.6,66.6) node    {$P'( x+dx,y+dy,z+dz)$};
	% Text Node
	\draw (302.6,191.6) node    {$P( x,y,z)$};
	% Text Node
	\draw (214.6,192.6) node    {$-q$};
	% Text Node
	\draw (332.6,62.6) node    {$+q$};


	\end{tikzpicture}
\end{figure}
\FloatBarrier
L'energia elettrostatica di un dipolo immerso in un campo elettrico:
\begin{align*}
	U_e &= qV(P') - qV(P) = q \left[ V(P')-V(P) \right] = q\,dV \\
	&= q\left[ \vec{\nabla} V \cdot \vec{d}  \right] =\vec{\nabla} V \cdot q\vec{d} =\vec{\nabla} V \cdot \vec{p} = \vec{p} \cdot \vec{\nabla} V = - \vec{p} \cdot \vec{E}
\end{align*}
Possiamo usare ragionamenti energetici per calcolare la risultante delle forze $\vec{R}$. Il lavoro compiuto da essa per spostare il dipolo di un tratto infinitesimo $d\vec{r}$ sarà pari a:
\[
	d\mathcal{L} = \vec{R} \cdot d\vec{r}
\]
Se c'è un lavoro positivo avremo una diminuzione dell'energia potenziale. Essendo infatti il campo conservativo, l'energia per compierlo viene sottratta da quella elettrostatica.
\begin{align*}
	d\mathcal{L} &= -dU_e \tag*{conservatività del campo}\\
		&= - \vec{\nabla} U_e \cdot d\vec{r}  \tag*{teorema del differenziale totale}  \\
		&= - \vec{\nabla} (- \vec{p} \cdot \vec{E}) \cdot d\vec{r} \tag*{risultato ottenuto prima} \\
		&= \vec{\nabla} (\vec{p} \cdot \vec{E}) \cdot d\vec{r} \\
		d\mathcal{L} &= \vec{R} \cdot d\vec{r} \tag*{definizione di lavoro}
\end{align*}
\begin{gather*}
	\implies \boxed{\vec{R} = \vec{\nabla} (\vec{p} \cdot \vec{E})} \quad \text{per campi conservativi} \\
	\boxed{\vec{R} = (\vec{p} \cdot \vec{\nabla} )\vec{E}} \quad \text{per tutti i campi}
\end{gather*}
Questo risultato (il primo) significa che il dipolo, una volta allineato, tende a portarsi nei punti in cui il campo è più intenso se il momento $\vec{p}$ è concorde al campo, mentre ad allontanarsene se $\vec{p}$ è discorde al campo.







































\section{I conduttori}

Il conduttore è un qualunque materiale caratterizzato dal fatto che al suo interno sono verificate particolari condizioni per cui è possibile il moto di alcune delle cariche che lo costituiscono. Un esempio di conduzione è ciò che accade ad esempio in una soluzione salina all'interno di una cella elettrolitica, in cui si osserva un moto ambipolare di ioni (entrambe le polarità si muovono). Questo corso tuttavia si limita allo studio di conduttori allo stato solido, che generalmente sono di carattere metallico. Solitamente in un conduttore metallico ogni atomo mette in compartecipazione con l'intero sistema uno dei suoi elettroni o più. Questa situazione può essere schematizzata come un gas di elettroni libero di muoversi all'interno del materiale. Se esso è imperturbato, ossia se non agisce un campo elettrico esterno, le cariche si muovono, come in un gas, di moto puramente caotico. Quando parliamo di campo elettrico del materiale non ci occupiamo del campo elettrico microscopico. Se lo facessimo infatti basterebbe spostarsi di un solo angstrom per avere delle perturbazioni enormi su scala spaziale. Tali perturbazioni avvengono anche su scala temporale, perché l'elettrone sta orbitando quindi quando esso si sposta il campo elettrico cambia. Non possiamo quindi studiare le proprietà elettriche su scala realmente microscopica ma introdurremo le varie grandezze su scala media. Ad esempio considereremo il campo elettrico medio presente in un cubetto microscopico (in cui ho comunque un numero molto elevato di atomi). Questo vale per il campo elettrico così come per altre considerazioni. Lo stesso approccio infatti ci porta anche a dire che la velocità vettoriale media delle particelle è pari a zero. Questo perché in un conduttore non sottoposto a campi elettrici, per ogni elettrone che si muove verso il basso ce ne sarà uno che si muove verso l'alto. Quando parliamo di tutte queste quantità, stiamo parlando di grandezze mediate su scale sufficientemente grandi ma sufficientemente piccole dal punto di vista macroscopico.







































\section{Conduttori in equilibrio elettrostatico}

Ci occuperemo in particolare di conduttori in equilibrio elettrostatico. Questo significa che se consideriamo un oggetto composto da materiale conduttore e vi deponiamo una carica, assumeremo che essa si distribuisca su di esso, ma attenderemo un tempo sufficientemente lungo per permettere al sistema di raggiungere una situazione stazionaria. Le cariche così saranno all'equilibrio compensando perfettamente tutte le forze di interazione fra di loro.
Nei conduttori sono gli elettroni a muoversi. Se vogliamo quindi caricare positivamente un oggetto quello che dobbiamo fare è andare a sottrarre degli elettroni. Siccome utilizzare questa logica risulta più contorto, quello che facciamo è di immaginare che sia possibile caricare un conduttore sia aggiungendo cariche positive che negative. Si tratta di una convenzione che non cambia nulla dal punto di vista concettuale.

\paragraph{Proprietà dei conduttori.} Supponiamo di aver caricato un conduttore. Nel momento in cui viene raggiunto l'equilibrio elettrostatico, il campo elettrico all'interno di tale conduttore è nullo. Esso contiene al suo interno un'enorme quantità di cariche libere di muoversi. Se $\vec{E} $ fosse diverso da zero, agirebbe una forza coulombiana che farebbe spostare le cariche, esse nel nostro caso però sono ferme. Questa è la proprietà fondamentale da cui deriveranno tutte le altre proprietà.
Ora consideriamo un punto $P$ qualsiasi all'interno del nostro conduttore. Ivi il campo elettrico, per quanto appena detto, è pari a zero e allo stesso tempo la prima equazione di Maxwell deve essere verificata:
\[
	\text{div}\vec{E} (P)=\frac{\rho (P)}{\varepsilon_0} = 0 \implies \rho (P)=0
\]
Troviamo che la densità di carica in tale punto deve essere nulla. Questo potrebbe sembrare un paradosso. Tuttavia tale densità deve essere interpretata in modo fisico come densità di carica netta. Questo significa che se consideriamo un cubettino piccolissimo, di dimensioni infinitesime, la somma delle cariche elettriche in quel punto è zero, ma non vuol dire che queste non siano presenti. Il numero di cariche positive è pari al numero di cariche negative: in ogni punto del conduttore c'è un bilanciamento perfetto fra $+$ e $-$. Se aggiungiamo della carica ad un conduttore essa non può andare a finire all'interno, l'unico punti in cui si può distribuire è la superficie. A questo punto consideriamo un percorso $ \gamma  $ che congiunge due punti all'interno del conduttore $P$ e $P'$. Possiamo osservare che, ricordando il fatto che $\vec{E} $ è nullo all'equilibrio:
\[
	V(P)-V(P') = \int_{\gamma}\vec{E} \cdot d\vec{l} =0 \implies V(P') = V(P)
\]
\begin{figure}[htpb]
	\centering
	

	% Pattern Info
	 
	\tikzset{
	pattern size/.store in=\mcSize, 
	pattern size = 5pt,
	pattern thickness/.store in=\mcThickness, 
	pattern thickness = 0.3pt,
	pattern radius/.store in=\mcRadius, 
	pattern radius = 1pt}
	\makeatletter
	\pgfutil@ifundefined{pgf@pattern@name@_axll6w49l}{
	\pgfdeclarepatternformonly[\mcThickness,\mcSize]{_axll6w49l}
	{\pgfqpoint{0pt}{0pt}}
	{\pgfpoint{\mcSize+\mcThickness}{\mcSize+\mcThickness}}
	{\pgfpoint{\mcSize}{\mcSize}}
	{
	\pgfsetcolor{\tikz@pattern@color}
	\pgfsetlinewidth{\mcThickness}
	\pgfpathmoveto{\pgfqpoint{0pt}{0pt}}
	\pgfpathlineto{\pgfpoint{\mcSize+\mcThickness}{\mcSize+\mcThickness}}
	\pgfusepath{stroke}
	}}
	\makeatother
	\tikzset{every picture/.style={line width=0.75pt}} %set default line width to 0.75pt        

	\begin{tikzpicture}[x=0.75pt,y=0.75pt,yscale=-1,xscale=1]
	%uncomment if require: \path (0,300); %set diagram left start at 0, and has height of 300

	%Shape: Circle [id:dp18105265613022237] 
	\draw  [pattern=_axll6w49l,pattern size=6pt,pattern thickness=0.75pt,pattern radius=0pt, pattern color={rgb, 255:red, 222; green, 222; blue, 222}] (192,145.25) .. controls (192,90.44) and (236.44,46) .. (291.25,46) .. controls (346.06,46) and (390.5,90.44) .. (390.5,145.25) .. controls (390.5,200.06) and (346.06,244.5) .. (291.25,244.5) .. controls (236.44,244.5) and (192,200.06) .. (192,145.25) -- cycle ;
	%Curve Lines [id:da6091614584116016] 
	\draw    (247,141) .. controls (268.5,100) and (322.5,199) .. (347,141) ;
	%Shape: Circle [id:dp1808319466488788] 
	\draw  [fill={rgb, 255:red, 0; green, 0; blue, 0 }  ,fill opacity=1 ] (245.14,141) .. controls (245.14,139.97) and (245.97,139.14) .. (247,139.14) .. controls (248.03,139.14) and (248.86,139.97) .. (248.86,141) .. controls (248.86,142.03) and (248.03,142.86) .. (247,142.86) .. controls (245.97,142.86) and (245.14,142.03) .. (245.14,141) -- cycle ;
	%Shape: Circle [id:dp4526096831203399] 
	\draw  [fill={rgb, 255:red, 0; green, 0; blue, 0 }  ,fill opacity=1 ] (345.14,141) .. controls (345.14,139.97) and (345.97,139.14) .. (347,139.14) .. controls (348.03,139.14) and (348.86,139.97) .. (348.86,141) .. controls (348.86,142.03) and (348.03,142.86) .. (347,142.86) .. controls (345.97,142.86) and (345.14,142.03) .. (345.14,141) -- cycle ;

	% Text Node
	\draw (291.25,36) node    {$+$};
	% Text Node
	\draw (291.25,256) node    {$+$};
	% Text Node
	\draw (401.25,146) node    {$+$};
	% Text Node
	\draw (181.25,146) node    {$+$};
	% Text Node
	\draw (369.03,68.22) node    {$+$};
	% Text Node
	\draw (213.47,223.78) node    {$+$};
	% Text Node
	\draw (369.03,223.78) node    {$+$};
	% Text Node
	\draw (213.47,68.22) node    {$+$};
	% Text Node
	\draw (334.23,44.74) node    {$+$};
	% Text Node
	\draw (248.27,247.26) node    {$+$};
	% Text Node
	\draw (392.51,188.98) node    {$+$};
	% Text Node
	\draw (189.99,103.02) node    {$+$};
	% Text Node
	\draw (393.24,104.79) node    {$+$};
	% Text Node
	\draw (189.26,187.21) node    {$+$};
	% Text Node
	\draw (332.46,247.99) node    {$+$};
	% Text Node
	\draw (250.04,44.01) node    {$+$};
	% Text Node
	\draw (238.57,150.73) node    {$P$};
	% Text Node
	\draw (359.42,134.73) node    {$P'$};
	% Text Node
	\draw (305.99,135.31) node    {$\gamma $};


	\end{tikzpicture}
\end{figure}
\FloatBarrier
Ciò significa che \textbf{il conduttore è equipotenziale}, tutti i punti al suo interno lo sono. Il risultato varrebbe anche se il punto $P'$ stesse sulla superficie. Se chiamiamo $ \Sigma  $ una superficie al suo interno, sarà sempre tale.







































\section{Il teorema di Coulomb}

Applichiamo ora le condizioni al contorno anche al conduttore, per legare i valori che assume all'interno con quelli che ha all'esterno. Chiamiamo $1$ la regione del vuoto che circonda il conduttore e $2$ quella del conduttore stesso. Come abbiamo visto in precedenza, possiamo legare fra di loro i valori dei campi elettrici in prossimità di un punto sulla superficie di separazione da un lato e dall'altro. Chiamiamoli rispettivamente $\vec{E}_1$ ed $\vec{E}_2$.
\begin{align*}
	E_{1t} &= \vec{E}_1\cdot \vec{u}_t \\
	E_{2t} &= \vec{E}_2\cdot \vec{u}_t = 0\\
\end{align*}
Ci aspettiamo che il campo elettrico $\vec{E}_1$ sulla superficie sia perpendicolare ad essa. Se esso avesse una componente tangente infatti le cariche presenti sulla superficie comincerebbero a circolare. Inoltre tale superficie è equipotenziale, ed essendo $ \vec{E} = -\text{grad}V $, il campo elettrico deve essere perpendicolare ad essa. Introduciamo poi una normale $n$ alla superficie e le componenti normali:
\begin{align*}
	E_{1n} &= \vec{E}_1\cdot \vec{n} \\
	E_{2n} &= \vec{E}_2\cdot \vec{n} = 0
\end{align*}
Applicando la condizione al contorno otteniamo che:
\[
	E_{1n} - \underbrace{E_{2n}}_{=0} = \frac{\sigma}{\varepsilon_0}  \implies \boxed{\vec{E}_1 = \frac{\sigma}{\varepsilon_0}\vec{n}}
\]
Questo risultato prende il nome di \textbf{teorema di Coulomb}.
\begin{figure}[htpb]
	\centering
	

	% Pattern Info
	 
	\tikzset{
	pattern size/.store in=\mcSize, 
	pattern size = 5pt,
	pattern thickness/.store in=\mcThickness, 
	pattern thickness = 0.3pt,
	pattern radius/.store in=\mcRadius, 
	pattern radius = 1pt}
	\makeatletter
	\pgfutil@ifundefined{pgf@pattern@name@_mrblqmza0}{
	\pgfdeclarepatternformonly[\mcThickness,\mcSize]{_mrblqmza0}
	{\pgfqpoint{0pt}{0pt}}
	{\pgfpoint{\mcSize+\mcThickness}{\mcSize+\mcThickness}}
	{\pgfpoint{\mcSize}{\mcSize}}
	{
	\pgfsetcolor{\tikz@pattern@color}
	\pgfsetlinewidth{\mcThickness}
	\pgfpathmoveto{\pgfqpoint{0pt}{0pt}}
	\pgfpathlineto{\pgfpoint{\mcSize+\mcThickness}{\mcSize+\mcThickness}}
	\pgfusepath{stroke}
	}}
	\makeatother
	\tikzset{every picture/.style={line width=0.75pt}} %set default line width to 0.75pt        

	\begin{tikzpicture}[x=0.75pt,y=0.75pt,yscale=-1,xscale=1]
	%uncomment if require: \path (0,300); %set diagram left start at 0, and has height of 300

	%Shape: Circle [id:dp3658297706442104] 
	\draw  [pattern=_mrblqmza0,pattern size=6pt,pattern thickness=0.75pt,pattern radius=0pt, pattern color={rgb, 255:red, 222; green, 222; blue, 222}] (212,165.25) .. controls (212,110.44) and (256.44,66) .. (311.25,66) .. controls (366.06,66) and (410.5,110.44) .. (410.5,165.25) .. controls (410.5,220.06) and (366.06,264.5) .. (311.25,264.5) .. controls (256.44,264.5) and (212,220.06) .. (212,165.25) -- cycle ;
	%Straight Lines [id:da9128998526785432] 
	\draw    (396,113.67) -- (341.97,30.52) ;
	\draw [shift={(340.33,28)}, rotate = 416.98] [fill={rgb, 255:red, 0; green, 0; blue, 0 }  ][line width=0.08]  [draw opacity=0] (10.72,-5.15) -- (0,0) -- (10.72,5.15) -- (7.12,0) -- cycle    ;
	%Straight Lines [id:da35503145261724756] 
	\draw    (396,113.67) -- (479.15,59.63) ;
	\draw [shift={(481.67,58)}, rotate = 506.98] [fill={rgb, 255:red, 0; green, 0; blue, 0 }  ][line width=0.08]  [draw opacity=0] (10.72,-5.15) -- (0,0) -- (10.72,5.15) -- (7.12,0) -- cycle    ;
	%Straight Lines [id:da4195765778648737] 
	\draw    (394,110.67) -- (437.32,82.52) ;
	\draw [shift={(439.83,80.88)}, rotate = 506.98] [fill={rgb, 255:red, 0; green, 0; blue, 0 }  ][line width=0.08]  [draw opacity=0] (10.72,-5.15) -- (0,0) -- (10.72,5.15) -- (7.12,0) -- cycle    ;

	% Text Node
	\draw (311.25,56) node    {$+$};
	% Text Node
	\draw (311.25,276) node    {$+$};
	% Text Node
	\draw (421.25,166) node    {$+$};
	% Text Node
	\draw (201.25,166) node    {$+$};
	% Text Node
	\draw (389.03,88.22) node    {$+$};
	% Text Node
	\draw (233.47,243.78) node    {$+$};
	% Text Node
	\draw (389.03,243.78) node    {$+$};
	% Text Node
	\draw (233.47,88.22) node    {$+$};
	% Text Node
	\draw (354.23,64.74) node    {$+$};
	% Text Node
	\draw (268.27,267.26) node    {$+$};
	% Text Node
	\draw (412.51,208.98) node    {$+$};
	% Text Node
	\draw (209.99,123.02) node    {$+$};
	% Text Node
	\draw (413.24,124.79) node    {$+$};
	% Text Node
	\draw (209.26,207.21) node    {$+$};
	% Text Node
	\draw (352.46,267.99) node    {$+$};
	% Text Node
	\draw (270.04,64.01) node    {$+$};
	% Text Node
	\draw (365.56,37.41) node    {$\vec{u}_{t}$};
	% Text Node
	\draw (495.56,66.08) node    {$\vec{u}_{n}$};
	% Text Node
	\draw (413.56,75.41) node    {$\vec{E}_{1}$};
	% Text Node
	\draw (311.25,165.25) node    {$\vec{E}_{2} =0$};


	\end{tikzpicture}
\end{figure}
\FloatBarrier
Una volta notato il fatto che la superficie esterna è equipotenziale, c'è un altra condizione che permette di dimostrare subito che il potenziale è costante anche dentro il conduttore. Infatti, se consideriamo ancora una volta il nostro conduttore carico, possiamo ricorrere all'equazione di Poisson:
\[
	\nabla^2 V = -\frac{\rho}{\varepsilon_0} = 0 \implies \nabla^2 V = 0\quad \text{sul conduttore}
\]
Ottenendo così l'equazione omogenea associata, che abbiamo visto essere l'equazione di Laplace. A questo punto ricordiamo che il teorema della media afferma che una funzione che soddisfa l'equazione di Laplace non ha né punti di minimo né di massimo. Ma siccome stiamo già vincolando il potenziale ad essere costante sulla superficie, deve essere costante anche in tutti gli altri punti del conduttore. La funzione potenziale è un equazione armonica, ecco perché il conduttore è equipotenziale non solo sulla superficie ma anche all'interno.







































\section{Il potere delle punte}

Immaginiamo di considerare due conduttori sferici: uno di raggio $R_1 $ e uno un po' più piccolo di raggio $R_2<R_1$. Poniamo di collegarli con un filo conduttore e assumiamo che tale filo abbia una quantità di carica trascurabile.

In questo modo si viene a costituire un unico corpo conduttore e in equilibrio vale ovunque la condizione $\vec{E}=0 $, $V=\text{costante}$: i conduttori a contatto hanno lo stesso potenziale. Inoltre poniamo che i conduttori siano molto lontani fra di loro, di modo tale che non si influenzino l'un con l'altro tramite i campi elettrici: sono praticamente isolati. Siccome tutto il sistema è conduttore, tutti i conduttori sono allo stesso potenziale. Chiamiamo $V_1$ il potenziale del conduttore più grande e $V_2$ quello di quello più piccolo.
\begin{figure}[htpb]
	\centering
	

	% Pattern Info
	 
	\tikzset{
	pattern size/.store in=\mcSize, 
	pattern size = 5pt,
	pattern thickness/.store in=\mcThickness, 
	pattern thickness = 0.3pt,
	pattern radius/.store in=\mcRadius, 
	pattern radius = 1pt}
	\makeatletter
	\pgfutil@ifundefined{pgf@pattern@name@_zaytzot61}{
	\pgfdeclarepatternformonly[\mcThickness,\mcSize]{_zaytzot61}
	{\pgfqpoint{0pt}{0pt}}
	{\pgfpoint{\mcSize+\mcThickness}{\mcSize+\mcThickness}}
	{\pgfpoint{\mcSize}{\mcSize}}
	{
	\pgfsetcolor{\tikz@pattern@color}
	\pgfsetlinewidth{\mcThickness}
	\pgfpathmoveto{\pgfqpoint{0pt}{0pt}}
	\pgfpathlineto{\pgfpoint{\mcSize+\mcThickness}{\mcSize+\mcThickness}}
	\pgfusepath{stroke}
	}}
	\makeatother

	% Pattern Info
	 
	\tikzset{
	pattern size/.store in=\mcSize, 
	pattern size = 5pt,
	pattern thickness/.store in=\mcThickness, 
	pattern thickness = 0.3pt,
	pattern radius/.store in=\mcRadius, 
	pattern radius = 1pt}
	\makeatletter
	\pgfutil@ifundefined{pgf@pattern@name@_sxfafmyuk}{
	\pgfdeclarepatternformonly[\mcThickness,\mcSize]{_sxfafmyuk}
	{\pgfqpoint{0pt}{0pt}}
	{\pgfpoint{\mcSize+\mcThickness}{\mcSize+\mcThickness}}
	{\pgfpoint{\mcSize}{\mcSize}}
	{
	\pgfsetcolor{\tikz@pattern@color}
	\pgfsetlinewidth{\mcThickness}
	\pgfpathmoveto{\pgfqpoint{0pt}{0pt}}
	\pgfpathlineto{\pgfpoint{\mcSize+\mcThickness}{\mcSize+\mcThickness}}
	\pgfusepath{stroke}
	}}
	\makeatother
	\tikzset{every picture/.style={line width=0.75pt}} %set default line width to 0.75pt        

	\begin{tikzpicture}[x=0.75pt,y=0.75pt,yscale=-0.8,xscale=0.8]
	%uncomment if require: \path (0,300); %set diagram left start at 0, and has height of 300

	%Shape: Circle [id:dp628173292640839] 
	\draw  [pattern=_zaytzot61,pattern size=6pt,pattern thickness=0.75pt,pattern radius=0pt, pattern color={rgb, 255:red, 222; green, 222; blue, 222}] (75,144.25) .. controls (75,89.44) and (119.44,45) .. (174.25,45) .. controls (229.06,45) and (273.5,89.44) .. (273.5,144.25) .. controls (273.5,199.06) and (229.06,243.5) .. (174.25,243.5) .. controls (119.44,243.5) and (75,199.06) .. (75,144.25) -- cycle ;
	%Straight Lines [id:da5492081302142064] 
	\draw    (174.25,144.25) -- (174.25,45) ;
	%Shape: Ellipse [id:dp9195499353482499] 
	\draw  [pattern=_sxfafmyuk,pattern size=6pt,pattern thickness=0.75pt,pattern radius=0pt, pattern color={rgb, 255:red, 222; green, 222; blue, 222}] (448.57,141.56) .. controls (448.57,112.03) and (472.5,88.09) .. (502.03,88.09) .. controls (531.56,88.09) and (555.5,112.03) .. (555.5,141.56) .. controls (555.5,171.09) and (531.56,195.02) .. (502.03,195.02) .. controls (472.5,195.02) and (448.57,171.09) .. (448.57,141.56) -- cycle ;
	%Straight Lines [id:da08931752397680315] 
	\draw    (502.03,140.56) -- (502.03,82.09) ;
	%Curve Lines [id:da21347577099567938] 
	\draw [line width=2.25]    (273.5,144.25) .. controls (336.5,57) and (398.5,229) .. (448.57,141.56) ;

	% Text Node
	\draw (174.25,35) node    {$+$};
	% Text Node
	\draw (174.25,255) node    {$+$};
	% Text Node
	\draw (284.25,145) node    {$+$};
	% Text Node
	\draw (64.25,145) node    {$+$};
	% Text Node
	\draw (252.03,67.22) node    {$+$};
	% Text Node
	\draw (96.47,222.78) node    {$+$};
	% Text Node
	\draw (252.03,222.78) node    {$+$};
	% Text Node
	\draw (96.47,67.22) node    {$+$};
	% Text Node
	\draw (217.23,43.74) node    {$+$};
	% Text Node
	\draw (131.27,246.26) node    {$+$};
	% Text Node
	\draw (275.51,187.98) node    {$+$};
	% Text Node
	\draw (72.99,102.02) node    {$+$};
	% Text Node
	\draw (276.24,103.79) node    {$+$};
	% Text Node
	\draw (72.26,186.21) node    {$+$};
	% Text Node
	\draw (215.46,246.99) node    {$+$};
	% Text Node
	\draw (133.04,43.01) node    {$+$};
	% Text Node
	\draw (189.56,98.41) node    {$R_{1}$};
	% Text Node
	\draw (269.56,41.41) node    {$Q_{1} ,V_{1}$};
	% Text Node
	\draw (502.03,76.2) node    {$+$};
	% Text Node
	\draw (502.03,205.8) node    {$+$};
	% Text Node
	\draw (566.84,141) node    {$+$};
	% Text Node
	\draw (437.23,141) node    {$+$};
	% Text Node
	\draw (547.86,95.18) node    {$+$};
	% Text Node
	\draw (456.21,186.82) node    {$+$};
	% Text Node
	\draw (547.86,186.82) node    {$+$};
	% Text Node
	\draw (456.21,95.18) node    {$+$};
	% Text Node
	\draw (527.35,81.35) node    {$+$};
	% Text Node
	\draw (476.71,200.65) node    {$+$};
	% Text Node
	\draw (561.68,166.32) node    {$+$};
	% Text Node
	\draw (442.38,115.68) node    {$+$};
	% Text Node
	\draw (562.12,116.72) node    {$+$};
	% Text Node
	\draw (441.95,165.28) node    {$+$};
	% Text Node
	\draw (526.31,201.08) node    {$+$};
	% Text Node
	\draw (477.76,80.92) node    {$+$};
	% Text Node
	\draw (516.05,113.55) node    {$R_{2}$};
	% Text Node
	\draw (576.18,77.98) node    {$Q_{2} ,V_{2}$};


	\end{tikzpicture}
\end{figure}
\FloatBarrier
Si trova che:
\begin{gather*}
	V(\infty) = 0 \qquad V_1=\frac{Q_1}{4\pi \varepsilon_0 R_1} \qquad V_2=\frac{Q_2}{4\pi \varepsilon_0 R_2} \qquad V_1=V_2 \\
	\frac{Q_1}{4\pi \varepsilon_0 R_1} = \frac{Q_2}{4\pi \varepsilon_0 R_2} \implies \frac{Q_1}{Q_2} = \frac{R_1}{R_2} \\
	\frac{\sigma_1}{\sigma_2} = \frac{\frac{Q_1}{4\pi R_1^2}}{\frac{Q_2}{4\pi R_2^2}} = \frac{Q_1}{Q_2} \frac{R_2^2}{R_1^2} = \frac{R_1}{R_2} \frac{R_2^2}{R_1^2} = \frac{R_2}{R_1} \\
	\boxed{\frac{\sigma_1}{\sigma_2} = \frac{R_2}{R_1}}
\end{gather*}
Otteniamo che la densità di carica superficiale è inversamente proporzionale al suo raggio di curvatura.
\begin{figure}[htpb]
	\centering
	

	\tikzset{every picture/.style={line width=0.75pt}} %set default line width to 0.75pt        

	\begin{tikzpicture}[x=0.75pt,y=0.75pt,yscale=-1,xscale=1]
	%uncomment if require: \path (0,300); %set diagram left start at 0, and has height of 300

	%Shape: Triangle [id:dp6864512559833931] 
	\draw   (370.65,122.5) -- (115,181) -- (115,64) -- cycle ;

	% Text Node
	\draw (105.46,82.99) node    {$+$};
	% Text Node
	\draw (105.46,122.99) node    {$+$};
	% Text Node
	\draw (105.46,162.99) node    {$+$};
	% Text Node
	\draw (363.46,108.49) node    {$+$};
	% Text Node
	\draw (349.96,104.49) node    {$+$};
	% Text Node
	\draw (338.46,100.49) node    {$+$};
	% Text Node
	\draw (324.46,97.49) node    {$+$};
	% Text Node
	\draw (308.96,93.49) node    {$+$};
	% Text Node
	\draw (271.96,87.99) node    {$+$};
	% Text Node
	\draw (243.46,81.49) node    {$+$};
	% Text Node
	\draw (216.46,74.49) node    {$+$};
	% Text Node
	\draw (190.46,67.49) node    {$+$};
	% Text Node
	\draw (165.96,62.49) node    {$+$};
	% Text Node
	\draw (130.96,54.49) node    {$+$};
	% Text Node
	\draw (363.46,131.54) node  [rotate=-180,xscale=-1]  {$+$};
	% Text Node
	\draw (349.96,135.96) node  [rotate=-180,xscale=-1]  {$+$};
	% Text Node
	\draw (338.46,139.38) node  [rotate=-180,xscale=-1]  {$+$};
	% Text Node
	\draw (324.46,141.69) node  [rotate=-180,xscale=-1]  {$+$};
	% Text Node
	\draw (308.96,146.11) node  [rotate=-180,xscale=-1]  {$+$};
	% Text Node
	\draw (273.96,155.19) node  [rotate=-180,xscale=-1]  {$+$};
	% Text Node
	\draw (243.46,161.37) node  [rotate=-180,xscale=-1]  {$+$};
	% Text Node
	\draw (216.46,169.1) node  [rotate=-180,xscale=-1]  {$+$};
	% Text Node
	\draw (190.46,174.84) node  [rotate=-180,xscale=-1]  {$+$};
	% Text Node
	\draw (165.96,180.36) node  [rotate=-180,xscale=-1]  {$+$};
	% Text Node
	\draw (130.96,189.2) node  [rotate=-180,xscale=-1]  {$+$};


	\end{tikzpicture}
\end{figure}
\FloatBarrier
Supponiamo di avere un conduttore dalla forma appuntita come in figura. In una certa posizione, il raggio di curvatura è piccolissimo. Siccome qui accade ciò, la densità di carica è molto elevata. Ecco quindi che la carica si distribuisce in modo più fitto in prossimità della punta. In quel punto avremo un campo elettrico molto intenso. Questa proprietà prende il nome di \textbf{potere delle punte}. Sperimentalmente si osserva che se prendiamo un conduttore con una punta e lo carichiamo, in prossimità di essa avremo un campo elettrico molto intenso tale da poter ionizzare le molecole d'aria intorno ad essa. Da questo effetto hanno origine svariati fenomeni, come la formazione di scintille tra elettrodi di forma appuntita in ambiente gassoso o l'effluvio di elettroni da punte cariche negativamente, che avviene anche nel vuoto.
Il parafulmine si basa su questo principio, la punta ionizza l'aria in prossimità e il fulmine cade in quel luogo perché c'è già un percorso conduttivo pronto. Vi sono moltissime altre applicazioni pratiche di questo concetto.







































\section{Conduttori cavi}

Un conduttore cavo è un conduttore che al suo interno ha una cavità vuota. Supponiamo di averne uno caricato come in figura.

Non potrebbe darsi che la carica sulla superficie attraversi la parete e si depositi sulla superficie interna?
\emph{Dimostrazione per assurdo.} Se consideriamo una superficie astratta gaussiana all'interno del conduttore e applichiamo ad essa il teorema di Gauss si ha che:
\begin{gather*}
	\Phi_{\Sigma_g}(\vec{E}) = \frac{Q_{\text{tot}}}{\varepsilon_0} = \int_{\Sigma_g} \vec{E} \cdot \vec{n} dS = 0 \implies Q_{\text{tot}} = 0
\end{gather*}
Quindi questa carica interna non può esistere perché il campo nel conduttore è nullo, e quindi anche il suo flusso attraverso la superficie $ \Sigma_g  $.

Supponiamo che per qualche motivo si formino cariche di segno opposto ai due estremi della cavità. Se il numero di cariche positive è uguale a quello di cariche negative, le condizioni riguardanti l'equilibrio elettrostatico del sistema non vengono violate. Tuttavia questa situazione non è possibile: per negare questa circostanza si ricorre alla proprietà di conservatività. Scegliamo un percorso chiuso che si ottiene come in figura.
\begin{gather*}
	\oint_{\gamma} \vec{E} \cdot d\vec{l} = 0 = \int_{\text{cavità}} \vec{E} \cdot d\vec{l}  + \underbrace{\int_{\text{conduttore}} \vec{E} \cdot d\vec{l}}_{=0} \\
	\implies \int_{\text{cavità}} \vec{E} \cdot d\vec{l} = 0
\end{gather*}
Arriviamo così ad un paradosso: la proprietà di circuitazione pari a $0$ ci dice che non ci può essere un campo elettrico all'interno della cavità. Ma allora non ci possono nemmeno essere cariche di segno opposto disposte in tal modo. Quindi è impossibile una separazione di carica sulle pareti della cavità. Qualunque sia la quantità di carica deposta sul conduttore, sulla superficie della cavità la carica elettrica va a zero. Inoltre è chiaro che il potenziale in un qualsiasi punto della cavità è uguale a quello del conduttore: se ci fosse una d.d.p. dovrebbe infatti esserci un campo diverso da $0$. In conclusione, \emph{la carica di un conduttore in equilibrio si distribuisce sempre e soltanto sulla superficie esterna, anche se il conduttore è cavo}.
È come se la cavità non risentisse dei fenomeni elettrostatici che avvengono al di fuori. Un'oggetto del genere viene chiamato \textbf{schermo elettrostatico}, tutto si arresta alla superficie, non penetra nel conduttore. Quando vogliamo proteggere un oggetto da fenomeni elettrostatici, basta circondarlo con un materiale conduttore. La gabbia metallica, chiamata anche \emph{gabbia di Faraday}, si basa proprio su questo principio ed è un modo per protegger qualcosa da disturbi elettromagnetici.
\begin{figure}[htpb]
	\centering
	

	% Pattern Info
	 
	\tikzset{
	pattern size/.store in=\mcSize, 
	pattern size = 5pt,
	pattern thickness/.store in=\mcThickness, 
	pattern thickness = 0.3pt,
	pattern radius/.store in=\mcRadius, 
	pattern radius = 1pt}
	\makeatletter
	\pgfutil@ifundefined{pgf@pattern@name@_n77hebug6}{
	\pgfdeclarepatternformonly[\mcThickness,\mcSize]{_n77hebug6}
	{\pgfqpoint{0pt}{0pt}}
	{\pgfpoint{\mcSize+\mcThickness}{\mcSize+\mcThickness}}
	{\pgfpoint{\mcSize}{\mcSize}}
	{
	\pgfsetcolor{\tikz@pattern@color}
	\pgfsetlinewidth{\mcThickness}
	\pgfpathmoveto{\pgfqpoint{0pt}{0pt}}
	\pgfpathlineto{\pgfpoint{\mcSize+\mcThickness}{\mcSize+\mcThickness}}
	\pgfusepath{stroke}
	}}
	\makeatother

	% Pattern Info
	 
	\tikzset{
	pattern size/.store in=\mcSize, 
	pattern size = 5pt,
	pattern thickness/.store in=\mcThickness, 
	pattern thickness = 0.3pt,
	pattern radius/.store in=\mcRadius, 
	pattern radius = 1pt}
	\makeatletter
	\pgfutil@ifundefined{pgf@pattern@name@_ly2th85di}{
	\pgfdeclarepatternformonly[\mcThickness,\mcSize]{_ly2th85di}
	{\pgfqpoint{0pt}{0pt}}
	{\pgfpoint{\mcSize+\mcThickness}{\mcSize+\mcThickness}}
	{\pgfpoint{\mcSize}{\mcSize}}
	{
	\pgfsetcolor{\tikz@pattern@color}
	\pgfsetlinewidth{\mcThickness}
	\pgfpathmoveto{\pgfqpoint{0pt}{0pt}}
	\pgfpathlineto{\pgfpoint{\mcSize+\mcThickness}{\mcSize+\mcThickness}}
	\pgfusepath{stroke}
	}}
	\makeatother

	% Pattern Info
	 
	\tikzset{
	pattern size/.store in=\mcSize, 
	pattern size = 5pt,
	pattern thickness/.store in=\mcThickness, 
	pattern thickness = 0.3pt,
	pattern radius/.store in=\mcRadius, 
	pattern radius = 1pt}
	\makeatletter
	\pgfutil@ifundefined{pgf@pattern@name@_1wjb81hf8}{
	\pgfdeclarepatternformonly[\mcThickness,\mcSize]{_1wjb81hf8}
	{\pgfqpoint{0pt}{0pt}}
	{\pgfpoint{\mcSize+\mcThickness}{\mcSize+\mcThickness}}
	{\pgfpoint{\mcSize}{\mcSize}}
	{
	\pgfsetcolor{\tikz@pattern@color}
	\pgfsetlinewidth{\mcThickness}
	\pgfpathmoveto{\pgfqpoint{0pt}{0pt}}
	\pgfpathlineto{\pgfpoint{\mcSize+\mcThickness}{\mcSize+\mcThickness}}
	\pgfusepath{stroke}
	}}
	\makeatother
	\tikzset{every picture/.style={line width=0.75pt}} %set default line width to 0.75pt        

	\begin{tikzpicture}[x=0.75pt,y=0.75pt,yscale=-0.6,xscale=0.6]
	%uncomment if require: \path (0,352); %set diagram left start at 0, and has height of 352

	%Shape: Path Data [id:dp44392596580582233] 
	\draw  [pattern=_n77hebug6,pattern size=6pt,pattern thickness=0.75pt,pattern radius=0pt, pattern color={rgb, 255:red, 222; green, 222; blue, 222}] (105.06,83.91) .. controls (152.78,83.91) and (191.47,122.6) .. (191.47,170.33) .. controls (191.47,218.06) and (152.78,256.75) .. (105.06,256.75) .. controls (57.33,256.75) and (18.64,218.06) .. (18.64,170.33) .. controls (18.64,122.6) and (57.33,83.91) .. (105.06,83.91) -- cycle (59.13,170.33) .. controls (59.13,195.7) and (79.69,216.26) .. (105.06,216.26) .. controls (130.42,216.26) and (150.98,195.7) .. (150.98,170.33) .. controls (150.98,144.96) and (130.42,124.4) .. (105.06,124.4) .. controls (79.69,124.4) and (59.13,144.96) .. (59.13,170.33) -- cycle ;
	%Shape: Path Data [id:dp25138489216387905] 
	\draw  [pattern=_ly2th85di,pattern size=6pt,pattern thickness=0.75pt,pattern radius=0pt, pattern color={rgb, 255:red, 222; green, 222; blue, 222}] (346.26,83.91) .. controls (393.99,83.91) and (432.68,122.6) .. (432.68,170.33) .. controls (432.68,218.06) and (393.99,256.75) .. (346.26,256.75) .. controls (298.54,256.75) and (259.85,218.06) .. (259.85,170.33) .. controls (259.85,122.6) and (298.54,83.91) .. (346.26,83.91) -- cycle (300.33,170.33) .. controls (300.33,195.7) and (320.9,216.26) .. (346.26,216.26) .. controls (371.63,216.26) and (392.19,195.7) .. (392.19,170.33) .. controls (392.19,144.96) and (371.63,124.4) .. (346.26,124.4) .. controls (320.9,124.4) and (300.33,144.96) .. (300.33,170.33) -- cycle ;
	\draw  [line width=3]  (333.2,285.26) -- (358.89,310.95)(358.89,285.26) -- (333.2,310.95) ;
	%Shape: Path Data [id:dp03545183947977337] 
	\draw  [pattern=_1wjb81hf8,pattern size=6pt,pattern thickness=0.75pt,pattern radius=0pt, pattern color={rgb, 255:red, 222; green, 222; blue, 222}] (586.26,83.91) .. controls (633.99,83.91) and (672.68,122.6) .. (672.68,170.33) .. controls (672.68,218.06) and (633.99,256.75) .. (586.26,256.75) .. controls (538.54,256.75) and (499.85,218.06) .. (499.85,170.33) .. controls (499.85,122.6) and (538.54,83.91) .. (586.26,83.91) -- cycle (540.33,170.33) .. controls (540.33,195.7) and (560.9,216.26) .. (586.26,216.26) .. controls (611.63,216.26) and (632.19,195.7) .. (632.19,170.33) .. controls (632.19,144.96) and (611.63,124.4) .. (586.26,124.4) .. controls (560.9,124.4) and (540.33,144.96) .. (540.33,170.33) -- cycle ;
	%Shape: Ellipse [id:dp7252735373705406] 
	\draw  [dash pattern={on 0.84pt off 2.51pt}] (523.79,170.55) .. controls (523.79,135.92) and (551.86,107.86) .. (586.48,107.86) .. controls (621.1,107.86) and (649.17,135.92) .. (649.17,170.55) .. controls (649.17,205.17) and (621.1,233.24) .. (586.48,233.24) .. controls (551.86,233.24) and (523.79,205.17) .. (523.79,170.55) -- cycle ;
	\draw  [line width=3]  (573.2,285.26) -- (598.89,310.95)(598.89,285.26) -- (573.2,310.95) ;
	%Shape: Ellipse [id:dp10723301414510167] 
	\draw  [dash pattern={on 0.84pt off 2.51pt}] (291.4,205.2) .. controls (291.4,191.28) and (317.01,180) .. (348.6,180) .. controls (380.19,180) and (405.8,191.28) .. (405.8,205.2) .. controls (405.8,219.12) and (380.19,230.4) .. (348.6,230.4) .. controls (317.01,230.4) and (291.4,219.12) .. (291.4,205.2) -- cycle ;
	%Straight Lines [id:da7941270572000036] 
	\draw [line width=3]    (92,303) -- (104,315) ;
	%Straight Lines [id:da5628523951031692] 
	\draw [line width=3]    (119.5,287.25) -- (104,315) ;

	% Text Node
	\draw (105.06,75.21) node    {$+$};
	% Text Node
	\draw (105.06,266.76) node    {$+$};
	% Text Node
	\draw (200.83,170.98) node    {$+$};
	% Text Node
	\draw (9.28,170.98) node    {$+$};
	% Text Node
	\draw (172.78,103.26) node    {$+$};
	% Text Node
	\draw (37.33,238.71) node    {$+$};
	% Text Node
	\draw (172.78,238.71) node    {$+$};
	% Text Node
	\draw (37.33,103.26) node    {$+$};
	% Text Node
	\draw (142.48,82.82) node    {$+$};
	% Text Node
	\draw (67.63,259.14) node    {$+$};
	% Text Node
	\draw (193.22,208.41) node    {$+$};
	% Text Node
	\draw (16.89,133.56) node    {$+$};
	% Text Node
	\draw (193.86,135.1) node    {$+$};
	% Text Node
	\draw (16.25,206.86) node    {$+$};
	% Text Node
	\draw (140.93,259.78) node    {$+$};
	% Text Node
	\draw (69.18,82.18) node    {$+$};
	% Text Node
	\draw (346.26,75.21) node    {$+$};
	% Text Node
	\draw (346.26,266.76) node    {$+$};
	% Text Node
	\draw (442.04,170.98) node    {$+$};
	% Text Node
	\draw (250.49,170.98) node    {$+$};
	% Text Node
	\draw (413.99,103.26) node    {$+$};
	% Text Node
	\draw (278.54,238.71) node    {$+$};
	% Text Node
	\draw (413.99,238.71) node    {$+$};
	% Text Node
	\draw (278.54,103.26) node    {$+$};
	% Text Node
	\draw (383.69,82.82) node    {$+$};
	% Text Node
	\draw (308.84,259.14) node    {$+$};
	% Text Node
	\draw (434.43,208.41) node    {$+$};
	% Text Node
	\draw (258.1,133.56) node    {$+$};
	% Text Node
	\draw (435.06,135.1) node    {$+$};
	% Text Node
	\draw (257.46,206.86) node    {$+$};
	% Text Node
	\draw (382.14,259.78) node    {$+$};
	% Text Node
	\draw (310.38,82.18) node    {$+$};
	% Text Node
	\draw (346.26,130.19) node    {$+$};
	% Text Node
	\draw (346.26,204.81) node    {$-$};
	% Text Node
	\draw (383.58,167.5) node    {$-$};
	% Text Node
	\draw (308.95,167.5) node    {$+$};
	% Text Node
	\draw (372.65,141.12) node    {$-$};
	% Text Node
	\draw (319.88,193.88) node    {$+$};
	% Text Node
	\draw (372.65,193.88) node    {$-$};
	% Text Node
	\draw (319.88,141.12) node    {$+$};
	% Text Node
	\draw (412.16,186.89) node    {$\gamma $};
	% Text Node
	\draw (586.26,75.21) node    {$+$};
	% Text Node
	\draw (586.26,266.76) node    {$+$};
	% Text Node
	\draw (682.04,170.98) node    {$+$};
	% Text Node
	\draw (490.49,170.98) node    {$+$};
	% Text Node
	\draw (653.99,103.26) node    {$+$};
	% Text Node
	\draw (518.54,238.71) node    {$+$};
	% Text Node
	\draw (653.99,238.71) node    {$+$};
	% Text Node
	\draw (518.54,103.26) node    {$+$};
	% Text Node
	\draw (623.69,82.82) node    {$+$};
	% Text Node
	\draw (548.84,259.14) node    {$+$};
	% Text Node
	\draw (674.43,208.41) node    {$+$};
	% Text Node
	\draw (498.1,133.56) node    {$+$};
	% Text Node
	\draw (675.06,135.1) node    {$+$};
	% Text Node
	\draw (497.46,206.86) node    {$+$};
	% Text Node
	\draw (622.14,259.78) node    {$+$};
	% Text Node
	\draw (550.38,82.18) node    {$+$};
	% Text Node
	\draw (586.26,130.19) node    {$+$};
	% Text Node
	\draw (586.26,204.81) node    {$+$};
	% Text Node
	\draw (623.58,167.5) node    {$+$};
	% Text Node
	\draw (548.95,167.5) node    {$+$};
	% Text Node
	\draw (612.65,141.12) node    {$+$};
	% Text Node
	\draw (559.88,193.88) node    {$+$};
	% Text Node
	\draw (612.65,193.88) node    {$+$};
	% Text Node
	\draw (559.88,141.12) node    {$+$};
	% Text Node
	\draw (578.16,98.89) node    {$\Sigma $};


	\end{tikzpicture}
\end{figure}
\FloatBarrier
Questa proprietà funziona anche nell'altro senso. Lo schermo elettrostatico infatti scherma anche quello che accade nella cavità dall'esterno. Poniamo di essere in grado di portare all'interno della cavità una carica positiva $Q$. Chiamiamo $Q_1$ la quantità di carica che si forma sulla superficie della cavità. Sulla superficie all'esterno si distribuirà allora una carica positiva per permettere di mantenere l'equilibrio elettrostatico (la somma delle cariche è sempre la stessa). Scegliamo una superficie astratta gaussiana completamente contenuta nel conduttore e applichiamo il teorema di Gauss.
\begin{gather*}
	\left. \begin{array}{r}
	  	\Phi_{\Sigma_g}(\vec{E} ) = \frac{Q_{\text{tot}}}{\varepsilon_0} = \frac{Q+Q_1}{\varepsilon_0} \\
		\Phi_{\Sigma_g}(\vec{E} ) = \int_{\Sigma_g}\vec{E} \cdot \vec{n} dS = 0
	 \end{array} \right\} \implies Q + Q_1 = 0 \\
	 Q_1 = -Q \qquad Q_2 = -Q_1 = -(-Q) = Q
\end{gather*}
La carica puntiforme più grande, \emph{conducente}, ha indotto una carica negativa uguale al suo opposto sulla superficie interna e automaticamente una carica positiva esattamente pari in modulo sulla superficie esterna. Quando la carica elettrica indotta sulla superficie di un conduttore è esattamente opposta alla carica inducente, si parla di \textbf{induzione elettrostatica completa}.
\begin{figure}[htpb]
	\centering
	

	% Pattern Info
	 
	\tikzset{
	pattern size/.store in=\mcSize, 
	pattern size = 5pt,
	pattern thickness/.store in=\mcThickness, 
	pattern thickness = 0.3pt,
	pattern radius/.store in=\mcRadius, 
	pattern radius = 1pt}
	\makeatletter
	\pgfutil@ifundefined{pgf@pattern@name@_40ui31n0d}{
	\pgfdeclarepatternformonly[\mcThickness,\mcSize]{_40ui31n0d}
	{\pgfqpoint{0pt}{0pt}}
	{\pgfpoint{\mcSize+\mcThickness}{\mcSize+\mcThickness}}
	{\pgfpoint{\mcSize}{\mcSize}}
	{
	\pgfsetcolor{\tikz@pattern@color}
	\pgfsetlinewidth{\mcThickness}
	\pgfpathmoveto{\pgfqpoint{0pt}{0pt}}
	\pgfpathlineto{\pgfpoint{\mcSize+\mcThickness}{\mcSize+\mcThickness}}
	\pgfusepath{stroke}
	}}
	\makeatother
	\tikzset{every picture/.style={line width=0.75pt}} %set default line width to 0.75pt        

	\begin{tikzpicture}[x=0.75pt,y=0.75pt,yscale=-0.8,xscale=0.8]
	%uncomment if require: \path (0,384); %set diagram left start at 0, and has height of 384

	%Shape: Path Data [id:dp9078267867868894] 
	\draw  [pattern=_40ui31n0d,pattern size=6pt,pattern thickness=0.75pt,pattern radius=0pt, pattern color={rgb, 255:red, 222; green, 222; blue, 222}] (309.64,72.54) .. controls (374.27,72.54) and (426.67,124.94) .. (426.67,189.57) .. controls (426.67,254.2) and (374.27,306.6) .. (309.64,306.6) .. controls (245.01,306.6) and (192.61,254.2) .. (192.61,189.57) .. controls (192.61,124.94) and (245.01,72.54) .. (309.64,72.54) -- cycle (247.44,189.57) .. controls (247.44,223.92) and (275.29,251.77) .. (309.64,251.77) .. controls (343.99,251.77) and (371.84,223.92) .. (371.84,189.57) .. controls (371.84,155.22) and (343.99,127.37) .. (309.64,127.37) .. controls (275.29,127.37) and (247.44,155.22) .. (247.44,189.57) -- cycle ;
	%Shape: Ellipse [id:dp18926462592798] 
	\draw  [dash pattern={on 0.84pt off 2.51pt}] (217.46,188.99) .. controls (217.46,138.16) and (258.66,96.97) .. (309.48,96.97) .. controls (360.3,96.97) and (401.5,138.16) .. (401.5,188.99) .. controls (401.5,239.81) and (360.3,281) .. (309.48,281) .. controls (258.66,281) and (217.46,239.81) .. (217.46,188.99) -- cycle ;

	% Text Node
	\draw (309.64,60.75) node    {$+$};
	% Text Node
	\draw (309.64,320.16) node    {$+$};
	% Text Node
	\draw (439.34,190.45) node    {$+$};
	% Text Node
	\draw (179.94,190.45) node    {$+$};
	% Text Node
	\draw (401.35,98.74) node    {$+$};
	% Text Node
	\draw (217.93,282.17) node    {$+$};
	% Text Node
	\draw (401.35,282.17) node    {$+$};
	% Text Node
	\draw (217.93,98.74) node    {$+$};
	% Text Node
	\draw (360.32,71.06) node    {$+$};
	% Text Node
	\draw (258.96,309.84) node    {$+$};
	% Text Node
	\draw (429.03,241.13) node    {$+$};
	% Text Node
	\draw (190.25,139.77) node    {$+$};
	% Text Node
	\draw (429.9,141.86) node    {$+$};
	% Text Node
	\draw (189.38,239.04) node    {$+$};
	% Text Node
	\draw (358.23,310.71) node    {$+$};
	% Text Node
	\draw (261.05,70.19) node    {$+$};
	% Text Node
	\draw  [fill={rgb, 255:red, 222; green, 222; blue, 222 }  ,fill opacity=1 ]  (310, 188.75) circle [x radius= 13.9, y radius= 13.9]   ;
	\draw (310,188.75) node    {$+$};
	% Text Node
	\draw (329.15,141.5) node    {$-$};
	% Text Node
	\draw (290.13,233.4) node    {$-$};
	% Text Node
	\draw (355.59,206.96) node    {$-$};
	% Text Node
	\draw (263.69,167.95) node    {$-$};
	% Text Node
	\draw (355.93,168.75) node    {$-$};
	% Text Node
	\draw (263.35,206.15) node    {$-$};
	% Text Node
	\draw (328.34,233.74) node    {$-$};
	% Text Node
	\draw (290.94,141.17) node    {$-$};
	% Text Node
	\draw (334.44,185.17) node    {$Q$};
	% Text Node
	\draw (360.94,134.17) node    {$Q_{1}$};
	% Text Node
	\draw (420.94,116.67) node    {$Q_{2}$};
	% Text Node
	\draw (297.27,83.8) node    {$\Sigma $};


	\end{tikzpicture}
\end{figure}
\FloatBarrier
Quando il conduttore circonda completamente la carica indotta si verifica questa situazione.
Se sposto un po' nella cavità la carica puntiforme, le cariche negative sulla superficie interna si spostano per schermare tale cambiamento in modo che all'interno del conduttore il campo elettrico rimanga zero. Le cariche positive sulla superficie esterna non si accorgono di nulla, vedono sempre un campo pari a $0$. Le due aree sono separate da una regione morta di campo elettrico nullo che blocca qualsiasi possibilità di trasferire informazione.







































\section{Legame fra carica presente in un conduttore e potenziale}

Consideriamo un conduttore il quale è stato caricato positivamente e immaginiamo di voler calcolare il potenziale generato da esso, $V_0$, conoscendo il campo elettrico. Prendiamo un percorso qualunque da un punto sulla superficie fino al punto del potenziale di riferimento $V(\infty)$, $P_0$.
\begin{figure}[htpb]
	\centering
	

	% Pattern Info
	 
	\tikzset{
	pattern size/.store in=\mcSize, 
	pattern size = 5pt,
	pattern thickness/.store in=\mcThickness, 
	pattern thickness = 0.3pt,
	pattern radius/.store in=\mcRadius, 
	pattern radius = 1pt}
	\makeatletter
	\pgfutil@ifundefined{pgf@pattern@name@_z18cobwuy}{
	\pgfdeclarepatternformonly[\mcThickness,\mcSize]{_z18cobwuy}
	{\pgfqpoint{0pt}{0pt}}
	{\pgfpoint{\mcSize+\mcThickness}{\mcSize+\mcThickness}}
	{\pgfpoint{\mcSize}{\mcSize}}
	{
	\pgfsetcolor{\tikz@pattern@color}
	\pgfsetlinewidth{\mcThickness}
	\pgfpathmoveto{\pgfqpoint{0pt}{0pt}}
	\pgfpathlineto{\pgfpoint{\mcSize+\mcThickness}{\mcSize+\mcThickness}}
	\pgfusepath{stroke}
	}}
	\makeatother
	\tikzset{every picture/.style={line width=0.75pt}} %set default line width to 0.75pt        

	\begin{tikzpicture}[x=0.75pt,y=0.75pt,yscale=-1,xscale=1]
	%uncomment if require: \path (0,300); %set diagram left start at 0, and has height of 300

	%Shape: Circle [id:dp10009141917725017] 
	\draw  [pattern=_z18cobwuy,pattern size=6pt,pattern thickness=0.75pt,pattern radius=0pt, pattern color={rgb, 255:red, 222; green, 222; blue, 222}] (95,164.25) .. controls (95,109.44) and (139.44,65) .. (194.25,65) .. controls (249.06,65) and (293.5,109.44) .. (293.5,164.25) .. controls (293.5,219.06) and (249.06,263.5) .. (194.25,263.5) .. controls (139.44,263.5) and (95,219.06) .. (95,164.25) -- cycle ;
	%Curve Lines [id:da36827921476913694] 
	\draw [line width=0.75]    (293.5,164.25) .. controls (356.5,77) and (418.5,249) .. (468.57,161.56) ;
	\draw [shift={(383.21,161.07)}, rotate = 215.04] [fill={rgb, 255:red, 0; green, 0; blue, 0 }  ][line width=0.08]  [draw opacity=0] (10.72,-5.15) -- (0,0) -- (10.72,5.15) -- (7.12,0) -- cycle    ;
	%Shape: Circle [id:dp033532967118578005] 
	\draw  [fill={rgb, 255:red, 0; green, 0; blue, 0 }  ,fill opacity=1 ] (465.57,161.56) .. controls (465.57,159.9) and (466.91,158.56) .. (468.57,158.56) .. controls (470.22,158.56) and (471.57,159.9) .. (471.57,161.56) .. controls (471.57,163.22) and (470.22,164.56) .. (468.57,164.56) .. controls (466.91,164.56) and (465.57,163.22) .. (465.57,161.56) -- cycle ;
	%Shape: Circle [id:dp01084754965521828] 
	\draw  [fill={rgb, 255:red, 0; green, 0; blue, 0 }  ,fill opacity=1 ] (290.5,164.25) .. controls (290.5,162.59) and (291.84,161.25) .. (293.5,161.25) .. controls (295.16,161.25) and (296.5,162.59) .. (296.5,164.25) .. controls (296.5,165.91) and (295.16,167.25) .. (293.5,167.25) .. controls (291.84,167.25) and (290.5,165.91) .. (290.5,164.25) -- cycle ;

	% Text Node
	\draw (194.25,55) node    {$+$};
	% Text Node
	\draw (194.25,275) node    {$+$};
	% Text Node
	\draw (304.25,165) node    {$+$};
	% Text Node
	\draw (84.25,165) node    {$+$};
	% Text Node
	\draw (272.03,87.22) node    {$+$};
	% Text Node
	\draw (116.47,242.78) node    {$+$};
	% Text Node
	\draw (272.03,242.78) node    {$+$};
	% Text Node
	\draw (116.47,87.22) node    {$+$};
	% Text Node
	\draw (237.23,63.74) node    {$+$};
	% Text Node
	\draw (151.27,266.26) node    {$+$};
	% Text Node
	\draw (295.51,207.98) node    {$+$};
	% Text Node
	\draw (92.99,122.02) node    {$+$};
	% Text Node
	\draw (296.24,123.79) node    {$+$};
	% Text Node
	\draw (92.26,206.21) node    {$+$};
	% Text Node
	\draw (235.46,266.99) node    {$+$};
	% Text Node
	\draw (153.04,63.01) node    {$+$};
	% Text Node
	\draw (295.96,90.21) node    {$V_{0}$};
	% Text Node
	\draw (500,160) node    {$P_{0}( \infty )$};
	% Text Node
	\draw (280,163) node    {$P$};


	\end{tikzpicture}
\end{figure}
\FloatBarrier
\begin{align*}
	V(P) - \underbrace{V(P_0=\infty)}_{=0} &= \int_P^{P_0 = \infty} \vec{E} \cdot d\vec{l} \\
	V(P)=V_0 &= \int_P^{P_0} \vec{E} \cdot d\vec{l}
\end{align*}
Nello spazio vuoto $ \rho =0 $. Qui la funzione $V$ deve soddisfare l'equazione di Poisson. Però se trascuriamo eventuali cariche presenti nello spazio, la relazione si riduce alla semplice equazione di Laplace:
\[
	\nabla^2 V = -\frac{\rho}{\varepsilon_0} = 0
\]
Fissate le condizioni al contorno, esiste una e una sola funzione $V(x,y,z)$ che ha queste condizioni al contorno, per la proprietà dell'equazione di Laplace.

Supponiamo di conoscere il potenziale. Possiamo calcolare il campo elettrico in tutti i punti dello spazio, perché esso è l'opposto del gradiente del potenziale. Date queste condizioni anche il campo elettrico è unico. C'è un unica distribuzione di carica possibile che deriva da queste condizioni al contorno.

Sulla superficie sigma: $ \vec{E} = \frac{\sigma}{\varepsilon_0} \vec{n}  $.
Se vogliamo calcolare la carica $Q$ presente sul conduttore, essa è: $ Q = \int_{\Sigma}\sigma \,dS  $.

Fissato un potenziale, automaticamente è fissata la carica presente sul conduttore. Si può facilmente verificare che se raddoppiamo la carica sul conduttore allora deve necessariamente raddoppiare anche il potenziale.
Si arriva a concludere che, pur di porre potenziale all'infinito pari a zero, il rapporto $ \frac{Q}{V_0} = C $ è costante ed indipendente dal valore di $Q$. Questa quantità $C$ si chiama \textbf{capacità del conduttore}. Le sue dimensioni sono:
\[
	\left[ \frac{Q}{V} \right] = \left( \frac{C}{V} \right) = F \quad \text{farad}
\]
Se abbiamo un conduttore sferico di raggio $R$ e vi deponiamo una certa carica $Q$, sono in grado di calcolare il potenziale ovunque come:
\[
	V_0 = \frac{Q}{4\pi \varepsilon_0 R} \implies C = \frac{Q}{V_0} = \frac{Q}{\frac{Q}{4\pi \varepsilon_0 R}} = 4\pi \varepsilon_0 R \sim 10^{-12} \quad \text{farad}
\]
Vediamo come il farad è un unità di misura estremamente sovradimensionata. Molto spesso le capacità con cui abbiamo a che fare sono frazioni molto più piccole del farad, $nF$, $\mu F$ e via dicendo.







































\section{Sistemi di conduttori in equilibrio statico}

Estendiamo questo problema a un caso più generale. Supponiamo di avere più conduttori ($1$, $2$ e $3$) e di voler stabilire una relazione fra i potenziali che ciascuno di essi assume all'equilibrio e le rispettive cariche. Immaginiamo che i conduttori $2$ e $3$ siano inizialmente neutri e che venga posta della carica $ Q_1  $ sul conduttore $1$.
\begin{figure}[htpb]
	\centering
	

	% Pattern Info
	 
	\tikzset{
	pattern size/.store in=\mcSize, 
	pattern size = 5pt,
	pattern thickness/.store in=\mcThickness, 
	pattern thickness = 0.3pt,
	pattern radius/.store in=\mcRadius, 
	pattern radius = 1pt}
	\makeatletter
	\pgfutil@ifundefined{pgf@pattern@name@_t0rw7xdim}{
	\pgfdeclarepatternformonly[\mcThickness,\mcSize]{_t0rw7xdim}
	{\pgfqpoint{0pt}{0pt}}
	{\pgfpoint{\mcSize+\mcThickness}{\mcSize+\mcThickness}}
	{\pgfpoint{\mcSize}{\mcSize}}
	{
	\pgfsetcolor{\tikz@pattern@color}
	\pgfsetlinewidth{\mcThickness}
	\pgfpathmoveto{\pgfqpoint{0pt}{0pt}}
	\pgfpathlineto{\pgfpoint{\mcSize+\mcThickness}{\mcSize+\mcThickness}}
	\pgfusepath{stroke}
	}}
	\makeatother

	% Pattern Info
	 
	\tikzset{
	pattern size/.store in=\mcSize, 
	pattern size = 5pt,
	pattern thickness/.store in=\mcThickness, 
	pattern thickness = 0.3pt,
	pattern radius/.store in=\mcRadius, 
	pattern radius = 1pt}
	\makeatletter
	\pgfutil@ifundefined{pgf@pattern@name@_ia5eax6xo}{
	\pgfdeclarepatternformonly[\mcThickness,\mcSize]{_ia5eax6xo}
	{\pgfqpoint{0pt}{0pt}}
	{\pgfpoint{\mcSize+\mcThickness}{\mcSize+\mcThickness}}
	{\pgfpoint{\mcSize}{\mcSize}}
	{
	\pgfsetcolor{\tikz@pattern@color}
	\pgfsetlinewidth{\mcThickness}
	\pgfpathmoveto{\pgfqpoint{0pt}{0pt}}
	\pgfpathlineto{\pgfpoint{\mcSize+\mcThickness}{\mcSize+\mcThickness}}
	\pgfusepath{stroke}
	}}
	\makeatother

	% Pattern Info
	 
	\tikzset{
	pattern size/.store in=\mcSize, 
	pattern size = 5pt,
	pattern thickness/.store in=\mcThickness, 
	pattern thickness = 0.3pt,
	pattern radius/.store in=\mcRadius, 
	pattern radius = 1pt}
	\makeatletter
	\pgfutil@ifundefined{pgf@pattern@name@_ucmmzdmku}{
	\pgfdeclarepatternformonly[\mcThickness,\mcSize]{_ucmmzdmku}
	{\pgfqpoint{0pt}{0pt}}
	{\pgfpoint{\mcSize+\mcThickness}{\mcSize+\mcThickness}}
	{\pgfpoint{\mcSize}{\mcSize}}
	{
	\pgfsetcolor{\tikz@pattern@color}
	\pgfsetlinewidth{\mcThickness}
	\pgfpathmoveto{\pgfqpoint{0pt}{0pt}}
	\pgfpathlineto{\pgfpoint{\mcSize+\mcThickness}{\mcSize+\mcThickness}}
	\pgfusepath{stroke}
	}}
	\makeatother
	\tikzset{every picture/.style={line width=0.75pt}} %set default line width to 0.75pt        

	\begin{tikzpicture}[x=0.75pt,y=0.75pt,yscale=-0.5,xscale=0.5]
	%uncomment if require: \path (0,507); %set diagram left start at 0, and has height of 507

	%Shape: Circle [id:dp463892695966174] 
	\draw  [pattern=_t0rw7xdim,pattern size=6pt,pattern thickness=0.75pt,pattern radius=0pt, pattern color={rgb, 255:red, 222; green, 222; blue, 222}] (95,164.25) .. controls (95,109.44) and (139.44,65) .. (194.25,65) .. controls (249.06,65) and (293.5,109.44) .. (293.5,164.25) .. controls (293.5,219.06) and (249.06,263.5) .. (194.25,263.5) .. controls (139.44,263.5) and (95,219.06) .. (95,164.25) -- cycle ;
	%Shape: Ellipse [id:dp3622162768048902] 
	\draw  [pattern=_ia5eax6xo,pattern size=6pt,pattern thickness=0.75pt,pattern radius=0pt, pattern color={rgb, 255:red, 222; green, 222; blue, 222}] (386.57,102.56) .. controls (386.57,73.03) and (410.5,49.09) .. (440.03,49.09) .. controls (469.56,49.09) and (493.5,73.03) .. (493.5,102.56) .. controls (493.5,132.09) and (469.56,156.02) .. (440.03,156.02) .. controls (410.5,156.02) and (386.57,132.09) .. (386.57,102.56) -- cycle ;
	%Shape: Ellipse [id:dp2897197347426077] 
	\draw  [pattern=_ucmmzdmku,pattern size=6pt,pattern thickness=0.75pt,pattern radius=0pt, pattern color={rgb, 255:red, 222; green, 222; blue, 222}] (378.58,294.76) .. controls (378.58,254.53) and (411.2,221.91) .. (451.43,221.91) .. controls (491.66,221.91) and (524.28,254.53) .. (524.28,294.76) .. controls (524.28,334.99) and (491.66,367.61) .. (451.43,367.61) .. controls (411.2,367.61) and (378.58,334.99) .. (378.58,294.76) -- cycle ;

	% Text Node
	\draw (194.25,55) node    {$+$};
	% Text Node
	\draw (194.25,275) node    {$+$};
	% Text Node
	\draw (304.25,165) node    {$+$};
	% Text Node
	\draw (84.25,165) node    {$+$};
	% Text Node
	\draw (272.03,87.22) node    {$+$};
	% Text Node
	\draw (116.47,242.78) node    {$+$};
	% Text Node
	\draw (272.03,242.78) node    {$+$};
	% Text Node
	\draw (116.47,87.22) node    {$+$};
	% Text Node
	\draw (237.23,63.74) node    {$+$};
	% Text Node
	\draw (151.27,266.26) node    {$+$};
	% Text Node
	\draw (295.51,207.98) node    {$+$};
	% Text Node
	\draw (92.99,122.02) node    {$+$};
	% Text Node
	\draw (296.24,123.79) node    {$+$};
	% Text Node
	\draw (92.26,206.21) node    {$+$};
	% Text Node
	\draw (235.46,266.99) node    {$+$};
	% Text Node
	\draw (153.04,63.01) node    {$+$};
	% Text Node
	\draw (194.25,164.25) node    {$1$};
	% Text Node
	\draw (440.03,37.2) node    {$-$};
	% Text Node
	\draw (440.03,166.8) node    {$+$};
	% Text Node
	\draw (504.84,102) node    {$+$};
	% Text Node
	\draw (375.23,102) node    {$-$};
	% Text Node
	\draw (485.86,56.18) node    {$+$};
	% Text Node
	\draw (394.21,147.82) node    {$-$};
	% Text Node
	\draw (485.86,147.82) node    {$+$};
	% Text Node
	\draw (394.21,56.18) node    {$-$};
	% Text Node
	\draw (465.35,42.35) node    {$+$};
	% Text Node
	\draw (414.71,161.65) node    {$-$};
	% Text Node
	\draw (499.68,127.32) node    {$+$};
	% Text Node
	\draw (380.38,76.68) node    {$-$};
	% Text Node
	\draw (500.12,77.72) node    {$+$};
	% Text Node
	\draw (379.95,126.28) node    {$-$};
	% Text Node
	\draw (464.31,162.08) node    {$+$};
	% Text Node
	\draw (415.76,41.92) node    {$-$};
	% Text Node
	\draw (440.03,102.56) node    {$2$};
	% Text Node
	\draw (451.43,205.7) node    {$-$};
	% Text Node
	\draw (451.43,382.3) node    {$+$};
	% Text Node
	\draw (539.72,294) node    {$+$};
	% Text Node
	\draw (363.13,294) node    {$-$};
	% Text Node
	\draw (513.86,231.57) node    {$+$};
	% Text Node
	\draw (389,356.43) node    {$-$};
	% Text Node
	\draw (513.86,356.43) node    {$+$};
	% Text Node
	\draw (389,231.57) node    {$-$};
	% Text Node
	\draw (485.93,212.72) node    {$+$};
	% Text Node
	\draw (416.93,375.28) node    {$-$};
	% Text Node
	\draw (532.71,328.5) node    {$+$};
	% Text Node
	\draw (370.15,259.5) node    {$-$};
	% Text Node
	\draw (533.3,260.92) node    {$+$};
	% Text Node
	\draw (369.56,327.08) node    {$-$};
	% Text Node
	\draw (484.51,375.87) node    {$+$};
	% Text Node
	\draw (418.35,212.13) node    {$-$};
	% Text Node
	\draw (451.43,294.76) node    {$3$};


	\end{tikzpicture}
\end{figure}
\FloatBarrier
Essa provoca per induzione elettrostatica uno spostamento di carica nei conduttori $2$ e $3$: anche se rimarranno complessivamente neutri, tali cariche una volta spostate saranno soggette a un campo elettrico, che andrà a influenzare tutti i conduttori, compreso il primo. Stabilire qual è la situazione definitiva non è così banale a causa di questa retroazione fra i conduttori. Facciamo un ragionamento simile a quello fatto nel caso di un conduttore: fissiamo la condizione al contorno che il potenziale all'infinito sia pari a $0$ e assuma un certo valore sulle varie superfici dei conduttori. Per l'unicità della soluzione dell'equazione di Laplace, sappiamo che ci sarà una sola funzione che soddisfa tali condizioni: possiamo calcolare in modo univoco il campo elettrico ovunque e di conseguenza in prossimità delle superfici dei vari conduttori potremo calcolare anche la densità di carica superficiale.
Per il teorema di Coulomb sappiamo che sul conduttore:
\[
	\vec{E} = - \vec{\nabla} V = \frac{\sigma}{\varepsilon_0}\vec{n}
\]
La carica $ Q_1  $ può inoltre essere scritta come: $ Q_1 = \int_{\Sigma_1}\sigma_1\,dS$. Per il principio di conservazione della carica, dato che i conduttore $2$ e $3$ erano inizialmente neutri, se su di essi non abbiamo portato carica in più, gli integrali:
\[
	Q_2=\int_{\Sigma_2}\sigma_2\,dS \qquad Q_3=\int_{\Sigma_3}\sigma_3\,dS
\]
devono valere $0$.
Fatte tali considerazioni, immaginiamo di moltiplicare il potenziale per una costante $ \alpha  $.
\[
	V'(x,y,z) = \alpha V(x,y,z)
\]
Con questa operazione, anche il campo elettrico aumenta di un fattore moltiplicativo $ \alpha  $. Ma allora, siccome tale campo elettrico lo possiamo calcolare ovunque, anche tutte le densità di carica e di conseguenza le cariche stesse aumenteranno di un fattore $\alpha$.
\begin{gather*}
	\vec{E}'(x,y,z) = \alpha \vec{E} (x,y,z) \\
	\Downarrow \\
	\sigma_i'=\sigma_i \\
	\Downarrow \\
	Q_i' = \alpha Q_i
\end{gather*}
In particolare per le cariche si avrà che:
\[
	Q_1' = \alpha Q_1 \qquad Q_2'=0 \qquad Q_3'= 0
\]
Se aumentiamo il potenziale di un fattore $\alpha$, esso aumenta anche sulle superfici dei conduttori.
\[
	V_1'=\alpha V_1 \qquad V_2'=\alpha V_2 \qquad V_3'=\alpha V_3
\]
Notiamo che il potenziale sul conduttore $1$ sicuramente è proporzionale alla sua carica. Ma anche i potenziali sugli altri conduttori devono essere proporzionali alla carica sul primo. Tutto ciò consente di affermare che il potenziale sul conduttore $i$-esimo si può scrivere come:
\[
	V_i = a_{ij}Q_j
\]
Nel caso ancora più generale in cui le cariche sono diverse da zero anche sugli altri $n-1$ conduttori, si considera altre $n-1$ situazioni del tipo appena visto in cui è diversa da zero la carica sull'$i$-esimo conduttore ed è nulla su tutti gli $n-1$ restanti. E quindi, a norma della proprietà additiva dei potenziali, il potenziale complessivo di ciascun conduttore si ottiene dalla somma dei singoli potenziali:
E\[
	V_i = \sum_{i,j}^n a_{ij}Q_j
\]
Dove i coefficienti $ a_{ij}  $ sono detti \textbf{coefficienti di potenziale}. Siccome la soluzione all'equazione di Laplace è unica, questo sistema di equazioni è sempre dotato di soluzione. Il determinante del sistema di equazioni è diverso da $0$. Il sistema allora è invertibile e possiamo trovare le cariche in funzione dei potenziali.
\[
	Q_i = \sum_{i,j}^n C_{ij}V_j
\]
Se $i=j$, $C_{ii} $ prende il nome di \textbf{coefficiente di capacità}, perché lega la carica al potenziale presente sul singolo conduttore. È in pratica la capacità del conduttore. Se $i\neq j$, il coefficiente prende il nome di \textbf{coefficiente di induzione}. Ci sta dicendo che effetto ha il potenziale su un conduttore sulla carica di un altro conduttore, descrive la loro interazione mutua.
Notiamo inoltre che:
\[
	a_{ij} = a_{ji}
\]
Questo significa semplicemente che il conduttore $i$ agisce su $j$ come $j$ agisce su $i$, i due coefficienti descrivono la stessa interazione.
Tutti questi coefficienti sono positivi perché una carica di un certo segno da un potenziale dello stesso segno.







































\section{Condensatori}

Poniamo di avere un conduttore cavo all'interno del quale poniamo un conduttore più piccolo. Se deponiamo sul conduttore $1$ della carica $Q_1$, ci troviamo di fronte a un caso di induzione elettrostatica completa. Sulla faccia della cavità interna si troveranno delle cariche negative tali che $Q_2=-Q_1$. All'esterno si formerà una carica positiva $Q_3$ sempre pari a $Q_1$. Un oggetto di questo tipo si chiama \textbf{condensatore}. Si tratta di un sistema di due conduttori in induzione elettrostatica completa.
\begin{figure}[htpb]
	\centering
	

	% Pattern Info
	 
	\tikzset{
	pattern size/.store in=\mcSize, 
	pattern size = 5pt,
	pattern thickness/.store in=\mcThickness, 
	pattern thickness = 0.3pt,
	pattern radius/.store in=\mcRadius, 
	pattern radius = 1pt}
	\makeatletter
	\pgfutil@ifundefined{pgf@pattern@name@_8tylpfyc0}{
	\pgfdeclarepatternformonly[\mcThickness,\mcSize]{_8tylpfyc0}
	{\pgfqpoint{0pt}{0pt}}
	{\pgfpoint{\mcSize+\mcThickness}{\mcSize+\mcThickness}}
	{\pgfpoint{\mcSize}{\mcSize}}
	{
	\pgfsetcolor{\tikz@pattern@color}
	\pgfsetlinewidth{\mcThickness}
	\pgfpathmoveto{\pgfqpoint{0pt}{0pt}}
	\pgfpathlineto{\pgfpoint{\mcSize+\mcThickness}{\mcSize+\mcThickness}}
	\pgfusepath{stroke}
	}}
	\makeatother

	% Pattern Info
	 
	\tikzset{
	pattern size/.store in=\mcSize, 
	pattern size = 5pt,
	pattern thickness/.store in=\mcThickness, 
	pattern thickness = 0.3pt,
	pattern radius/.store in=\mcRadius, 
	pattern radius = 1pt}
	\makeatletter
	\pgfutil@ifundefined{pgf@pattern@name@_d6zmalwtt}{
	\pgfdeclarepatternformonly[\mcThickness,\mcSize]{_d6zmalwtt}
	{\pgfqpoint{0pt}{0pt}}
	{\pgfpoint{\mcSize+\mcThickness}{\mcSize+\mcThickness}}
	{\pgfpoint{\mcSize}{\mcSize}}
	{
	\pgfsetcolor{\tikz@pattern@color}
	\pgfsetlinewidth{\mcThickness}
	\pgfpathmoveto{\pgfqpoint{0pt}{0pt}}
	\pgfpathlineto{\pgfpoint{\mcSize+\mcThickness}{\mcSize+\mcThickness}}
	\pgfusepath{stroke}
	}}
	\makeatother
	\tikzset{every picture/.style={line width=0.75pt}} %set default line width to 0.75pt        

	\begin{tikzpicture}[x=0.75pt,y=0.75pt,yscale=-1,xscale=1]
	%uncomment if require: \path (0,373); %set diagram left start at 0, and has height of 373

	%Shape: Path Data [id:dp7080683984931779] 
	\draw  [pattern=_8tylpfyc0,pattern size=6pt,pattern thickness=0.75pt,pattern radius=0pt, pattern color={rgb, 255:red, 222; green, 222; blue, 222}] (296.63,59.47) .. controls (360.94,59.47) and (413.09,111.61) .. (413.09,175.93) .. controls (413.09,240.25) and (360.94,292.39) .. (296.63,292.39) .. controls (232.31,292.39) and (180.16,240.25) .. (180.16,175.93) .. controls (180.16,111.61) and (232.31,59.47) .. (296.63,59.47) -- cycle (207.7,175.93) .. controls (207.7,225.04) and (247.51,264.86) .. (296.63,264.86) .. controls (345.74,264.86) and (385.55,225.04) .. (385.55,175.93) .. controls (385.55,126.81) and (345.74,87) .. (296.63,87) .. controls (247.51,87) and (207.7,126.81) .. (207.7,175.93) -- cycle ;
	%Shape: Ellipse [id:dp3439458197477021] 
	\draw  [pattern=_d6zmalwtt,pattern size=6pt,pattern thickness=0.75pt,pattern radius=0pt, pattern color={rgb, 255:red, 222; green, 222; blue, 222}] (258.7,175.93) .. controls (258.7,154.98) and (275.68,138) .. (296.63,138) .. controls (317.57,138) and (334.55,154.98) .. (334.55,175.93) .. controls (334.55,196.88) and (317.57,213.86) .. (296.63,213.86) .. controls (275.68,213.86) and (258.7,196.88) .. (258.7,175.93) -- cycle ;

	% Text Node
	\draw (296.63,128.56) node    {$+$};
	% Text Node
	\draw (296.63,219.05) node    {$+$};
	% Text Node
	\draw (341.87,173.81) node    {$+$};
	% Text Node
	\draw (251.38,173.81) node    {$+$};
	% Text Node
	\draw (328.62,141.81) node    {$+$};
	% Text Node
	\draw (264.63,205.8) node    {$+$};
	% Text Node
	\draw (328.62,205.8) node    {$+$};
	% Text Node
	\draw (264.63,141.81) node    {$+$};
	% Text Node
	\draw (314.3,132.16) node    {$+$};
	% Text Node
	\draw (278.95,215.46) node    {$+$};
	% Text Node
	\draw (338.27,191.49) node    {$+$};
	% Text Node
	\draw (254.98,156.13) node    {$+$};
	% Text Node
	\draw (338.58,156.86) node    {$+$};
	% Text Node
	\draw (254.67,190.76) node    {$+$};
	% Text Node
	\draw (313.57,215.76) node    {$+$};
	% Text Node
	\draw (279.68,131.86) node    {$+$};
	% Text Node
	\draw (348,136) node    {$Q_{1}$};
	% Text Node
	\draw (296.63,46.96) node    {$-$};
	% Text Node
	\draw (296.63,300.66) node    {$-$};
	% Text Node
	\draw (423.47,173.81) node    {$-$};
	% Text Node
	\draw (169.78,173.81) node    {$-$};
	% Text Node
	\draw (386.32,84.11) node    {$-$};
	% Text Node
	\draw (206.93,263.5) node    {$-$};
	% Text Node
	\draw (386.32,263.5) node    {$-$};
	% Text Node
	\draw (206.93,84.11) node    {$-$};
	% Text Node
	\draw (346.19,57.04) node    {$-$};
	% Text Node
	\draw (247.06,290.57) node    {$-$};
	% Text Node
	\draw (413.39,223.37) node    {$-$};
	% Text Node
	\draw (179.86,124.24) node    {$-$};
	% Text Node
	\draw (414.24,126.29) node    {$-$};
	% Text Node
	\draw (179.01,221.33) node    {$-$};
	% Text Node
	\draw (344.14,291.42) node    {$-$};
	% Text Node
	\draw (249.11,56.2) node    {$-$};
	% Text Node
	\draw (420,99) node    {$Q_{2}$};


	\end{tikzpicture}
\end{figure}
\FloatBarrier
Se colleghiamo il conduttore esterno a terra, le cariche esterne tendono ad allontanarsi perché si respingono e fuggono tutte verso terra (si dice verso massa). Ciò che si trova all'interno della cavità non si accorge di questo cambiamento, quindi le cariche interne rimarranno inalterate.
Vediamo come varia il potenziale:
\[
	V_1 = a_{11}Q_1 + a_{12}Q_2 \qquad V_2=a_{21}Q_1+a_{22}Q_2
\]
Essendo in induzione completa $ Q_1=-Q_2   $ e ricordando $ a_{ij}=a_{ji}   $:
\begin{gather*}
	V_1 = a_{11}Q_1 + a_{12}Q_2 \qquad V_2=a_{21}Q_1+a_{22}Q_2 \\
	V_1 = a_{11}Q_1 - a_{12}Q_1 \qquad V_2=a_{12}Q_1-a_{22}Q_1 \\
	\Delta V = V_1 - V_2 = a_{11}Q_1 - a_{12}Q_1  - a_{12}Q_1 + a_{22}Q_1  \\
	\Delta V = Q_1 \underbrace{[a_{11} - 2a_{12} + a_{22}]}_{1/C}
\end{gather*}
Abbiamo ottenuto che la differenza di potenziale fra i due conduttori è proporzionale alla carica $ Q_1  $. Evidentemente il rapporto $ Q_1/\Delta V  $ è anche esso una costante. Per analogia questo rapporto prende il nome di \textbf{capacità del condensatore}. Anche tale rapporto è funzione della geometria del condensatore
\[
	C = \frac{Q}{\Delta V}
\]
Molto spesso i due conduttori che costituiscono il condensatore prendono il nome di \textbf{armature del condensatore}. Il condensatore sferico è l'unico per il quale sicuramente siamo in induzione completa. Molto spesso vengono chiamati condensatori oggetti che non lo sono davvero a rigore. Per esempio viene considerato condensatore anche una struttura composta da due porzioni di superfici cilindriche coassiali, una di raggio $ R_1  $ e un'altra di raggio $ R_2 > R_1   $, di eguale lunghezza $d$, grande rispetto ai raggi. Si parla di \textbf{condensatore cilindrico}.
Un altro esempio di condensatore è il \textbf{condensatore piano}. Esso è costituito da due conduttori piani paralleli. La carica positiva q è distribuita con densità uniforme sull'armatura positiva e quella negativa con densità uniforme sull'altra armatura.

Questi conduttori non sono in induzione elettrostatica completa. Solo in quest'ultimo caso le cariche sono esattamente l'una l'opposto dell'alta. Consideriamo il caso in figura. La superficie gaussiana che potrei rappresentare è in parte interna al metallo. Ma c'è una porzione di superficie che si trova nello spazio vuoto. Nessuno ci garantisce che il flusso del campo elettrico attraverso tale parte di superficie sia zero, come potremmo dire nel caso del condensatore sferico. Le configurazioni regolari del campo $\vec{E}$ infatti non sono realmente realizzabili nella pratica. Esse sarebbero corrette se l'estensione fosse indefinita; per una dimensione finita si avrebbe, nella zona del bordo, un passaggio brusco dalla regione in cui esiste un campo elettrico regolare alla regione con un campo elettrico nullo e sarebbe possibile trovare un percorso chiuso $ \gamma  $ come in figura tale che:
\[
	\oint_{\gamma} \vec{E} \cdot d\vec{l} \neq 0
\]
\begin{figure}[htpb]
	\centering
	

	% Pattern Info
	 
	\tikzset{
	pattern size/.store in=\mcSize, 
	pattern size = 5pt,
	pattern thickness/.store in=\mcThickness, 
	pattern thickness = 0.3pt,
	pattern radius/.store in=\mcRadius, 
	pattern radius = 1pt}
	\makeatletter
	\pgfutil@ifundefined{pgf@pattern@name@_3ubi1q8ek}{
	\pgfdeclarepatternformonly[\mcThickness,\mcSize]{_3ubi1q8ek}
	{\pgfqpoint{0pt}{0pt}}
	{\pgfpoint{\mcSize+\mcThickness}{\mcSize+\mcThickness}}
	{\pgfpoint{\mcSize}{\mcSize}}
	{
	\pgfsetcolor{\tikz@pattern@color}
	\pgfsetlinewidth{\mcThickness}
	\pgfpathmoveto{\pgfqpoint{0pt}{0pt}}
	\pgfpathlineto{\pgfpoint{\mcSize+\mcThickness}{\mcSize+\mcThickness}}
	\pgfusepath{stroke}
	}}
	\makeatother

	% Pattern Info
	 
	\tikzset{
	pattern size/.store in=\mcSize, 
	pattern size = 5pt,
	pattern thickness/.store in=\mcThickness, 
	pattern thickness = 0.3pt,
	pattern radius/.store in=\mcRadius, 
	pattern radius = 1pt}
	\makeatletter
	\pgfutil@ifundefined{pgf@pattern@name@_b3y094ngj}{
	\pgfdeclarepatternformonly[\mcThickness,\mcSize]{_b3y094ngj}
	{\pgfqpoint{0pt}{0pt}}
	{\pgfpoint{\mcSize+\mcThickness}{\mcSize+\mcThickness}}
	{\pgfpoint{\mcSize}{\mcSize}}
	{
	\pgfsetcolor{\tikz@pattern@color}
	\pgfsetlinewidth{\mcThickness}
	\pgfpathmoveto{\pgfqpoint{0pt}{0pt}}
	\pgfpathlineto{\pgfpoint{\mcSize+\mcThickness}{\mcSize+\mcThickness}}
	\pgfusepath{stroke}
	}}
	\makeatother
	\tikzset{every picture/.style={line width=0.75pt}} %set default line width to 0.75pt        

	\begin{tikzpicture}[x=0.75pt,y=0.75pt,yscale=-1,xscale=1]
	%uncomment if require: \path (0,300); %set diagram left start at 0, and has height of 300

	%Shape: Rectangle [id:dp6085496164711728] 
	\draw  [pattern=_3ubi1q8ek,pattern size=6pt,pattern thickness=0.75pt,pattern radius=0pt, pattern color={rgb, 255:red, 222; green, 222; blue, 222}] (196,49) -- (214.5,49) -- (214.5,271.75) -- (196,271.75) -- cycle ;
	%Shape: Rectangle [id:dp781196133860844] 
	\draw  [pattern=_b3y094ngj,pattern size=6pt,pattern thickness=0.75pt,pattern radius=0pt, pattern color={rgb, 255:red, 222; green, 222; blue, 222}] (356,49) -- (374.5,49) -- (374.5,271.75) -- (356,271.75) -- cycle ;
	%Straight Lines [id:da16020172523607368] 
	\draw    (215,70) -- (352.67,70) ;
	\draw [shift={(355.67,70)}, rotate = 180] [fill={rgb, 255:red, 0; green, 0; blue, 0 }  ][line width=0.08]  [draw opacity=0] (10.72,-5.15) -- (0,0) -- (10.72,5.15) -- (7.12,0) -- cycle    ;
	%Straight Lines [id:da7517143629477214] 
	\draw    (215,100) -- (352.67,100) ;
	\draw [shift={(355.67,100)}, rotate = 180] [fill={rgb, 255:red, 0; green, 0; blue, 0 }  ][line width=0.08]  [draw opacity=0] (10.72,-5.15) -- (0,0) -- (10.72,5.15) -- (7.12,0) -- cycle    ;
	%Straight Lines [id:da06685263891412374] 
	\draw    (215,130) -- (352.67,130) ;
	\draw [shift={(355.67,130)}, rotate = 180] [fill={rgb, 255:red, 0; green, 0; blue, 0 }  ][line width=0.08]  [draw opacity=0] (10.72,-5.15) -- (0,0) -- (10.72,5.15) -- (7.12,0) -- cycle    ;
	%Straight Lines [id:da6077615910716951] 
	\draw    (215,160) -- (352.67,160) ;
	\draw [shift={(355.67,160)}, rotate = 180] [fill={rgb, 255:red, 0; green, 0; blue, 0 }  ][line width=0.08]  [draw opacity=0] (10.72,-5.15) -- (0,0) -- (10.72,5.15) -- (7.12,0) -- cycle    ;
	%Straight Lines [id:da18653546680564426] 
	\draw    (215,190) -- (352.67,190) ;
	\draw [shift={(355.67,190)}, rotate = 180] [fill={rgb, 255:red, 0; green, 0; blue, 0 }  ][line width=0.08]  [draw opacity=0] (10.72,-5.15) -- (0,0) -- (10.72,5.15) -- (7.12,0) -- cycle    ;
	%Straight Lines [id:da31737719728233493] 
	\draw    (215,220) -- (352.67,220) ;
	\draw [shift={(355.67,220)}, rotate = 180] [fill={rgb, 255:red, 0; green, 0; blue, 0 }  ][line width=0.08]  [draw opacity=0] (10.72,-5.15) -- (0,0) -- (10.72,5.15) -- (7.12,0) -- cycle    ;
	%Straight Lines [id:da8133959062018439] 
	\draw    (215,250) -- (352.67,250) ;
	\draw [shift={(355.67,250)}, rotate = 180] [fill={rgb, 255:red, 0; green, 0; blue, 0 }  ][line width=0.08]  [draw opacity=0] (10.72,-5.15) -- (0,0) -- (10.72,5.15) -- (7.12,0) -- cycle    ;
	%Curve Lines [id:da7774271303818328] 
	\draw    (356,49) .. controls (321.7,25.15) and (251.87,23.4) .. (216.61,47.49) ;
	\draw [shift={(214.5,49)}, rotate = 323.35] [fill={rgb, 255:red, 0; green, 0; blue, 0 }  ][line width=0.08]  [draw opacity=0] (10.72,-5.15) -- (0,0) -- (10.72,5.15) -- (7.12,0) -- cycle    ;
	%Shape: Rectangle [id:dp07531599169392056] 
	\draw   (242.8,241.4) -- (317,241.4) -- (317,288) -- (242.8,288) -- cycle ;
	\draw   (282.6,283.6) -- (291.4,288) -- (282.6,292.4) ;

	% Text Node
	\draw (206.06,59) node    {$+$};
	% Text Node
	\draw (206.06,79) node    {$+$};
	% Text Node
	\draw (206.06,99) node    {$+$};
	% Text Node
	\draw (206.06,119) node    {$+$};
	% Text Node
	\draw (206.06,139) node    {$+$};
	% Text Node
	\draw (206.06,159) node    {$+$};
	% Text Node
	\draw (206.06,179) node    {$+$};
	% Text Node
	\draw (206.06,199) node    {$+$};
	% Text Node
	\draw (206.06,219) node    {$+$};
	% Text Node
	\draw (206.06,239) node    {$+$};
	% Text Node
	\draw (206.06,259) node    {$+$};
	% Text Node
	\draw (365.06,59) node    {$-$};
	% Text Node
	\draw (365.06,79) node    {$-$};
	% Text Node
	\draw (365.06,99) node    {$-$};
	% Text Node
	\draw (365.06,119) node    {$-$};
	% Text Node
	\draw (365.06,139) node    {$-$};
	% Text Node
	\draw (365.06,159) node    {$-$};
	% Text Node
	\draw (365.06,179) node    {$-$};
	% Text Node
	\draw (365.06,199) node    {$-$};
	% Text Node
	\draw (365.06,219) node    {$-$};
	% Text Node
	\draw (365.06,239) node    {$-$};
	% Text Node
	\draw (365.06,259) node    {$-$};
	% Text Node
	\draw (231.66,280.6) node    {$\gamma $};


	\end{tikzpicture}
\end{figure}
\FloatBarrier
Vi è infatti un contributo positivo nella parte interna al condensatore, mentre nella parte esterna non troviamo contributo né positivo né negativo. Essendo il campo conservativo, questa possibilità va esclusa e in effetti il campo è regolare solo nella zona centrale del condensatore, mentre vicino ai bordi le linee di forza sono deformate ed escono all'esterno, assumendo una configurazione che assicura la nullità della circuitazione del campo elettrico; il valore del campo elettrico è ad ogni modo rapidamente decrescente verso l'esterno.
Trascurando gli effetti di bordo, ponendo $ \vec{E} =0 $ all'esterno del condensatore, allora possiamo dire che le due cariche sulle armature sono esattamente l'una l'opposto dell'altra, ma questa è un'approssimazione.

Consideriamo un condensatore a due lastre parallele.
\begin{figure}[htpb]
	\centering
	

	% Pattern Info
	 
	\tikzset{
	pattern size/.store in=\mcSize, 
	pattern size = 5pt,
	pattern thickness/.store in=\mcThickness, 
	pattern thickness = 0.3pt,
	pattern radius/.store in=\mcRadius, 
	pattern radius = 1pt}
	\makeatletter
	\pgfutil@ifundefined{pgf@pattern@name@_vqzqwkgac}{
	\pgfdeclarepatternformonly[\mcThickness,\mcSize]{_vqzqwkgac}
	{\pgfqpoint{0pt}{0pt}}
	{\pgfpoint{\mcSize+\mcThickness}{\mcSize+\mcThickness}}
	{\pgfpoint{\mcSize}{\mcSize}}
	{
	\pgfsetcolor{\tikz@pattern@color}
	\pgfsetlinewidth{\mcThickness}
	\pgfpathmoveto{\pgfqpoint{0pt}{0pt}}
	\pgfpathlineto{\pgfpoint{\mcSize+\mcThickness}{\mcSize+\mcThickness}}
	\pgfusepath{stroke}
	}}
	\makeatother

	% Pattern Info
	 
	\tikzset{
	pattern size/.store in=\mcSize, 
	pattern size = 5pt,
	pattern thickness/.store in=\mcThickness, 
	pattern thickness = 0.3pt,
	pattern radius/.store in=\mcRadius, 
	pattern radius = 1pt}
	\makeatletter
	\pgfutil@ifundefined{pgf@pattern@name@_nrv1g1isg}{
	\pgfdeclarepatternformonly[\mcThickness,\mcSize]{_nrv1g1isg}
	{\pgfqpoint{0pt}{0pt}}
	{\pgfpoint{\mcSize+\mcThickness}{\mcSize+\mcThickness}}
	{\pgfpoint{\mcSize}{\mcSize}}
	{
	\pgfsetcolor{\tikz@pattern@color}
	\pgfsetlinewidth{\mcThickness}
	\pgfpathmoveto{\pgfqpoint{0pt}{0pt}}
	\pgfpathlineto{\pgfpoint{\mcSize+\mcThickness}{\mcSize+\mcThickness}}
	\pgfusepath{stroke}
	}}
	\makeatother
	\tikzset{every picture/.style={line width=0.75pt}} %set default line width to 0.75pt        

	\begin{tikzpicture}[x=0.75pt,y=0.75pt,yscale=-1,xscale=1]
	%uncomment if require: \path (0,369); %set diagram left start at 0, and has height of 369

	%Shape: Rectangle [id:dp6987487718206593] 
	\draw  [pattern=_vqzqwkgac,pattern size=6pt,pattern thickness=0.75pt,pattern radius=0pt, pattern color={rgb, 255:red, 222; green, 222; blue, 222}] (216,46.6) -- (234.5,46.6) -- (234.5,269.35) -- (216,269.35) -- cycle ;
	%Shape: Rectangle [id:dp03229727638294522] 
	\draw  [pattern=_nrv1g1isg,pattern size=6pt,pattern thickness=0.75pt,pattern radius=0pt, pattern color={rgb, 255:red, 222; green, 222; blue, 222}] (376,46.6) -- (394.5,46.6) -- (394.5,269.35) -- (376,269.35) -- cycle ;
	%Straight Lines [id:da8987019493735644] 
	\draw    (235,67.6) -- (372.67,67.6) ;
	\draw [shift={(375.67,67.6)}, rotate = 180] [fill={rgb, 255:red, 0; green, 0; blue, 0 }  ][line width=0.08]  [draw opacity=0] (10.72,-5.15) -- (0,0) -- (10.72,5.15) -- (7.12,0) -- cycle    ;
	%Straight Lines [id:da7199404640623852] 
	\draw    (235,97.6) -- (372.67,97.6) ;
	\draw [shift={(375.67,97.6)}, rotate = 180] [fill={rgb, 255:red, 0; green, 0; blue, 0 }  ][line width=0.08]  [draw opacity=0] (10.72,-5.15) -- (0,0) -- (10.72,5.15) -- (7.12,0) -- cycle    ;
	%Straight Lines [id:da9957718405450486] 
	\draw    (235,127.6) -- (372.67,127.6) ;
	\draw [shift={(375.67,127.6)}, rotate = 180] [fill={rgb, 255:red, 0; green, 0; blue, 0 }  ][line width=0.08]  [draw opacity=0] (10.72,-5.15) -- (0,0) -- (10.72,5.15) -- (7.12,0) -- cycle    ;
	%Straight Lines [id:da9804821897489686] 
	\draw    (235,157.6) -- (372.67,157.6) ;
	\draw [shift={(375.67,157.6)}, rotate = 180] [fill={rgb, 255:red, 0; green, 0; blue, 0 }  ][line width=0.08]  [draw opacity=0] (10.72,-5.15) -- (0,0) -- (10.72,5.15) -- (7.12,0) -- cycle    ;
	%Straight Lines [id:da5064027925429426] 
	\draw    (235,187.6) -- (372.67,187.6) ;
	\draw [shift={(375.67,187.6)}, rotate = 180] [fill={rgb, 255:red, 0; green, 0; blue, 0 }  ][line width=0.08]  [draw opacity=0] (10.72,-5.15) -- (0,0) -- (10.72,5.15) -- (7.12,0) -- cycle    ;
	%Straight Lines [id:da4067274419286031] 
	\draw    (235,217.6) -- (372.67,217.6) ;
	\draw [shift={(375.67,217.6)}, rotate = 180] [fill={rgb, 255:red, 0; green, 0; blue, 0 }  ][line width=0.08]  [draw opacity=0] (10.72,-5.15) -- (0,0) -- (10.72,5.15) -- (7.12,0) -- cycle    ;
	%Straight Lines [id:da9046841925696891] 
	\draw    (235,247.6) -- (372.67,247.6) ;
	\draw [shift={(375.67,247.6)}, rotate = 180] [fill={rgb, 255:red, 0; green, 0; blue, 0 }  ][line width=0.08]  [draw opacity=0] (10.72,-5.15) -- (0,0) -- (10.72,5.15) -- (7.12,0) -- cycle    ;
	%Curve Lines [id:da35438118374229366] 
	\draw    (376,46.6) .. controls (341.7,22.75) and (271.87,21) .. (236.61,45.09) ;
	\draw [shift={(234.5,46.6)}, rotate = 323.35] [fill={rgb, 255:red, 0; green, 0; blue, 0 }  ][line width=0.08]  [draw opacity=0] (10.72,-5.15) -- (0,0) -- (10.72,5.15) -- (7.12,0) -- cycle    ;
	%Straight Lines [id:da6301391204108429] 
	\draw    (235,287.6) -- (375.67,287.6) ;
	\draw [shift={(375.67,287.6)}, rotate = 180] [color={rgb, 255:red, 0; green, 0; blue, 0 }  ][line width=0.75]    (0,5.59) -- (0,-5.59)   ;
	\draw [shift={(235,287.6)}, rotate = 180] [color={rgb, 255:red, 0; green, 0; blue, 0 }  ][line width=0.75]    (0,5.59) -- (0,-5.59)   ;

	% Text Node
	\draw (305.66,299.2) node    {$d$};
	% Text Node
	\draw (313.66,15.2) node    {$\Delta V$};
	% Text Node
	\draw (384.66,33.2) node    {$-\sigma $};
	% Text Node
	\draw (221.66,33.2) node    {$+\sigma $};
	% Text Node
	\draw (199.66,149.2) node    {$A$};


	\end{tikzpicture}
\end{figure}
\FloatBarrier
Chiamiamo $A$ l'area delle due lastre. Se immaginiamo di avere una carica positiva $Q$ a sinistra, trascurando gli effetti di bordo, avremo una carica $-Q$ a destra. Calcoliamo la densità superficiale di carica $\sigma$ come rapporto: $ \sigma = \frac{Q}{A} $.
Se introduciamo la normale $ \vec{n}  $, sappiamo che il campo elettrico in prossimità del bordo lo possiamo scrivere come:
\[
	\vec{E} =\frac{\sigma}{\varepsilon_0}\vec{n} = \frac{Q}{\varepsilon_0 A} \vec{n}
\]
Tale condizione vale anche a destra. Quello che possiamo dire è che il campo elettrico è lo stesso ovunque, anche nello spazio circostante compreso tra le armature. Si fa infatti l'ipotesi di avere un campo elettrico uniforme.
Calcolando il potenziale si ha:
\[
	\Delta V = \int_{\gamma} \vec{E} \cdot d\vec{l} =\frac{Q\vec{n}}{\varepsilon_0 A} \underbrace{\left( \int_{\gamma}\vec{n} dl  \right)}_{\vec{n} d}= \frac{Qd}{\varepsilon_0 A}
\]
Si deduce quindi che la capacità di un condensatore piano è data da:
\[
	C = \frac{Q}{\Delta V} = \frac{Q}{\frac{Qd}{\varepsilon_0 A}} = \frac{\varepsilon_0 A}{d}
\]







































\section{Collegamento di condensatori}

Può apparire interessante studiare cosa accade quando colleghiamo fra di loro più condensatori. Tramite opportuni collegamenti conduttivi esterni è possibile far fluire la carica negativa da un'armatura all'altra, generando una corrente elettrica che scarica il condensatore. Esistono sostanzialmente due grandi famiglie di connessioni fra oggetti. La prima che andiamo a considerare è la connessione in parallelo.




















\subsection{Condensatori in parallelo}

Supponiamo di conoscere le capacità dei condensatori e che questi siano collegati fra due punti di cui chiamiamo $\Delta V$ la differenza di potenziale. La differenza di potenziale è la stessa ai capi di tutti i condensatori.
\begin{figure}[htpb]
	\centering
	

	\tikzset{every picture/.style={line width=0.75pt}} %set default line width to 0.75pt        

	\begin{tikzpicture}[x=0.75pt,y=0.75pt,yscale=-1,xscale=1]
	%uncomment if require: \path (0,300); %set diagram left start at 0, and has height of 300

	%Shape: Capacitor [id:dp558240700229865] 
	\draw   (168,109) -- (168,145) (188,153) -- (148,153) (188,145) -- (148,145) (168,153) -- (168,189) ;
	%Straight Lines [id:da8359633083468658] 
	\draw    (168,109) -- (368,109) ;
	%Shape: Capacitor [id:dp9010751755313184] 
	\draw   (368,109) -- (368,145) (388,153) -- (348,153) (388,145) -- (348,145) (368,153) -- (368,189) ;
	%Shape: Capacitor [id:dp5238033061656202] 
	\draw   (268,109) -- (268,145) (288,153) -- (248,153) (288,145) -- (248,145) (268,153) -- (268,189) ;
	%Straight Lines [id:da24800240212761704] 
	\draw    (168,189) -- (368,189) ;
	%Straight Lines [id:da4614347285040261] 
	\draw    (268,109) -- (268,76.5) ;
	%Straight Lines [id:da8785973966177691] 
	\draw    (268,221.5) -- (268,189) ;
	%Curve Lines [id:da026221795222580502] 
	\draw    (138,189) .. controls (129.23,164.14) and (125.2,141.65) .. (137.06,111.35) ;
	\draw [shift={(138,109)}, rotate = 472.43] [fill={rgb, 255:red, 0; green, 0; blue, 0 }  ][line width=0.08]  [draw opacity=0] (10.72,-5.15) -- (0,0) -- (10.72,5.15) -- (7.12,0) -- cycle    ;

	% Text Node
	\draw (112,148) node    {$\Delta V$};


	\end{tikzpicture}
\end{figure}
\FloatBarrier
Avremo determinate quantità di carica su ogni condensatore. Vorremmo legame la carica totale alla differenza di potenziale. Notiamo che essa può essere vista come:
\[
	Q_{\text{tot}}=\sum_{i=1}^n Q_i
\]
Possiamo in pratica riscrivere la le cariche come:
\[
	Q_1=C_1\Delta V \qquad Q_2=C_2\Delta V  \qquad Q_3=C_3\Delta V
\]
Questo sistema di condensatori si sta comportando come se fosse un unico condensatore tale per cui, applicata ai suoi cavi una differenza di potenziale $ \Delta V $, avrà una carica sulle armature pari a $Q_{\text{tot}}$. Chiameremo la capacità di questo condensatore $C_{\text{equivalente}}$. Essa per definizione è:
\[
	\boxed{C_{\text{equivalente}}^{\text{parallelo}} = \frac{Q_{\text{tot}}}{\Delta V} = \frac{\Delta V \sum_{i=1}^n C_i}{\Delta V} = \sum_{i=1}^n C_i}
\]




















\subsection{Collegamento in serie}

Un altro tipico modo di collegare i condensatori è la connessione in serie.
Supponiamo di applicare a questi condensatori una differenza di potenziale $\Delta V$.
\begin{figure}[htpb]
	\centering
	

	\tikzset{every picture/.style={line width=0.75pt}} %set default line width to 0.75pt        

	\begin{tikzpicture}[x=0.75pt,y=0.75pt,yscale=-1,xscale=1]
	%uncomment if require: \path (0,300); %set diagram left start at 0, and has height of 300

	%Shape: Capacitor [id:dp4523136331925126] 
	\draw   (112.51,159) -- (163.27,159) (174.55,139) -- (174.55,179) (163.27,139) -- (163.27,179) (174.55,159) -- (225.32,159) ;
	%Shape: Capacitor [id:dp26663032854279867] 
	\draw   (225.32,159) -- (276.08,159) (287.36,139) -- (287.36,179) (276.08,139) -- (276.08,179) (287.36,159) -- (338.12,159) ;
	%Shape: Capacitor [id:dp034889466412811654] 
	\draw   (338.12,159) -- (388.88,159) (400.16,139) -- (400.16,179) (388.88,139) -- (388.88,179) (400.16,159) -- (450.92,159) ;
	%Shape: Rectangle [id:dp34598128118367466] 
	\draw  [dash pattern={on 0.84pt off 2.51pt}] (169.63,117.6) -- (281,117.6) -- (281,200.4) -- (169.63,200.4) -- cycle ;
	%Shape: Circle [id:dp745967235012474] 
	\draw  [fill={rgb, 255:red, 0; green, 0; blue, 0 }  ,fill opacity=1 ] (110.42,159) .. controls (110.42,157.85) and (111.36,156.91) .. (112.51,156.91) .. controls (113.66,156.91) and (114.6,157.85) .. (114.6,159) .. controls (114.6,160.15) and (113.66,161.09) .. (112.51,161.09) .. controls (111.36,161.09) and (110.42,160.15) .. (110.42,159) -- cycle ;
	%Shape: Circle [id:dp1899769930971913] 
	\draw  [fill={rgb, 255:red, 0; green, 0; blue, 0 }  ,fill opacity=1 ] (223.23,159) .. controls (223.23,157.85) and (224.16,156.91) .. (225.32,156.91) .. controls (226.47,156.91) and (227.4,157.85) .. (227.4,159) .. controls (227.4,160.15) and (226.47,161.09) .. (225.32,161.09) .. controls (224.16,161.09) and (223.23,160.15) .. (223.23,159) -- cycle ;
	%Shape: Circle [id:dp39131917210538325] 
	\draw  [fill={rgb, 255:red, 0; green, 0; blue, 0 }  ,fill opacity=1 ] (336.03,159) .. controls (336.03,157.85) and (336.97,156.91) .. (338.12,156.91) .. controls (339.27,156.91) and (340.21,157.85) .. (340.21,159) .. controls (340.21,160.15) and (339.27,161.09) .. (338.12,161.09) .. controls (336.97,161.09) and (336.03,160.15) .. (336.03,159) -- cycle ;
	%Shape: Circle [id:dp7997296340536011] 
	\draw  [fill={rgb, 255:red, 0; green, 0; blue, 0 }  ,fill opacity=1 ] (448.84,159) .. controls (448.84,157.85) and (449.77,156.91) .. (450.92,156.91) .. controls (452.08,156.91) and (453.01,157.85) .. (453.01,159) .. controls (453.01,160.15) and (452.08,161.09) .. (450.92,161.09) .. controls (449.77,161.09) and (448.84,160.15) .. (448.84,159) -- cycle ;

	% Text Node
	\draw (108.99,172) node    {$A$};
	% Text Node
	\draw (226.02,173) node    {$P_{1}$};
	% Text Node
	\draw (338.82,173) node    {$P_{2}$};
	% Text Node
	\draw (451.63,172) node    {$B$};
	% Text Node
	\draw (150.02,133) node    {$Q$};
	% Text Node
	\draw (190.02,133) node    {$-Q$};
	% Text Node
	\draw (263.02,133) node    {$Q$};
	% Text Node
	\draw (303.02,133) node    {$-Q$};
	% Text Node
	\draw (376.02,133) node    {$Q$};
	% Text Node
	\draw (416.02,133) node    {$-Q$};


	\end{tikzpicture}
\end{figure}
\FloatBarrier
La parte del sistema racchiusa nella parte tratteggiata è un conduttore isolato. Quindi la carica totale su di esso, se inizialmente era zero, deve essere ancora tale.
Quindi per induzione ho la stessa carica $Q$ anche sul secondo condensatore. E via dicendo per tutti i condensatori.  Al di là del valore delle varie capacità, la carica presente sui vari condensatori è la stessa. Se inseriamo dei punti intermedi, possiamo attuare una operazione di questo tipo:
\begin{align*}
	\Delta V &= V_A-V_B \\
	&= V_A-V_{P_1}+V_{P_1} -V_{P_2}+V_{P_2} - V_B \\
	&= \Delta V_1 + \Delta V_2 + \Delta V_3
\end{align*}
Posso riscrivere $ \Delta V $ come
\[
	\Delta V = \sum_{i=1}^n \Delta V_i = \sum_{i=1}^n \frac{Q}{C_i} = Q \sum_{i=1}^n \frac{1}{C_i}
\]
Posso anche in questo caso introdurre una capacità equivalente per un condensatore equivalente, che può essere vista come:
\[
	\frac{\Delta V}{Q} = \boxed{\frac{1}{C_{\text{equivalente}}^{\text{serie}}} = \sum_{i=1}^n \frac{1}{C_i}}
\]
Possiamo a questo punto determinare la capacità equivalente di un circuito equivalente quanto complesso si voglia.
\emph{Notiamo che le connessioni in serie o in parallelo sono significative, dal punto di vista della capacità equivalente, solo se i valori delle capacità non sono molto diversi tra loro. In caso contrario, nel parallelo predomina la capacità maggiore, nella serie quella minore.}







































\section{Energia elettrostatica di un condensatore carico}

Quando consideriamo un sistema di cariche puntiformi, è necessario, come abbiamo visto, spendere energia. Questa energia viene detta \textbf{energia elettrostatica} e può essere scritta come:
\[
	U_e =\frac{1}{2} \sum_{i\neq j} \frac{q_iq_j}{4\pi \varepsilon_0 r_{ij}} = \frac{1}{2} \sum_{i\neq j} V_{ji}Q_i
\]
Dove ciascuno dei contributi della sommatoria rappresenta il lavoro necessario per portare la carica $ q_i  $ dall'infinito, dove il potenziale vale zero, a distanza $ r_{ij}  $. Poiché la sommatoria è simmetrica negli indici, cioè conta per ogni coppia sia il lavoro per portare $ q_i  $ nel campo di $ q_j  $ sia l'opposto, ci vuole il fattore $1/2$. Vorremmo generalizzare questo calcolo alla situazione in cui abbiamo, invece di una distribuzione di carica discreta, una distribuzione di carica continua. Per una generica distribuzione di carica lineare, superficiale, o di volume, si scrive rispettivamente:
\[
	U_e=\frac{1}{2} \int_{\Lambda}\lambda V dl \qquad U_e=\int_{\Sigma}\sigma V dS \qquad U_e=\frac{1}{2} \int_{\tau}\rho V d\tau
\]
Applichiamo questa generalizzazione al caso dei conduttori. Supponiamo di avere un conduttore carico positivamente e chiamiamo $ V_c  $ il potenziale sul conduttore. Vogliamo capire quanta energia si deve spendere per portare le cariche sul conduttore:
\[
	U_e = \frac{1}{2} \int_{\Sigma}\sigma V dS= \frac{1}{2} V_c \int_{\Sigma}\sigma dS = \frac{1}{2} V_c Q
\]
Di conseguenza
\[
	C=\frac{Q}{V} \implies U_e = \frac{1}{2} CV_c^2 = \frac{1}{2} \frac{Q^2}{C}
\]
Sapendo come calcolare l'energia per caricare un conduttore, possiamo anche calcolare l'energia per caricare un condensatore. Il processo di carica di un condensatore consiste infatti in una definitiva separazione di cariche e richiede un determinato lavoro che, essendo il campo conservativo, dipende solo dallo stato iniziale e dallo stato finale, ma non dalle modalità con cui avviene il processo. Il lavoro compiuto in questo processo viene effettuato contro la forza elettrostatica, che si oppone ad un accumulo di cariche dello stesso segno e viene immagazzinato nel sistema sotto forma di energia potenziale elettrostatica.
\begin{align*}
	U_e &= \frac{1}{2} Q_1V_1 + \frac{1}{2}  Q_2V_2 \\
	&= \frac{1}{2} Q_1V_1 - \frac{1}{2}  Q_1V_2 \\
	&= \frac{1}{2} Q_1 (V_1 - V_2) \\
	&= \frac{1}{2} Q \Delta V
\end{align*}
Possiamo scrivere $Q$ invece di $ Q_1  $ ricordando che la intendiamo positiva. Analogamente a prima si ha:
\[
	\boxed{U_e = \frac{1}{2} Q\Delta V = \frac{1}{2} C \Delta V^2 = \frac{1}{2} \frac{Q^2}{C}}
\]
Il condensatore viene spesso usato come metodo per conservare energia all'interno del sistema.
Alle stesse espressioni si arriva per l'energia elettrostatica di un conduttore carico isolato immaginando il processo di carica come un trasporto di carica dall'infinito, dove $V=0$, alla superficie del conduttore. Ciò torna formalmente con l'idea di considerare un conduttore isolato come un condensatore con un'armatura all'infinito.







































\section{Densità di energia elettrostatica}

Il ragionamento svolto per il calcolo dell'energia del condensatore lega l'energia alle cariche, che la possiedono in quanto si trovano ad un certo potenziale: l'energia totale è la somma delle energie potenziali delle singole cariche. È però possibile trovare un'espressione alternativa dell'energia, legata al campo prodotto dal sistema di cariche piuttosto che alle sorgenti del campo stesso.
Consideriamo una distribuzione di carica con una certa densità di carica di volume $\rho$ e poniamo di conoscere il potenziale in ogni punto dello spazio. L'energia elettrostatica si può calcolare come: $ U_e = \frac{1}{2} \int_{\tau}\rho Vd\tau$.
Il contributo all'integrale della regione esterna è nullo perché $ \rho  $ è pari a zero all'esterno di quella distribuzione. Riprendiamo la prima equazione di Maxwell secondo la quale: $\text{div}\vec{E} =\frac{\rho}{\varepsilon_0} \implies \rho = \varepsilon_0 \text{div}\vec{E} $.
Allora
\[
	U_e=\frac{1}{2} \int_{\tau}(\varepsilon_0 \text{div}\vec{E} ) Vd\tau = \frac{1}{2} \varepsilon_0\int_{\tau}V \text{div}\vec{E} d\tau
\]
Siano ora
\begin{itemize}
	\item $ \vec{v} =\vec{v} (x,y,z)  $ un campo vettoriale
	\item $ f=f(x,y,z)  $ un campo scalare
\end{itemize}
Allora
\[
	\text{div}[f\vec{v}] = \vec{\nabla} \cdot [f\vec{v}] = (\vec{\nabla} \cdot \vec{v} )f + \vec{v} \cdot \vec{\nabla} f = f\text{div}\vec{v} +\vec{v} \cdot \text{grad}f
\]
Quindi
\[
	f\text{div}\vec{v} = \text{div}[f\vec{v}] - \vec{v} \cdot \text{grad}f
\]
Ora considerando $V$ come il campo $ f $ ed $ \vec{E}  $ come il campo $ \vec{v}  $ si ha
\[
	 V\text{div}\vec{E} = \text{div}[V\vec{E}] - \vec{E} \cdot \text{grad}V = \text{div}[V\vec{E}] + E^2
\]
Sostituendo il risultato nella relazione di prima si ha:
\begin{align*}
	U_e &= \frac{1}{2} \varepsilon_0\int_{\tau}  \{\text{div}[V\vec{E}] + E^2 \}  d\tau \\
	&=  \frac{1}{2} \varepsilon_0\int_{\tau}  \text{div}[V\vec{E}] d\tau + \frac{1}{2} \varepsilon_0\int_{\tau} E^2  d\tau \\
	&= \frac{1}{2} \varepsilon_0 \underbrace{\int_{\Sigma}  V\vec{E}\cdot \vec{n} dS}_{\Phi_{\Sigma}(V\vec{E})} + \frac{1}{2} \varepsilon_0\int_{\tau} E^2  d\tau
\end{align*}
Dove l'ultimo passaggio è giustificato applicando il teorema della Divergenza. Ora consideriamo cosa accade per $ r \to \infty  $
\begin{gather*}
	V \propto  \frac{1}{r} \qquad
	|\vec{E} | \propto \frac{1}{r^2} \qquad
	A_{\Sigma} \propto r^2  \\
	\implies  V\vec{E} \propto \frac{1}{r^3} \implies \Phi_{\Sigma}(V\vec{E}) \propto \frac{1}{r} \xrightarrow{r\rightarrow \infty} 0
\end{gather*}
\begin{figure}[htpb]
	\centering
	

	\tikzset{every picture/.style={line width=0.75pt}} %set default line width to 0.75pt        

	\begin{tikzpicture}[x=0.75pt,y=0.75pt,yscale=-1,xscale=1]
	%uncomment if require: \path (0,300); %set diagram left start at 0, and has height of 300

	%Shape: Circle [id:dp619797634009047] 
	\draw   (195,148.75) .. controls (195,85.38) and (246.38,34) .. (309.75,34) .. controls (373.12,34) and (424.5,85.38) .. (424.5,148.75) .. controls (424.5,212.12) and (373.12,263.5) .. (309.75,263.5) .. controls (246.38,263.5) and (195,212.12) .. (195,148.75) -- cycle ;
	%Shape: Cloud [id:dp2524059018632465] 
	\draw   (270.37,134.35) .. controls (269.53,128.14) and (272.3,122) .. (277.51,118.52) .. controls (282.72,115.05) and (289.46,114.85) .. (294.86,118.02) .. controls (296.78,114.41) and (300.28,111.92) .. (304.31,111.29) .. controls (308.35,110.67) and (312.44,112) .. (315.34,114.86) .. controls (316.97,111.59) and (320.18,109.39) .. (323.81,109.05) .. controls (327.45,108.7) and (331.01,110.26) .. (333.22,113.17) .. controls (336.17,109.7) and (340.86,108.24) .. (345.26,109.42) .. controls (349.66,110.6) and (352.98,114.21) .. (353.79,118.68) .. controls (357.4,119.67) and (360.4,122.17) .. (362.03,125.55) .. controls (363.66,128.93) and (363.74,132.85) .. (362.27,136.29) .. controls (365.83,140.92) and (366.66,147.09) .. (364.46,152.5) .. controls (362.25,157.9) and (357.35,161.73) .. (351.57,162.56) .. controls (351.53,167.63) and (348.75,172.29) .. (344.3,174.73) .. controls (339.86,177.18) and (334.43,177.03) .. (330.13,174.34) .. controls (328.29,180.42) and (323.13,184.89) .. (316.87,185.82) .. controls (310.61,186.75) and (304.38,183.98) .. (300.86,178.71) .. controls (296.55,181.31) and (291.38,182.06) .. (286.51,180.78) .. controls (281.64,179.51) and (277.49,176.33) .. (274.99,171.95) .. controls (270.58,172.46) and (266.32,170.18) .. (264.32,166.23) .. controls (262.32,162.28) and (263.01,157.5) .. (266.04,154.27) .. controls (262.11,151.95) and (260.11,147.36) .. (261.07,142.88) .. controls (262.04,138.41) and (265.75,135.06) .. (270.28,134.59) ; \draw   (266.04,154.27) .. controls (267.9,155.36) and (270.04,155.86) .. (272.18,155.69)(274.99,171.95) .. controls (275.91,171.84) and (276.81,171.61) .. (277.67,171.27)(300.86,178.71) .. controls (300.21,177.73) and (299.67,176.69) .. (299.24,175.6)(330.13,174.34) .. controls (330.46,173.23) and (330.68,172.09) .. (330.77,170.94)(351.57,162.56) .. controls (351.61,157.16) and (348.55,152.21) .. (343.69,149.85)(362.27,136.29) .. controls (361.48,138.13) and (360.28,139.76) .. (358.76,141.06)(353.79,118.68) .. controls (353.92,119.43) and (353.98,120.18) .. (353.97,120.94)(333.22,113.17) .. controls (332.49,114.04) and (331.88,115) .. (331.43,116.04)(315.34,114.86) .. controls (314.95,115.65) and (314.66,116.48) .. (314.47,117.34)(294.86,118.02) .. controls (296,118.69) and (297.06,119.49) .. (298.01,120.42)(270.37,134.35) .. controls (270.49,135.21) and (270.67,136.05) .. (270.92,136.88) ;
	%Straight Lines [id:da48436721532501514] 
	\draw    (307.5,160) -- (227.82,225.1) ;
	\draw [shift={(225.5,227)}, rotate = 320.75] [fill={rgb, 255:red, 0; green, 0; blue, 0 }  ][line width=0.08]  [draw opacity=0] (10.72,-5.15) -- (0,0) -- (10.72,5.15) -- (7.12,0) -- cycle    ;

	% Text Node
	\draw (217,60) node    {$\Sigma $};
	% Text Node
	\draw (313,143) node    {$\rho ( x,y,z)$};
	% Text Node
	\draw (273,55) node    {$\tau $};
	% Text Node
	\draw (349,215) node    {$\rho =0$};
	% Text Node
	\draw (257,214) node    {$r$};


	\end{tikzpicture}
\end{figure}
\FloatBarrier
Quindi facendo tendere $ r \to \infty  $ l'espressione diventa
\[
	\boxed{U_e= \int_{\text{tutto lo spazio}}   \frac{1}{2} \varepsilon_0 E^2  d\tau}
\]
La quantità all'interno dell'integrale moltiplicata per il volume dà un'energia. Quindi la quantità:
\[
	\boxed{u_e = \frac{1}{2} \varepsilon_0 E^2}
\]
ha la dimensione di un volume e viene chiamata \textbf{densità di energia elettrostatica per unità di volume}.
Quando ho un campo elettrico, ad esso ad ogni punto dello spazio si associa una densità di energia. Per creare un campo elettrico, dobbiamo creare una distribuzione e per farla si deve spendere del lavoro. Ecco perché ad $\vec{E}$ si associa un'energia per unità di volume, punto per punto. Questo non vale solo nel caso statico, anche le onde elettromagnetiche, composte da un campo elettrico e uno magnetico oscillanti, sono caratterizzate da una certa densità di energia. Notiamo subito la generalità di questa formula, in cui non compare alcun elemento caratteristico del sistema per cui il calcolo è stato eseguito, ma soltanto il valore del campo e una proprietà del mezzo (in questo caso il vuoto).

\emph{Esempio: energia elettrostatica di un conduttore.} Consideriamo un conduttore sferico di raggio $R$. Se volessi calcolare l'energia elettrostatica di tale componente possiamo utilizzare due approcci differenti:
\begin{itemize}
	\item Utilizzare la formula: $ U_e=\frac{1}{2} QV_c = \frac{1}{2} Q \frac{Q}{4\pi \varepsilon_0 R} = \frac{Q^2}{8\pi \varepsilon_0 R} $
	\item Calcolare la densità di energia elettrostatica e integrarla su tutto lo spazio. È possibile attuare questo metodo nel momento in cui siamo a conoscenza del campo elettrico presente in ogni punto dello spazio.
	\[
		\vec{E} = \left\{ \begin{array}{ll}
		 	0 & r<R \\
			\frac{Q}{4\pi \varepsilon_0 r^2} & r>R
		\end{array} \right. \qquad U_e = \int_{\tau}\frac{1}{2} \varepsilon_0 E^2 d\tau
	\]
	Per calcolare tale integrale, immaginiamo che il volume infinitesimo $ d\tau  $ possa essere visto come un guscio sferico che circonda il nostro conduttore. Il volume che corrisponde a questo elemento si può calcolare come superficie del guscio moltiplicata per lo spessore: $ d\tau =4\pi r^2 dr $. E l'energia elettrostatica sarà data da:
	\begin{align*}
		U_e&=\int_R^{\infty} \frac{1}{2} \varepsilon_0 \frac{Q^2}{(4\pi \varepsilon_0 r^2 )^2} \overbrace{4\pi r^2 dr}^{d\tau} = \int_R^{\infty} \frac{\varepsilon_0 Q^2}{8\pi \varepsilon_0^2 r^2} dr \\
		&= \frac{Q^2}{8\pi \varepsilon_0} \int_R^{\infty} \frac{dr}{r^2} = \frac{Q^2}{8\pi \varepsilon_0}\left[ -\frac{1}{r} \right]_R^{\infty} = \frac{Q^2}{8\pi \varepsilon_0 R}
	\end{align*}
\end{itemize}

\emph{Esempio: energia elettrostatica di un condensatore piano.} Immaginiamo un condensatore piano in cui trascuriamo gli effetti di bordo.
Avevamo già calcolato la capacità del condensatore come:
\[
	C=\frac{\varepsilon_0 S}{d} \qquad U_e=\frac{1}{2} \frac{Q^2}{C} = \frac{Q^2d}{2\varepsilon_0 S}
\]
Per il teorema di Coulomb sappiamo inoltre che:
\[
	\vec{E} = \frac{\sigma}{\varepsilon_0} \vec{u}_n = \frac{Q}{\varepsilon_0 S}\vec{n}
\]
Calcoliamo l'energia elettrostatica come l'integrale della densità di energia. Il campo elettrico al di fuori del condensatore è nullo perché stiamo trascurando gli effetti di bordo.
\begin{align*}
	U_e &= \int_{\text{spazio}}\frac{1}{2} \varepsilon_0 E^2 d\tau = \int_{\text{condensatore}} \frac{1}{2} \varepsilon_0 E^2 d\tau \\
	&= \frac{1}{2} \varepsilon_0 E^2 \int_{\text{condensatore}}d\tau = \frac{1}{2} \varepsilon_0 E^2 Sd \\
	&=\frac{1}{2} \varepsilon_0 \left( \frac{Q}{\varepsilon_0 S} \right)^2 Sd = \frac{Q^2 d}{2\varepsilon_0 S}
\end{align*}
E vediamo che otteniamo la stessa energia elettrostatica ottenuta con l'altra modalità di calcolo:
\[
	U_e = \frac{1}{2} Q V_c
\]







































\section{Forza tra le armature di un condensatore: pressione elettrostatica}

\subsection{Caso I: Carica costante}

Consideriamo ancora un condensatore piano. Tra le armature, cariche di segno opposto, si esercita una forza $\vec{F}$ attrattiva che, per ragioni di simmetria, è parallela al campo, cioè ortogonale alle armature. Supponiamo di mantenere fissa l'armatura negativa e di lasciare avvicinare, sotto l'azione della forza, l'armatura positiva di una quantità $dx$.
\begin{figure}[htpb]
	\centering
	

	% Pattern Info
	 
	\tikzset{
	pattern size/.store in=\mcSize, 
	pattern size = 5pt,
	pattern thickness/.store in=\mcThickness, 
	pattern thickness = 0.3pt,
	pattern radius/.store in=\mcRadius, 
	pattern radius = 1pt}
	\makeatletter
	\pgfutil@ifundefined{pgf@pattern@name@_ssq9nr2n6}{
	\pgfdeclarepatternformonly[\mcThickness,\mcSize]{_ssq9nr2n6}
	{\pgfqpoint{0pt}{0pt}}
	{\pgfpoint{\mcSize+\mcThickness}{\mcSize+\mcThickness}}
	{\pgfpoint{\mcSize}{\mcSize}}
	{
	\pgfsetcolor{\tikz@pattern@color}
	\pgfsetlinewidth{\mcThickness}
	\pgfpathmoveto{\pgfqpoint{0pt}{0pt}}
	\pgfpathlineto{\pgfpoint{\mcSize+\mcThickness}{\mcSize+\mcThickness}}
	\pgfusepath{stroke}
	}}
	\makeatother

	% Pattern Info
	 
	\tikzset{
	pattern size/.store in=\mcSize, 
	pattern size = 5pt,
	pattern thickness/.store in=\mcThickness, 
	pattern thickness = 0.3pt,
	pattern radius/.store in=\mcRadius, 
	pattern radius = 1pt}
	\makeatletter
	\pgfutil@ifundefined{pgf@pattern@name@_gqzuhdnmb}{
	\pgfdeclarepatternformonly[\mcThickness,\mcSize]{_gqzuhdnmb}
	{\pgfqpoint{0pt}{0pt}}
	{\pgfpoint{\mcSize+\mcThickness}{\mcSize+\mcThickness}}
	{\pgfpoint{\mcSize}{\mcSize}}
	{
	\pgfsetcolor{\tikz@pattern@color}
	\pgfsetlinewidth{\mcThickness}
	\pgfpathmoveto{\pgfqpoint{0pt}{0pt}}
	\pgfpathlineto{\pgfpoint{\mcSize+\mcThickness}{\mcSize+\mcThickness}}
	\pgfusepath{stroke}
	}}
	\makeatother
	\tikzset{every picture/.style={line width=0.75pt}} %set default line width to 0.75pt        

	\begin{tikzpicture}[x=0.75pt,y=0.75pt,yscale=-1,xscale=1]
	%uncomment if require: \path (0,378); %set diagram left start at 0, and has height of 378

	%Shape: Rectangle [id:dp498484159010417] 
	\draw  [pattern=_ssq9nr2n6,pattern size=6pt,pattern thickness=0.75pt,pattern radius=0pt, pattern color={rgb, 255:red, 222; green, 222; blue, 222}] (376,51) -- (394.5,51) -- (394.5,273.75) -- (376,273.75) -- cycle ;
	%Shape: Rectangle [id:dp6143944971864774] 
	\draw  [pattern=_gqzuhdnmb,pattern size=6pt,pattern thickness=0.75pt,pattern radius=0pt, pattern color={rgb, 255:red, 222; green, 222; blue, 222}] (216,51) -- (234.5,51) -- (234.5,273.75) -- (216,273.75) -- cycle ;
	%Straight Lines [id:da24065184641167936] 
	\draw    (162,291) -- (455.5,291) ;
	\draw [shift={(458.5,291)}, rotate = 180] [fill={rgb, 255:red, 0; green, 0; blue, 0 }  ][line width=0.08]  [draw opacity=0] (10.72,-5.15) -- (0,0) -- (10.72,5.15) -- (7.12,0) -- cycle    ;
	%Straight Lines [id:da7699557261492547] 
	\draw    (225,284.25) -- (225,297.5) ;
	%Straight Lines [id:da07913726278772404] 
	\draw    (385,284.25) -- (385,297.5) ;
	%Straight Lines [id:da09060319574415887] 
	\draw    (376.5,157) -- (305,157) ;
	\draw [shift={(302,157)}, rotate = 360] [fill={rgb, 255:red, 0; green, 0; blue, 0 }  ][line width=0.08]  [draw opacity=0] (10.72,-5.15) -- (0,0) -- (10.72,5.15) -- (7.12,0) -- cycle    ;

	% Text Node
	\draw (386.06,61) node    {$+$};
	% Text Node
	\draw (386.06,81) node    {$+$};
	% Text Node
	\draw (386.06,101) node    {$+$};
	% Text Node
	\draw (386.06,121) node    {$+$};
	% Text Node
	\draw (386.06,141) node    {$+$};
	% Text Node
	\draw (386.06,161) node    {$+$};
	% Text Node
	\draw (386.06,181) node    {$+$};
	% Text Node
	\draw (386.06,201) node    {$+$};
	% Text Node
	\draw (386.06,221) node    {$+$};
	% Text Node
	\draw (386.06,241) node    {$+$};
	% Text Node
	\draw (386.06,261) node    {$+$};
	% Text Node
	\draw (225.06,61) node    {$-$};
	% Text Node
	\draw (225.06,81) node    {$-$};
	% Text Node
	\draw (225.06,101) node    {$-$};
	% Text Node
	\draw (225.06,121) node    {$-$};
	% Text Node
	\draw (225.06,141) node    {$-$};
	% Text Node
	\draw (225.06,161) node    {$-$};
	% Text Node
	\draw (225.06,181) node    {$-$};
	% Text Node
	\draw (225.06,201) node    {$-$};
	% Text Node
	\draw (225.06,221) node    {$-$};
	% Text Node
	\draw (225.06,241) node    {$-$};
	% Text Node
	\draw (225.06,261) node    {$-$};
	% Text Node
	\draw (385,306.5) node    {$x$};
	% Text Node
	\draw (225,306.5) node    {$O$};
	% Text Node
	\draw (336.56,137.5) node    {$\vec{F}$};
	% Text Node
	\draw (385.06,35) node    {$+Q$};
	% Text Node
	\draw (225.06,35) node    {$-Q$};


	\end{tikzpicture}
\end{figure}
\FloatBarrier
Non potrei calcolare tale forza $\vec{F}$ semplicemente come $\vec{F}=\vec{E}q$ perché il campo elettrico è la somma di quello dato da due cariche: quelle sull'armatura positiva e quelle sull'armatura negativa. Quando calcoliamo la forza elettrica sulla carica, dobbiamo considerare il campo generato da altre cariche, non da se stessa. Si procede quindi come segue.
Supponiamo che l'armatura positiva si avvicini all'armatura negativa
di un tratto infinitesimo. La forza $\vec{F}$ compirà un lavoro infinitesimo dato da:
\[
	d\mathcal{L} =\vec{F} \cdot d\vec{r} = F_x dx
\]
Quando abbiamo un sistema che non scambia energia ma può compiere lavoro, la variazione di energia interna dovrà essere compensata da una stessa variazione energetica di segno opposto da un'altra parte del sistema. La variazione di energia elettrostatica del condensatore è l'opposto del lavoro compiuto dalle forze elettrostatiche. Quindi scriviamo il bilancio energetico.
\[
	\boxed{dU_e = -d\mathcal{L}} = - F_xdx  \implies F_x = - \frac{\partial U_e}{\partial x} = - \frac{Q^2}{2\varepsilon_0 S}
\]
Otteniamo così che la derivata dell'energia elettrostatica rispetto alla distanza ci da automaticamente la forza agente. Nel caso tridimensionale avremo per le altre componenti forze analoghe, e si otterrà che:
\[
	\boxed{\vec{F} = - \vec{\nabla} U_e}
\]
Se ci calcoliamo il modulo della forza e dividiamo a destra e a sinistra per la superficie $S$:
\[
	|\vec{F}| = |F_x| = \frac{Q^2}{2\varepsilon_0 S} \implies \frac{|\vec{F}|}{S} = \frac{Q^2}{2\varepsilon_0 S^2} = \frac{1}{2\varepsilon_0} \left( \frac{Q}{S} \right)^2 = \frac{\sigma^2}{2\varepsilon_0}
\]
Se una forza $\vec{F}$ elettrostatica agisce su una superficie $S$, il rapporto fra l'area $S$ e la forza che vi agisce viene chiamato \textbf{pressione elettrostatica}.
\[
	\boxed{p_e = \frac{\sigma^2}{2\varepsilon_0}}
\]
Tale risultato può essere dimostrato anche considerando un elemento di carica $dq=\sigma dS$ sulla superficie di un conduttore sferico.
\begin{figure}[htpb]
	\centering
	

	\tikzset{every picture/.style={line width=0.75pt}} %set default line width to 0.75pt        

	\begin{tikzpicture}[x=0.75pt,y=0.75pt,yscale=-0.8,xscale=0.8]
	%uncomment if require: \path (0,300); %set diagram left start at 0, and has height of 300

	%Straight Lines [id:da7991625339498252] 
	\draw    (272,159) -- (320.75,159) ;
	\draw [shift={(269,159)}, rotate = 0] [fill={rgb, 255:red, 0; green, 0; blue, 0 }  ][line width=0.08]  [draw opacity=0] (10.72,-5.15) -- (0,0) -- (10.72,5.15) -- (7.12,0) -- cycle    ;
	%Shape: Circle [id:dp4399052341626799] 
	\draw   (29,154.75) .. controls (29,104.08) and (70.08,63) .. (120.75,63) .. controls (171.42,63) and (212.5,104.08) .. (212.5,154.75) .. controls (212.5,205.42) and (171.42,246.5) .. (120.75,246.5) .. controls (70.08,246.5) and (29,205.42) .. (29,154.75) -- cycle ;
	%Shape: Ellipse [id:dp7162737382390345] 
	\draw  [fill={rgb, 255:red, 222; green, 222; blue, 222 }  ,fill opacity=1 ] (176.85,105.38) .. controls (170.06,96.67) and (167.64,87.2) .. (171.45,84.23) .. controls (175.26,81.26) and (183.85,85.91) .. (190.65,94.62) .. controls (197.44,103.33) and (199.86,112.8) .. (196.05,115.77) .. controls (192.24,118.74) and (183.65,114.09) .. (176.85,105.38) -- cycle ;
	%Straight Lines [id:da0023035086983758113] 
	\draw    (181.75,95) -- (231.21,52.94) ;
	\draw [shift={(233.5,51)}, rotate = 499.63] [fill={rgb, 255:red, 0; green, 0; blue, 0 }  ][line width=0.08]  [draw opacity=0] (10.72,-5.15) -- (0,0) -- (10.72,5.15) -- (7.12,0) -- cycle    ;
	%Straight Lines [id:da6515601487889628] 
	\draw    (187.75,103) -- (217.22,77.94) ;
	\draw [shift={(219.51,76)}, rotate = 499.63] [fill={rgb, 255:red, 0; green, 0; blue, 0 }  ][line width=0.08]  [draw opacity=0] (10.72,-5.15) -- (0,0) -- (10.72,5.15) -- (7.12,0) -- cycle    ;
	%Shape: Ellipse [id:dp01894524455581892] 
	\draw  [fill={rgb, 255:red, 222; green, 222; blue, 222 }  ,fill opacity=1 ] (312,159) .. controls (312,147.95) and (315.92,139) .. (320.75,139) .. controls (325.58,139) and (329.5,147.95) .. (329.5,159) .. controls (329.5,170.05) and (325.58,179) .. (320.75,179) .. controls (315.92,179) and (312,170.05) .. (312,159) -- cycle ;
	%Straight Lines [id:da6725260589505533] 
	\draw    (320.75,159) -- (369.5,159) ;
	\draw [shift={(372.5,159)}, rotate = 180] [fill={rgb, 255:red, 0; green, 0; blue, 0 }  ][line width=0.08]  [draw opacity=0] (10.72,-5.15) -- (0,0) -- (10.72,5.15) -- (7.12,0) -- cycle    ;
	%Curve Lines [id:da587494170370523] 
	\draw    (479.5,77) .. controls (503.5,106) and (522.5,161) .. (504.5,235) ;

	% Text Node
	\draw (223,97) node    {$\vec{n}$};
	% Text Node
	\draw (244,49) node    {$\vec{E}$};
	% Text Node
	\draw (165,114) node    {$dS$};
	% Text Node
	\draw (364,136) node    {$\vec{E}_{1}$};
	% Text Node
	\draw (284,136) node    {$\vec{E}_{2}$};
	% Text Node
	\draw (564,150) node    {$\vec{E} =\overrightarrow{E'} +\vec{E}_{1}$};
	% Text Node
	\draw (464,150) node    {$\vec{E} =\overrightarrow{E'} +\vec{E}_{2}$};


	\end{tikzpicture}
\end{figure}
\FloatBarrier
Supponiamo di considerare una porzione $dS$ del conduttore e di conoscere la densità di carica superficiale $ \sigma  $. Il campo elettrico complessivo nelle immediate vicinanze di $dq$ all'esterno del conduttore, per il teorema di Coulomb vale:
\[
	\vec{E} = \frac{\sigma}{\varepsilon_0} \vec{n}
\]
Invece all'interno il campo è nullo. D'altro canto lo strato di carica $dq$ sulla superficie $dS$ genera nei punti immediatamente vicini il campo elettrico, rispettivamente all'interno e all'esterno:
\[
	\vec{E}_2 = - \frac{\sigma}{2\varepsilon_0} \vec{n} \qquad \vec{E}_1=\frac{\sigma}{2\varepsilon_0}\vec{n}
\]
Il campo complessivo $\vec{E}$ è dato dalla somma dei contributi dei campi elettrici generati da tutte le altre cariche sulla sfera, più quello generato da $dq$:
\begin{align*}
	\vec{E}' + \vec{E}_2 &= \vec{E}' - \frac{\sigma}{2\varepsilon_0}\vec{n} = 0 \\
	\vec{E}' + \vec{E}_1 &= \vec{E}' + \frac{\sigma}{2\varepsilon_0}\vec{n} = \frac{\sigma}{\varepsilon_0} \vec{n}
\end{align*}
Da queste relazioni segue immediatamente che il campo $\vec{E}'$ generato da tutte le altre cariche nella zona in cui si trova la carica $q$ è:
\[
	\vec{E}'=\frac{\sigma}{2\varepsilon_0} \vec{n}
\]
E questo è il campo che dobbiamo considerare per calcolare la forza subita da $dq$:
\[
	d\vec{F} = dq\,\vec{E}' = \sigma dS\,\vec{E}' = \frac{\sigma^2}{2\varepsilon_0}dS\vec{n}
\]
Diretta sempre verso l'esterno qualunque sia il segno della carica. A questa forza corrisponde la pressione:
\[
	p_e = \frac{d\vec{F} \cdot \vec{n}}{dS} = \frac{\sigma^2}{2\varepsilon_0}
\]
\textbf{Osservazione.} Ricordando quanto detto in precedenza sul potere delle punte, sappiamo che in prossimità di superfici di questo tipo (appuntite) la densità di carica è molto più elevata. Abbiamo ora gli strumenti per dire che ivi anche la pressione elettrostatica sarà di intensità notevole, talmente forte che alcune cariche saranno in grado di abbandonare il conduttore.




















\subsection{Caso II: Potenziale costante}

Invece che a carica costante il processo di spostamento dell'armatura del condensatore può avvenire a potenziale costante. Si connettono le armature ai poli di un generatore di tensione che è un dispositivo in grado di mantenere costante la d.d.p. tra due conduttori ad esso collegati. Vorremmo calcolare la forza $\vec{F}$ con cui l'armatura negativa viene attratta verso l'armatura positiva.
\begin{figure}[htpb]
	\centering
	

	% Pattern Info
	 
	\tikzset{
	pattern size/.store in=\mcSize, 
	pattern size = 5pt,
	pattern thickness/.store in=\mcThickness, 
	pattern thickness = 0.3pt,
	pattern radius/.store in=\mcRadius, 
	pattern radius = 1pt}
	\makeatletter
	\pgfutil@ifundefined{pgf@pattern@name@_vz54ibxs5}{
	\pgfdeclarepatternformonly[\mcThickness,\mcSize]{_vz54ibxs5}
	{\pgfqpoint{0pt}{0pt}}
	{\pgfpoint{\mcSize+\mcThickness}{\mcSize+\mcThickness}}
	{\pgfpoint{\mcSize}{\mcSize}}
	{
	\pgfsetcolor{\tikz@pattern@color}
	\pgfsetlinewidth{\mcThickness}
	\pgfpathmoveto{\pgfqpoint{0pt}{0pt}}
	\pgfpathlineto{\pgfpoint{\mcSize+\mcThickness}{\mcSize+\mcThickness}}
	\pgfusepath{stroke}
	}}
	\makeatother

	% Pattern Info
	 
	\tikzset{
	pattern size/.store in=\mcSize, 
	pattern size = 5pt,
	pattern thickness/.store in=\mcThickness, 
	pattern thickness = 0.3pt,
	pattern radius/.store in=\mcRadius, 
	pattern radius = 1pt}
	\makeatletter
	\pgfutil@ifundefined{pgf@pattern@name@_0l43vzxrp}{
	\pgfdeclarepatternformonly[\mcThickness,\mcSize]{_0l43vzxrp}
	{\pgfqpoint{0pt}{0pt}}
	{\pgfpoint{\mcSize+\mcThickness}{\mcSize+\mcThickness}}
	{\pgfpoint{\mcSize}{\mcSize}}
	{
	\pgfsetcolor{\tikz@pattern@color}
	\pgfsetlinewidth{\mcThickness}
	\pgfpathmoveto{\pgfqpoint{0pt}{0pt}}
	\pgfpathlineto{\pgfpoint{\mcSize+\mcThickness}{\mcSize+\mcThickness}}
	\pgfusepath{stroke}
	}}
	\makeatother
	\tikzset{every picture/.style={line width=0.75pt}} %set default line width to 0.75pt        

	\begin{tikzpicture}[x=0.75pt,y=0.75pt,yscale=-1,xscale=1]
	%uncomment if require: \path (0,385); %set diagram left start at 0, and has height of 385

	%Shape: Rectangle [id:dp28459746051411217] 
	\draw  [pattern=_vz54ibxs5,pattern size=6pt,pattern thickness=0.75pt,pattern radius=0pt, pattern color={rgb, 255:red, 222; green, 222; blue, 222}] (395,76.5) -- (413.5,76.5) -- (413.5,299.25) -- (395,299.25) -- cycle ;
	%Shape: Rectangle [id:dp768785510835789] 
	\draw  [pattern=_0l43vzxrp,pattern size=6pt,pattern thickness=0.75pt,pattern radius=0pt, pattern color={rgb, 255:red, 222; green, 222; blue, 222}] (235,76.5) -- (253.5,76.5) -- (253.5,299.25) -- (235,299.25) -- cycle ;
	%Straight Lines [id:da25028214418576145] 
	\draw    (181,316.5) -- (474.5,316.5) ;
	\draw [shift={(477.5,316.5)}, rotate = 180] [fill={rgb, 255:red, 0; green, 0; blue, 0 }  ][line width=0.08]  [draw opacity=0] (10.72,-5.15) -- (0,0) -- (10.72,5.15) -- (7.12,0) -- cycle    ;
	%Straight Lines [id:da0685560206511322] 
	\draw    (244,309.75) -- (244,323) ;
	%Straight Lines [id:da3208800571656274] 
	\draw    (404,309.75) -- (404,323) ;
	%Straight Lines [id:da6933873827368091] 
	\draw    (395.5,182.5) -- (324,182.5) ;
	\draw [shift={(321,182.5)}, rotate = 360] [fill={rgb, 255:red, 0; green, 0; blue, 0 }  ][line width=0.08]  [draw opacity=0] (10.72,-5.15) -- (0,0) -- (10.72,5.15) -- (7.12,0) -- cycle    ;
	%Shape: Battery [id:dp9851220206275462] 
	\draw  [fill={rgb, 255:red, 0; green, 0; blue, 0 }  ,fill opacity=1 ] (245.5,43.25) -- (317.05,43.25) (332.95,13.25) -- (332.95,73.25) (332.95,43.25) -- (404.5,43.25) (310.69,28.25) -- (317.05,28.25) -- (317.05,58.25) -- (310.69,58.25) -- (310.69,28.25) -- cycle ;
	%Straight Lines [id:da9230478693041511] 
	\draw    (245.5,43.25) -- (245.5,76.5) ;
	%Straight Lines [id:da40213787812146173] 
	\draw    (404.5,43.25) -- (404.5,76.5) ;
	%Curve Lines [id:da8689710937205839] 
	\draw    (273.5,62.5) .. controls (303.39,88.23) and (340.48,91.14) .. (371.6,64.19) ;
	\draw [shift={(373.5,62.5)}, rotate = 497.61] [fill={rgb, 255:red, 0; green, 0; blue, 0 }  ][line width=0.08]  [draw opacity=0] (10.72,-5.15) -- (0,0) -- (10.72,5.15) -- (7.12,0) -- cycle    ;

	% Text Node
	\draw (405.06,86.5) node    {$+$};
	% Text Node
	\draw (405.06,106.5) node    {$+$};
	% Text Node
	\draw (405.06,126.5) node    {$+$};
	% Text Node
	\draw (405.06,146.5) node    {$+$};
	% Text Node
	\draw (405.06,166.5) node    {$+$};
	% Text Node
	\draw (405.06,186.5) node    {$+$};
	% Text Node
	\draw (405.06,206.5) node    {$+$};
	% Text Node
	\draw (405.06,226.5) node    {$+$};
	% Text Node
	\draw (405.06,246.5) node    {$+$};
	% Text Node
	\draw (405.06,266.5) node    {$+$};
	% Text Node
	\draw (405.06,286.5) node    {$+$};
	% Text Node
	\draw (244.06,86.5) node    {$-$};
	% Text Node
	\draw (244.06,106.5) node    {$-$};
	% Text Node
	\draw (244.06,126.5) node    {$-$};
	% Text Node
	\draw (244.06,146.5) node    {$-$};
	% Text Node
	\draw (244.06,166.5) node    {$-$};
	% Text Node
	\draw (244.06,186.5) node    {$-$};
	% Text Node
	\draw (244.06,206.5) node    {$-$};
	% Text Node
	\draw (244.06,226.5) node    {$-$};
	% Text Node
	\draw (244.06,246.5) node    {$-$};
	% Text Node
	\draw (244.06,266.5) node    {$-$};
	% Text Node
	\draw (244.06,286.5) node    {$-$};
	% Text Node
	\draw (404,332) node    {$x$};
	% Text Node
	\draw (244,332) node    {$O$};
	% Text Node
	\draw (355.56,163) node    {$\vec{F}$};
	% Text Node
	\draw (325.06,96) node    {$\Delta V$};


	\end{tikzpicture}
\end{figure}
\FloatBarrier
La formula vista prima: $U_e = \frac{Q^2 d}{2\varepsilon_0 S}$ vale solo nel momento in cui il condensatore è isolato, non collegato con un generatore.
\[
	C=\frac{\varepsilon_0 S}{x} \implies Q=C\Delta V=\frac{\varepsilon_0 S \Delta V}{x}
\]
Il lavoro compiuto deve essere fatto a scapito di qualche energia interna. Ma il nostro sistema comprende questa volta anche il generatore.
Abbiamo quindi due forme di energia interna: l'energia potenziale del generatore e l'energia elettrostatica accumulata nel condensatore. Andremo a scrivere che il lavoro della forza $\vec{F}$ sarà uguale alla somma di due variazioni energetiche.
\[
	\boxed{d\mathcal{L} = -d(U_e+U_g)} = -dU_e - dU_g
\]
Osserviamo che il generatore mentre l'armatura si sposta deve compiere del lavoro. Infatti quando $x$ diminuisce la capacita aumenta e se ciò accade la carica sulle armature del condensatore aumenta anch'essa in modulo. Vediamo come questa vada ad aumentare:
\[
	dQ=\frac{dQ}{dx}dx = \frac{d(C\Delta V)}{dx} dx = \frac{d}{dx} \left( \frac{\varepsilon_0 S\Delta V}{x} \right) dx = - \frac{\varepsilon_0 S\Delta V}{x^2} dx
\]
Se la carica aumenta significa che il condensatore la sposterà da un'armatura all'altra, facendole superare una certa differenza di potenziale e compiendo quindi un certo lavoro pari a:
\begin{gather*}
	d\mathcal{L}_g=dQ\cdot \Delta V =-\frac{\varepsilon_0 S\Delta V}{x^2}dx \cdot \Delta V = - \frac{\varepsilon_0 S\Delta V^2}{x^2} dx \\
	dU_g = -d\mathcal{L}_g = \frac{\varepsilon_0 S\Delta V^2}{x^2}dx
\end{gather*}
E la sola variazione di energia potenziale del generatore sarà data dall'opposto di tale lavoro compiuto. Sappiamo inoltre che l'energia potenziale elettrica del condensatore è data da:
\begin{gather*}
	U_e = \frac{1}{2} C\Delta V^2 = \frac{\varepsilon_0 S \Delta V^2}{2x} \\
	dU_e = \frac{dU_e}{dx} dx= - \frac{\varepsilon_0 S \Delta V^2}{2 x^2} \implies dU_g = -2dU_e
\end{gather*}
L'energia interna del generatore diminuisce, la metà dell'energia così resasi disponibile la ritroviamo sottoforma di energia elettrostatica, l'altra metà deve corrispondere al lavoro per avvicinare le armature.
Sostituendo questi risultati nell'espressione iniziale si ha:
\begin{align*}
	d\mathcal{L} &= -dU_e - dU_g \\
	d\mathcal{L} &= -dU_e - (-2dU_e) \\
	d\mathcal{L} &= -dU_e + 2dU_e \\
	\vec{F} \cdot d\vec{r} &= dU_e \\
	F_xdx &= dU_e \\
	F_x &= \frac{dU_e}{dx}
\end{align*}
Concludiamo osservando che spesso, anche per sistemi non complessi, il calcolo delle forze appare in generale più semplice se affrontato partendo dall'energia elettrostatica del sistema. Il calcolo diretto sulla base della legge di Coulomb richiede, anche nei casi più banali, conti laboriosi. Nota l'energia elettrostatica, la componente della forza nella direzione $x$ si ottiene derivando l'energia elettrostatica rispetto ad $x$. Bisogna avere l'avvertenza di verificare se il processo si svolge a carica costante o a potenziale costante.
\[
	\boxed{\begin{array}{rl}
		 	q = \text{costante} & F_x = - \frac{dU_e}{dx} \\ \\
			V = \text{costante} & F_x = + \frac{dU_e}{dx} \\
		\end{array}}
\]







































\section{Dielettrici in equilibrio elettrostatico}

Abbiamo distinto i materiali in due grandi categorie in base alle proprietà elettrostatiche:
\begin{itemize}
	\item I conduttori, la cui carica può essere facilmente cambiata, dando origine a una nuova situazione elettrostatica. Le cariche infatti vengono facilmente sottratte o cedute attraverso il contatto con altri corpi carichi o il collegamento con generatori.
	\item Gli isolanti, in cui gli elettroni sono fortemente legati al loro atomo di appartenenza e non possono spostarsi.
\end{itemize}
I fenomeni che avvengono nei materiali dielettrici sono in realtà molto complessi. Cominciamo andando a studiare come viene modificato il campo elettrostatico nello spazio tra conduttori carichi quando questo viene riempito con un materiale isolante. Abbiamo visto che, considerato un condensatore piano con le due armature di area $A$ fra di loro distanti una quantità d e poste nel vuoto, se lo carichiamo di una carica $Q$, abbiamo una certa differenza di potenziale $\Delta V_0 $. Fra tale differenza e la carica vi è proporzionalità. Il loro rapporto è costante ed è pari a $ \varepsilon_0 A / d $. Nel 1831 Faraday comincio a condurre una serie di esperienze di carattere sistematico che mostrano che, nel momento in cui viene introdotta una lastra di materiale isolante (non permette il passaggio di carica) fra le armature del condensatore, la d.d.p. misurata è minore rispetto a quella osservata con lo stesso condensatore vuoto. Chiamando $ \Delta V_d  $ la differenza di potenziale in presenza del dielettrico si ha:
\[
	\Delta V_d < \Delta V_0
\]
Le sostanze isolanti che hanno questa proprietà di ridurre la differenza di potenziale tra le armature, e quindi il campo elettrico, si chiamano anche \emph{sostanze dielettriche o dielettrici}. La capacità del condensatore in presenza del dielettrico, in conseguenza alla diminuzione della d.d.p., sarà maggiore di quella dello stesso condensatore nel vuoto. Definiamo quindi, nel caso di un materiale omogeneo e isotropo, il rapporto adimensionale:
\[
	\varepsilon_r = \frac{C_d}{C_0} \qquad \text{costante dielettrica relativa del dielettrico}
\]
La capacità del condensatore aumenta di un fattore $ \varepsilon_r  $, lo stesso fattore di diminuzione del potenziale e del campo elettrico. Essendo la capacita $ C_d  $ sempre maggiore rispetto alla capacità del condensatore vuoto, la costante $ \varepsilon_r  $ è sempre maggiore o uguale a $1$. Ad esempio per l'aria $ \varepsilon_r \sim 1,0006 $ e per l'acqua $ \varepsilon_r \sim 80 $.
Il caso di $ C_d=C_0   $, si ha quando il mezzo è il vuoto. Anche esso infatti può essere assimilato a un dielettrico con costante dielettrica relativa pari a $1$. Per quanto detto finora, possiamo capire come le formule ottenute in precedenza sulle capacità del condensatore piano, cilindrico, sferico, restano valide purché si sostituisca a $ \varepsilon_0  $ la grandezza:
\[
	\varepsilon = \varepsilon_0 \varepsilon_r \qquad \text{costante dielettrica assoluta del materiale}
\]
Introdotte queste quantità possiamo scrivere esplicitamente che in un condensatore piano in presenza di dielettrico la capacità sarà
\[
	C_d=\varepsilon_r C_0 \qquad C_0=\frac{\varepsilon_0 A}{d} \qquad \implies C_d=\frac{\varepsilon_0 \varepsilon_r A}{d}=\frac{\varepsilon A}{d}
\]
Definiamo inoltre
\[
	\chi=\varepsilon_r -1 \qquad \text{suscettività elettrica, } \chi \geq 0
\]







































\section{Polarizzazione dei dielettrici}

Dobbiamo ora chiederci quali siano i fenomeni microscopici che hanno luogo in un materiale isolante in presenza di campo elettrico. Ricordiamo come il fenomeno dell'induzione elettrostatica, che rende possibile la separazione delle cariche dei due segni nei conduttori, sia dovuto al fatto che nei conduttori un certo numero di elettroni per atomo è separato dall'atomo stesso: all'interno dei conduttori esiste un gas di elettroni praticamente liberi. Negli isolanti invece tutti gli elettroni sono legati agli atomi e non se ne allontanano spontaneamente. Per far avvenire la separazione occorre agire dall'esterno ad esempio tramite lo strofinio con un panno. Questo significa che normalmente qualunque pezzo di materiale isolante ha carica nulla: non possiamo estrarre o cedere cariche. Possiamo ad esempio introdurre una perturbazione molto forte, ad esempio sfregando il materiale molto intensamente o attraverso una scarica elettrica o ionizzare il materiale con i raggi X. A meno di non fare questo assumeremo che il materiale isolante sia sempre complessivamente neutro.
Se si applica al dielettrico un campo elettrico esterno avviene uno spostamento locale delle cariche che costituiscono gli atomi, viene indotta la presenza di dipoli elettrici. Possiamo quindi definire la polarizzazione di un dielettrico come la formazione di un dipolo orientato in modo tale da contrastare il campo elettrico esterno: tale dipolo è dato dalle deformazioni della struttura elettronica microscopica degli atomi attorno alla posizione di equilibrio, oppure dal loro orientamento. Questo rende possibile la distinzione di due tipi di polarizzazione: la polarizzazione per deformazione e la polarizzazione per orientamento.

\paragraph{Polarizzazione per deformazione.} Il primo ha luogo in qualunque materiale. In un atomo in condizioni normali e in assenza di campo elettrico esterno la distribuzione degli elettroni è in media simmetrica rispetto al nucleo: essa viene rappresentata come una nube di carica negativa che occupa una zona intorno al nucleo di raggio $R$ pari alle dimensioni dell'atomo. Il centro di massa della carica negativa $(-Ze)$ coincide con il nucleo positivo $(+Ze)$. Sotto l'azione di un campo $\vec{E}$ il centro (di massa) della nube negativa subisce uno spostamento in verso contrario al campo, il nucleo in senso concorde al campo e si raggiunge una posizione di equilibrio in cui questo effetto è bilanciato dall'attrazione fra le cariche di segno opposto. È come se potessimo imaginare che in questa condizione di equilibrio tutta la carica negativa sia concentrata nel baricentro della sfera. Avremo un piccolo dipolo in cui le due cariche si saranno separate di una certa distanza $\vec{r}$. Ha senso introdurre in questo modo un momento di dipolo, dato dal prodotto della carica $Ze$ per lo spostamento:
\[
	\boxed{\vec{p} = Ze\vec{r}}
\]
Questo processo prende in nome di \textbf{polarizzazione per deformazione}. Si dimostra che se il campo elettrico totale non è troppo intenso, lo spostamento $ \vec{r}  $ è proporzionale al campo elettrico esterno applicato.
\[
	\vec{r} \propto \vec{E}_l \implies \vec{p} \propto \vec{E}_l \implies \vec{p} = \alpha_d  \vec{E}_l
\]
Si dimostra che:
\[
	\alpha_d = 4\pi \varepsilon_0 R^3 \sim 10^{-41} \; Fm^2
\]
Si può comprendere quindi perché atomi con nucleo atomico maggiore sono più polarizzabili.
Se abbiamo tanti atomi concentrati in un cubetto e applichiamo ad esso un campo elettrico, vediamo che tutti i momenti degli gli atomi si orientano allo stesso modo e di conseguenza il momento di dipolo medio coincide con il momento di dipolo del singolo atomo. Il campo elettrico totale è la somma dei campi elettrici prodotti da tutte le sorgenti.
\[
	\langle \vec{p} \rangle = \frac{\sum_i \vec{p}_i}{N} = \frac{N\alpha_d \vec{E}_l}{N} = \alpha_d\vec{E}_l
\]
Il campo $ \vec{E}_{\text{locale}}  $ agente sul singolo atomo a rigore non coincide con il campo macroscopico $ \vec{E}  $ presente nel dielettrico, in cui è compreso il contributo dell'atomo interessato. Questo contributo è trascurabile in un gas in condizioni standard di temperatura e pressione. Allo stato condensato invece, la differenza tra campo totale e campo locale è molto più accentuata.

\paragraph{Polarizzazione per orientamento.} Esistono delle sostanze in cui le molecole presentano un \emph{momento di dipolo intrinseco}. Si tratta di molecole poliatomiche formate da specie atomiche diverse in cui il centro delle cariche positive non coincide con il centro delle cariche negative.
\begin{figure}[htpb]
	\centering
	

	\tikzset{every picture/.style={line width=0.75pt}} %set default line width to 0.75pt        

	\begin{tikzpicture}[x=0.75pt,y=0.75pt,yscale=-0.9,xscale=0.9]
	%uncomment if require: \path (0,300); %set diagram left start at 0, and has height of 300

	%Shape: Rectangle [id:dp6576962240219826] 
	\draw  [fill={rgb, 255:red, 222; green, 222; blue, 222 }  ,fill opacity=1 ] (56,77) -- (274.5,77) -- (274.5,218) -- (56,218) -- cycle ;
	%Straight Lines [id:da746633193541568] 
	\draw    (80,97) -- (124.1,130.2) ;
	\draw [shift={(126.5,132)}, rotate = 216.97] [fill={rgb, 255:red, 0; green, 0; blue, 0 }  ][line width=0.08]  [draw opacity=0] (10.72,-5.15) -- (0,0) -- (10.72,5.15) -- (7.12,0) -- cycle    ;
	%Shape: Rectangle [id:dp45155476595525257] 
	\draw  [fill={rgb, 255:red, 222; green, 222; blue, 222 }  ,fill opacity=1 ] (326,77) -- (544.5,77) -- (544.5,218) -- (326,218) -- cycle ;
	%Straight Lines [id:da1302435157948829] 
	\draw    (252,139) -- (220.8,112.92) ;
	\draw [shift={(218.5,111)}, rotate = 399.89] [fill={rgb, 255:red, 0; green, 0; blue, 0 }  ][line width=0.08]  [draw opacity=0] (10.72,-5.15) -- (0,0) -- (10.72,5.15) -- (7.12,0) -- cycle    ;
	%Straight Lines [id:da8601858024938456] 
	\draw    (185,191) -- (204.9,159.54) ;
	\draw [shift={(206.5,157)}, rotate = 482.31] [fill={rgb, 255:red, 0; green, 0; blue, 0 }  ][line width=0.08]  [draw opacity=0] (10.72,-5.15) -- (0,0) -- (10.72,5.15) -- (7.12,0) -- cycle    ;
	%Straight Lines [id:da9344740713006228] 
	\draw    (133,182) -- (84.43,192.37) ;
	\draw [shift={(81.5,193)}, rotate = 347.94] [fill={rgb, 255:red, 0; green, 0; blue, 0 }  ][line width=0.08]  [draw opacity=0] (10.72,-5.15) -- (0,0) -- (10.72,5.15) -- (7.12,0) -- cycle    ;
	%Straight Lines [id:da737086394537414] 
	\draw    (152,143) -- (161.91,93.94) ;
	\draw [shift={(162.5,91)}, rotate = 461.42] [fill={rgb, 255:red, 0; green, 0; blue, 0 }  ][line width=0.08]  [draw opacity=0] (10.72,-5.15) -- (0,0) -- (10.72,5.15) -- (7.12,0) -- cycle    ;
	%Straight Lines [id:da5569072759879605] 
	\draw    (219.5,90) -- (190.13,135.48) ;
	\draw [shift={(188.5,138)}, rotate = 302.86] [fill={rgb, 255:red, 0; green, 0; blue, 0 }  ][line width=0.08]  [draw opacity=0] (10.72,-5.15) -- (0,0) -- (10.72,5.15) -- (7.12,0) -- cycle    ;
	%Straight Lines [id:da3780313133880817] 
	\draw    (223,148) -- (238.56,195.15) ;
	\draw [shift={(239.5,198)}, rotate = 251.74] [fill={rgb, 255:red, 0; green, 0; blue, 0 }  ][line width=0.08]  [draw opacity=0] (10.72,-5.15) -- (0,0) -- (10.72,5.15) -- (7.12,0) -- cycle    ;
	%Straight Lines [id:da1755853991450964] 
	\draw    (165,180) -- (129.1,159.49) ;
	\draw [shift={(126.5,158)}, rotate = 389.74] [fill={rgb, 255:red, 0; green, 0; blue, 0 }  ][line width=0.08]  [draw opacity=0] (10.72,-5.15) -- (0,0) -- (10.72,5.15) -- (7.12,0) -- cycle    ;
	%Straight Lines [id:da5614600138989143] 
	\draw    (82,169) -- (101.02,135.61) ;
	\draw [shift={(102.5,133)}, rotate = 479.66] [fill={rgb, 255:red, 0; green, 0; blue, 0 }  ][line width=0.08]  [draw opacity=0] (10.72,-5.15) -- (0,0) -- (10.72,5.15) -- (7.12,0) -- cycle    ;
	%Straight Lines [id:da8579471343252436] 
	\draw    (362.5,116) -- (362.5,161) ;
	\draw [shift={(362.5,164)}, rotate = 270] [fill={rgb, 255:red, 0; green, 0; blue, 0 }  ][line width=0.08]  [draw opacity=0] (10.72,-5.15) -- (0,0) -- (10.72,5.15) -- (7.12,0) -- cycle    ;
	%Straight Lines [id:da15722289884503238] 
	\draw    (392.5,126) -- (392.5,171) ;
	\draw [shift={(392.5,174)}, rotate = 270] [fill={rgb, 255:red, 0; green, 0; blue, 0 }  ][line width=0.08]  [draw opacity=0] (10.72,-5.15) -- (0,0) -- (10.72,5.15) -- (7.12,0) -- cycle    ;
	%Straight Lines [id:da2301237669359808] 
	\draw    (422.5,116) -- (422.5,161) ;
	\draw [shift={(422.5,164)}, rotate = 270] [fill={rgb, 255:red, 0; green, 0; blue, 0 }  ][line width=0.08]  [draw opacity=0] (10.72,-5.15) -- (0,0) -- (10.72,5.15) -- (7.12,0) -- cycle    ;
	%Straight Lines [id:da19196603646506527] 
	\draw    (452.5,136) -- (452.5,181) ;
	\draw [shift={(452.5,184)}, rotate = 270] [fill={rgb, 255:red, 0; green, 0; blue, 0 }  ][line width=0.08]  [draw opacity=0] (10.72,-5.15) -- (0,0) -- (10.72,5.15) -- (7.12,0) -- cycle    ;
	%Straight Lines [id:da32584235732099254] 
	\draw    (482.5,116) -- (482.5,161) ;
	\draw [shift={(482.5,164)}, rotate = 270] [fill={rgb, 255:red, 0; green, 0; blue, 0 }  ][line width=0.08]  [draw opacity=0] (10.72,-5.15) -- (0,0) -- (10.72,5.15) -- (7.12,0) -- cycle    ;
	%Straight Lines [id:da5302442677332262] 
	\draw    (512.5,126) -- (512.5,171) ;
	\draw [shift={(512.5,174)}, rotate = 270] [fill={rgb, 255:red, 0; green, 0; blue, 0 }  ][line width=0.08]  [draw opacity=0] (10.72,-5.15) -- (0,0) -- (10.72,5.15) -- (7.12,0) -- cycle    ;

	% Text Node
	\draw (454,105) node    {$\vec{E}$};


	\end{tikzpicture}
\end{figure}
\FloatBarrier
In assenza di campo elettrico esterno i dipoli molecolari sono orientati a caso, per via degli urti dovuti ai moti di agitazione termica che distruggono eventuali configurazioni ordinate dovute alle interazione tra i dipoli. È per questo che se calcoliamo il momento di dipolo medio in un dielettrico in assenza di campi elettrici esterni, la somma di tutti i dipoli sarà zero. Se applichiamo al materiale un campo elettrico, su ciascuno dei dipoli agisce il momento delle forze:
\[
	\vec{M} =\vec{p} \times \vec{E}_l
\]
che ne causa un orientamento con il campo soltanto parziale perché disturbato dall'agitazione termica. Il grado di allineamento aumenta al diminuire della temperatura e all'aumentare dell'intensità del campo elettrico esterno.
Si dimostra tramite un ragionamento di tipo statistico che il momento di dipolo medio si può scrivere come:
\[
	\langle \vec{p} \rangle = \alpha_o \vec{E}_l
\]
$ \alpha_0  $ prende il nome di \emph{polarizzabilità per orientamento} e questo fenomeno è detto \textbf{polarizzazione per orientamento}.
\[
	\alpha_o \sim \frac{P_0^2}{3k_BT}
\]
Confermiamo che se $T$ aumenta aumentano le collisioni e diminuisce il valore medio del momento di dipolo. Viceversa se il momento di dipolo è molto alto, l'effetto di allineamento è più forte.

Introduciamo una nuova grandezza che ci descrive lo stato di polarizzazione all'interno del materiale punto per punto. Immaginiamo di avere un blocchetto sul quale un campo elettrico provoca un effetto di polarizzazione. Poniamo di prendere all'interno del materiale un punto P di volume macroscopico che chiamiamo $ \Delta \tau $. Se sommiamo i momenti di dipolo troveremo il momento di dipolo totale che chiamiamo $ \Delta \vec{p} $. Definiamo \textbf{vettore polarizzazione del dielettrico}:
\[
	\boxed{\vec{P} = \lim_{\Delta \tau  \to 0} \frac{\Delta \vec{p}}{\Delta \tau}}
\]
Il passaggio al limite si effettua immaginando di far tendere a zero il volume (anche il momento di dipolo totale tenderà a zero) e si ottiene un vettore che ha le dimensioni di un momento di dipolo per unità di volume. Se facciamo un analisi dimensionale si ha:
\[
	[\vec{P}] = \frac{[Q][L]}{[L^3]} = \frac{[Q]}{[L^2]} = \left( \frac{C}{m^2} \right)
\]
Troviamo le stesse dimensioni di una densità di carica superficiale. Possiamo calcolare il momento di dipolo totale del cubetto come:
\[
	\Delta \vec{p} = \langle \vec{p} \rangle \cdot \Delta N
\]
Dove $N$ rappresenta il numero di atomi o molecole contenuti nel volume considerato. Il vettore polarizzazione può allora essere scritto come:
\[
	\vec{P} (P) = \lim_{\Delta \tau  \to 0} \frac{\Delta \vec{p}}{\Delta \tau} = \lim_{\Delta \tau  \to 0} \langle \vec{p} \rangle \frac{\Delta N}{\Delta \tau} = \langle \vec{p} \rangle \underbrace{\lim_{\Delta \tau  \to 0} \frac{\Delta N}{\Delta \tau}}_n
\]
$n=$ numero di atomi o molecole per unità di volume.
Bisogna specificare che il passaggio al limite deve essere inteso nel senso che $ \Delta \tau  $ è piccolo su scala macroscopica, ma sempre abbastanza grande da contenere un numero tale di atomi per cui abbia senso affermare che $ \langle \vec{p} \rangle $ rappresenta il momento di dipolo elettrico medio di ciascun elemento.

\paragraph{Campo elettrico prodotto da un dielettrico polarizzato.}Consideriamo una lastra di materiale dielettrico polarizzato uniformemente. Ci aspettiamo che i momenti di dipolo medi siano allineati nella direzione di $\vec{E}$.
\begin{figure}[htpb]
	\centering
	

	\tikzset{every picture/.style={line width=0.75pt}} %set default line width to 0.75pt        

	\begin{tikzpicture}[x=0.75pt,y=0.75pt,yscale=-1,xscale=1]
	%uncomment if require: \path (0,300); %set diagram left start at 0, and has height of 300

	%Shape: Circle [id:dp19164084868005893] 
	\draw   (112.5,139.5) .. controls (112.5,135.36) and (115.86,132) .. (120,132) .. controls (124.14,132) and (127.5,135.36) .. (127.5,139.5) .. controls (127.5,143.64) and (124.14,147) .. (120,147) .. controls (115.86,147) and (112.5,143.64) .. (112.5,139.5) -- cycle ;

	%Shape: Circle [id:dp4711974201949678] 
	\draw   (94.5,139.5) .. controls (94.5,135.36) and (97.86,132) .. (102,132) .. controls (106.14,132) and (109.5,135.36) .. (109.5,139.5) .. controls (109.5,143.64) and (106.14,147) .. (102,147) .. controls (97.86,147) and (94.5,143.64) .. (94.5,139.5) -- cycle ;

	%Shape: Circle [id:dp1412969076711239] 
	\draw   (112.5,156.5) .. controls (112.5,152.36) and (115.86,149) .. (120,149) .. controls (124.14,149) and (127.5,152.36) .. (127.5,156.5) .. controls (127.5,160.64) and (124.14,164) .. (120,164) .. controls (115.86,164) and (112.5,160.64) .. (112.5,156.5) -- cycle ;

	%Shape: Circle [id:dp027966587924755926] 
	\draw   (94.5,156.5) .. controls (94.5,152.36) and (97.86,149) .. (102,149) .. controls (106.14,149) and (109.5,152.36) .. (109.5,156.5) .. controls (109.5,160.64) and (106.14,164) .. (102,164) .. controls (97.86,164) and (94.5,160.64) .. (94.5,156.5) -- cycle ;

	%Shape: Circle [id:dp6942327579518035] 
	\draw   (112.5,173.5) .. controls (112.5,169.36) and (115.86,166) .. (120,166) .. controls (124.14,166) and (127.5,169.36) .. (127.5,173.5) .. controls (127.5,177.64) and (124.14,181) .. (120,181) .. controls (115.86,181) and (112.5,177.64) .. (112.5,173.5) -- cycle ;

	%Shape: Circle [id:dp8714292795806315] 
	\draw   (94.5,173.5) .. controls (94.5,169.36) and (97.86,166) .. (102,166) .. controls (106.14,166) and (109.5,169.36) .. (109.5,173.5) .. controls (109.5,177.64) and (106.14,181) .. (102,181) .. controls (97.86,181) and (94.5,177.64) .. (94.5,173.5) -- cycle ;

	%Shape: Circle [id:dp2792300850364897] 
	\draw   (112.5,190.5) .. controls (112.5,186.36) and (115.86,183) .. (120,183) .. controls (124.14,183) and (127.5,186.36) .. (127.5,190.5) .. controls (127.5,194.64) and (124.14,198) .. (120,198) .. controls (115.86,198) and (112.5,194.64) .. (112.5,190.5) -- cycle ;

	%Shape: Circle [id:dp4704935059937605] 
	\draw   (94.5,190.5) .. controls (94.5,186.36) and (97.86,183) .. (102,183) .. controls (106.14,183) and (109.5,186.36) .. (109.5,190.5) .. controls (109.5,194.64) and (106.14,198) .. (102,198) .. controls (97.86,198) and (94.5,194.64) .. (94.5,190.5) -- cycle ;

	%Shape: Circle [id:dp3149780737642973] 
	\draw   (112.5,207.5) .. controls (112.5,203.36) and (115.86,200) .. (120,200) .. controls (124.14,200) and (127.5,203.36) .. (127.5,207.5) .. controls (127.5,211.64) and (124.14,215) .. (120,215) .. controls (115.86,215) and (112.5,211.64) .. (112.5,207.5) -- cycle ;

	%Shape: Circle [id:dp3792710061577118] 
	\draw   (94.5,207.5) .. controls (94.5,203.36) and (97.86,200) .. (102,200) .. controls (106.14,200) and (109.5,203.36) .. (109.5,207.5) .. controls (109.5,211.64) and (106.14,215) .. (102,215) .. controls (97.86,215) and (94.5,211.64) .. (94.5,207.5) -- cycle ;

	%Shape: Circle [id:dp25571012860277675] 
	\draw   (112.5,224.5) .. controls (112.5,220.36) and (115.86,217) .. (120,217) .. controls (124.14,217) and (127.5,220.36) .. (127.5,224.5) .. controls (127.5,228.64) and (124.14,232) .. (120,232) .. controls (115.86,232) and (112.5,228.64) .. (112.5,224.5) -- cycle ;

	%Shape: Circle [id:dp512259061057881] 
	\draw   (94.5,224.5) .. controls (94.5,220.36) and (97.86,217) .. (102,217) .. controls (106.14,217) and (109.5,220.36) .. (109.5,224.5) .. controls (109.5,228.64) and (106.14,232) .. (102,232) .. controls (97.86,232) and (94.5,228.64) .. (94.5,224.5) -- cycle ;

	%Shape: Circle [id:dp3302529449077396] 
	\draw   (162.5,139.5) .. controls (162.5,135.36) and (165.86,132) .. (170,132) .. controls (174.14,132) and (177.5,135.36) .. (177.5,139.5) .. controls (177.5,143.64) and (174.14,147) .. (170,147) .. controls (165.86,147) and (162.5,143.64) .. (162.5,139.5) -- cycle ;

	%Shape: Circle [id:dp26425636086936133] 
	\draw   (144.5,139.5) .. controls (144.5,135.36) and (147.86,132) .. (152,132) .. controls (156.14,132) and (159.5,135.36) .. (159.5,139.5) .. controls (159.5,143.64) and (156.14,147) .. (152,147) .. controls (147.86,147) and (144.5,143.64) .. (144.5,139.5) -- cycle ;

	%Shape: Circle [id:dp1245109699141056] 
	\draw   (162.5,156.5) .. controls (162.5,152.36) and (165.86,149) .. (170,149) .. controls (174.14,149) and (177.5,152.36) .. (177.5,156.5) .. controls (177.5,160.64) and (174.14,164) .. (170,164) .. controls (165.86,164) and (162.5,160.64) .. (162.5,156.5) -- cycle ;

	%Shape: Circle [id:dp46231724294315657] 
	\draw   (144.5,156.5) .. controls (144.5,152.36) and (147.86,149) .. (152,149) .. controls (156.14,149) and (159.5,152.36) .. (159.5,156.5) .. controls (159.5,160.64) and (156.14,164) .. (152,164) .. controls (147.86,164) and (144.5,160.64) .. (144.5,156.5) -- cycle ;

	%Shape: Circle [id:dp1900913512441038] 
	\draw   (162.5,173.5) .. controls (162.5,169.36) and (165.86,166) .. (170,166) .. controls (174.14,166) and (177.5,169.36) .. (177.5,173.5) .. controls (177.5,177.64) and (174.14,181) .. (170,181) .. controls (165.86,181) and (162.5,177.64) .. (162.5,173.5) -- cycle ;

	%Shape: Circle [id:dp9452779973114231] 
	\draw   (144.5,173.5) .. controls (144.5,169.36) and (147.86,166) .. (152,166) .. controls (156.14,166) and (159.5,169.36) .. (159.5,173.5) .. controls (159.5,177.64) and (156.14,181) .. (152,181) .. controls (147.86,181) and (144.5,177.64) .. (144.5,173.5) -- cycle ;

	%Shape: Circle [id:dp12026290511889726] 
	\draw   (162.5,190.5) .. controls (162.5,186.36) and (165.86,183) .. (170,183) .. controls (174.14,183) and (177.5,186.36) .. (177.5,190.5) .. controls (177.5,194.64) and (174.14,198) .. (170,198) .. controls (165.86,198) and (162.5,194.64) .. (162.5,190.5) -- cycle ;

	%Shape: Circle [id:dp676886513349807] 
	\draw   (144.5,190.5) .. controls (144.5,186.36) and (147.86,183) .. (152,183) .. controls (156.14,183) and (159.5,186.36) .. (159.5,190.5) .. controls (159.5,194.64) and (156.14,198) .. (152,198) .. controls (147.86,198) and (144.5,194.64) .. (144.5,190.5) -- cycle ;

	%Shape: Circle [id:dp7919018684910728] 
	\draw   (162.5,207.5) .. controls (162.5,203.36) and (165.86,200) .. (170,200) .. controls (174.14,200) and (177.5,203.36) .. (177.5,207.5) .. controls (177.5,211.64) and (174.14,215) .. (170,215) .. controls (165.86,215) and (162.5,211.64) .. (162.5,207.5) -- cycle ;

	%Shape: Circle [id:dp4593544722439482] 
	\draw   (144.5,207.5) .. controls (144.5,203.36) and (147.86,200) .. (152,200) .. controls (156.14,200) and (159.5,203.36) .. (159.5,207.5) .. controls (159.5,211.64) and (156.14,215) .. (152,215) .. controls (147.86,215) and (144.5,211.64) .. (144.5,207.5) -- cycle ;

	%Shape: Circle [id:dp5718514674723356] 
	\draw   (162.5,224.5) .. controls (162.5,220.36) and (165.86,217) .. (170,217) .. controls (174.14,217) and (177.5,220.36) .. (177.5,224.5) .. controls (177.5,228.64) and (174.14,232) .. (170,232) .. controls (165.86,232) and (162.5,228.64) .. (162.5,224.5) -- cycle ;

	%Shape: Circle [id:dp286314008013018] 
	\draw   (144.5,224.5) .. controls (144.5,220.36) and (147.86,217) .. (152,217) .. controls (156.14,217) and (159.5,220.36) .. (159.5,224.5) .. controls (159.5,228.64) and (156.14,232) .. (152,232) .. controls (147.86,232) and (144.5,228.64) .. (144.5,224.5) -- cycle ;

	%Shape: Circle [id:dp3350616079956419] 
	\draw   (212.5,139.5) .. controls (212.5,135.36) and (215.86,132) .. (220,132) .. controls (224.14,132) and (227.5,135.36) .. (227.5,139.5) .. controls (227.5,143.64) and (224.14,147) .. (220,147) .. controls (215.86,147) and (212.5,143.64) .. (212.5,139.5) -- cycle ;

	%Shape: Circle [id:dp9875918744845518] 
	\draw   (194.5,139.5) .. controls (194.5,135.36) and (197.86,132) .. (202,132) .. controls (206.14,132) and (209.5,135.36) .. (209.5,139.5) .. controls (209.5,143.64) and (206.14,147) .. (202,147) .. controls (197.86,147) and (194.5,143.64) .. (194.5,139.5) -- cycle ;

	%Shape: Circle [id:dp09009059961354371] 
	\draw   (212.5,156.5) .. controls (212.5,152.36) and (215.86,149) .. (220,149) .. controls (224.14,149) and (227.5,152.36) .. (227.5,156.5) .. controls (227.5,160.64) and (224.14,164) .. (220,164) .. controls (215.86,164) and (212.5,160.64) .. (212.5,156.5) -- cycle ;

	%Shape: Circle [id:dp04381701106928748] 
	\draw   (194.5,156.5) .. controls (194.5,152.36) and (197.86,149) .. (202,149) .. controls (206.14,149) and (209.5,152.36) .. (209.5,156.5) .. controls (209.5,160.64) and (206.14,164) .. (202,164) .. controls (197.86,164) and (194.5,160.64) .. (194.5,156.5) -- cycle ;

	%Shape: Circle [id:dp839790661869483] 
	\draw   (212.5,173.5) .. controls (212.5,169.36) and (215.86,166) .. (220,166) .. controls (224.14,166) and (227.5,169.36) .. (227.5,173.5) .. controls (227.5,177.64) and (224.14,181) .. (220,181) .. controls (215.86,181) and (212.5,177.64) .. (212.5,173.5) -- cycle ;

	%Shape: Circle [id:dp12769245667931517] 
	\draw   (194.5,173.5) .. controls (194.5,169.36) and (197.86,166) .. (202,166) .. controls (206.14,166) and (209.5,169.36) .. (209.5,173.5) .. controls (209.5,177.64) and (206.14,181) .. (202,181) .. controls (197.86,181) and (194.5,177.64) .. (194.5,173.5) -- cycle ;

	%Shape: Circle [id:dp47307370720138087] 
	\draw   (212.5,190.5) .. controls (212.5,186.36) and (215.86,183) .. (220,183) .. controls (224.14,183) and (227.5,186.36) .. (227.5,190.5) .. controls (227.5,194.64) and (224.14,198) .. (220,198) .. controls (215.86,198) and (212.5,194.64) .. (212.5,190.5) -- cycle ;

	%Shape: Circle [id:dp7906392079870947] 
	\draw   (194.5,190.5) .. controls (194.5,186.36) and (197.86,183) .. (202,183) .. controls (206.14,183) and (209.5,186.36) .. (209.5,190.5) .. controls (209.5,194.64) and (206.14,198) .. (202,198) .. controls (197.86,198) and (194.5,194.64) .. (194.5,190.5) -- cycle ;

	%Shape: Circle [id:dp6831223709389529] 
	\draw   (212.5,207.5) .. controls (212.5,203.36) and (215.86,200) .. (220,200) .. controls (224.14,200) and (227.5,203.36) .. (227.5,207.5) .. controls (227.5,211.64) and (224.14,215) .. (220,215) .. controls (215.86,215) and (212.5,211.64) .. (212.5,207.5) -- cycle ;

	%Shape: Circle [id:dp27793654220673414] 
	\draw   (194.5,207.5) .. controls (194.5,203.36) and (197.86,200) .. (202,200) .. controls (206.14,200) and (209.5,203.36) .. (209.5,207.5) .. controls (209.5,211.64) and (206.14,215) .. (202,215) .. controls (197.86,215) and (194.5,211.64) .. (194.5,207.5) -- cycle ;

	%Shape: Circle [id:dp9378917273548486] 
	\draw   (212.5,224.5) .. controls (212.5,220.36) and (215.86,217) .. (220,217) .. controls (224.14,217) and (227.5,220.36) .. (227.5,224.5) .. controls (227.5,228.64) and (224.14,232) .. (220,232) .. controls (215.86,232) and (212.5,228.64) .. (212.5,224.5) -- cycle ;

	%Shape: Circle [id:dp2910685753031026] 
	\draw   (194.5,224.5) .. controls (194.5,220.36) and (197.86,217) .. (202,217) .. controls (206.14,217) and (209.5,220.36) .. (209.5,224.5) .. controls (209.5,228.64) and (206.14,232) .. (202,232) .. controls (197.86,232) and (194.5,228.64) .. (194.5,224.5) -- cycle ;

	%Shape: Circle [id:dp875640502618519] 
	\draw   (262.5,139.5) .. controls (262.5,135.36) and (265.86,132) .. (270,132) .. controls (274.14,132) and (277.5,135.36) .. (277.5,139.5) .. controls (277.5,143.64) and (274.14,147) .. (270,147) .. controls (265.86,147) and (262.5,143.64) .. (262.5,139.5) -- cycle ;

	%Shape: Circle [id:dp7726240726317197] 
	\draw   (244.5,139.5) .. controls (244.5,135.36) and (247.86,132) .. (252,132) .. controls (256.14,132) and (259.5,135.36) .. (259.5,139.5) .. controls (259.5,143.64) and (256.14,147) .. (252,147) .. controls (247.86,147) and (244.5,143.64) .. (244.5,139.5) -- cycle ;

	%Shape: Circle [id:dp42416738420649125] 
	\draw   (262.5,156.5) .. controls (262.5,152.36) and (265.86,149) .. (270,149) .. controls (274.14,149) and (277.5,152.36) .. (277.5,156.5) .. controls (277.5,160.64) and (274.14,164) .. (270,164) .. controls (265.86,164) and (262.5,160.64) .. (262.5,156.5) -- cycle ;

	%Shape: Circle [id:dp9792443489291947] 
	\draw   (244.5,156.5) .. controls (244.5,152.36) and (247.86,149) .. (252,149) .. controls (256.14,149) and (259.5,152.36) .. (259.5,156.5) .. controls (259.5,160.64) and (256.14,164) .. (252,164) .. controls (247.86,164) and (244.5,160.64) .. (244.5,156.5) -- cycle ;

	%Shape: Circle [id:dp8828004218874803] 
	\draw   (262.5,173.5) .. controls (262.5,169.36) and (265.86,166) .. (270,166) .. controls (274.14,166) and (277.5,169.36) .. (277.5,173.5) .. controls (277.5,177.64) and (274.14,181) .. (270,181) .. controls (265.86,181) and (262.5,177.64) .. (262.5,173.5) -- cycle ;

	%Shape: Circle [id:dp8135086241131206] 
	\draw   (244.5,173.5) .. controls (244.5,169.36) and (247.86,166) .. (252,166) .. controls (256.14,166) and (259.5,169.36) .. (259.5,173.5) .. controls (259.5,177.64) and (256.14,181) .. (252,181) .. controls (247.86,181) and (244.5,177.64) .. (244.5,173.5) -- cycle ;

	%Shape: Circle [id:dp8183070500746967] 
	\draw   (262.5,190.5) .. controls (262.5,186.36) and (265.86,183) .. (270,183) .. controls (274.14,183) and (277.5,186.36) .. (277.5,190.5) .. controls (277.5,194.64) and (274.14,198) .. (270,198) .. controls (265.86,198) and (262.5,194.64) .. (262.5,190.5) -- cycle ;

	%Shape: Circle [id:dp8771987952634577] 
	\draw   (244.5,190.5) .. controls (244.5,186.36) and (247.86,183) .. (252,183) .. controls (256.14,183) and (259.5,186.36) .. (259.5,190.5) .. controls (259.5,194.64) and (256.14,198) .. (252,198) .. controls (247.86,198) and (244.5,194.64) .. (244.5,190.5) -- cycle ;

	%Shape: Circle [id:dp8547826702722316] 
	\draw   (262.5,207.5) .. controls (262.5,203.36) and (265.86,200) .. (270,200) .. controls (274.14,200) and (277.5,203.36) .. (277.5,207.5) .. controls (277.5,211.64) and (274.14,215) .. (270,215) .. controls (265.86,215) and (262.5,211.64) .. (262.5,207.5) -- cycle ;

	%Shape: Circle [id:dp9809578657766269] 
	\draw   (244.5,207.5) .. controls (244.5,203.36) and (247.86,200) .. (252,200) .. controls (256.14,200) and (259.5,203.36) .. (259.5,207.5) .. controls (259.5,211.64) and (256.14,215) .. (252,215) .. controls (247.86,215) and (244.5,211.64) .. (244.5,207.5) -- cycle ;

	%Shape: Circle [id:dp8798473406470462] 
	\draw   (262.5,224.5) .. controls (262.5,220.36) and (265.86,217) .. (270,217) .. controls (274.14,217) and (277.5,220.36) .. (277.5,224.5) .. controls (277.5,228.64) and (274.14,232) .. (270,232) .. controls (265.86,232) and (262.5,228.64) .. (262.5,224.5) -- cycle ;

	%Shape: Circle [id:dp7853219619689527] 
	\draw   (244.5,224.5) .. controls (244.5,220.36) and (247.86,217) .. (252,217) .. controls (256.14,217) and (259.5,220.36) .. (259.5,224.5) .. controls (259.5,228.64) and (256.14,232) .. (252,232) .. controls (247.86,232) and (244.5,228.64) .. (244.5,224.5) -- cycle ;

	%Shape: Circle [id:dp9656716411444544] 
	\draw   (312.5,139.5) .. controls (312.5,135.36) and (315.86,132) .. (320,132) .. controls (324.14,132) and (327.5,135.36) .. (327.5,139.5) .. controls (327.5,143.64) and (324.14,147) .. (320,147) .. controls (315.86,147) and (312.5,143.64) .. (312.5,139.5) -- cycle ;

	%Shape: Circle [id:dp886437198293027] 
	\draw   (294.5,139.5) .. controls (294.5,135.36) and (297.86,132) .. (302,132) .. controls (306.14,132) and (309.5,135.36) .. (309.5,139.5) .. controls (309.5,143.64) and (306.14,147) .. (302,147) .. controls (297.86,147) and (294.5,143.64) .. (294.5,139.5) -- cycle ;

	%Shape: Circle [id:dp7420510641067599] 
	\draw   (312.5,156.5) .. controls (312.5,152.36) and (315.86,149) .. (320,149) .. controls (324.14,149) and (327.5,152.36) .. (327.5,156.5) .. controls (327.5,160.64) and (324.14,164) .. (320,164) .. controls (315.86,164) and (312.5,160.64) .. (312.5,156.5) -- cycle ;

	%Shape: Circle [id:dp27984106972355893] 
	\draw   (294.5,156.5) .. controls (294.5,152.36) and (297.86,149) .. (302,149) .. controls (306.14,149) and (309.5,152.36) .. (309.5,156.5) .. controls (309.5,160.64) and (306.14,164) .. (302,164) .. controls (297.86,164) and (294.5,160.64) .. (294.5,156.5) -- cycle ;

	%Shape: Circle [id:dp45267946605936715] 
	\draw   (312.5,173.5) .. controls (312.5,169.36) and (315.86,166) .. (320,166) .. controls (324.14,166) and (327.5,169.36) .. (327.5,173.5) .. controls (327.5,177.64) and (324.14,181) .. (320,181) .. controls (315.86,181) and (312.5,177.64) .. (312.5,173.5) -- cycle ;

	%Shape: Circle [id:dp47176953309178127] 
	\draw   (294.5,173.5) .. controls (294.5,169.36) and (297.86,166) .. (302,166) .. controls (306.14,166) and (309.5,169.36) .. (309.5,173.5) .. controls (309.5,177.64) and (306.14,181) .. (302,181) .. controls (297.86,181) and (294.5,177.64) .. (294.5,173.5) -- cycle ;

	%Shape: Circle [id:dp4522992215286874] 
	\draw   (312.5,190.5) .. controls (312.5,186.36) and (315.86,183) .. (320,183) .. controls (324.14,183) and (327.5,186.36) .. (327.5,190.5) .. controls (327.5,194.64) and (324.14,198) .. (320,198) .. controls (315.86,198) and (312.5,194.64) .. (312.5,190.5) -- cycle ;

	%Shape: Circle [id:dp80510293643717] 
	\draw   (294.5,190.5) .. controls (294.5,186.36) and (297.86,183) .. (302,183) .. controls (306.14,183) and (309.5,186.36) .. (309.5,190.5) .. controls (309.5,194.64) and (306.14,198) .. (302,198) .. controls (297.86,198) and (294.5,194.64) .. (294.5,190.5) -- cycle ;

	%Shape: Circle [id:dp6513760445442778] 
	\draw   (312.5,207.5) .. controls (312.5,203.36) and (315.86,200) .. (320,200) .. controls (324.14,200) and (327.5,203.36) .. (327.5,207.5) .. controls (327.5,211.64) and (324.14,215) .. (320,215) .. controls (315.86,215) and (312.5,211.64) .. (312.5,207.5) -- cycle ;

	%Shape: Circle [id:dp8348478744395829] 
	\draw   (294.5,207.5) .. controls (294.5,203.36) and (297.86,200) .. (302,200) .. controls (306.14,200) and (309.5,203.36) .. (309.5,207.5) .. controls (309.5,211.64) and (306.14,215) .. (302,215) .. controls (297.86,215) and (294.5,211.64) .. (294.5,207.5) -- cycle ;

	%Shape: Circle [id:dp7822794310209413] 
	\draw   (312.5,224.5) .. controls (312.5,220.36) and (315.86,217) .. (320,217) .. controls (324.14,217) and (327.5,220.36) .. (327.5,224.5) .. controls (327.5,228.64) and (324.14,232) .. (320,232) .. controls (315.86,232) and (312.5,228.64) .. (312.5,224.5) -- cycle ;

	%Shape: Circle [id:dp05820807585887722] 
	\draw   (294.5,224.5) .. controls (294.5,220.36) and (297.86,217) .. (302,217) .. controls (306.14,217) and (309.5,220.36) .. (309.5,224.5) .. controls (309.5,228.64) and (306.14,232) .. (302,232) .. controls (297.86,232) and (294.5,228.64) .. (294.5,224.5) -- cycle ;

	%Straight Lines [id:da8114546191950183] 
	\draw    (127.5,139.5) -- (144.5,139.5) ;
	%Straight Lines [id:da5397729772664339] 
	\draw    (127.5,156.5) -- (144.5,156.5) ;
	%Straight Lines [id:da22017195744423446] 
	\draw    (127.5,173.5) -- (144.5,173.5) ;
	%Straight Lines [id:da3280947652086823] 
	\draw    (127.5,190.5) -- (144.5,190.5) ;
	%Straight Lines [id:da6372066143296944] 
	\draw    (127.5,207.5) -- (144.5,207.5) ;
	%Straight Lines [id:da5472936403127473] 
	\draw    (127.5,224.5) -- (144.5,224.5) ;
	%Straight Lines [id:da3177974539882309] 
	\draw    (177.5,139.5) -- (194.5,139.5) ;
	%Straight Lines [id:da670731851387866] 
	\draw    (177.5,156.5) -- (194.5,156.5) ;
	%Straight Lines [id:da18046734708581846] 
	\draw    (177.5,173.5) -- (194.5,173.5) ;
	%Straight Lines [id:da8948129865243182] 
	\draw    (177.5,190.5) -- (194.5,190.5) ;
	%Straight Lines [id:da016270853134040975] 
	\draw    (177.5,207.5) -- (194.5,207.5) ;
	%Straight Lines [id:da8922634629198649] 
	\draw    (177.5,224.5) -- (194.5,224.5) ;
	%Straight Lines [id:da8219981603071091] 
	\draw    (227.5,139.5) -- (244.5,139.5) ;
	%Straight Lines [id:da4338403557538508] 
	\draw    (227.5,156.5) -- (244.5,156.5) ;
	%Straight Lines [id:da5080322322617772] 
	\draw    (227.5,173.5) -- (244.5,173.5) ;
	%Straight Lines [id:da9765484430266556] 
	\draw    (227.5,190.5) -- (244.5,190.5) ;
	%Straight Lines [id:da5331696213916248] 
	\draw    (227.5,207.5) -- (244.5,207.5) ;
	%Straight Lines [id:da3955648042053568] 
	\draw    (227.5,224.5) -- (244.5,224.5) ;
	%Straight Lines [id:da5902876054342023] 
	\draw    (277.5,139.5) -- (294.5,139.5) ;
	%Straight Lines [id:da7296681159520706] 
	\draw    (277.5,156.5) -- (294.5,156.5) ;
	%Straight Lines [id:da8121629814839808] 
	\draw    (277.5,173.5) -- (294.5,173.5) ;
	%Straight Lines [id:da8172817994404065] 
	\draw    (277.5,190.5) -- (294.5,190.5) ;
	%Straight Lines [id:da16045024607549663] 
	\draw    (277.5,207.5) -- (294.5,207.5) ;
	%Straight Lines [id:da22800502295434] 
	\draw    (277.5,224.5) -- (294.5,224.5) ;
	%Shape: Ellipse [id:dp5574870645017052] 
	\draw   (89.33,178.33) .. controls (89.33,144.64) and (94.71,117.33) .. (101.33,117.33) .. controls (107.96,117.33) and (113.33,144.64) .. (113.33,178.33) .. controls (113.33,212.02) and (107.96,239.33) .. (101.33,239.33) .. controls (94.71,239.33) and (89.33,212.02) .. (89.33,178.33) -- cycle ;
	%Shape: Ellipse [id:dp6391350284755926] 
	\draw   (309.33,178.33) .. controls (309.33,144.64) and (314.71,117.33) .. (321.33,117.33) .. controls (327.96,117.33) and (333.33,144.64) .. (333.33,178.33) .. controls (333.33,212.02) and (327.96,239.33) .. (321.33,239.33) .. controls (314.71,239.33) and (309.33,212.02) .. (309.33,178.33) -- cycle ;

	% Text Node
	\draw (102,137) node    {$-$};
	% Text Node
	\draw (120,137) node    {$+$};
	% Text Node
	\draw (102,154) node    {$-$};
	% Text Node
	\draw (120,154) node    {$+$};
	% Text Node
	\draw (102,188) node    {$-$};
	% Text Node
	\draw (120,188) node    {$+$};
	% Text Node
	\draw (102,171) node    {$-$};
	% Text Node
	\draw (120,171) node    {$+$};
	% Text Node
	\draw (102,222) node    {$-$};
	% Text Node
	\draw (120,222) node    {$+$};
	% Text Node
	\draw (102,205) node    {$-$};
	% Text Node
	\draw (120,205) node    {$+$};
	% Text Node
	\draw (152,222) node    {$-$};
	% Text Node
	\draw (170,222) node    {$+$};
	% Text Node
	\draw (152,205) node    {$-$};
	% Text Node
	\draw (170,205) node    {$+$};
	% Text Node
	\draw (152,188) node    {$-$};
	% Text Node
	\draw (170,188) node    {$+$};
	% Text Node
	\draw (152,171) node    {$-$};
	% Text Node
	\draw (170,171) node    {$+$};
	% Text Node
	\draw (152,154) node    {$-$};
	% Text Node
	\draw (170,154) node    {$+$};
	% Text Node
	\draw (152,137) node    {$-$};
	% Text Node
	\draw (170,137) node    {$+$};
	% Text Node
	\draw (202,222) node    {$-$};
	% Text Node
	\draw (220,222) node    {$+$};
	% Text Node
	\draw (202,205) node    {$-$};
	% Text Node
	\draw (220,205) node    {$+$};
	% Text Node
	\draw (202,188) node    {$-$};
	% Text Node
	\draw (220,188) node    {$+$};
	% Text Node
	\draw (202,171) node    {$-$};
	% Text Node
	\draw (220,171) node    {$+$};
	% Text Node
	\draw (202,154) node    {$-$};
	% Text Node
	\draw (220,154) node    {$+$};
	% Text Node
	\draw (202,137) node    {$-$};
	% Text Node
	\draw (220,137) node    {$+$};
	% Text Node
	\draw (252,222) node    {$-$};
	% Text Node
	\draw (270,222) node    {$+$};
	% Text Node
	\draw (252,205) node    {$-$};
	% Text Node
	\draw (270,205) node    {$+$};
	% Text Node
	\draw (252,188) node    {$-$};
	% Text Node
	\draw (270,188) node    {$+$};
	% Text Node
	\draw (252,171) node    {$-$};
	% Text Node
	\draw (270,171) node    {$+$};
	% Text Node
	\draw (252,154) node    {$-$};
	% Text Node
	\draw (270,154) node    {$+$};
	% Text Node
	\draw (252,137) node    {$-$};
	% Text Node
	\draw (270,137) node    {$+$};
	% Text Node
	\draw (302,222) node    {$-$};
	% Text Node
	\draw (320,222) node    {$+$};
	% Text Node
	\draw (302,205) node    {$-$};
	% Text Node
	\draw (320,205) node    {$+$};
	% Text Node
	\draw (302,188) node    {$-$};
	% Text Node
	\draw (320,188) node    {$+$};
	% Text Node
	\draw (302,171) node    {$-$};
	% Text Node
	\draw (320,171) node    {$+$};
	% Text Node
	\draw (302,154) node    {$-$};
	% Text Node
	\draw (320,154) node    {$+$};
	% Text Node
	\draw (302,137) node    {$-$};
	% Text Node
	\draw (320,137) node    {$+$};


	\end{tikzpicture}
\end{figure}
\FloatBarrier
Per ogni carica negativa c'è sempre un altra carica positiva, la somma di tutte le cariche sarà zero. Inoltre, se prendiamo un qualunque volumetto all'interno del materiale, avremo sempre un uguale numero di cariche positive e negative. Possiamo dire che all'interno del dielettrico polarizzato uniformemente avviene una compensazione delle cariche spostate dalle posizioni di equilibrio. Ciò non accade sulla superficie limite, dove la discontinuità del mezzo impedisce la compensazione. Qui la carica è localizzata entro uno strato di spessore pari alle dimensioni atomiche ed è a tutti gli effetti trattabile come una distribuzione superficiale di carica, localizzata sulle facce. Definiremo $ \sigma_{p1}  $ la distribuzione di carica positiva e $ \sigma_{p2}  $ la distribuzione di carica negativa. Queste cariche sono dette \textbf{cariche di polarizzazione} e non sono libere come nei conduttori: esse si manifestano a causa degli spostamenti microscopici locali ma rimangono vincolate agli atomi o alle molecole. Per lo stesso motivo, quando cerchiamo di prelevarne un campione non riusciamo ad asportarne nemmeno una piccola quantità misurabile. Per quantificare queste cariche, dobbiamo fare alcuni ragionamenti.
Prima di tutto si avrà:
\[
	\sigma_{p1} = \frac{Q_p}{S} = - \sigma_{p2}
\]
Consideriamo un cubetto infinitesimo.
\begin{figure}[htpb]
	\centering
	

	\tikzset{every picture/.style={line width=0.75pt}} %set default line width to 0.75pt        

	\begin{tikzpicture}[x=0.75pt,y=0.75pt,yscale=-1,xscale=1]
	%uncomment if require: \path (0,300); %set diagram left start at 0, and has height of 300

	%Shape: Cube [id:dp8215866987656137] 
	\draw   (245,127.45) -- (284.45,88) -- (376.5,88) -- (376.5,182.55) -- (337.05,222) -- (245,222) -- cycle ; \draw   (376.5,88) -- (337.05,127.45) -- (245,127.45) ; \draw   (337.05,127.45) -- (337.05,222) ;
	%Straight Lines [id:da7254517296928906] 
	\draw  [dash pattern={on 0.84pt off 2.51pt}]  (284.45,182.55) -- (284.45,88) ;
	%Straight Lines [id:da28526796535119736] 
	\draw  [dash pattern={on 0.84pt off 2.51pt}]  (376.5,182.55) -- (284,182.55) ;
	%Straight Lines [id:da46518653987443304] 
	\draw  [dash pattern={on 0.84pt off 2.51pt}]  (284.45,182.55) -- (245.33,221.67) ;
	%Straight Lines [id:da08935691832654435] 
	\draw    (358.21,160) -- (434,160) ;
	\draw [shift={(437,160)}, rotate = 180] [fill={rgb, 255:red, 0; green, 0; blue, 0 }  ][line width=0.08]  [draw opacity=0] (10.72,-5.15) -- (0,0) -- (10.72,5.15) -- (7.12,0) -- cycle    ;
	%Straight Lines [id:da011424130890764994] 
	\draw    (263.55,158.67) -- (202.67,158.67) ;
	\draw [shift={(199.67,158.67)}, rotate = 360] [fill={rgb, 255:red, 0; green, 0; blue, 0 }  ][line width=0.08]  [draw opacity=0] (10.72,-5.15) -- (0,0) -- (10.72,5.15) -- (7.12,0) -- cycle    ;
	%Straight Lines [id:da5620390285149068] 
	\draw    (262.88,172) -- (354.67,172) ;
	\draw [shift={(357.67,172)}, rotate = 180] [fill={rgb, 255:red, 0; green, 0; blue, 0 }  ][line width=0.08]  [draw opacity=0] (10.72,-5.15) -- (0,0) -- (10.72,5.15) -- (7.12,0) -- cycle    ;
	%Straight Lines [id:da6129228168046186] 
	\draw    (245.33,234.67) -- (334.05,234.67) ;
	\draw [shift={(337.05,234.67)}, rotate = 180] [fill={rgb, 255:red, 0; green, 0; blue, 0 }  ][line width=0.08]  [draw opacity=0] (10.72,-5.15) -- (0,0) -- (10.72,5.15) -- (7.12,0) -- cycle    ;

	% Text Node
	\draw (359,126.67) node    {$dS$};
	% Text Node
	\draw (447.33,156) node    {$\vec{n}$};
	% Text Node
	\draw (307.33,155.33) node    {$\vec{p}$};
	% Text Node
	\draw (185,155.33) node    {$-\vec{n}$};
	% Text Node
	\draw (272.33,115.33) node    {$-$};
	% Text Node
	\draw (258.33,134) node    {$-$};
	% Text Node
	\draw (269.67,143.33) node    {$-$};
	% Text Node
	\draw (252.33,172.67) node    {$-$};
	% Text Node
	\draw (269,178.67) node    {$-$};
	% Text Node
	\draw (255,192.67) node    {$-$};
	% Text Node
	\draw (346.33,198) node    {$+$};
	% Text Node
	\draw (355,186.67) node    {$+$};
	% Text Node
	\draw (365.67,172) node    {$+$};
	% Text Node
	\draw (346.33,158) node    {$+$};
	% Text Node
	\draw (363,148.67) node    {$+$};
	% Text Node
	\draw (347,140.67) node    {$+$};
	% Text Node
	\draw (366.33,110) node    {$+$};
	% Text Node
	\draw (294.67,244.67) node    {$l$};


	\end{tikzpicture}
\end{figure}
\FloatBarrier
Definiamo $\vec{l}$ il vettore parallelo allo spigolo. Chiamiamo $dS$ la superficie di una faccia del cubo. Il momento di dipolo infinitesimo si potrà calcolare in due modi.
\[
	\left. \begin{array}{r}
	 	d\vec{p} = \vec{P} d\tau = \vec{P} \,l\,dS = P\,dS\,\vec{l} \\
		d\vec{p} = \sigma_p\,dS\,\vec{l}
	\end{array} \right\} \implies P = \sigma_p
\]
Ecco perché le dimensioni di un vettore di polarizzazione sono quelle di una densità di carica superficiale.
Definendo i vettori normali alle superfici, $\vec{n}$ e $-\vec{n}$, si avrà:
\[
	\boxed{\sigma_p = \vec{P} \cdot \vec{n}}
\]
Possiamo estendere il risultato a un dielettrico di forma qualunque, sempre polarizzato uniformemente. Se introduciamo la normale $\vec{n}$ sulla faccia inclinata, abbiamo un certo angolo $\vartheta$ e calcolando la densità di carica troveremo:
\[
	\sigma_{p1} = P \cos \vartheta \implies |\sigma_{p1}| < |\sigma_{p2} |
\]
Possiamo allora dire che in generale la densità di carica superficiale delle cariche di polarizzazione in un dielettrico è uguale alla componente di $\vec{P}$ lungo la normale alla superficie. Se la polarizzazione è uniforme non si manifestano cariche all'interno del dielettrico e quindi la carica totale superficiale deve essere nulla. Di conseguenza avremo sempre una parte della superficie carica positivamente e l'altra carica negativamente.
\begin{figure}[htpb]
	\centering
	

	\tikzset{every picture/.style={line width=0.75pt}} %set default line width to 0.75pt        

	\begin{tikzpicture}[x=0.75pt,y=0.75pt,yscale=-1,xscale=1]
	%uncomment if require: \path (0,300); %set diagram left start at 0, and has height of 300

	%Shape: Circle [id:dp5674579527452632] 
	\draw   (157.5,99.5) .. controls (157.5,95.36) and (160.86,92) .. (165,92) .. controls (169.14,92) and (172.5,95.36) .. (172.5,99.5) .. controls (172.5,103.64) and (169.14,107) .. (165,107) .. controls (160.86,107) and (157.5,103.64) .. (157.5,99.5) -- cycle ;

	%Shape: Circle [id:dp31248580205689125] 
	\draw   (139.5,99.5) .. controls (139.5,95.36) and (142.86,92) .. (147,92) .. controls (151.14,92) and (154.5,95.36) .. (154.5,99.5) .. controls (154.5,103.64) and (151.14,107) .. (147,107) .. controls (142.86,107) and (139.5,103.64) .. (139.5,99.5) -- cycle ;

	%Shape: Circle [id:dp24973894309400047] 
	\draw   (157.5,116.5) .. controls (157.5,112.36) and (160.86,109) .. (165,109) .. controls (169.14,109) and (172.5,112.36) .. (172.5,116.5) .. controls (172.5,120.64) and (169.14,124) .. (165,124) .. controls (160.86,124) and (157.5,120.64) .. (157.5,116.5) -- cycle ;

	%Shape: Circle [id:dp5348710377486483] 
	\draw   (139.5,116.5) .. controls (139.5,112.36) and (142.86,109) .. (147,109) .. controls (151.14,109) and (154.5,112.36) .. (154.5,116.5) .. controls (154.5,120.64) and (151.14,124) .. (147,124) .. controls (142.86,124) and (139.5,120.64) .. (139.5,116.5) -- cycle ;

	%Shape: Circle [id:dp2624224603993852] 
	\draw   (157.5,133.5) .. controls (157.5,129.36) and (160.86,126) .. (165,126) .. controls (169.14,126) and (172.5,129.36) .. (172.5,133.5) .. controls (172.5,137.64) and (169.14,141) .. (165,141) .. controls (160.86,141) and (157.5,137.64) .. (157.5,133.5) -- cycle ;

	%Shape: Circle [id:dp9126413471984727] 
	\draw   (139.5,133.5) .. controls (139.5,129.36) and (142.86,126) .. (147,126) .. controls (151.14,126) and (154.5,129.36) .. (154.5,133.5) .. controls (154.5,137.64) and (151.14,141) .. (147,141) .. controls (142.86,141) and (139.5,137.64) .. (139.5,133.5) -- cycle ;

	%Shape: Circle [id:dp5137035004474502] 
	\draw   (157.5,150.5) .. controls (157.5,146.36) and (160.86,143) .. (165,143) .. controls (169.14,143) and (172.5,146.36) .. (172.5,150.5) .. controls (172.5,154.64) and (169.14,158) .. (165,158) .. controls (160.86,158) and (157.5,154.64) .. (157.5,150.5) -- cycle ;

	%Shape: Circle [id:dp08195447101411246] 
	\draw   (139.5,150.5) .. controls (139.5,146.36) and (142.86,143) .. (147,143) .. controls (151.14,143) and (154.5,146.36) .. (154.5,150.5) .. controls (154.5,154.64) and (151.14,158) .. (147,158) .. controls (142.86,158) and (139.5,154.64) .. (139.5,150.5) -- cycle ;

	%Shape: Circle [id:dp04241561070984834] 
	\draw   (157.5,167.5) .. controls (157.5,163.36) and (160.86,160) .. (165,160) .. controls (169.14,160) and (172.5,163.36) .. (172.5,167.5) .. controls (172.5,171.64) and (169.14,175) .. (165,175) .. controls (160.86,175) and (157.5,171.64) .. (157.5,167.5) -- cycle ;

	%Shape: Circle [id:dp6697945523742617] 
	\draw   (139.5,167.5) .. controls (139.5,163.36) and (142.86,160) .. (147,160) .. controls (151.14,160) and (154.5,163.36) .. (154.5,167.5) .. controls (154.5,171.64) and (151.14,175) .. (147,175) .. controls (142.86,175) and (139.5,171.64) .. (139.5,167.5) -- cycle ;

	%Shape: Circle [id:dp7972299761937769] 
	\draw   (157.5,184.5) .. controls (157.5,180.36) and (160.86,177) .. (165,177) .. controls (169.14,177) and (172.5,180.36) .. (172.5,184.5) .. controls (172.5,188.64) and (169.14,192) .. (165,192) .. controls (160.86,192) and (157.5,188.64) .. (157.5,184.5) -- cycle ;

	%Shape: Circle [id:dp4370732695403827] 
	\draw   (139.5,184.5) .. controls (139.5,180.36) and (142.86,177) .. (147,177) .. controls (151.14,177) and (154.5,180.36) .. (154.5,184.5) .. controls (154.5,188.64) and (151.14,192) .. (147,192) .. controls (142.86,192) and (139.5,188.64) .. (139.5,184.5) -- cycle ;

	%Shape: Circle [id:dp4751475486684724] 
	\draw   (207.5,99.5) .. controls (207.5,95.36) and (210.86,92) .. (215,92) .. controls (219.14,92) and (222.5,95.36) .. (222.5,99.5) .. controls (222.5,103.64) and (219.14,107) .. (215,107) .. controls (210.86,107) and (207.5,103.64) .. (207.5,99.5) -- cycle ;

	%Shape: Circle [id:dp12771181457548964] 
	\draw   (189.5,99.5) .. controls (189.5,95.36) and (192.86,92) .. (197,92) .. controls (201.14,92) and (204.5,95.36) .. (204.5,99.5) .. controls (204.5,103.64) and (201.14,107) .. (197,107) .. controls (192.86,107) and (189.5,103.64) .. (189.5,99.5) -- cycle ;

	%Shape: Circle [id:dp4385629415265657] 
	\draw   (207.5,116.5) .. controls (207.5,112.36) and (210.86,109) .. (215,109) .. controls (219.14,109) and (222.5,112.36) .. (222.5,116.5) .. controls (222.5,120.64) and (219.14,124) .. (215,124) .. controls (210.86,124) and (207.5,120.64) .. (207.5,116.5) -- cycle ;

	%Shape: Circle [id:dp12359862768827745] 
	\draw   (189.5,116.5) .. controls (189.5,112.36) and (192.86,109) .. (197,109) .. controls (201.14,109) and (204.5,112.36) .. (204.5,116.5) .. controls (204.5,120.64) and (201.14,124) .. (197,124) .. controls (192.86,124) and (189.5,120.64) .. (189.5,116.5) -- cycle ;

	%Shape: Circle [id:dp2783544070113577] 
	\draw   (207.5,133.5) .. controls (207.5,129.36) and (210.86,126) .. (215,126) .. controls (219.14,126) and (222.5,129.36) .. (222.5,133.5) .. controls (222.5,137.64) and (219.14,141) .. (215,141) .. controls (210.86,141) and (207.5,137.64) .. (207.5,133.5) -- cycle ;

	%Shape: Circle [id:dp03829939483390721] 
	\draw   (189.5,133.5) .. controls (189.5,129.36) and (192.86,126) .. (197,126) .. controls (201.14,126) and (204.5,129.36) .. (204.5,133.5) .. controls (204.5,137.64) and (201.14,141) .. (197,141) .. controls (192.86,141) and (189.5,137.64) .. (189.5,133.5) -- cycle ;

	%Shape: Circle [id:dp46350221314251816] 
	\draw   (207.5,150.5) .. controls (207.5,146.36) and (210.86,143) .. (215,143) .. controls (219.14,143) and (222.5,146.36) .. (222.5,150.5) .. controls (222.5,154.64) and (219.14,158) .. (215,158) .. controls (210.86,158) and (207.5,154.64) .. (207.5,150.5) -- cycle ;

	%Shape: Circle [id:dp28725369404867473] 
	\draw   (189.5,150.5) .. controls (189.5,146.36) and (192.86,143) .. (197,143) .. controls (201.14,143) and (204.5,146.36) .. (204.5,150.5) .. controls (204.5,154.64) and (201.14,158) .. (197,158) .. controls (192.86,158) and (189.5,154.64) .. (189.5,150.5) -- cycle ;

	%Shape: Circle [id:dp5658901230662008] 
	\draw   (207.5,167.5) .. controls (207.5,163.36) and (210.86,160) .. (215,160) .. controls (219.14,160) and (222.5,163.36) .. (222.5,167.5) .. controls (222.5,171.64) and (219.14,175) .. (215,175) .. controls (210.86,175) and (207.5,171.64) .. (207.5,167.5) -- cycle ;

	%Shape: Circle [id:dp20560481947709341] 
	\draw   (189.5,167.5) .. controls (189.5,163.36) and (192.86,160) .. (197,160) .. controls (201.14,160) and (204.5,163.36) .. (204.5,167.5) .. controls (204.5,171.64) and (201.14,175) .. (197,175) .. controls (192.86,175) and (189.5,171.64) .. (189.5,167.5) -- cycle ;

	%Shape: Circle [id:dp32754842423634556] 
	\draw   (207.5,184.5) .. controls (207.5,180.36) and (210.86,177) .. (215,177) .. controls (219.14,177) and (222.5,180.36) .. (222.5,184.5) .. controls (222.5,188.64) and (219.14,192) .. (215,192) .. controls (210.86,192) and (207.5,188.64) .. (207.5,184.5) -- cycle ;

	%Shape: Circle [id:dp37879556764976674] 
	\draw   (189.5,184.5) .. controls (189.5,180.36) and (192.86,177) .. (197,177) .. controls (201.14,177) and (204.5,180.36) .. (204.5,184.5) .. controls (204.5,188.64) and (201.14,192) .. (197,192) .. controls (192.86,192) and (189.5,188.64) .. (189.5,184.5) -- cycle ;

	%Shape: Circle [id:dp3906228362632065] 
	\draw   (239.5,99.5) .. controls (239.5,95.36) and (242.86,92) .. (247,92) .. controls (251.14,92) and (254.5,95.36) .. (254.5,99.5) .. controls (254.5,103.64) and (251.14,107) .. (247,107) .. controls (242.86,107) and (239.5,103.64) .. (239.5,99.5) -- cycle ;

	%Shape: Circle [id:dp39446739583773116] 
	\draw   (257.5,116.5) .. controls (257.5,112.36) and (260.86,109) .. (265,109) .. controls (269.14,109) and (272.5,112.36) .. (272.5,116.5) .. controls (272.5,120.64) and (269.14,124) .. (265,124) .. controls (260.86,124) and (257.5,120.64) .. (257.5,116.5) -- cycle ;

	%Shape: Circle [id:dp8222104547960365] 
	\draw   (239.5,116.5) .. controls (239.5,112.36) and (242.86,109) .. (247,109) .. controls (251.14,109) and (254.5,112.36) .. (254.5,116.5) .. controls (254.5,120.64) and (251.14,124) .. (247,124) .. controls (242.86,124) and (239.5,120.64) .. (239.5,116.5) -- cycle ;

	%Shape: Circle [id:dp7506262046541383] 
	\draw   (257.5,133.5) .. controls (257.5,129.36) and (260.86,126) .. (265,126) .. controls (269.14,126) and (272.5,129.36) .. (272.5,133.5) .. controls (272.5,137.64) and (269.14,141) .. (265,141) .. controls (260.86,141) and (257.5,137.64) .. (257.5,133.5) -- cycle ;

	%Shape: Circle [id:dp3625817781602867] 
	\draw   (239.5,133.5) .. controls (239.5,129.36) and (242.86,126) .. (247,126) .. controls (251.14,126) and (254.5,129.36) .. (254.5,133.5) .. controls (254.5,137.64) and (251.14,141) .. (247,141) .. controls (242.86,141) and (239.5,137.64) .. (239.5,133.5) -- cycle ;

	%Shape: Circle [id:dp42002419402672997] 
	\draw   (257.5,150.5) .. controls (257.5,146.36) and (260.86,143) .. (265,143) .. controls (269.14,143) and (272.5,146.36) .. (272.5,150.5) .. controls (272.5,154.64) and (269.14,158) .. (265,158) .. controls (260.86,158) and (257.5,154.64) .. (257.5,150.5) -- cycle ;

	%Shape: Circle [id:dp5742957539476088] 
	\draw   (239.5,150.5) .. controls (239.5,146.36) and (242.86,143) .. (247,143) .. controls (251.14,143) and (254.5,146.36) .. (254.5,150.5) .. controls (254.5,154.64) and (251.14,158) .. (247,158) .. controls (242.86,158) and (239.5,154.64) .. (239.5,150.5) -- cycle ;

	%Shape: Circle [id:dp7855747581921195] 
	\draw   (257.5,167.5) .. controls (257.5,163.36) and (260.86,160) .. (265,160) .. controls (269.14,160) and (272.5,163.36) .. (272.5,167.5) .. controls (272.5,171.64) and (269.14,175) .. (265,175) .. controls (260.86,175) and (257.5,171.64) .. (257.5,167.5) -- cycle ;

	%Shape: Circle [id:dp25976337027338436] 
	\draw   (239.5,167.5) .. controls (239.5,163.36) and (242.86,160) .. (247,160) .. controls (251.14,160) and (254.5,163.36) .. (254.5,167.5) .. controls (254.5,171.64) and (251.14,175) .. (247,175) .. controls (242.86,175) and (239.5,171.64) .. (239.5,167.5) -- cycle ;

	%Shape: Circle [id:dp06350917894276797] 
	\draw   (257.5,184.5) .. controls (257.5,180.36) and (260.86,177) .. (265,177) .. controls (269.14,177) and (272.5,180.36) .. (272.5,184.5) .. controls (272.5,188.64) and (269.14,192) .. (265,192) .. controls (260.86,192) and (257.5,188.64) .. (257.5,184.5) -- cycle ;

	%Shape: Circle [id:dp49110512077037516] 
	\draw   (239.5,184.5) .. controls (239.5,180.36) and (242.86,177) .. (247,177) .. controls (251.14,177) and (254.5,180.36) .. (254.5,184.5) .. controls (254.5,188.64) and (251.14,192) .. (247,192) .. controls (242.86,192) and (239.5,188.64) .. (239.5,184.5) -- cycle ;

	%Shape: Circle [id:dp29810898502622596] 
	\draw   (289.5,150.5) .. controls (289.5,146.36) and (292.86,143) .. (297,143) .. controls (301.14,143) and (304.5,146.36) .. (304.5,150.5) .. controls (304.5,154.64) and (301.14,158) .. (297,158) .. controls (292.86,158) and (289.5,154.64) .. (289.5,150.5) -- cycle ;

	%Shape: Circle [id:dp021078483094720823] 
	\draw   (307.5,167.5) .. controls (307.5,163.36) and (310.86,160) .. (315,160) .. controls (319.14,160) and (322.5,163.36) .. (322.5,167.5) .. controls (322.5,171.64) and (319.14,175) .. (315,175) .. controls (310.86,175) and (307.5,171.64) .. (307.5,167.5) -- cycle ;

	%Shape: Circle [id:dp001341301351222679] 
	\draw   (289.5,167.5) .. controls (289.5,163.36) and (292.86,160) .. (297,160) .. controls (301.14,160) and (304.5,163.36) .. (304.5,167.5) .. controls (304.5,171.64) and (301.14,175) .. (297,175) .. controls (292.86,175) and (289.5,171.64) .. (289.5,167.5) -- cycle ;

	%Shape: Circle [id:dp2137227377404305] 
	\draw   (307.5,184.5) .. controls (307.5,180.36) and (310.86,177) .. (315,177) .. controls (319.14,177) and (322.5,180.36) .. (322.5,184.5) .. controls (322.5,188.64) and (319.14,192) .. (315,192) .. controls (310.86,192) and (307.5,188.64) .. (307.5,184.5) -- cycle ;

	%Shape: Circle [id:dp7497044454766757] 
	\draw   (289.5,184.5) .. controls (289.5,180.36) and (292.86,177) .. (297,177) .. controls (301.14,177) and (304.5,180.36) .. (304.5,184.5) .. controls (304.5,188.64) and (301.14,192) .. (297,192) .. controls (292.86,192) and (289.5,188.64) .. (289.5,184.5) -- cycle ;

	%Straight Lines [id:da3254115736577341] 
	\draw    (172.5,99.5) -- (189.5,99.5) ;
	%Straight Lines [id:da8064887731786978] 
	\draw    (172.5,116.5) -- (189.5,116.5) ;
	%Straight Lines [id:da2547937464010419] 
	\draw    (172.5,133.5) -- (189.5,133.5) ;
	%Straight Lines [id:da733773613619549] 
	\draw    (172.5,150.5) -- (189.5,150.5) ;
	%Straight Lines [id:da5834847016913209] 
	\draw    (172.5,167.5) -- (189.5,167.5) ;
	%Straight Lines [id:da8760325679403016] 
	\draw    (172.5,184.5) -- (189.5,184.5) ;
	%Straight Lines [id:da6628371312751844] 
	\draw    (222.5,99.5) -- (239.5,99.5) ;
	%Straight Lines [id:da2388876028335436] 
	\draw    (222.5,116.5) -- (239.5,116.5) ;
	%Straight Lines [id:da5423634193431548] 
	\draw    (222.5,133.5) -- (239.5,133.5) ;
	%Straight Lines [id:da8753153495110384] 
	\draw    (222.5,150.5) -- (239.5,150.5) ;
	%Straight Lines [id:da7665314297040351] 
	\draw    (222.5,167.5) -- (239.5,167.5) ;
	%Straight Lines [id:da731166079179139] 
	\draw    (222.5,184.5) -- (239.5,184.5) ;
	%Straight Lines [id:da47015521173307984] 
	\draw    (272.5,150.5) -- (289.5,150.5) ;
	%Straight Lines [id:da17894529311159602] 
	\draw    (272.5,167.5) -- (289.5,167.5) ;
	%Straight Lines [id:da5383428168783206] 
	\draw    (272.5,184.5) -- (289.5,184.5) ;
	%Straight Lines [id:da8320236171918673] 
	\draw    (288.17,129.83) -- (372.17,129.83) ;
	\draw [shift={(375.17,129.83)}, rotate = 180] [fill={rgb, 255:red, 0; green, 0; blue, 0 }  ][line width=0.08]  [draw opacity=0] (10.72,-5.15) -- (0,0) -- (10.72,5.15) -- (7.12,0) -- cycle    ;
	%Shape: Polygon [id:ds9601352475932101] 
	\draw   (243.5,84) -- (354.5,197.33) -- (132.5,197.33) -- (132.5,84) -- cycle ;
	%Straight Lines [id:da3940010496093549] 
	\draw    (288.17,129.83) -- (333.52,85.59) ;
	\draw [shift={(335.67,83.5)}, rotate = 495.71] [fill={rgb, 255:red, 0; green, 0; blue, 0 }  ][line width=0.08]  [draw opacity=0] (10.72,-5.15) -- (0,0) -- (10.72,5.15) -- (7.12,0) -- cycle    ;

	% Text Node
	\draw (297,182) node    {$-$};
	% Text Node
	\draw (315,182) node    {$+$};
	% Text Node
	\draw (297,165) node    {$-$};
	% Text Node
	\draw (315,165) node    {$+$};
	% Text Node
	\draw (297,148) node    {$-$};
	% Text Node
	\draw (247,182) node    {$-$};
	% Text Node
	\draw (265,182) node    {$+$};
	% Text Node
	\draw (247,165) node    {$-$};
	% Text Node
	\draw (265,165) node    {$+$};
	% Text Node
	\draw (247,148) node    {$-$};
	% Text Node
	\draw (265,148) node    {$+$};
	% Text Node
	\draw (247,131) node    {$-$};
	% Text Node
	\draw (265,131) node    {$+$};
	% Text Node
	\draw (247,114) node    {$-$};
	% Text Node
	\draw (265,114) node    {$+$};
	% Text Node
	\draw (247,97) node    {$-$};
	% Text Node
	\draw (197,182) node    {$-$};
	% Text Node
	\draw (215,182) node    {$+$};
	% Text Node
	\draw (197,165) node    {$-$};
	% Text Node
	\draw (215,165) node    {$+$};
	% Text Node
	\draw (197,148) node    {$-$};
	% Text Node
	\draw (215,148) node    {$+$};
	% Text Node
	\draw (197,131) node    {$-$};
	% Text Node
	\draw (215,131) node    {$+$};
	% Text Node
	\draw (197,114) node    {$-$};
	% Text Node
	\draw (215,114) node    {$+$};
	% Text Node
	\draw (197,97) node    {$-$};
	% Text Node
	\draw (215,97) node    {$+$};
	% Text Node
	\draw (147,182) node    {$-$};
	% Text Node
	\draw (165,182) node    {$+$};
	% Text Node
	\draw (147,165) node    {$-$};
	% Text Node
	\draw (165,165) node    {$+$};
	% Text Node
	\draw (147,148) node    {$-$};
	% Text Node
	\draw (165,148) node    {$+$};
	% Text Node
	\draw (147,131) node    {$-$};
	% Text Node
	\draw (165,131) node    {$+$};
	% Text Node
	\draw (147,114) node    {$-$};
	% Text Node
	\draw (165,114) node    {$+$};
	% Text Node
	\draw (147,97) node    {$-$};
	% Text Node
	\draw (165,97) node    {$+$};
	% Text Node
	\draw (384.67,127) node    {$\vec{P}$};
	% Text Node
	\draw (346,75) node    {$\vec{n}$};


	\end{tikzpicture}
\end{figure}
\FloatBarrier
Consideriamo il caso di un dielettrico polarizzato non uniformemente. In questa situazione la quantità di carica totale è ancora zero.
\begin{figure}[htpb]
	\centering
	

	\tikzset{every picture/.style={line width=0.75pt}} %set default line width to 0.75pt        

	\begin{tikzpicture}[x=0.75pt,y=0.75pt,yscale=-1,xscale=1]
	%uncomment if require: \path (0,300); %set diagram left start at 0, and has height of 300

	%Shape: Circle [id:dp3997031955099213] 
	\draw   (177.5,119.5) .. controls (177.5,115.36) and (180.86,112) .. (185,112) .. controls (189.14,112) and (192.5,115.36) .. (192.5,119.5) .. controls (192.5,123.64) and (189.14,127) .. (185,127) .. controls (180.86,127) and (177.5,123.64) .. (177.5,119.5) -- cycle ;

	%Shape: Circle [id:dp4349931488939849] 
	\draw   (159.5,119.5) .. controls (159.5,115.36) and (162.86,112) .. (167,112) .. controls (171.14,112) and (174.5,115.36) .. (174.5,119.5) .. controls (174.5,123.64) and (171.14,127) .. (167,127) .. controls (162.86,127) and (159.5,123.64) .. (159.5,119.5) -- cycle ;


	%Shape: Circle [id:dp581016149468] 
	\draw   (184,155.5) .. controls (184,151.36) and (187.36,148) .. (191.5,148) .. controls (195.64,148) and (199,151.36) .. (199,155.5) .. controls (199,159.64) and (195.64,163) .. (191.5,163) .. controls (187.36,163) and (184,159.64) .. (184,155.5) -- cycle ;

	%Shape: Circle [id:dp6863631747381511] 
	\draw   (166,155.5) .. controls (166,151.36) and (169.36,148) .. (173.5,148) .. controls (177.64,148) and (181,151.36) .. (181,155.5) .. controls (181,159.64) and (177.64,163) .. (173.5,163) .. controls (169.36,163) and (166,159.64) .. (166,155.5) -- cycle ;


	%Shape: Circle [id:dp689682689487148] 
	\draw   (177,173.5) .. controls (177,169.36) and (180.36,166) .. (184.5,166) .. controls (188.64,166) and (192,169.36) .. (192,173.5) .. controls (192,177.64) and (188.64,181) .. (184.5,181) .. controls (180.36,181) and (177,177.64) .. (177,173.5) -- cycle ;

	%Shape: Circle [id:dp3133810354139708] 
	\draw   (159,173.5) .. controls (159,169.36) and (162.36,166) .. (166.5,166) .. controls (170.64,166) and (174,169.36) .. (174,173.5) .. controls (174,177.64) and (170.64,181) .. (166.5,181) .. controls (162.36,181) and (159,177.64) .. (159,173.5) -- cycle ;


	%Shape: Circle [id:dp058911854760762994] 
	\draw   (180,204) .. controls (180,199.86) and (183.36,196.5) .. (187.5,196.5) .. controls (191.64,196.5) and (195,199.86) .. (195,204) .. controls (195,208.14) and (191.64,211.5) .. (187.5,211.5) .. controls (183.36,211.5) and (180,208.14) .. (180,204) -- cycle ;

	%Shape: Circle [id:dp568686883862441] 
	\draw   (162,204) .. controls (162,199.86) and (165.36,196.5) .. (169.5,196.5) .. controls (173.64,196.5) and (177,199.86) .. (177,204) .. controls (177,208.14) and (173.64,211.5) .. (169.5,211.5) .. controls (165.36,211.5) and (162,208.14) .. (162,204) -- cycle ;


	%Shape: Circle [id:dp19525928763138922] 
	\draw   (210.5,188.5) .. controls (210.5,184.36) and (213.86,181) .. (218,181) .. controls (222.14,181) and (225.5,184.36) .. (225.5,188.5) .. controls (225.5,192.64) and (222.14,196) .. (218,196) .. controls (213.86,196) and (210.5,192.64) .. (210.5,188.5) -- cycle ;

	%Shape: Circle [id:dp256767487144576] 
	\draw   (192.5,188.5) .. controls (192.5,184.36) and (195.86,181) .. (200,181) .. controls (204.14,181) and (207.5,184.36) .. (207.5,188.5) .. controls (207.5,192.64) and (204.14,196) .. (200,196) .. controls (195.86,196) and (192.5,192.64) .. (192.5,188.5) -- cycle ;


	%Straight Lines [id:da6552875977852031] 
	\draw    (308.17,149.83) -- (392.17,149.83) ;
	\draw [shift={(395.17,149.83)}, rotate = 180] [fill={rgb, 255:red, 0; green, 0; blue, 0 }  ][line width=0.08]  [draw opacity=0] (10.72,-5.15) -- (0,0) -- (10.72,5.15) -- (7.12,0) -- cycle    ;
	%Shape: Polygon [id:ds6462598494864005] 
	\draw   (263.5,104) -- (374.5,217.33) -- (152.5,217.33) -- (152.5,104) -- cycle ;
	%Straight Lines [id:da4641669338489087] 
	\draw    (308.17,149.83) -- (353.52,105.59) ;
	\draw [shift={(355.67,103.5)}, rotate = 495.71] [fill={rgb, 255:red, 0; green, 0; blue, 0 }  ][line width=0.08]  [draw opacity=0] (10.72,-5.15) -- (0,0) -- (10.72,5.15) -- (7.12,0) -- cycle    ;
	%Shape: Circle [id:dp7541837135978344] 
	\draw   (247,121.5) .. controls (247,117.36) and (250.36,114) .. (254.5,114) .. controls (258.64,114) and (262,117.36) .. (262,121.5) .. controls (262,125.64) and (258.64,129) .. (254.5,129) .. controls (250.36,129) and (247,125.64) .. (247,121.5) -- cycle ;

	%Shape: Circle [id:dp8239938142851762] 
	\draw   (229,121.5) .. controls (229,117.36) and (232.36,114) .. (236.5,114) .. controls (240.64,114) and (244,117.36) .. (244,121.5) .. controls (244,125.64) and (240.64,129) .. (236.5,129) .. controls (232.36,129) and (229,125.64) .. (229,121.5) -- cycle ;


	%Shape: Circle [id:dp48430591279413493] 
	\draw   (260,142.5) .. controls (260,138.36) and (263.36,135) .. (267.5,135) .. controls (271.64,135) and (275,138.36) .. (275,142.5) .. controls (275,146.64) and (271.64,150) .. (267.5,150) .. controls (263.36,150) and (260,146.64) .. (260,142.5) -- cycle ;

	%Shape: Circle [id:dp9983815651307262] 
	\draw   (242,142.5) .. controls (242,138.36) and (245.36,135) .. (249.5,135) .. controls (253.64,135) and (257,138.36) .. (257,142.5) .. controls (257,146.64) and (253.64,150) .. (249.5,150) .. controls (245.36,150) and (242,146.64) .. (242,142.5) -- cycle ;


	%Shape: Circle [id:dp9858162727509685] 
	\draw   (221.5,155) .. controls (221.5,150.86) and (224.86,147.5) .. (229,147.5) .. controls (233.14,147.5) and (236.5,150.86) .. (236.5,155) .. controls (236.5,159.14) and (233.14,162.5) .. (229,162.5) .. controls (224.86,162.5) and (221.5,159.14) .. (221.5,155) -- cycle ;

	%Shape: Circle [id:dp13823482414225197] 
	\draw   (203.5,155) .. controls (203.5,150.86) and (206.86,147.5) .. (211,147.5) .. controls (215.14,147.5) and (218.5,150.86) .. (218.5,155) .. controls (218.5,159.14) and (215.14,162.5) .. (211,162.5) .. controls (206.86,162.5) and (203.5,159.14) .. (203.5,155) -- cycle ;


	%Shape: Circle [id:dp21970221878274065] 
	\draw   (275,165) .. controls (275,160.86) and (278.36,157.5) .. (282.5,157.5) .. controls (286.64,157.5) and (290,160.86) .. (290,165) .. controls (290,169.14) and (286.64,172.5) .. (282.5,172.5) .. controls (278.36,172.5) and (275,169.14) .. (275,165) -- cycle ;

	%Shape: Circle [id:dp7778845902481999] 
	\draw   (257,165) .. controls (257,160.86) and (260.36,157.5) .. (264.5,157.5) .. controls (268.64,157.5) and (272,160.86) .. (272,165) .. controls (272,169.14) and (268.64,172.5) .. (264.5,172.5) .. controls (260.36,172.5) and (257,169.14) .. (257,165) -- cycle ;


	%Shape: Circle [id:dp3400563060095192] 
	\draw   (322,195.5) .. controls (322,191.36) and (325.36,188) .. (329.5,188) .. controls (333.64,188) and (337,191.36) .. (337,195.5) .. controls (337,199.64) and (333.64,203) .. (329.5,203) .. controls (325.36,203) and (322,199.64) .. (322,195.5) -- cycle ;

	%Shape: Circle [id:dp37013071315348856] 
	\draw   (304,195.5) .. controls (304,191.36) and (307.36,188) .. (311.5,188) .. controls (315.64,188) and (319,191.36) .. (319,195.5) .. controls (319,199.64) and (315.64,203) .. (311.5,203) .. controls (307.36,203) and (304,199.64) .. (304,195.5) -- cycle ;


	%Shape: Circle [id:dp7727269308430249] 
	\draw   (249.5,202.5) .. controls (249.5,198.36) and (252.86,195) .. (257,195) .. controls (261.14,195) and (264.5,198.36) .. (264.5,202.5) .. controls (264.5,206.64) and (261.14,210) .. (257,210) .. controls (252.86,210) and (249.5,206.64) .. (249.5,202.5) -- cycle ;

	%Shape: Circle [id:dp2687878363233702] 
	\draw   (231.5,202.5) .. controls (231.5,198.36) and (234.86,195) .. (239,195) .. controls (243.14,195) and (246.5,198.36) .. (246.5,202.5) .. controls (246.5,206.64) and (243.14,210) .. (239,210) .. controls (234.86,210) and (231.5,206.64) .. (231.5,202.5) -- cycle ;


	%Shape: Circle [id:dp027664617434867145] 
	\draw   (221,171.5) .. controls (221,167.36) and (224.36,164) .. (228.5,164) .. controls (232.64,164) and (236,167.36) .. (236,171.5) .. controls (236,175.64) and (232.64,179) .. (228.5,179) .. controls (224.36,179) and (221,175.64) .. (221,171.5) -- cycle ;

	%Shape: Circle [id:dp3053253298206451] 
	\draw   (203,171.5) .. controls (203,167.36) and (206.36,164) .. (210.5,164) .. controls (214.64,164) and (218,167.36) .. (218,171.5) .. controls (218,175.64) and (214.64,179) .. (210.5,179) .. controls (206.36,179) and (203,175.64) .. (203,171.5) -- cycle ;


	%Shape: Circle [id:dp7973636898934802] 
	\draw   (288.5,186.5) .. controls (288.5,182.36) and (291.86,179) .. (296,179) .. controls (300.14,179) and (303.5,182.36) .. (303.5,186.5) .. controls (303.5,190.64) and (300.14,194) .. (296,194) .. controls (291.86,194) and (288.5,190.64) .. (288.5,186.5) -- cycle ;

	%Shape: Circle [id:dp5358553850599468] 
	\draw   (270.5,186.5) .. controls (270.5,182.36) and (273.86,179) .. (278,179) .. controls (282.14,179) and (285.5,182.36) .. (285.5,186.5) .. controls (285.5,190.64) and (282.14,194) .. (278,194) .. controls (273.86,194) and (270.5,190.64) .. (270.5,186.5) -- cycle ;


	%Shape: Circle [id:dp7802639170581402] 
	\draw   (175,171) .. controls (175,157.19) and (186.19,146) .. (200,146) .. controls (213.81,146) and (225,157.19) .. (225,171) .. controls (225,184.81) and (213.81,196) .. (200,196) .. controls (186.19,196) and (175,184.81) .. (175,171) -- cycle ;

	% Text Node
	\draw (404.67,147) node    {$\vec{P}$};
	% Text Node
	\draw (366,95) node    {$\vec{n}$};
	% Text Node
	\draw (200,186) node    {$-$};
	% Text Node
	\draw (218,186) node    {$+$};
	% Text Node
	\draw (169.5,201.5) node    {$-$};
	% Text Node
	\draw (187.5,201.5) node    {$+$};
	% Text Node
	\draw (166.5,171) node    {$-$};
	% Text Node
	\draw (184.5,171) node    {$+$};
	% Text Node
	\draw (173.5,153) node    {$-$};
	% Text Node
	\draw (191.5,153) node    {$+$};
	% Text Node
	\draw (167,117) node    {$-$};
	% Text Node
	\draw (185,117) node    {$+$};
	% Text Node
	\draw (236.5,119) node    {$-$};
	% Text Node
	\draw (254.5,119) node    {$+$};
	% Text Node
	\draw (249.5,140) node    {$-$};
	% Text Node
	\draw (267.5,140) node    {$+$};
	% Text Node
	\draw (211,152.5) node    {$-$};
	% Text Node
	\draw (229,152.5) node    {$+$};
	% Text Node
	\draw (264.5,162.5) node    {$-$};
	% Text Node
	\draw (282.5,162.5) node    {$+$};
	% Text Node
	\draw (311.5,193) node    {$-$};
	% Text Node
	\draw (329.5,193) node    {$+$};
	% Text Node
	\draw (239,200) node    {$-$};
	% Text Node
	\draw (257,200) node    {$+$};
	% Text Node
	\draw (210.5,169) node    {$-$};
	% Text Node
	\draw (228.5,169) node    {$+$};
	% Text Node
	\draw (278,184) node    {$-$};
	% Text Node
	\draw (296,184) node    {$+$};


	\end{tikzpicture}
\end{figure}
\FloatBarrier
Ma notiamo che se a questo punto prendiamo un volumetto all'interno del materiale, potremmo trovare un numero diverso di cariche positive o negative. Ecco quindi che possiamo misurare la carica in un volume con la densità di carica di polarizzazione per unità di volume, $ \rho (P) $. Quindi in un dielettrico in cui la polarizzazione non sia uniforme, oltre alla densità superficiale esiste una densità spaziale di carica di polarizzazione.
Detta $Q_p$ la carica totale presente nel dielettrico, si ha che, dovendo essere nulla:
\[
	Q_p = 0 = \int_{\Sigma}\sigma_pdS + \int_{\tau} \rho_pd\tau = \int_{\Sigma}\vec{P} \cdot \vec{n} dS + \int_{\tau} \rho_p d\tau
\]
Per il teorema della divergenza, il primo integrale può essere trasformato
nell'integrale di volume. Si avrà perciò:
\begin{align*}
	\int_{\Sigma}\vec{P} \cdot \vec{n} dS + \int_{\tau} \rho_p d\tau &= 0 \\
	\int_{\tau}\text{div}\vec{P} \, d\tau + \int_{\tau} \rho_p d\tau &= 0 \\
	\int_{\tau} \left[ \text{div}\vec{P} + \rho_p  \right] &= 0
\end{align*}
Le due distribuzioni di carica superficiale e spaziale si compensano globalmente dando carica totale nulla.
\[
	\boxed{\rho_p = - \text{div}\vec{P}}
\]
La conoscenza della polarizzazione $\vec{P}$ permette dunque di calcolare il potenziale e il campo elettrico in ogni punto $Q$ esterno al dielettrico. Avevamo infatti visto che il potenziale elettrostatico di distribuzioni continue di carica rispettivamente superficiali e volumetriche si calcola come:
\[
	V(x,y,z) = \frac{1}{4\pi \varepsilon_0}\int_{\Sigma}\frac{\sigma dS}{r} \qquad V(x,y,z) = \frac{1}{4\pi \varepsilon_0} \int_{\tau} \frac{\rho d\tau}{r}
\]
il quindi il potenziale di un dielettrico sarà dato da:
\begin{align*}
	V(x,y,z) &= \frac{1}{4\pi \varepsilon_0}\int_{\Sigma}\frac{\sigma_p  dS}{r} + \frac{1}{4\pi \varepsilon_0} \int_{\tau} \frac{\rho_p  d\tau}{r} \\
	&= \frac{1}{4\pi \varepsilon_0} \left[ \int_{\Sigma}\frac{\vec{P} \cdot \vec{n}}{r}dS + \int_{\tau}\frac{\vec{\nabla} \cdot \vec{P}}{r} d\tau  \right]
\end{align*}
Dove $r$ è la distanza da $Q$ all'elemento di carica di polarizzazione. $\vec{E}$ sarà semplicemente $ \vec{E} = - \text{grad}V $.







































\section{Dipendenza della polarizzazione dal campo elettrico}

A seguito dello stato di polarizzazione del dielettrico compaiono delle cariche sulla superficie del dielettrico. Introdotta la normale $\vec{n}$ alla superficie, possiamo calcolare la densità di carica sulla superficie e quella volumetrica come:
\[
	\sigma_p = \vec{P} \cdot \vec{n}  \qquad \rho_p=-\text{div}\vec{P}
\]
Dove il vettore polarizzazione è $\vec{P} = n \langle \vec{p} \rangle = n\alpha \vec{E}_l$.
Il problema di rappresentare la polarizzazione in questa forma è che il campo locale spesso non lo conosciamo e non è nemmeno il campo totale, perché va escluso il contributo della molecola stessa.

Si puà determinare una relazione fra $\vec{P}$ e campo elettrico complessivo, che racchiude i contributi di tutte le molecole. Torniamo al semplice caso di un condensatore piano inizialmente vuoto sulle cui armature abbiamo una distribuzione superficiale di carica positiva da una parte, negativa dall'altra. Sappiamo che nello spazio vuoto tra le armature il campo elettrico valga:
\[
	\vec{E}_0 = \frac{\sigma_0}{\varepsilon_0}\vec{n}
\]
Se immaginiamo di posizionare una lastra di isolante in mezzo al condensatore, il campo elettrico provocherà una polarizzazione del dielettrico.
\begin{figure}[htpb]
	\centering
	

	% Pattern Info
	 
	\tikzset{
	pattern size/.store in=\mcSize, 
	pattern size = 5pt,
	pattern thickness/.store in=\mcThickness, 
	pattern thickness = 0.3pt,
	pattern radius/.store in=\mcRadius, 
	pattern radius = 1pt}
	\makeatletter
	\pgfutil@ifundefined{pgf@pattern@name@_qkgxi2fjx}{
	\pgfdeclarepatternformonly[\mcThickness,\mcSize]{_qkgxi2fjx}
	{\pgfqpoint{0pt}{0pt}}
	{\pgfpoint{\mcSize+\mcThickness}{\mcSize+\mcThickness}}
	{\pgfpoint{\mcSize}{\mcSize}}
	{
	\pgfsetcolor{\tikz@pattern@color}
	\pgfsetlinewidth{\mcThickness}
	\pgfpathmoveto{\pgfqpoint{0pt}{0pt}}
	\pgfpathlineto{\pgfpoint{\mcSize+\mcThickness}{\mcSize+\mcThickness}}
	\pgfusepath{stroke}
	}}
	\makeatother

	% Pattern Info
	 
	\tikzset{
	pattern size/.store in=\mcSize, 
	pattern size = 5pt,
	pattern thickness/.store in=\mcThickness, 
	pattern thickness = 0.3pt,
	pattern radius/.store in=\mcRadius, 
	pattern radius = 1pt}
	\makeatletter
	\pgfutil@ifundefined{pgf@pattern@name@_l7mmy55f0}{
	\pgfdeclarepatternformonly[\mcThickness,\mcSize]{_l7mmy55f0}
	{\pgfqpoint{0pt}{0pt}}
	{\pgfpoint{\mcSize+\mcThickness}{\mcSize+\mcThickness}}
	{\pgfpoint{\mcSize}{\mcSize}}
	{
	\pgfsetcolor{\tikz@pattern@color}
	\pgfsetlinewidth{\mcThickness}
	\pgfpathmoveto{\pgfqpoint{0pt}{0pt}}
	\pgfpathlineto{\pgfpoint{\mcSize+\mcThickness}{\mcSize+\mcThickness}}
	\pgfusepath{stroke}
	}}
	\makeatother
	\tikzset{every picture/.style={line width=0.75pt}} %set default line width to 0.75pt        

	\begin{tikzpicture}[x=0.75pt,y=0.75pt,yscale=-1,xscale=1]
	%uncomment if require: \path (0,397); %set diagram left start at 0, and has height of 397

	%Shape: Rectangle [id:dp8333572023914577] 
	\draw  [pattern=_qkgxi2fjx,pattern size=6pt,pattern thickness=0.75pt,pattern radius=0pt, pattern color={rgb, 255:red, 222; green, 222; blue, 222}] (379,51) -- (397.5,51) -- (397.5,273.75) -- (379,273.75) -- cycle ;
	%Shape: Rectangle [id:dp9121196834897161] 
	\draw  [pattern=_l7mmy55f0,pattern size=6pt,pattern thickness=0.75pt,pattern radius=0pt, pattern color={rgb, 255:red, 222; green, 222; blue, 222}] (219,51) -- (237.5,51) -- (237.5,273.75) -- (219,273.75) -- cycle ;
	%Shape: Rectangle [id:dp9389885307301062] 
	\draw  [draw opacity=0][fill={rgb, 255:red, 222; green, 222; blue, 222 }  ,fill opacity=1 ] (249,51) -- (370.5,51) -- (370.5,273.75) -- (249,273.75) -- cycle ;
	%Straight Lines [id:da8522020220086368] 
	\draw    (280.25,112.38) -- (344.25,112.38) ;
	\draw [shift={(277.25,112.38)}, rotate = 0] [fill={rgb, 255:red, 0; green, 0; blue, 0 }  ][line width=0.08]  [draw opacity=0] (10.72,-5.15) -- (0,0) -- (10.72,5.15) -- (7.12,0) -- cycle    ;
	%Straight Lines [id:da6368325319824204] 
	\draw    (277.25,172.38) -- (341.25,172.38) ;
	\draw [shift={(344.25,172.38)}, rotate = 180] [fill={rgb, 255:red, 0; green, 0; blue, 0 }  ][line width=0.08]  [draw opacity=0] (10.72,-5.15) -- (0,0) -- (10.72,5.15) -- (7.12,0) -- cycle    ;
	%Straight Lines [id:da8391058151953443] 
	\draw    (240.5,233.75) -- (379,233.75) ;
	\draw [shift={(237.5,233.75)}, rotate = 0] [fill={rgb, 255:red, 0; green, 0; blue, 0 }  ][line width=0.08]  [draw opacity=0] (10.72,-5.15) -- (0,0) -- (10.72,5.15) -- (7.12,0) -- cycle    ;
	%Curve Lines [id:da9656519554783083] 
	\draw    (241.5,281.5) .. controls (283.56,307.36) and (335.81,310.17) .. (379.51,282.77) ;
	\draw [shift={(381.5,281.5)}, rotate = 506.9] [fill={rgb, 255:red, 0; green, 0; blue, 0 }  ][line width=0.08]  [draw opacity=0] (10.72,-5.15) -- (0,0) -- (10.72,5.15) -- (7.12,0) -- cycle    ;

	% Text Node
	\draw (389.06,61) node    {$+$};
	% Text Node
	\draw (389.06,81) node    {$+$};
	% Text Node
	\draw (389.06,101) node    {$+$};
	% Text Node
	\draw (389.06,121) node    {$+$};
	% Text Node
	\draw (389.06,141) node    {$+$};
	% Text Node
	\draw (389.06,161) node    {$+$};
	% Text Node
	\draw (389.06,181) node    {$+$};
	% Text Node
	\draw (389.06,201) node    {$+$};
	% Text Node
	\draw (389.06,221) node    {$+$};
	% Text Node
	\draw (389.06,241) node    {$+$};
	% Text Node
	\draw (389.06,261) node    {$+$};
	% Text Node
	\draw (228.06,61) node    {$-$};
	% Text Node
	\draw (228.06,81) node    {$-$};
	% Text Node
	\draw (228.06,101) node    {$-$};
	% Text Node
	\draw (228.06,121) node    {$-$};
	% Text Node
	\draw (228.06,141) node    {$-$};
	% Text Node
	\draw (228.06,161) node    {$-$};
	% Text Node
	\draw (228.06,181) node    {$-$};
	% Text Node
	\draw (228.06,201) node    {$-$};
	% Text Node
	\draw (228.06,221) node    {$-$};
	% Text Node
	\draw (228.06,241) node    {$-$};
	% Text Node
	\draw (228.06,261) node    {$-$};
	% Text Node
	\draw (226,36) node    {$-\sigma _{l}$};
	% Text Node
	\draw (392,36) node    {$+\sigma _{l}$};
	% Text Node
	\draw (358.06,61) node    {$-$};
	% Text Node
	\draw (358.06,81) node    {$-$};
	% Text Node
	\draw (358.06,101) node    {$-$};
	% Text Node
	\draw (358.06,121) node    {$-$};
	% Text Node
	\draw (358.06,141) node    {$-$};
	% Text Node
	\draw (358.06,161) node    {$-$};
	% Text Node
	\draw (358.06,181) node    {$-$};
	% Text Node
	\draw (358.06,201) node    {$-$};
	% Text Node
	\draw (358.06,221) node    {$-$};
	% Text Node
	\draw (358.06,241) node    {$-$};
	% Text Node
	\draw (358.06,261) node    {$-$};
	% Text Node
	\draw (259.06,61) node    {$+$};
	% Text Node
	\draw (259.06,81) node    {$+$};
	% Text Node
	\draw (259.06,101) node    {$+$};
	% Text Node
	\draw (259.06,121) node    {$+$};
	% Text Node
	\draw (259.06,141) node    {$+$};
	% Text Node
	\draw (259.06,161) node    {$+$};
	% Text Node
	\draw (259.06,181) node    {$+$};
	% Text Node
	\draw (259.06,201) node    {$+$};
	% Text Node
	\draw (259.06,221) node    {$+$};
	% Text Node
	\draw (259.06,241) node    {$+$};
	% Text Node
	\draw (259.06,261) node    {$+$};
	% Text Node
	\draw (262,36) node    {$+\sigma _{p}$};
	% Text Node
	\draw (356,36) node    {$-\sigma _{p}$};
	% Text Node
	\draw (313.67,96) node    {$\vec{P}$};
	% Text Node
	\draw (313.67,156) node    {$\vec{E} '$};
	% Text Node
	\draw (313.67,216) node    {$\vec{E}_{0}$};
	% Text Node
	\draw (313.06,315) node    {$\Delta V_{d}$};


	\end{tikzpicture}
\end{figure}
\FloatBarrier
Possiamo anche in questo caso determinare la densità di carica di polarizzazione. Esse parzialmente cercano di compensare le cariche libere. Il campo elettrico all'interno del dielettrico sarà dato da:
\begin{align*}
		\left. \begin{array}{r}
	 	E_d = \frac{V_d}{h} = \frac{V_0}{\varepsilon_r h} = \frac{E_0}{\varepsilon_r} \\
	 	\sigma_p = \vec{P} \cdot \vec{n} \\
	 	\vec{E}_d = \vec{E}_0 + \vec{E}'= \vec{E}_0 - \frac{\sigma_p}{\varepsilon_0}
	\end{array} \right\} \implies \vec{E}_d &= \varepsilon_r \vec{E}_d - \frac{\vec{P}}{\varepsilon_0} \\
	\frac{\vec{P}}{\varepsilon_0} &= \vec{E}_d(\varepsilon_r -1) \\
	\Aboxed{\vec{P} &= \varepsilon_0 (\varepsilon_r - 1) \vec{E}_d} \\
	\vec{P} &= \varepsilon_0 \chi \vec{E}_d
\end{align*}
Se consideriamo il vuoto come dielettrico particolare, da queste formule deriva che $\vec{P}$ nel vuoto è uguale a zero, non ci può essere infatti polarizzazione nel vuoto.







































\section{Equazioni generali dell'elettrostatica in presenza di dielettrici}

Definito il legame tra campo elettrico e polarizzazione, vediamo come esprimere le equazioni di Maxwell in presenza del dielettrico. Sappiamo che il campo elettrico prodotto da cariche ferme è conservativo anche in presenza di dielettrici polarizzati. Continuano pertanto a valere le due formulazioni, integrale e locale:
\[
	\int_{\Sigma} \vec{E} \cdot \vec{n} dS = 0 \qquad \vec{\nabla} \times \vec{E} =0
\]
e la proprietà equivalente che il campo elettrico si possa ottenere come gradiente della funzione potenziale. Anche la legge di Gauss resta valida in presenza di dielettrici polarizzati, perché si tenga conto delle cariche di polarizzazione oltre che delle cariche libere.
\[
	\int_{\Sigma} \vec{E} \cdot \vec{n} dS = \frac{q+q_p}{\varepsilon_0}
\]
Il flusso del campo elettrico attraverso una superficie chiusa è uguale alla somma delle cariche libere e delle cariche di polarizzazione contenute all'interno della superficie. In forma differenziale:
\[
	\text{div}\vec{E} = \frac{\rho_{\text{tot}}}{\varepsilon_0}=\frac{\rho_l+\rho_p}{\varepsilon_0}
\]
Non possiamo stabilire a priori quanti valga la densità di carica volumetrica di polarizzazione nelle equazioni di Maxwell. Ricordando che essa può essere espressa come l'opposto della divergenza del vettore polarizzazione:
\begin{align*}
	\vec{\nabla} \cdot \vec{E} &= \frac{\rho_l+\rho_p}{\varepsilon_0} \\
	\vec{\nabla} \cdot \vec{E} &= \frac{\rho_l - \vec{\nabla} \cdot \vec{P}}{\varepsilon_0} \\
	\varepsilon_0 \vec{\nabla} \cdot \vec{E} &= \rho_l - \vec{\nabla} \cdot \vec{P} \\
	\vec{\nabla} \cdot (\varepsilon_0 \vec{E} + \vec{P} ) &= \rho_l
\end{align*}
Abbiamo ottenuto così una nuova equazione che dipende solo dalle cariche libere, di cui conosciamo la disposizione. Possiamo a questo punto introdurre una nuova grandezza vettoriale che chiameremo \textbf{campo induzione dielettrica}:
\[
	\boxed{\vec{D} = \varepsilon_0 \vec{E} + \vec{P}}
\]
Le leggi precedenti allora si scrivono:
\[
	\vec{\nabla} \cdot \vec{D} = \rho_l \qquad \int_{\Sigma}\vec{D} \cdot \vec{n} dS = Q_{\text{libere}}
\]
Il flusso del vettore $\vec{D}$ attraverso una superficie chiusa, contenente in generale sia cariche libere che cariche di polarizzazione, \emph{dipende soltanto dalle cariche libere}. La proprietà non è banale perché la superficie chiusa $\Sigma$ può intersecare un dielettrico, invece che contenerlo interamente, per cui la carica di polarizzazione all'interno di $\Sigma$ non è nulla. Per chiarire l'argomento, possiamo applicare il teorema della divergenza sulla superficie in figura.
\begin{figure}[htpb]
	\centering
	

	\tikzset{every picture/.style={line width=0.75pt}} %set default line width to 0.75pt        

	\begin{tikzpicture}[x=0.75pt,y=0.75pt,yscale=-1,xscale=1]
	%uncomment if require: \path (0,300); %set diagram left start at 0, and has height of 300

	%Shape: Circle [id:dp833508616976691] 
	\draw  [fill={rgb, 255:red, 0; green, 0; blue, 0 }  ,fill opacity=1 ] (348.25,129) .. controls (348.25,127.48) and (349.48,126.25) .. (351,126.25) .. controls (352.52,126.25) and (353.75,127.48) .. (353.75,129) .. controls (353.75,130.52) and (352.52,131.75) .. (351,131.75) .. controls (349.48,131.75) and (348.25,130.52) .. (348.25,129) -- cycle ;
	%Shape: Circle [id:dp823362303447388] 
	\draw  [fill={rgb, 255:red, 0; green, 0; blue, 0 }  ,fill opacity=1 ] (298.25,149) .. controls (298.25,147.48) and (299.48,146.25) .. (301,146.25) .. controls (302.52,146.25) and (303.75,147.48) .. (303.75,149) .. controls (303.75,150.52) and (302.52,151.75) .. (301,151.75) .. controls (299.48,151.75) and (298.25,150.52) .. (298.25,149) -- cycle ;
	%Shape: Circle [id:dp03974441898811176] 
	\draw  [fill={rgb, 255:red, 0; green, 0; blue, 0 }  ,fill opacity=1 ] (338.25,189) .. controls (338.25,187.48) and (339.48,186.25) .. (341,186.25) .. controls (342.52,186.25) and (343.75,187.48) .. (343.75,189) .. controls (343.75,190.52) and (342.52,191.75) .. (341,191.75) .. controls (339.48,191.75) and (338.25,190.52) .. (338.25,189) -- cycle ;
	%Straight Lines [id:da7059820690490635] 
	\draw    (209.5,159) -- (247.5,159) ;
	\draw [shift={(206.5,159)}, rotate = 0] [fill={rgb, 255:red, 0; green, 0; blue, 0 }  ][line width=0.08]  [draw opacity=0] (10.72,-5.15) -- (0,0) -- (10.72,5.15) -- (7.12,0) -- cycle    ;
	%Shape: Circle [id:dp45490059660765114] 
	\draw  [draw opacity=0][fill={rgb, 255:red, 222; green, 222; blue, 222 }  ,fill opacity=1 ] (365.5,109.5) .. controls (365.5,83.54) and (386.54,62.5) .. (412.5,62.5) .. controls (438.46,62.5) and (459.5,83.54) .. (459.5,109.5) .. controls (459.5,135.46) and (438.46,156.5) .. (412.5,156.5) .. controls (386.54,156.5) and (365.5,135.46) .. (365.5,109.5) -- cycle ;
	%Shape: Circle [id:dp7498608518321022] 
	\draw   (247.5,159) .. controls (247.5,111.23) and (286.23,72.5) .. (334,72.5) .. controls (381.77,72.5) and (420.5,111.23) .. (420.5,159) .. controls (420.5,206.77) and (381.77,245.5) .. (334,245.5) .. controls (286.23,245.5) and (247.5,206.77) .. (247.5,159) -- cycle ;

	% Text Node
	\draw (302.06,167) node    {$Q_{1}$};
	% Text Node
	\draw (353.06,147) node    {$Q_{2}$};
	% Text Node
	\draw (347.06,205) node    {$Q_{n}$};
	% Text Node
	\draw (196.06,156) node    {$\vec{n}$};
	% Text Node
	\draw (289.06,95) node    {$\tau $};
	% Text Node
	\draw (320.06,89) node    {$\Sigma _{1}$};
	% Text Node
	\draw (390.06,119) node    {$\Sigma _{2}$};
	% Text Node
	\draw (419.06,214) node    {$\Sigma _{g}$};
	% Text Node
	\draw (417,85) node  [font=\footnotesize] [align=left] {dielettrico};


	\end{tikzpicture}
\end{figure}
\FloatBarrier
La superficie Gaussiana contiene al suo interno delle cariche libere e una parte del dielettrico polarizzato.
\begin{align*}
	\int_{\tau} \text{div}\vec{D} d\tau &= \int_{\Sigma}\vec{D} \cdot \vec{n} dS = \Phi_{\Sigma}(\vec{D}) \\
	\int_{\tau} \text{div}\vec{D} d\tau &= \int_{\Sigma} \rho_ld\tau = Q_{\text{libere}}^{\text{int. a}\Sigma}
\end{align*}
Otteniamo così il teorema di Gauss per D:
\[
	\boxed{\Phi_{\Sigma}(\vec{D}) = Q_{\text{libere}}^{\text{int. a}\Sigma}}
\]
Ribadiamo che l'utilità di questa relazione sta nel fatto che di norma sono le cariche libere ad essere conosciute e non le cariche di polarizzazione.
\[
	\left. \begin{array}{r}
	 	\vec{D} =\varepsilon_0 \vec{E} + \vec{P} \\
		\vec{P} = \varepsilon_0 \chi \vec{E}
	\end{array} \right\} \implies \vec{D} = \varepsilon_0 \vec{E} +\varepsilon_0 \chi \vec{E} = \varepsilon_0 (1+\chi)\vec{E} = \varepsilon_0 \varepsilon_r \vec{E}
\]
Abbiamo trovato quindi che:
\[
	\boxed{\vec{P} = \varepsilon_0 \chi \vec{E} \qquad \vec{D} = \varepsilon_0 \varepsilon_r \vec{E} = \varepsilon \vec{E}}
\]
Un dielettrico per cui valgono queste relazioni si dice \emph{dielettrico lineare}. In un dielettrico lineare l'induzione dielettrica, il campo elettrico e la polarizzazione sono vettori tra loro paralleli. Sappiamo che $ \text{rot}\vec{E} =0 $ perché il campo elettrostatico è conservativo. Proviamo a calcolare il rotore di $ \vec{D}  $ per vedere se anche esso è conservativo:
\[
	\text{rot}\vec{D} =\text{rot}(\varepsilon_0 \vec{E} +\vec{P}) = \varepsilon_0 \underbrace{\text{rot}\vec{E}}_{=0} + \text{rot}\vec{P} = \text{rot}\vec{P}
\]
Consideriamo una percorso chiuso $ \gamma  $ fatto come in figura. Il lato coincide con le linee di flusso di $ \vec{D}  $. Per testare se un campo vettoriale è conservativo il criterio è quello di calcolare la circuitazione del campo lungo un percorso chiuso.
\begin{figure}[htpb]
	\centering
	

	\tikzset{every picture/.style={line width=0.75pt}} %set default line width to 0.75pt        

	\begin{tikzpicture}[x=0.75pt,y=0.75pt,yscale=-1,xscale=1]
	%uncomment if require: \path (0,300); %set diagram left start at 0, and has height of 300

	%Shape: Rectangle [id:dp2533190145026505] 
	\draw  [draw opacity=0][fill={rgb, 255:red, 222; green, 222; blue, 222 }  ,fill opacity=1 ] (249,30) -- (370.5,30) -- (370.5,252.75) -- (249,252.75) -- cycle ;
	%Straight Lines [id:da6449464912660832] 
	\draw    (342.25,224) -- (342.25,56) ;
	\draw [shift={(342.25,227)}, rotate = 270] [fill={rgb, 255:red, 0; green, 0; blue, 0 }  ][line width=0.08]  [draw opacity=0] (10.72,-5.15) -- (0,0) -- (10.72,5.15) -- (7.12,0) -- cycle    ;
	%Straight Lines [id:da3245611589473396] 
	\draw    (312.25,224) -- (312.25,56) ;
	\draw [shift={(312.25,227)}, rotate = 270] [fill={rgb, 255:red, 0; green, 0; blue, 0 }  ][line width=0.08]  [draw opacity=0] (10.72,-5.15) -- (0,0) -- (10.72,5.15) -- (7.12,0) -- cycle    ;
	%Straight Lines [id:da6676595391607743] 
	\draw    (282.25,224) -- (282.25,56) ;
	\draw [shift={(282.25,227)}, rotate = 270] [fill={rgb, 255:red, 0; green, 0; blue, 0 }  ][line width=0.08]  [draw opacity=0] (10.72,-5.15) -- (0,0) -- (10.72,5.15) -- (7.12,0) -- cycle    ;
	%Shape: Rectangle [id:dp8869556569700139] 
	\draw   (355,102) -- (429.5,102) -- (429.5,182) -- (355,182) -- cycle ;
	\draw   (434.33,136) -- (429.67,145.33) -- (425,136) ;

	% Text Node
	\draw (263.67,101) node    {$\vec{P}$};
	% Text Node
	\draw (444,181.33) node    {$\gamma $};


	\end{tikzpicture}
\end{figure}
\FloatBarrier
Ci accorgiamo subito che lungo i tratti all'esterno del percorso $ \vec{P}  $ è zero. I lati perpendicolari alle linee di $\vec{P}$ non contribuiscono. L'unico tratto che contribuisce è quello del rettangolo parallelo a $\vec{P}$. L'integrale
\[
	\oint_{\gamma} \vec{P} \cdot d\vec{l} \neq 0 \implies \text{rot}\vec{P} \neq 0
\]
sarà diverso da zero sicuramente.
Di conseguenza anche $ \text{rot}\vec{D} $ non sarà zero e il campo non è conservativo. Se immaginiamo un caso astratto in cui il dielettrico permea tutto lo spazio in modo omogeneo, la situazione cambia perché, se assumiamo che $ \varepsilon_r  $ \emph{non dipende dalla posizione}, abbiamo che:
\[
	\vec{D} =\varepsilon_0 \varepsilon_r \vec{E} \qquad \text{rot}\vec{D} = \text{rot}(\varepsilon_0 \varepsilon_r \vec{E} ) = \varepsilon_0 \varepsilon_r \text{rot}\vec{E} = 0
\]
Si può concludere che allora in questa particolare situazione $\vec{D}$ e $\vec{P}$ sono conservativi. In casi generali, $\varepsilon_r$ \emph{dipende dalla posizione} e non può essere portato fuori dal rotore.

Inoltre
\begin{align*}
	\left. \begin{array}{r}
	 	\vec{D} =\varepsilon_0 \varepsilon_r \vec{E} \\
		\vec{\nabla} \cdot \vec{D} = \rho_l
	\end{array} \right\} \vec{\nabla} \cdot (\varepsilon_0 \varepsilon_r \vec{E} ) = \rho_l  &= \varepsilon_0 \varepsilon_r \vec{\nabla} \cdot \vec{E} \\
	\Aboxed{\vec{\nabla} \cdot \vec{E} &= \frac{\rho_l}{\varepsilon_0 \varepsilon_r} = \frac{\rho_l}{\varepsilon}}
\end{align*}
In presenza di un dielettrico omogeneo possiamo anche non preoccuparci di come sono fatte le cariche di polarizzazione. Basta conoscere le cariche libere e correggere l'equazione, per considerare anche l'influenza di quelle di polarizzazione.
Se immaginiamo la stessa distribuzione nel vuoto in cui abbiamo le cariche $ Q_1  $, $ Q_2  $, $ Q_n  $, allora in questo caso abbiamo che il flusso attraverso $\Sigma$ del campo elettrico è:
\[
	\Phi_{\Sigma}(\vec{E} ) = \int_{\tau}\text{div}\vec{E} d\tau = \int_{\tau}\frac{\rho_l}{\varepsilon_0 \varepsilon_r} d\tau = \int \frac{dq}{\varepsilon_0 \varepsilon_r} = \frac{Q_l}{\varepsilon_0 \varepsilon_r} = \frac{Q_1 + Q_2 + \ldots + Q_n}{\varepsilon_0 \varepsilon_r}
\]
Questo suggerisce una strategia per risolvere i problemi di elettrostatica in presenza di dielettrici.
Teniamo presente il caso di una sfera di raggio $R$. Nel vuoto sappiamo come calcolare il campo elettrico:
\[
	\vec{E} (r) = \left\{ \begin{array}{ll}
	 	0 & r<R \\
		\frac{Q}{4\pi \varepsilon_0 r^2} & r>R
	\end{array} \right.
\]
Se immaginiamo di avere invece del vuoto un dielettrico:
\[
	\vec{E} (r) = \left\{ \begin{array}{ll}
	 	0 & r<R \\
		\frac{Q}{4\pi \varepsilon_0 \varepsilon_r r^2} & r>R
	\end{array} \right.
\]







































\section{Discontinuità dei campi sulla superficie di separazione tra due dielettrici}

Occupiamoci ora delle condizioni al contorno in prossimità della superficie di separazione fra due dielettrici differenti. Abbiamo già studiato la discontinuità del campo elettrico nell'attraversare una superficie su cui è depositata una carica. Una situazione di questo tipo si ritrova sempre quando si attraversa la superficie limite di un dielettrico polarizzato e quindi anche sulla superficie $\Sigma$ di separazione tra due dielettrici diversi, di costanti dielettriche relative $\varepsilon_{r1}$ e $\varepsilon_{r2}$. Poniamo che sulla superficie di separazione siano presenti anche delle cariche libere di densità superficiale $ \sigma_l  $. Se consideriamo un percorso rettangolare come in figura, con il lato più lungo parallelo alla superficie e il lato più corto di lunghezza trascurabile, se calcoliamo la circuitazione su $\gamma$ del campo elettrico, essa deve essere nulla per la conservatività del campo elettrico.
\begin{figure}[htpb]
	\centering
	

	\tikzset{every picture/.style={line width=0.75pt}} %set default line width to 0.75pt        

	\begin{tikzpicture}[x=0.75pt,y=0.75pt,yscale=-1,xscale=1]
	%uncomment if require: \path (0,300); %set diagram left start at 0, and has height of 300

	%Shape: Rectangle [id:dp9027983971629476] 
	\draw  [draw opacity=0][fill={rgb, 255:red, 196; green, 196; blue, 196 }  ,fill opacity=1 ] (154,161) -- (483.33,161) -- (483.33,278) -- (154,278) -- cycle ;
	%Shape: Rectangle [id:dp9530947479101171] 
	\draw  [draw opacity=0][fill={rgb, 255:red, 222; green, 222; blue, 222 }  ,fill opacity=1 ] (154,44) -- (483.33,44) -- (483.33,161) -- (154,161) -- cycle ;
	%Straight Lines [id:da14386266924254976] 
	\draw    (154,161) -- (483.5,161) ;
	%Straight Lines [id:da333858239102355] 
	\draw    (316.99,161.11) -- (350.19,80.61) ;
	\draw [shift={(351.33,77.83)}, rotate = 472.41] [fill={rgb, 255:red, 0; green, 0; blue, 0 }  ][line width=0.08]  [draw opacity=0] (10.72,-5.15) -- (0,0) -- (10.72,5.15) -- (7.12,0) -- cycle    ;
	%Straight Lines [id:da5494115338762517] 
	\draw    (253.33,229.17) -- (314.94,163.3) ;
	\draw [shift={(316.99,161.11)}, rotate = 493.09] [fill={rgb, 255:red, 0; green, 0; blue, 0 }  ][line width=0.08]  [draw opacity=0] (10.72,-5.15) -- (0,0) -- (10.72,5.15) -- (7.12,0) -- cycle    ;
	%Straight Lines [id:da6228664250334108] 
	\draw    (260.33,145.33) -- (440.33,145.33) ;
	\draw [shift={(350.33,145.33)}, rotate = 180] [fill={rgb, 255:red, 0; green, 0; blue, 0 }  ][line width=0.08]  [draw opacity=0] (10.72,-5.15) -- (0,0) -- (10.72,5.15) -- (7.12,0) -- cycle    ;
	%Straight Lines [id:da4897252836030672] 
	\draw    (260.33,174) -- (440.33,174) ;
	\draw [shift={(350.33,174)}, rotate = 0] [fill={rgb, 255:red, 0; green, 0; blue, 0 }  ][line width=0.08]  [draw opacity=0] (10.72,-5.15) -- (0,0) -- (10.72,5.15) -- (7.12,0) -- cycle    ;
	%Straight Lines [id:da14323434536708257] 
	\draw    (440.33,145.33) -- (440.33,174) ;
	%Straight Lines [id:da7181018507690913] 
	\draw    (260.33,145.33) -- (260.33,174) ;
	%Straight Lines [id:da32490904028770395] 
	\draw    (402.66,119.77) -- (442.67,119.77) ;
	\draw [shift={(445.67,119.77)}, rotate = 180] [fill={rgb, 255:red, 0; green, 0; blue, 0 }  ][line width=0.08]  [draw opacity=0] (10.72,-5.15) -- (0,0) -- (10.72,5.15) -- (7.12,0) -- cycle    ;
	%Straight Lines [id:da8352740902431424] 
	\draw  [dash pattern={on 0.84pt off 2.51pt}]  (316.99,229.17) -- (316.99,78.5) ;
	%Straight Lines [id:da2977705925120635] 
	\draw    (316.99,77.83) -- (348.33,77.83) ;
	\draw [shift={(351.33,77.83)}, rotate = 180] [fill={rgb, 255:red, 0; green, 0; blue, 0 }  ][line width=0.08]  [draw opacity=0] (10.72,-5.15) -- (0,0) -- (10.72,5.15) -- (7.12,0) -- cycle    ;
	%Straight Lines [id:da9098455531286174] 
	\draw    (253.33,229.17) -- (313.99,229.17) ;
	\draw [shift={(316.99,229.17)}, rotate = 180] [fill={rgb, 255:red, 0; green, 0; blue, 0 }  ][line width=0.08]  [draw opacity=0] (10.72,-5.15) -- (0,0) -- (10.72,5.15) -- (7.12,0) -- cycle    ;
	%Shape: Arc [id:dp9514879995065608] 
	\draw  [draw opacity=0] (317.22,191.11) .. controls (317.15,191.11) and (317.07,191.11) .. (316.99,191.11) .. controls (308.94,191.11) and (301.63,187.94) .. (296.24,182.78) -- (316.99,161.11) -- cycle ; \draw   (317.22,191.11) .. controls (317.15,191.11) and (317.07,191.11) .. (316.99,191.11) .. controls (308.94,191.11) and (301.63,187.94) .. (296.24,182.78) ;
	%Shape: Arc [id:dp550715929473472] 
	\draw  [draw opacity=0] (317,131.11) .. controls (321.05,131.11) and (324.91,131.91) .. (328.44,133.37) -- (316.99,161.11) -- cycle ; \draw   (317,131.11) .. controls (321.05,131.11) and (324.91,131.91) .. (328.44,133.37) ;

	% Text Node
	\draw (170,152) node    {$+$};
	% Text Node
	\draw (134,131) node    {$1$};
	% Text Node
	\draw (134,192) node    {$2$};
	% Text Node
	\draw (190,152) node    {$+$};
	% Text Node
	\draw (210,152) node    {$+$};
	% Text Node
	\draw (230,152) node    {$+$};
	% Text Node
	\draw (250,152) node    {$+$};
	% Text Node
	\draw (270,152) node    {$+$};
	% Text Node
	\draw (290,152) node    {$+$};
	% Text Node
	\draw (310,152) node    {$+$};
	% Text Node
	\draw (330,152) node    {$+$};
	% Text Node
	\draw (350,152) node    {$+$};
	% Text Node
	\draw (370,152) node    {$+$};
	% Text Node
	\draw (390,152) node    {$+$};
	% Text Node
	\draw (410,152) node    {$+$};
	% Text Node
	\draw (430,152) node    {$+$};
	% Text Node
	\draw (450,152) node    {$+$};
	% Text Node
	\draw (470,152) node    {$+$};
	% Text Node
	\draw (353,109.33) node    {$\vec{E}_{1}$};
	% Text Node
	\draw (254.33,202) node    {$\vec{E}_{2}$};
	% Text Node
	\draw (498.67,161) node    {$\Sigma $};
	% Text Node
	\draw (424,105.67) node    {$\vec{u}_{t}$};
	% Text Node
	\draw (264,130) node    {$\gamma $};
	% Text Node
	\draw (177.5,131) node    {$\sigma _{1}$};
	% Text Node
	\draw (177.5,179.33) node    {$\sigma _{2}$};
	% Text Node
	\draw (217.5,131) node    {$\varepsilon _{r1}$};
	% Text Node
	\draw (217.5,179.33) node    {$\varepsilon _{r2}$};
	% Text Node
	\draw (329.67,60.67) node    {$\vec{E}_{t1}$};
	% Text Node
	\draw (288.67,245) node    {$\vec{E}_{t2}$};
	% Text Node
	\draw (327.33,106) node    {$\vartheta _{1}$};
	% Text Node
	\draw (304,197.33) node    {$\vartheta _{2}$};


	\end{tikzpicture}
\end{figure}
\FloatBarrier
Se introduciamo il versore tangente alla superficie e chiamiamo con $\vec{E}_1$ ed $\vec{E}_2$ i campi elettrici presenti in prossimità della superficie dal lato $1$ e $2$, questa condizione porta a stabilire che:
\[
	\oint_{\gamma} \vec{E} \cdot d\vec{l} = \vec{E}_1\cdot \vec{u}_t - \vec{E}_2\cdot \vec{u}_t = 0 \implies  E_{t1}=E_{t2}
\]
Se introduciamo la normale $ \vec{n}  $ alla superficie, introdotte le componenti normali del campo, esse subiscono una discontinuità e se non conosciamo le cariche di polarizzazione non sappiamo quantificarla.
Poniamo:
\[
	E_{n1} = \vec{E}_1\cdot \vec{n} \qquad E_{n2}=\vec{E} \cdot \vec{n}
\]
Applichiamo la legge di Gauss al vettore $ \vec{D}  $, scegliendo come superficie di integrazione una scatola cilindrica di altezza infinitesima $h$ ortogonale a $\Sigma$ e con le basi $dS$ all'interno dei due dielettrici.
\begin{figure}[htpb]
	\centering
	

	\tikzset{every picture/.style={line width=0.75pt}} %set default line width to 0.75pt        

	\begin{tikzpicture}[x=0.75pt,y=0.75pt,yscale=-1,xscale=1]
	%uncomment if require: \path (0,343); %set diagram left start at 0, and has height of 343

	%Shape: Rectangle [id:dp7294611060731351] 
	\draw  [draw opacity=0][fill={rgb, 255:red, 222; green, 222; blue, 222 }  ,fill opacity=1 ] (151,43) -- (480.33,43) -- (480.33,160) -- (151,160) -- cycle ;
	%Shape: Rectangle [id:dp7551459350179142] 
	\draw  [draw opacity=0][fill={rgb, 255:red, 196; green, 196; blue, 196 }  ,fill opacity=1 ] (151,160) -- (480.33,160) -- (480.33,277) -- (151,277) -- cycle ;
	%Straight Lines [id:da34936287587630344] 
	\draw    (457.33,168.7) -- (457.33,139.5) ;
	\draw [shift={(457.33,139.5)}, rotate = 450] [color={rgb, 255:red, 0; green, 0; blue, 0 }  ][line width=0.75]    (0,5.59) -- (0,-5.59)   ;
	\draw [shift={(457.33,168.7)}, rotate = 450] [color={rgb, 255:red, 0; green, 0; blue, 0 }  ][line width=0.75]    (0,5.59) -- (0,-5.59)   ;
	%Straight Lines [id:da23320189135833136] 
	\draw    (151,160) -- (480.5,160) ;
	%Straight Lines [id:da8288250365936289] 
	\draw    (313.99,160.11) -- (347.19,79.61) ;
	\draw [shift={(348.33,76.83)}, rotate = 472.41] [fill={rgb, 255:red, 0; green, 0; blue, 0 }  ][line width=0.08]  [draw opacity=0] (10.72,-5.15) -- (0,0) -- (10.72,5.15) -- (7.12,0) -- cycle    ;
	%Straight Lines [id:da8647977351173188] 
	\draw    (250.33,228.17) -- (311.94,162.3) ;
	\draw [shift={(313.99,160.11)}, rotate = 493.09] [fill={rgb, 255:red, 0; green, 0; blue, 0 }  ][line width=0.08]  [draw opacity=0] (10.72,-5.15) -- (0,0) -- (10.72,5.15) -- (7.12,0) -- cycle    ;
	%Straight Lines [id:da41476616214789863] 
	\draw    (424.66,113.87) -- (464.67,113.87) ;
	\draw [shift={(467.67,113.87)}, rotate = 180] [fill={rgb, 255:red, 0; green, 0; blue, 0 }  ][line width=0.08]  [draw opacity=0] (10.72,-5.15) -- (0,0) -- (10.72,5.15) -- (7.12,0) -- cycle    ;
	%Straight Lines [id:da03518666561559369] 
	\draw    (313.99,160.11) -- (313.99,80.5) ;
	\draw [shift={(313.99,77.5)}, rotate = 450] [fill={rgb, 255:red, 0; green, 0; blue, 0 }  ][line width=0.08]  [draw opacity=0] (10.72,-5.15) -- (0,0) -- (10.72,5.15) -- (7.12,0) -- cycle    ;
	%Straight Lines [id:da8461812416508665] 
	\draw  [dash pattern={on 0.84pt off 2.51pt}]  (313.99,76.83) -- (348.33,76.83) ;
	%Straight Lines [id:da6883334376186454] 
	\draw  [dash pattern={on 0.84pt off 2.51pt}]  (250.33,228.17) -- (313.99,228.17) ;
	%Shape: Arc [id:dp5767902112592809] 
	\draw  [draw opacity=0] (314.25,193.22) .. controls (314.16,193.22) and (314.08,193.22) .. (313.99,193.22) .. controls (305.11,193.22) and (297.04,189.72) .. (291.09,184.02) -- (313.99,160.11) -- cycle ; \draw   (314.25,193.22) .. controls (314.16,193.22) and (314.08,193.22) .. (313.99,193.22) .. controls (305.11,193.22) and (297.04,189.72) .. (291.09,184.02) ;
	%Shape: Arc [id:dp3547713129305532] 
	\draw  [draw opacity=0] (314,127) .. controls (318.47,127) and (322.73,127.89) .. (326.62,129.49) -- (313.99,160.11) -- cycle ; \draw   (314,127) .. controls (318.47,127) and (322.73,127.89) .. (326.62,129.49) ;
	%Shape: Can [id:dp402632050012663] 
	\draw   (400,140.27) -- (400,169.2) .. controls (400,172.62) and (363.36,175.4) .. (318.17,175.4) .. controls (272.97,175.4) and (236.33,172.62) .. (236.33,169.2) -- (236.33,140.27) .. controls (236.33,136.84) and (272.97,134.07) .. (318.17,134.07) .. controls (363.36,134.07) and (400,136.84) .. (400,140.27) .. controls (400,143.69) and (363.36,146.47) .. (318.17,146.47) .. controls (272.97,146.47) and (236.33,143.69) .. (236.33,140.27) ;
	%Straight Lines [id:da6048807373583354] 
	\draw    (371.33,139.77) -- (371.33,94.33) ;
	\draw [shift={(371.33,91.33)}, rotate = 450] [fill={rgb, 255:red, 0; green, 0; blue, 0 }  ][line width=0.08]  [draw opacity=0] (10.72,-5.15) -- (0,0) -- (10.72,5.15) -- (7.12,0) -- cycle    ;
	%Straight Lines [id:da3849904308148171] 
	\draw    (371.33,220.77) -- (371.33,173.53) ;
	\draw [shift={(371.33,223.77)}, rotate = 270] [fill={rgb, 255:red, 0; green, 0; blue, 0 }  ][line width=0.08]  [draw opacity=0] (10.72,-5.15) -- (0,0) -- (10.72,5.15) -- (7.12,0) -- cycle    ;
	%Straight Lines [id:da8027470362565412] 
	\draw  [dash pattern={on 0.84pt off 2.51pt}]  (371.33,173.53) -- (371.33,164.53) ;
	%Straight Lines [id:da7347351988099999] 
	\draw    (313.99,228.17) -- (313.99,163.11) ;
	\draw [shift={(313.99,160.11)}, rotate = 450] [fill={rgb, 255:red, 0; green, 0; blue, 0 }  ][line width=0.08]  [draw opacity=0] (10.72,-5.15) -- (0,0) -- (10.72,5.15) -- (7.12,0) -- cycle    ;

	% Text Node
	\draw (458.33,179.07) node    {$dh$};
	% Text Node
	\draw (412.67,175.67) node    {$\Sigma _{g}$};
	% Text Node
	\draw (167,151) node    {$+$};
	% Text Node
	\draw (131,130) node    {$1$};
	% Text Node
	\draw (131,191) node    {$2$};
	% Text Node
	\draw (187,151) node    {$+$};
	% Text Node
	\draw (207,151) node    {$+$};
	% Text Node
	\draw (227,151) node    {$+$};
	% Text Node
	\draw (247,151) node    {$+$};
	% Text Node
	\draw (267,151) node    {$+$};
	% Text Node
	\draw (287,151) node    {$+$};
	% Text Node
	\draw (307,151) node    {$+$};
	% Text Node
	\draw (327,151) node    {$+$};
	% Text Node
	\draw (347,151) node    {$+$};
	% Text Node
	\draw (367,151) node    {$+$};
	% Text Node
	\draw (387,151) node    {$+$};
	% Text Node
	\draw (407,151) node    {$+$};
	% Text Node
	\draw (427,151) node    {$+$};
	% Text Node
	\draw (447,151) node    {$+$};
	% Text Node
	\draw (467,151) node    {$+$};
	% Text Node
	\draw (350,108.33) node    {$\vec{E}_{1}$};
	% Text Node
	\draw (251.33,201) node    {$\vec{E}_{2}$};
	% Text Node
	\draw (495.67,160) node    {$\Sigma $};
	% Text Node
	\draw (446,99.77) node    {$\vec{u}_{t}$};
	% Text Node
	\draw (174.5,130) node    {$\sigma _{1}$};
	% Text Node
	\draw (174.5,178.33) node    {$\sigma _{2}$};
	% Text Node
	\draw (214.5,130) node    {$\varepsilon _{r1}$};
	% Text Node
	\draw (214.5,178.33) node    {$\varepsilon _{r2}$};
	% Text Node
	\draw (293.87,81.67) node    {$\vec{E}_{n1}$};
	% Text Node
	\draw (331.67,221.6) node    {$\vec{E}_{n2}$};
	% Text Node
	\draw (324.33,105) node    {$\vartheta _{1}$};
	% Text Node
	\draw (299.4,202.33) node    {$\vartheta _{2}$};
	% Text Node
	\draw (386.93,91.87) node    {$\vec{n}_{1}$};
	% Text Node
	\draw (389.33,218.33) node    {$\vec{n}_{2}$};
	% Text Node
	\draw (257,120.33) node    {$dS$};


	\end{tikzpicture}
\end{figure}
\FloatBarrier
\begin{align*}
	\Phi_{\Sigma_g}(\vec{D}) &= Q_l^{\Sigma} = \sigma_l dS \\
	\Phi_{\Sigma_g}(\vec{D}) &= \vec{D}_1\cdot \vec{n} dS - \vec{D}_2\cdot \vec{n} dS \tag*{se $ h \ll \sqrt{dS} $} \\
	\Downarrow \\
	(\vec{D}_1\cdot \vec{n} - \vec{D}_2\cdot \vec{n})dS &= \sigma_l dS \\
	\Aboxed{D_{n1}-D_{n2} &= \sigma_l}\\
	D_{n1} &= (\varepsilon_0 \vec{E}_1 + \vec{P}_1) \cdot \vec{u}_n = \varepsilon_0 E_{n1} + \sigma_1 \\
	D_{n2} &= (\varepsilon_0 \vec{E}_2 + \vec{P}_2) \cdot \vec{u}_n = \varepsilon_0 E_{n2} + \sigma_2\\
	\underbrace{\varepsilon_0 E_{n1} + \sigma_1 - \varepsilon_0 E_{n2} - \sigma_2}_{D_{n1}-D_{n2}} &= \sigma_l \\
	\Aboxed{E_{n1} - E_{n2} &= \frac{\sigma_l +\sigma_2 -\sigma_1}{\varepsilon_0}}
\end{align*}
Se fossimo nella stessa situazione, ma in assenza di cariche libere:
\begin{align*}
	\Phi_{\Sigma}(\vec{D} ) = 0 \\
	\Downarrow \\
	D_{n1} &= D_{n2} \\
	\varepsilon_0 \varepsilon_{r1}E_{n1} &= \varepsilon_0 \varepsilon_{r2}E_{n2} \\
	\varepsilon_{r1}E_{n1} &= \varepsilon_{r2}E_{n2} \\
	\varepsilon_{r1}E_1\cos \vartheta_1  &= \varepsilon_{r2}E_2\cos \vartheta_2 \\
	\frac{E_1}{E_2} &= \frac{\varepsilon_{r2}}{\varepsilon_{r1}} \frac{\cos \vartheta_2}{\cos \vartheta_1} \\
	E_{t1} = E_{t2} & \implies E_1 \sin \vartheta_1 = E_2 \sin \vartheta_2 \\
	&\implies \frac{E_1}{E_2} = \frac{\sin \vartheta_2}{\sin \vartheta_1} \\
	\frac{E_1}{E_2} = \frac{\sin \vartheta_2}{\sin \vartheta_1} &= \frac{\varepsilon_{r2}}{\varepsilon_{r1}} \frac{\cos \vartheta_2}{\cos \vartheta_1} \\
	\Aboxed{\frac{\tan \vartheta_2}{\tan \vartheta_1} &= \frac{\varepsilon_{r2}}{\varepsilon_{r1}}}
\end{align*}
Si vede come la discontinuità della componente normale di $ \vec{E}  $, insieme alla continuità della componente tangenziale, comporta un cambiamento di direzione delle linee di forza del campo elettrico, fenomeno che si chiama anche \emph{rifrazione delle linee di forza}.







































\section{Cenni ai dielettrici anisotropi}

Per i dielettrici è conveniente introdurre il nuovo campo vettoriale $\vec{D}$ definito, per i dielettrici isotropi, ossia dotati di simmetria spaziale, come:
\[
	\vec{D} =\varepsilon_0 \vec{E} + \vec{P} = \varepsilon_0 \varepsilon_r \vec{E}
\]
Tuttavia non tutti i dielettrici sono isotropi. Per i cosiddetti dielettrici \emph{anisotropi}, come i cristalli, la polarizzazione $\vec{P}$ non è parallela al campo $\vec{E}$ così come $\vec{D}$ ed $\vec{E}$ non sono paralleli. In questi materiali esistono tuttavia tre direzioni caratteristiche, tra loro ortogonali, lungo le quali $\vec{E}$, $\vec{P}$ e $\vec{D}$ sono paralleli. Detta $i$ una qualsiasi di queste direzioni, si ha:
\[
	D_i = \varepsilon_0 \sum_j \varepsilon_{ij} E_j \qquad i=x,y,z
\]
Il legame fra $\vec{D}$ ed $\vec{E}$ avviene non più tramite proporzionalità diretta, ma attraverso una matrice che prende il nome di \textbf{tensore dielettrico relativo}. Le tre particolari direzioni prendono il nome di assi cristallografici o assi ottici del dielettrico.







































\section{Energia elettrostatica nei dielettrici}

Potremmo chiederci qualcosa riguarda all'energia elettrostatica necessaria da spendere quando vogliamo creare una distribuzione di carica all'interno di un dielettrico. Oltre a dover vincere la repulsione fra le varie cariche, dobbiamo anche spendere del lavoro per polarizzare l'unità di volume del dielettrico (lavoro che nel vuoto è nullo). Stiamo infatti generando all'interno del dielettrico un campo elettrico che lo polarizza: si separano leggermente cariche positive e negative, cosa che richiede a sua volta una spesa di energia. Vorremmo dunque calcolare tale energia elettrostatica che immagazziniamo in una distribuzione creata all'interno del dielettrico.

Abbiamo visto che la divergenza di $\vec{D}$ dipende esclusivamente dalle cariche libere e quindi: $ \text{div}\vec{D} =\rho_l  $. La quantità:
\[
	U_e = \frac{1}{2} \int_{\tau}\rho_lVd\tau = \frac{1}{2} \int_{\tau}(\text{div}\vec{D})Vd\tau
\]
Rappresenta la \emph{densità di energia elettrostatica}. Notiamo che:
\[
	\vec{\nabla} \cdot (V\vec{D}) = V\vec{\nabla} \cdot \vec{D} + \vec{D} \cdot \vec{\nabla} V
\]
Ricordando che:
\begin{gather*}
	\left. \begin{array}{r}
	 	\vec{E} = - \vec{\nabla} V \\
		\vec{\nabla} \cdot (V\vec{D}) = V\vec{\nabla} \cdot \vec{D} + \vec{D} \cdot \vec{\nabla} V
	\end{array} \right\} \implies \\
	\implies \vec{\nabla} \cdot (V\vec{D}) = V\vec{\nabla} \cdot \vec{D} -  \vec{D} \cdot \vec{E} \\
	V\vec{\nabla} \cdot \vec{D} = \vec{\nabla} \cdot (V\vec{D}) + \vec{D} \cdot \vec{E}
\end{gather*}
Possiamo a questo punto applicare il teorema della divergenza.
\begin{align*}
	U_e &= \frac{1}{2} \int_{\tau}\vec{D} \cdot \vec{E} d\tau + \frac{1}{2} \int_{\tau} \vec{\nabla} \cdot (V\vec{D}) d\tau \\
	&= \frac{1}{2} \int_{\tau}\vec{D} \cdot \vec{E} d\tau + \underbrace{\frac{1}{2} \int_{\Sigma} (V\vec{D})\cdot \vec{n} \,dS}_{\to 0 \text{ per}\Sigma \to \infty}
\end{align*}
Per $\Sigma$ che si estende all'infinito, $ V\vec{D} $ è asintotico a $1/r^3$, si ha quindi che l'ultimo integrale della somma tende a zero. Possiamo concludere che:
\[
	\boxed{U_e = \frac{1}{2} \int_{\text{tutto lo spazio}}\vec{D} \cdot \vec{E} \,d\tau}
\]
La densità di energia elettrostatica per unità di volume sarà semplicemente:
\[
	\boxed{u_e=\frac{1}{2} \vec{D} \cdot \vec{E}}
\]
In realtà questa è l'espressione più generale di densità di energia elettrostatica. Essa vale anche per i dielettrici anisotropi, infatti il prodotto scalare mette in evidenza che $\vec{D}$ ed $\vec{E}$ non sono necessariamente paralleli, e può essere ricondotta al caso del vuoto o del dielettrico isotropo. Nel vuoto infatti:
\[
	\vec{D} =\varepsilon_0 \vec{E} \implies U_e=\frac{1}{2} \varepsilon_0 E^2
\]
Su un dielettrico isotropo:
\[
	U_e = \frac{1}{2} (\varepsilon_0 \varepsilon_r \vec{E} )\vec{E} = \frac{1}{2} \varepsilon_0 \varepsilon_r E^2
\]
Questa energia è $ \varepsilon_r  $ volte più alta rispetto a quella da spendere nel caso del vuoto. Il motivo è che per creare la distribuzione di carica del dielettrico dobbiamo compiere un per vincere la repulsione elettrostatica tra le cariche, ma anche uno per via della polarizzazione delle molecole del dielettrico.




























































\chapter{Corrente elettrica nei conduttori}

\section{Fenomeni di conduzione di carica elettrica}

Ritorniamo ai materiali conduttori in equilibrio elettrostatico. Abbiamo visto che in tali condizioni le eventuali cariche in eccesso sono depositate solo sulla superficie, all'interno il campo elettrico è nullo e la densità di carica libera anche essa è zero. Per ogni elettrone c'è un nucleo positivo vicino che lo neutralizza, non c'è squilibrio di cariche, inoltre esse all'equilibrio sono in quiete. Questo concetto ha senso solo dal punto di vista statistico. Se consideriamo un pezzettino di conduttore, all'equilibrio avremo tante cariche negative che si muovono in modo casuale secondo un moto di agitazione termica molto simile a quello che abbiamo in un gas di molecole. Se calcoliamo il valore medio della velocità vettoriale, anche prendendo un piccolissimo volume, avremo che essa è zero. Questo accade perché per ogni elettrone che si muove in una direzione c'è ne è sicuramente un altro che si muove in direzione opposta.
\[
	\langle \vec{v} \rangle = \frac{\sum_i \vec{v}_i}{N}
\]
Dal punto di vista concettuale, questo ci comunica che non esiste una direzione di moto preferenziale per gli elettroni. In realtà, anche in un volume piccolissimo, essi sono numerosi e hanno una certa velocità. Nel rame e nell'argento, la densità di elettroni per unità di volume è dell'ordine di $10^{28}/m^3$; l'ordine di grandezza è lo stesso per tutti i conduttori metallici. Capiamo quindi Che s proviamo a calcolare la velocità quadratica media, questa è ben lontana dall'essere pari a $0$.
\[
	\overline{v} = \sqrt{\langle v^2 \rangle}
\]
A $300 K$, la velocità quadratica media è $120\,km/s$.
\begin{figure}[htpb]
	\centering
	

	% Pattern Info
	 
	\tikzset{
	pattern size/.store in=\mcSize, 
	pattern size = 5pt,
	pattern thickness/.store in=\mcThickness, 
	pattern thickness = 0.3pt,
	pattern radius/.store in=\mcRadius, 
	pattern radius = 1pt}
	\makeatletter
	\pgfutil@ifundefined{pgf@pattern@name@_7pir68r1b}{
	\pgfdeclarepatternformonly[\mcThickness,\mcSize]{_7pir68r1b}
	{\pgfqpoint{0pt}{0pt}}
	{\pgfpoint{\mcSize+\mcThickness}{\mcSize+\mcThickness}}
	{\pgfpoint{\mcSize}{\mcSize}}
	{
	\pgfsetcolor{\tikz@pattern@color}
	\pgfsetlinewidth{\mcThickness}
	\pgfpathmoveto{\pgfqpoint{0pt}{0pt}}
	\pgfpathlineto{\pgfpoint{\mcSize+\mcThickness}{\mcSize+\mcThickness}}
	\pgfusepath{stroke}
	}}
	\makeatother

	% Pattern Info
	 
	\tikzset{
	pattern size/.store in=\mcSize, 
	pattern size = 5pt,
	pattern thickness/.store in=\mcThickness, 
	pattern thickness = 0.3pt,
	pattern radius/.store in=\mcRadius, 
	pattern radius = 1pt}
	\makeatletter
	\pgfutil@ifundefined{pgf@pattern@name@_9503dhk18}{
	\pgfdeclarepatternformonly[\mcThickness,\mcSize]{_9503dhk18}
	{\pgfqpoint{0pt}{0pt}}
	{\pgfpoint{\mcSize+\mcThickness}{\mcSize+\mcThickness}}
	{\pgfpoint{\mcSize}{\mcSize}}
	{
	\pgfsetcolor{\tikz@pattern@color}
	\pgfsetlinewidth{\mcThickness}
	\pgfpathmoveto{\pgfqpoint{0pt}{0pt}}
	\pgfpathlineto{\pgfpoint{\mcSize+\mcThickness}{\mcSize+\mcThickness}}
	\pgfusepath{stroke}
	}}
	\makeatother
	\tikzset{every picture/.style={line width=0.75pt}} %set default line width to 0.75pt        

	\begin{tikzpicture}[x=0.75pt,y=0.75pt,yscale=-0.7,xscale=0.7]
	%uncomment if require: \path (0,300); %set diagram left start at 0, and has height of 300

	%Shape: Circle [id:dp9219367414045321] 
	\draw  [pattern=_7pir68r1b,pattern size=6pt,pattern thickness=0.75pt,pattern radius=0pt, pattern color={rgb, 255:red, 222; green, 222; blue, 222}] (95,164.25) .. controls (95,109.44) and (139.44,65) .. (194.25,65) .. controls (249.06,65) and (293.5,109.44) .. (293.5,164.25) .. controls (293.5,219.06) and (249.06,263.5) .. (194.25,263.5) .. controls (139.44,263.5) and (95,219.06) .. (95,164.25) -- cycle ;
	%Shape: Ellipse [id:dp8917923509185433] 
	\draw  [pattern=_9503dhk18,pattern size=6pt,pattern thickness=0.75pt,pattern radius=0pt, pattern color={rgb, 255:red, 222; green, 222; blue, 222}] (468.57,161.56) .. controls (468.57,132.03) and (492.5,108.09) .. (522.03,108.09) .. controls (551.56,108.09) and (575.5,132.03) .. (575.5,161.56) .. controls (575.5,191.09) and (551.56,215.02) .. (522.03,215.02) .. controls (492.5,215.02) and (468.57,191.09) .. (468.57,161.56) -- cycle ;
	%Curve Lines [id:da025791305803244624] 
	\draw [line width=2.25]    (293.5,164.25) .. controls (356.5,77) and (418.5,249) .. (468.57,161.56) ;
	%Curve Lines [id:da0029303514799270403] 
	\draw    (334,130) .. controls (373.2,127.55) and (377.83,151.51) .. (413.76,166.59) ;
	\draw [shift={(416,167.5)}, rotate = 201.54] [fill={rgb, 255:red, 0; green, 0; blue, 0 }  ][line width=0.08]  [draw opacity=0] (10.72,-5.15) -- (0,0) -- (10.72,5.15) -- (7.12,0) -- cycle    ;
	%Curve Lines [id:da8872938466077263] 
	\draw    (373.67,170.83) .. controls (382.13,177.1) and (395.89,187.2) .. (409.69,192.81) ;
	\draw [shift={(412.33,193.83)}, rotate = 199.98] [fill={rgb, 255:red, 0; green, 0; blue, 0 }  ][line width=0.08]  [draw opacity=0] (10.72,-5.15) -- (0,0) -- (10.72,5.15) -- (7.12,0) -- cycle    ;

	% Text Node
	\draw (194.25,55) node    {$+$};
	% Text Node
	\draw (194.25,275) node    {$+$};
	% Text Node
	\draw (304.25,165) node    {$+$};
	% Text Node
	\draw (84.25,165) node    {$+$};
	% Text Node
	\draw (272.03,87.22) node    {$+$};
	% Text Node
	\draw (116.47,242.78) node    {$+$};
	% Text Node
	\draw (272.03,242.78) node    {$+$};
	% Text Node
	\draw (116.47,87.22) node    {$+$};
	% Text Node
	\draw (237.23,63.74) node    {$+$};
	% Text Node
	\draw (151.27,266.26) node    {$+$};
	% Text Node
	\draw (295.51,207.98) node    {$+$};
	% Text Node
	\draw (92.99,122.02) node    {$+$};
	% Text Node
	\draw (296.24,123.79) node    {$+$};
	% Text Node
	\draw (92.26,206.21) node    {$+$};
	% Text Node
	\draw (235.46,266.99) node    {$+$};
	% Text Node
	\draw (153.04,63.01) node    {$+$};
	% Text Node
	\draw (283.56,55.41) node    {$V_{1}$};
	% Text Node
	\draw (522.03,96.2) node    {$+$};
	% Text Node
	\draw (522.03,225.8) node    {$+$};
	% Text Node
	\draw (586.84,161) node    {$+$};
	% Text Node
	\draw (457.23,161) node    {$+$};
	% Text Node
	\draw (567.86,115.18) node    {$+$};
	% Text Node
	\draw (476.21,206.82) node    {$+$};
	% Text Node
	\draw (567.86,206.82) node    {$+$};
	% Text Node
	\draw (476.21,115.18) node    {$+$};
	% Text Node
	\draw (547.35,101.35) node    {$+$};
	% Text Node
	\draw (496.71,220.65) node    {$+$};
	% Text Node
	\draw (581.68,186.32) node    {$+$};
	% Text Node
	\draw (462.38,135.68) node    {$+$};
	% Text Node
	\draw (582.12,136.72) node    {$+$};
	% Text Node
	\draw (461.95,185.28) node    {$+$};
	% Text Node
	\draw (546.31,221.08) node    {$+$};
	% Text Node
	\draw (497.76,100.92) node    {$+$};
	% Text Node
	\draw (626.18,100.98) node    {$V_{2} < V_{1}$};
	% Text Node
	\draw (386.56,125.41) node    {$\vec{E}$};
	% Text Node
	\draw (366.38,163.68) node    {$+$};


	\end{tikzpicture}
\end{figure}
\FloatBarrier
Immaginiamo di avere due conduttori inizialmente in equilibrio elettrostatico che si trovano a due potenziali elettrostatici differenti $V_1$ e $V_2$. Se li colleghiamo con un filo di materiale conduttore, accade che all'interno di esso apparirà un campo elettrico. Le cariche cominciano a spostarsi da un conduttore all'altro. Il fenomeno è detto di \textbf{conduzione} di carica elettrica, o trasporto. Quando abbiamo un moto ordinato di cariche in una certa direzione, diciamo che in quel conduttore è presente una \textbf{corrente elettrica}. Questo processo di moto di cariche è transitorio e terminerà quando i potenziali Dei conduttori saranno uguali. La corrente elettrica in questo caso specifico dura per un tempo molto limitato che impedisce l'esecuzione di studi sistematici del fenomeno.

I fenomeni di conduzione di carica per questo sono rimasti inesplorati fino all'800, in cui Alessandro Volta inventò il \textbf{generatore di forza elettromotrice}.
Si tratta di un dispositivo in grado di mantenere una differenza di potenziale costante fra due conduttori a contatto. Così facendo il flusso di elettroni può durare per molto più tempo e quindi nel conduttore si instaura una corrente elettrica stabile, in un regime di equilibrio dinamico e non più di equilibrio elettrostatico. Nella sua versione originale la pila di Volt o cella voltaica consta di una serie di elementi ciascuno dei quali è costituito da un disco di zinco, un tampone imbevuto di una soluzione acquosa di acido solforico e un disco di rame. Esso risulta carico positivamente e quello di zinco negativamente. Se si misura con uno strumento elettrostatico la d.d.p. tra i due dischi si trova un valore fisso caratteristico della coppia di metalli che è detto $fem$ della pila. Collegando alle estremità della pila un conduttore, ad esempio un filo metallico, viene stabilita in questo una corrente elettrica costante nel tempo, denominata come corrente continua. In realtà con una pila di Volta la corrente diminuisce molto lentamente nel tempo.

Immaginiamo di collegare il nostro generatore di forza elettromotrice ad un oggetto costituito da materiale conduttore. Inizierà ad esserci un flusso ordinato di cariche elettriche dovuto al campo elettrico prodotto dal generatore.
\begin{figure}[htpb]
	\centering
	

	\tikzset{every picture/.style={line width=0.75pt}} %set default line width to 0.75pt        

	\begin{tikzpicture}[x=0.75pt,y=0.75pt,yscale=-1,xscale=1]
	%uncomment if require: \path (0,300); %set diagram left start at 0, and has height of 300

	%Shape: Battery [id:dp7947896056729546] 
	\draw  [fill={rgb, 255:red, 0; green, 0; blue, 0 }  ,fill opacity=1 ] (148.3,130.4) -- (148.3,94.4) (118.3,86.4) -- (178.3,86.4) (148.3,86.4) -- (148.3,50.4) (133.3,97.6) -- (133.3,94.4) -- (163.3,94.4) -- (163.3,97.6) -- (133.3,97.6) -- cycle ;
	%Straight Lines [id:da8636566827035768] 
	\draw    (148.3,50.4) -- (292.8,50.4) ;
	%Shape: Rectangle [id:dp5030241254746104] 
	\draw   (248.7,71.6) -- (335.8,71.6) -- (335.8,172.6) -- (248.7,172.6) -- cycle ;
	%Shape: Circle [id:dp11503800279281395] 
	\draw   (265.4,112.3) .. controls (265.4,108.16) and (268.76,104.8) .. (272.9,104.8) .. controls (277.04,104.8) and (280.4,108.16) .. (280.4,112.3) .. controls (280.4,116.44) and (277.04,119.8) .. (272.9,119.8) .. controls (268.76,119.8) and (265.4,116.44) .. (265.4,112.3) -- cycle ;

	%Straight Lines [id:da5166399090598475] 
	\draw    (272.9,119.8) -- (272.9,159) ;
	\draw [shift={(272.9,162)}, rotate = 270] [fill={rgb, 255:red, 0; green, 0; blue, 0 }  ][line width=0.08]  [draw opacity=0] (10.72,-5.15) -- (0,0) -- (10.72,5.15) -- (7.12,0) -- cycle    ;
	%Straight Lines [id:da10335817060324226] 
	\draw    (312.9,119.8) -- (312.9,159) ;
	\draw [shift={(312.9,162)}, rotate = 270] [fill={rgb, 255:red, 0; green, 0; blue, 0 }  ][line width=0.08]  [draw opacity=0] (10.72,-5.15) -- (0,0) -- (10.72,5.15) -- (7.12,0) -- cycle    ;
	%Straight Lines [id:da9745597516547506] 
	\draw    (292.8,71.2) -- (292.8,50.4) ;
	%Straight Lines [id:da9311759273417342] 
	\draw    (292.8,193.2) -- (292.8,172.4) ;
	%Straight Lines [id:da04193895564665562] 
	\draw    (148.3,193.2) -- (292.8,193.2) ;
	%Straight Lines [id:da4895770815624165] 
	\draw    (148.3,193.2) -- (148.3,130.4) ;
	%Curve Lines [id:da18786427144500872] 
	\draw    (115.4,120.4) .. controls (103.63,105.73) and (100.05,81.38) .. (115.63,62.44) ;
	\draw [shift={(117.4,60.4)}, rotate = 492.51] [fill={rgb, 255:red, 0; green, 0; blue, 0 }  ][line width=0.08]  [draw opacity=0] (10.72,-5.15) -- (0,0) -- (10.72,5.15) -- (7.12,0) -- cycle    ;

	% Text Node
	\draw (272.9,109.8) node    {$+$};
	% Text Node
	\draw (273.6,88.2) node    {$q$};
	% Text Node
	\draw (313.2,103) node    {$\vec{E}$};
	% Text Node
	\draw (84,89) node    {$\Delta V$};


	\end{tikzpicture}
\end{figure}
\FloatBarrier
\textbf{Osservazione.} \emph{In un metallo gli unici sul 9 portatori di carica sono gli elettroni, ma sappiamo che in altri casi, come la pila o il plasma, le cariche possono anche essere positive.}
Il flusso che ci aspettiamo è la combinazione di un moto caotico a velocità elevatissima sovrapposto a un moto molto lento detto di deriva delle cariche nella direzione del campo elettrico $\vec{E}$. Non ci concentreremo sulla velocità microscopica di ciascun elettrone ma solo su questa \emph{velocità di deriva}, $ v_d  $. In questo caso il valore medio delle velocità degli elettroni non sarà più zero, ma sarà uguale a $v_d (0.15\,mm/s)$. Vediamo quindi come la perturbazione introdotta dalla conduzione sulla velocità dell'elettrone sia molto piccola.
\[
	\langle \vec{v} \rangle = \vec{v}_d
\]
Abbiamo quindi il nostro conduttore, al cui interno è presente un campo elettrico e le cui cariche positive si muovono per questo a una certa velocità di deriva $ v_d  $. Siamo in grado di calcolarla in tutti i punti del conduttore in maniera precisa, definendo il campo vettoriale:
\[
	\vec{v}_d = \vec{v}_d(x,y,z)
\]
\begin{figure}[htpb]
	\centering
	

	% Pattern Info
	 
	\tikzset{
	pattern size/.store in=\mcSize, 
	pattern size = 5pt,
	pattern thickness/.store in=\mcThickness, 
	pattern thickness = 0.3pt,
	pattern radius/.store in=\mcRadius, 
	pattern radius = 1pt}
	\makeatletter
	\pgfutil@ifundefined{pgf@pattern@name@_3zx8u6z0b}{
	\pgfdeclarepatternformonly[\mcThickness,\mcSize]{_3zx8u6z0b}
	{\pgfqpoint{0pt}{0pt}}
	{\pgfpoint{\mcSize+\mcThickness}{\mcSize+\mcThickness}}
	{\pgfpoint{\mcSize}{\mcSize}}
	{
	\pgfsetcolor{\tikz@pattern@color}
	\pgfsetlinewidth{\mcThickness}
	\pgfpathmoveto{\pgfqpoint{0pt}{0pt}}
	\pgfpathlineto{\pgfpoint{\mcSize+\mcThickness}{\mcSize+\mcThickness}}
	\pgfusepath{stroke}
	}}
	\makeatother
	\tikzset{every picture/.style={line width=0.75pt}} %set default line width to 0.75pt        

	\begin{tikzpicture}[x=0.75pt,y=0.75pt,yscale=-1,xscale=1]
	%uncomment if require: \path (0,321); %set diagram left start at 0, and has height of 321

	%Shape: Ellipse [id:dp6145612949263457] 
	\draw   (356.94,221.86) .. controls (342.78,213.69) and (342.9,186.98) .. (357.2,162.2) .. controls (371.51,137.43) and (394.58,123.97) .. (408.74,132.14) .. controls (422.89,140.31) and (422.77,167.02) .. (408.47,191.8) .. controls (394.17,216.57) and (371.09,230.03) .. (356.94,221.86) -- cycle ;
	%Straight Lines [id:da5353682425174289] 
	\draw    (364.34,224.08) -- (155.66,224.08) ;
	%Curve Lines [id:da42914586957077305] 
	\draw  [dash pattern={on 0.84pt off 2.51pt}]  (194.16,130.08) .. controls (229.91,144.25) and (195.41,223.75) .. (155.66,224.08) ;
	%Straight Lines [id:da08156155803614062] 
	\draw    (402.84,130.08) -- (194.16,130.08) ;
	%Curve Lines [id:da8302824113941996] 
	\draw    (194.16,130.08) .. controls (149.41,130.25) and (112.91,219.25) .. (155.66,224.08) ;
	%Straight Lines [id:da707902971274881] 
	\draw    (469.34,177) -- (382.84,177) ;
	\draw [shift={(472.34,177)}, rotate = 180] [fill={rgb, 255:red, 0; green, 0; blue, 0 }  ][line width=0.08]  [draw opacity=0] (10.72,-5.15) -- (0,0) -- (10.72,5.15) -- (7.12,0) -- cycle    ;
	%Straight Lines [id:da9219809989976684] 
	\draw    (457.75,220.25) -- (382.84,177) ;
	\draw [shift={(460.35,221.75)}, rotate = 210] [fill={rgb, 255:red, 0; green, 0; blue, 0 }  ][line width=0.08]  [draw opacity=0] (10.72,-5.15) -- (0,0) -- (10.72,5.15) -- (7.12,0) -- cycle    ;
	%Shape: Arc [id:dp7521632869570065] 
	\draw  [draw opacity=0] (422.59,176.76) .. controls (422.59,176.84) and (422.59,176.92) .. (422.59,177) .. controls (422.59,184.13) and (420.71,190.83) .. (417.41,196.62) -- (382.84,177) -- cycle ; \draw   (422.59,176.76) .. controls (422.59,176.84) and (422.59,176.92) .. (422.59,177) .. controls (422.59,184.13) and (420.71,190.83) .. (417.41,196.62) ;
	%Shape: Ellipse [id:dp32703450933859113] 
	\draw  [pattern=_3zx8u6z0b,pattern size=6pt,pattern thickness=0.75pt,pattern radius=0pt, pattern color={rgb, 255:red, 222; green, 222; blue, 222}] (251.59,130.25) .. controls (266.43,130.25) and (278.45,151.29) .. (278.45,177.25) .. controls (278.45,203.21) and (266.43,224.25) .. (251.59,224.25) .. controls (236.76,224.25) and (224.74,203.21) .. (224.74,177.25) .. controls (224.74,151.29) and (236.76,130.25) .. (251.59,130.25) -- cycle ;
	%Shape: Circle [id:dp2722917258779334] 
	\draw   (301.07,158.3) .. controls (301.07,154.16) and (304.42,150.8) .. (308.57,150.8) .. controls (312.71,150.8) and (316.07,154.16) .. (316.07,158.3) .. controls (316.07,162.44) and (312.71,165.8) .. (308.57,165.8) .. controls (304.42,165.8) and (301.07,162.44) .. (301.07,158.3) -- cycle ;

	%Straight Lines [id:da5661845749281902] 
	\draw    (316.07,158.3) -- (339.67,158.3) ;
	\draw [shift={(342.67,158.3)}, rotate = 180] [fill={rgb, 255:red, 0; green, 0; blue, 0 }  ][line width=0.08]  [draw opacity=0] (10.72,-5.15) -- (0,0) -- (10.72,5.15) -- (7.12,0) -- cycle    ;
	%Shape: Circle [id:dp13344174764785333] 
	\draw   (294.07,188.3) .. controls (294.07,184.16) and (297.42,180.8) .. (301.57,180.8) .. controls (305.71,180.8) and (309.07,184.16) .. (309.07,188.3) .. controls (309.07,192.44) and (305.71,195.8) .. (301.57,195.8) .. controls (297.42,195.8) and (294.07,192.44) .. (294.07,188.3) -- cycle ;

	%Straight Lines [id:da20562510595451156] 
	\draw    (309.07,188.3) -- (332.67,188.3) ;
	\draw [shift={(335.67,188.3)}, rotate = 180] [fill={rgb, 255:red, 0; green, 0; blue, 0 }  ][line width=0.08]  [draw opacity=0] (10.72,-5.15) -- (0,0) -- (10.72,5.15) -- (7.12,0) -- cycle    ;
	%Straight Lines [id:da7301870407518438] 
	\draw    (364.34,264.08) -- (155.66,264.08) ;
	\draw [shift={(155.66,264.08)}, rotate = 360] [color={rgb, 255:red, 0; green, 0; blue, 0 }  ][line width=0.75]    (0,5.59) -- (0,-5.59)   ;
	\draw [shift={(364.34,264.08)}, rotate = 360] [color={rgb, 255:red, 0; green, 0; blue, 0 }  ][line width=0.75]    (0,5.59) -- (0,-5.59)   ;
	%Straight Lines [id:da8249226126729325] 
	\draw    (447,107) -- (127.5,107) ;
	\draw [shift={(287.25,107)}, rotate = 180] [fill={rgb, 255:red, 0; green, 0; blue, 0 }  ][line width=0.08]  [draw opacity=0] (10.72,-5.15) -- (0,0) -- (10.72,5.15) -- (7.12,0) -- cycle    ;
	%Straight Lines [id:da11077874341684812] 
	\draw    (447,77) -- (127.5,77) ;
	\draw [shift={(287.25,77)}, rotate = 180] [fill={rgb, 255:red, 0; green, 0; blue, 0 }  ][line width=0.08]  [draw opacity=0] (10.72,-5.15) -- (0,0) -- (10.72,5.15) -- (7.12,0) -- cycle    ;

	% Text Node
	\draw (430.5,188) node    {$\vartheta $};
	% Text Node
	\draw (487,176.5) node    {$\vec{v}_{d}$};
	% Text Node
	\draw (471,227) node    {$\vec{n}$};
	% Text Node
	\draw (394.5,153) node    {$dS$};
	% Text Node
	\draw (253,234.5) node    {$dS_{n}$};
	% Text Node
	\draw (308.57,155.8) node    {$+$};
	% Text Node
	\draw (301.57,185.8) node    {$+$};
	% Text Node
	\draw (253,274.5) node    {$dl$};
	% Text Node
	\draw (243,90.33) node    {$\vec{E}$};


	\end{tikzpicture}
\end{figure}
\FloatBarrier
Immaginiamo di costruire una superficie tubolare (tubo di flusso) che ha le linee di flusso del campo $\vec{E}$ come generatrici. Consideriamo all'interno di tale tubo di flusso di volume infinitesimo una sezione infinitesima di area $dS$. Poniamo che n sia la normale a questa $dS$. Con queste condizioni, studiamo il moto delle cariche in un intervallo di tempo infinitesimo, chiedendoci quanta carica $dQ$ attraversi questa sezione $dS$.
Dato che tutte le cariche si spostano con la stessa velocità media $ v_d  $, lo spazio da esse percorso è il medesimo: $dl =v_d\,dt$. Per cui, se $n$ è il numero di portatori di carica $q$ per unità di volume, la carica complessiva che passa attraverso $dS$ nel tempo $dt$ è quella contenuta nel volume infinitesimo dato da:
\begin{align*}
	d\tau &= dS_ndl = \overbrace{dS\cos \vartheta}^{dS_n}\,dl = dS\cos \vartheta \,\overbrace{v_d\,dt}^{dl} \\
	dQ &= d\tau \cdot n \cdot q \\
	&= \underbrace{dS\cos \vartheta \,v_d\,dt}_{d\tau}\cdot n\cdot q
\end{align*}
Possiamo a questo punto introdurre l'intensità di corrente elettrica come il rapporto fra la quantità di carica $dQ$ che attraversa la sezione $dS$ e l'intervallo di tempo $dt$ in cui ciò accade.
\[
	\boxed{dI=\frac{dQ}{dt}=nq\vec{v}_d\cdot \vec{n} \,dS} \qquad [I]=\frac{[Q]}{[T]} = \left( \frac{C}{S} \right) = A \quad \text{Ampere}
\]
Introduciamo inoltre una nuova grandezza vettoriale che chiamiamo vettore \textbf{densità di corrente elettrica}, generalmente indicato con la lettera $\vec{J}$, pari a:
\[
	\vec{J} =nq\vec{v}_d \implies dI=\vec{J} \cdot \vec{n} \,dS = d\Phi (\vec{J})
\]
La corrente è visibile anche come flusso di questo vettore. Se abbiamo un conduttore macroscopico:
\[
	I=\int_{\Sigma}dI = \int_{\Sigma}\vec{J} \cdot \vec{n} \,dS = \Phi_{\Sigma} (\vec{J})
\]
Se avessimo un conduttore i cui portatori di carica sono negativi, la velocità di deriva sarebbe diretta in verso opposto. Andando a calcolare $\vec{J}$, avremmo il prodotto della carica del singolo portatore per la velocità. Quindi calcolando il vettore $\vec{J}$ troveremmo un vettore diretto come il campo elettrico, perché il segno della carica compensa il segno della $ v_d  $. Non ci occuperemo quindi del fatto che la carica sia positiva o negativa.







































\section{Legge di continuità della corrente elettrica}

Consideriamo all'interno di un conduttore una regione di spazio di volume $ \tau  $ delimitato da una superficie chiusa $\Sigma$. Se la regione è sede di corrente elettrica, definita dal vettore densità di corrente $\vec{J}$, avremo che:
\[
	Q_i(t+dt) - Q_i(t)=-dQ
\]
La carica all'interno della superficie è diminuita. Ci aspettiamo che abbia attraversato la superficie $\Sigma$ e sia andata via. A $-dQ$ deve corrispondere un flusso di cariche verso l'esterno e quindi, in accordo con quanto detto, una corrente elettrica che sta attraversando questa superficie.
\begin{gather*}
	I=\frac{dQ}{dt} \\
	\frac{dQ_i}{dt}=\frac{Q_i(t+dt)-Q_i(t)}{dt} = -\frac{dQ}{dt} = -I = -\int_{\Sigma}\vec{J} \cdot \vec{n} dS
\end{gather*}
Introdotto il versore normale alla superficie, $\vec{n}$, diretta verso l'esterno in ogni punto di $\Sigma$, si ha:
\[
	\frac{dQ_i}{dt} = \frac{d}{dt} \int_{\tau}\rho_ld\tau = -\int_{\Sigma}\vec{J} \cdot \vec{n} dS = -I
\]
Ora possiamo applicare all'ultimo termine il teorema della divergenza
\begin{align*}
	\int_{\tau}\frac{d}{dt}\rho_ld\tau &= -\int_{\tau}\text{div}\vec{J} d\tau \\
	\int_{\tau}\left[ \frac{d\rho_l}{dt}+\text{div}\vec{J}  \right] d\tau &= 0
\end{align*}
Il che implica che la funzione integranda sia nulla, non avendo fatto ipotesi di alcun tipo sulla forma:
\[
	\boxed{\text{div}\vec{J} = - \frac{\partial \rho_l}{\partial t}}
\]
Per capire il significato di questa relazione, supponiamo che $\vec{J}$ abbia una sorgente delle linee di flusso. Se la corrente si sta allontanando dal punto considerato, dato che non possiamo creare o distruggere carica, in tal punto avremo una diminuzione di carica, a cui corrisponderà derivata negativa. La relazione è nota come \textbf{equazione di continuità della corrente elettrica} ed esprime quindi il \emph{principio di conservazione della carica elettrica}.

Un caso particolare si ha quando la carica contenuta all'interno della superficie non varia, per cui la divergenza di $\vec{J}$ è uguale a zero. Tale regime è noto come \textbf{regime di conduzione stazionario}. Se $\vec{J}$ ha divergenza sempre nulla la derivata di $ \rho $ rispetto al tempo è pari a zero ovunque.
\[
	\text{div}\vec{J} = 0 \qquad \text{regime stazionario}
\]
Espressione che si chiama \textbf{equazione di continuità della corrente elettrica in regime stazionario}.
\begin{figure}[htpb]
	\centering
	

	\tikzset{every picture/.style={line width=0.75pt}} %set default line width to 0.75pt        

	\begin{tikzpicture}[x=0.75pt,y=0.75pt,yscale=-1,xscale=1]
	%uncomment if require: \path (0,300); %set diagram left start at 0, and has height of 300

	%Shape: Can [id:dp017404037796793226] 
	\draw   (113.45,64) -- (300.05,64) .. controls (310.24,64) and (318.5,91.53) .. (318.5,125.5) .. controls (318.5,159.47) and (310.24,187) .. (300.05,187) -- (113.45,187) .. controls (103.26,187) and (95,159.47) .. (95,125.5) .. controls (95,91.53) and (103.26,64) .. (113.45,64) .. controls (123.64,64) and (131.9,91.53) .. (131.9,125.5) .. controls (131.9,159.47) and (123.64,187) .. (113.45,187) ;
	%Curve Lines [id:da8183553062560818] 
	\draw  [dash pattern={on 0.84pt off 2.51pt}]  (300.05,64) .. controls (275.1,64.75) and (274.6,186.25) .. (300.05,187) ;
	%Shape: Circle [id:dp011877859992850626] 
	\draw   (329.07,88) .. controls (329.07,83.86) and (332.42,80.5) .. (336.57,80.5) .. controls (340.71,80.5) and (344.07,83.86) .. (344.07,88) .. controls (344.07,92.14) and (340.71,95.5) .. (336.57,95.5) .. controls (332.42,95.5) and (329.07,92.14) .. (329.07,88) -- cycle ;

	%Straight Lines [id:da44264635934188323] 
	\draw    (344.07,88) -- (367.67,88) ;
	\draw [shift={(370.67,88)}, rotate = 180] [fill={rgb, 255:red, 0; green, 0; blue, 0 }  ][line width=0.08]  [draw opacity=0] (10.72,-5.15) -- (0,0) -- (10.72,5.15) -- (7.12,0) -- cycle    ;
	%Shape: Circle [id:dp2671896177393527] 
	\draw   (330.07,117) .. controls (330.07,112.86) and (333.42,109.5) .. (337.57,109.5) .. controls (341.71,109.5) and (345.07,112.86) .. (345.07,117) .. controls (345.07,121.14) and (341.71,124.5) .. (337.57,124.5) .. controls (333.42,124.5) and (330.07,121.14) .. (330.07,117) -- cycle ;

	%Straight Lines [id:da8934265394022132] 
	\draw    (345.07,117) -- (368.67,117) ;
	\draw [shift={(371.67,117)}, rotate = 180] [fill={rgb, 255:red, 0; green, 0; blue, 0 }  ][line width=0.08]  [draw opacity=0] (10.72,-5.15) -- (0,0) -- (10.72,5.15) -- (7.12,0) -- cycle    ;
	%Shape: Circle [id:dp24014124719456853] 
	\draw   (42.07,101) .. controls (42.07,96.86) and (45.42,93.5) .. (49.57,93.5) .. controls (53.71,93.5) and (57.07,96.86) .. (57.07,101) .. controls (57.07,105.14) and (53.71,108.5) .. (49.57,108.5) .. controls (45.42,108.5) and (42.07,105.14) .. (42.07,101) -- cycle ;

	%Straight Lines [id:da728429653230475] 
	\draw    (57.07,101) -- (80.67,101) ;
	\draw [shift={(83.67,101)}, rotate = 180] [fill={rgb, 255:red, 0; green, 0; blue, 0 }  ][line width=0.08]  [draw opacity=0] (10.72,-5.15) -- (0,0) -- (10.72,5.15) -- (7.12,0) -- cycle    ;
	%Shape: Circle [id:dp707704073297533] 
	\draw   (42.07,127) .. controls (42.07,122.86) and (45.42,119.5) .. (49.57,119.5) .. controls (53.71,119.5) and (57.07,122.86) .. (57.07,127) .. controls (57.07,131.14) and (53.71,134.5) .. (49.57,134.5) .. controls (45.42,134.5) and (42.07,131.14) .. (42.07,127) -- cycle ;

	%Straight Lines [id:da3887493517750147] 
	\draw    (57.07,127) -- (80.67,127) ;
	\draw [shift={(83.67,127)}, rotate = 180] [fill={rgb, 255:red, 0; green, 0; blue, 0 }  ][line width=0.08]  [draw opacity=0] (10.72,-5.15) -- (0,0) -- (10.72,5.15) -- (7.12,0) -- cycle    ;
	%Shape: Circle [id:dp7135595126886061] 
	\draw   (42.07,155) .. controls (42.07,150.86) and (45.42,147.5) .. (49.57,147.5) .. controls (53.71,147.5) and (57.07,150.86) .. (57.07,155) .. controls (57.07,159.14) and (53.71,162.5) .. (49.57,162.5) .. controls (45.42,162.5) and (42.07,159.14) .. (42.07,155) -- cycle ;

	%Straight Lines [id:da6009103508315261] 
	\draw    (57.07,155) -- (80.67,155) ;
	\draw [shift={(83.67,155)}, rotate = 180] [fill={rgb, 255:red, 0; green, 0; blue, 0 }  ][line width=0.08]  [draw opacity=0] (10.72,-5.15) -- (0,0) -- (10.72,5.15) -- (7.12,0) -- cycle    ;
	%Straight Lines [id:da9479029623698474] 
	\draw    (116.07,154) -- (159.5,154) ;
	\draw [shift={(162.5,154)}, rotate = 180] [fill={rgb, 255:red, 0; green, 0; blue, 0 }  ][line width=0.08]  [draw opacity=0] (10.72,-5.15) -- (0,0) -- (10.72,5.15) -- (7.12,0) -- cycle    ;
	%Straight Lines [id:da6793377751043193] 
	\draw    (166.07,109) -- (229.5,109) ;
	\draw [shift={(232.5,109)}, rotate = 180] [fill={rgb, 255:red, 0; green, 0; blue, 0 }  ][line width=0.08]  [draw opacity=0] (10.72,-5.15) -- (0,0) -- (10.72,5.15) -- (7.12,0) -- cycle    ;
	%Straight Lines [id:da8906300638878115] 
	\draw    (299.07,154) -- (342.5,154) ;
	\draw [shift={(345.5,154)}, rotate = 180] [fill={rgb, 255:red, 0; green, 0; blue, 0 }  ][line width=0.08]  [draw opacity=0] (10.72,-5.15) -- (0,0) -- (10.72,5.15) -- (7.12,0) -- cycle    ;

	% Text Node
	\draw (336.57,85.5) node    {$+$};
	% Text Node
	\draw (337.57,114.5) node    {$+$};
	% Text Node
	\draw (49.57,98.5) node    {$+$};
	% Text Node
	\draw (49.57,124.5) node    {$+$};
	% Text Node
	\draw (49.57,152.5) node    {$+$};
	% Text Node
	\draw (177,152) node    {$\vec{n}_{1}$};
	% Text Node
	\draw (245,107) node    {$\vec{J}$};
	% Text Node
	\draw (360,152) node    {$\vec{n}_{2}$};


	\end{tikzpicture}
\end{figure}
\FloatBarrier
Tale equazione sta a significare che se consideriamo un tratto di conduttore in regime stazionario, \emph{tante cariche entrano, tante ne devono uscire}: non ci deve essere accumulo di carica in un conduttore. La superficie laterale del conduttore è allora un tubo di flusso per $\vec{J}$ perché le sue linee devono essere parallele per poter evitare accumulo di carica.
\begin{align*}
	\Phi_{\Sigma}(\vec{J}) &= \int_{\tau}\text{div}\vec{J} d\tau = 0 \\
	&= \underbrace{\int_{\text{sup. lat.}}\vec{J} \cdot \vec{n} dS}_{=0} +\int_{B_2}\vec{J} \cdot \vec{n}_2 dS + \int_{B_1}\vec{J} \cdot (-\vec{n}_1 ) dS \\
	&= \int_{B_2}\vec{J} \cdot \vec{n}_2 dS - \int_{B_1}\vec{J} \cdot \vec{n}_1 dS = 0\\
	&\implies \underbrace{\int_{B_2}\vec{J} \cdot \vec{n}_2 dS}_{I_2} = \underbrace{\int_{B_1}\vec{J} \cdot \vec{n}_1 dS}_{I_1}
\end{align*}
I due integrali rappresentano le correnti che stanno entrando e uscendo dalle basi uno e due. Questa formula ci dice che in regime stazionario la corrente $I_2$ in uscita dal conduttore è pari alla corrente $I_1$ in ingresso.
In condizioni stazionarie l'intensità di corrente è la stessa attraverso ogni sezione del conduttore.




















\subsection{Leggi di Kirchhoff}

I circuiti elettrici sono sistemi costituiti dalle interconnessioni di semplici dispositivi detti elementi. Per descrivere il comportamento di tali elementi si utilizzano le variabili tensione e corrente. Gli elementi geometrici distintivi di un circuito sono i nodi e i rami. Un nodo è un punto nel quale convengono almeno tre conduttori. I nodi sono collegati da rami, in cui possono esserci componenti attivi (generatori) o componenti passivi (resistori). All'interno di una rete è possibile individuare determinati cammini chiusi, detti maglie, costituiti da più rami.

Se siamo in regime stazionario, considerando un nodo e chiudendolo in una superficie $\Sigma$, la divergenza di $\vec{J}$ è zero, il flusso di $\vec{J}$ attraverso la superficie chiusa è nullo. Ma, dato che questo flusso di $\vec{J}$ si può interpretare in questo modo:
\[
	\Phi_{\Sigma}(\vec{J} )= 0 = I_1+I_2+I_3 \implies \boxed{\sum_i I_i=0}
\]
considerando positive le correnti uscenti.

Arriviamo a concludere che la sommatoria delle correnti entranti e uscenti da un nodo è zero. Questo risultato è noto come \textbf{prima legge di Kirchhoff}.
Consideriamo ora una maglia del circuito.
\begin{figure}[htpb]
	\centering
	

	\tikzset{every picture/.style={line width=0.75pt}} %set default line width to 0.75pt        

	\begin{tikzpicture}[x=0.75pt,y=0.75pt,yscale=-1,xscale=1]
	%uncomment if require: \path (0,300); %set diagram left start at 0, and has height of 300

	%Curve Lines [id:da11127780512440166] 
	\draw    (154.5,66) .. controls (214.5,71) and (237.5,132) .. (240.5,175) ;
	%Curve Lines [id:da10636274290526448] 
	\draw    (154.5,66) .. controls (198.5,28) and (267.62,34.69) .. (304.5,57) ;
	%Curve Lines [id:da8804350288523184] 
	\draw    (240.5,175) .. controls (240.5,124) and (268.5,78) .. (304.5,57) ;
	%Curve Lines [id:da869349992627678] 
	\draw    (103.5,190) .. controls (91.79,133.05) and (112.5,81) .. (154.5,66) ;
	%Curve Lines [id:da43043148147300325] 
	\draw    (240.5,175) .. controls (197.5,207) and (143.42,206.27) .. (103.5,190) ;
	%Curve Lines [id:da40348725078307845] 
	\draw    (141.5,145) .. controls (183.5,199) and (215.5,112) .. (162.5,115) ;
	\draw [shift={(186.54,151.27)}, rotate = 485.9] [fill={rgb, 255:red, 0; green, 0; blue, 0 }  ][line width=0.08]  [draw opacity=0] (10.72,-5.15) -- (0,0) -- (10.72,5.15) -- (7.12,0) -- cycle    ;
	%Shape: Circle [id:dp4401661675359796] 
	\draw  [fill={rgb, 255:red, 0; green, 0; blue, 0 }  ,fill opacity=1 ] (237.75,175) .. controls (237.75,173.48) and (238.98,172.25) .. (240.5,172.25) .. controls (242.02,172.25) and (243.25,173.48) .. (243.25,175) .. controls (243.25,176.52) and (242.02,177.75) .. (240.5,177.75) .. controls (238.98,177.75) and (237.75,176.52) .. (237.75,175) -- cycle ;
	%Shape: Circle [id:dp7453620979054527] 
	\draw  [fill={rgb, 255:red, 0; green, 0; blue, 0 }  ,fill opacity=1 ] (100.75,190) .. controls (100.75,188.48) and (101.98,187.25) .. (103.5,187.25) .. controls (105.02,187.25) and (106.25,188.48) .. (106.25,190) .. controls (106.25,191.52) and (105.02,192.75) .. (103.5,192.75) .. controls (101.98,192.75) and (100.75,191.52) .. (100.75,190) -- cycle ;
	%Shape: Circle [id:dp05292139092767889] 
	\draw  [fill={rgb, 255:red, 0; green, 0; blue, 0 }  ,fill opacity=1 ] (151.75,66) .. controls (151.75,64.48) and (152.98,63.25) .. (154.5,63.25) .. controls (156.02,63.25) and (157.25,64.48) .. (157.25,66) .. controls (157.25,67.52) and (156.02,68.75) .. (154.5,68.75) .. controls (152.98,68.75) and (151.75,67.52) .. (151.75,66) -- cycle ;
	%Shape: Circle [id:dp5530571859485116] 
	\draw  [fill={rgb, 255:red, 0; green, 0; blue, 0 }  ,fill opacity=1 ] (301.75,57) .. controls (301.75,55.48) and (302.98,54.25) .. (304.5,54.25) .. controls (306.02,54.25) and (307.25,55.48) .. (307.25,57) .. controls (307.25,58.52) and (306.02,59.75) .. (304.5,59.75) .. controls (302.98,59.75) and (301.75,58.52) .. (301.75,57) -- cycle ;

	% Text Node
	\draw (152,122) node    {$\gamma $};
	% Text Node
	\draw (98.4,202.8) node    {$A$};
	% Text Node
	\draw (250.4,182) node    {$B$};
	% Text Node
	\draw (141.6,55.6) node    {$C$};


	\end{tikzpicture}
\end{figure}
\FloatBarrier
Ricordando che il campo elettrostatico è conservativo, abbiamo che:
\[
	\oint_{\gamma} \vec{E} \cdot d\vec{l} = 0 = (V_a-V_b  ) + (V_b-V_c  )+ (V_c-V_a  ) = \Delta V_{ab} + \Delta V_{bc} + \Delta V_{ca}
\]
Allora
\[
	\boxed{\sum_i \Delta V_i=0}
\]
Otteniamo così un altro legge, affermante che la sommatoria delle differenze di potenziale calcolate lungo un percorso chiuso deve essere pari a zero. Essa è nota come \textbf{legge di Kirchhoff delle tensioni}.
Queste leggi si applicano a tutti i nodi e a tutte le maglie del circuito. Danno un sistema di equazioni per risolverlo.







































\section{Conduttori Ohmici}

Finora abbiamo suddiviso i materiali tra isolanti e conduttori perfetti, ma in realtà la suddivisione non è così netta. Ci sono materiali che a seconda delle situazioni possono comportarsi o da isolanti o da conduttori. Gli isolanti non perfetti ad esempio presentano proprietà conduttrici. Si parla di conduttori ohmici perché fu Ohm a studiarne le proprietà conduttive. Egli scoprì che in regime stazionario, la d.d.p. applicata ai capi di un conduttore metallico e l'intensità di corrente che a seguito di ciò lo attraversa, è pari ad una grandezza, detta resistenza del conduttore, che dipende solamente dal conduttore e dalle sue dimensioni. Quindi i conduttori ohmici sono quei conduttori la cui qualità di opporsi al transito di una corrente di cariche è misurata dalla seguente legge.
\[
	\boxed{\Delta V= RI} \qquad \text{legge di Ohm}
\]
Per i conduttori a sezione costante, la resistenza $ R $ del conduttore ha la seguente forma
\[
	R = \rho \frac{l}{S} = \frac{1}{\sigma}\frac{l}{S}
\]
Dove $ \rho $ è la resistività del materiale conduttore.
Inoltre $ \frac{1}{\rho}= \sigma $ è detta conducibilità (o conduttività).
\begin{figure}[htpb]
	\centering
	

	\tikzset{every picture/.style={line width=0.75pt}} %set default line width to 0.75pt        

	\begin{tikzpicture}[x=0.75pt,y=0.75pt,yscale=-1,xscale=1]
	%uncomment if require: \path (0,300); %set diagram left start at 0, and has height of 300

	%Shape: Battery [id:dp6291306635732892] 
	\draw  [fill={rgb, 255:red, 0; green, 0; blue, 0 }  ,fill opacity=1 ] (168.3,180.8) -- (168.3,144.8) (138.3,136.8) -- (198.3,136.8) (168.3,136.8) -- (168.3,100.8) (153.3,148) -- (153.3,144.8) -- (183.3,144.8) -- (183.3,148) -- (153.3,148) -- cycle ;
	%Straight Lines [id:da46883367559070654] 
	\draw    (168.3,70.4) -- (312.8,70.4) ;
	%Straight Lines [id:da743496792064678] 
	\draw    (312.8,99.2) -- (312.8,70.4) ;
	%Straight Lines [id:da1990389962662249] 
	\draw    (312.8,213.2) -- (312.8,189.4) ;
	%Straight Lines [id:da11413330658688636] 
	\draw    (168.3,213.2) -- (312.8,213.2) ;
	%Straight Lines [id:da38694099187961495] 
	\draw    (168.3,213.2) -- (168.3,180.8) ;
	%Curve Lines [id:da2620092915211867] 
	\draw    (138.2,172) .. controls (126.43,157.33) and (122.85,132.98) .. (138.43,114.04) ;
	\draw [shift={(140.2,112)}, rotate = 492.51] [fill={rgb, 255:red, 0; green, 0; blue, 0 }  ][line width=0.08]  [draw opacity=0] (10.72,-5.15) -- (0,0) -- (10.72,5.15) -- (7.12,0) -- cycle    ;
	%Shape: Can [id:dp7370813713893971] 
	\draw   (339.9,99.33) -- (339.9,181.37) .. controls (339.9,185.86) and (327.77,189.5) .. (312.8,189.5) .. controls (297.83,189.5) and (285.7,185.86) .. (285.7,181.37) -- (285.7,99.33) .. controls (285.7,94.84) and (297.83,91.2) .. (312.8,91.2) .. controls (327.77,91.2) and (339.9,94.84) .. (339.9,99.33) .. controls (339.9,103.82) and (327.77,107.46) .. (312.8,107.46) .. controls (297.83,107.46) and (285.7,103.82) .. (285.7,99.33) ;
	%Straight Lines [id:da5695340545261522] 
	\draw    (168.3,100.8) -- (168.3,70.4) ;

	% Text Node
	\draw (106.8,140.6) node    {$\Delta V$};


	\end{tikzpicture}
\end{figure}
\FloatBarrier
Dalla legge di Ohm deriva che le dimensioni della resistenza sono quelle di una differenza di potenziale divisa una corrente. Vista l'importanza della grandezza, ribattezziamo questa unità di misura come $\Omega$. Invece di ragionare in termini di resistenza, si può ragionare in termini di inverso di resistenza: $ G=1/R $. Tale grandezza viene chiamata \textbf{conduttanza}. La sua unità di misura sarebbe $\Omega^{-1}$, anche se viene ribattezzata \emph{Siemens} (o anche \emph{mho}).
\begin{figure}[htpb]
	\centering
	

	\tikzset{every picture/.style={line width=0.75pt}} %set default line width to 0.75pt        

	\begin{tikzpicture}[x=0.75pt,y=0.75pt,yscale=-1,xscale=1]
	%uncomment if require: \path (0,300); %set diagram left start at 0, and has height of 300

	%Shape: Battery [id:dp22509077003774736] 
	\draw  [fill={rgb, 255:red, 0; green, 0; blue, 0 }  ,fill opacity=1 ] (188.3,171.95) -- (188.3,135.95) (158.3,127.95) -- (218.3,127.95) (188.3,127.95) -- (188.3,91.95) (173.3,139.15) -- (173.3,135.95) -- (203.3,135.95) -- (203.3,139.15) -- (173.3,139.15) -- cycle ;
	%Straight Lines [id:da7219445667701057] 
	\draw    (188.3,61.55) -- (332.8,61.55) ;
	%Straight Lines [id:da19489332319438146] 
	\draw    (332.8,90.35) -- (332.8,61.55) ;
	%Straight Lines [id:da7739814888159238] 
	\draw    (332.8,249.25) -- (332.8,234.75) ;
	%Straight Lines [id:da34455176440645685] 
	\draw    (188.3,249.25) -- (332.8,249.25) ;
	%Straight Lines [id:da19332381310739422] 
	\draw    (188.3,249.25) -- (188.3,171.95) ;
	%Curve Lines [id:da7997142849254137] 
	\draw    (156.53,163) .. controls (144.76,148.33) and (141.18,123.98) .. (156.76,105.04) ;
	\draw [shift={(158.53,103)}, rotate = 492.51] [fill={rgb, 255:red, 0; green, 0; blue, 0 }  ][line width=0.08]  [draw opacity=0] (10.72,-5.15) -- (0,0) -- (10.72,5.15) -- (7.12,0) -- cycle    ;
	%Shape: Can [id:dp14099858415964261] 
	\draw   (385,91.61) -- (385,219.09) .. controls (385,227.74) and (361.63,234.75) .. (332.8,234.75) .. controls (303.97,234.75) and (280.6,227.74) .. (280.6,219.09) -- (280.6,91.61) .. controls (280.6,82.96) and (303.97,75.95) .. (332.8,75.95) .. controls (361.63,75.95) and (385,82.96) .. (385,91.61) .. controls (385,100.26) and (361.63,107.27) .. (332.8,107.27) .. controls (303.97,107.27) and (280.6,100.26) .. (280.6,91.61) ;
	%Straight Lines [id:da26449592927404764] 
	\draw    (188.3,91.95) -- (188.3,61.55) ;
	%Straight Lines [id:da7576221257386357] 
	\draw    (308,151.25) -- (308,208.5) ;
	\draw [shift={(308,211.5)}, rotate = 270] [fill={rgb, 255:red, 0; green, 0; blue, 0 }  ][line width=0.08]  [draw opacity=0] (10.72,-5.15) -- (0,0) -- (10.72,5.15) -- (7.12,0) -- cycle    ;
	%Shape: Circle [id:dp6264100912992732] 
	\draw  [fill={rgb, 255:red, 0; green, 0; blue, 0 }  ,fill opacity=1 ] (330.05,61.55) .. controls (330.05,60.03) and (331.28,58.8) .. (332.8,58.8) .. controls (334.32,58.8) and (335.55,60.03) .. (335.55,61.55) .. controls (335.55,63.07) and (334.32,64.3) .. (332.8,64.3) .. controls (331.28,64.3) and (330.05,63.07) .. (330.05,61.55) -- cycle ;
	%Straight Lines [id:da13078567924346718] 
	\draw    (333,151.25) -- (333,208.5) ;
	\draw [shift={(333,211.5)}, rotate = 270] [fill={rgb, 255:red, 0; green, 0; blue, 0 }  ][line width=0.08]  [draw opacity=0] (10.72,-5.15) -- (0,0) -- (10.72,5.15) -- (7.12,0) -- cycle    ;
	%Straight Lines [id:da8093452580702538] 
	\draw    (358,151.25) -- (358,208.5) ;
	\draw [shift={(358,211.5)}, rotate = 270] [fill={rgb, 255:red, 0; green, 0; blue, 0 }  ][line width=0.08]  [draw opacity=0] (10.72,-5.15) -- (0,0) -- (10.72,5.15) -- (7.12,0) -- cycle    ;
	%Shape: Circle [id:dp24588222328028175] 
	\draw  [fill={rgb, 255:red, 0; green, 0; blue, 0 }  ,fill opacity=1 ] (330.05,249.25) .. controls (330.05,247.73) and (331.28,246.5) .. (332.8,246.5) .. controls (334.32,246.5) and (335.55,247.73) .. (335.55,249.25) .. controls (335.55,250.77) and (334.32,252) .. (332.8,252) .. controls (331.28,252) and (330.05,250.77) .. (330.05,249.25) -- cycle ;
	%Shape: Boxed Bezier Curve [id:dp0474784471536458] 
	\draw    (401.16,228.67) .. controls (426.17,191.82) and (433.41,130.14) .. (398.62,83.14) ;
	\draw [shift={(397,81)}, rotate = 412.22] [fill={rgb, 255:red, 0; green, 0; blue, 0 }  ][line width=0.08]  [draw opacity=0] (10.72,-5.15) -- (0,0) -- (10.72,5.15) -- (7.12,0) -- cycle    ;
	%Straight Lines [id:da7317171609485253] 
	\draw    (259.6,219.09) -- (259.6,91.61) ;
	\draw [shift={(259.6,91.61)}, rotate = 450] [color={rgb, 255:red, 0; green, 0; blue, 0 }  ][line width=0.75]    (0,5.59) -- (0,-5.59)   ;
	\draw [shift={(259.6,219.09)}, rotate = 450] [color={rgb, 255:red, 0; green, 0; blue, 0 }  ][line width=0.75]    (0,5.59) -- (0,-5.59)   ;

	% Text Node
	\draw (126.8,129.93) node    {$\Delta V$};
	% Text Node
	\draw (309,130) node    {$\vec{J}$};
	% Text Node
	\draw (347.4,57.3) node    {$A$};
	% Text Node
	\draw (335,134) node    {$I$};
	% Text Node
	\draw (359,130) node    {$\vec{E}$};
	% Text Node
	\draw (348.9,249.3) node    {$B$};
	% Text Node
	\draw (440.13,154.6) node    {$\Delta V$};
	% Text Node
	\draw (254.4,153.3) node    {$l$};


	\end{tikzpicture}
\end{figure}
\FloatBarrier
Consideriamo ora un regime di conduzione stazionaria e un conduttore omogeneo:
\begin{equation}
	\begin{aligned}
		\Delta V &= \int_A^B \vec{E} \cdot d\vec{l} = El \\
		RI       &=El \\
		\rho \frac{l}{S} \cdot I &= El \\
		\rho \frac{l}{S} \cdot JS &= El \\
		\rho J &= E \implies \boxed{\vec{E} = \rho \vec{J}} \qquad \text{oppure} \qquad \boxed{\vec{J} = \sigma\vec{E}}
	\end{aligned}
\end{equation}
Le ultime due relazioni sono note come legge di Ohm in forma locale.

Per un conduttore non-ohmico, si ha $ \vec{J} = f(\vec{E} ) $ ossia i due vettori non sono in relazione lineare tra loro.

Nel caso di conduttori non isotropi, introducendo $ \sigma $ come tensore conducibilità si ha $J_i= \Sigma_k\sigma_{ik}E_k$.







































\section{Modello di Drude-Lorentz della conduzione}

Agli inizi nel 900 fu introdotto un modello microscopico che spiega il legame fra $J$ ed $E$ in base alla dinamica microscopica dei portatori di carica. Tale modello fu proposto da Drude e successivamente sviluppato da Lorentz. Si suppone che gli ioni del reticolo cristallino siano fissi e che gli elettroni si muovano attraverso il reticolo in modo completamente disordinato.
\begin{figure}[htpb]
	\centering
	

	\tikzset{every picture/.style={line width=0.75pt}} %set default line width to 0.75pt        

	\begin{tikzpicture}[x=0.75pt,y=0.75pt,yscale=-1,xscale=1]
	%uncomment if require: \path (0,300); %set diagram left start at 0, and has height of 300

	%Shape: Circle [id:dp6561318663921207] 
	\draw  [fill={rgb, 255:red, 222; green, 222; blue, 222 }  ,fill opacity=1 ] (226.27,82.6) .. controls (226.27,78.46) and (229.62,75.1) .. (233.77,75.1) .. controls (237.91,75.1) and (241.27,78.46) .. (241.27,82.6) .. controls (241.27,86.74) and (237.91,90.1) .. (233.77,90.1) .. controls (229.62,90.1) and (226.27,86.74) .. (226.27,82.6) -- cycle ;

	%Straight Lines [id:da4580037288020633] 
	\draw    (147,120) -- (189.5,166) ;
	%Straight Lines [id:da29394536279820804] 
	\draw    (189.5,166) -- (178.5,81) ;
	\draw [shift={(184,123.5)}, rotate = 442.63] [fill={rgb, 255:red, 0; green, 0; blue, 0 }  ][line width=0.08]  [draw opacity=0] (10.72,-5.15) -- (0,0) -- (10.72,5.15) -- (7.12,0) -- cycle    ;
	%Straight Lines [id:da3085778409379849] 
	\draw    (178.5,81) -- (153.5,191) ;
	%Straight Lines [id:da30789847434685935] 
	\draw    (153.5,191) -- (239.5,146) ;
	\draw [shift={(196.5,168.5)}, rotate = 512.38] [fill={rgb, 255:red, 0; green, 0; blue, 0 }  ][line width=0.08]  [draw opacity=0] (10.72,-5.15) -- (0,0) -- (10.72,5.15) -- (7.12,0) -- cycle    ;
	%Straight Lines [id:da31112340431688956] 
	\draw    (239.5,146) -- (203.5,204) ;
	%Straight Lines [id:da6816397380460644] 
	\draw    (203.5,204) -- (233.5,90) ;
	%Straight Lines [id:da14431912970181138] 
	\draw    (309,123.2) -- (233.5,90) ;
	\draw [shift={(271.25,106.6)}, rotate = 203.74] [fill={rgb, 255:red, 0; green, 0; blue, 0 }  ][line width=0.08]  [draw opacity=0] (10.72,-5.15) -- (0,0) -- (10.72,5.15) -- (7.12,0) -- cycle    ;
	%Shape: Circle [id:dp922295589505471] 
	\draw  [fill={rgb, 255:red, 222; green, 222; blue, 222 }  ,fill opacity=1 ] (135.27,114.6) .. controls (135.27,110.46) and (138.62,107.1) .. (142.77,107.1) .. controls (146.91,107.1) and (150.27,110.46) .. (150.27,114.6) .. controls (150.27,118.74) and (146.91,122.1) .. (142.77,122.1) .. controls (138.62,122.1) and (135.27,118.74) .. (135.27,114.6) -- cycle ;


	% Text Node
	\draw (233.77,80.1) node    {$+$};
	% Text Node
	\draw (142.77,112.1) node    {$-$};
	% Text Node
	\draw (121.6,112) node    {$-e$};


	\end{tikzpicture}
\end{figure}
\FloatBarrier
Nel loro moto gli elettroni subiscono continue interazioni con gli ioni, che chiamiamo urti: tra un urto e il successivo il moto è libero e la traiettoria rettilinea, cosicché la traiettoria di ciascun elettrone è costituita da una successione di segmenti rettilinei, con direzione e lunghezza variabili. L'insieme delle traiettorie è completamente casuale e non si ha un flusso di carica netto, cioè una corrente, in nessuna direzione. Il modello di D-L suppone che la velocità con cui un elettrone emerge dall'urto sia completamente casuale. Si introducono due quantità:
\begin{itemize}
	\item libero cammino medio del portatore di carica, $l$: è la distanza media che un elettrone percorre fra due urti. Si tratta di un processo molto casuale, ecco perché parliamo di una quantità media.
	\item Tempo medio fra due urti, $\tau$. Tale tempo è legato al cammino medio perché se conosciamo la velocità quadratica media, possiamo legare le due quantità come
	\[
		\tau = \frac{l}{\overline{v}}
	\]
\end{itemize}
La velocità quadratica media dipende dalla temperatura del materiale. Si può dimostrare che:
\[
	\frac{1}{2} m_e \overline{v}^2 = \frac{3}{2} kT
\]
Quando si applica un campo elettrico $E$ ciascun elettrone acquista Un'accelerazione $a=\frac{F}{m}=\frac{-eE}{m}$ opposta al campo elettrico, e i tratti rettilinei tra due urti diventano archi di parabola. Alla distribuzione casuale e isotropa della velocità si sovrappone una velocità $ v_d  $ di deriva. Essendo questa velocità piccola rispetto a quella propria degli elettroni, il tempo medio $ \tau  $ fra due urti non cambia.
\[
	\vec{F} = -e\vec{E} \qquad \vec{a} =\frac{\vec{F}}{m_e} = - \frac{e\vec{E}}{m_e}
\]
Supponiamo che l'elettrone $i$-esimo abbia subito un urto e che ne esca con una certa velocità che chiameremo $v_i^{(-)}$, completamente scorrelata da quella passata. Se indichiamo con $v_i^{(+)}$ la velocità dell'elettrone subito prima dell'urto successivo, abbiamo:
\begin{gather*}
	\vec{v}_i^{(+)}(t) = \vec{v}_i^{(-)} + \vec{a}t = \vec{v}_i^{(-)} - \frac{e\vec{E}}{m_e} t \\
	\vec{v}_i^{(+)}(\tau_i) = \vec{v}_i^{(-)} - \frac{e\vec{E}}{m_e} \tau_i
\end{gather*}
Facendo la media su un grande numero di urti possiamo definire la velocità di deriva come:
\[
	\langle \vec{v}_i^{(+)} (\tau_i) \rangle = \vec{v}_d \qquad \langle \vec{v}_i^{(+)}(\tau_i) \rangle = \langle \vec{v}_i^{(-)} \rangle - \frac{e\vec{E}}{m_e} \langle \tau_i \rangle
\]
La media non cambia il termine contenente il campo elettrico, che è uguale per tutti gli elettroni. Inoltre la $ \langle \vec{v}_i^{(-)} \rangle = 0 $ in quanto dopo ogni urto la distribuzione delle velocità rimane casuale. Pertanto:
\[
	\vec{v}_d  = \underbrace{\langle \vec{v}_i^{(-)} \rangle}_{=0} - \frac{e\vec{E}}{m_e} \underbrace{\langle \tau_i \rangle}_{=\tau} = - \frac{e\vec{E}}{m_e} \tau
\]
Per effetto del campo elettrico $E$ ciascun elettrone nel metallo acquista una velocità $ v_d  $ nella direzione del campo elettrico, che è proporzionale al campo elettrico stesso. In sostanza si ammette che in media l'urto cancelli la direzione preferenziale del moto dovuta all'azione del campo elettrico e che questa venga ristabilita durante il tempo $\tau$. La densità di corrente elettrica che consegue a questo moto ordinato è:
\[
	\vec{J} =nq\vec{v}_d \implies J = n(-e)\left( -\frac{e\vec{E}\tau}{m_e} \right) = \underbrace{\left( \frac{ne^2 \tau}{m_e} \right)}_{\sigma}  \vec{E}
\]
Ritorna il risultato $ \vec{J} = \sigma \vec{E}  $ con $ \sigma = \frac{ne^2 \tau}{m_e} $.
Per avere elevata conducibilità è necessario avere un gran numero di portatori di carica e un tempo medio fra i due urti molto grande, perché questi disturbano il processo di conduzione. Al limite, in un super conduttore, $\tau$ tende ad infinito.
Se aumentiamo la temperatura, aumenta la velocità quadratica media e quindi l'intervallo di tempo fra due urti diminuisce e diminuisce la conduttività. La relazione trovata è nota come legge di Ohm della conduzione elettrica e stabilisce che il rapporto tra la densità di corrente $J$ e il campo elettrico applicato $\vec{E}$ è dato da una grandezza caratteristica del conduttore. Il fatto che la conduttività sia sempre positiva, ribadisce il fatto che la densità di corrente è concorde al campo elettrico indipendentemente dal segno dei portatori di carica.




















\subsection{Conduttore ohmico di forma generica}

Abbiamo visto che nel caso di un conduttore di sezione regolare si ha: $ R = \frac{l}{S}\rho $.
Non sempre abbiamo conduttori dotati di una forma tale per cui la sezione è regolare. Stiamo considerando conduttori ohmici, per cui la legge di Ohm continua a valere, il problema è trovarne la resistenza. In tal caso possiamo immaginare di suddividere il conduttore in tante fettine sottili di spessore $dl$ e sezione $S$, che dipenderà dalla posizione.
\begin{figure}[htpb]
	\centering
	

	% Pattern Info
	 
	\tikzset{
	pattern size/.store in=\mcSize, 
	pattern size = 5pt,
	pattern thickness/.store in=\mcThickness, 
	pattern thickness = 0.3pt,
	pattern radius/.store in=\mcRadius, 
	pattern radius = 1pt}
	\makeatletter
	\pgfutil@ifundefined{pgf@pattern@name@_y3lhcvofb}{
	\pgfdeclarepatternformonly[\mcThickness,\mcSize]{_y3lhcvofb}
	{\pgfqpoint{0pt}{0pt}}
	{\pgfpoint{\mcSize+\mcThickness}{\mcSize+\mcThickness}}
	{\pgfpoint{\mcSize}{\mcSize}}
	{
	\pgfsetcolor{\tikz@pattern@color}
	\pgfsetlinewidth{\mcThickness}
	\pgfpathmoveto{\pgfqpoint{0pt}{0pt}}
	\pgfpathlineto{\pgfpoint{\mcSize+\mcThickness}{\mcSize+\mcThickness}}
	\pgfusepath{stroke}
	}}
	\makeatother
	\tikzset{every picture/.style={line width=0.75pt}} %set default line width to 0.75pt        

	\begin{tikzpicture}[x=0.75pt,y=0.75pt,yscale=-1,xscale=1]
	%uncomment if require: \path (0,300); %set diagram left start at 0, and has height of 300

	%Shape: Ellipse [id:dp6484887179203112] 
	\draw   (104,145) .. controls (104,100.82) and (114.86,65) .. (128.25,65) .. controls (141.64,65) and (152.5,100.82) .. (152.5,145) .. controls (152.5,189.18) and (141.64,225) .. (128.25,225) .. controls (114.86,225) and (104,189.18) .. (104,145) -- cycle ;
	%Curve Lines [id:da8157123722318489] 
	\draw    (128.25,65) .. controls (195.5,65) and (225.5,163) .. (335.5,110) ;
	%Curve Lines [id:da5839389048384083] 
	\draw    (128.25,225) .. controls (195.5,225) and (225.5,130.3) .. (335.5,181.51) ;
	%Shape: Ellipse [id:dp8581699542787002] 
	\draw  [pattern=_y3lhcvofb,pattern size=6pt,pattern thickness=0.75pt,pattern radius=0pt, pattern color={rgb, 255:red, 222; green, 222; blue, 222}] (208.4,146) .. controls (208.4,126.23) and (213.26,110.2) .. (219.25,110.2) .. controls (225.24,110.2) and (230.1,126.23) .. (230.1,146) .. controls (230.1,165.77) and (225.24,181.8) .. (219.25,181.8) .. controls (213.26,181.8) and (208.4,165.77) .. (208.4,146) -- cycle ;
	%Curve Lines [id:da06702468969463737] 
	\draw    (335.5,110) .. controls (349.4,110) and (349.4,180.8) .. (335.5,181.51) ;
	%Curve Lines [id:da7199387975350722] 
	\draw  [dash pattern={on 0.84pt off 2.51pt}]  (335.5,181.51) .. controls (323.8,181.6) and (323,110.8) .. (335.5,110) ;
	%Curve Lines [id:da3014728401967599] 
	\draw    (211.87,198.2) .. controls (219.44,191.99) and (231.42,185.96) .. (242.68,181.78) ;
	\draw [shift={(245.4,180.8)}, rotate = 520.79] [fill={rgb, 255:red, 0; green, 0; blue, 0 }  ][line width=0.08]  [draw opacity=0] (10.72,-5.15) -- (0,0) -- (10.72,5.15) -- (7.12,0) -- cycle    ;
	\draw [shift={(209.6,200.2)}, rotate = 316.51] [fill={rgb, 255:red, 0; green, 0; blue, 0 }  ][line width=0.08]  [draw opacity=0] (10.72,-5.15) -- (0,0) -- (10.72,5.15) -- (7.12,0) -- cycle    ;
	%Curve Lines [id:da1333721569231685] 
	\draw    (138.98,256.95) .. controls (198.67,275.37) and (288.44,265.66) .. (346.85,210.06) ;
	\draw [shift={(135.36,255.78)}, rotate = 18.51] [fill={rgb, 255:red, 0; green, 0; blue, 0 }  ][line width=0.08]  [draw opacity=0] (10.72,-5.15) -- (0,0) -- (10.72,5.15) -- (7.12,0) -- cycle    ;
	%Straight Lines [id:da5495637004127791] 
	\draw    (47.5,145) -- (125.25,145) ;
	\draw [shift={(128.25,145)}, rotate = 180] [fill={rgb, 255:red, 0; green, 0; blue, 0 }  ][line width=0.08]  [draw opacity=0] (10.72,-5.15) -- (0,0) -- (10.72,5.15) -- (7.12,0) -- cycle    ;

	% Text Node
	\draw (230.4,202.7) node    {$dl$};
	% Text Node
	\draw (126.8,237.5) node    {$A$};
	% Text Node
	\draw (336,192.7) node    {$B$};
	% Text Node
	\draw (307.13,252.6) node    {$\Delta V$};
	% Text Node
	\draw (61.4,131.7) node    {$I$};
	% Text Node
	\draw (251.4,145.7) node    {$S( P)$};


	\end{tikzpicture}
\end{figure}
\FloatBarrier
Supporremo che la conduzione avvenga in regime stazionario, e che quindi la corrente sia la stessa attraverso qualsiasi sezione del conduttore.
\begin{align*}
	\Delta V &= \int_A^B \vec{E} \cdot d\vec{l} =\int_A^B \rho \vec{J} \cdot d\vec{l} =  \\
	&= \int_A^B \rho Jdl = \int_A^B \rho \frac{I}{S}dl= \\
	&= I \underbrace{\int_A^B \frac{\rho dl}{S}}_R = I \cdot R
\end{align*}
Allora
\[
	\boxed{R = \int_A^B \frac{\rho dl}{S}}
\]







































\section{Resistori in serie o in parallelo}

Conduttori ohmici caratterizzati da un determinato valore della resistenza sono elementi molto usati nei circuiti elettrici: essi vengono chiamati resistori. Più resistori possono essere collegati tra loro, tipicamente da fili o piattine metallici, la cui resistenza è di norma completamente trascurabile. I collegamenti di base, similmente a quanto già visto per i condensatori, sono in serie e in parallelo.




















\subsection{Resistori in serie}

Due resistori sono collegati in serie quando hanno un estremo in comune: in regime stazionario l'intensità di corrente che li attraversa è la stessa.
\begin{figure}[htpb]
	\centering
	

	\tikzset{every picture/.style={line width=0.75pt}} %set default line width to 0.75pt        

	\begin{tikzpicture}[x=0.75pt,y=0.75pt,yscale=-0.8,xscale=0.8]
	%uncomment if require: \path (0,300); %set diagram left start at 0, and has height of 300

	%Shape: Resistor [id:dp7046608827348249] 
	\draw   (59,103) -- (73.4,103) -- (76.6,83) -- (83,123) -- (89.4,83) -- (95.8,123) -- (102.2,83) -- (108.6,123) -- (115,83) -- (121.4,123) -- (124.6,103) -- (139,103) ;
	%Straight Lines [id:da221312075114799] 
	\draw    (339,103) -- (419,103) ;
	%Shape: Resistor [id:dp4547309195494975] 
	\draw   (139,103) -- (153.4,103) -- (156.6,83) -- (163,123) -- (169.4,83) -- (175.8,123) -- (182.2,83) -- (188.6,123) -- (195,83) -- (201.4,123) -- (204.6,103) -- (219,103) ;
	%Shape: Resistor [id:dp5319707454231064] 
	\draw   (219,103) -- (233.4,103) -- (236.6,83) -- (243,123) -- (249.4,83) -- (255.8,123) -- (262.2,83) -- (268.6,123) -- (275,83) -- (281.4,123) -- (284.6,103) -- (299,103) ;
	%Curve Lines [id:da4202906418080372] 
	\draw    (74.24,133.66) .. controls (128.61,164.09) and (218.39,173.69) .. (287.33,132.3) ;
	\draw [shift={(70.95,131.77)}, rotate = 30.58] [fill={rgb, 255:red, 0; green, 0; blue, 0 }  ][line width=0.08]  [draw opacity=0] (10.72,-5.15) -- (0,0) -- (10.72,5.15) -- (7.12,0) -- cycle    ;
	%Shape: Resistor [id:dp10924022597816418] 
	\draw   (419,103) -- (433.4,103) -- (436.6,83) -- (443,123) -- (449.4,83) -- (455.8,123) -- (462.2,83) -- (468.6,123) -- (475,83) -- (481.4,123) -- (484.6,103) -- (499,103) ;
	%Curve Lines [id:da3418165600857359] 
	\draw    (354.24,133.66) .. controls (408.61,164.09) and (498.39,173.69) .. (567.33,132.3) ;
	\draw [shift={(350.95,131.77)}, rotate = 30.58] [fill={rgb, 255:red, 0; green, 0; blue, 0 }  ][line width=0.08]  [draw opacity=0] (10.72,-5.15) -- (0,0) -- (10.72,5.15) -- (7.12,0) -- cycle    ;
	%Straight Lines [id:da6514510472385655] 
	\draw    (499,103) -- (579,103) ;

	% Text Node
	\draw (185.13,173.6) node    {$\Delta V$};
	% Text Node
	\draw (465.13,173.6) node    {$\Delta V$};
	% Text Node
	\draw (100.13,60.6) node    {$R_{1}$};
	% Text Node
	\draw (180.13,60.6) node    {$R_{2}$};
	% Text Node
	\draw (260.13,60.6) node    {$R_{3}$};
	% Text Node
	\draw (460.13,60.6) node    {$R_{eq}$};


	\end{tikzpicture}
\end{figure}
\FloatBarrier
Applicando la legge di Ohm ai resistori si dimostra che $N$ resistori in serie si comportano come un unico resistore di resistenza equivalente:
\[
	\boxed{R_{eq} = \sum_i^N R_i}
\]




















\subsection{Resistori in parallelo}

Due resistori si dicono in parallelo quando sono collegati tra loro entrambi gli estremi. In questo caso l'elemento in comune fra i resistori è la differenza di potenziale e quindi, in base alla legge di Ohm, essi sono attraversati da due correnti diverse se i valori delle resistenze sono diverse. Si può dimostrare che $N$ resistori in parallelo si comportano come un unico resistore equivalente di resistenza tale per cui
\[
	\boxed{\frac{1}{R_{eq}} = \sum_i^N \frac{1}{R_i}}
\]







































\section{Proprietà generali dei conduttori ohmici in regime stazionario}

Nel caso di conduttori non perfetti, che presentano quindi delle molecole polarizzabili, oltre alla resistività, possiamo attribuire al conduttore anche una costante dielettrica relativa. Avremo allora:
\begin{gather*}
	\text{div}\vec{J} = 0 \qquad \frac{\partial \rho_l}{\partial t} =0\\
	\vec{J} = \sigma \vec{E} \qquad \vec{E} = \rho \vec{J} \qquad \text{div}(\rho \vec{J} ) = \frac{\rho_l}{\varepsilon}
\end{gather*}
Consideriamo due conduttori in serie collegati come in figura, con resistività differenti.
\begin{figure}[htpb]
	\centering
	

	% Pattern Info
	 
	\tikzset{
	pattern size/.store in=\mcSize, 
	pattern size = 5pt,
	pattern thickness/.store in=\mcThickness, 
	pattern thickness = 0.3pt,
	pattern radius/.store in=\mcRadius, 
	pattern radius = 1pt}
	\makeatletter
	\pgfutil@ifundefined{pgf@pattern@name@_94gxm1s4l}{
	\pgfdeclarepatternformonly[\mcThickness,\mcSize]{_94gxm1s4l}
	{\pgfqpoint{0pt}{0pt}}
	{\pgfpoint{\mcSize+\mcThickness}{\mcSize+\mcThickness}}
	{\pgfpoint{\mcSize}{\mcSize}}
	{
	\pgfsetcolor{\tikz@pattern@color}
	\pgfsetlinewidth{\mcThickness}
	\pgfpathmoveto{\pgfqpoint{0pt}{0pt}}
	\pgfpathlineto{\pgfpoint{\mcSize+\mcThickness}{\mcSize+\mcThickness}}
	\pgfusepath{stroke}
	}}
	\makeatother
	\tikzset{every picture/.style={line width=0.75pt}} %set default line width to 0.75pt        

	\begin{tikzpicture}[x=0.75pt,y=0.75pt,yscale=-0.9,xscale=0.9]
	%uncomment if require: \path (0,300); %set diagram left start at 0, and has height of 300

	%Straight Lines [id:da5492661277472597] 
	\draw    (36.33,145.5) -- (111.22,145.5) ;
	\draw [shift={(114.22,145.5)}, rotate = 180] [fill={rgb, 255:red, 0; green, 0; blue, 0 }  ][line width=0.08]  [draw opacity=0] (10.72,-5.15) -- (0,0) -- (10.72,5.15) -- (7.12,0) -- cycle    ;
	%Straight Lines [id:da3932330408325475] 
	\draw    (180.07,140) -- (243.5,140) ;
	\draw [shift={(246.5,140)}, rotate = 180] [fill={rgb, 255:red, 0; green, 0; blue, 0 }  ][line width=0.08]  [draw opacity=0] (10.72,-5.15) -- (0,0) -- (10.72,5.15) -- (7.12,0) -- cycle    ;
	%Shape: Can [id:dp33923703592669496] 
	\draw   (109.45,84) -- (507.05,84) .. controls (517.24,84) and (525.5,111.53) .. (525.5,145.5) .. controls (525.5,179.47) and (517.24,207) .. (507.05,207) -- (109.45,207) .. controls (99.26,207) and (91,179.47) .. (91,145.5) .. controls (91,111.53) and (99.26,84) .. (109.45,84) .. controls (119.64,84) and (127.9,111.53) .. (127.9,145.5) .. controls (127.9,179.47) and (119.64,207) .. (109.45,207) ;
	%Curve Lines [id:da0015840017294836972] 
	\draw  [dash pattern={on 0.84pt off 2.51pt}]  (510.05,84) .. controls (485.1,84.75) and (484.6,206.25) .. (510.05,207) ;
	%Shape: Ellipse [id:dp9413302304121538] 
	\draw  [pattern=_94gxm1s4l,pattern size=6pt,pattern thickness=0.75pt,pattern radius=0pt, pattern color={rgb, 255:red, 222; green, 222; blue, 222}] (309.45,84) .. controls (319.64,84) and (327.9,111.46) .. (327.9,145.33) .. controls (327.9,179.21) and (319.64,206.67) .. (309.45,206.67) .. controls (299.26,206.67) and (291,179.21) .. (291,145.33) .. controls (291,111.46) and (299.26,84) .. (309.45,84) -- cycle ;
	%Straight Lines [id:da9333169066315126] 
	\draw    (180.07,176) -- (243.5,176) ;
	\draw [shift={(246.5,176)}, rotate = 180] [fill={rgb, 255:red, 0; green, 0; blue, 0 }  ][line width=0.08]  [draw opacity=0] (10.72,-5.15) -- (0,0) -- (10.72,5.15) -- (7.12,0) -- cycle    ;
	%Straight Lines [id:da891234474697852] 
	\draw    (380.07,140) -- (443.5,140) ;
	\draw [shift={(446.5,140)}, rotate = 180] [fill={rgb, 255:red, 0; green, 0; blue, 0 }  ][line width=0.08]  [draw opacity=0] (10.72,-5.15) -- (0,0) -- (10.72,5.15) -- (7.12,0) -- cycle    ;
	%Straight Lines [id:da7377291792229808] 
	\draw    (380.07,176) -- (443.5,176) ;
	\draw [shift={(446.5,176)}, rotate = 180] [fill={rgb, 255:red, 0; green, 0; blue, 0 }  ][line width=0.08]  [draw opacity=0] (10.72,-5.15) -- (0,0) -- (10.72,5.15) -- (7.12,0) -- cycle    ;

	% Text Node
	\draw (165,138) node    {$\vec{J}$};
	% Text Node
	\draw (166.67,96) node    {$\rho _{1} ,\varepsilon _{r1}$};
	% Text Node
	\draw (366.67,96) node    {$\rho _{2} ,\varepsilon _{r2}$};
	% Text Node
	\draw (309.67,69.67) node    {$\Sigma $};
	% Text Node
	\draw (109,120.33) node    {$S$};
	% Text Node
	\draw (59,133.67) node    {$I$};
	% Text Node
	\draw (165,174) node    {$\vec{E}_{1}$};
	% Text Node
	\draw (365,138) node    {$\vec{J}$};
	% Text Node
	\draw (365,174) node    {$\vec{E}_{2}$};


	\end{tikzpicture}
\end{figure}
\FloatBarrier
Possiamo assumere che il vettore densità di corrente sia lo stesso.
Siccome:
\[
	\rho \underbrace{\text{div}\vec{J}}_{=0} = \frac{\rho_l}{\varepsilon} \implies \rho_l = 0
\]
Essendo le resistività diverse avremo:
\begin{gather*}
	\vec{E}_1 = \rho_1 \vec{J} \qquad \vec{E}_2 = \rho_2 \vec{J} \\\\
	E_1 \neq E_2 \quad \text{se} \quad \rho_1 \neq \rho_2
\end{gather*}
Vi sarà quindi una discontinuità del campo elettrico. Applichiamo le condizioni al contorno per il vettore $\vec{D}$ per vedere se accade qualcosa di interessante su quella superficie.
\begin{align*}
	D_{n2}-D_{n1}=D_2-D_1&=\sigma_l \\
	\varepsilon_0 (\varepsilon_{r2}E_2 - \varepsilon_{r1} E_1 ) &= \sigma_l \tag*{$ \vec{E} = \rho \vec{J}  $} \\
	\varepsilon_0 J(\varepsilon_{r2}\rho_2 - \varepsilon_{r1} \rho_1 ) &= \sigma_l
\end{align*}
Ci sarà della carica libera che si accumula su questa $ \Sigma  $ di separazione. Nel caso di conduttori con $\rho$ variabile non è più vero. In generale si accumula della carica elettrica anche sulla superficie laterale dei conduttori. Se consideriamo una regione di spazio vuota e un conduttore, il campo elettrico a seconda dei due casi avrà l'andamento:
\begin{figure}[htpb]
	\centering
	

	% Pattern Info
	 
	\tikzset{
	pattern size/.store in=\mcSize, 
	pattern size = 5pt,
	pattern thickness/.store in=\mcThickness, 
	pattern thickness = 0.3pt,
	pattern radius/.store in=\mcRadius, 
	pattern radius = 1pt}
	\makeatletter
	\pgfutil@ifundefined{pgf@pattern@name@_o7lu03xiy}{
	\pgfdeclarepatternformonly[\mcThickness,\mcSize]{_o7lu03xiy}
	{\pgfqpoint{0pt}{0pt}}
	{\pgfpoint{\mcSize+\mcThickness}{\mcSize+\mcThickness}}
	{\pgfpoint{\mcSize}{\mcSize}}
	{
	\pgfsetcolor{\tikz@pattern@color}
	\pgfsetlinewidth{\mcThickness}
	\pgfpathmoveto{\pgfqpoint{0pt}{0pt}}
	\pgfpathlineto{\pgfpoint{\mcSize+\mcThickness}{\mcSize+\mcThickness}}
	\pgfusepath{stroke}
	}}
	\makeatother

	% Pattern Info
	 
	\tikzset{
	pattern size/.store in=\mcSize, 
	pattern size = 5pt,
	pattern thickness/.store in=\mcThickness, 
	pattern thickness = 0.3pt,
	pattern radius/.store in=\mcRadius, 
	pattern radius = 1pt}
	\makeatletter
	\pgfutil@ifundefined{pgf@pattern@name@_3pn64mvo8}{
	\pgfdeclarepatternformonly[\mcThickness,\mcSize]{_3pn64mvo8}
	{\pgfqpoint{0pt}{0pt}}
	{\pgfpoint{\mcSize+\mcThickness}{\mcSize+\mcThickness}}
	{\pgfpoint{\mcSize}{\mcSize}}
	{
	\pgfsetcolor{\tikz@pattern@color}
	\pgfsetlinewidth{\mcThickness}
	\pgfpathmoveto{\pgfqpoint{0pt}{0pt}}
	\pgfpathlineto{\pgfpoint{\mcSize+\mcThickness}{\mcSize+\mcThickness}}
	\pgfusepath{stroke}
	}}
	\makeatother

	% Pattern Info
	 
	\tikzset{
	pattern size/.store in=\mcSize, 
	pattern size = 5pt,
	pattern thickness/.store in=\mcThickness, 
	pattern thickness = 0.3pt,
	pattern radius/.store in=\mcRadius, 
	pattern radius = 1pt}
	\makeatletter
	\pgfutil@ifundefined{pgf@pattern@name@_ltcbzqzjs}{
	\pgfdeclarepatternformonly[\mcThickness,\mcSize]{_ltcbzqzjs}
	{\pgfqpoint{0pt}{0pt}}
	{\pgfpoint{\mcSize+\mcThickness}{\mcSize+\mcThickness}}
	{\pgfpoint{\mcSize}{\mcSize}}
	{
	\pgfsetcolor{\tikz@pattern@color}
	\pgfsetlinewidth{\mcThickness}
	\pgfpathmoveto{\pgfqpoint{0pt}{0pt}}
	\pgfpathlineto{\pgfpoint{\mcSize+\mcThickness}{\mcSize+\mcThickness}}
	\pgfusepath{stroke}
	}}
	\makeatother

	% Pattern Info
	 
	\tikzset{
	pattern size/.store in=\mcSize, 
	pattern size = 5pt,
	pattern thickness/.store in=\mcThickness, 
	pattern thickness = 0.3pt,
	pattern radius/.store in=\mcRadius, 
	pattern radius = 1pt}
	\makeatletter
	\pgfutil@ifundefined{pgf@pattern@name@_4sedj10oo}{
	\pgfdeclarepatternformonly[\mcThickness,\mcSize]{_4sedj10oo}
	{\pgfqpoint{0pt}{0pt}}
	{\pgfpoint{\mcSize+\mcThickness}{\mcSize+\mcThickness}}
	{\pgfpoint{\mcSize}{\mcSize}}
	{
	\pgfsetcolor{\tikz@pattern@color}
	\pgfsetlinewidth{\mcThickness}
	\pgfpathmoveto{\pgfqpoint{0pt}{0pt}}
	\pgfpathlineto{\pgfpoint{\mcSize+\mcThickness}{\mcSize+\mcThickness}}
	\pgfusepath{stroke}
	}}
	\makeatother
	\tikzset{every picture/.style={line width=0.75pt}} %set default line width to 0.75pt        

	\begin{tikzpicture}[x=0.75pt,y=0.75pt,yscale=-0.9,xscale=0.9]
	%uncomment if require: \path (0,300); %set diagram left start at 0, and has height of 300

	%Shape: Rectangle [id:dp4388537023359036] 
	\draw  [pattern=_o7lu03xiy,pattern size=6pt,pattern thickness=0.75pt,pattern radius=0pt, pattern color={rgb, 255:red, 222; green, 222; blue, 222}] (66,37) -- (74,37) -- (74,183.5) -- (66,183.5) -- cycle ;
	%Shape: Rectangle [id:dp40332462303911787] 
	\draw  [pattern=_3pn64mvo8,pattern size=6pt,pattern thickness=0.75pt,pattern radius=0pt, pattern color={rgb, 255:red, 222; green, 222; blue, 222}] (206,37) -- (214,37) -- (214,183.5) -- (206,183.5) -- cycle ;
	%Straight Lines [id:da7104754448668555] 
	\draw    (74,77) -- (206,77) ;
	\draw [shift={(140,77)}, rotate = 180] [fill={rgb, 255:red, 0; green, 0; blue, 0 }  ][line width=0.08]  [draw opacity=0] (10.72,-5.15) -- (0,0) -- (10.72,5.15) -- (7.12,0) -- cycle    ;
	%Straight Lines [id:da08715799863832752] 
	\draw    (74,97) -- (206,97) ;
	\draw [shift={(140,97)}, rotate = 180] [fill={rgb, 255:red, 0; green, 0; blue, 0 }  ][line width=0.08]  [draw opacity=0] (10.72,-5.15) -- (0,0) -- (10.72,5.15) -- (7.12,0) -- cycle    ;
	%Straight Lines [id:da12968140512915705] 
	\draw    (74,117) -- (206,117) ;
	\draw [shift={(140,117)}, rotate = 180] [fill={rgb, 255:red, 0; green, 0; blue, 0 }  ][line width=0.08]  [draw opacity=0] (10.72,-5.15) -- (0,0) -- (10.72,5.15) -- (7.12,0) -- cycle    ;
	%Straight Lines [id:da6917414277281675] 
	\draw    (74,137) -- (206,137) ;
	\draw [shift={(140,137)}, rotate = 180] [fill={rgb, 255:red, 0; green, 0; blue, 0 }  ][line width=0.08]  [draw opacity=0] (10.72,-5.15) -- (0,0) -- (10.72,5.15) -- (7.12,0) -- cycle    ;
	%Shape: Rectangle [id:dp16580824435663089] 
	\draw  [pattern=_ltcbzqzjs,pattern size=6pt,pattern thickness=0.75pt,pattern radius=0pt, pattern color={rgb, 255:red, 222; green, 222; blue, 222}] (366,37) -- (374,37) -- (374,183.5) -- (366,183.5) -- cycle ;
	%Shape: Rectangle [id:dp10201364970647253] 
	\draw  [pattern=_4sedj10oo,pattern size=6pt,pattern thickness=0.75pt,pattern radius=0pt, pattern color={rgb, 255:red, 222; green, 222; blue, 222}] (506,37) -- (514,37) -- (514,183.5) -- (506,183.5) -- cycle ;
	%Curve Lines [id:da5440374698496118] 
	\draw    (374,77) .. controls (415,35.5) and (462,114.5) .. (506,77) ;
	\draw [shift={(440.22,76.09)}, rotate = 203.7] [fill={rgb, 255:red, 0; green, 0; blue, 0 }  ][line width=0.08]  [draw opacity=0] (10.72,-5.15) -- (0,0) -- (10.72,5.15) -- (7.12,0) -- cycle    ;
	%Curve Lines [id:da5129362441117902] 
	\draw    (374,97) .. controls (415,55.5) and (462,134.5) .. (506,97) ;
	\draw [shift={(440.22,96.09)}, rotate = 203.7] [fill={rgb, 255:red, 0; green, 0; blue, 0 }  ][line width=0.08]  [draw opacity=0] (10.72,-5.15) -- (0,0) -- (10.72,5.15) -- (7.12,0) -- cycle    ;
	%Curve Lines [id:da5329072845060521] 
	\draw    (374,117) .. controls (415,75.5) and (462,154.5) .. (506,117) ;
	\draw [shift={(440.22,116.09)}, rotate = 203.7] [fill={rgb, 255:red, 0; green, 0; blue, 0 }  ][line width=0.08]  [draw opacity=0] (10.72,-5.15) -- (0,0) -- (10.72,5.15) -- (7.12,0) -- cycle    ;
	%Curve Lines [id:da33121188984697136] 
	\draw    (374,137) .. controls (415,95.5) and (462,174.5) .. (506,137) ;
	\draw [shift={(440.22,136.09)}, rotate = 203.7] [fill={rgb, 255:red, 0; green, 0; blue, 0 }  ][line width=0.08]  [draw opacity=0] (10.72,-5.15) -- (0,0) -- (10.72,5.15) -- (7.12,0) -- cycle    ;

	% Text Node
	\draw (53,45) node    {$V_{1}$};
	% Text Node
	\draw (228,48) node    {$V_{2}$};
	% Text Node
	\draw (353,45) node    {$V_{1}$};
	% Text Node
	\draw (528,48) node    {$V_{2}$};
	% Text Node
	\draw (136,54) node    {$\vec{E}$};
	% Text Node
	\draw (454,59) node    {$\vec{E}$};


	\end{tikzpicture}
\end{figure}
\FloatBarrier
Le cariche sulla superficie laterale spiegano tale andamento.







































\section{Aspetti energetici della conduzione di carica elettrica}

Il lavoro compiuto dalla forza elettrica $\vec{F}$ per spostare la carica $q$ di un tratto infinitesimo $dl$ è dato da:
\[
	d\mathcal{L} =\vec{F} \cdot d\vec{l}
\]
\begin{figure}[htpb]
	\centering
	

	\tikzset{every picture/.style={line width=0.75pt}} %set default line width to 0.75pt        

	\begin{tikzpicture}[x=0.75pt,y=0.75pt,yscale=-1,xscale=1]
	%uncomment if require: \path (0,300); %set diagram left start at 0, and has height of 300

	%Shape: Circle [id:dp22141999262004308] 
	\draw  [fill={rgb, 255:red, 0; green, 0; blue, 0 }  ,fill opacity=1 ] (156.05,179.55) .. controls (156.05,178.03) and (157.28,176.8) .. (158.8,176.8) .. controls (160.32,176.8) and (161.55,178.03) .. (161.55,179.55) .. controls (161.55,181.07) and (160.32,182.3) .. (158.8,182.3) .. controls (157.28,182.3) and (156.05,181.07) .. (156.05,179.55) -- cycle ;
	%Straight Lines [id:da597861486650308] 
	\draw    (158.8,179.55) -- (295.88,83.06) ;
	\draw [shift={(298.33,81.33)}, rotate = 504.86] [fill={rgb, 255:red, 0; green, 0; blue, 0 }  ][line width=0.08]  [draw opacity=0] (10.72,-5.15) -- (0,0) -- (10.72,5.15) -- (7.12,0) -- cycle    ;
	%Straight Lines [id:da2655454358408562] 
	\draw    (158.8,181.55) -- (365.33,181.55) ;
	\draw [shift={(368.33,181.55)}, rotate = 180] [fill={rgb, 255:red, 0; green, 0; blue, 0 }  ][line width=0.08]  [draw opacity=0] (10.72,-5.15) -- (0,0) -- (10.72,5.15) -- (7.12,0) -- cycle    ;
	%Straight Lines [id:da9395712963407883] 
	\draw    (158.8,178.55) -- (316.67,178.55) ;
	\draw [shift={(319.67,178.55)}, rotate = 180] [fill={rgb, 255:red, 0; green, 0; blue, 0 }  ][line width=0.08]  [draw opacity=0] (10.72,-5.15) -- (0,0) -- (10.72,5.15) -- (7.12,0) -- cycle    ;

	% Text Node
	\draw (233.4,101.97) node    {$\vec{F}$};
	% Text Node
	\draw (276.73,165.3) node    {$\vec{v}$};
	% Text Node
	\draw (370.07,189.97) node    {$d\vec{l}$};


	\end{tikzpicture}
\end{figure}
\FloatBarrier
Supponiamo che tale spostamento avvenga in un intervallo di tempo $dt$. Chiamiamo il rapporto fra il lavoro compiuto dalla forza e l'intervallo di tempo in cui è stato compiuto \textbf{potenza dissipata dalla forza}. L'unità di misura della grandezza è $J/S$ e viene ribattezzata \emph{watt}. Possiamo esprimere la potenza come:
\[
	W=\frac{d\mathcal{L}}{dt}=\frac{\vec{F} \cdot d\vec{l}}{dt} = \vec{F} \cdot \vec{v}
\]
Consideriamo un conduttore di forma generica e concentriamoci su quello che accade in un istante infinitesimo. Le cariche si spostano entrambe della medesima quantità: chiamiamo dq la quantità di carica che attraversa la d.d.p. Per questo spostamento viene dissipata la potenza:
\[
	\boxed{W=\frac{d\mathcal{L}}{dt} = \frac{dQ\,\Delta V}{dt} = I\cdot \Delta V}
\]
Per i conduttori ohmici, quindi nel caso in cui vale la legge di Ohm, la potenza può anche essere riscritta come:
\[
	W=\frac{\Delta V^2}{R} = RI^2
\]
Questa legge che abbiamo trovato è detta \textbf{legge di Joule}. Essa mette in evidenza come il lavoro speso per far circolare la corrente elettrica sia necessario a vincere la resistenza opposta al reticolo cristallino al moto ordinario degli elettroni e, dal punto di vista termodinamico, esso viene assorbito dal conduttore la cui energia interna aumenta. Di conseguenza aumenta la temperatura del conduttore: se esso è isolato termicamente dall'ambiente circostante il processo porta alla fusione del metallo; se invece il conduttore è in contatto termico con l'ambiente, la sua temperatura cresce fino a che si raggiunge uno stato di equilibrio in cui l'energia interna non varia più e il lavoro elettrico viene ceduto all'ambiente sotto forma di calore (perché naturalmente la temperatura di equilibrio sia inferiore alla temperatura di fusione del conduttore). L'effetto di riscaldamento di un conduttore percorso da corrente si chiama \textbf{effetto Joule}.
I portatori di carica urtano in modo anaelastico con gli ioni del reticolo cristallino, che si scalda. Questo processo è quello che noi chiamiamo effetto Joule.

La relazione integrale $ W=\Delta V\cdot I $ si può anche estendere al caso microscopico.
Immaginiamo di avere un grande oggetto conduttore e di considerare al suo interno un campo elettrico.
Vogliamo studiare quello che accade in un istante infinitesimo. Immaginiamo di prendere una piccola porzione di conduttore. Vogliamo che il suo volume sia $d\tau$.
\begin{figure}[htpb]
	\centering
	

	\tikzset{every picture/.style={line width=0.75pt}} %set default line width to 0.75pt        

	\begin{tikzpicture}[x=0.75pt,y=0.75pt,yscale=-1,xscale=1]
	%uncomment if require: \path (0,300); %set diagram left start at 0, and has height of 300

	%Shape: Can [id:dp757877143522663] 
	\draw   (254.03,131.67) -- (314.63,131.67) .. controls (318.89,131.67) and (322.33,143.16) .. (322.33,157.33) .. controls (322.33,171.51) and (318.89,183) .. (314.63,183) -- (254.03,183) .. controls (249.78,183) and (246.33,171.51) .. (246.33,157.33) .. controls (246.33,143.16) and (249.78,131.67) .. (254.03,131.67) .. controls (258.29,131.67) and (261.73,143.16) .. (261.73,157.33) .. controls (261.73,171.51) and (258.29,183) .. (254.03,183) ;
	%Curve Lines [id:da3467365634870494] 
	\draw  [dash pattern={on 0.84pt off 2.51pt}]  (315.76,131.67) .. controls (305.13,131.98) and (304.92,182.69) .. (315.76,183) ;
	%Shape: Circle [id:dp9534971715270861] 
	\draw  [fill={rgb, 255:red, 222; green, 222; blue, 222 }  ,fill opacity=1 ] (220.27,80.6) .. controls (220.27,76.46) and (223.62,73.1) .. (227.77,73.1) .. controls (231.91,73.1) and (235.27,76.46) .. (235.27,80.6) .. controls (235.27,84.74) and (231.91,88.1) .. (227.77,88.1) .. controls (223.62,88.1) and (220.27,84.74) .. (220.27,80.6) -- cycle ;

	%Straight Lines [id:da8479493413613088] 
	\draw    (235.27,80.6) -- (317.93,80.6) ;
	\draw [shift={(320.93,80.6)}, rotate = 180] [fill={rgb, 255:red, 0; green, 0; blue, 0 }  ][line width=0.08]  [draw opacity=0] (10.72,-5.15) -- (0,0) -- (10.72,5.15) -- (7.12,0) -- cycle    ;
	%Straight Lines [id:da33699563020269463] 
	\draw    (361.93,103.27) -- (444.6,103.27) ;
	\draw [shift={(447.6,103.27)}, rotate = 180] [fill={rgb, 255:red, 0; green, 0; blue, 0 }  ][line width=0.08]  [draw opacity=0] (10.72,-5.15) -- (0,0) -- (10.72,5.15) -- (7.12,0) -- cycle    ;
	%Shape: Rectangle [id:dp5473248941985187] 
	\draw   (175,43.33) -- (468.33,43.33) -- (468.33,201.33) -- (175,201.33) -- cycle ;
	%Straight Lines [id:da8728669115931522] 
	\draw    (314.33,157.33) -- (367,157.33) ;
	\draw [shift={(370,157.33)}, rotate = 180] [fill={rgb, 255:red, 0; green, 0; blue, 0 }  ][line width=0.08]  [draw opacity=0] (10.72,-5.15) -- (0,0) -- (10.72,5.15) -- (7.12,0) -- cycle    ;
	%Shape: Circle [id:dp291256225043645] 
	\draw  [fill={rgb, 255:red, 222; green, 222; blue, 222 }  ,fill opacity=1 ] (267.6,144.6) .. controls (267.6,140.46) and (270.96,137.1) .. (275.1,137.1) .. controls (279.24,137.1) and (282.6,140.46) .. (282.6,144.6) .. controls (282.6,148.74) and (279.24,152.1) .. (275.1,152.1) .. controls (270.96,152.1) and (267.6,148.74) .. (267.6,144.6) -- cycle ;

	%Shape: Circle [id:dp8259031963097847] 
	\draw  [fill={rgb, 255:red, 222; green, 222; blue, 222 }  ,fill opacity=1 ] (289.6,161.93) .. controls (289.6,157.79) and (292.96,154.43) .. (297.1,154.43) .. controls (301.24,154.43) and (304.6,157.79) .. (304.6,161.93) .. controls (304.6,166.08) and (301.24,169.43) .. (297.1,169.43) .. controls (292.96,169.43) and (289.6,166.08) .. (289.6,161.93) -- cycle ;

	%Shape: Circle [id:dp6700087166522872] 
	\draw  [fill={rgb, 255:red, 222; green, 222; blue, 222 }  ,fill opacity=1 ] (269.6,169.27) .. controls (269.6,165.12) and (272.96,161.77) .. (277.1,161.77) .. controls (281.24,161.77) and (284.6,165.12) .. (284.6,169.27) .. controls (284.6,173.41) and (281.24,176.77) .. (277.1,176.77) .. controls (272.96,176.77) and (269.6,173.41) .. (269.6,169.27) -- cycle ;


	% Text Node
	\draw (227.77,78.1) node    {$+$};
	% Text Node
	\draw (211.33,77.33) node    {$q$};
	% Text Node
	\draw (280.67,66.67) node    {$\vec{v}_{d}$};
	% Text Node
	\draw (406.67,88) node    {$\vec{E}$};
	% Text Node
	\draw (232.67,168.67) node    {$d\tau $};
	% Text Node
	\draw (344,143.33) node    {$d\vec{l}$};
	% Text Node
	\draw (275.1,142.1) node    {$+$};
	% Text Node
	\draw (297.1,159.43) node    {$+$};
	% Text Node
	\draw (277.1,166.77) node    {$+$};


	\end{tikzpicture}
\end{figure}
\FloatBarrier
Calcoliamo quanto vale il lavoro delle forze elettriche per spostare nell'istante $dt$ tutte le cariche del volumetto di uno spostamento $dl$.
\begin{align*}
	d\mathcal{L}^{(d\tau )} &= q\vec{E} \cdot d\vec{l} \, n \, d\tau \\
	d\mathcal{L}^{(d\tau )} &= q\vec{E} \underbrace{\vec{v}_ddt}_{d\vec{l}} \, n \, d\tau = \vec{E} \cdot \vec{J} \, dt \, d\tau \\
	dW^{(d\tau )} &= \frac{d\mathcal{L}^{(d\tau )}}{dt}= \frac{\vec{E} \cdot \vec{J} \, dt \, d\tau}{dt} = \vec{E} \cdot \vec{J}  \, d\tau
\end{align*}
E possiamo definire la \textbf{potenza dissipata per unità di volume}
\[
	\boxed{w = \frac{dW^{(d\tau )}}{d\tau} = \vec{E}  \cdot  \vec{J}}
\]
Se nello specifico stiamo considerando un conduttore ohmico avremo
\begin{gather*}
	\vec{J} = \sigma \vec{E}  \implies  w = \sigma E^2 \\
	\vec{E} = \rho \vec{J}  \implies  w = \rho J^2
\end{gather*}







































\section{Forza elettromotrice}

Il regime stazionario corrisponde all'equazione $ \text{div}\vec{J} =0 $.
Quando un campo ha questa proprietà, è detto \textbf{solenoidale}. In questi casi non ci sono sorgenti di linee di flusso. Esse non possono essere altro che delle linee chiuse. Il campo elettrico non è solenoidale perché vi sono sempre da qualche parte delle sorgenti, punti in cui la divergenza non è nulla. Per avere il regime stazionario, dobbiamo avere un percorso conduttivo in cui le linee di flusso descrivono dei percorsi chiusi.
\begin{figure}[htpb]
	\centering
	

	\tikzset{every picture/.style={line width=0.75pt}} %set default line width to 0.75pt        

	\begin{tikzpicture}[x=0.75pt,y=0.75pt,yscale=-1,xscale=1]
	%uncomment if require: \path (0,300); %set diagram left start at 0, and has height of 300

	%Straight Lines [id:da9476843193458615] 
	\draw    (402.5,46) -- (402.5,141) ;
	\draw [shift={(402.5,93.5)}, rotate = 270] [fill={rgb, 255:red, 0; green, 0; blue, 0 }  ][line width=0.08]  [draw opacity=0] (10.72,-5.15) -- (0,0) -- (10.72,5.15) -- (7.12,0) -- cycle    ;
	%Straight Lines [id:da614935104864186] 
	\draw    (402.5,46) -- (147.5,46) ;
	%Straight Lines [id:da07877041729371093] 
	\draw    (402.5,141) -- (147.5,141) ;
	%Straight Lines [id:da5055683085293592] 
	\draw    (147.5,46) -- (147.5,141) ;
	%Straight Lines [id:da5941319138495598] 
	\draw    (428.6,67.77) -- (428.6,107.17) ;
	\draw [shift={(428.6,110.17)}, rotate = 270] [fill={rgb, 255:red, 0; green, 0; blue, 0 }  ][line width=0.08]  [draw opacity=0] (10.72,-5.15) -- (0,0) -- (10.72,5.15) -- (7.12,0) -- cycle    ;

	% Text Node
	\draw (438.67,88.22) node    {$I$};


	\end{tikzpicture}
\end{figure}
\FloatBarrier
Per la legge di Ohm in forma integrale applicata ad un circuito chiuso, sappiamo che:
\[
	\oint_{\gamma} \vec{E} \cdot d\vec{l} = R_t i
\]
Dove $ R_t  $ è la resistenza totale del circuito stesso. Vediamo come il processo di conduzione stazionario sia un processo dissipativo. Il lavoro per condurre la carica lungo il circuito non sarà nullo. Potremmo chiederci quale è la forza che compie questo lavoro, Ogni carica che scorre nel circuito è sottoposto ad una forza elettrica.
Tuttavia, dal momento che E è conservativo, si ha che:
\[
	\mathcal{L} =\oint_{\gamma} \vec{E} \cdot d\vec{l} = \oint_{\gamma} q\vec{E} \cdot d\vec{l} =0
\]
Il campo elettrico non può compiere lavoro su un percorso chiuso con una conseguente impossibilità di sostenere un processo di conduzione stazionario.
La circolazione delle cariche attraverso il circuito è garantita dal generatore di forza elettromotrice, che ha al suo interno delle forze di natura non elettrostatica, non conservative, che possono determinare il moto continuo delle cariche.
\begin{figure}[htpb]
	\centering
	

	\tikzset{every picture/.style={line width=0.75pt}} %set default line width to 0.75pt        

	\begin{tikzpicture}[x=0.75pt,y=0.75pt,yscale=-1,xscale=1]
	%uncomment if require: \path (0,300); %set diagram left start at 0, and has height of 300

	%Straight Lines [id:da9462970824287664] 
	\draw    (474.5,119.33) -- (474.5,238) ;
	%Straight Lines [id:da005797569061894325] 
	\draw    (154.5,119.33) -- (112,119.33) ;
	%Straight Lines [id:da6061661728139769] 
	\draw    (474.5,238) -- (112,238) ;
	%Straight Lines [id:da6122974189521297] 
	\draw    (112,119.33) -- (112,238) ;
	%Shape: Rectangle [id:dp03509844529887096] 
	\draw   (169,43) -- (417.5,43) -- (417.5,195) -- (169,195) -- cycle ;
	%Shape: Rectangle [id:dp2522424612146723] 
	\draw   (169,43) -- (154.5,43) -- (154.5,195) -- (169,195) -- cycle ;
	%Shape: Rectangle [id:dp11615323912801867] 
	\draw   (432,43) -- (417.5,43) -- (417.5,195) -- (432,195) -- cycle ;
	%Straight Lines [id:da33287730947959715] 
	\draw    (474.5,119.33) -- (432,119.33) ;
	%Shape: Circle [id:dp16159178135504249] 
	\draw  [fill={rgb, 255:red, 222; green, 222; blue, 222 }  ,fill opacity=1 ] (338.27,117.93) .. controls (338.27,113.79) and (341.62,110.43) .. (345.77,110.43) .. controls (349.91,110.43) and (353.27,113.79) .. (353.27,117.93) .. controls (353.27,122.08) and (349.91,125.43) .. (345.77,125.43) .. controls (341.62,125.43) and (338.27,122.08) .. (338.27,117.93) -- cycle ;

	%Straight Lines [id:da9308438901155345] 
	\draw    (338.27,117.93) -- (275.33,117.93) ;
	\draw [shift={(272.33,117.93)}, rotate = 360] [fill={rgb, 255:red, 0; green, 0; blue, 0 }  ][line width=0.08]  [draw opacity=0] (10.72,-5.15) -- (0,0) -- (10.72,5.15) -- (7.12,0) -- cycle    ;

	% Text Node
	\draw (161.67,58.33) node    {$+$};
	% Text Node
	\draw (161.67,78.33) node    {$+$};
	% Text Node
	\draw (161.67,98.33) node    {$+$};
	% Text Node
	\draw (161.67,118.33) node    {$+$};
	% Text Node
	\draw (161.67,138.33) node    {$+$};
	% Text Node
	\draw (161.67,158.33) node    {$+$};
	% Text Node
	\draw (161.67,178.33) node    {$+$};
	% Text Node
	\draw (424.67,58.33) node    {$-$};
	% Text Node
	\draw (424.67,78.33) node    {$-$};
	% Text Node
	\draw (424.67,98.33) node    {$-$};
	% Text Node
	\draw (424.67,118.33) node    {$-$};
	% Text Node
	\draw (424.67,138.33) node    {$-$};
	% Text Node
	\draw (424.67,158.33) node    {$-$};
	% Text Node
	\draw (424.67,178.33) node    {$-$};
	% Text Node
	\draw (367.33,110) node    {$q$};
	% Text Node
	\draw (345.77,115.43) node    {$+$};
	% Text Node
	\draw (300.67,103.33) node    {$\vec{F} *$};


	\end{tikzpicture}
\end{figure}
\FloatBarrier
Esso, grazie a queste forze, sposta delle cariche da un lato all'altro accumulando su estremo carica positiva e su un altro carica negativa. Produrrà all'interno del circuito un campo elettrico e quando le cariche arrivano ad un suo estremo, il generatore le porta all'estremo successivo. Per comprendere questo meccanismo, introduciamo un nuovo campo vettoriale nel generatore, detto \textbf{campo elettromotore}, pari al rapporto fra la forza non conservativa che trasferiamo all'interno del conservatore e la carica su cui questa forza agisce. Si tratta sempre di un campo elettrico coulombiano, si misura con la stessa unità di misura. Chiamiamo i due estremi del generatore poli.




















\subsection{Generatore a circuito aperto}

Quando il generatore è staccato dal conduttore esterno, mano a mano che accumulo cariche sui poli, esse producono un campo elettrostatico in aumento, diretto dalle cariche positive verso quelle negative.
\begin{figure}[htpb]
	\centering
	

	\tikzset{every picture/.style={line width=0.75pt}} %set default line width to 0.75pt        

	\begin{tikzpicture}[x=0.75pt,y=0.75pt,yscale=-1,xscale=1]
	%uncomment if require: \path (0,300); %set diagram left start at 0, and has height of 300

	%Shape: Rectangle [id:dp5181084226757073] 
	\draw   (170,75) -- (418.5,75) -- (418.5,227) -- (170,227) -- cycle ;
	%Shape: Rectangle [id:dp08607693266497796] 
	\draw   (170,75) -- (155.5,75) -- (155.5,227) -- (170,227) -- cycle ;
	%Shape: Rectangle [id:dp8799864593554507] 
	\draw   (433,75) -- (418.5,75) -- (418.5,227) -- (433,227) -- cycle ;
	%Shape: Circle [id:dp1343846334943457] 
	\draw  [fill={rgb, 255:red, 222; green, 222; blue, 222 }  ,fill opacity=1 ] (284.27,114.43) .. controls (284.27,110.29) and (287.62,106.93) .. (291.77,106.93) .. controls (295.91,106.93) and (299.27,110.29) .. (299.27,114.43) .. controls (299.27,118.58) and (295.91,121.93) .. (291.77,121.93) .. controls (287.62,121.93) and (284.27,118.58) .. (284.27,114.43) -- cycle ;

	%Straight Lines [id:da9326497794989177] 
	\draw    (284.27,114.43) -- (221.33,114.43) ;
	\draw [shift={(218.33,114.43)}, rotate = 360] [fill={rgb, 255:red, 0; green, 0; blue, 0 }  ][line width=0.08]  [draw opacity=0] (10.72,-5.15) -- (0,0) -- (10.72,5.15) -- (7.12,0) -- cycle    ;
	%Straight Lines [id:da5145606566488408] 
	\draw    (418.5,212) -- (170,212) ;
	\draw [shift={(294.25,212)}, rotate = 180] [fill={rgb, 255:red, 0; green, 0; blue, 0 }  ][line width=0.08]  [draw opacity=0] (10.72,-5.15) -- (0,0) -- (10.72,5.15) -- (7.12,0) -- cycle    ;
	%Straight Lines [id:da05978377752574082] 
	\draw    (362.2,114.43) -- (299.27,114.43) ;
	\draw [shift={(365.2,114.43)}, rotate = 180] [fill={rgb, 255:red, 0; green, 0; blue, 0 }  ][line width=0.08]  [draw opacity=0] (10.72,-5.15) -- (0,0) -- (10.72,5.15) -- (7.12,0) -- cycle    ;
	%Straight Lines [id:da06615514171888282] 
	\draw    (284.27,172.43) -- (221.33,172.43) ;
	\draw [shift={(218.33,172.43)}, rotate = 360] [fill={rgb, 255:red, 0; green, 0; blue, 0 }  ][line width=0.08]  [draw opacity=0] (10.72,-5.15) -- (0,0) -- (10.72,5.15) -- (7.12,0) -- cycle    ;
	%Straight Lines [id:da5274416088289138] 
	\draw    (362.2,172.43) -- (299.27,172.43) ;
	\draw [shift={(365.2,172.43)}, rotate = 180] [fill={rgb, 255:red, 0; green, 0; blue, 0 }  ][line width=0.08]  [draw opacity=0] (10.72,-5.15) -- (0,0) -- (10.72,5.15) -- (7.12,0) -- cycle    ;

	% Text Node
	\draw (162.67,90.33) node    {$+$};
	% Text Node
	\draw (162.67,110.33) node    {$+$};
	% Text Node
	\draw (162.67,130.33) node    {$+$};
	% Text Node
	\draw (162.67,150.33) node    {$+$};
	% Text Node
	\draw (162.67,170.33) node    {$+$};
	% Text Node
	\draw (162.67,190.33) node    {$+$};
	% Text Node
	\draw (162.67,210.33) node    {$+$};
	% Text Node
	\draw (425.67,90.33) node    {$-$};
	% Text Node
	\draw (425.67,110.33) node    {$-$};
	% Text Node
	\draw (425.67,130.33) node    {$-$};
	% Text Node
	\draw (425.67,150.33) node    {$-$};
	% Text Node
	\draw (425.67,170.33) node    {$-$};
	% Text Node
	\draw (425.67,190.33) node    {$-$};
	% Text Node
	\draw (425.67,210.33) node    {$-$};
	% Text Node
	\draw (294.33,91) node    {$q$};
	% Text Node
	\draw (246.67,99.83) node    {$\vec{F} *$};
	% Text Node
	\draw (291.77,111.93) node    {$+$};
	% Text Node
	\draw (195.83,197.5) node    {$\gamma $};
	% Text Node
	\draw (336.67,99.83) node    {$\vec{F}$};
	% Text Node
	\draw (246.67,155.83) node    {$\vec{E}_{e}$};
	% Text Node
	\draw (336.67,155.83) node    {$\vec{E}$};


	\end{tikzpicture}
\end{figure}
\FloatBarrier
A questo punto sulla carica agiscono due forze: la forza non conservativa e quella coulombiana che tende a respingere la carica. In regime stazionario arriviamo a una situazione di equilibrio in cui la risultante delle forze agenti sulla carica diventerà pari a zero. Questa situazione è dovuta al fatto che l'accumulo di carica sui poli impedisce un ulteriore spostamento di carica (per repulsione elettrostatica).
\[
	\vec{R} = \vec{F} + \vec{F}^* = 0 \qquad q(\vec{E} +\vec{E}_e ) = 0\implies \vec{E} = - \vec{E}_e
\]
Consideriamo un percorso $\gamma$ che va dal polo positivo al polo negativo. Calcoliamoci l'integrale di linea lungo il percorso nei due casi:
\begin{gather*}
	\int_{A,\gamma}^B \vec{E} \cdot d\vec{l} = - \int_A^B \vec{E}_e \cdot d\vec{l} \\
	V_A-V_B = \Delta V = \int_{B,\gamma}^A \vec{E}_e\cdot d\vec{l} = f \\
	\boxed{V_A-V_B=fem}
\end{gather*}
Ciò che abbiamo ottenuto è detta forza elettromotrice ($fem$). Non si tratta tuttavia di una forza, ha le dimensioni di un potenziale ed è proprio la d.d.p. misurata ai capi del generatore a circuito aperto.




















\subsection{Generatore a circuito chiuso}

Proviamo a collegare il generatore a un circuito. Esso provoca la circolazione di corrente. Sulla carica all'interno del generatore agiranno entrambi i campi sopra definiti.
\begin{figure}[htpb]
	\centering
	

	\tikzset{every picture/.style={line width=0.75pt}} %set default line width to 0.75pt        

	\begin{tikzpicture}[x=0.75pt,y=0.75pt,yscale=-1,xscale=1]
	%uncomment if require: \path (0,300); %set diagram left start at 0, and has height of 300

	%Straight Lines [id:da6043179485749723] 
	\draw    (494.5,139.33) -- (494.5,258) ;
	%Straight Lines [id:da10221033382838973] 
	\draw    (174.5,139.33) -- (132,139.33) ;
	%Straight Lines [id:da6558814232584502] 
	\draw    (494.5,258) -- (132,258) ;
	%Straight Lines [id:da62131025025839] 
	\draw    (132,139.33) -- (132,258) ;
	%Shape: Rectangle [id:dp0952509793199432] 
	\draw   (189,63) -- (437.5,63) -- (437.5,215) -- (189,215) -- cycle ;
	%Shape: Rectangle [id:dp16008206437810846] 
	\draw   (189,63) -- (174.5,63) -- (174.5,215) -- (189,215) -- cycle ;
	%Shape: Rectangle [id:dp7768738225428726] 
	\draw   (452,63) -- (437.5,63) -- (437.5,215) -- (452,215) -- cycle ;
	%Straight Lines [id:da8208347839397301] 
	\draw    (494.5,139.33) -- (452,139.33) ;
	%Shape: Circle [id:dp33343919986697257] 
	\draw  [fill={rgb, 255:red, 222; green, 222; blue, 222 }  ,fill opacity=1 ] (324.27,114.43) .. controls (324.27,110.29) and (327.62,106.93) .. (331.77,106.93) .. controls (335.91,106.93) and (339.27,110.29) .. (339.27,114.43) .. controls (339.27,118.58) and (335.91,121.93) .. (331.77,121.93) .. controls (327.62,121.93) and (324.27,118.58) .. (324.27,114.43) -- cycle ;

	%Straight Lines [id:da053558009454149236] 
	\draw    (324.27,114.43) -- (235.5,114.43) ;
	\draw [shift={(232.5,114.43)}, rotate = 360] [fill={rgb, 255:red, 0; green, 0; blue, 0 }  ][line width=0.08]  [draw opacity=0] (10.72,-5.15) -- (0,0) -- (10.72,5.15) -- (7.12,0) -- cycle    ;
	%Straight Lines [id:da19431546395832155] 
	\draw    (399.5,114.43) -- (339.27,114.43) ;
	\draw [shift={(402.5,114.43)}, rotate = 180] [fill={rgb, 255:red, 0; green, 0; blue, 0 }  ][line width=0.08]  [draw opacity=0] (10.72,-5.15) -- (0,0) -- (10.72,5.15) -- (7.12,0) -- cycle    ;
	%Straight Lines [id:da6578194814557985] 
	\draw    (355.5,164.43) -- (280.5,164.43) ;
	\draw [shift={(277.5,164.43)}, rotate = 360] [fill={rgb, 255:red, 0; green, 0; blue, 0 }  ][line width=0.08]  [draw opacity=0] (10.72,-5.15) -- (0,0) -- (10.72,5.15) -- (7.12,0) -- cycle    ;

	% Text Node
	\draw (181.67,78.33) node    {$+$};
	% Text Node
	\draw (181.67,98.33) node    {$+$};
	% Text Node
	\draw (181.67,118.33) node    {$+$};
	% Text Node
	\draw (181.67,138.33) node    {$+$};
	% Text Node
	\draw (181.67,158.33) node    {$+$};
	% Text Node
	\draw (181.67,178.33) node    {$+$};
	% Text Node
	\draw (181.67,198.33) node    {$+$};
	% Text Node
	\draw (444.67,78.33) node    {$-$};
	% Text Node
	\draw (444.67,98.33) node    {$-$};
	% Text Node
	\draw (444.67,118.33) node    {$-$};
	% Text Node
	\draw (444.67,138.33) node    {$-$};
	% Text Node
	\draw (444.67,158.33) node    {$-$};
	% Text Node
	\draw (444.67,178.33) node    {$-$};
	% Text Node
	\draw (444.67,198.33) node    {$-$};
	% Text Node
	\draw (334.33,91) node    {$q$};
	% Text Node
	\draw (266.67,99.83) node    {$\vec{F} *$};
	% Text Node
	\draw (376.67,99.83) node    {$-\vec{F}$};
	% Text Node
	\draw (331.77,111.93) node    {$+$};
	% Text Node
	\draw (162.33,125) node    {$A$};
	% Text Node
	\draw (462.33,125) node    {$B$};
	% Text Node
	\draw (220.33,199) node    {$\rho _{g} ,\sigma _{g}$};
	% Text Node
	\draw (175.33,244) node    {$\rho _{c} ,\sigma _{c}$};
	% Text Node
	\draw (370.67,160.83) node    {$\vec{J}$};


	\end{tikzpicture}
\end{figure}
\FloatBarrier
Essi non possono più essere uguali e opposti perché se lo fossero le cariche non si sposterebbero. È necessario che la forza non conservativa vinca sulla forza di repulsione elettrostatica. Il campo elettromotore in modulo deve diventare maggiore del campo elettrico:
\[
	|\vec{E}_e| > |\vec{E}|
\]
Sappiamo che al conduttore viene sempre associata una resistività $ \rho_c  $, tale che:
\[
	\vec{E} = \rho_c\vec{J} \qquad \vec{J} =\sigma_c\vec{E}
\]
Anche il generatore ha una resistività interna caratteristica che possiamo definire come:
\[
	\boxed{(\vec{E} +\vec{E}_e ) = \rho_g\vec{J} \qquad \vec{J} = \sigma_g (\vec{E} + \vec{E}_e )}
\]
Tale relazione prende il nome di \emph{legge di Ohm generalizzata}, perché viene estesa all'interno del generatore.
\begin{align*}
	\underbrace{\int_{A,\gamma}^B \vec{E} \cdot d\vec{l}}_{\Delta V} + \int_{A,\gamma}^B \vec{E}_e\cdot d\vec{l} &= \int_A^B \rho_g \vec{J} \cdot d\vec{l}\\
	\Delta V &= - \int_{A,\gamma}^B \vec{E}_e\cdot d\vec{l} + \int_A^B \rho_g \vec{J} \cdot d\vec{l} \\
	\Delta V &= \underbrace{\int_{B,\gamma}^A \vec{E}_e\cdot d\vec{l}}_{fem} + \underbrace{\int_A^B \rho_g \vec{J} \cdot d\vec{l}}_{<0} \\
\end{align*}
La quantità con nel secondo integrale è pari a:
\[
	\int_A^B \rho_g \vec{J} \cdot d\vec{l} = \int_A^B \rho_g \frac{I}{S} dl = -I \underbrace{\rho_g\frac{l}{S}}_r = -Ir
\]
Si ottiene allora:
\[
	\boxed{\Delta V = fem - rI}
\]
La relazione esprime come parte della capacità di spostare cariche serve per vincere la resistenza, tutto il resto è differenza di potenziale ai lati disponibile.

Dal momento che il generatore oltre alla $fem$ ha anche resistenza interna, si schematizza un generatore reale come un generatore ideale in serie con una resistenza interna. Questo oggetto viene poi collegato al resistore esterno.




























































\chapter{Campo Magnetostatico}

\section{Fenomeni magnetostatici e Campo induzione magnetica B}

La proprietà di attirare la limatura di ferro, mostrata da alcuni materiali di ferro e in particolare dalla magnetite (combinazione di ossidi di ferro), era gia nota nel VII secolo a.C. Il nome magnetite derivò da quello della città greca di Magnesia, in Asia minore, dove si trovavano giacimenti di minerale, e la proprietà osservata prese il nome di magnetismo.
Tale proprieta di attrazione non è uniformemente presente nel materiale. Sia questi oggetti che altri con diversa geometria si indicano con il nome di \emph{magneti} e le parti in cui si localizza la proprietà di attrazione si chiamano i \emph{poli del magnete}.
Nel 1600 Gilbert ipotizza che le forze fra i pianeti siano forze di tipo magnetico e compie una serie di esperienze con magneti, aventi lo scopo di mettere in evidenza le caratteristiche del magnetismo e le differenze con i fenomeni di elettrostatica, da lui stesso studiati. I risultati sullo studio delle interazioni fra poli magnetici sono riassunti nei seguenti punti.
\begin{itemize}
	\item Se ad un magnete sospeso nel centro tramite un filo si avvicina un secondo magnete, si osserva che questo esercita sul primo una certa forza. Come per le forze di natura elettrostatica possiamo interpretare il fatto dicendo che un magnete genera un campo, chiamato \emph{campo magnetico}, e che l'altro magnete risente dell'azione che il campo magnetico esercita nella posizione da esso occupata. Un'analisi sistemata porta a stabilire che la forza di interazione tra i due magneti è attrattiva o repulsiva a seconda dei poli dei magneti che vengono affacciati e che esistono soltanto due specie di poli, positivi e negativi; inoltre si trova che i poli di uno stesso magnete sono sempre di segno opposto.
	\item Se si avvicina a un pezzo di magnetite una bacchetta sottile di ferro, essa acquista la proprietà di attirare la limatura di ferro, principalmente in vicinanza delle estremità: la bacchetta di ferro immersa nel campo magnetico generato dalla magnetite è diventata pertanto un magnete, ovvero \emph{si è magnetizzata}. Posso indurre uno stato di magnetizzazione che prima non c'era con interazione a distanza. Questo fenomeno prende il nome di \emph{magnetizzazione}. Tale bacchetta lunga e sottile viene detta magnete artificiale o calamita e presenta due poli magnetici di segno opposto. Quando un magnete permanente ha una forma allungata, lo possiamo usare come una specie di sonda, si parla di \emph{ago magnetico} e viene utilizzato per sondare lo spazio e le proprietà magnetiche prodotte.
	\item Se sospendiamo con un filo l'ago magnetico sopra definito e lo lasciamo libero di ruotare, osserviamo che esso tende a disporsi approssimativamente \emph{parallelo al meridiano terrestre}; spostato da questa posizione di equilibrio l'ago compie intorno ad essa oscillazioni. L'esperienza quindi mostra l'esistenza di un campo magnetico naturale, il \emph{campo magnetico terrestre} e mette in evidenza un comportamento dell'ago magnetico del tutto analogo a quello di un dipolo elettrico posto in un campo elettrico. L'ago magnetico si comporta come un dipolo magnetico che, lasciato libero, \emph{si orienta nella direzione e verso del campo magnetico esistente nel punto in cui è posto}. Per motivi storici, il polo dell'ago magnetico che si orienta verso il polo nord geografico prende il nome di polo nord magnetico e gli attribuisce segno positivo. L'altro viene detto polo sud magnetico e gli si attribuisce segno negativo. Dato che poli dello stesso tipo si respingono, se il polo nord punta verso il nord vuol dire \emph{sul polo nord terrestre c'è un polo sud magnetico}. Si trova sempre che l'interazione fra poli magnetici dello stesso segno è positiva, mentre quella fra poli di segno discorde è attrattiva.
	\item Lo studio quantitativo della forza magnetica tra i poli di due magneti, svolto da Coulomb, dimostrò in tale caso un andamento inversamente proporzionale al quadrato della distanza, almeno per poli puntiformi, come sono con buona approssimazione quelli agli estremi di sbarre lunghe e sottili. Si potrebbe pertanto enunciare una legge di Coulomb per l'interazione magnetica tra due poli, data dalla forza:
	\[
	|\vec{F}_m | = k_m\frac{|q_{m1}\, q_{m2}  |}{r^2}
	\]
	dove $q_m$ rappresenta la massa marginetica e $k_m$ una costante il cui valore dipende dal mezzo in cui avviene l'interazione.
	\begin{figure}[htpb]
		\centering
		

		\tikzset{every picture/.style={line width=0.75pt}} %set default line width to 0.75pt        

		\begin{tikzpicture}[x=0.75pt,y=0.75pt,yscale=-0.6,xscale=0.6]
		%uncomment if require: \path (0,300); %set diagram left start at 0, and has height of 300

		%Shape: Rectangle [id:dp4379328648322769] 
		\draw   (100,110) -- (260,110) -- (260,150) -- (100,150) -- cycle ;
		%Shape: Rectangle [id:dp6350525414581176] 
		\draw   (260,110) -- (300,110) -- (300,150) -- (260,150) -- cycle ;
		%Shape: Rectangle [id:dp4096511233686948] 
		\draw   (60,110) -- (100,110) -- (100,150) -- (60,150) -- cycle ;
		%Shape: Rectangle [id:dp1736195095098798] 
		\draw   (390,110) -- (550,110) -- (550,150) -- (390,150) -- cycle ;
		%Shape: Rectangle [id:dp6389792905995233] 
		\draw   (550,110) -- (590,110) -- (590,150) -- (550,150) -- cycle ;
		%Shape: Rectangle [id:dp7014855427891589] 
		\draw   (350,110) -- (390,110) -- (390,150) -- (350,150) -- cycle ;

		% Text Node
		\draw (80,130) node    {$S$};
		% Text Node
		\draw (370,130) node    {$S$};
		% Text Node
		\draw (280,130) node    {$N$};
		% Text Node
		\draw (570,130) node    {$N$};


		\end{tikzpicture}
	\end{figure}
	\FloatBarrier
	Sebbene la struttura della formula sia identica a quella della forza fra due cariche elettriche, c'è una differenza fondamentale. \emph{Una carica elettrica, positiva o negativa, puo sempre essere isolata}; ciò è una conseguenza dell'esistenza della carica elementare positiva portata dal protone e della carica elementare negativa dell'elettrone, la possibilità di separazione esiste già a livello elementare. \emph{Non è invece mai stato possibile ottenere un polo magnetico isolato}. I poli magnetici sembrano esistere sempre a coppie di egual valore e segno opposto, si manifestano solamente sotto forma di dipoli magnetici. L'indicazione classica è costituita dall'esperimento della calamita spezzata. Se si taglia a meta la calamita compaiono sempre due poli di segno opposto nella zona del taglio, che precedentemente a questo non mostrava la proprietà di attirare la limatura di ferro.
\end{itemize}
La prima relazione tra fenomeni magnetici ed elettrici fu scoperta da Oersted nel 1811 e successivamente l'argomento venne approfondito soprattutto da Ampere intorno al 1820. La sperimentazione fu resa possibile dall'utilizzo della pila di Volta che, permettendo la produzione di correnti elettriche costanti e intense, aprì il campo dello studio dell'interazione tra circuiti percorsi da corrente e magneti. Oersted mostrò che un ago magnetico, posto in prossimità di un filo percorso da corrente, tende ad assumere una ben definita posizione di equilibrio. Alla luce di quanto visto finora il risultato si interpreta dicendo che il filo percorso da corrente produce un campo magnetico e che l'ago si orienta parallelamente al campo esistente nel punto in cui viene posto. In seguito Ampere dimostrò che anche due fili percorsi da corrente interagiscono e intuì che \emph{le azioni magnetiche non sono altro che la manifestazione dell'interazione tra cariche elettriche in movimento}. Egli nel 1827 pubblicò una teoria ancora oggi valida, affermante che le interazioni elettromagnetiche sono legate a correnti di tipo microscopico. Bisogna pensare che in ogni atomo o in ogni molecola devono esistere delle correnti microscopiche locali, che prendono il nome di correnti molecolari di Ampere o \emph{correnti amperiane}; l'interazione tra un circuito percorso da corrente e un magnete è allora risultato delle interazioni tra gli elettroni liberi in moto nel conduttore e le microcorrenti presenti nel materiale magnetizzato.

\textbf{Osservazione.} L'elettrone è puntiforme, per cui il momento magnetico appare proprio come una proprietà intrinseca, legata al momento angolare intrinseco (spin).







































\section{Teorema di Gauss per il campo magnetico, III Equazione di Maxwell (anche in regime tempovariante)}

Per interpretare le interazioni di tipo magnetico si utilizza un approccio simile a quello usato per i campi elettrici. Considerato un filo percorso da corrente, esso genera attorno a sé una modifica delle proprietà dello spazio rappresentata da un campo vettoriale detto \emph{campo di induzione magnetica} $\vec{B}$.
\begin{figure}[htpb]
	\centering
	

	\tikzset{every picture/.style={line width=0.75pt}} %set default line width to 0.75pt        

	\begin{tikzpicture}[x=0.75pt,y=0.75pt,yscale=-1,xscale=1]
	%uncomment if require: \path (0,300); %set diagram left start at 0, and has height of 300

	%Straight Lines [id:da8308841647454677] 
	\draw    (179.5,228) -- (179.5,78) ;
	\draw [shift={(179.5,75)}, rotate = 450] [fill={rgb, 255:red, 0; green, 0; blue, 0 }  ][line width=0.08]  [draw opacity=0] (10.72,-5.15) -- (0,0) -- (10.72,5.15) -- (7.12,0) -- cycle    ;
	%Shape: Ellipse [id:dp5745765474173075] 
	\draw   (144.5,151) .. controls (144.5,139.95) and (160.17,131) .. (179.5,131) .. controls (198.83,131) and (214.5,139.95) .. (214.5,151) .. controls (214.5,162.05) and (198.83,171) .. (179.5,171) .. controls (160.17,171) and (144.5,162.05) .. (144.5,151) -- cycle ;
	%Shape: Triangle [id:dp08320703214712943] 
	\draw  [fill={rgb, 255:red, 222; green, 222; blue, 222 }  ,fill opacity=1 ] (280.46,157) -- (263.37,195.27) -- (244.63,178.73) -- cycle ;
	%Shape: Triangle [id:dp9261976129323173] 
	\draw   (227.54,217) -- (244.63,178.73) -- (263.37,195.27) -- cycle ;

	%Shape: Ellipse [id:dp20987696581529236] 
	\draw   (126.5,151) .. controls (126.5,134.27) and (150.23,120.71) .. (179.5,120.71) .. controls (208.77,120.71) and (232.5,134.27) .. (232.5,151) .. controls (232.5,167.73) and (208.77,181.29) .. (179.5,181.29) .. controls (150.23,181.29) and (126.5,167.73) .. (126.5,151) -- cycle ;
	%Shape: Ellipse [id:dp11041624129618621] 
	\draw   (106.5,151) .. controls (106.5,127.96) and (139.18,109.29) .. (179.5,109.29) .. controls (219.82,109.29) and (252.5,127.96) .. (252.5,151) .. controls (252.5,174.04) and (219.82,192.71) .. (179.5,192.71) .. controls (139.18,192.71) and (106.5,174.04) .. (106.5,151) -- cycle ;




	\end{tikzpicture}
\end{figure}
\FloatBarrier
Se vediamo che l'ago del magnete esploratore interagisce con il filo, lì è presente un campo magnetico. Per definire il campo magnetico in forma qualitativa si utilizza un ago magnetico. Nel caso dell'interazione fra campi elettrici e dipoli elettrici, quando un dipolo viene immerso in un campo $\vec{E}$ ruota in modo tale da rendere parallelo al campo il suo momento di dipolo. Una situazione analoga si ha con il campo magnetico. Si può studiare l'andamento di $\vec{B}$ con un piccolo ago magnetico: questo si orienta parallelamente a $\vec{B}$ indicandone cosi la direzione, mentre il verso è quello dal polo sud al polo nord dell'ago. Se esso è abbastanza piccolo si può supporre che $\vec{B}$ non vari lungo l'ago e quindi la misura sia effettivamente puntuale e non mediata.

Possiamo osservare che anche la legge di Coulomb magnetica ha una forza tale per cui l'interazione \emph{decresce con il quadrato della distanza ed è radiale}.

Questa è stata una condizione fondamentale per dimostrare il teorema di Gauss per il campo elettrico, quindi ci aspettiamo che \emph{anche per i campi magnetici valga un teorema di Gauss}. Siccome i magneti sono sempre composti da masse magnetiche opposte che non riusciamo a separare, qualunque superficie chiusa io prenda, all'interno avremo sempre lo stesso numero di masse positive e negative. \emph{Il flusso del campo magnetico attraverso una superficie chiusa sarà nullo}.
\begin{figure}[htpb]
	\centering
	

	% Pattern Info
	 
	\tikzset{
	pattern size/.store in=\mcSize, 
	pattern size = 5pt,
	pattern thickness/.store in=\mcThickness, 
	pattern thickness = 0.3pt,
	pattern radius/.store in=\mcRadius, 
	pattern radius = 1pt}
	\makeatletter
	\pgfutil@ifundefined{pgf@pattern@name@_1aq31xh1f}{
	\pgfdeclarepatternformonly[\mcThickness,\mcSize]{_1aq31xh1f}
	{\pgfqpoint{0pt}{0pt}}
	{\pgfpoint{\mcSize+\mcThickness}{\mcSize+\mcThickness}}
	{\pgfpoint{\mcSize}{\mcSize}}
	{
	\pgfsetcolor{\tikz@pattern@color}
	\pgfsetlinewidth{\mcThickness}
	\pgfpathmoveto{\pgfqpoint{0pt}{0pt}}
	\pgfpathlineto{\pgfpoint{\mcSize+\mcThickness}{\mcSize+\mcThickness}}
	\pgfusepath{stroke}
	}}
	\makeatother
	\tikzset{every picture/.style={line width=0.75pt}} %set default line width to 0.75pt        

	\begin{tikzpicture}[x=0.75pt,y=0.75pt,yscale=-1,xscale=1]
	%uncomment if require: \path (0,300); %set diagram left start at 0, and has height of 300

	%Shape: Ellipse [id:dp9222605722816104] 
	\draw  [pattern=_1aq31xh1f,pattern size=6pt,pattern thickness=0.75pt,pattern radius=0pt, pattern color={rgb, 255:red, 222; green, 222; blue, 222}] (214.01,99.16) .. controls (214.01,65.28) and (241.47,37.81) .. (275.36,37.81) .. controls (309.24,37.81) and (336.71,65.28) .. (336.71,99.16) .. controls (336.71,133.04) and (309.24,160.51) .. (275.36,160.51) .. controls (241.47,160.51) and (214.01,133.04) .. (214.01,99.16) -- cycle ;
	%Shape: Ellipse [id:dp4016085591912175] 
	\draw   (246.68,158.33) .. controls (246.68,142.29) and (269.43,129.29) .. (297.5,129.29) .. controls (325.57,129.29) and (348.32,142.29) .. (348.32,158.33) .. controls (348.32,174.37) and (325.57,187.37) .. (297.5,187.37) .. controls (269.43,187.37) and (246.68,174.37) .. (246.68,158.33) -- cycle ;
	%Shape: Ellipse [id:dp36375410760407245] 
	\draw   (220.54,158.33) .. controls (220.54,134.04) and (255,114.36) .. (297.5,114.36) .. controls (340,114.36) and (374.46,134.04) .. (374.46,158.33) .. controls (374.46,182.62) and (340,202.31) .. (297.5,202.31) .. controls (255,202.31) and (220.54,182.62) .. (220.54,158.33) -- cycle ;
	%Shape: Ellipse [id:dp8010793881307494] 
	\draw   (191.5,158.33) .. controls (191.5,124.88) and (238.96,97.76) .. (297.5,97.76) .. controls (356.04,97.76) and (403.5,124.88) .. (403.5,158.33) .. controls (403.5,191.78) and (356.04,218.9) .. (297.5,218.9) .. controls (238.96,218.9) and (191.5,191.78) .. (191.5,158.33) -- cycle ;
	\draw   (286.75,180.75) -- (300.25,187.5) -- (286.75,194.25) ;
	\draw   (302.25,194.25) -- (315.75,201) -- (302.25,207.75) ;
	\draw   (281.25,211.75) -- (294.75,218.5) -- (281.25,225.25) ;

	% Text Node
	\draw (210.5,53.5) node    {$\Sigma _{g}$};
	% Text Node
	\draw (414,150) node    {$\vec{B}$};


	\end{tikzpicture}
\end{figure}
\FloatBarrier
Possiamo a questo punto ricavare la \textbf{III Equazione di Maxwell} applicando il teorema della Divergenza.
\[
	\Phi_{\Sigma}(\vec{B}) = \int_{\Sigma}\vec{B} \cdot \vec{n} dS = \int_{\tau} \text{div}\vec{B} \, d\tau  = 0
\]
Quindi la divergenza di $\vec{B}$ è sempre nulla in tutti i punti dello spazio.
\[
	\text{div}\vec{B} =0
\]
Questa equazione vale in qualsiasi regime, sia stazionario che tempovariante. Quando un campo vettoriale ha divergenza pari a zero ovunque, esso viene chiamato \emph{solenoidale}. Quindi il campo magnetico è un campo solenoidale. Siccome la divergenza ci dice dove sono le sorgenti di flusso, per questo particolare campo vettoriale non ci sono sorgenti positive o negative, \emph{le linee di flusso sono chiuse}.
\begin{figure}[htpb]
	\centering
	

	\tikzset{every picture/.style={line width=0.75pt}} %set default line width to 0.75pt        

	\begin{tikzpicture}[x=0.75pt,y=0.75pt,yscale=-1,xscale=1]
	%uncomment if require: \path (0,300); %set diagram left start at 0, and has height of 300

	%Shape: Ellipse [id:dp4993447536693967] 
	\draw   (276.73,187.62) .. controls (260.7,187.65) and (247.66,164.91) .. (247.61,136.84) .. controls (247.57,108.78) and (260.54,86) .. (276.57,85.98) .. controls (292.61,85.95) and (305.65,108.69) .. (305.7,136.75) .. controls (305.74,164.82) and (292.77,187.6) .. (276.73,187.62) -- cycle ;
	%Curve Lines [id:da3193485518740673] 
	\draw    (276.73,187.62) .. controls (153.41,192.99) and (137.26,94.02) .. (276.57,85.98) ;
	%Curve Lines [id:da3253006873399502] 
	\draw    (276.73,187.62) .. controls (314.58,194.15) and (346.54,220.14) .. (368.5,224.76) .. controls (390.47,229.37) and (402.44,212.6) .. (400.32,133.6) .. controls (398.19,54.61) and (383.1,61.57) .. (361.5,63) .. controls (339.9,64.43) and (312.9,82.33) .. (276.57,85.98) ;
	%Straight Lines [id:da29819932081250955] 
	\draw    (178,138) -- (141.67,138) ;
	\draw [shift={(138.67,138)}, rotate = 360] [fill={rgb, 255:red, 0; green, 0; blue, 0 }  ][line width=0.08]  [draw opacity=0] (10.72,-5.15) -- (0,0) -- (10.72,5.15) -- (7.12,0) -- cycle    ;
	%Straight Lines [id:da5245812337382556] 
	\draw    (400.33,138.67) -- (364,138.67) ;
	\draw [shift={(361,138.67)}, rotate = 360] [fill={rgb, 255:red, 0; green, 0; blue, 0 }  ][line width=0.08]  [draw opacity=0] (10.72,-5.15) -- (0,0) -- (10.72,5.15) -- (7.12,0) -- cycle    ;
	%Straight Lines [id:da542773982948495] 
	\draw    (436.67,138.67) -- (400.33,138.67) ;
	\draw [shift={(439.67,138.67)}, rotate = 180] [fill={rgb, 255:red, 0; green, 0; blue, 0 }  ][line width=0.08]  [draw opacity=0] (10.72,-5.15) -- (0,0) -- (10.72,5.15) -- (7.12,0) -- cycle    ;

	% Text Node
	\draw (275.33,194.33) node    {$\gamma $};
	% Text Node
	\draw (214,80.33) node    {$\Sigma _{1}$};
	% Text Node
	\draw (336.67,45.67) node    {$\Sigma _{2}$};
	% Text Node
	\draw (127.33,133.67) node    {$\vec{n}_{1}$};
	% Text Node
	\draw (349.67,134.33) node    {$\vec{n}_{2}$};
	% Text Node
	\draw (459.67,138.33) node    {$-\vec{n}_{2}$};


	\end{tikzpicture}
\end{figure}
\FloatBarrier
Immaginiamo di considerare una linea chiusa $\gamma$ su cui è costruita una superficie aperta $\Sigma_1$ e una $\Sigma_2$. Dato che sono due superfici aperte, possiamo scegliere la normale $\vec{n}$ come vogliamo. Immaginiamo che in questa regione dello spazio ci sia un campo magnetico. L'insieme delle due superfici è una superficie chiusa. Chiamiamo $\Sigma$ la loro unione.
\begin{equation*}
	\begin{aligned}
		\Phi_{\Sigma} (\vec{B} ) = 0 &= \int_{\Sigma_1}\vec{B} \cdot \vec{n}_1  dS + \int_{\Sigma_2}\vec{B} \cdot (-\vec{n}_2)dS \\
		&= \underbrace{\int_{\Sigma_1}\vec{B} \cdot \vec{n}_1  dS}_{\Phi_{\Sigma_1}(\vec{B} )}  - \underbrace{\int_{\Sigma_2}\vec{B} \cdot \vec{n}_1 dS}_{\Phi_{\Sigma_2}(\vec{B} )}  \implies \Phi_{\Sigma_1}(\vec{B} ) = \Phi_{\Sigma_2}(\vec{B} )
	\end{aligned}
\end{equation*}
Quindi considerata una linea chiusa $\gamma$, qualunque superficie che abbia tale linea come contorno avrà lo stesso flusso di $\vec{B}$.
Possiamo chiamare questo flusso: \emph{flusso del campo magnetico concatenato a} $\gamma$.







































\section{Forza di Lorentz}

Abbiamo detto che i campi magnetici sono prodotti da cariche in moto che agiscono su altre cariche in moto. Possiamo immaginare come sonda nella regione di spazio in cui è presente un campo $\vec{B}$ una carica in moto.
Si verifica sperimentalmente che sulla carica agisce una forza pari a:
\[
	\boxed{\vec{F} = q\vec{v} \times \vec{B}}
\]
La prima proprietà da osservare è che se la velocità della carica
è zero la forza di Lorentz è pari a zero. Infatti le forze magnetiche agiscono fra cariche in moto. Se la carica è in quiete non c'è alcuna forza che agisce su di essa. Supponiamo di trovarci su un altro sistema di riferimento inerziale che viaggia ad una certa velocità. In tale sistema di riferimento vedo la carica in moto. Sembrerebbe che questa formula dipenda dal sistema di riferimento su cui si trova la carica.
\begin{figure}[htpb]
	\centering
	

	\tikzset{every picture/.style={line width=0.75pt}} %set default line width to 0.75pt        

	\begin{tikzpicture}[x=0.75pt,y=0.75pt,yscale=-1,xscale=1]
	%uncomment if require: \path (0,300); %set diagram left start at 0, and has height of 300

	%Straight Lines [id:da24951894038501976] 
	\draw    (407.7,221.67) -- (256.77,163.43) ;
	\draw [shift={(410.5,222.75)}, rotate = 201.1] [fill={rgb, 255:red, 0; green, 0; blue, 0 }  ][line width=0.08]  [draw opacity=0] (10.72,-5.15) -- (0,0) -- (10.72,5.15) -- (7.12,0) -- cycle    ;
	%Straight Lines [id:da5487854509807417] 
	\draw    (398.24,102.93) -- (256.77,163.43) ;
	\draw [shift={(401,101.75)}, rotate = 156.85] [fill={rgb, 255:red, 0; green, 0; blue, 0 }  ][line width=0.08]  [draw opacity=0] (10.72,-5.15) -- (0,0) -- (10.72,5.15) -- (7.12,0) -- cycle    ;
	%Straight Lines [id:da24462379929793854] 
	\draw    (256.77,57.75) -- (256.77,163.43) ;
	\draw [shift={(256.77,54.75)}, rotate = 90] [fill={rgb, 255:red, 0; green, 0; blue, 0 }  ][line width=0.08]  [draw opacity=0] (10.72,-5.15) -- (0,0) -- (10.72,5.15) -- (7.12,0) -- cycle    ;
	%Shape: Circle [id:dp3064944724065122] 
	\draw  [fill={rgb, 255:red, 222; green, 222; blue, 222 }  ,fill opacity=1 ] (249.27,163.43) .. controls (249.27,159.29) and (252.62,155.93) .. (256.77,155.93) .. controls (260.91,155.93) and (264.27,159.29) .. (264.27,163.43) .. controls (264.27,167.58) and (260.91,170.93) .. (256.77,170.93) .. controls (252.62,170.93) and (249.27,167.58) .. (249.27,163.43) -- cycle ;


	% Text Node
	\draw (256.77,160.93) node    {$+$};
	% Text Node
	\draw (410.5,97.5) node    {$\vec{B}$};
	% Text Node
	\draw (422.5,221) node    {$\vec{v}$};
	% Text Node
	\draw (271,67) node    {$\vec{F}$};
	% Text Node
	\draw (236.5,163) node    {$q$};


	\end{tikzpicture}
\end{figure}
\FloatBarrier
Se la velocità non è nulla ma è parallela a $\vec{B}$, la forza è zero. Se la velocità è perpendicolare a $\vec{B}$ il modulo della forza è massimo. Comunque $\vec{F}$ è sempre perpendicolare a $\vec{v}$. Abbiamo definitivo la potenza dissipata dalla forza come:
\[
	w=\vec{F} \cdot \vec{v} = 0
\]
Tale prodotto scalare diventa allora zero. Significa che la forza di Lorentz non dissipa potenza, ovvero, se consideriamo la nostra carica in moto in presenza di campo magnetico, la forza sarà fatta come in figura.
Se calcoliamo il lavoro compiuto dalla forza infinitesima:
\[
	\mathcal{L}_{AB}=\int_A^B \vec{F} \cdot d\vec{l} =0 \implies \Delta K=0
\]
Il modulo della velocità rimane quindi costante perché $\Delta K$ dipende dal quadrato della velocità, ma non vi è una sua variazione del tempo. Segue che il moto di una carica in un campo magnetico è uniforme. A parità di intensità del campo magnetico, la carica negativa e la carica positiva subiranno forze in versi opposti. Il campo magnetico in un punto è definito come quel vettore tale per cui la forza agente sulla carica in tal punto è pari alla forza di Lorentz.
\[
	|\vec{v} |=v_s = \text{costante}
\]
Dimensionalmente abbiamo
\[
	[F]=\frac{[Q][L]}{[T]}\cdot [B] \qquad \implies [B]=\frac{[F][T]}{[Q][L]}
\]
\[
	\left( \frac{N}{C}\cdot \frac{S}{m} \right) =\left( \frac{V}{m}\cdot \frac{S}{m} \right) = \left( \frac{V\cdot S}{m^2} \right) = \left( \frac{Wb}{m^2}\right)=T
\]
Con $ T =$ Tesla e $ Wb= $ Weber.

Il tesla è una unità di misura sovra-dimensionata. Sulla superficie terrestre, il valore del campo varia in intensità, dall'equatore ai poli, da circa $ 20000 nT $ a $ 70000 nT $.
Si definisce quindi il Gauss: $ 1 G = 10^{-4} T $







































\section{Moto di una carica in un campo magnetico uniforme}

Consideriamo una carica in moto in una regione in cui è presente un campo magnetico $\vec{B}$, con velocità $\vec{v}$ perpendicolare a $ \vec{B}$. L'unica forza che agisce è quella di Lorentz. La deviazione della direzione della velocità avviene sul piano.
\[
	\vec{F} = q\vec{v} \times \vec{B}
\]
\begin{figure}[htpb]
	\centering
	

	\tikzset{every picture/.style={line width=0.75pt}} %set default line width to 0.75pt        

	\begin{tikzpicture}[x=0.75pt,y=0.75pt,yscale=-0.9,xscale=0.9]
	%uncomment if require: \path (0,300); %set diagram left start at 0, and has height of 300

	%Shape: Circle [id:dp23188527005327408] 
	\draw  [fill={rgb, 255:red, 0; green, 0; blue, 0 }  ,fill opacity=1 ] (186.05,29.55) .. controls (186.05,28.03) and (187.28,26.8) .. (188.8,26.8) .. controls (190.32,26.8) and (191.55,28.03) .. (191.55,29.55) .. controls (191.55,31.07) and (190.32,32.3) .. (188.8,32.3) .. controls (187.28,32.3) and (186.05,31.07) .. (186.05,29.55) -- cycle ;
	%Shape: Circle [id:dp3151248833710725] 
	\draw  [fill={rgb, 255:red, 0; green, 0; blue, 0 }  ,fill opacity=1 ] (186.05,99.55) .. controls (186.05,98.03) and (187.28,96.8) .. (188.8,96.8) .. controls (190.32,96.8) and (191.55,98.03) .. (191.55,99.55) .. controls (191.55,101.07) and (190.32,102.3) .. (188.8,102.3) .. controls (187.28,102.3) and (186.05,101.07) .. (186.05,99.55) -- cycle ;
	%Shape: Circle [id:dp33386804817308025] 
	\draw  [fill={rgb, 255:red, 0; green, 0; blue, 0 }  ,fill opacity=1 ] (186.05,169.55) .. controls (186.05,168.03) and (187.28,166.8) .. (188.8,166.8) .. controls (190.32,166.8) and (191.55,168.03) .. (191.55,169.55) .. controls (191.55,171.07) and (190.32,172.3) .. (188.8,172.3) .. controls (187.28,172.3) and (186.05,171.07) .. (186.05,169.55) -- cycle ;
	%Shape: Circle [id:dp7042195434142586] 
	\draw  [fill={rgb, 255:red, 0; green, 0; blue, 0 }  ,fill opacity=1 ] (186.05,239.55) .. controls (186.05,238.03) and (187.28,236.8) .. (188.8,236.8) .. controls (190.32,236.8) and (191.55,238.03) .. (191.55,239.55) .. controls (191.55,241.07) and (190.32,242.3) .. (188.8,242.3) .. controls (187.28,242.3) and (186.05,241.07) .. (186.05,239.55) -- cycle ;
	%Shape: Circle [id:dp6237047411495382] 
	\draw   (224,142.25) .. controls (224,84.12) and (271.12,37) .. (329.25,37) .. controls (387.38,37) and (434.5,84.12) .. (434.5,142.25) .. controls (434.5,200.38) and (387.38,247.5) .. (329.25,247.5) .. controls (271.12,247.5) and (224,200.38) .. (224,142.25) -- cycle ;
	%Straight Lines [id:da6231155453525645] 
	\draw    (329.25,142.25) -- (434.5,142.25) ;
	%Straight Lines [id:da6167820082823245] 
	\draw    (330.27,36.43) -- (432.52,36.43) ;
	\draw [shift={(435.52,36.43)}, rotate = 180] [fill={rgb, 255:red, 0; green, 0; blue, 0 }  ][line width=0.08]  [draw opacity=0] (10.72,-5.15) -- (0,0) -- (10.72,5.15) -- (7.12,0) -- cycle    ;
	%Straight Lines [id:da571191659258169] 
	\draw    (330.27,36.43) -- (330.27,84.25) ;
	\draw [shift={(330.27,87.25)}, rotate = 270] [fill={rgb, 255:red, 0; green, 0; blue, 0 }  ][line width=0.08]  [draw opacity=0] (10.72,-5.15) -- (0,0) -- (10.72,5.15) -- (7.12,0) -- cycle    ;
	%Shape: Circle [id:dp01683522605612775] 
	\draw  [fill={rgb, 255:red, 222; green, 222; blue, 222 }  ,fill opacity=1 ] (322.77,36.43) .. controls (322.77,32.29) and (326.12,28.93) .. (330.27,28.93) .. controls (334.41,28.93) and (337.77,32.29) .. (337.77,36.43) .. controls (337.77,40.58) and (334.41,43.93) .. (330.27,43.93) .. controls (326.12,43.93) and (322.77,40.58) .. (322.77,36.43) -- cycle ;

	%Shape: Circle [id:dp5265854373631442] 
	\draw  [fill={rgb, 255:red, 0; green, 0; blue, 0 }  ,fill opacity=1 ] (116.05,29.55) .. controls (116.05,28.03) and (117.28,26.8) .. (118.8,26.8) .. controls (120.32,26.8) and (121.55,28.03) .. (121.55,29.55) .. controls (121.55,31.07) and (120.32,32.3) .. (118.8,32.3) .. controls (117.28,32.3) and (116.05,31.07) .. (116.05,29.55) -- cycle ;
	%Shape: Circle [id:dp35288825789599043] 
	\draw  [fill={rgb, 255:red, 0; green, 0; blue, 0 }  ,fill opacity=1 ] (116.05,99.55) .. controls (116.05,98.03) and (117.28,96.8) .. (118.8,96.8) .. controls (120.32,96.8) and (121.55,98.03) .. (121.55,99.55) .. controls (121.55,101.07) and (120.32,102.3) .. (118.8,102.3) .. controls (117.28,102.3) and (116.05,101.07) .. (116.05,99.55) -- cycle ;
	%Shape: Circle [id:dp8684657825124835] 
	\draw  [fill={rgb, 255:red, 0; green, 0; blue, 0 }  ,fill opacity=1 ] (116.05,169.55) .. controls (116.05,168.03) and (117.28,166.8) .. (118.8,166.8) .. controls (120.32,166.8) and (121.55,168.03) .. (121.55,169.55) .. controls (121.55,171.07) and (120.32,172.3) .. (118.8,172.3) .. controls (117.28,172.3) and (116.05,171.07) .. (116.05,169.55) -- cycle ;
	%Shape: Circle [id:dp7100765936828766] 
	\draw  [fill={rgb, 255:red, 0; green, 0; blue, 0 }  ,fill opacity=1 ] (116.05,239.55) .. controls (116.05,238.03) and (117.28,236.8) .. (118.8,236.8) .. controls (120.32,236.8) and (121.55,238.03) .. (121.55,239.55) .. controls (121.55,241.07) and (120.32,242.3) .. (118.8,242.3) .. controls (117.28,242.3) and (116.05,241.07) .. (116.05,239.55) -- cycle ;
	%Shape: Circle [id:dp3104752082490132] 
	\draw  [fill={rgb, 255:red, 0; green, 0; blue, 0 }  ,fill opacity=1 ] (556.05,29.55) .. controls (556.05,28.03) and (557.28,26.8) .. (558.8,26.8) .. controls (560.32,26.8) and (561.55,28.03) .. (561.55,29.55) .. controls (561.55,31.07) and (560.32,32.3) .. (558.8,32.3) .. controls (557.28,32.3) and (556.05,31.07) .. (556.05,29.55) -- cycle ;
	%Shape: Circle [id:dp24861452344507984] 
	\draw  [fill={rgb, 255:red, 0; green, 0; blue, 0 }  ,fill opacity=1 ] (556.05,99.55) .. controls (556.05,98.03) and (557.28,96.8) .. (558.8,96.8) .. controls (560.32,96.8) and (561.55,98.03) .. (561.55,99.55) .. controls (561.55,101.07) and (560.32,102.3) .. (558.8,102.3) .. controls (557.28,102.3) and (556.05,101.07) .. (556.05,99.55) -- cycle ;
	%Shape: Circle [id:dp8507140470140022] 
	\draw  [fill={rgb, 255:red, 0; green, 0; blue, 0 }  ,fill opacity=1 ] (556.05,169.55) .. controls (556.05,168.03) and (557.28,166.8) .. (558.8,166.8) .. controls (560.32,166.8) and (561.55,168.03) .. (561.55,169.55) .. controls (561.55,171.07) and (560.32,172.3) .. (558.8,172.3) .. controls (557.28,172.3) and (556.05,171.07) .. (556.05,169.55) -- cycle ;
	%Shape: Circle [id:dp8349390271245463] 
	\draw  [fill={rgb, 255:red, 0; green, 0; blue, 0 }  ,fill opacity=1 ] (556.05,239.55) .. controls (556.05,238.03) and (557.28,236.8) .. (558.8,236.8) .. controls (560.32,236.8) and (561.55,238.03) .. (561.55,239.55) .. controls (561.55,241.07) and (560.32,242.3) .. (558.8,242.3) .. controls (557.28,242.3) and (556.05,241.07) .. (556.05,239.55) -- cycle ;
	%Shape: Circle [id:dp7166248238687152] 
	\draw  [fill={rgb, 255:red, 0; green, 0; blue, 0 }  ,fill opacity=1 ] (486.05,29.55) .. controls (486.05,28.03) and (487.28,26.8) .. (488.8,26.8) .. controls (490.32,26.8) and (491.55,28.03) .. (491.55,29.55) .. controls (491.55,31.07) and (490.32,32.3) .. (488.8,32.3) .. controls (487.28,32.3) and (486.05,31.07) .. (486.05,29.55) -- cycle ;
	%Shape: Circle [id:dp05061953433231636] 
	\draw  [fill={rgb, 255:red, 0; green, 0; blue, 0 }  ,fill opacity=1 ] (486.05,99.55) .. controls (486.05,98.03) and (487.28,96.8) .. (488.8,96.8) .. controls (490.32,96.8) and (491.55,98.03) .. (491.55,99.55) .. controls (491.55,101.07) and (490.32,102.3) .. (488.8,102.3) .. controls (487.28,102.3) and (486.05,101.07) .. (486.05,99.55) -- cycle ;
	%Shape: Circle [id:dp7111663881487575] 
	\draw  [fill={rgb, 255:red, 0; green, 0; blue, 0 }  ,fill opacity=1 ] (486.05,169.55) .. controls (486.05,168.03) and (487.28,166.8) .. (488.8,166.8) .. controls (490.32,166.8) and (491.55,168.03) .. (491.55,169.55) .. controls (491.55,171.07) and (490.32,172.3) .. (488.8,172.3) .. controls (487.28,172.3) and (486.05,171.07) .. (486.05,169.55) -- cycle ;
	%Shape: Circle [id:dp9560662872299377] 
	\draw  [fill={rgb, 255:red, 0; green, 0; blue, 0 }  ,fill opacity=1 ] (486.05,239.55) .. controls (486.05,238.03) and (487.28,236.8) .. (488.8,236.8) .. controls (490.32,236.8) and (491.55,238.03) .. (491.55,239.55) .. controls (491.55,241.07) and (490.32,242.3) .. (488.8,242.3) .. controls (487.28,242.3) and (486.05,241.07) .. (486.05,239.55) -- cycle ;

	% Text Node
	\draw (330.27,33.93) node    {$+$};
	% Text Node
	\draw (385,128) node    {$R$};
	% Text Node
	\draw (315.5,65) node    {$\vec{F}$};
	% Text Node
	\draw (444.5,35.5) node    {$\vec{v}$};
	% Text Node
	\draw (536.17,31.67) node    {$\vec{B}$};


	\end{tikzpicture}
\end{figure}
\FloatBarrier
Il modulo della forza è costante del tempo. Si tratta di un moto circolare uniforme, di cui si osservano tutte le proprietà caratteristiche. Possiamo definire il raggio $ R $ della traiettoria e provare a stabilire un legame fra $ R $ e la forza, che possiamo pensare come centripeta (diretta verso il centro)
\[
	\vec{F} =m\vec{a}_c \qquad |\vec{F} |=m a_c=\frac{mv^2}{R}=qvB \implies \boxed{R=\frac{mv}{qB}}
\]
Ricordiamo ora dalla Fisica I la seguente relazione vettoriale
\[
	\vec{a}_c=\vec{\omega} \times \vec{v}
\]
Pertanto avremo, sostituendo nelle relazioni precedenti:
\begin{equation*}
	\begin{aligned}
		\vec{F} =q\vec{v} \times \vec{B} &= m\vec{a}_c \\
		\vec{v} \times q\vec{B}  &= m\vec{\omega} \times \vec{v} \\
		\vec{v} \times q\vec{B}  &=\vec{v} \times (-m\vec{\omega} )
	\end{aligned}
\end{equation*}
Dove è stata applicata la proprietà anticommutativa del prodotto vettoriale, infine abbiamo
\[
	q\vec{B} = -m\vec{\omega} \implies \boxed{\vec{\omega} = -\frac{q\vec{B}}{m}}
\]
Vettorialmente vediamo quindi che $ \omega $ è sempre opposta a $\vec{B}$.

Nel caso in cui $\vec{v}$ non sia ortogonale al campo ma formi un certo angolo $\vartheta$ con esso dovremo considerare la componente perpendicolare.
\[
	\vec{F} =q(\vec{v}_{\parallel}+\vec{v}_{\bot})\times \vec{B} = q\vec{v}_{\bot}\times \vec{B}
\]
\begin{figure}[htpb]
	\centering
	

	\tikzset{every picture/.style={line width=0.75pt}} %set default line width to 0.75pt        

	\begin{tikzpicture}[x=0.75pt,y=0.75pt,yscale=-0.7,xscale=0.7]
	%uncomment if require: \path (0,300); %set diagram left start at 0, and has height of 300

	%Shape: Spring [id:dp9279391676836501] 
	\draw  [color={rgb, 255:red, 0; green, 0; blue, 0 }  ,draw opacity=1 ] (345,250.09) .. controls (349.98,258.03) and (356.42,263) .. (364.5,263) .. controls (404.5,263) and (404.5,141) .. (384.5,141) .. controls (364.5,141) and (364.5,263) .. (404.5,263) .. controls (444.5,263) and (444.5,141) .. (424.5,141) .. controls (404.5,141) and (404.5,263) .. (444.5,263) .. controls (484.5,263) and (484.5,141) .. (464.5,141) .. controls (444.5,141) and (444.5,263) .. (484.5,263) .. controls (524.5,263) and (524.5,141) .. (504.5,141) .. controls (484.5,141) and (484.5,263) .. (524.5,263) .. controls (529.43,263) and (533.75,261.15) .. (537.5,257.9) ;
	%Straight Lines [id:da7085363059097909] 
	\draw    (36,200) -- (183,200) ;
	\draw [shift={(186,200)}, rotate = 180] [fill={rgb, 255:red, 0; green, 0; blue, 0 }  ][line width=0.08]  [draw opacity=0] (10.72,-5.15) -- (0,0) -- (10.72,5.15) -- (7.12,0) -- cycle    ;
	%Straight Lines [id:da8596236417176182] 
	\draw    (36,200) -- (183.5,101.66) ;
	\draw [shift={(186,100)}, rotate = 506.31] [fill={rgb, 255:red, 0; green, 0; blue, 0 }  ][line width=0.08]  [draw opacity=0] (10.72,-5.15) -- (0,0) -- (10.72,5.15) -- (7.12,0) -- cycle    ;
	%Straight Lines [id:da4245362862630413] 
	\draw    (36,200) -- (36,103) ;
	\draw [shift={(36,100)}, rotate = 450] [fill={rgb, 255:red, 0; green, 0; blue, 0 }  ][line width=0.08]  [draw opacity=0] (10.72,-5.15) -- (0,0) -- (10.72,5.15) -- (7.12,0) -- cycle    ;
	%Straight Lines [id:da5299214536869359] 
	\draw  [dash pattern={on 0.84pt off 2.51pt}]  (36,100) -- (186,100) ;
	%Straight Lines [id:da31044452089539964] 
	\draw  [dash pattern={on 0.84pt off 2.51pt}]  (186,200) -- (186,100) ;
	%Straight Lines [id:da3319296169460262] 
	\draw    (36,202) -- (253,202) ;
	\draw [shift={(256,202)}, rotate = 180] [fill={rgb, 255:red, 0; green, 0; blue, 0 }  ][line width=0.08]  [draw opacity=0] (10.72,-5.15) -- (0,0) -- (10.72,5.15) -- (7.12,0) -- cycle    ;
	%Shape: Arc [id:dp7996534054265312] 
	\draw  [draw opacity=0] (60.86,183.21) .. controls (64.09,187.97) and (65.98,193.71) .. (66,199.89) -- (36,200) -- cycle ; \draw   (60.86,183.21) .. controls (64.09,187.97) and (65.98,193.71) .. (66,199.89) ;
	%Straight Lines [id:da0474297745656187] 
	\draw  [dash pattern={on 0.84pt off 2.51pt}]  (384,200) -- (384,141.33) ;
	%Straight Lines [id:da9478959187866598] 
	\draw    (336,202) -- (550.5,202) ;
	\draw [shift={(553.5,202)}, rotate = 180] [fill={rgb, 255:red, 0; green, 0; blue, 0 }  ][line width=0.08]  [draw opacity=0] (10.72,-5.15) -- (0,0) -- (10.72,5.15) -- (7.12,0) -- cycle    ;

	% Text Node
	\draw (159,181) node    {$\vec{v}_{p}$};
	% Text Node
	\draw (22,136) node    {$\vec{v}_{n}$};
	% Text Node
	\draw (201,91) node    {$\vec{v}_{0}$};
	% Text Node
	\draw (271,201) node    {$\vec{B}$};
	% Text Node
	\draw (75.33,186) node    {$\vartheta $};
	% Text Node
	\draw (355,179) node    {$R$};
	% Text Node
	\draw (570,199.33) node    {$\vec{B}$};


	\end{tikzpicture}
\end{figure}
\FloatBarrier
Abbiamo quindi la composizione del moto circolare uniforme in un piano ortogonale a $\vec{B}$ e del moto circolare uniforme lungo $\vec{B}$, che è un moto elicoidale uniforme avente come asse la direzione di $\vec{B}$. Nel tempo la particella si sposta lungo $\vec{B}$ della quantità:
\[
	p=v_{\parallel}\, t= \frac{2\pi mv\cos \vartheta}{qB}
\]
detto \emph{passo dell'elica}.







































\section{Forza magnetica agente su conduttori percorsi da corrente (II legge di Laplace)}

Se consideriamo il caso di un conduttore in cui fluisce corrente, avremo delle cariche in moto all'interno di esso. Che cosa accade se espongo il conduttore percorso da corrente ad un campo magnetico? Ciascuna delle cariche in moto subirà una forza di Lorentz. Il moto delle cariche è molto complesso: caotico sovrapposto a un moto di deriva. Se calcoliamo il valor medio della forza di Lorentz agente sulla singola carica avremo:
\[
	\langle \vec{F} \rangle = q\vec{v}_d\times \vec{B}
\]
\begin{figure}[htpb]
	\centering
	

	\tikzset{every picture/.style={line width=0.75pt}} %set default line width to 0.75pt        

	\begin{tikzpicture}[x=0.75pt,y=0.75pt,yscale=-0.9,xscale=0.9]
	%uncomment if require: \path (0,300); %set diagram left start at 0, and has height of 300

	%Straight Lines [id:da7320324873567214] 
	\draw    (332.75,76) -- (145.25,190) ;
	%Straight Lines [id:da5802726838205565] 
	\draw    (239,133) -- (456,133) ;
	\draw [shift={(459,133)}, rotate = 180] [fill={rgb, 255:red, 0; green, 0; blue, 0 }  ][line width=0.08]  [draw opacity=0] (10.72,-5.15) -- (0,0) -- (10.72,5.15) -- (7.12,0) -- cycle    ;
	%Straight Lines [id:da4587584641746505] 
	\draw    (218.94,131.28) -- (155.25,170) ;
	\draw [shift={(221.5,129.72)}, rotate = 148.7] [fill={rgb, 255:red, 0; green, 0; blue, 0 }  ][line width=0.08]  [draw opacity=0] (10.72,-5.15) -- (0,0) -- (10.72,5.15) -- (7.12,0) -- cycle    ;
	%Straight Lines [id:da8949125682974581] 
	\draw    (275.94,106.72) -- (236,131) ;
	\draw [shift={(278.5,105.16)}, rotate = 148.7] [fill={rgb, 255:red, 0; green, 0; blue, 0 }  ][line width=0.08]  [draw opacity=0] (10.72,-5.15) -- (0,0) -- (10.72,5.15) -- (7.12,0) -- cycle    ;
	%Straight Lines [id:da667896447295699] 
	\draw    (239,133) -- (239,209) ;
	\draw [shift={(239,212)}, rotate = 270] [fill={rgb, 255:red, 0; green, 0; blue, 0 }  ][line width=0.08]  [draw opacity=0] (10.72,-5.15) -- (0,0) -- (10.72,5.15) -- (7.12,0) -- cycle    ;

	% Text Node
	\draw (470,132) node    {$\vec{B}$};
	% Text Node
	\draw (187,139) node    {$I$};
	% Text Node
	\draw (249,105) node    {$d\vec{l}$};
	% Text Node
	\draw (257,199) node    {$d\vec{F}$};


	\end{tikzpicture}
\end{figure}
\FloatBarrier
Consideriamo un tratto $dl$ di lunghezza infinitesima sul nostro conduttore, parallelo alla superficie del conduttore e diretto nel verso in cui scorre la corrente. Consideriamo anche il volume di conduttore $ d\tau$, avremo:
\[
	d\vec{F} = \langle \vec{F} \rangle n \, d\tau \qquad d\tau =S\,dl
\]
\[
	d\vec{F} =q\vec{v}_d\times \vec{B} \,n\,S\,dl = \vec{J} \times \vec{B} \,d\tau
\]
Chiameremo allora forza magnetica per unità di volume il vettore:
\[
	\vec{f} = \frac{d\vec{F}}{d\tau}= \vec{J} \times \vec{B}
\]
Nota l'espressione di $\vec{f}$ possiamo calcolare la forza totale agente sul conduttore come
\[
	\vec{F} =\int_{\tau} \vec{f} \, d\tau = \int_{\tau}\vec{J} \times \vec{B} \,d\tau
\]
Applichiamo questo al caso di un filo, in cui $ d\vec{F} =\vec{J} \times \vec{B} \,dl\,S $ e in regime stazionario $ \vec{J} \parallel d\vec{l}$.
\begin{align*}
	d\vec{F} &=\vec{J} \times \vec{B} \,dl\,S \\
	&= d\vec{l} \times \vec{B} JS\\
	&= I\,d\vec{l} \times \vec{B}\\
	\Aboxed{d\vec{F}&= I\,d\vec{l} \times \vec{B}}
\end{align*}
Tale formula prende il nome di \textbf{II legge di Laplace}.

Non è possibile dimostrare sperimentalmente la formula trovata. Per farlo dovremo in qualche modo isolare il tratto dal resto del circuito e calcolare il campo magnetico. Ma se siamo in regime stazionario questo tratto deve appartenente al circuito, non può essere separato da esso. Tuttavia, se usiamo questa formula per calcolare la forza agente sull'intero circuito, il risultato ottenuto coincide con ciò che si osserva sperimentalmente. In altre parole, non ha senso l'elemento infinitesimo aperto di corrente in un conduttore, se non come strumento matematico di calcolo.

Questa legge esprime il fatto che la forza magnetica su un tratto infinitesimo di filo percorso da corrente è ortogonale al filo e al campo magnetico ed è orientata rispetto a $ d\vec{l}$ e $ \vec{B}$ secondo la regola della vite destrorsa. Osserviamo che le caratteristiche della forza non dipendono dal segno dei portatori di carica e che essa è in ogni caso proporzionale all'intensità di corrente.

Se vogliamo considerare la forza agente su un tratto macroscopico di filo $ AB $ in una regione in cui è presente un campo magnetico, utilizzeremo la formula di Laplace insieme al principio di sovrapposizione degli effetti
\[
	\vec{F}_{AB}=\int_{A\,\gamma}^B d\vec{F} = \int_{A\,\gamma}^B I\,d\vec{l} \times \vec{B}
\]
E in particolare per una $\gamma$ chiusa:
\[
	\vec{F} = \oint_{\gamma} I\,d\vec{l} \times \vec{B}
\]
Se consideriamo il caso di un filo immerso in campo magnetico uniforme, avremo che $\vec{B}$ è costante. $I$ è costante perché siamo in regime stazionario e quindi:
\[
	\vec{F}_{AB}= \int_{A\,\gamma}^B I\,d\vec{l} \times \vec{B} =I \left( \int_{A\,\gamma}^B d\vec{l}  \right) \times \vec{B} =I\,\vec{l} \times \vec{B}
\]
L'integrale è una somma di vettori spostamento lungo il percorso $\gamma$ che ci dà il vettore spostamento in linea retta che da $A$ punta verso $B$.
Fissati $A$ e $B$, qualunque sia il percorso del filo, la formula restituisce sempre lo stesso risultato. Se consideriamo l'intero circuito e calcoliamo la forza agente:
\[
	\vec{F} = \oint_{\gamma} I\,d\vec{l} \times \vec{B} = I\left( \oint_{\gamma} d\vec{l}  \right) \times \vec{B} = 0
\]
\begin{figure}[htpb]
	\centering
	

	\tikzset{every picture/.style={line width=0.75pt}} %set default line width to 0.75pt        

	\begin{tikzpicture}[x=0.75pt,y=0.75pt,yscale=-1,xscale=1]
	%uncomment if require: \path (0,300); %set diagram left start at 0, and has height of 300

	%Shape: Ellipse [id:dp81007255868101] 
	\draw   (153,146) .. controls (153,115.62) and (217.14,91) .. (296.25,91) .. controls (375.36,91) and (439.5,115.62) .. (439.5,146) .. controls (439.5,176.38) and (375.36,201) .. (296.25,201) .. controls (217.14,201) and (153,176.38) .. (153,146) -- cycle ;
	\draw   (286,85) -- (298,91) -- (286,97) ;
	%Straight Lines [id:da5676795332498741] 
	\draw    (93,146) -- (496.5,146) ;
	\draw [shift={(499.5,146)}, rotate = 180] [fill={rgb, 255:red, 0; green, 0; blue, 0 }  ][line width=0.08]  [draw opacity=0] (10.72,-5.15) -- (0,0) -- (10.72,5.15) -- (7.12,0) -- cycle    ;

	% Text Node
	\draw (513,143) node    {$\vec{B}$};
	% Text Node
	\draw (310,78) node    {$I$};
	% Text Node
	\draw (151,168) node    {$\gamma $};


	\end{tikzpicture}
\end{figure}
\FloatBarrier
Il fatto che la forza sia nulla non significa che non ci sia alcuna interazione fra campo magnetico e circuito. \emph{Non c'è traslazione ma ci potrà essere rotazione}.







































\section{Campo magnetico agente su una spira e momento magnetico}

Consideriamo una spira piana, rettangolare e rigida come in figura.
\begin{figure}[htpb]
	\centering
	

	\tikzset{every picture/.style={line width=0.75pt}} %set default line width to 0.75pt        

	\begin{tikzpicture}[x=0.75pt,y=0.75pt,yscale=-1,xscale=1]
	%uncomment if require: \path (0,340); %set diagram left start at 0, and has height of 340

	%Shape: Rectangle [id:dp314599602769986] 
	\draw  [line width=1.5]  (243.5,23.64) -- (243.5,207.39) -- (138.5,256.36) -- (138.5,72.61) -- cycle ;
	%Straight Lines [id:da2971679923140418] 
	\draw    (118.5,72.61) -- (118.5,256.36) ;
	\draw [shift={(118.5,256.36)}, rotate = 270] [color={rgb, 255:red, 0; green, 0; blue, 0 }  ][line width=0.75]    (0,5.59) -- (0,-5.59)   ;
	\draw [shift={(118.5,72.61)}, rotate = 270] [color={rgb, 255:red, 0; green, 0; blue, 0 }  ][line width=0.75]    (0,5.59) -- (0,-5.59)   ;
	%Straight Lines [id:da3837571296766522] 
	\draw    (253.5,227.39) -- (148.5,276.36) ;
	\draw [shift={(148.5,276.36)}, rotate = 335] [color={rgb, 255:red, 0; green, 0; blue, 0 }  ][line width=0.75]    (0,5.59) -- (0,-5.59)   ;
	\draw [shift={(253.5,227.39)}, rotate = 335] [color={rgb, 255:red, 0; green, 0; blue, 0 }  ][line width=0.75]    (0,5.59) -- (0,-5.59)   ;
	%Straight Lines [id:da9220289196713451] 
	\draw    (191,140) -- (295,140) ;
	\draw [shift={(298,140)}, rotate = 180] [fill={rgb, 255:red, 0; green, 0; blue, 0 }  ][line width=0.08]  [draw opacity=0] (10.72,-5.15) -- (0,0) -- (10.72,5.15) -- (7.12,0) -- cycle    ;
	%Straight Lines [id:da49361292246796284] 
	\draw    (191,140) -- (232.53,168.79) ;
	\draw [shift={(235,170.5)}, rotate = 214.73] [fill={rgb, 255:red, 0; green, 0; blue, 0 }  ][line width=0.08]  [draw opacity=0] (10.72,-5.15) -- (0,0) -- (10.72,5.15) -- (7.12,0) -- cycle    ;
	%Straight Lines [id:da05198423899342308] 
	\draw    (243.5,23.64) -- (138.5,72.61) ;
	\draw [shift={(191,48.12)}, rotate = 335] [fill={rgb, 255:red, 0; green, 0; blue, 0 }  ][line width=0.08]  [draw opacity=0] (10.72,-5.15) -- (0,0) -- (10.72,5.15) -- (7.12,0) -- cycle    ;
	%Straight Lines [id:da5634924894683115] 
	\draw    (243.5,207.39) -- (138.5,256.36) ;
	\draw [shift={(191,231.88)}, rotate = 155] [fill={rgb, 255:red, 0; green, 0; blue, 0 }  ][line width=0.08]  [draw opacity=0] (10.72,-5.15) -- (0,0) -- (10.72,5.15) -- (7.12,0) -- cycle    ;
	%Straight Lines [id:da6157860000143798] 
	\draw [line width=1.5]    (403.5,153.36) -- (508.5,124.39) ;
	%Straight Lines [id:da5362688552515869] 
	\draw    (508.5,124.39) -- (508.5,78.5) ;
	\draw [shift={(508.5,75.5)}, rotate = 450] [fill={rgb, 255:red, 0; green, 0; blue, 0 }  ][line width=0.08]  [draw opacity=0] (10.72,-5.15) -- (0,0) -- (10.72,5.15) -- (7.12,0) -- cycle    ;
	%Straight Lines [id:da880907260663218] 
	\draw    (403.5,199.25) -- (403.5,153.36) ;
	\draw [shift={(403.5,202.25)}, rotate = 270] [fill={rgb, 255:red, 0; green, 0; blue, 0 }  ][line width=0.08]  [draw opacity=0] (10.72,-5.15) -- (0,0) -- (10.72,5.15) -- (7.12,0) -- cycle    ;
	%Straight Lines [id:da24430913160095735] 
	\draw    (456,138.88) -- (469.68,188.48) ;
	\draw [shift={(470.48,191.38)}, rotate = 254.57999999999998] [fill={rgb, 255:red, 0; green, 0; blue, 0 }  ][line width=0.08]  [draw opacity=0] (10.72,-5.15) -- (0,0) -- (10.72,5.15) -- (7.12,0) -- cycle    ;
	%Straight Lines [id:da5766759564623969] 
	\draw    (456,138.88) -- (528.8,138.88) ;
	\draw [shift={(531.8,138.88)}, rotate = 180] [fill={rgb, 255:red, 0; green, 0; blue, 0 }  ][line width=0.08]  [draw opacity=0] (10.72,-5.15) -- (0,0) -- (10.72,5.15) -- (7.12,0) -- cycle    ;
	%Shape: Arc [id:dp19135976738545968] 
	\draw  [draw opacity=0] (481.52,139.08) .. controls (481.43,150.8) and (473.44,160.64) .. (462.61,163.54) -- (456,138.88) -- cycle ; \draw   (481.52,139.08) .. controls (481.43,150.8) and (473.44,160.64) .. (462.61,163.54) ;
	%Curve Lines [id:da74730732313342] 
	\draw    (450.6,120) .. controls (452.16,83.73) and (397.06,82.82) .. (394.68,116.91) ;
	\draw [shift={(394.6,119.6)}, rotate = 269.38] [fill={rgb, 255:red, 0; green, 0; blue, 0 }  ][line width=0.08]  [draw opacity=0] (10.72,-5.15) -- (0,0) -- (10.72,5.15) -- (7.12,0) -- cycle    ;
	%Shape: Circle [id:dp7235812700074382] 
	\draw  [fill={rgb, 255:red, 0; green, 0; blue, 0 }  ,fill opacity=1 ] (412.8,121) .. controls (412.8,119.9) and (413.7,119) .. (414.8,119) .. controls (415.9,119) and (416.8,119.9) .. (416.8,121) .. controls (416.8,122.1) and (415.9,123) .. (414.8,123) .. controls (413.7,123) and (412.8,122.1) .. (412.8,121) -- cycle ;

	% Text Node
	\draw (108.5,164.48) node    {$a$};
	% Text Node
	\draw (212.5,261.52) node    {$b$};
	% Text Node
	\draw (310.5,138.48) node    {$\vec{B}$};
	% Text Node
	\draw (199.5,165.48) node    {$\vec{n}$};
	% Text Node
	\draw (195.5,60.48) node  [font=\footnotesize]  {$1$};
	% Text Node
	\draw (149.5,161.48) node  [font=\footnotesize]  {$2$};
	% Text Node
	\draw (201.5,218.48) node  [font=\footnotesize]  {$3$};
	% Text Node
	\draw (237.5,118.48) node  [font=\footnotesize]  {$4$};
	% Text Node
	\draw (184.5,36.48) node  [font=\normalsize]  {$I$};
	% Text Node
	\draw (528.5,82.48) node    {$\vec{F}_{4}$};
	% Text Node
	\draw (419.5,183.48) node    {$\vec{F}_{2}$};
	% Text Node
	\draw (469.9,204.68) node    {$\vec{n}$};
	% Text Node
	\draw (543.3,142.08) node    {$\vec{B}$};
	% Text Node
	\draw (483.3,156.88) node    {$\vartheta $};
	% Text Node
	\draw (431.1,117.48) node    {$\vec{M}$};


	\end{tikzpicture}
\end{figure}
\FloatBarrier
Assumiamo che ci sia una normale ortogonale al piano della spira e che essa si trovi in una regione in cui è presente un campo magnetico uniforme diretto in modo tale da essere perpendicolare ai lati maggiori di lunghezza $a$.
I fili percorsi da correnti sono sottoposti a forze magnetiche. Possiamo calcolare per ciascuno dei quattro lati l'andamento delle forze. Come si deduce dalla figura, le forze $\vec{F}_1$ ed $\vec{F}_3$ sono eguali e contrarie e hanno la stessa retta di azione. Nel loro insieme formano una coppia di braccio nullo e quindi un momento nullo. Per quanto riguarda $\vec{F}_4$ ed $\vec{F}_2$, anche tali forze hanno lo stesso modulo ma costituiscono una coppia di braccio $b\sin \vartheta$, dove $ \vartheta$ è l'angolo che $\vec{B}$ forma con la normale.
\begin{gather*}
	\vec{F} =I\vec{l} \times \vec{B} \qquad \vec{F}_3=-\vec{F}_1 \qquad \vec{F}_2 = -\vec{F}_4 \\
	\vec{R} = \vec{F}_1+\vec{F}_2+\vec{F}_3+\vec{F}_4
\end{gather*}
Calcoliamo il modulo della forza $ \vec{F}_4$ e il momento ad essa associato.
\[
	|\vec{F}_4 |=I\,a\,B \qquad |\vec{M} |=|\vec{F}_4 |\cdot \text{braccio} = I\,a\,B\,b\,\sin \vartheta = ISB\,\sin \vartheta
\]
Quindi
\[
	\vec{M} = IS\vec{n} \times \vec{B}
\]
Per analogia col campo elettrico si introduce il \emph{momento di dipolo magnetico della forza} come
\[
	\boxed{\vec{m} =IS\vec{n}} \implies \boxed{\vec{M} =\vec{m} \times \vec{B}}
\]
Questa formula ci dice che se consideriamo un magnete permanente immerso in un campo magnetico $\vec{B}$, anche su questo magnete agisce un campo delle forze che tende a farlo ruotare e quindi potremmo scrivere il momento delle forze agenti sul magente come abbiamo visto. Esso è una proprietà dell'ago magnetico che va misurata.
\begin{figure}[htpb]
	\centering
	

	\tikzset{every picture/.style={line width=0.75pt}} %set default line width to 0.75pt        

	\begin{tikzpicture}[x=0.75pt,y=0.75pt,yscale=-1,xscale=1]
	%uncomment if require: \path (0,300); %set diagram left start at 0, and has height of 300

	%Straight Lines [id:da05135261111632694] 
	\draw    (82.5,135) -- (325.8,135) ;
	\draw [shift={(328.8,135)}, rotate = 180] [fill={rgb, 255:red, 0; green, 0; blue, 0 }  ][line width=0.08]  [draw opacity=0] (10.72,-5.15) -- (0,0) -- (10.72,5.15) -- (7.12,0) -- cycle    ;
	%Shape: Triangle [id:dp9590788161473491] 
	\draw  [fill={rgb, 255:red, 222; green, 222; blue, 222 }  ,fill opacity=1 ] (254,135) -- (214,147.5) -- (214,122.5) -- cycle ;
	%Shape: Triangle [id:dp8011408545698926] 
	\draw  [fill={rgb, 255:red, 255; green, 255; blue, 255 }  ,fill opacity=1 ] (174,135) -- (214,122.5) -- (214,147.5) -- cycle ;


	% Text Node
	\draw (344.3,132.08) node    {$\vec{B}$};
	% Text Node
	\draw (253.3,116.08) node    {$\vec{m}$};


	\end{tikzpicture}
\end{figure}
\FloatBarrier
Il risultato è valido in realtà per un circuito piano di forma qualunque immerso in un campo magnetico uniforme. Infatti il circuito si può sempre approssimare con n circuiti rettangolari adiacenti, tutti percorsi da una stessa corrente i con verso tale da avere le normali concordemente orientate. Le correnti che passano nei lati in comune sono eguali ed opposte per cui gli effetti del campo magnetico si annullano e rimangono solo gli effetti prodotti sul contorno esterno, coincidente con il circuito.
\begin{figure}[htpb]
	\centering
	

	\tikzset{every picture/.style={line width=0.75pt}} %set default line width to 0.75pt        

	\begin{tikzpicture}[x=0.75pt,y=0.75pt,yscale=-1,xscale=1]
	%uncomment if require: \path (0,300); %set diagram left start at 0, and has height of 300

	%Shape: Rectangle [id:dp2081961356269224] 
	\draw  [line width=1.5]  (388.36,213.5) -- (204.61,213.5) -- (155.64,108.5) -- (339.39,108.5) -- cycle ;
	%Straight Lines [id:da47375466830675284] 
	\draw    (388.36,213.5) -- (339.39,108.5) ;
	\draw [shift={(363.88,161)}, rotate = 425] [fill={rgb, 255:red, 0; green, 0; blue, 0 }  ][line width=0.08]  [draw opacity=0] (10.72,-5.15) -- (0,0) -- (10.72,5.15) -- (7.12,0) -- cycle    ;
	%Straight Lines [id:da1877948623602992] 
	\draw    (204.61,213.5) -- (155.64,108.5) ;
	\draw [shift={(180.12,161)}, rotate = 245] [fill={rgb, 255:red, 0; green, 0; blue, 0 }  ][line width=0.08]  [draw opacity=0] (10.72,-5.15) -- (0,0) -- (10.72,5.15) -- (7.12,0) -- cycle    ;
	%Straight Lines [id:da5368840401436481] 
	\draw    (292,161) -- (292,43) ;
	\draw [shift={(292,40)}, rotate = 450] [fill={rgb, 255:red, 0; green, 0; blue, 0 }  ][line width=0.08]  [draw opacity=0] (10.72,-5.15) -- (0,0) -- (10.72,5.15) -- (7.12,0) -- cycle    ;
	%Straight Lines [id:da8370373990593616] 
	\draw    (242,161) -- (242,43) ;
	\draw [shift={(242,40)}, rotate = 450] [fill={rgb, 255:red, 0; green, 0; blue, 0 }  ][line width=0.08]  [draw opacity=0] (10.72,-5.15) -- (0,0) -- (10.72,5.15) -- (7.12,0) -- cycle    ;

	% Text Node
	\draw (311.3,48.08) node    {$\vec{m}$};
	% Text Node
	\draw (261.3,48.08) node    {$\vec{n}$};
	% Text Node
	\draw (467.3,146.08) node    {$\vec{m} =IdS\vec{n}$};
	% Text Node
	\draw (378.3,152.08) node    {$I$};
	% Text Node
	\draw (335.3,192.08) node    {$dS$};


	\end{tikzpicture}
\end{figure}
\FloatBarrier
Il momento risulta nullo soltanto se $\vec{m}$ è parallelo a $\vec{B}$. La posizione con $ \vartheta =0 $ è di equilibrio stabile, con $\vartheta =\pi$, di equilibrio instabile. Per qualsiasi altro angolo $\vec{M}$ tende a far ruotare la spira in modo che il momento magnetico $\vec{m}$ diventi parallelo a $\vec{B}$. Sospendendo opportunamente la spira è possibile generare in questo modo un moto oscillatorio. Ed è così che tramite l'orientazione di un piccolo circuito si possono ottenere direzione e verso di $\vec{B}$, mentre dalle piccole oscillazioni se ne deduce il modulo. Anche un ago magnetico ha un comportamento del tutto simile quando posto in un campo magnetico. Questa identità di comportamento fra quest'ultimo e la spira nei riguardi delle azioni meccaniche subite quando posti in un campo magnetico uniforme venne generalizzata da Ampere sottoforma di un postulato detto principio di equivalenza di Ampere. Afferma che una spira di area $ d\Sigma$ percorsa dalla corrente i equivale agli effetti magnetici a un dipolo elementare di momento magnetico perpendicolare al piano della spira e orientato rispetto al verso della corrente secondo la regola della vite. In analogia con quanto visto per il dipolo elettrico, anche per il dipolo magnetico si definisce un energia potenziale, legata alla posizione angolare rispetto alla direzione di $\vec{B}$. Si parla di energia potenziale magnetostatica ed è pari a:
\[
	\boxed{U_m = -\vec{m} \cdot \vec{B}}
\]
Questo deriva dal fatto che le interazioni tra il campo e il dipolo sono del tutto analoghe.







































\section{Espressioni di forza, momento e lavoro tramite flusso magnetico}

Immaginiamo di considerare un circuito $\gamma$ di forma generica percorso da corrente $I$. Immaginiamo di costruire su questo circuito una superficie $\Sigma$ che abbia $\gamma$ come contorno o, come si dice, che si appoggi sul circuito $\gamma$. Possiamo pensare questo circuito come ricoperto da tante spire elementari percorse dalla stessa corrente $I$. Se prendiamo due spirette vicine, le correnti sui lati adiacenti si cancellano e rimangono solo le correnti sul bordo che rappresentano il circuito percorso da corrente. A questo punto possiamo immaginare che ci sia un campo magnetico e chiederci quanto valga l'energia magnetostatica. Dato che il circuito è equivalente alla scomposizione in tante spire, possiamo scrivere l'energia potenziale come la somma dell'energia potenziale di ciascuna di esse.
\[
	U_m=\int_{\Sigma}(-I\,dS\,\vec{n} ) \cdot \vec{B} = -I\int_{\Sigma}dS\,\vec{n} \cdot \vec{B} = -I\int_{\Sigma}\vec{B} \cdot \vec{n} \,dS
\]
L'ultimo integrale è il flusso di $\vec{B}$ concatenato al circuito.
\begin{figure}[htpb]
	\centering
	

	\tikzset{every picture/.style={line width=0.75pt}} %set default line width to 0.75pt        

	\begin{tikzpicture}[x=0.75pt,y=0.75pt,yscale=-1,xscale=1]
	%uncomment if require: \path (0,300); %set diagram left start at 0, and has height of 300

	%Shape: Ellipse [id:dp09744228681476397] 
	\draw   (201,200) .. controls (201,181.77) and (234.47,167) .. (275.75,167) .. controls (317.03,167) and (350.5,181.77) .. (350.5,200) .. controls (350.5,218.23) and (317.03,233) .. (275.75,233) .. controls (234.47,233) and (201,218.23) .. (201,200) -- cycle ;
	%Curve Lines [id:da7130016733280224] 
	\draw    (201,200) .. controls (195.25,100.5) and (237.38,58.25) .. (278.56,62) .. controls (319.75,65.75) and (360,115.5) .. (350.5,200) ;
	%Curve Lines [id:da7554171546591242] 
	\draw    (233,263) .. controls (283.5,182) and (347.5,151) .. (401.5,122) ;
	\draw [shift={(307.23,179.65)}, rotate = 501.13] [fill={rgb, 255:red, 0; green, 0; blue, 0 }  ][line width=0.08]  [draw opacity=0] (10.72,-5.15) -- (0,0) -- (10.72,5.15) -- (7.12,0) -- cycle    ;
	%Curve Lines [id:da41275262296503756] 
	\draw    (285.5,80) .. controls (308,88.25) and (319.5,106.75) .. (328,129.25) ;
	%Curve Lines [id:da020627173159082357] 
	\draw    (285.5,80) .. controls (291.5,75.75) and (296,71.75) .. (301.5,69.25) ;
	%Curve Lines [id:da8534082517799799] 
	\draw    (304.5,91) .. controls (308.6,87.1) and (312.5,83.75) .. (318,81.25) ;
	%Curve Lines [id:da5576041821757687] 
	\draw    (318.3,107.4) .. controls (322.4,103.5) and (326.3,100.15) .. (331.8,97.65) ;
	%Curve Lines [id:da2810794442426652] 
	\draw    (328,129.25) .. controls (332.1,125.35) and (336.7,121.4) .. (342.2,118.9) ;

	% Text Node
	\draw (386.3,148.08) node    {$\vec{B}$};
	% Text Node
	\draw (358.3,214.08) node    {$\gamma $};


	\end{tikzpicture}
\end{figure}
\FloatBarrier
Al circuito possiamo associare un energia magnetica pari a
\[
	\boxed{U_m=-I\cdot \Phi_{\Sigma}(\vec{B})}
\]
Il flusso di $\vec{B}$, essendo il campo solenoidale, non dipende dalla particolare superficie $\Sigma$ che si appoggia su $\gamma$. La formula per $U_m$ è determinata fissato il circuito e afferma che l'energia potenziale di un circuito percorso da corrente $I$ e immerso in un campo magnetico $\vec{B}$ è eguale al prodotto cambiato di segno della corrente per il flusso di $\vec{B}$ concatenato al circuito.
Il circuito può subire traslazione, deformazione o rotazione. A seguito di una di queste trasformazioni infinitesime il flusso del campo magnetico cambierà. Avremo una variazione dell'energia $dU$. Per calcolarla assumeremo che la corrente che scorre nel circuito rimanga la stessa (condizione essenziale per la validità delle seguenti formule). Notiamo che:
\begin{align*}
	dU_m=-I\,d\Phi \qquad d\mathcal{L} &=-dU_m=-(-I\,d\Phi )= I\,d\Phi\\
	\Aboxed{d\mathcal{L} &= I\,d\Phi}
\end{align*}
Possiamo allora riscrivere la risultante delle forze come
\begin{align*}
	\vec{R} &= - \vec{\nabla} U_m \\
	&= -\vec{\nabla} (-I\,\Phi (\vec{B} )) \\
	&= \vec{\nabla}(I\,\Phi (\vec{B} )) \\
	\Aboxed{\vec{R} &= \vec{\nabla}(I\,\Phi (\vec{B} ))}
\end{align*}
Questo risultato comunica che un circuito immerso in un campo magnetico variabile, è sottoposto a una risultante che punta nella direzione di massima crescita del flusso. Notiamo che si tratta di un risultato analogo a quello ottenuto per i bipoli elettrici sottoposti ad un campo $\vec{E}$, che sono portati ad allinearsi nella direzione di $\vec{E}$:
\[
	\vec{R} = - \vec{\nabla} U_e = -\vec{\nabla} (-p\vec{E} )
\]







































\section{Campo magnetico prodotto da un circuito percorso da corrente (I legge di Laplace)}

Si dimostra che, considerato un tratto di circuito infinitesimo in cui scorre una corrente $I$, detto $dl$ il solito vettore spostamento parallelo al tratto di filo, considerato un punto $P$ dello spazio e introdotto un vettore $\vec{r}$ che punta al punto $P$, vale la legge:
\[
	\boxed{d\vec{B} (P)=\frac{\mu_0}{4\pi}\frac{I\,d\vec{l} \times \vec{u}_r}{r^2}}
\]
Nota come prima \textbf{legge di Laplace}.
\begin{figure}[htpb]
	\centering
	

	\tikzset{every picture/.style={line width=0.75pt}} %set default line width to 0.75pt        

	\begin{tikzpicture}[x=0.75pt,y=0.75pt,yscale=-1,xscale=1]
	%uncomment if require: \path (0,300); %set diagram left start at 0, and has height of 300

	%Straight Lines [id:da8727482290757451] 
	\draw    (352.75,96) -- (165.25,210) ;
	%Straight Lines [id:da8359628816789753] 
	\draw    (259,153) -- (427,153) ;
	\draw [shift={(430,153)}, rotate = 180] [fill={rgb, 255:red, 0; green, 0; blue, 0 }  ][line width=0.08]  [draw opacity=0] (10.72,-5.15) -- (0,0) -- (10.72,5.15) -- (7.12,0) -- cycle    ;
	%Straight Lines [id:da4970483138809967] 
	\draw    (238.94,151.28) -- (175.25,190) ;
	\draw [shift={(241.5,149.72)}, rotate = 148.7] [fill={rgb, 255:red, 0; green, 0; blue, 0 }  ][line width=0.08]  [draw opacity=0] (10.72,-5.15) -- (0,0) -- (10.72,5.15) -- (7.12,0) -- cycle    ;
	%Straight Lines [id:da159245909182834] 
	\draw    (295.94,126.72) -- (256,151) ;
	\draw [shift={(298.5,125.16)}, rotate = 148.7] [fill={rgb, 255:red, 0; green, 0; blue, 0 }  ][line width=0.08]  [draw opacity=0] (10.72,-5.15) -- (0,0) -- (10.72,5.15) -- (7.12,0) -- cycle    ;
	%Straight Lines [id:da43119170963545916] 
	\draw    (255,156) -- (323,156) ;
	\draw [shift={(326,156)}, rotate = 180] [fill={rgb, 255:red, 0; green, 0; blue, 0 }  ][line width=0.08]  [draw opacity=0] (10.72,-5.15) -- (0,0) -- (10.72,5.15) -- (7.12,0) -- cycle    ;
	%Shape: Circle [id:dp13708648389322597] 
	\draw  [fill={rgb, 255:red, 0; green, 0; blue, 0 }  ,fill opacity=1 ] (430.86,153) .. controls (430.86,151.82) and (431.82,150.86) .. (433,150.86) .. controls (434.18,150.86) and (435.14,151.82) .. (435.14,153) .. controls (435.14,154.18) and (434.18,155.14) .. (433,155.14) .. controls (431.82,155.14) and (430.86,154.18) .. (430.86,153) -- cycle ;

	% Text Node
	\draw (443.71,151.14) node    {$P$};
	% Text Node
	\draw (207,159) node    {$I$};
	% Text Node
	\draw (269,125) node    {$d\vec{l}$};
	% Text Node
	\draw (294,171) node    {$\vec{u}_{r}$};


	\end{tikzpicture}
\end{figure}
\FloatBarrier
Il campo magnetico elementare di un tratto infinitesimo di circuito risulta proporzionale alla corrente e inversamente proporzionale al quadrato della distanza. Tale formula ha validità generale, però soltanto come strumento matematico di calcolo; valgono infatti in proposito osservazioni analoghe a quelle fatte per la seconda legge elementare di Laplace: sperimentalmente non è possibile misurare in alcun modo il contributo di un elemento infinitesimo di filo, che a sua volta non può esistere da solo.
Per calcolare il campo magnetico di un circuito chiuso percorso da corrente si applica il principio di sovrapposizione degli effetti. Si scompone il circuito in tanti trattini infinitesimi, per ciascuno di essi si individua il vettore $\vec{r}$. Il campo complessivo sarà dato da
\[
	\vec{B} (P) = \int_{\gamma} \frac{\mu_0}{4\pi}\frac{I\,d\vec{l} \times \vec{r}}{r^3}
\]
Dove la costante vale $ \mu_0 = 4\pi \cdot 10^{-7} Tm/A $.

Per creare campi magnetici Intensi sono necessarie correnti molto elevate, visto il valore piccolo della costante.
Tale legge si può applicare anche nel caso di conduttori macroscopici immaginandoli come un fascio di fili percorsi da corrente. Infatti si considera un elemento di sezione $dS$ percorso da corrente di densità $\vec{J}(x,y,z)$. Allora si ha:
\begin{gather*}
	di=J\,dS \implies d\vec{B} (P)=\frac{\mu_0}{4\pi}\frac{I\,d\vec{l} \times \vec{r}}{r^3} = \frac{\mu_0}{4\pi} \frac{\vec{J} \times \vec{r} \,\overbrace{dS\,dl}^{d\tau}}{r^3} \\
	\boxed{\vec{B} (P)= \int_{\gamma}\frac{\mu_0}{4\pi}\frac{\vec{J} \times \vec{r}}{r^3}d\tau}
\end{gather*}







































\section{Campo magnetico prodotto da una carica in moto}

Avevamo introdotto una formula approssimata per il campo elettrico generato da un dipolo, che può essere scritto come
\[
	\vec{E} (P)= \frac{3(\vec{p} \cdot \vec{u}_r)\vec{u}_r-\vec{p}}{4\pi r^3}
\]
Nel caso del magnete, considerato un punto alla distanza $P$, detta $r$ la distanza fra l'oggetto e il punto $P$, avremo che il campo magnetico avrà una formula analoga
\[
	\vec{B} (P)=\frac{\mu_0}{4\pi}\frac{3(\vec{m} \cdot \vec{u}_r)\vec{u}_r-\vec{m}}{r^3}
\]
Se in un conduttore di corrente $\vec{J}$ consideriamo un tratto infinitesimo, possiamo calcolare il contributo al campo magnetico di questa porzione di conduttore utilizzando le solite formule. Il contributo al campo magnetico nel punto $P$ era:
\[
	d\vec{B} (P)= \frac{\mu_0}{4\pi}\frac{\vec{J} \times \vec{r}}{r^3}d\tau =\frac{\mu_0}{4\pi}\frac{\vec{J} \times \vec{u}_r}{r^2}d\tau
\]
Ricordiamo che la densità di corrente è legata alla velocità dei portatori di carica e al loro numero per unità di volume da:
\begin{gather*}
	\vec{J} =n\,q\,\vec{v}_d \qquad d\tau \cdot n=N \\
	d\vec{B} (P)= \frac{\mu_0}{4\pi}\frac{q\,n\,\vec{v}_d\times \vec{u}_r}{r^2}d\tau = \frac{\mu_0}{4\pi}\frac{q\,\vec{v}_d\times \vec{u}_r}{r^2}dN
\end{gather*}
Guardando la formula di questo modo, osserviamo che il campo magnetico è il prodotto di qualcosa per il numero di cariche presenti nell'elementino. Se dividiamo per il numero di cariche otteniamo il campo magnetico prodotto dalla singola carica.
\[
	\frac{d\vec{B}}{dN}=\frac{\mu_0}{4\pi}\frac{q\,\vec{v}_d\times \vec{u}_r}{r^2} = \vec{B}_{\text{singola carica}}
\]
Una carica elettrica in moto o in quiete genera anche un campo elettrico. Nel punto $P$ oltre al campo magnetico ci sarà anche un campo elettrico:
\[
	\vec{E} (P)_{\text{singola carica}} = \frac{q}{4\pi \varepsilon_0} \vec{u}_r
\]
Possiamo cercare una connessione fra le due formule. Possiamo vedere il campo magnetico come:
\[
	\vec{B} (P)_{\text{sc}} = \frac{\vec{v}_d\times q\vec{u}_r}{4\pi r^2 \varepsilon_0} \varepsilon_0 \mu_0  \implies  \vec{B} (P) = \mu_0 \varepsilon_0 \,\vec{v} \times \vec{E} (P) \qquad \mu_0 \varepsilon_0 = \frac{1}{c^2}
\]
Stabilendo una stretta relazione tra campo elettrico e magnetico prodotti da una carica in moto, essendo c la velocità della luce nel vuoto. La formula trovata per il campo magnetico prodotto dalla singola carica vale finché la velocità $v$ è trascurabile rispetto a, ossia finché $ (v/c)^2 \ll 1 $. Anche l'espressione del campo elettrico ha le stesse limitazioni.







































\section{Legge di Biot-Savart (campo magnetico prodotto da un filo rettilineo infinito)}

Un sistema di riferimento è dato da:
\begin{itemize}
	\item $\vec{u}_n$ versore perpendicolare al filo, diretto verso $P$
	\item $\vec{u}_{\varphi}$ versore perpendicolare al piano del foglio con verso scelto con la regola della vite destrorsa (regola della mano destra)
	\item $\vec{u}_t$ versore tangente al filo
\end{itemize}
\[
	\implies \vec{u}_{\varphi}=\vec{u}_t\times \vec{u}_n
\]
\begin{figure}[htpb]
	\centering
	

	\tikzset{every picture/.style={line width=0.75pt}} %set default line width to 0.75pt        

	\begin{tikzpicture}[x=0.75pt,y=0.75pt,yscale=-1,xscale=1]
	%uncomment if require: \path (0,300); %set diagram left start at 0, and has height of 300

	%Straight Lines [id:da400490808663277] 
	\draw    (200,29) -- (200,263) ;
	%Straight Lines [id:da45879608678252115] 
	\draw    (200,83.4) -- (200,42) ;
	\draw [shift={(200,39)}, rotate = 450] [fill={rgb, 255:red, 0; green, 0; blue, 0 }  ][line width=0.08]  [draw opacity=0] (10.72,-5.15) -- (0,0) -- (10.72,5.15) -- (7.12,0) -- cycle    ;
	%Straight Lines [id:da4761857626496162] 
	\draw  [dash pattern={on 0.84pt off 2.51pt}]  (200,119) -- (351.5,119) ;
	%Straight Lines [id:da24289172597636122] 
	\draw    (200,236) -- (349.13,120.83) ;
	\draw [shift={(351.5,119)}, rotate = 502.32] [fill={rgb, 255:red, 0; green, 0; blue, 0 }  ][line width=0.08]  [draw opacity=0] (10.72,-5.15) -- (0,0) -- (10.72,5.15) -- (7.12,0) -- cycle    ;
	%Shape: Circle [id:dp6692413966506952] 
	\draw  [fill={rgb, 255:red, 0; green, 0; blue, 0 }  ,fill opacity=1 ] (351.5,119) .. controls (351.5,117.82) and (352.46,116.86) .. (353.64,116.86) .. controls (354.83,116.86) and (355.79,117.82) .. (355.79,119) .. controls (355.79,120.18) and (354.83,121.14) .. (353.64,121.14) .. controls (352.46,121.14) and (351.5,120.18) .. (351.5,119) -- cycle ;
	%Straight Lines [id:da9483799877637442] 
	\draw    (200,148) -- (235.6,148) ;
	\draw [shift={(238.6,148)}, rotate = 180] [fill={rgb, 255:red, 0; green, 0; blue, 0 }  ][line width=0.08]  [draw opacity=0] (10.72,-5.15) -- (0,0) -- (10.72,5.15) -- (7.12,0) -- cycle    ;
	%Shape: Rectangle [id:dp2511783145268558] 
	\draw  [fill={rgb, 255:red, 0; green, 0; blue, 0 }  ,fill opacity=1 ] (198.6,228.4) -- (201.4,228.4) -- (201.4,243.6) -- (198.6,243.6) -- cycle ;
	%Shape: Arc [id:dp06720596056852024] 
	\draw  [draw opacity=0] (199.91,206) .. controls (199.94,206) and (199.97,206) .. (200,206) .. controls (209.65,206) and (218.23,210.56) .. (223.72,217.63) -- (200,236) -- cycle ; \draw   (199.91,206) .. controls (199.94,206) and (199.97,206) .. (200,206) .. controls (209.65,206) and (218.23,210.56) .. (223.72,217.63) ;
	%Straight Lines [id:da6085603141380644] 
	\draw    (180,236) -- (180,119) ;
	\draw [shift={(180,119)}, rotate = 450] [color={rgb, 255:red, 0; green, 0; blue, 0 }  ][line width=0.75]    (0,5.59) -- (0,-5.59)   ;
	\draw [shift={(180,236)}, rotate = 450] [color={rgb, 255:red, 0; green, 0; blue, 0 }  ][line width=0.75]    (0,5.59) -- (0,-5.59)   ;
	%Shape: Circle [id:dp6334725741706344] 
	\draw  [fill={rgb, 255:red, 0; green, 0; blue, 0 }  ,fill opacity=1 ] (197.86,119) .. controls (197.86,117.82) and (198.82,116.86) .. (200,116.86) .. controls (201.18,116.86) and (202.14,117.82) .. (202.14,119) .. controls (202.14,120.18) and (201.18,121.14) .. (200,121.14) .. controls (198.82,121.14) and (197.86,120.18) .. (197.86,119) -- cycle ;
	\draw   (278.3,60.7) -- (288.5,70.9)(288.5,60.7) -- (278.3,70.9) ;
	%Straight Lines [id:da5318167544382137] 
	\draw    (161,92.4) -- (161,51) ;
	\draw [shift={(161,48)}, rotate = 450] [fill={rgb, 255:red, 0; green, 0; blue, 0 }  ][line width=0.08]  [draw opacity=0] (10.72,-5.15) -- (0,0) -- (10.72,5.15) -- (7.12,0) -- cycle    ;

	% Text Node
	\draw (190.6,43.2) node    {$I$};
	% Text Node
	\draw (288,184) node    {$\vec{r}$};
	% Text Node
	\draw (366.4,117.2) node    {$P$};
	% Text Node
	\draw (252,146) node    {$\vec{u}_{n}$};
	% Text Node
	\draw (218,198.8) node    {$\vartheta $};
	% Text Node
	\draw (210.4,108.4) node    {$O$};
	% Text Node
	\draw (153.6,173.6) node    {$l< 0$};
	% Text Node
	\draw (212.8,240.4) node    {$dl$};
	% Text Node
	\draw (303.6,65.2) node    {$\vec{u}_{\varphi }$};
	% Text Node
	\draw (148.6,73.8) node    {$\vec{u}_{\vartheta }$};


	\end{tikzpicture}
\end{figure}
\FloatBarrier
Applichiamo la legge di Laplace
\begin{gather*}
	d\vec{B} (P)=\frac{\mu_0}{4\pi}\frac{I\,d\vec{l} \times \vec{r}}{r^3} = \frac{\mu_0}{4\pi}\frac{I\,dl\,\sin \vartheta}{r^2}\vec{u}_{\varphi} \\
	\vec{B} (P)= \int_{\text{filo}}d\vec{B} (P) = \int_{\text{filo}} \frac{\mu_0}{4\pi}\frac{I\,dl\,\sin \vartheta}{r^2}\vec{u}_{\varphi}
\end{gather*}
Iniziamo a calcolare le solite relazioni per esprimere tutto in funzione degli angoli.
\begin{gather*}
	R=r\,\sin \vartheta \implies r=\frac{R}{\sin \vartheta} \qquad l= r\,\cos \vartheta = -R\,\text{cotan} \vartheta
\end{gather*}
Esprimiamo $ dl $ come
\begin{gather*}
	dl=\frac{dl}{d\vartheta}d\vartheta =\frac{d}{d\vartheta}(-R\,\text{cotan}\vartheta  )d\vartheta = -\left( -\frac{R}{\sin^2 \vartheta} \right) d\vartheta = \frac{R}{\sin^2 \vartheta} d\vartheta \\
	d\vec{B} (P)= \frac{\mu_0}{4\pi}\frac{I\, \overbrace{\frac{R}{\sin^2 \vartheta}d\vartheta}^{dl} \sin \vartheta}{\underbrace{\frac{R^2}{\sin^2 \vartheta}}_{r^2}}\vec{u}_{\varphi} = \frac{\mu_0}{4\pi}\frac{I\,\sin \vartheta}{R}\vec{u}_{\varphi}d\vartheta
\end{gather*}
Possiamo ora procedere con l'integrazione finale tra gli estremi dell'angolo
\begin{equation*}
	\begin{aligned}
		\vec{B} (P)= \int_0^{\pi} \frac{\mu_0 \,I\,\sin \vartheta}{4\pi R}\vec{u}_{\varphi}d\vartheta &= \frac{\mu_0 \,I\,\vec{u}_{\varphi}}{4\pi R}\int_0^{\pi} \sin \vartheta d\vartheta \\
		&= \frac{\mu_0 \,I\,\vec{u}_{\varphi}}{4\pi R} \underbrace{[-\cos \vartheta ]_0^{\pi}}_2 = \frac{\mu_0 I}{2\pi R}\vec{u}_{\varphi}
	\end{aligned}
\end{equation*}
Nel piano mediano il campo magnetico $\vec{B}$ è costante su ogni circonferenza di raggio $R$ ed è tangente a tale circonferenza. Detto $ \vec{u}_{\varphi} $ il versore tangente alla circonferenza e orientato rispetto al verso della corrente secondo la regola della vite, possiamo scrivere:
\[
	\vec{B} (P)=\frac{\mu_0 I}{2\pi R}\vec{u}_{\varphi}
\]
Tale risultato è noto come \textbf{legge di Biot-Savart} e afferma che il campo magnetico di un filo rettilineo indefinito dipende solo dalla distanza dal filo, in modo inversamente proporzionale, le sue linee sono circonferenze concentriche al filo.







































\section{Interazione tra circuiti percorsi da corrente}

Calcoliamo ora la forza tra circuiti percorsi da corrente, partendo dalle leggi elementari di Laplace. Scomponiamo entrambi i circuiti in tanti elementi. Chiamiamo $r$ la distanza fra due elementi generici $dl_1$ e $dl_2$ e individuiamo con $\vec{u}_1$ il versore che punta da $dl_1$ a $dl_2$ e $\vec{u}_2$ il versore che punta in verso opposto.
\begin{figure}[htpb]
	\centering
	

	\tikzset{every picture/.style={line width=0.75pt}} %set default line width to 0.75pt        

	\begin{tikzpicture}[x=0.75pt,y=0.75pt,yscale=-1,xscale=1]
	%uncomment if require: \path (0,300); %set diagram left start at 0, and has height of 300

	%Shape: Ellipse [id:dp5039867707891208] 
	\draw   (85.33,188.11) .. controls (64.95,159) and (85.8,109.24) .. (131.89,76.96) .. controls (177.99,44.68) and (231.88,42.11) .. (252.26,71.22) .. controls (272.65,100.33) and (251.8,150.1) .. (205.71,182.37) .. controls (159.61,214.65) and (105.72,217.22) .. (85.33,188.11) -- cycle ;
	%Shape: Ellipse [id:dp4465357244059345] 
	\draw   (511.39,210.14) .. controls (486.81,235.81) and (433.94,225.06) .. (393.3,186.14) .. controls (352.65,147.22) and (339.63,94.86) .. (364.21,69.19) .. controls (388.79,43.53) and (441.66,54.27) .. (482.3,93.19) .. controls (522.95,132.11) and (535.97,184.47) .. (511.39,210.14) -- cycle ;
	%Shape: Rectangle [id:dp4793662673196728] 
	\draw  [fill={rgb, 255:red, 0; green, 0; blue, 0 }  ,fill opacity=1 ] (216.79,171.86) -- (218.45,174.12) -- (206.21,183.14) -- (204.55,180.88) -- cycle ;
	%Straight Lines [id:da36718365960822563] 
	\draw    (211.5,177.5) -- (385,178) ;
	\draw   (127.85,86.51) -- (115.82,89.67) -- (120.51,78.15) ;
	\draw   (480.42,99.36) -- (476.61,87.53) -- (488.36,91.58) ;
	%Shape: Rectangle [id:dp7299100063793511] 
	\draw  [fill={rgb, 255:red, 0; green, 0; blue, 0 }  ,fill opacity=1 ] (378.87,173.3) -- (380.94,171.42) -- (391.13,182.7) -- (389.06,184.58) -- cycle ;
	%Straight Lines [id:da6851940673219525] 
	\draw    (208.5,181.5) -- (240.5,181.5) ;
	\draw [shift={(243.5,181.5)}, rotate = 180] [fill={rgb, 255:red, 0; green, 0; blue, 0 }  ][line width=0.08]  [draw opacity=0] (10.72,-5.15) -- (0,0) -- (10.72,5.15) -- (7.12,0) -- cycle    ;
	%Straight Lines [id:da6416264107401324] 
	\draw    (354.5,181.5) -- (386.5,181.5) ;
	\draw [shift={(351.5,181.5)}, rotate = 0] [fill={rgb, 255:red, 0; green, 0; blue, 0 }  ][line width=0.08]  [draw opacity=0] (10.72,-5.15) -- (0,0) -- (10.72,5.15) -- (7.12,0) -- cycle    ;
	%Straight Lines [id:da46776015006269556] 
	\draw    (181.5,188.75) -- (209.11,167.81) ;
	\draw [shift={(211.5,166)}, rotate = 502.83] [fill={rgb, 255:red, 0; green, 0; blue, 0 }  ][line width=0.08]  [draw opacity=0] (10.72,-5.15) -- (0,0) -- (10.72,5.15) -- (7.12,0) -- cycle    ;
	%Straight Lines [id:da29499082211518535] 
	\draw    (211.5,166) -- (190.56,138.39) ;
	\draw [shift={(188.75,136)}, rotate = 412.83000000000004] [fill={rgb, 255:red, 0; green, 0; blue, 0 }  ][line width=0.08]  [draw opacity=0] (10.72,-5.15) -- (0,0) -- (10.72,5.15) -- (7.12,0) -- cycle    ;
	%Straight Lines [id:da5556095860689843] 
	\draw    (409.83,193.18) -- (388.5,172.75) ;
	\draw [shift={(412,195.25)}, rotate = 223.75] [fill={rgb, 255:red, 0; green, 0; blue, 0 }  ][line width=0.08]  [draw opacity=0] (10.72,-5.15) -- (0,0) -- (10.72,5.15) -- (7.12,0) -- cycle    ;
	%Straight Lines [id:da48897677161774156] 
	\draw    (388.5,172.75) -- (408.93,151.42) ;
	\draw [shift={(411,149.25)}, rotate = 493.75] [fill={rgb, 255:red, 0; green, 0; blue, 0 }  ][line width=0.08]  [draw opacity=0] (10.72,-5.15) -- (0,0) -- (10.72,5.15) -- (7.12,0) -- cycle    ;

	% Text Node
	\draw (304.5,165) node    {$r_{12}$};
	% Text Node
	\draw (226,195) node    {$\vec{u}_{1}$};
	% Text Node
	\draw (504,86.5) node    {$\gamma _{2}$};
	% Text Node
	\draw (102.5,80) node    {$\gamma _{1}$};
	% Text Node
	\draw (372.5,196) node    {$\vec{u}_{2}$};
	% Text Node
	\draw (165,181.5) node    {$d\vec{l}_{1}$};
	% Text Node
	\draw (199,121) node    {$d\vec{F}_{12}$};
	% Text Node
	\draw (428.5,188) node    {$d\vec{l}_{2}$};
	% Text Node
	\draw (431,138) node    {$d\vec{F}_{21}$};


	\end{tikzpicture}
\end{figure}
\FloatBarrier
Utilizzando la seconda legge di Laplace, sappiamo che la forza $d\vec{F}_{21}$ agente sull'elemento $dl_2$ a causa del campo magnetico $d\vec{B}_1$ prodotto da $dl_1$ nel posto in cui si trova $dl_2$ è
\[
	d\vec{F}_{21}=I_2d\vec{l}_2\times d\vec{B}_1
\]
Per la prima legge di Laplace si ha:
\begin{gather*}
	d\vec{B}_1=\frac{\mu_0}{4\pi}\frac{I_1d\vec{l}_1\times \vec{u}_1}{r^2} \\
	d\vec{F}_{21} = I_2d\vec{l}_2\times \left[ \frac{\mu_0}{4\pi}\frac{I_1d\vec{l}_1\times \vec{u}_1}{r^2}  \right] =\frac{\mu_0 I_1I_2}{4\pi r^2}d\vec{l}_2\times (d\vec{l}_1\times \vec{u}_1) \\
	d\vec{F}_{12} = I_1d\vec{l}_1\times \left[ \frac{\mu_0}{4\pi}\frac{I_2d\vec{l}_2\times \vec{u}_2}{r^2}  \right] = \frac{\mu_0 I_1I_2}{4\pi r^2}d\vec{l}_1\times (d\vec{l}_2\times \vec{u}_2 )
\end{gather*}
Quindi $ d\vec{F}_{21} \neq d\vec{F}_{12}$. Sembra che esse non rispettino il terzo principio della dinamica. Ma le due formule di Laplace non vanno prese come formule che rappresentano interazioni di tipo fondamentale, vanno prese come ausili matematici e integrate su tutto il circuito. A noi interessano per il principio azione reazione le forze totali. Le leggi elementari di Laplace infatti non si applicano a sistemi fisicamente realizzabili.

La forza risultante fra i due circuiti si ottiene con una doppia integrazione estesa ai due circuiti che tiene conto di tutte le coppie di elementi $dl_1$ e $dl_2$ tra i quali si esercitano le forze $d\vec{F}$:
\begin{align*}
	\left.\begin{array}{ll}
		\vec{F}_{21} = \oint_{c1} \oint_{c2} d\vec{F}_{21} \\
		\vec{F}_{12} = \oint_{c2} \oint_{c1} d\vec{F}_{12}
	\end{array}\right\}
		\implies \vec{F}_{21} = \vec{F}_{12}
\end{align*}
Si può verificare che la forza tra i due circuiti obbedisce al principio di azione reazione.







































\section{Coefficiente di mutua e auto-induzione}

Il campo magnetico $B_1$ generato da un circuito percorso dalla corrente i1 determina un certo flusso magnetico attraverso un qualsiasi altro circuito presente nella regione in cui agisce $B_1$.
\begin{figure}[htpb]
	\centering
	

	% Pattern Info
	 
	\tikzset{
	pattern size/.store in=\mcSize, 
	pattern size = 5pt,
	pattern thickness/.store in=\mcThickness, 
	pattern thickness = 0.3pt,
	pattern radius/.store in=\mcRadius, 
	pattern radius = 1pt}
	\makeatletter
	\pgfutil@ifundefined{pgf@pattern@name@_x03l2yg0j}{
	\pgfdeclarepatternformonly[\mcThickness,\mcSize]{_x03l2yg0j}
	{\pgfqpoint{0pt}{0pt}}
	{\pgfpoint{\mcSize+\mcThickness}{\mcSize+\mcThickness}}
	{\pgfpoint{\mcSize}{\mcSize}}
	{
	\pgfsetcolor{\tikz@pattern@color}
	\pgfsetlinewidth{\mcThickness}
	\pgfpathmoveto{\pgfqpoint{0pt}{0pt}}
	\pgfpathlineto{\pgfpoint{\mcSize+\mcThickness}{\mcSize+\mcThickness}}
	\pgfusepath{stroke}
	}}
	\makeatother

	% Pattern Info
	 
	\tikzset{
	pattern size/.store in=\mcSize, 
	pattern size = 5pt,
	pattern thickness/.store in=\mcThickness, 
	pattern thickness = 0.3pt,
	pattern radius/.store in=\mcRadius, 
	pattern radius = 1pt}
	\makeatletter
	\pgfutil@ifundefined{pgf@pattern@name@_191ys6rbz}{
	\pgfdeclarepatternformonly[\mcThickness,\mcSize]{_191ys6rbz}
	{\pgfqpoint{0pt}{0pt}}
	{\pgfpoint{\mcSize+\mcThickness}{\mcSize+\mcThickness}}
	{\pgfpoint{\mcSize}{\mcSize}}
	{
	\pgfsetcolor{\tikz@pattern@color}
	\pgfsetlinewidth{\mcThickness}
	\pgfpathmoveto{\pgfqpoint{0pt}{0pt}}
	\pgfpathlineto{\pgfpoint{\mcSize+\mcThickness}{\mcSize+\mcThickness}}
	\pgfusepath{stroke}
	}}
	\makeatother
	\tikzset{every picture/.style={line width=0.75pt}} %set default line width to 0.75pt        

	\begin{tikzpicture}[x=0.75pt,y=0.75pt,yscale=-1,xscale=1]
	%uncomment if require: \path (0,300); %set diagram left start at 0, and has height of 300

	%Shape: Ellipse [id:dp5421393934180025] 
	\draw  [pattern=_x03l2yg0j,pattern size=6pt,pattern thickness=0.75pt,pattern radius=0pt, pattern color={rgb, 255:red, 222; green, 222; blue, 222}] (400.48,243.91) .. controls (371.1,223.92) and (372.94,170) .. (404.6,123.47) .. controls (436.25,76.94) and (485.73,55.43) .. (515.12,75.42) .. controls (544.5,95.41) and (542.65,149.34) .. (511,195.86) .. controls (479.34,242.39) and (429.86,263.9) .. (400.48,243.91) -- cycle ;
	%Shape: Ellipse [id:dp4806549724492446] 
	\draw  [pattern=_191ys6rbz,pattern size=6pt,pattern thickness=0.75pt,pattern radius=0pt, pattern color={rgb, 255:red, 222; green, 222; blue, 222}] (105.33,208.11) .. controls (84.95,179) and (105.8,129.24) .. (151.89,96.96) .. controls (197.99,64.68) and (251.88,62.11) .. (272.26,91.22) .. controls (292.65,120.33) and (271.8,170.1) .. (225.71,202.37) .. controls (179.61,234.65) and (125.72,237.22) .. (105.33,208.11) -- cycle ;
	\draw   (147.85,106.51) -- (135.82,109.67) -- (140.51,98.15) ;
	\draw   (504.6,195.03) -- (515.64,189.31) -- (513.6,201.57) ;
	%Straight Lines [id:da3399806485135666] 
	\draw    (457.8,159.67) -- (369.48,99.58) ;
	\draw [shift={(367,97.89)}, rotate = 394.23] [fill={rgb, 255:red, 0; green, 0; blue, 0 }  ][line width=0.08]  [draw opacity=0] (10.72,-5.15) -- (0,0) -- (10.72,5.15) -- (7.12,0) -- cycle    ;
	%Straight Lines [id:da06607489606794448] 
	\draw    (188.8,149.67) -- (126.72,61.01) ;
	\draw [shift={(125,58.55)}, rotate = 415] [fill={rgb, 255:red, 0; green, 0; blue, 0 }  ][line width=0.08]  [draw opacity=0] (10.72,-5.15) -- (0,0) -- (10.72,5.15) -- (7.12,0) -- cycle    ;

	% Text Node
	\draw (112,52) node    {$\vec{n}_{1}$};
	% Text Node
	\draw (551,161.5) node    {$\gamma _{2}$};
	% Text Node
	\draw (84.5,175) node    {$\gamma _{1}$};
	% Text Node
	\draw (230,219.5) node    {$d\vec{l}_{1}$};
	% Text Node
	\draw (374.5,236) node    {$d\vec{l}_{2}$};
	% Text Node
	\draw (354,92) node    {$\vec{n}_{2}$};
	% Text Node
	\draw (249,98.5) node    {$\Sigma _{1}$};
	% Text Node
	\draw (496,89.5) node    {$\Sigma _{2}$};
	% Text Node
	\draw (108.5,118) node    {$I_{1}$};
	% Text Node
	\draw (517.5,217) node    {$I_{2}$};


	\end{tikzpicture}
\end{figure}
\FloatBarrier
Si ottiene
\begin{gather*}
	\vec{B}_1(P)= \oint_{\gamma_1} \frac{\mu_0}{4\pi}\frac{I_1d\vec{l}_1\times \vec{u}_r}{r^2} \\
\end{gather*}
Calcoliamo il flusso di questo campo concatenato a un'altro circuito.
\begin{equation*}
	\begin{aligned}
		\Phi_{\Sigma_2}=\int_{\Sigma_2} \vec{B}_1(P)\cdot \vec{n}_2\,dS &= \int_{\Sigma_2} \left[ \oint_{\gamma_1} \frac{\mu_0}{4\pi}\frac{I_1d\vec{l}_1\times \vec{u}_r}{r^2} \right] \cdot \vec{n}_2 \,dS \\
		&= I_1 \underbrace{\int_{\Sigma_2} \left[ \oint_{\gamma_1} \frac{\mu_0}{4\pi}\frac{d\vec{l}_1\times \vec{u}_r}{r^2} \right] \cdot \vec{n}_2 \,dS}_{M_{21}}  = I_1\,M_{21}
	\end{aligned}
\end{equation*}
Dove $\Sigma_2$ è una qualsiasi superficie che si appoggia sul secondo circuito, $r$ è la distanza dall'elemento $d\vec{l}_1$ del primo circuito all'elemento di area $dS$ e $\vec{u}_r$ è il versore della direzione orientata $r$.
L'espressione del flusso può essere riscritta sinteticamente come:
\[
	\Phi_{21}=M_{21}I_1 \qquad \Phi_{12}=M_{12}I_2
\]
Conglobando in $M_{12}$ tutti i fattori geometrici e l'eventuale dipendenza dalle proprietà magnetiche del mezzo in cui sono immersi i circuiti (vale lo stesso discorso per il circuito $2$ rispetto all'$1$). Mi dice quanto vale il flusso attraverso il circuito $2$ a causa del campo prodotto dal circuito $1$ e prende il nome di \textbf{coefficiente di mutua induzione}. Si può dimostrare che i due coefficienti $M_{12}$ e $M_{21}$ sono uguali.
\[
	M_{21}=M_{12}
\]
Il campo magnetico generato da un circuito produce un flusso anche attraverso il circuito stesso, detto \textbf{autoflusso del circuito}, che si scrive:
\begin{gather*}
	\vec{B} = \oint_{\gamma} \frac{\mu_0}{4\pi}\frac{I\,d\vec{l} \times \vec{u}_r}{r^2}
\end{gather*}
\begin{equation*}
	\begin{aligned}
		\Phi_{\Sigma}(\vec{B} ) = \int_{\Sigma}\vec{B} \cdot \vec{n} \,dS &= \int_{\Sigma} \left[ \oint_{\gamma} \frac{\mu_0}{4\pi}\frac{I\,d\vec{l} \times \vec{u}_r}{r^2} \right]   \cdot \vec{n} \,dS \\
		&= I \underbrace{\int_{\Sigma} \left[ \oint_{\gamma} \frac{\mu_0}{4\pi}\frac{d\vec{l} \times \vec{u}_r}{r^2} \right] \cdot \vec{n} \,dS}_L = I\,L
	\end{aligned}
\end{equation*}
Dove il fattore $L$ si chiama \textbf{coefficiente di autoinduzione} del circuito.
\begin{figure}[htpb]
	\centering
	

	% Pattern Info
	 
	\tikzset{
	pattern size/.store in=\mcSize, 
	pattern size = 5pt,
	pattern thickness/.store in=\mcThickness, 
	pattern thickness = 0.3pt,
	pattern radius/.store in=\mcRadius, 
	pattern radius = 1pt}
	\makeatletter
	\pgfutil@ifundefined{pgf@pattern@name@_t68lb0l9b}{
	\pgfdeclarepatternformonly[\mcThickness,\mcSize]{_t68lb0l9b}
	{\pgfqpoint{0pt}{0pt}}
	{\pgfpoint{\mcSize+\mcThickness}{\mcSize+\mcThickness}}
	{\pgfpoint{\mcSize}{\mcSize}}
	{
	\pgfsetcolor{\tikz@pattern@color}
	\pgfsetlinewidth{\mcThickness}
	\pgfpathmoveto{\pgfqpoint{0pt}{0pt}}
	\pgfpathlineto{\pgfpoint{\mcSize+\mcThickness}{\mcSize+\mcThickness}}
	\pgfusepath{stroke}
	}}
	\makeatother
	\tikzset{every picture/.style={line width=0.75pt}} %set default line width to 0.75pt        

	\begin{tikzpicture}[x=0.75pt,y=0.75pt,yscale=-1,xscale=1]
	%uncomment if require: \path (0,300); %set diagram left start at 0, and has height of 300

	%Shape: Ellipse [id:dp9059340509175016] 
	\draw  [pattern=_t68lb0l9b,pattern size=6pt,pattern thickness=0.75pt,pattern radius=0pt, pattern color={rgb, 255:red, 222; green, 222; blue, 222}] (174.63,160.83) .. controls (174.71,125.3) and (220.4,96.6) .. (276.68,96.74) .. controls (332.95,96.87) and (378.5,125.79) .. (378.41,161.33) .. controls (378.33,196.87) and (332.64,225.56) .. (276.36,225.43) .. controls (220.09,225.29) and (174.54,196.37) .. (174.63,160.83) -- cycle ;
	\draw   (273.47,219.66) -- (284.39,225.6) -- (273.09,230.78) ;
	%Straight Lines [id:da417258539993012] 
	\draw    (276.52,161.08) -- (276.78,54.26) ;
	\draw [shift={(276.79,51.26)}, rotate = 450.14] [fill={rgb, 255:red, 0; green, 0; blue, 0 }  ][line width=0.08]  [draw opacity=0] (10.72,-5.15) -- (0,0) -- (10.72,5.15) -- (7.12,0) -- cycle    ;
	%Curve Lines [id:da8196902834456317] 
	\draw    (182,258) .. controls (232.5,177) and (245.5,100) .. (201.5,47) ;
	\draw [shift={(225.62,152.9)}, rotate = 460.89] [fill={rgb, 255:red, 0; green, 0; blue, 0 }  ][line width=0.08]  [draw opacity=0] (10.72,-5.15) -- (0,0) -- (10.72,5.15) -- (7.12,0) -- cycle    ;
	%Curve Lines [id:da6478763752060417] 
	\draw    (330,270) .. controls (298.5,177) and (295.5,101) .. (366.5,58) ;
	\draw [shift={(309.05,154.01)}, rotate = 452.21] [fill={rgb, 255:red, 0; green, 0; blue, 0 }  ][line width=0.08]  [draw opacity=0] (10.72,-5.15) -- (0,0) -- (10.72,5.15) -- (7.12,0) -- cycle    ;

	% Text Node
	\draw (388,174.5) node    {$\gamma $};
	% Text Node
	\draw (348.5,221) node    {$d\vec{l}$};
	% Text Node
	\draw (295,55) node    {$\vec{n}$};
	% Text Node
	\draw (356,154.5) node    {$\Sigma $};
	% Text Node
	\draw (265.5,239) node    {$I$};
	% Text Node
	\draw (381.3,52.08) node    {$\vec{B}$};


	\end{tikzpicture}
\end{figure}
\FloatBarrier
Esso dipende dalla forma del circuito e dalle proprietà magnetiche del mezzo ed è costante se il circuito è indeformabile. A differenza di quanto accade per $M$, $L$ è sempre positivo.
Quando facciamo scorrere corrente per il circuito, esso ha un energia potenziale. Accendendo il circuito stiamo fornendo energia non legata al fatto che c'è una dissipazione, ma al fatto che dobbiamo creare questo campo magnetico e tale energia dipende dalla corrente che facciamo scorrere. Per creare un campo magnetico, come succedeva con $\vec{E}$, dobbiamo spendere energia.
\[
	[M]=[L]=\frac{[\Phi ]}{[I]} = \left( \frac{Wb}{A} \right) =\left( \frac{V\,S}{A} \right) = (\Omega\,S) = H  \qquad \text{Henry}
\]
Si ribattezza una nuova unità di misura, Henry, $H$.







































\section{Legge di Ampere, concatenazione, ordine di concatenazione, IV Equazione di Maxwell (solo in regime stazionario)}

Riprendendo la relazione
\[
	\vec{B} (P) = \frac{\mu_0 I}{2\pi R}\vec{u}_{\varphi}
\]
Consideriamo una linea chiusa $\gamma$ orientata, concatenata alla corrente $I$. `concatenata' significa che la corrente attraversa una qualunque superficie delimitata da $\gamma$.
\begin{figure}[htpb]
	\centering
	

	\tikzset{every picture/.style={line width=0.75pt}} %set default line width to 0.75pt        

	\begin{tikzpicture}[x=0.75pt,y=0.75pt,yscale=-1,xscale=1]
	%uncomment if require: \path (0,300); %set diagram left start at 0, and has height of 300

	%Straight Lines [id:da9211068985865756] 
	\draw    (220,49) -- (220,283) ;
	%Straight Lines [id:da36112272009272406] 
	\draw    (220,103.4) -- (220,62) ;
	\draw [shift={(220,59)}, rotate = 450] [fill={rgb, 255:red, 0; green, 0; blue, 0 }  ][line width=0.08]  [draw opacity=0] (10.72,-5.15) -- (0,0) -- (10.72,5.15) -- (7.12,0) -- cycle    ;
	%Straight Lines [id:da06976981758467105] 
	\draw    (220,109) -- (368.5,109) ;
	\draw [shift={(371.5,109)}, rotate = 180] [fill={rgb, 255:red, 0; green, 0; blue, 0 }  ][line width=0.08]  [draw opacity=0] (10.72,-5.15) -- (0,0) -- (10.72,5.15) -- (7.12,0) -- cycle    ;
	%Straight Lines [id:da477050807121032] 
	\draw    (220,138) -- (255.6,138) ;
	\draw [shift={(258.6,138)}, rotate = 180] [fill={rgb, 255:red, 0; green, 0; blue, 0 }  ][line width=0.08]  [draw opacity=0] (10.72,-5.15) -- (0,0) -- (10.72,5.15) -- (7.12,0) -- cycle    ;
	\draw   (277.1,75.5) -- (287.3,85.7)(287.3,75.5) -- (277.1,85.7) ;
	%Curve Lines [id:da009775526404201873] 
	\draw    (159,173.68) .. controls (242,141.83) and (330,211.6) .. (278,235.87) ;
	\draw [shift={(248.81,173.68)}, rotate = 19.39] [fill={rgb, 255:red, 0; green, 0; blue, 0 }  ][line width=0.08]  [draw opacity=0] (10.72,-5.15) -- (0,0) -- (10.72,5.15) -- (7.12,0) -- cycle    ;
	%Curve Lines [id:da9385892682251267] 
	\draw    (278,235.87) .. controls (196,274.55) and (106,196.43) .. (159,173.68) ;
	\draw [shift={(187.37,238.04)}, rotate = 200.57] [fill={rgb, 255:red, 0; green, 0; blue, 0 }  ][line width=0.08]  [draw opacity=0] (10.72,-5.15) -- (0,0) -- (10.72,5.15) -- (7.12,0) -- cycle    ;
	%Straight Lines [id:da9153580905085494] 
	\draw    (220.2,201.2) -- (278,235.87) ;
	%Straight Lines [id:da722540916164276] 
	\draw    (220.2,201.2) -- (250.2,244.8) ;
	%Curve Lines [id:da7608604455030328] 
	\draw    (235.2,223) .. controls (237.4,221.6) and (241.4,219.6) .. (245,216.4) ;

	% Text Node
	\draw (210.6,63.2) node    {$I$};
	% Text Node
	\draw (386.4,107.2) node    {$P$};
	% Text Node
	\draw (272,136) node    {$\vec{u}_{n}$};
	% Text Node
	\draw (302.4,80) node    {$\vec{u}_{\varphi }$};
	% Text Node
	\draw (288.4,179.2) node    {$\gamma $};
	% Text Node
	\draw (272.4,250) node    {$d\varphi $};


	\end{tikzpicture}
\end{figure}
\FloatBarrier
Introduciamo un sistema di riferimento in coordinate cilindriche in cui ogni punto è determinato da $ P=P(z,R,\varphi) $.
Abbiamo che
\[
	d\vec{l} = dz\;\vec{u}_z + dR\;\vec{u}_n + Rd\varphi \; \vec{u}_{\varphi}
\]
Calcoliamo la circuitazione di $\vec{B}$
\[
	\oint_{\gamma} \vec{B} \cdot d\vec{l} =\oint_{\gamma} \frac{\mu_0 I}{2\pi R}\vec{u}_{\varphi} \; [dz\;\vec{u}_z + dR\;\vec{u}_n + Rd\varphi \; \vec{u}_{\varphi}] = \oint_{\gamma} \frac{\mu_0 I}{2\pi R}Rd\varphi
\]
Dove abbiamo usato le proprietà del prodotto scalare e la formula per il campo magnetico prodotto da un filo percorso da corrente. Quindi
\[
	\oint_{\gamma} \vec{B} \cdot d\vec{l} = \frac{\mu_0 I}{2\pi}\oint_{\gamma} d\varphi = \frac{\mu_0 I}{2\pi} \cdot 2\pi = \mu_0 I
\]
Introduciamo anche il concetto di \emph{ordine di concatenazione}, nel senso che una linea $\gamma$ può avvolgersi $n$ volte attorno al filo percorso da corrente.
\begin{figure}[htpb]
	\centering
	

	\tikzset{every picture/.style={line width=0.75pt}} %set default line width to 0.75pt        

	\begin{tikzpicture}[x=0.75pt,y=0.75pt,yscale=-1,xscale=1]
	%uncomment if require: \path (0,300); %set diagram left start at 0, and has height of 300

	%Straight Lines [id:da7476669646127272] 
	\draw    (252,49) -- (252,242) ;
	%Straight Lines [id:da29768359794142274] 
	\draw    (252,93.87) -- (252,60.25) ;
	\draw [shift={(252,57.25)}, rotate = 450] [fill={rgb, 255:red, 0; green, 0; blue, 0 }  ][line width=0.08]  [draw opacity=0] (10.72,-5.15) -- (0,0) -- (10.72,5.15) -- (7.12,0) -- cycle    ;
	%Curve Lines [id:da3466057001566656] 
	\draw    (193.5,96) .. controls (276.5,64.15) and (356,119.6) .. (304,143.87) ;
	\draw [shift={(277.71,90.5)}, rotate = 13.28] [fill={rgb, 255:red, 0; green, 0; blue, 0 }  ][line width=0.08]  [draw opacity=0] (10.72,-5.15) -- (0,0) -- (10.72,5.15) -- (7.12,0) -- cycle    ;
	%Curve Lines [id:da9829470547669434] 
	\draw    (208.5,145) .. controls (334.5,102) and (388.5,228) .. (196.5,211) ;
	%Curve Lines [id:da1539438051158717] 
	\draw    (304,143.87) .. controls (222,182.55) and (155.5,167.75) .. (208.5,145) ;
	\draw [shift={(234.37,165.93)}, rotate = 170.42] [fill={rgb, 255:red, 0; green, 0; blue, 0 }  ][line width=0.08]  [draw opacity=0] (10.72,-5.15) -- (0,0) -- (10.72,5.15) -- (7.12,0) -- cycle    ;
	%Curve Lines [id:da9265603983516404] 
	\draw    (193.5,96) .. controls (122.5,125) and (144.5,210) .. (196.5,211) ;

	% Text Node
	\draw (242.6,60.71) node    {$I$};
	% Text Node
	\draw (364.6,142.71) node    {$n=2$};


	\end{tikzpicture}
\end{figure}
\FloatBarrier
In tal caso
\[
	\oint_{\gamma} \vec{B} \cdot d\vec{l} = n\;\mu_0 I
\]
Precisiamo il segno da usare per le correnti
\begin{itemize}
	\item $ I>0 $ se concorde al verso di $\gamma$
	\item $ I<0 $ se discorde al verso di $\gamma$
\end{itemize}
Se la corrente non risulta concatenata a $\gamma$ i contributi mentre si integra l'angolo $\varphi$ si cancellano.
\begin{figure}[htpb]
	\centering
	

	\tikzset{every picture/.style={line width=0.75pt}} %set default line width to 0.75pt        

	\begin{tikzpicture}[x=0.75pt,y=0.75pt,yscale=-1,xscale=1]
	%uncomment if require: \path (0,300); %set diagram left start at 0, and has height of 300

	%Straight Lines [id:da40927297486999903] 
	\draw    (225,69) -- (225,262) ;
	%Straight Lines [id:da8720550292630895] 
	\draw    (225,113.87) -- (225,80.25) ;
	\draw [shift={(225,77.25)}, rotate = 450] [fill={rgb, 255:red, 0; green, 0; blue, 0 }  ][line width=0.08]  [draw opacity=0] (10.72,-5.15) -- (0,0) -- (10.72,5.15) -- (7.12,0) -- cycle    ;
	%Shape: Polygon Curved [id:ds3996198548298875] 
	\draw   (308.33,142) .. controls (283.83,112) and (418.33,122) .. (398.33,142) .. controls (378.33,162) and (378.33,172) .. (398.33,202) .. controls (418.33,232) and (348.83,227) .. (308.33,202) .. controls (267.83,177) and (332.83,172) .. (308.33,142) -- cycle ;
	%Straight Lines [id:da21112185649434645] 
	\draw    (308.33,129.67) -- (225,120.67) ;
	%Straight Lines [id:da4371563579602411] 
	\draw    (308.33,142) -- (225,120.67) ;
	%Straight Lines [id:da6581159860746844] 
	\draw    (331,213.33) -- (225,220.67) ;
	%Straight Lines [id:da7631853339077679] 
	\draw    (354.33,220) -- (225,220.67) ;
	%Shape: Arc [id:dp6176391255763463] 
	\draw  [draw opacity=0] (254.87,123.44) .. controls (254.72,125.1) and (254.43,126.72) .. (254.02,128.29) -- (225,120.67) -- cycle ; \draw   (254.87,123.44) .. controls (254.72,125.1) and (254.43,126.72) .. (254.02,128.29) ;
	%Shape: Arc [id:dp7056007805282605] 
	\draw  [draw opacity=0] (254.91,218.36) .. controls (254.97,219.12) and (255,219.89) .. (255,220.67) .. controls (255,220.8) and (255,220.94) .. (255,221.07) -- (225,220.67) -- cycle ; \draw   (254.91,218.36) .. controls (254.97,219.12) and (255,219.89) .. (255,220.67) .. controls (255,220.8) and (255,220.94) .. (255,221.07) ;

	% Text Node
	\draw (215.6,80.71) node    {$I$};
	% Text Node
	\draw (328.93,138.71) node    {$d\vec{l}$};
	% Text Node
	\draw (348.93,200.05) node    {$d\vec{l}$};
	% Text Node
	\draw (410.27,134.05) node    {$\gamma $};


	\end{tikzpicture}
\end{figure}
\FloatBarrier
Dal momento che $ \vec{B}  $ gode del principio di sovrapposizione degli effetti, si può considerare la somma delle correnti concatenate per generalizzare il concetto.
\begin{align*}
	\oint_{\gamma} \vec{B}_{\text{tot}}\cdot d\vec{l} &=  \oint_{\gamma} (\Sigma_i\vec{B}_i  ) \cdot d\vec{l} \\
	&= \Sigma_i \oint_{\gamma} \vec{B}_i\cdot d\vec{l} \\
	&= \Sigma_i \; \mu_0 I_i^{\text{concatenata a}\gamma} \\
	&= \mu_0 I_{\text{tot}}^{\text{concatenata a}\gamma}
\end{align*}
Possiamo quindi enunciare la legge di Ampere
\[
	\boxed{\oint_{\gamma} \vec{B} \cdot d\vec{l} = \mu_0 I^{\text{concatenata a}\gamma}_{\text{tot}}}
\]
Vale anche se il circuito non è rettilineo.




















\subsection{IV Equazione di Maxwell}

Consideriamo una situazione in regime stazionario, quindi con la corrente costante in tutto il circuito.

Definiamo $\vec{J}$ anche all'esterno dei fili, ossia $\vec{J}=0$ in assenza di corrente.
\begin{figure}[htpb]
	\centering
	

	\tikzset{every picture/.style={line width=0.75pt}} %set default line width to 0.75pt        

	\begin{tikzpicture}[x=0.75pt,y=0.75pt,yscale=-1,xscale=1]
	%uncomment if require: \path (0,300); %set diagram left start at 0, and has height of 300

	%Curve Lines [id:da9285892682977372] 
	\draw    (189,190) .. controls (181.5,250) and (404.5,242) .. (397.5,193) ;
	%Curve Lines [id:da7924907437346282] 
	\draw  [dash pattern={on 0.84pt off 2.51pt}]  (189,190) .. controls (190.5,142) and (395.5,153) .. (397.5,193) ;
	%Curve Lines [id:da020508186457237843] 
	\draw    (189,190) .. controls (190.5,19) and (388.5,27) .. (397.5,193) ;
	%Curve Lines [id:da7775810360309536] 
	\draw  [dash pattern={on 0.84pt off 2.51pt}]  (214,103) .. controls (233.5,144) and (241.4,189.2) .. (232.6,226.4) ;
	%Shape: Ellipse [id:dp5111089961649995] 
	\draw  [dash pattern={on 0.84pt off 2.51pt}] (236.5,189) .. controls (236.5,184.94) and (242.25,181.66) .. (249.35,181.66) .. controls (256.45,181.66) and (262.2,184.94) .. (262.2,189) .. controls (262.2,193.06) and (256.45,196.34) .. (249.35,196.34) .. controls (242.25,196.34) and (236.5,193.06) .. (236.5,189) -- cycle ;
	%Curve Lines [id:da5839381113347033] 
	\draw  [dash pattern={on 0.84pt off 2.51pt}]  (231.8,85) .. controls (251.3,126) and (272.6,173.2) .. (256.6,230.8) ;
	%Curve Lines [id:da4469723860447563] 
	\draw    (211.4,42.8) .. controls (220.2,59.2) and (226.2,72) .. (231.8,85) ;
	%Curve Lines [id:da6134223361402396] 
	\draw    (193.6,60.8) .. controls (202.4,77.2) and (208.4,90) .. (214,103) ;
	%Curve Lines [id:da4664167643230339] 
	\draw    (232.6,226.4) .. controls (227.8,242.8) and (224.2,255.2) .. (218.2,267.6) ;
	%Curve Lines [id:da13402819752656003] 
	\draw    (256.6,230.8) .. controls (251.8,247.2) and (248.2,259.6) .. (242.2,272) ;
	%Curve Lines [id:da559353079188736] 
	\draw  [dash pattern={on 0.84pt off 2.51pt}]  (366.22,103.6) .. controls (346.72,144.6) and (338.82,189.8) .. (347.62,227) ;
	%Shape: Ellipse [id:dp7162312556907064] 
	\draw  [dash pattern={on 0.84pt off 2.51pt}] (343.72,189.6) .. controls (343.72,185.54) and (337.97,182.26) .. (330.87,182.26) .. controls (323.77,182.26) and (318.02,185.54) .. (318.02,189.6) .. controls (318.02,193.66) and (323.77,196.94) .. (330.87,196.94) .. controls (337.97,196.94) and (343.72,193.66) .. (343.72,189.6) -- cycle ;
	%Curve Lines [id:da3010331939362052] 
	\draw  [dash pattern={on 0.84pt off 2.51pt}]  (348.42,85.6) .. controls (328.92,126.6) and (307.62,173.8) .. (323.62,231.4) ;
	%Curve Lines [id:da4468898215223951] 
	\draw    (368.82,43.4) .. controls (360.02,59.8) and (354.02,72.6) .. (348.42,85.6) ;
	%Curve Lines [id:da9455644848772078] 
	\draw    (386.62,61.4) .. controls (377.82,77.8) and (371.82,90.6) .. (366.22,103.6) ;
	%Curve Lines [id:da9875240906730798] 
	\draw    (347.62,227) .. controls (352.42,243.4) and (356.02,255.8) .. (362.02,268.2) ;
	%Curve Lines [id:da7459083423841357] 
	\draw    (323.62,231.4) .. controls (328.42,247.8) and (332.02,260.2) .. (338.02,272.6) ;
	%Straight Lines [id:da20399307729549032] 
	\draw    (216.67,83) -- (195.03,40.67) ;
	\draw [shift={(193.67,38)}, rotate = 422.93] [fill={rgb, 255:red, 0; green, 0; blue, 0 }  ][line width=0.08]  [draw opacity=0] (10.72,-5.15) -- (0,0) -- (10.72,5.15) -- (7.12,0) -- cycle    ;
	%Straight Lines [id:da20561143386928005] 
	\draw    (361.33,86.33) -- (383.04,40.71) ;
	\draw [shift={(384.33,38)}, rotate = 475.45] [fill={rgb, 255:red, 0; green, 0; blue, 0 }  ][line width=0.08]  [draw opacity=0] (10.72,-5.15) -- (0,0) -- (10.72,5.15) -- (7.12,0) -- cycle    ;

	% Text Node
	\draw (404.67,160) node    {$\Sigma $};
	% Text Node
	\draw (176.67,40.67) node    {$I_{1}$};
	% Text Node
	\draw (400.67,46) node    {$I_{2}$};


	\end{tikzpicture}
\end{figure}
\FloatBarrier
Ora ricordando che
\[
	I_{\text{tot}}^{\text{conc. a}\gamma} = \int_{\Sigma}\vec{J} \cdot \vec{n} dS
\]
Possiamo riscrivere la legge di Ampere e applicare il teorema di Stokes
\begin{align*}
	\oint_{\gamma} \vec{B} \cdot d\vec{l} &= \mu_0 \; I^{\text{conc. a}\gamma}_{\text{tot}}  \\
	\oint_{\gamma} \vec{B} \cdot d\vec{l} &= \mu_0 \int_{\Sigma}\vec{J} \cdot \vec{n} dS \\
	\int_{\Sigma} \text{rot}\vec{B} \cdot \vec{n} dS &=  \int_{\Sigma}\mu_0\vec{J} \cdot \vec{n} dS
\end{align*}
Ma dal momento che non abbiamo fatto supposizioni di alcun tipo sulla forma possiamo concludere la \textbf{IV Equazione di Maxwell}
\[
	\boxed{\text{rot}\vec{B} =\mu_0 \vec{J}}
\]
Valida solo in regime stazionario.

Ne deduciamo che $ \vec{B}  $ non è conservativo, non essendo irrotazionale, infatti presenta delle linee chiuse.

Le linee di $\vec{B}$ descrivono un vortice e la corrente sta nel suo centro. L'operatore rotore comunica qualcosa sulla geometria del campo, perché è non nullo nei punti che sono al centro dei vortici delle linee di flusso. Il rotore inoltre ci informa anche sull'asse attorno al quale le linee stanno ruotando e quindi il piano in cui esse descrivono i percorsi chiusi.

Il $ \text{rot}\vec{B}  $ è $ \neq 0 $ nei punti in cui sono presenti dei vortici e ci dice l'orientazione di questi vortici.

N.B.
\[
	\begin{array}{c}
		\text{rot} \vec{v} =0 \\
		\text{div} \vec{v} =0
	\end{array}
	\iff \vec{v} =0
\]
A riprova che la IV Equazione di Maxwell vale implica che ci si trovi in regime stazionario ($ \text{div}\vec{J} =0 $ ), possiamo applicare la divergenza a entrambi i termini, ricordando l'identità operatoriale $ \text{div}(\text{rot}\vec{v} )=0 $
\begin{align*}
	\text{rot}\vec{B}  &= \mu_0 \vec{J}  \\
	\text{div}(\text{rot}\vec{B} ) &=  \text{div}(\mu_0\vec{J} ) \\
	0= \text{div}(\text{rot}\vec{B} ) &= \mu_0 \text{div}\vec{J} \implies \text{div}\vec{J} =0
\end{align*}







































\section{Condizioni al contorno di B}

Abbiamo visto che il campo elettrostatico è discontinuo nell'attraversamento di una superficie carica. Analogamente dimostriamo che il campo magnetico è discontinuo nell'attraversamento di una superficie sede di correnti. Non possiamo usare il vettore $\vec{J}$ per descrivere un vettore superficiale.
\begin{figure}[htpb]
	\centering
	

	\tikzset{every picture/.style={line width=0.75pt}} %set default line width to 0.75pt        

	\begin{tikzpicture}[x=0.75pt,y=0.75pt,yscale=-1,xscale=1]
	%uncomment if require: \path (0,300); %set diagram left start at 0, and has height of 300

	%Straight Lines [id:da40290349945353854] 
	\draw    (154,165) -- (483.5,165) ;
	%Shape: Circle [id:dp7722867521050865] 
	\draw  [fill={rgb, 255:red, 0; green, 0; blue, 0 }  ,fill opacity=1 ] (164.5,165) .. controls (164.5,164.31) and (165.06,163.75) .. (165.75,163.75) .. controls (166.44,163.75) and (167,164.31) .. (167,165) .. controls (167,165.69) and (166.44,166.25) .. (165.75,166.25) .. controls (165.06,166.25) and (164.5,165.69) .. (164.5,165) -- cycle ;
	%Straight Lines [id:da6036238931546025] 
	\draw    (299.99,165.11) -- (327.33,100.76) ;
	\draw [shift={(328.5,98)}, rotate = 473.01] [fill={rgb, 255:red, 0; green, 0; blue, 0 }  ][line width=0.08]  [draw opacity=0] (10.72,-5.15) -- (0,0) -- (10.72,5.15) -- (7.12,0) -- cycle    ;
	%Straight Lines [id:da876809805291892] 
	\draw    (234.5,230) -- (297.86,167.22) ;
	\draw [shift={(299.99,165.11)}, rotate = 495.26] [fill={rgb, 255:red, 0; green, 0; blue, 0 }  ][line width=0.08]  [draw opacity=0] (10.72,-5.15) -- (0,0) -- (10.72,5.15) -- (7.12,0) -- cycle    ;
	%Shape: Circle [id:dp10580150585865411] 
	\draw  [fill={rgb, 255:red, 0; green, 0; blue, 0 }  ,fill opacity=1 ] (184.5,165) .. controls (184.5,164.31) and (185.06,163.75) .. (185.75,163.75) .. controls (186.44,163.75) and (187,164.31) .. (187,165) .. controls (187,165.69) and (186.44,166.25) .. (185.75,166.25) .. controls (185.06,166.25) and (184.5,165.69) .. (184.5,165) -- cycle ;
	%Shape: Circle [id:dp15575276541461003] 
	\draw  [fill={rgb, 255:red, 0; green, 0; blue, 0 }  ,fill opacity=1 ] (204.5,165) .. controls (204.5,164.31) and (205.06,163.75) .. (205.75,163.75) .. controls (206.44,163.75) and (207,164.31) .. (207,165) .. controls (207,165.69) and (206.44,166.25) .. (205.75,166.25) .. controls (205.06,166.25) and (204.5,165.69) .. (204.5,165) -- cycle ;
	%Shape: Circle [id:dp7242236667809356] 
	\draw  [fill={rgb, 255:red, 0; green, 0; blue, 0 }  ,fill opacity=1 ] (224.5,165) .. controls (224.5,164.31) and (225.06,163.75) .. (225.75,163.75) .. controls (226.44,163.75) and (227,164.31) .. (227,165) .. controls (227,165.69) and (226.44,166.25) .. (225.75,166.25) .. controls (225.06,166.25) and (224.5,165.69) .. (224.5,165) -- cycle ;
	%Shape: Circle [id:dp19764606296007692] 
	\draw  [fill={rgb, 255:red, 0; green, 0; blue, 0 }  ,fill opacity=1 ] (244.5,165) .. controls (244.5,164.31) and (245.06,163.75) .. (245.75,163.75) .. controls (246.44,163.75) and (247,164.31) .. (247,165) .. controls (247,165.69) and (246.44,166.25) .. (245.75,166.25) .. controls (245.06,166.25) and (244.5,165.69) .. (244.5,165) -- cycle ;
	%Shape: Circle [id:dp9900072173023722] 
	\draw  [fill={rgb, 255:red, 0; green, 0; blue, 0 }  ,fill opacity=1 ] (264.5,165) .. controls (264.5,164.31) and (265.06,163.75) .. (265.75,163.75) .. controls (266.44,163.75) and (267,164.31) .. (267,165) .. controls (267,165.69) and (266.44,166.25) .. (265.75,166.25) .. controls (265.06,166.25) and (264.5,165.69) .. (264.5,165) -- cycle ;
	%Shape: Circle [id:dp24295923777511308] 
	\draw  [fill={rgb, 255:red, 0; green, 0; blue, 0 }  ,fill opacity=1 ] (284.5,165) .. controls (284.5,164.31) and (285.06,163.75) .. (285.75,163.75) .. controls (286.44,163.75) and (287,164.31) .. (287,165) .. controls (287,165.69) and (286.44,166.25) .. (285.75,166.25) .. controls (285.06,166.25) and (284.5,165.69) .. (284.5,165) -- cycle ;
	%Shape: Circle [id:dp5824761646667302] 
	\draw  [fill={rgb, 255:red, 0; green, 0; blue, 0 }  ,fill opacity=1 ] (304.5,165) .. controls (304.5,164.31) and (305.06,163.75) .. (305.75,163.75) .. controls (306.44,163.75) and (307,164.31) .. (307,165) .. controls (307,165.69) and (306.44,166.25) .. (305.75,166.25) .. controls (305.06,166.25) and (304.5,165.69) .. (304.5,165) -- cycle ;
	%Shape: Circle [id:dp3131757951408054] 
	\draw  [fill={rgb, 255:red, 0; green, 0; blue, 0 }  ,fill opacity=1 ] (324.5,165) .. controls (324.5,164.31) and (325.06,163.75) .. (325.75,163.75) .. controls (326.44,163.75) and (327,164.31) .. (327,165) .. controls (327,165.69) and (326.44,166.25) .. (325.75,166.25) .. controls (325.06,166.25) and (324.5,165.69) .. (324.5,165) -- cycle ;
	%Shape: Circle [id:dp3016076811973367] 
	\draw  [fill={rgb, 255:red, 0; green, 0; blue, 0 }  ,fill opacity=1 ] (344.5,165) .. controls (344.5,164.31) and (345.06,163.75) .. (345.75,163.75) .. controls (346.44,163.75) and (347,164.31) .. (347,165) .. controls (347,165.69) and (346.44,166.25) .. (345.75,166.25) .. controls (345.06,166.25) and (344.5,165.69) .. (344.5,165) -- cycle ;
	%Shape: Circle [id:dp35138916967723177] 
	\draw  [fill={rgb, 255:red, 0; green, 0; blue, 0 }  ,fill opacity=1 ] (364.5,165) .. controls (364.5,164.31) and (365.06,163.75) .. (365.75,163.75) .. controls (366.44,163.75) and (367,164.31) .. (367,165) .. controls (367,165.69) and (366.44,166.25) .. (365.75,166.25) .. controls (365.06,166.25) and (364.5,165.69) .. (364.5,165) -- cycle ;
	%Shape: Circle [id:dp5071323612099659] 
	\draw  [fill={rgb, 255:red, 0; green, 0; blue, 0 }  ,fill opacity=1 ] (384.5,165) .. controls (384.5,164.31) and (385.06,163.75) .. (385.75,163.75) .. controls (386.44,163.75) and (387,164.31) .. (387,165) .. controls (387,165.69) and (386.44,166.25) .. (385.75,166.25) .. controls (385.06,166.25) and (384.5,165.69) .. (384.5,165) -- cycle ;
	%Shape: Circle [id:dp09177828943056299] 
	\draw  [fill={rgb, 255:red, 0; green, 0; blue, 0 }  ,fill opacity=1 ] (404.5,165) .. controls (404.5,164.31) and (405.06,163.75) .. (405.75,163.75) .. controls (406.44,163.75) and (407,164.31) .. (407,165) .. controls (407,165.69) and (406.44,166.25) .. (405.75,166.25) .. controls (405.06,166.25) and (404.5,165.69) .. (404.5,165) -- cycle ;
	%Shape: Circle [id:dp9193795776307525] 
	\draw  [fill={rgb, 255:red, 0; green, 0; blue, 0 }  ,fill opacity=1 ] (424.5,165) .. controls (424.5,164.31) and (425.06,163.75) .. (425.75,163.75) .. controls (426.44,163.75) and (427,164.31) .. (427,165) .. controls (427,165.69) and (426.44,166.25) .. (425.75,166.25) .. controls (425.06,166.25) and (424.5,165.69) .. (424.5,165) -- cycle ;
	%Shape: Circle [id:dp27302929469440107] 
	\draw  [fill={rgb, 255:red, 0; green, 0; blue, 0 }  ,fill opacity=1 ] (444.5,165) .. controls (444.5,164.31) and (445.06,163.75) .. (445.75,163.75) .. controls (446.44,163.75) and (447,164.31) .. (447,165) .. controls (447,165.69) and (446.44,166.25) .. (445.75,166.25) .. controls (445.06,166.25) and (444.5,165.69) .. (444.5,165) -- cycle ;
	%Shape: Circle [id:dp48175179756314135] 
	\draw  [fill={rgb, 255:red, 0; green, 0; blue, 0 }  ,fill opacity=1 ] (464.5,165) .. controls (464.5,164.31) and (465.06,163.75) .. (465.75,163.75) .. controls (466.44,163.75) and (467,164.31) .. (467,165) .. controls (467,165.69) and (466.44,166.25) .. (465.75,166.25) .. controls (465.06,166.25) and (464.5,165.69) .. (464.5,165) -- cycle ;

	% Text Node
	\draw (134,135) node    {$1$};
	% Text Node
	\draw (134,196) node    {$2$};
	% Text Node
	\draw (348.67,101.33) node    {$\vec{B}_{1}$};
	% Text Node
	\draw (269,223) node    {$\vec{B}_{2}$};
	% Text Node
	\draw (498.67,165) node    {$\Sigma $};
	% Text Node
	\draw (402.67,147.8) node    {$j_{s}$};


	\end{tikzpicture}
\end{figure}
\FloatBarrier
Se consideriamo un tratto $dl$ sulla superficie, possiamo pensare che su di esso stia scorrendo della corrente e introdurre una nuova quantità che indichiamo come \textbf{densità di corrente superficiale} pari al rapporto della corrente $di$ sulla lunghezza $dl$.
\[
	\boxed{j_s = \frac{di}{dl}}
\]
Per determinare le condizioni per la componente normale consideriamo un cilindro di area $dS$ e altezza $dh$ a ridosso della superficie di separazione dei due mezzi. Sia $dh \ll \sqrt{dS}$ in modo da trascurare il flusso laterale.
\begin{figure}[htpb]
	\centering
	

	\tikzset{every picture/.style={line width=0.75pt}} %set default line width to 0.75pt        

	\begin{tikzpicture}[x=0.75pt,y=0.75pt,yscale=-1,xscale=1]
	%uncomment if require: \path (0,300); %set diagram left start at 0, and has height of 300

	%Straight Lines [id:da6213004317923272] 
	\draw    (159.33,194.33) -- (488.83,194.33) ;
	%Straight Lines [id:da7110584731917358] 
	\draw    (325.33,194.44) -- (365.77,104.07) ;
	\draw [shift={(367,101.33)}, rotate = 474.11] [fill={rgb, 255:red, 0; green, 0; blue, 0 }  ][line width=0.08]  [draw opacity=0] (10.72,-5.15) -- (0,0) -- (10.72,5.15) -- (7.12,0) -- cycle    ;
	%Shape: Can [id:dp8536022843616089] 
	\draw   (447,166.5) -- (447,215.5) .. controls (447,221.3) and (397.15,226) .. (335.67,226) .. controls (274.18,226) and (224.33,221.3) .. (224.33,215.5) -- (224.33,166.5) .. controls (224.33,160.7) and (274.18,156) .. (335.67,156) .. controls (397.15,156) and (447,160.7) .. (447,166.5) .. controls (447,172.3) and (397.15,177) .. (335.67,177) .. controls (274.18,177) and (224.33,172.3) .. (224.33,166.5) ;
	%Straight Lines [id:da4110358980175308] 
	\draw    (206.33,215.5) -- (206.33,166.5) ;
	\draw [shift={(206.33,166.5)}, rotate = 450] [color={rgb, 255:red, 0; green, 0; blue, 0 }  ][line width=0.75]    (0,5.59) -- (0,-5.59)   ;
	\draw [shift={(206.33,215.5)}, rotate = 450] [color={rgb, 255:red, 0; green, 0; blue, 0 }  ][line width=0.75]    (0,5.59) -- (0,-5.59)   ;
	%Straight Lines [id:da410379630838986] 
	\draw    (403.33,165.77) -- (403.33,120.33) ;
	\draw [shift={(403.33,117.33)}, rotate = 450] [fill={rgb, 255:red, 0; green, 0; blue, 0 }  ][line width=0.08]  [draw opacity=0] (10.72,-5.15) -- (0,0) -- (10.72,5.15) -- (7.12,0) -- cycle    ;
	%Straight Lines [id:da8371791645883637] 
	\draw    (403.33,249.77) -- (403.33,224.33) ;
	\draw [shift={(403.33,252.77)}, rotate = 270] [fill={rgb, 255:red, 0; green, 0; blue, 0 }  ][line width=0.08]  [draw opacity=0] (10.72,-5.15) -- (0,0) -- (10.72,5.15) -- (7.12,0) -- cycle    ;
	%Straight Lines [id:da24523180797428168] 
	\draw  [dash pattern={on 0.84pt off 2.51pt}]  (403.33,224.33) -- (403.33,204.33) ;
	%Straight Lines [id:da5308318657083659] 
	\draw    (241.5,278) -- (323.2,196.56) ;
	\draw [shift={(325.33,194.44)}, rotate = 495.09] [fill={rgb, 255:red, 0; green, 0; blue, 0 }  ][line width=0.08]  [draw opacity=0] (10.72,-5.15) -- (0,0) -- (10.72,5.15) -- (7.12,0) -- cycle    ;
	%Shape: Circle [id:dp16041020967866237] 
	\draw  [fill={rgb, 255:red, 0; green, 0; blue, 0 }  ,fill opacity=1 ] (172.5,194.44) .. controls (172.5,193.75) and (173.06,193.19) .. (173.75,193.19) .. controls (174.44,193.19) and (175,193.75) .. (175,194.44) .. controls (175,195.13) and (174.44,195.69) .. (173.75,195.69) .. controls (173.06,195.69) and (172.5,195.13) .. (172.5,194.44) -- cycle ;
	%Shape: Circle [id:dp8427889866553417] 
	\draw  [fill={rgb, 255:red, 0; green, 0; blue, 0 }  ,fill opacity=1 ] (192.5,194.44) .. controls (192.5,193.75) and (193.06,193.19) .. (193.75,193.19) .. controls (194.44,193.19) and (195,193.75) .. (195,194.44) .. controls (195,195.13) and (194.44,195.69) .. (193.75,195.69) .. controls (193.06,195.69) and (192.5,195.13) .. (192.5,194.44) -- cycle ;
	%Shape: Circle [id:dp6742655422084851] 
	\draw  [fill={rgb, 255:red, 0; green, 0; blue, 0 }  ,fill opacity=1 ] (212.5,194.44) .. controls (212.5,193.75) and (213.06,193.19) .. (213.75,193.19) .. controls (214.44,193.19) and (215,193.75) .. (215,194.44) .. controls (215,195.13) and (214.44,195.69) .. (213.75,195.69) .. controls (213.06,195.69) and (212.5,195.13) .. (212.5,194.44) -- cycle ;
	%Shape: Circle [id:dp2559509551421366] 
	\draw  [fill={rgb, 255:red, 0; green, 0; blue, 0 }  ,fill opacity=1 ] (232.5,194.44) .. controls (232.5,193.75) and (233.06,193.19) .. (233.75,193.19) .. controls (234.44,193.19) and (235,193.75) .. (235,194.44) .. controls (235,195.13) and (234.44,195.69) .. (233.75,195.69) .. controls (233.06,195.69) and (232.5,195.13) .. (232.5,194.44) -- cycle ;
	%Shape: Circle [id:dp25389345637442107] 
	\draw  [fill={rgb, 255:red, 0; green, 0; blue, 0 }  ,fill opacity=1 ] (252.5,194.44) .. controls (252.5,193.75) and (253.06,193.19) .. (253.75,193.19) .. controls (254.44,193.19) and (255,193.75) .. (255,194.44) .. controls (255,195.13) and (254.44,195.69) .. (253.75,195.69) .. controls (253.06,195.69) and (252.5,195.13) .. (252.5,194.44) -- cycle ;
	%Shape: Circle [id:dp25508561739913915] 
	\draw  [fill={rgb, 255:red, 0; green, 0; blue, 0 }  ,fill opacity=1 ] (272.5,194.44) .. controls (272.5,193.75) and (273.06,193.19) .. (273.75,193.19) .. controls (274.44,193.19) and (275,193.75) .. (275,194.44) .. controls (275,195.13) and (274.44,195.69) .. (273.75,195.69) .. controls (273.06,195.69) and (272.5,195.13) .. (272.5,194.44) -- cycle ;
	%Shape: Circle [id:dp44631343518992095] 
	\draw  [fill={rgb, 255:red, 0; green, 0; blue, 0 }  ,fill opacity=1 ] (292.5,194.44) .. controls (292.5,193.75) and (293.06,193.19) .. (293.75,193.19) .. controls (294.44,193.19) and (295,193.75) .. (295,194.44) .. controls (295,195.13) and (294.44,195.69) .. (293.75,195.69) .. controls (293.06,195.69) and (292.5,195.13) .. (292.5,194.44) -- cycle ;
	%Shape: Circle [id:dp5911616852292005] 
	\draw  [fill={rgb, 255:red, 0; green, 0; blue, 0 }  ,fill opacity=1 ] (312.5,194.44) .. controls (312.5,193.75) and (313.06,193.19) .. (313.75,193.19) .. controls (314.44,193.19) and (315,193.75) .. (315,194.44) .. controls (315,195.13) and (314.44,195.69) .. (313.75,195.69) .. controls (313.06,195.69) and (312.5,195.13) .. (312.5,194.44) -- cycle ;
	%Shape: Circle [id:dp10546600814340201] 
	\draw  [fill={rgb, 255:red, 0; green, 0; blue, 0 }  ,fill opacity=1 ] (332.5,194.44) .. controls (332.5,193.75) and (333.06,193.19) .. (333.75,193.19) .. controls (334.44,193.19) and (335,193.75) .. (335,194.44) .. controls (335,195.13) and (334.44,195.69) .. (333.75,195.69) .. controls (333.06,195.69) and (332.5,195.13) .. (332.5,194.44) -- cycle ;
	%Shape: Circle [id:dp13547297521940904] 
	\draw  [fill={rgb, 255:red, 0; green, 0; blue, 0 }  ,fill opacity=1 ] (352.5,194.44) .. controls (352.5,193.75) and (353.06,193.19) .. (353.75,193.19) .. controls (354.44,193.19) and (355,193.75) .. (355,194.44) .. controls (355,195.13) and (354.44,195.69) .. (353.75,195.69) .. controls (353.06,195.69) and (352.5,195.13) .. (352.5,194.44) -- cycle ;
	%Shape: Circle [id:dp09508956969941629] 
	\draw  [fill={rgb, 255:red, 0; green, 0; blue, 0 }  ,fill opacity=1 ] (372.5,194.44) .. controls (372.5,193.75) and (373.06,193.19) .. (373.75,193.19) .. controls (374.44,193.19) and (375,193.75) .. (375,194.44) .. controls (375,195.13) and (374.44,195.69) .. (373.75,195.69) .. controls (373.06,195.69) and (372.5,195.13) .. (372.5,194.44) -- cycle ;
	%Shape: Circle [id:dp7963240077414149] 
	\draw  [fill={rgb, 255:red, 0; green, 0; blue, 0 }  ,fill opacity=1 ] (392.5,194.44) .. controls (392.5,193.75) and (393.06,193.19) .. (393.75,193.19) .. controls (394.44,193.19) and (395,193.75) .. (395,194.44) .. controls (395,195.13) and (394.44,195.69) .. (393.75,195.69) .. controls (393.06,195.69) and (392.5,195.13) .. (392.5,194.44) -- cycle ;
	%Shape: Circle [id:dp7134148353618073] 
	\draw  [fill={rgb, 255:red, 0; green, 0; blue, 0 }  ,fill opacity=1 ] (412.5,194.44) .. controls (412.5,193.75) and (413.06,193.19) .. (413.75,193.19) .. controls (414.44,193.19) and (415,193.75) .. (415,194.44) .. controls (415,195.13) and (414.44,195.69) .. (413.75,195.69) .. controls (413.06,195.69) and (412.5,195.13) .. (412.5,194.44) -- cycle ;
	%Shape: Circle [id:dp6805815488388596] 
	\draw  [fill={rgb, 255:red, 0; green, 0; blue, 0 }  ,fill opacity=1 ] (432.5,194.44) .. controls (432.5,193.75) and (433.06,193.19) .. (433.75,193.19) .. controls (434.44,193.19) and (435,193.75) .. (435,194.44) .. controls (435,195.13) and (434.44,195.69) .. (433.75,195.69) .. controls (433.06,195.69) and (432.5,195.13) .. (432.5,194.44) -- cycle ;
	%Shape: Circle [id:dp751825076031954] 
	\draw  [fill={rgb, 255:red, 0; green, 0; blue, 0 }  ,fill opacity=1 ] (452.5,194.44) .. controls (452.5,193.75) and (453.06,193.19) .. (453.75,193.19) .. controls (454.44,193.19) and (455,193.75) .. (455,194.44) .. controls (455,195.13) and (454.44,195.69) .. (453.75,195.69) .. controls (453.06,195.69) and (452.5,195.13) .. (452.5,194.44) -- cycle ;
	%Shape: Circle [id:dp23506561777408042] 
	\draw  [fill={rgb, 255:red, 0; green, 0; blue, 0 }  ,fill opacity=1 ] (472.5,194.44) .. controls (472.5,193.75) and (473.06,193.19) .. (473.75,193.19) .. controls (474.44,193.19) and (475,193.75) .. (475,194.44) .. controls (475,195.13) and (474.44,195.69) .. (473.75,195.69) .. controls (473.06,195.69) and (472.5,195.13) .. (472.5,194.44) -- cycle ;

	% Text Node
	\draw (139.33,164.33) node    {$1$};
	% Text Node
	\draw (139.33,225.33) node    {$2$};
	% Text Node
	\draw (345,101.33) node    {$\vec{B}_{1}$};
	% Text Node
	\draw (293.33,261) node    {$\vec{B}_{2}$};
	% Text Node
	\draw (504,194.33) node    {$\Sigma $};
	% Text Node
	\draw (444,148.33) node    {$dS$};
	% Text Node
	\draw (205.33,225.67) node    {$dh$};
	% Text Node
	\draw (421.33,121.67) node    {$\vec{n}_{1}$};
	% Text Node
	\draw (421.33,247.33) node    {$\vec{n}_{2}$};
	% Text Node
	\draw (450.67,227.67) node    {$\Sigma _{g}$};
	% Text Node
	\draw (173.67,177.3) node    {$j_{s}$};


	\end{tikzpicture}
\end{figure}
\FloatBarrier
Ricordando che il flusso di $\vec{B}$ è sempre nullo su una superficie chiusa, possiamo scrivere
\begin{align*}
	0 = \Phi_{\Sigma g}(\vec{B} )  &\simeq  \vec{B}_1\cdot \vec{n}_1  dS + \vec{B}_2\cdot \vec{n}_2  dS    \\
	0 &= (\vec{B}_1\vec{n}_1 - \vec{B}_2\vec{n}_1 )dS \\
\end{align*}
Chiamando
\begin{gather*}
	B_{1n} = \vec{B}_1\cdot \vec{n}_1 \\
	B_{2n} = \vec{B}_2\cdot \vec{n}_1
\end{gather*}
Abbiamo
\[
	\boxed{B_{1n} = B_{2n}}
\]
Questo ci dice che la componente normale del campo magnetico si conserva.
Consideriamo a questo punto il versore tangente alla superficie e individuiamo un percorso chiuso $\gamma$ di forma rettangolare a cavallo della superficie, di altezza $dh$ e larghezza $dl$ con $ dh \ll dl $. Possiamo dire le seguenti cose.
\begin{figure}[htpb]
	\centering
	

	\tikzset{every picture/.style={line width=0.75pt}} %set default line width to 0.75pt        

	\begin{tikzpicture}[x=0.75pt,y=0.75pt,yscale=-1,xscale=1]
	%uncomment if require: \path (0,300); %set diagram left start at 0, and has height of 300

	%Straight Lines [id:da3977022913494037] 
	\draw    (174,185) -- (503.5,185) ;
	%Straight Lines [id:da7732099440082854] 
	\draw    (319.99,185.11) -- (360.44,94.74) ;
	\draw [shift={(361.67,92)}, rotate = 474.11] [fill={rgb, 255:red, 0; green, 0; blue, 0 }  ][line width=0.08]  [draw opacity=0] (10.72,-5.15) -- (0,0) -- (10.72,5.15) -- (7.12,0) -- cycle    ;
	%Straight Lines [id:da5500797242504052] 
	\draw    (240.67,169.33) -- (460.33,169.33) ;
	\draw [shift={(350.5,169.33)}, rotate = 180] [fill={rgb, 255:red, 0; green, 0; blue, 0 }  ][line width=0.08]  [draw opacity=0] (10.72,-5.15) -- (0,0) -- (10.72,5.15) -- (7.12,0) -- cycle    ;
	%Straight Lines [id:da6756783602076346] 
	\draw    (240.67,198) -- (460.33,198) ;
	\draw [shift={(350.5,198)}, rotate = 0] [fill={rgb, 255:red, 0; green, 0; blue, 0 }  ][line width=0.08]  [draw opacity=0] (10.72,-5.15) -- (0,0) -- (10.72,5.15) -- (7.12,0) -- cycle    ;
	%Straight Lines [id:da253951117937784] 
	\draw    (460.33,169.33) -- (460.33,198) ;
	%Straight Lines [id:da6384193071075221] 
	\draw    (240.67,169.33) -- (240.67,198) ;
	%Straight Lines [id:da29680057316531405] 
	\draw    (422.66,143.77) -- (462.67,143.77) ;
	\draw [shift={(465.67,143.77)}, rotate = 180] [fill={rgb, 255:red, 0; green, 0; blue, 0 }  ][line width=0.08]  [draw opacity=0] (10.72,-5.15) -- (0,0) -- (10.72,5.15) -- (7.12,0) -- cycle    ;
	%Straight Lines [id:da7403301396741371] 
	\draw    (238.5,265) -- (317.85,187.21) ;
	\draw [shift={(319.99,185.11)}, rotate = 495.57] [fill={rgb, 255:red, 0; green, 0; blue, 0 }  ][line width=0.08]  [draw opacity=0] (10.72,-5.15) -- (0,0) -- (10.72,5.15) -- (7.12,0) -- cycle    ;
	%Shape: Circle [id:dp8471426621089055] 
	\draw  [fill={rgb, 255:red, 0; green, 0; blue, 0 }  ,fill opacity=1 ] (184.5,185) .. controls (184.5,184.31) and (185.06,183.75) .. (185.75,183.75) .. controls (186.44,183.75) and (187,184.31) .. (187,185) .. controls (187,185.69) and (186.44,186.25) .. (185.75,186.25) .. controls (185.06,186.25) and (184.5,185.69) .. (184.5,185) -- cycle ;
	%Shape: Circle [id:dp6247963925507645] 
	\draw  [fill={rgb, 255:red, 0; green, 0; blue, 0 }  ,fill opacity=1 ] (204.5,185) .. controls (204.5,184.31) and (205.06,183.75) .. (205.75,183.75) .. controls (206.44,183.75) and (207,184.31) .. (207,185) .. controls (207,185.69) and (206.44,186.25) .. (205.75,186.25) .. controls (205.06,186.25) and (204.5,185.69) .. (204.5,185) -- cycle ;
	%Shape: Circle [id:dp8166862233747876] 
	\draw  [fill={rgb, 255:red, 0; green, 0; blue, 0 }  ,fill opacity=1 ] (224.5,185) .. controls (224.5,184.31) and (225.06,183.75) .. (225.75,183.75) .. controls (226.44,183.75) and (227,184.31) .. (227,185) .. controls (227,185.69) and (226.44,186.25) .. (225.75,186.25) .. controls (225.06,186.25) and (224.5,185.69) .. (224.5,185) -- cycle ;
	%Shape: Circle [id:dp09582189768683791] 
	\draw  [fill={rgb, 255:red, 0; green, 0; blue, 0 }  ,fill opacity=1 ] (244.5,185) .. controls (244.5,184.31) and (245.06,183.75) .. (245.75,183.75) .. controls (246.44,183.75) and (247,184.31) .. (247,185) .. controls (247,185.69) and (246.44,186.25) .. (245.75,186.25) .. controls (245.06,186.25) and (244.5,185.69) .. (244.5,185) -- cycle ;
	%Shape: Circle [id:dp9482119358944006] 
	\draw  [fill={rgb, 255:red, 0; green, 0; blue, 0 }  ,fill opacity=1 ] (264.5,185) .. controls (264.5,184.31) and (265.06,183.75) .. (265.75,183.75) .. controls (266.44,183.75) and (267,184.31) .. (267,185) .. controls (267,185.69) and (266.44,186.25) .. (265.75,186.25) .. controls (265.06,186.25) and (264.5,185.69) .. (264.5,185) -- cycle ;
	%Shape: Circle [id:dp9627900255160409] 
	\draw  [fill={rgb, 255:red, 0; green, 0; blue, 0 }  ,fill opacity=1 ] (284.5,185) .. controls (284.5,184.31) and (285.06,183.75) .. (285.75,183.75) .. controls (286.44,183.75) and (287,184.31) .. (287,185) .. controls (287,185.69) and (286.44,186.25) .. (285.75,186.25) .. controls (285.06,186.25) and (284.5,185.69) .. (284.5,185) -- cycle ;
	%Shape: Circle [id:dp9256729969047921] 
	\draw  [fill={rgb, 255:red, 0; green, 0; blue, 0 }  ,fill opacity=1 ] (304.5,185) .. controls (304.5,184.31) and (305.06,183.75) .. (305.75,183.75) .. controls (306.44,183.75) and (307,184.31) .. (307,185) .. controls (307,185.69) and (306.44,186.25) .. (305.75,186.25) .. controls (305.06,186.25) and (304.5,185.69) .. (304.5,185) -- cycle ;
	%Shape: Circle [id:dp30974022568739357] 
	\draw  [fill={rgb, 255:red, 0; green, 0; blue, 0 }  ,fill opacity=1 ] (324.5,185) .. controls (324.5,184.31) and (325.06,183.75) .. (325.75,183.75) .. controls (326.44,183.75) and (327,184.31) .. (327,185) .. controls (327,185.69) and (326.44,186.25) .. (325.75,186.25) .. controls (325.06,186.25) and (324.5,185.69) .. (324.5,185) -- cycle ;
	%Shape: Circle [id:dp0766862258644907] 
	\draw  [fill={rgb, 255:red, 0; green, 0; blue, 0 }  ,fill opacity=1 ] (344.5,185) .. controls (344.5,184.31) and (345.06,183.75) .. (345.75,183.75) .. controls (346.44,183.75) and (347,184.31) .. (347,185) .. controls (347,185.69) and (346.44,186.25) .. (345.75,186.25) .. controls (345.06,186.25) and (344.5,185.69) .. (344.5,185) -- cycle ;
	%Shape: Circle [id:dp9393455596382134] 
	\draw  [fill={rgb, 255:red, 0; green, 0; blue, 0 }  ,fill opacity=1 ] (364.5,185) .. controls (364.5,184.31) and (365.06,183.75) .. (365.75,183.75) .. controls (366.44,183.75) and (367,184.31) .. (367,185) .. controls (367,185.69) and (366.44,186.25) .. (365.75,186.25) .. controls (365.06,186.25) and (364.5,185.69) .. (364.5,185) -- cycle ;
	%Shape: Circle [id:dp9958337185030302] 
	\draw  [fill={rgb, 255:red, 0; green, 0; blue, 0 }  ,fill opacity=1 ] (384.5,185) .. controls (384.5,184.31) and (385.06,183.75) .. (385.75,183.75) .. controls (386.44,183.75) and (387,184.31) .. (387,185) .. controls (387,185.69) and (386.44,186.25) .. (385.75,186.25) .. controls (385.06,186.25) and (384.5,185.69) .. (384.5,185) -- cycle ;
	%Shape: Circle [id:dp05334937723511701] 
	\draw  [fill={rgb, 255:red, 0; green, 0; blue, 0 }  ,fill opacity=1 ] (404.5,185) .. controls (404.5,184.31) and (405.06,183.75) .. (405.75,183.75) .. controls (406.44,183.75) and (407,184.31) .. (407,185) .. controls (407,185.69) and (406.44,186.25) .. (405.75,186.25) .. controls (405.06,186.25) and (404.5,185.69) .. (404.5,185) -- cycle ;
	%Shape: Circle [id:dp14065239917369898] 
	\draw  [fill={rgb, 255:red, 0; green, 0; blue, 0 }  ,fill opacity=1 ] (424.5,185) .. controls (424.5,184.31) and (425.06,183.75) .. (425.75,183.75) .. controls (426.44,183.75) and (427,184.31) .. (427,185) .. controls (427,185.69) and (426.44,186.25) .. (425.75,186.25) .. controls (425.06,186.25) and (424.5,185.69) .. (424.5,185) -- cycle ;
	%Shape: Circle [id:dp6198608535462575] 
	\draw  [fill={rgb, 255:red, 0; green, 0; blue, 0 }  ,fill opacity=1 ] (444.5,185) .. controls (444.5,184.31) and (445.06,183.75) .. (445.75,183.75) .. controls (446.44,183.75) and (447,184.31) .. (447,185) .. controls (447,185.69) and (446.44,186.25) .. (445.75,186.25) .. controls (445.06,186.25) and (444.5,185.69) .. (444.5,185) -- cycle ;
	%Shape: Circle [id:dp8759882873535005] 
	\draw  [fill={rgb, 255:red, 0; green, 0; blue, 0 }  ,fill opacity=1 ] (464.5,185) .. controls (464.5,184.31) and (465.06,183.75) .. (465.75,183.75) .. controls (466.44,183.75) and (467,184.31) .. (467,185) .. controls (467,185.69) and (466.44,186.25) .. (465.75,186.25) .. controls (465.06,186.25) and (464.5,185.69) .. (464.5,185) -- cycle ;
	%Shape: Circle [id:dp4544633676859924] 
	\draw  [fill={rgb, 255:red, 0; green, 0; blue, 0 }  ,fill opacity=1 ] (484.5,185) .. controls (484.5,184.31) and (485.06,183.75) .. (485.75,183.75) .. controls (486.44,183.75) and (487,184.31) .. (487,185) .. controls (487,185.69) and (486.44,186.25) .. (485.75,186.25) .. controls (485.06,186.25) and (484.5,185.69) .. (484.5,185) -- cycle ;

	% Text Node
	\draw (154,155) node    {$1$};
	% Text Node
	\draw (154,216) node    {$2$};
	% Text Node
	\draw (374.67,98.33) node    {$\vec{B}_{1}$};
	% Text Node
	\draw (283,248) node    {$\vec{B}_{2}$};
	% Text Node
	\draw (518.67,185) node    {$\Sigma $};
	% Text Node
	\draw (362,151.67) node    {$d\vec{l}$};
	% Text Node
	\draw (444,129.67) node    {$\vec{u}_{t}$};
	% Text Node
	\draw (247.33,154.33) node    {$\gamma $};
	% Text Node
	\draw (200.17,166.8) node    {$j_{s}$};


	\end{tikzpicture}
\end{figure}
\FloatBarrier
Applicando la legge di Ampere:
\begin{align*}
	\oint_{\gamma} \vec{B} \cdot d\vec{l}  &= \mu_0 I^{conc. a \gamma}  \\
	\vec{B}_1\cdot dl \, \vec{u}_t - \vec{B}_2\cdot dl \, \vec{u}_t &\simeq \mu_0 \, dl \, j_s  \\
	\vec{B}_1\cdot  \vec{u}_t - \vec{B}_2\cdot  \vec{u}_t &\simeq \mu_0  \, j_s
\end{align*}
Chiamando:
\begin{gather*}
	B_{1t} = \vec{B}_1\cdot \vec{u}_t \\
	B_{2t} = \vec{B}_2\cdot \vec{u}_t
\end{gather*}
Abbiamo:
\[
	\boxed{B_{1t}-B_{2t} = \mu_0  \, j_s}
\]
La componente tangente del campo non si conserva se vi è la presenza di una corrente superficiale. I campi considerati sono i campi totali, non solo quelli generati dalle correnti ma anche altri campi esterni generati da altri circuiti non rappresentati.







































\section{Potenziale magnetico vettore}

Abbiamo visto che il potenziale soddisfa all'equazione di Poisson:
\[
	\vec{E} = - \vec{\nabla} V \qquad \nabla^2 V = -\frac{\rho}{\varepsilon_0}
\]
la cui soluzione è data da:
\[
	V(P) = \int \frac{\rho (\vec{r'})\,d\tau}{4\pi \varepsilon_0 |\vec{r} -\vec{r'} |}
\]
Il fatto che il rotore del campo magnetico sia proporzionale alla densità di corrente e quindi non sia identicamente nullo, impedisce di definire un potenziale scalare magnetico. Il campo magnetico ha però divergenza sempre nulla, quindi può sempre essere espresso come rotore di un altro vettore $\vec{A}$.

Definiamo \textbf{potenziale magnetico vettore} quel vettore $ \vec{A}  $ tale che
\[
	\boxed{\vec{B} = \text{rot}\vec{A}}
\]
Notiamo intanto che questa relazione soddisfa immediatamente il fatto che $ \text{div}\vec{B} =0 $ dal momento che
\[
	\text{div}\vec{B} = \text{div}(\text{rot}\vec{A})= \vec{\nabla} \cdot (\vec{\nabla} \times \vec{A} ) = 0 \qquad \text{ (identità operatoriale)}
\]
Consideriamo poi una \emph{invarianza di gauge}, ossia una trasformazione che lascia invariata la relazione del potenziale vettore.
\begin{gather*}
	\vec{A}' = \vec{A} +\vec{\nabla} f \\
	\text{rot}\vec{A}' = \text{rot}\vec{A} + \underbrace{\text{rot}(\vec{\nabla} f)}_{=0} = \text{rot}\vec{A}
\end{gather*}
Quindi il potenziale, come ci si poteva aspettare, è definito a meno di un termine.
Consideriamo la IV Equazione di Maxwell e sostituiamo il potenziale.
\[
	\text{rot}\vec{B} = \mu_0 \vec{J} \implies \text{rot}(\text{rot}\vec{A} ) = \mu_0  \vec{J}
\]
Che possiamo riscrivere con la notazione nabla e usare un'identità vettoriale
\begin{align*}
	\vec{\nabla} \times (\vec{\nabla} \times \vec{A} ) &= \mu_0 \vec{J} \\
	\vec{\nabla} (\vec{\nabla} \cdot \vec{A} ) - \nabla^2 \vec{A} &= \mu_0 \vec{J}
\end{align*}
A questo punto calcoliamo la divergenza di $ \vec{A}' $
\[
	\vec{\nabla} \cdot \vec{A}' = \vec{\nabla} \cdot \vec{A} +\vec{\nabla} \cdot (\vec{\nabla} S)= \vec{\nabla} \cdot \vec{A} + \nabla^2 S
\]
Quindi possiamo scegliere un campo $ S $ che soddisfi $ \vec{\nabla} \cdot \vec{A} + \nabla^2 S = 0 $ ossia $ \text{div}\vec{A} ' = 0 $.

In tutto questo ricordiamoci che non abbiamo modificato il campo magnetico.

Quindi riscriviamo l'equazione di prima, ma in termini di $\vec{A}'$
\[
	\left. \begin{array}{r}
	 	\vec{\nabla} (\vec{\nabla} \cdot \vec{A}' ) - \nabla^2 \vec{A}' = \mu_0 \vec{J} \\
		\vec{\nabla} \cdot \vec{A} ' = 0
	\end{array} \right\}
	\implies \boxed{ \nabla^2 \vec{A} = -\mu_0 \vec{J}}
\]
dove l'apice può cadere rinominando il potenziale vettore.
L'equazione trovata alla fine corrisponde a 3 equazioni di Poisson analoghe a quelle trovate per il campo $ \vec{E}  $,
\[
	\nabla^2 A_x = -\mu_0 \, J_x \qquad \nabla^2 A_y = -\mu_0 \, J_y \qquad \nabla^2 A_z = -\mu_0 \, J_z
\]
Ciascuna di queste equazioni è matematicamente eguale all'equazione di Poisson, quindi, a parte il significato fisico, la soluzione deve essere la stessa. Tenendo conto della relazione vettoriale presente questa volta abbiamo
\begin{equation*}
	\boxed{\vec{A}(\vec{r}) = \int_{\tau}\frac{\mu_0 \vec{J} (\vec{r'})}{4\pi |\vec{r} - \vec{r'}|}d\tau \qquad \text{con} \qquad \vec{B} = \text{rot}\vec{A}
}
\end{equation*}
Infine, se applichiamo il calcolo del flusso di $\vec{B}$ inserendo $ \vec{B} =\text{rot}\vec{A}  $ e applicando il teorema di Stokes:
\[
	\Phi_{\Sigma}(\vec{B}) = \int_{\Sigma}\vec{B} \cdot \vec{n} dS = \int_{\Sigma}(\text{rot}\vec{A} ) \cdot \vec{n} dS = \oint_{\gamma} \vec{A} d\vec{l}
\]
Che è la forma integrale della definizione locale. La circuitazione
di $\vec{A}$ lungo una linea chiusa è eguale al flusso di $\vec{B}$ attraverso una qualsiasi superficie $ \Sigma  $ appoggiata sulla linea chiusa (relazione molto utile in presenza di simmetrie).
\begin{figure}[htpb]
	\centering
	

	\tikzset{every picture/.style={line width=0.75pt}} %set default line width to 0.75pt        

	\begin{tikzpicture}[x=0.75pt,y=0.75pt,yscale=-1,xscale=1]
	%uncomment if require: \path (0,300); %set diagram left start at 0, and has height of 300

	%Curve Lines [id:da7272074769202108] 
	\draw    (209,210) .. controls (201.5,270) and (424.5,262) .. (417.5,213) ;
	%Curve Lines [id:da23093259879535255] 
	\draw  [dash pattern={on 0.84pt off 2.51pt}]  (209,210) .. controls (210.5,162) and (415.5,173) .. (417.5,213) ;
	%Curve Lines [id:da39327262602797664] 
	\draw    (209,210) .. controls (210.5,39) and (408.5,47) .. (417.5,213) ;
	%Straight Lines [id:da3821458906147317] 
	\draw    (247.67,108) -- (218.78,68.91) ;
	\draw [shift={(217,66.5)}, rotate = 413.53999999999996] [fill={rgb, 255:red, 0; green, 0; blue, 0 }  ][line width=0.08]  [draw opacity=0] (10.72,-5.15) -- (0,0) -- (10.72,5.15) -- (7.12,0) -- cycle    ;
	%Straight Lines [id:da5165175504778032] 
	\draw    (247.67,108) -- (267.07,48.35) ;
	\draw [shift={(268,45.5)}, rotate = 468.02] [fill={rgb, 255:red, 0; green, 0; blue, 0 }  ][line width=0.08]  [draw opacity=0] (10.72,-5.15) -- (0,0) -- (10.72,5.15) -- (7.12,0) -- cycle    ;

	% Text Node
	\draw (414.67,130) node    {$\Sigma $};
	% Text Node
	\draw (202.67,70) node    {$\vec{n}$};
	% Text Node
	\draw (281.67,56) node    {$\vec{B}$};


	\end{tikzpicture}
\end{figure}
\FloatBarrier







































\section{Fenomeni magnetostatici nella materia}

Per parlare di questo argomento dobbiamo introdurre il solenoide rettilineo infinito, con $N_s$ numero di spire per unità di lunghezza. È un circuito in cui un filo conduttore viene avvolto su un supporto cilindrico cavo.
\begin{figure}[htpb]
	\centering
	

	\tikzset{every picture/.style={line width=0.75pt}} %set default line width to 0.75pt        

	\begin{tikzpicture}[x=0.75pt,y=0.75pt,yscale=-0.8,xscale=0.8]
	%uncomment if require: \path (0,300); %set diagram left start at 0, and has height of 300

	%Shape: Can [id:dp045013765137410955] 
	\draw   (386.81,167.5) -- (128.72,167.5) .. controls (121.97,167.5) and (116.5,149.26) .. (116.5,126.75) .. controls (116.5,104.24) and (121.97,86) .. (128.72,86) -- (386.81,86) .. controls (393.56,86) and (399.03,104.24) .. (399.03,126.75) .. controls (399.03,149.26) and (393.56,167.5) .. (386.81,167.5) .. controls (380.06,167.5) and (374.58,149.26) .. (374.58,126.75) .. controls (374.58,104.24) and (380.06,86) .. (386.81,86) ;
	%Curve Lines [id:da22142605208872257] 
	\draw    (215.8,48.4) .. controls (171.8,63.6) and (173,167.6) .. (189.81,167.5) ;
	%Curve Lines [id:da21365082906938282] 
	\draw    (209.81,86) .. controls (193,85.2) and (193,167.6) .. (209.81,167.5) ;
	%Curve Lines [id:da22930047139455945] 
	\draw    (229.81,86) .. controls (213,85.2) and (213,167.6) .. (229.81,167.5) ;
	%Curve Lines [id:da7450044653523427] 
	\draw    (249.81,86) .. controls (233,85.2) and (233,167.6) .. (249.81,167.5) ;
	%Curve Lines [id:da45467405986713083] 
	\draw    (269.81,86) .. controls (253,85.2) and (253,167.6) .. (269.81,167.5) ;
	%Curve Lines [id:da9056531400522023] 
	\draw    (289.81,86) .. controls (273,85.2) and (273,167.6) .. (289.81,167.5) ;
	%Curve Lines [id:da5705660849247445] 
	\draw    (309.81,86) .. controls (293,85.2) and (293,167.6) .. (309.81,167.5) ;
	%Curve Lines [id:da46102047817662717] 
	\draw    (329.81,86) .. controls (313,85.2) and (305.4,174.4) .. (351.8,194.8) ;




	\end{tikzpicture}
\end{figure}
\FloatBarrier
Si dimostra che, se facciamo circolare nel solenoide una corrente stazionaria $I$, genereremo all'interno di esso un campo magnetico uniforme. Il verso di tale campo è determinato dalla regola del cavatappi, la direzione è quella dell'asse del solenoide. Tale campo può essere calcolato con la legge di Ampere. Se introduciamo il versore uz parallelo all'asse, possiamo scrivere questo campo come: $\vec{B}_0 = \mu_0 N_sI\vec{u}_z$.
\begin{figure}[htpb]
	\centering
	

	\tikzset{every picture/.style={line width=0.75pt}} %set default line width to 0.75pt        

	\begin{tikzpicture}[x=0.75pt,y=0.75pt,yscale=-1,xscale=1]
	%uncomment if require: \path (0,443); %set diagram left start at 0, and has height of 443

	%Shape: Rectangle [id:dp3570439802483778] 
	\draw  [draw opacity=0][fill={rgb, 255:red, 222; green, 222; blue, 222 }  ,fill opacity=1 ] (136,289) -- (465.5,289) -- (465.5,379) -- (136,379) -- cycle ;
	%Straight Lines [id:da36567456999813075] 
	\draw    (136,109) -- (465.5,109) ;
	%Shape: Circle [id:dp06167117646965958] 
	\draw  [fill={rgb, 255:red, 0; green, 0; blue, 0 }  ,fill opacity=1 ] (146.5,99) .. controls (146.5,98.31) and (147.06,97.75) .. (147.75,97.75) .. controls (148.44,97.75) and (149,98.31) .. (149,99) .. controls (149,99.69) and (148.44,100.25) .. (147.75,100.25) .. controls (147.06,100.25) and (146.5,99.69) .. (146.5,99) -- cycle ;
	%Straight Lines [id:da4511644058526041] 
	\draw    (253,140) -- (336.5,140) ;
	\draw [shift={(339.5,140)}, rotate = 180] [fill={rgb, 255:red, 0; green, 0; blue, 0 }  ][line width=0.08]  [draw opacity=0] (10.72,-5.15) -- (0,0) -- (10.72,5.15) -- (7.12,0) -- cycle    ;
	%Shape: Circle [id:dp5660420390804164] 
	\draw  [fill={rgb, 255:red, 0; green, 0; blue, 0 }  ,fill opacity=1 ] (166.5,99) .. controls (166.5,98.31) and (167.06,97.75) .. (167.75,97.75) .. controls (168.44,97.75) and (169,98.31) .. (169,99) .. controls (169,99.69) and (168.44,100.25) .. (167.75,100.25) .. controls (167.06,100.25) and (166.5,99.69) .. (166.5,99) -- cycle ;
	%Shape: Circle [id:dp07328533443252061] 
	\draw  [fill={rgb, 255:red, 0; green, 0; blue, 0 }  ,fill opacity=1 ] (186.5,99) .. controls (186.5,98.31) and (187.06,97.75) .. (187.75,97.75) .. controls (188.44,97.75) and (189,98.31) .. (189,99) .. controls (189,99.69) and (188.44,100.25) .. (187.75,100.25) .. controls (187.06,100.25) and (186.5,99.69) .. (186.5,99) -- cycle ;
	%Shape: Circle [id:dp18300326771712694] 
	\draw  [fill={rgb, 255:red, 0; green, 0; blue, 0 }  ,fill opacity=1 ] (206.5,99) .. controls (206.5,98.31) and (207.06,97.75) .. (207.75,97.75) .. controls (208.44,97.75) and (209,98.31) .. (209,99) .. controls (209,99.69) and (208.44,100.25) .. (207.75,100.25) .. controls (207.06,100.25) and (206.5,99.69) .. (206.5,99) -- cycle ;
	%Shape: Circle [id:dp608039457212683] 
	\draw  [fill={rgb, 255:red, 0; green, 0; blue, 0 }  ,fill opacity=1 ] (226.5,99) .. controls (226.5,98.31) and (227.06,97.75) .. (227.75,97.75) .. controls (228.44,97.75) and (229,98.31) .. (229,99) .. controls (229,99.69) and (228.44,100.25) .. (227.75,100.25) .. controls (227.06,100.25) and (226.5,99.69) .. (226.5,99) -- cycle ;
	%Shape: Circle [id:dp5829775483860742] 
	\draw  [fill={rgb, 255:red, 0; green, 0; blue, 0 }  ,fill opacity=1 ] (246.5,99) .. controls (246.5,98.31) and (247.06,97.75) .. (247.75,97.75) .. controls (248.44,97.75) and (249,98.31) .. (249,99) .. controls (249,99.69) and (248.44,100.25) .. (247.75,100.25) .. controls (247.06,100.25) and (246.5,99.69) .. (246.5,99) -- cycle ;
	%Shape: Circle [id:dp8365393693985355] 
	\draw  [fill={rgb, 255:red, 0; green, 0; blue, 0 }  ,fill opacity=1 ] (266.5,99) .. controls (266.5,98.31) and (267.06,97.75) .. (267.75,97.75) .. controls (268.44,97.75) and (269,98.31) .. (269,99) .. controls (269,99.69) and (268.44,100.25) .. (267.75,100.25) .. controls (267.06,100.25) and (266.5,99.69) .. (266.5,99) -- cycle ;
	%Shape: Circle [id:dp9211812805367365] 
	\draw  [fill={rgb, 255:red, 0; green, 0; blue, 0 }  ,fill opacity=1 ] (286.5,99) .. controls (286.5,98.31) and (287.06,97.75) .. (287.75,97.75) .. controls (288.44,97.75) and (289,98.31) .. (289,99) .. controls (289,99.69) and (288.44,100.25) .. (287.75,100.25) .. controls (287.06,100.25) and (286.5,99.69) .. (286.5,99) -- cycle ;
	%Shape: Circle [id:dp7579180001977182] 
	\draw  [fill={rgb, 255:red, 0; green, 0; blue, 0 }  ,fill opacity=1 ] (306.5,99) .. controls (306.5,98.31) and (307.06,97.75) .. (307.75,97.75) .. controls (308.44,97.75) and (309,98.31) .. (309,99) .. controls (309,99.69) and (308.44,100.25) .. (307.75,100.25) .. controls (307.06,100.25) and (306.5,99.69) .. (306.5,99) -- cycle ;
	%Shape: Circle [id:dp4615158289408525] 
	\draw  [fill={rgb, 255:red, 0; green, 0; blue, 0 }  ,fill opacity=1 ] (326.5,99) .. controls (326.5,98.31) and (327.06,97.75) .. (327.75,97.75) .. controls (328.44,97.75) and (329,98.31) .. (329,99) .. controls (329,99.69) and (328.44,100.25) .. (327.75,100.25) .. controls (327.06,100.25) and (326.5,99.69) .. (326.5,99) -- cycle ;
	%Shape: Circle [id:dp46502921133551056] 
	\draw  [fill={rgb, 255:red, 0; green, 0; blue, 0 }  ,fill opacity=1 ] (346.5,99) .. controls (346.5,98.31) and (347.06,97.75) .. (347.75,97.75) .. controls (348.44,97.75) and (349,98.31) .. (349,99) .. controls (349,99.69) and (348.44,100.25) .. (347.75,100.25) .. controls (347.06,100.25) and (346.5,99.69) .. (346.5,99) -- cycle ;
	%Shape: Circle [id:dp6969228625535551] 
	\draw  [fill={rgb, 255:red, 0; green, 0; blue, 0 }  ,fill opacity=1 ] (366.5,99) .. controls (366.5,98.31) and (367.06,97.75) .. (367.75,97.75) .. controls (368.44,97.75) and (369,98.31) .. (369,99) .. controls (369,99.69) and (368.44,100.25) .. (367.75,100.25) .. controls (367.06,100.25) and (366.5,99.69) .. (366.5,99) -- cycle ;
	%Shape: Circle [id:dp7115874827419801] 
	\draw  [fill={rgb, 255:red, 0; green, 0; blue, 0 }  ,fill opacity=1 ] (386.5,99) .. controls (386.5,98.31) and (387.06,97.75) .. (387.75,97.75) .. controls (388.44,97.75) and (389,98.31) .. (389,99) .. controls (389,99.69) and (388.44,100.25) .. (387.75,100.25) .. controls (387.06,100.25) and (386.5,99.69) .. (386.5,99) -- cycle ;
	%Shape: Circle [id:dp9086685091559468] 
	\draw  [fill={rgb, 255:red, 0; green, 0; blue, 0 }  ,fill opacity=1 ] (406.5,99) .. controls (406.5,98.31) and (407.06,97.75) .. (407.75,97.75) .. controls (408.44,97.75) and (409,98.31) .. (409,99) .. controls (409,99.69) and (408.44,100.25) .. (407.75,100.25) .. controls (407.06,100.25) and (406.5,99.69) .. (406.5,99) -- cycle ;
	%Shape: Circle [id:dp026018660569895546] 
	\draw  [fill={rgb, 255:red, 0; green, 0; blue, 0 }  ,fill opacity=1 ] (426.5,99) .. controls (426.5,98.31) and (427.06,97.75) .. (427.75,97.75) .. controls (428.44,97.75) and (429,98.31) .. (429,99) .. controls (429,99.69) and (428.44,100.25) .. (427.75,100.25) .. controls (427.06,100.25) and (426.5,99.69) .. (426.5,99) -- cycle ;
	%Shape: Circle [id:dp4647458311205046] 
	\draw  [fill={rgb, 255:red, 0; green, 0; blue, 0 }  ,fill opacity=1 ] (446.5,99) .. controls (446.5,98.31) and (447.06,97.75) .. (447.75,97.75) .. controls (448.44,97.75) and (449,98.31) .. (449,99) .. controls (449,99.69) and (448.44,100.25) .. (447.75,100.25) .. controls (447.06,100.25) and (446.5,99.69) .. (446.5,99) -- cycle ;
	\draw   (144.63,205.13) -- (151.38,211.88)(151.38,205.13) -- (144.63,211.88) ;
	\draw   (164.63,205.13) -- (171.38,211.88)(171.38,205.13) -- (164.63,211.88) ;
	\draw   (184.63,205.13) -- (191.38,211.88)(191.38,205.13) -- (184.63,211.88) ;
	\draw   (204.63,205.13) -- (211.38,211.88)(211.38,205.13) -- (204.63,211.88) ;
	\draw   (224.63,205.13) -- (231.38,211.88)(231.38,205.13) -- (224.63,211.88) ;
	\draw   (244.63,205.13) -- (251.38,211.88)(251.38,205.13) -- (244.63,211.88) ;
	\draw   (264.63,205.13) -- (271.38,211.88)(271.38,205.13) -- (264.63,211.88) ;
	\draw   (284.63,205.13) -- (291.38,211.88)(291.38,205.13) -- (284.63,211.88) ;
	\draw   (304.63,205.13) -- (311.38,211.88)(311.38,205.13) -- (304.63,211.88) ;
	\draw   (324.63,205.13) -- (331.38,211.88)(331.38,205.13) -- (324.63,211.88) ;
	\draw   (344.63,205.13) -- (351.38,211.88)(351.38,205.13) -- (344.63,211.88) ;
	\draw   (364.63,205.13) -- (371.38,211.88)(371.38,205.13) -- (364.63,211.88) ;
	\draw   (384.63,205.13) -- (391.38,211.88)(391.38,205.13) -- (384.63,211.88) ;
	\draw   (404.63,205.13) -- (411.38,211.88)(411.38,205.13) -- (404.63,211.88) ;
	\draw   (424.63,205.13) -- (431.38,211.88)(431.38,205.13) -- (424.63,211.88) ;
	\draw   (444.63,205.13) -- (451.38,211.88)(451.38,205.13) -- (444.63,211.88) ;
	%Straight Lines [id:da011930091129263687] 
	\draw    (136,199) -- (465.5,199) ;
	%Straight Lines [id:da8541905804450896] 
	\draw    (213,170) -- (255.5,170) ;
	\draw [shift={(258.5,170)}, rotate = 180] [fill={rgb, 255:red, 0; green, 0; blue, 0 }  ][line width=0.08]  [draw opacity=0] (10.72,-5.15) -- (0,0) -- (10.72,5.15) -- (7.12,0) -- cycle    ;
	%Straight Lines [id:da5131266363848697] 
	\draw    (136,289) -- (465.5,289) ;
	%Shape: Circle [id:dp7651147863040559] 
	\draw  [fill={rgb, 255:red, 0; green, 0; blue, 0 }  ,fill opacity=1 ] (146.5,279) .. controls (146.5,278.31) and (147.06,277.75) .. (147.75,277.75) .. controls (148.44,277.75) and (149,278.31) .. (149,279) .. controls (149,279.69) and (148.44,280.25) .. (147.75,280.25) .. controls (147.06,280.25) and (146.5,279.69) .. (146.5,279) -- cycle ;
	%Straight Lines [id:da4325803733652427] 
	\draw    (253,320) -- (336.5,320) ;
	\draw [shift={(339.5,320)}, rotate = 180] [fill={rgb, 255:red, 0; green, 0; blue, 0 }  ][line width=0.08]  [draw opacity=0] (10.72,-5.15) -- (0,0) -- (10.72,5.15) -- (7.12,0) -- cycle    ;
	%Shape: Circle [id:dp4114711006213514] 
	\draw  [fill={rgb, 255:red, 0; green, 0; blue, 0 }  ,fill opacity=1 ] (166.5,279) .. controls (166.5,278.31) and (167.06,277.75) .. (167.75,277.75) .. controls (168.44,277.75) and (169,278.31) .. (169,279) .. controls (169,279.69) and (168.44,280.25) .. (167.75,280.25) .. controls (167.06,280.25) and (166.5,279.69) .. (166.5,279) -- cycle ;
	%Shape: Circle [id:dp6622056757528201] 
	\draw  [fill={rgb, 255:red, 0; green, 0; blue, 0 }  ,fill opacity=1 ] (186.5,279) .. controls (186.5,278.31) and (187.06,277.75) .. (187.75,277.75) .. controls (188.44,277.75) and (189,278.31) .. (189,279) .. controls (189,279.69) and (188.44,280.25) .. (187.75,280.25) .. controls (187.06,280.25) and (186.5,279.69) .. (186.5,279) -- cycle ;
	%Shape: Circle [id:dp08439361314545968] 
	\draw  [fill={rgb, 255:red, 0; green, 0; blue, 0 }  ,fill opacity=1 ] (206.5,279) .. controls (206.5,278.31) and (207.06,277.75) .. (207.75,277.75) .. controls (208.44,277.75) and (209,278.31) .. (209,279) .. controls (209,279.69) and (208.44,280.25) .. (207.75,280.25) .. controls (207.06,280.25) and (206.5,279.69) .. (206.5,279) -- cycle ;
	%Shape: Circle [id:dp208931867674091] 
	\draw  [fill={rgb, 255:red, 0; green, 0; blue, 0 }  ,fill opacity=1 ] (226.5,279) .. controls (226.5,278.31) and (227.06,277.75) .. (227.75,277.75) .. controls (228.44,277.75) and (229,278.31) .. (229,279) .. controls (229,279.69) and (228.44,280.25) .. (227.75,280.25) .. controls (227.06,280.25) and (226.5,279.69) .. (226.5,279) -- cycle ;
	%Shape: Circle [id:dp24176455894853666] 
	\draw  [fill={rgb, 255:red, 0; green, 0; blue, 0 }  ,fill opacity=1 ] (246.5,279) .. controls (246.5,278.31) and (247.06,277.75) .. (247.75,277.75) .. controls (248.44,277.75) and (249,278.31) .. (249,279) .. controls (249,279.69) and (248.44,280.25) .. (247.75,280.25) .. controls (247.06,280.25) and (246.5,279.69) .. (246.5,279) -- cycle ;
	%Shape: Circle [id:dp7134986809671351] 
	\draw  [fill={rgb, 255:red, 0; green, 0; blue, 0 }  ,fill opacity=1 ] (266.5,279) .. controls (266.5,278.31) and (267.06,277.75) .. (267.75,277.75) .. controls (268.44,277.75) and (269,278.31) .. (269,279) .. controls (269,279.69) and (268.44,280.25) .. (267.75,280.25) .. controls (267.06,280.25) and (266.5,279.69) .. (266.5,279) -- cycle ;
	%Shape: Circle [id:dp4301034087995177] 
	\draw  [fill={rgb, 255:red, 0; green, 0; blue, 0 }  ,fill opacity=1 ] (286.5,279) .. controls (286.5,278.31) and (287.06,277.75) .. (287.75,277.75) .. controls (288.44,277.75) and (289,278.31) .. (289,279) .. controls (289,279.69) and (288.44,280.25) .. (287.75,280.25) .. controls (287.06,280.25) and (286.5,279.69) .. (286.5,279) -- cycle ;
	%Shape: Circle [id:dp3141736429267601] 
	\draw  [fill={rgb, 255:red, 0; green, 0; blue, 0 }  ,fill opacity=1 ] (306.5,279) .. controls (306.5,278.31) and (307.06,277.75) .. (307.75,277.75) .. controls (308.44,277.75) and (309,278.31) .. (309,279) .. controls (309,279.69) and (308.44,280.25) .. (307.75,280.25) .. controls (307.06,280.25) and (306.5,279.69) .. (306.5,279) -- cycle ;
	%Shape: Circle [id:dp7904314939792563] 
	\draw  [fill={rgb, 255:red, 0; green, 0; blue, 0 }  ,fill opacity=1 ] (326.5,279) .. controls (326.5,278.31) and (327.06,277.75) .. (327.75,277.75) .. controls (328.44,277.75) and (329,278.31) .. (329,279) .. controls (329,279.69) and (328.44,280.25) .. (327.75,280.25) .. controls (327.06,280.25) and (326.5,279.69) .. (326.5,279) -- cycle ;
	%Shape: Circle [id:dp8568388528321926] 
	\draw  [fill={rgb, 255:red, 0; green, 0; blue, 0 }  ,fill opacity=1 ] (346.5,279) .. controls (346.5,278.31) and (347.06,277.75) .. (347.75,277.75) .. controls (348.44,277.75) and (349,278.31) .. (349,279) .. controls (349,279.69) and (348.44,280.25) .. (347.75,280.25) .. controls (347.06,280.25) and (346.5,279.69) .. (346.5,279) -- cycle ;
	%Shape: Circle [id:dp4917928487527021] 
	\draw  [fill={rgb, 255:red, 0; green, 0; blue, 0 }  ,fill opacity=1 ] (366.5,279) .. controls (366.5,278.31) and (367.06,277.75) .. (367.75,277.75) .. controls (368.44,277.75) and (369,278.31) .. (369,279) .. controls (369,279.69) and (368.44,280.25) .. (367.75,280.25) .. controls (367.06,280.25) and (366.5,279.69) .. (366.5,279) -- cycle ;
	%Shape: Circle [id:dp35777580605743875] 
	\draw  [fill={rgb, 255:red, 0; green, 0; blue, 0 }  ,fill opacity=1 ] (386.5,279) .. controls (386.5,278.31) and (387.06,277.75) .. (387.75,277.75) .. controls (388.44,277.75) and (389,278.31) .. (389,279) .. controls (389,279.69) and (388.44,280.25) .. (387.75,280.25) .. controls (387.06,280.25) and (386.5,279.69) .. (386.5,279) -- cycle ;
	%Shape: Circle [id:dp4453998005278139] 
	\draw  [fill={rgb, 255:red, 0; green, 0; blue, 0 }  ,fill opacity=1 ] (406.5,279) .. controls (406.5,278.31) and (407.06,277.75) .. (407.75,277.75) .. controls (408.44,277.75) and (409,278.31) .. (409,279) .. controls (409,279.69) and (408.44,280.25) .. (407.75,280.25) .. controls (407.06,280.25) and (406.5,279.69) .. (406.5,279) -- cycle ;
	%Shape: Circle [id:dp6724734928969023] 
	\draw  [fill={rgb, 255:red, 0; green, 0; blue, 0 }  ,fill opacity=1 ] (426.5,279) .. controls (426.5,278.31) and (427.06,277.75) .. (427.75,277.75) .. controls (428.44,277.75) and (429,278.31) .. (429,279) .. controls (429,279.69) and (428.44,280.25) .. (427.75,280.25) .. controls (427.06,280.25) and (426.5,279.69) .. (426.5,279) -- cycle ;
	%Shape: Circle [id:dp16204380586234945] 
	\draw  [fill={rgb, 255:red, 0; green, 0; blue, 0 }  ,fill opacity=1 ] (446.5,279) .. controls (446.5,278.31) and (447.06,277.75) .. (447.75,277.75) .. controls (448.44,277.75) and (449,278.31) .. (449,279) .. controls (449,279.69) and (448.44,280.25) .. (447.75,280.25) .. controls (447.06,280.25) and (446.5,279.69) .. (446.5,279) -- cycle ;
	\draw   (144.63,385.13) -- (151.38,391.88)(151.38,385.13) -- (144.63,391.88) ;
	\draw   (164.63,385.13) -- (171.38,391.88)(171.38,385.13) -- (164.63,391.88) ;
	\draw   (184.63,385.13) -- (191.38,391.88)(191.38,385.13) -- (184.63,391.88) ;
	\draw   (204.63,385.13) -- (211.38,391.88)(211.38,385.13) -- (204.63,391.88) ;
	\draw   (224.63,385.13) -- (231.38,391.88)(231.38,385.13) -- (224.63,391.88) ;
	\draw   (244.63,385.13) -- (251.38,391.88)(251.38,385.13) -- (244.63,391.88) ;
	\draw   (264.63,385.13) -- (271.38,391.88)(271.38,385.13) -- (264.63,391.88) ;
	\draw   (284.63,385.13) -- (291.38,391.88)(291.38,385.13) -- (284.63,391.88) ;
	\draw   (304.63,385.13) -- (311.38,391.88)(311.38,385.13) -- (304.63,391.88) ;
	\draw   (324.63,385.13) -- (331.38,391.88)(331.38,385.13) -- (324.63,391.88) ;
	\draw   (344.63,385.13) -- (351.38,391.88)(351.38,385.13) -- (344.63,391.88) ;
	\draw   (364.63,385.13) -- (371.38,391.88)(371.38,385.13) -- (364.63,391.88) ;
	\draw   (384.63,385.13) -- (391.38,391.88)(391.38,385.13) -- (384.63,391.88) ;
	\draw   (404.63,385.13) -- (411.38,391.88)(411.38,385.13) -- (404.63,391.88) ;
	\draw   (424.63,385.13) -- (431.38,391.88)(431.38,385.13) -- (424.63,391.88) ;
	\draw   (444.63,385.13) -- (451.38,391.88)(451.38,385.13) -- (444.63,391.88) ;
	%Straight Lines [id:da6433239934847432] 
	\draw    (136,379) -- (465.5,379) ;
	%Straight Lines [id:da1810416487609825] 
	\draw    (213,350) -- (255.5,350) ;
	\draw [shift={(258.5,350)}, rotate = 180] [fill={rgb, 255:red, 0; green, 0; blue, 0 }  ][line width=0.08]  [draw opacity=0] (10.72,-5.15) -- (0,0) -- (10.72,5.15) -- (7.12,0) -- cycle    ;

	% Text Node
	\draw (358,137) node    {$\vec{B}_{0}$};
	% Text Node
	\draw (276,169) node    {$\vec{u}_{z}$};
	% Text Node
	\draw (358,317) node    {$\vec{B}$};
	% Text Node
	\draw (276,349) node    {$\vec{u}_{z}$};


	\end{tikzpicture}
\end{figure}
\FloatBarrier
Se, a parità di corrente, inseriamo all'interno del solenoide un materiale avremo $ \vec{B}  \neq \vec{B}_0  $. Definiamo allora
\begin{gather*}
	\frac{B}{B_0}= \mu_r \qquad \text{permeabilità magnetica relativa del materiale} \\
	\mu = \mu_0 \mu_r \qquad \text{permeabilità magnetica assoluta del materiale}
\end{gather*}
Infine
\[
	\chi_m = \mu_r -1 \qquad \text{suscettività magnetica relativa del materiale}
\]
Possiamo quindi dire
\[
	\vec{B} = \mu_r \vec{B}_0 = (1 + \chi_m) \vec{B}_0 = \vec{B}_0 + \chi_m\vec{B}_0
\]
Il secondo termine rappresenta l'effetto del mezzo magnetizzato, che risulta identico a quello che sarebbe prodotto da un secondo solenoide eguale al primo, ma percorso da una corrente di densità lineare differente. Questa interpretazione non è fittizia: sulla superficie del mezzo magnetizzato esiste veramente una corrente, risultato di correnti di origine atomica che si formano per effetto del campo magnetico prodotto dalla corrente di conduzione. Tale correnti sono dette amperiane.
Il vuoto stesso può essere consderato un mezzo particolare di permeabilità magnetica relativa $ \mu_r = 1 $.

Sperimentalmente, a questo punto ci si presentano 3 casi.
\begin{itemize}
	\item \textbf{Materiali diamagnetici} (es. argento, oro, rame, elio, $\dots$) \\
	In questo caso si ha che $\boxed{ \mu_r < 1 \implies \chi_m < 0}$\\
	$ \chi_m  $ risulta indipendente dalla temperatura e ha valori compresi tra $ -10^{-5} < \chi_m < -10^{-9} $

	\item \textbf{Materiali paramagnetici} (es. alluminio, cromo, ossigeno, $\dots$) \\
	In questo caso si ha che $\boxed{ \mu_r > 1 \implies \chi_m > 0} $\\
	$ \chi_m  $ dipende dalla temperatura $ \left( \chi_m=\frac{c\rho}{T} \right)   $ (\emph{legge di Curie})  e ha valori compresi tra \\ $10^{-6}<\chi_m<10^{-2}$

	\item \textbf{Materiali ferromagnetici}\\
	In questo caso si ha che $\boxed{ \mu_r \sim [10^2\ldots 10^5]}$\\
	$ \chi_m  $ dipende dalla temperatura $ \left( \chi_m=\frac{c\rho}{T-T_c} \right) $ (\emph{legge di Curie-Weiss})\\
	Per temperature $ T>T_c $ (temperatura di Curie), i materiali ferromagnetici diventano paramagnetici.
\end{itemize}
In analogia con le cariche di polarizzazione viste per il campo elettrico, il fatto che $ \vec{B} \neq \vec{B}_0  $ si può ricondurre a delle \textbf{correnti di magnetizzazione superficiali}, dette anche correnti \emph{correnti amperiane}.







































\section{Correnti di magnetizzazione superficiali (o correnti amperiane)}

Abbiamo visto che si può dimostrare che in un materiale omogeneo isotropo esposto a un campo magnetico compaiono correnti sulla sua superficie che generano un campo magnetico che va a sommarsi a $ \vec{B}_0  $.

Consideriamo un atomo di quelli che compongono il materiale e schematizziamolo come una carica positiva attorno alla quale orbita un elettrone. L'oggetto può essere visto come un circuito microscopico percorso da corrente. Il verso della corrente equivalente è opposto rispetto alla velocità dell'elettrone. Tale circuito genera a sua volta un momento di dipolo magnetico se immerso in un campo $\vec{B}$ e si comporta come un dipolo elettrico immerso in un campo $\vec{E}$.

L'atomo può essere visto come una spira piana di area $ \pi r^2  $
Percorsa dalla corrente $ i_a  $ (verso opposto alla velocità dell'elettrone), la quale equivale, agli effetti magnetici, per il principio di equivalenza di Ampere, a un dipolo elementare di momento magnetico:
\[
	\vec{m} =IS\vec{n} = I\pi r^2 \vec{n}
\]
\begin{figure}[htpb]
	\centering
	

	\tikzset{every picture/.style={line width=0.75pt}} %set default line width to 0.75pt        

	\begin{tikzpicture}[x=0.75pt,y=0.75pt,yscale=-1,xscale=1]
	%uncomment if require: \path (0,300); %set diagram left start at 0, and has height of 300

	%Shape: Circle [id:dp2712735899479499] 
	\draw   (189,166.5) .. controls (189,114.86) and (230.86,73) .. (282.5,73) .. controls (334.14,73) and (376,114.86) .. (376,166.5) .. controls (376,218.14) and (334.14,260) .. (282.5,260) .. controls (230.86,260) and (189,218.14) .. (189,166.5) -- cycle ;
	%Straight Lines [id:da6534809017110852] 
	\draw    (282.5,73) -- (353,73) ;
	\draw [shift={(356,73)}, rotate = 180] [fill={rgb, 255:red, 0; green, 0; blue, 0 }  ][line width=0.08]  [draw opacity=0] (10.72,-5.15) -- (0,0) -- (10.72,5.15) -- (7.12,0) -- cycle    ;
	\draw   (215.7,107.76) -- (205.77,113.14) -- (207.43,101.96) ;
	%Straight Lines [id:da03521882584841718] 
	\draw  [dash pattern={on 0.84pt off 2.51pt}]  (282.5,166.5) -- (376,166.5) ;
	%Shape: Circle [id:dp8491457574334582] 
	\draw  [fill={rgb, 255:red, 0; green, 0; blue, 0 }  ,fill opacity=1 ] (230,162.33) .. controls (230,161.6) and (230.6,161) .. (231.33,161) .. controls (232.07,161) and (232.67,161.6) .. (232.67,162.33) .. controls (232.67,163.07) and (232.07,163.67) .. (231.33,163.67) .. controls (230.6,163.67) and (230,163.07) .. (230,162.33) -- cycle ;
	%Shape: Circle [id:dp8737946612729182] 
	\draw  [fill={rgb, 255:red, 0; green, 0; blue, 0 }  ,fill opacity=1 ] (270,212.33) .. controls (270,211.6) and (270.6,211) .. (271.33,211) .. controls (272.07,211) and (272.67,211.6) .. (272.67,212.33) .. controls (272.67,213.07) and (272.07,213.67) .. (271.33,213.67) .. controls (270.6,213.67) and (270,213.07) .. (270,212.33) -- cycle ;

	% Text Node
	\draw  [fill={rgb, 255:red, 222; green, 222; blue, 222 }  ,fill opacity=1 ]  (282.5, 73) circle [x radius= 15.91, y radius= 15.91]   ;
	\draw (282.5,73) node    {$-e$};
	% Text Node
	\draw (194.67,105) node    {$i_{a}$};
	% Text Node
	\draw  [fill={rgb, 255:red, 222; green, 222; blue, 222 }  ,fill opacity=1 ]  (282.5, 166.5) circle [x radius= 13.9, y radius= 13.9]   ;
	\draw (282.5,166.5) node    {$+$};
	% Text Node
	\draw (336.33,155.33) node    {$r$};
	% Text Node
	\draw (285,212) node    {$\vec{n}$};
	% Text Node
	\draw (232.33,146.67) node    {$\vec{m}$};
	% Text Node
	\draw (369.33,69.67) node    {$\vec{v}$};


	\end{tikzpicture}
\end{figure}
\FloatBarrier
Inoltre anche gli elettroni possiedono un momento di dipolo magnetico proprio, di spin, che si somma a quello orbitale. In realtà per studiare come si sommano vettorialmente i momenti di dipolo bisognerebbe passare al modello quantistico (non classico come questo).

Si parla di magnetizzazione in presenza di un momento di dipolo magnetico macroscopico nel materiale.
Esistono due meccanismi di magnetizzazione:
\begin{itemize}
	\item \textbf{Precessione di Larmor.} Dobbiamo distinguere fra campo magnetico totale macroscopico e campo magnetico locale. Quest'ultimo non comprende infatti il contributo del singolo atomo. Il moto degli elettroni intorno al nucleo può essere assimilato a correnti microscopiche, alle quali è associato un momento magnetico. Nella maggior parte dei casi questi momenti si compensano e l'atomo non ha momento magnetico, mentre quando agisce un campo magnetico esterno il moto degli elettroni è perturbato e ha origine un momento magnetico (di Larmor) che è opposto al campo esterno: questo è il meccanismo classico del diamagnetismo.
	\item \textbf{Magnetizzazione per orientamento.} In alcune sostanze vi sono condizioni di asimmetria per cui le molecole possono avere un momento magnetico intrinseco. In assenza di campo esterno c'è una distribuzione casuale dei momenti di dipolo. In assenza di perturbazione, a causa dell'agitazione termica si ha:
	\[
		\langle\vec{m} \rangle=0
	\]
	Sotto l'azione magnetica c'è un fenomeno di orientazione parziale e ha origine un momento magnetico parallelo e concorde al campo esterno, che supera l'effetto diamagnetico. Questo è ciò che accade nei materiali paramagnetici.
	\begin{figure}[htpb]
		\centering
		

		\tikzset{every picture/.style={line width=0.75pt}} %set default line width to 0.75pt        

		\begin{tikzpicture}[x=0.75pt,y=0.75pt,yscale=-1,xscale=1]
		%uncomment if require: \path (0,300); %set diagram left start at 0, and has height of 300

		%Shape: Rectangle [id:dp501883048787948] 
		\draw   (127,73) -- (465.5,73) -- (465.5,211) -- (127,211) -- cycle ;
		%Straight Lines [id:da5698452711857578] 
		\draw    (179.11,130.14) -- (205.42,93.57) ;
		\draw [shift={(207.17,91.14)}, rotate = 485.73] [fill={rgb, 255:red, 0; green, 0; blue, 0 }  ][line width=0.08]  [draw opacity=0] (10.72,-5.15) -- (0,0) -- (10.72,5.15) -- (7.12,0) -- cycle    ;
		%Shape: Ellipse [id:dp16811796547817082] 
		\draw   (159.77,116.22) .. controls (164.16,110.12) and (176.38,111.4) .. (187.07,119.09) .. controls (197.75,126.77) and (202.85,137.95) .. (198.46,144.06) .. controls (194.07,150.16) and (181.85,148.88) .. (171.16,141.2) .. controls (160.48,133.51) and (155.37,122.33) .. (159.77,116.22) -- cycle ;

		%Straight Lines [id:da5185904290633843] 
		\draw    (262.2,161.5) -- (219.37,175.46) ;
		\draw [shift={(216.52,176.39)}, rotate = 341.94] [fill={rgb, 255:red, 0; green, 0; blue, 0 }  ][line width=0.08]  [draw opacity=0] (10.72,-5.15) -- (0,0) -- (10.72,5.15) -- (7.12,0) -- cycle    ;
		%Shape: Ellipse [id:dp6467364427859881] 
		\draw   (269.59,184.16) .. controls (262.44,186.49) and (253.33,178.23) .. (249.25,165.72) .. controls (245.17,153.2) and (247.66,141.17) .. (254.81,138.84) .. controls (261.96,136.51) and (271.07,144.76) .. (275.15,157.27) .. controls (279.23,169.79) and (276.74,181.82) .. (269.59,184.16) -- cycle ;

		%Straight Lines [id:da304296926921781] 
		\draw    (315.85,117.12) -- (354.18,140.79) ;
		\draw [shift={(356.74,142.36)}, rotate = 211.69] [fill={rgb, 255:red, 0; green, 0; blue, 0 }  ][line width=0.08]  [draw opacity=0] (10.72,-5.15) -- (0,0) -- (10.72,5.15) -- (7.12,0) -- cycle    ;
		%Shape: Ellipse [id:dp658133593462523] 
		\draw   (328.37,96.84) .. controls (334.77,100.8) and (334.36,113.08) .. (327.44,124.28) .. controls (320.53,135.48) and (309.73,141.35) .. (303.33,137.4) .. controls (296.93,133.45) and (297.35,121.17) .. (304.26,109.97) .. controls (311.18,98.77) and (321.97,92.89) .. (328.37,96.84) -- cycle ;

		%Straight Lines [id:da9060060516988924] 
		\draw    (393.88,175.22) -- (425.16,142.81) ;
		\draw [shift={(427.25,140.65)}, rotate = 493.99] [fill={rgb, 255:red, 0; green, 0; blue, 0 }  ][line width=0.08]  [draw opacity=0] (10.72,-5.15) -- (0,0) -- (10.72,5.15) -- (7.12,0) -- cycle    ;
		%Shape: Ellipse [id:dp3999948978632739] 
		\draw   (376.73,158.67) .. controls (381.95,153.25) and (393.87,156.28) .. (403.34,165.42) .. controls (412.81,174.56) and (416.25,186.36) .. (411.02,191.77) .. controls (405.8,197.18) and (393.89,194.16) .. (384.42,185.02) .. controls (374.95,175.87) and (371.51,164.08) .. (376.73,158.67) -- cycle ;





		\end{tikzpicture}
	\end{figure}
	\FloatBarrier
	Tale orientamento è disturbato dalle collisioni quindi ci sarà equilibrio fra tendenza ad orientamento e tendenza a distribuzione casuale. Possiamo allora introdurre un momento magnetico medio come
	\[
		\vec{m}_L = -\alpha_L\vec{B}_L
	\]
	Qui $ \alpha  $ dipende dalla temperatura perché più essa aumenta più aumentano le collisioni. Tale dipendenza non c'era nel processo di Larmor perché si tratta di un processo che avviene nell'atomo.
	\[
		\langle \vec{m} \rangle = \alpha_0 \vec{B}_L \qquad \alpha_0\sim \frac{m_0^2}{3k_0T}
	\]
\end{itemize}
Consideriamo nel nostro materiale un punto e un volumetto $ \Delta \tau  $ e supponiamo che la somma di tutti i momenti magnetici all'interno del cubo sia un vettore $ \Delta \vec{m}  $. Definiamo vettore magnetizzazione nel punto $P$ il vettore:
\begin{gather*}
	\boxed{\vec{M} (P)=\lim_{\Delta \tau \to 0} \frac{\Delta \vec{m}}{\Delta \tau}} \\
	[M]=\frac{[I][L^2 ]}{[L^3 ]} = \frac{I}{L} \qquad \left( \frac{A}{m} \right)
\end{gather*}
Questa grandezza $A/m$ l'abbiamo già introdotta quando abbiamo parlato di correnti superficiali.
Consideriamo il caso di un cilindro di materiale uniformemente magnetizzato. Il vettore $\vec{M}$ in tutti i punti è costante in modulo, direzione e verso.
\begin{figure}[htpb]
	\centering
	

	\tikzset{every picture/.style={line width=0.75pt}} %set default line width to 0.75pt        

	\begin{tikzpicture}[x=0.75pt,y=0.75pt,yscale=-1,xscale=1]
	%uncomment if require: \path (0,486); %set diagram left start at 0, and has height of 486

	%Shape: Can [id:dp3230615374202459] 
	\draw   (285.64,247) -- (285.64,373.75) .. controls (285.64,384.38) and (253.94,393) .. (214.82,393) .. controls (175.71,393) and (144,384.38) .. (144,373.75) -- (144,247) .. controls (144,236.37) and (175.71,227.75) .. (214.82,227.75) .. controls (253.94,227.75) and (285.64,236.37) .. (285.64,247) .. controls (285.64,257.63) and (253.94,266.24) .. (214.82,266.24) .. controls (175.71,266.24) and (144,257.63) .. (144,247) ;
	%Straight Lines [id:da17259453016301607] 
	\draw  [dash pattern={on 0.84pt off 2.51pt}]  (144,247) -- (122,100.05) ;
	%Curve Lines [id:da4364396764680176] 
	\draw    (144,327) .. controls (144,353.33) and (285.33,352.67) .. (285.64,327) ;
	%Curve Lines [id:da04570346435057404] 
	\draw    (144,307) .. controls (144,333.33) and (285.33,332.67) .. (285.64,307) ;
	%Curve Lines [id:da5518993535514329] 
	\draw  [dash pattern={on 0.84pt off 2.51pt}]  (285.64,306.5) .. controls (285.64,280.16) and (144.31,280.83) .. (144,306.5) ;
	%Curve Lines [id:da40078306322834467] 
	\draw  [dash pattern={on 0.84pt off 2.51pt}]  (285.64,326.5) .. controls (285.64,300.16) and (144.31,300.83) .. (144,326.5) ;
	%Shape: Ellipse [id:dp7888143418444162] 
	\draw   (122,100.05) .. controls (122,48.97) and (163.41,7.55) .. (214.5,7.55) .. controls (265.59,7.55) and (307,48.97) .. (307,100.05) .. controls (307,151.14) and (265.59,192.55) .. (214.5,192.55) .. controls (163.41,192.55) and (122,151.14) .. (122,100.05) -- cycle ;
	%Shape: Ellipse [id:dp8462081397971879] 
	\draw   (199.87,22.18) .. controls (199.87,14.1) and (206.42,7.55) .. (214.5,7.55) .. controls (222.58,7.55) and (229.13,14.1) .. (229.13,22.18) .. controls (229.13,30.26) and (222.58,36.81) .. (214.5,36.81) .. controls (206.42,36.81) and (199.87,30.26) .. (199.87,22.18) -- cycle ;
	%Shape: Ellipse [id:dp6804585921218491] 
	\draw   (277.74,100.05) .. controls (277.74,91.97) and (284.29,85.42) .. (292.37,85.42) .. controls (300.45,85.42) and (307,91.97) .. (307,100.05) .. controls (307,108.13) and (300.45,114.68) .. (292.37,114.68) .. controls (284.29,114.68) and (277.74,108.13) .. (277.74,100.05) -- cycle ;
	%Shape: Ellipse [id:dp7704312400046178] 
	\draw   (271.3,70.8) .. controls (271.3,62.72) and (277.85,56.17) .. (285.93,56.17) .. controls (294.01,56.17) and (300.56,62.72) .. (300.56,70.8) .. controls (300.56,78.88) and (294.01,85.42) .. (285.93,85.42) .. controls (277.85,85.42) and (271.3,78.88) .. (271.3,70.8) -- cycle ;
	%Shape: Ellipse [id:dp6850228040411508] 
	\draw   (254.87,46.12) .. controls (254.87,38.04) and (261.42,31.49) .. (269.5,31.49) .. controls (277.58,31.49) and (284.13,38.04) .. (284.13,46.12) .. controls (284.13,54.2) and (277.58,60.75) .. (269.5,60.75) .. controls (261.42,60.75) and (254.87,54.2) .. (254.87,46.12) -- cycle ;
	%Shape: Ellipse [id:dp12566774187130147] 
	\draw   (229.61,28.94) .. controls (229.61,20.86) and (236.16,14.31) .. (244.23,14.31) .. controls (252.31,14.31) and (258.86,20.86) .. (258.86,28.94) .. controls (258.86,37.02) and (252.31,43.57) .. (244.23,43.57) .. controls (236.16,43.57) and (229.61,37.02) .. (229.61,28.94) -- cycle ;
	%Shape: Ellipse [id:dp40074758889659146] 
	\draw   (229.74,177.75) .. controls (229.74,185.83) and (223.19,192.37) .. (215.11,192.37) .. controls (207.03,192.37) and (200.48,185.83) .. (200.48,177.75) .. controls (200.48,169.67) and (207.03,163.12) .. (215.11,163.12) .. controls (223.19,163.12) and (229.74,169.67) .. (229.74,177.75) -- cycle ;
	%Shape: Ellipse [id:dp517785856049154] 
	\draw   (151.87,101.16) .. controls (151.87,109.24) and (145.32,115.79) .. (137.24,115.79) .. controls (129.16,115.79) and (122.61,109.24) .. (122.61,101.16) .. controls (122.61,93.08) and (129.16,86.53) .. (137.24,86.53) .. controls (145.32,86.53) and (151.87,93.08) .. (151.87,101.16) -- cycle ;
	%Shape: Ellipse [id:dp8454827342266542] 
	\draw   (158.31,130.42) .. controls (158.31,138.5) and (151.76,145.05) .. (143.68,145.05) .. controls (135.6,145.05) and (129.05,138.5) .. (129.05,130.42) .. controls (129.05,122.34) and (135.6,115.79) .. (143.68,115.79) .. controls (151.76,115.79) and (158.31,122.34) .. (158.31,130.42) -- cycle ;
	%Shape: Ellipse [id:dp5753891243085612] 
	\draw   (174.73,155.09) .. controls (174.73,163.17) and (168.18,169.72) .. (160.11,169.72) .. controls (152.03,169.72) and (145.48,163.17) .. (145.48,155.09) .. controls (145.48,147.01) and (152.03,140.46) .. (160.11,140.46) .. controls (168.18,140.46) and (174.73,147.01) .. (174.73,155.09) -- cycle ;
	%Shape: Ellipse [id:dp45554462058415024] 
	\draw   (200,172.27) .. controls (200,180.35) and (193.45,186.9) .. (185.37,186.9) .. controls (177.29,186.9) and (170.74,180.35) .. (170.74,172.27) .. controls (170.74,164.2) and (177.29,157.65) .. (185.37,157.65) .. controls (193.45,157.65) and (200,164.2) .. (200,172.27) -- cycle ;
	%Shape: Ellipse [id:dp23478204402502456] 
	\draw   (184.59,43.83) .. controls (176.51,43.83) and (169.96,37.28) .. (169.96,29.2) .. controls (169.96,21.12) and (176.51,14.57) .. (184.59,14.57) .. controls (192.67,14.57) and (199.22,21.12) .. (199.22,29.2) .. controls (199.22,37.28) and (192.67,43.83) .. (184.59,43.83) -- cycle ;
	%Shape: Ellipse [id:dp7914099804082926] 
	\draw   (159.92,60.26) .. controls (151.84,60.26) and (145.29,53.71) .. (145.29,45.63) .. controls (145.29,37.55) and (151.84,31) .. (159.92,31) .. controls (168,31) and (174.55,37.55) .. (174.55,45.63) .. controls (174.55,53.71) and (168,60.26) .. (159.92,60.26) -- cycle ;
	%Shape: Ellipse [id:dp8918275920693646] 
	\draw   (142.74,85.53) .. controls (134.66,85.53) and (128.11,78.98) .. (128.11,70.9) .. controls (128.11,62.82) and (134.66,56.27) .. (142.74,56.27) .. controls (150.82,56.27) and (157.37,62.82) .. (157.37,70.9) .. controls (157.37,78.98) and (150.82,85.53) .. (142.74,85.53) -- cycle ;
	%Shape: Ellipse [id:dp2640445250152359] 
	\draw   (244.47,156.71) .. controls (252.55,156.71) and (259.1,163.26) .. (259.1,171.34) .. controls (259.1,179.42) and (252.55,185.97) .. (244.47,185.97) .. controls (236.39,185.97) and (229.84,179.42) .. (229.84,171.34) .. controls (229.84,163.26) and (236.39,156.71) .. (244.47,156.71) -- cycle ;
	%Shape: Ellipse [id:dp4077385577120709] 
	\draw   (269.14,140.28) .. controls (277.22,140.28) and (283.77,146.83) .. (283.77,154.91) .. controls (283.77,162.99) and (277.22,169.54) .. (269.14,169.54) .. controls (261.06,169.54) and (254.51,162.99) .. (254.51,154.91) .. controls (254.51,146.83) and (261.06,140.28) .. (269.14,140.28) -- cycle ;
	%Shape: Ellipse [id:dp04802538812976165] 
	\draw   (286.32,115.01) .. controls (294.4,115.01) and (300.95,121.56) .. (300.95,129.64) .. controls (300.95,137.72) and (294.4,144.27) .. (286.32,144.27) .. controls (278.24,144.27) and (271.69,137.72) .. (271.69,129.64) .. controls (271.69,121.56) and (278.24,115.01) .. (286.32,115.01) -- cycle ;
	%Straight Lines [id:da6913923038588099] 
	\draw  [dash pattern={on 0.84pt off 2.51pt}]  (285.64,247) -- (307,100.05) ;
	%Shape: Ellipse [id:dp8052035361415502] 
	\draw   (203.35,230.47) .. controls (203.35,228.77) and (208.34,227.4) .. (214.5,227.4) .. controls (220.66,227.4) and (225.65,228.77) .. (225.65,230.47) .. controls (225.65,232.16) and (220.66,233.54) .. (214.5,233.54) .. controls (208.34,233.54) and (203.35,232.16) .. (203.35,230.47) -- cycle ;
	%Shape: Ellipse [id:dp7838484261606344] 
	\draw   (262.7,246.8) .. controls (262.7,245.11) and (267.69,243.73) .. (273.85,243.73) .. controls (280.01,243.73) and (285,245.11) .. (285,246.8) .. controls (285,248.49) and (280.01,249.87) .. (273.85,249.87) .. controls (267.69,249.87) and (262.7,248.49) .. (262.7,246.8) -- cycle ;
	%Shape: Ellipse [id:dp6774679030733199] 
	\draw   (257.79,240.66) .. controls (257.79,238.97) and (262.79,237.6) .. (268.94,237.6) .. controls (275.1,237.6) and (280.09,238.97) .. (280.09,240.66) .. controls (280.09,242.36) and (275.1,243.73) .. (268.94,243.73) .. controls (262.79,243.73) and (257.79,242.36) .. (257.79,240.66) -- cycle ;
	%Shape: Ellipse [id:dp4507601366530445] 
	\draw   (245.27,235.49) .. controls (245.27,233.79) and (250.26,232.42) .. (256.42,232.42) .. controls (262.58,232.42) and (267.57,233.79) .. (267.57,235.49) .. controls (267.57,237.18) and (262.58,238.56) .. (256.42,238.56) .. controls (250.26,238.56) and (245.27,237.18) .. (245.27,235.49) -- cycle ;
	%Shape: Ellipse [id:dp7650145431327668] 
	\draw   (226.01,231.89) .. controls (226.01,230.19) and (231,228.82) .. (237.16,228.82) .. controls (243.32,228.82) and (248.31,230.19) .. (248.31,231.89) .. controls (248.31,233.58) and (243.32,234.95) .. (237.16,234.95) .. controls (231,234.95) and (226.01,233.58) .. (226.01,231.89) -- cycle ;
	%Shape: Ellipse [id:dp3717906398797597] 
	\draw   (226.11,263.09) .. controls (226.11,264.79) and (221.12,266.16) .. (214.96,266.16) .. controls (208.81,266.16) and (203.81,264.79) .. (203.81,263.09) .. controls (203.81,261.4) and (208.81,260.03) .. (214.96,260.03) .. controls (221.12,260.03) and (226.11,261.4) .. (226.11,263.09) -- cycle ;
	%Shape: Ellipse [id:dp8203902612422316] 
	\draw   (166.76,247.03) .. controls (166.76,248.73) and (161.77,250.1) .. (155.61,250.1) .. controls (149.46,250.1) and (144.46,248.73) .. (144.46,247.03) .. controls (144.46,245.34) and (149.46,243.96) .. (155.61,243.96) .. controls (161.77,243.96) and (166.76,245.34) .. (166.76,247.03) -- cycle ;
	%Shape: Ellipse [id:dp9196990012889128] 
	\draw   (171.67,253.17) .. controls (171.67,254.86) and (166.68,256.24) .. (160.52,256.24) .. controls (154.36,256.24) and (149.37,254.86) .. (149.37,253.17) .. controls (149.37,251.47) and (154.36,250.1) .. (160.52,250.1) .. controls (166.68,250.1) and (171.67,251.47) .. (171.67,253.17) -- cycle ;
	%Shape: Ellipse [id:dp2181154557657321] 
	\draw   (184.19,258.34) .. controls (184.19,260.04) and (179.2,261.41) .. (173.04,261.41) .. controls (166.88,261.41) and (161.89,260.04) .. (161.89,258.34) .. controls (161.89,256.65) and (166.88,255.27) .. (173.04,255.27) .. controls (179.2,255.27) and (184.19,256.65) .. (184.19,258.34) -- cycle ;
	%Shape: Ellipse [id:dp24123893743436953] 
	\draw   (203.45,261.95) .. controls (203.45,263.64) and (198.46,265.02) .. (192.3,265.02) .. controls (186.14,265.02) and (181.15,263.64) .. (181.15,261.95) .. controls (181.15,260.25) and (186.14,258.88) .. (192.3,258.88) .. controls (198.46,258.88) and (203.45,260.25) .. (203.45,261.95) -- cycle ;
	%Shape: Ellipse [id:dp27847989786994987] 
	\draw   (191.71,235.01) .. controls (185.55,235.01) and (180.56,233.63) .. (180.56,231.94) .. controls (180.56,230.25) and (185.55,228.87) .. (191.71,228.87) .. controls (197.86,228.87) and (202.86,230.25) .. (202.86,231.94) .. controls (202.86,233.63) and (197.86,235.01) .. (191.71,235.01) -- cycle ;
	%Shape: Ellipse [id:dp2053955489101582] 
	\draw   (172.9,238.45) .. controls (166.74,238.45) and (161.75,237.08) .. (161.75,235.39) .. controls (161.75,233.69) and (166.74,232.32) .. (172.9,232.32) .. controls (179.06,232.32) and (184.05,233.69) .. (184.05,235.39) .. controls (184.05,237.08) and (179.06,238.45) .. (172.9,238.45) -- cycle ;
	%Shape: Ellipse [id:dp30004909590665463] 
	\draw   (159.81,243.75) .. controls (153.65,243.75) and (148.66,242.38) .. (148.66,240.68) .. controls (148.66,238.99) and (153.65,237.62) .. (159.81,237.62) .. controls (165.96,237.62) and (170.96,238.99) .. (170.96,240.68) .. controls (170.96,242.38) and (165.96,243.75) .. (159.81,243.75) -- cycle ;
	%Shape: Ellipse [id:dp918768832199206] 
	\draw   (237.34,258.68) .. controls (243.5,258.68) and (248.49,260.06) .. (248.49,261.75) .. controls (248.49,263.44) and (243.5,264.82) .. (237.34,264.82) .. controls (231.18,264.82) and (226.19,263.44) .. (226.19,261.75) .. controls (226.19,260.06) and (231.18,258.68) .. (237.34,258.68) -- cycle ;
	%Shape: Ellipse [id:dp8131837210548161] 
	\draw   (256.14,255.24) .. controls (262.3,255.24) and (267.29,256.61) .. (267.29,258.3) .. controls (267.29,260) and (262.3,261.37) .. (256.14,261.37) .. controls (249.99,261.37) and (244.99,260) .. (244.99,258.3) .. controls (244.99,256.61) and (249.99,255.24) .. (256.14,255.24) -- cycle ;
	%Shape: Ellipse [id:dp985932681474267] 
	\draw   (269.24,249.94) .. controls (275.4,249.94) and (280.39,251.31) .. (280.39,253.01) .. controls (280.39,254.7) and (275.4,256.07) .. (269.24,256.07) .. controls (263.08,256.07) and (258.09,254.7) .. (258.09,253.01) .. controls (258.09,251.31) and (263.08,249.94) .. (269.24,249.94) -- cycle ;
	\draw   (209.4,188.6) -- (216.6,192.2) -- (209.4,195.8) ;
	\draw   (243.67,183.73) -- (251.7,184.17) -- (246.53,190.33) ;
	\draw   (273.52,166.37) -- (280.95,163.28) -- (278.97,171.08) ;
	\draw   (294.71,138.8) -- (300.25,132.96) -- (301.6,140.89) ;
	\draw   (303.2,106.4) -- (306.8,99.2) -- (310.4,106.4) ;
	\draw   (219.8,11.2) -- (212.6,7.6) -- (219.8,4) ;
	\draw   (185.53,16.07) -- (177.5,15.63) -- (182.67,9.47) ;
	\draw   (155.68,33.43) -- (148.25,36.52) -- (150.23,28.72) ;
	\draw   (134.49,61) -- (128.95,66.84) -- (127.6,58.91) ;
	\draw   (126,93.4) -- (122.4,100.6) -- (118.8,93.4) ;
	\draw   (299.05,71.52) -- (299.49,63.48) -- (305.66,68.66) ;
	\draw   (281.69,41.67) -- (278.6,34.23) -- (286.4,36.22) ;
	\draw   (254.12,20.48) -- (248.28,14.94) -- (256.21,13.59) ;
	\draw   (131.29,129.59) -- (130.84,137.63) -- (124.68,132.45) ;
	\draw   (148.64,159.45) -- (151.74,166.88) -- (143.93,164.89) ;
	\draw   (176.21,180.63) -- (182.06,186.17) -- (174.12,187.52) ;
	\draw   (281.6,95.8) -- (278,103) -- (274.4,95.8) ;
	\draw   (277.69,120.2) -- (272.15,126.04) -- (270.8,118.11) ;
	\draw   (264.88,143.43) -- (257.45,146.52) -- (259.43,138.72) ;
	\draw   (243.93,159.67) -- (235.9,159.23) -- (241.07,153.07) ;
	\draw   (219,166.8) -- (211.8,163.2) -- (219,159.6) ;
	\draw   (148,105.4) -- (151.6,98.2) -- (155.2,105.4) ;
	\draw   (151.91,79) -- (157.45,73.16) -- (158.8,81.09) ;
	\draw   (165.12,56.97) -- (172.55,53.88) -- (170.57,61.68) ;
	\draw   (185.27,41.13) -- (193.3,41.57) -- (188.13,47.73) ;
	\draw   (210,33.2) -- (217.2,36.8) -- (210,40.4) ;
	\draw   (171.69,150.47) -- (168.6,143.03) -- (176.4,145.02) ;
	\draw   (155.45,130.32) -- (155.89,122.28) -- (162.06,127.46) ;
	\draw   (192.72,163.28) -- (186.88,157.74) -- (194.81,156.39) ;
	\draw   (235.41,37.43) -- (241.26,42.97) -- (233.32,44.32) ;
	\draw   (258.44,51.05) -- (261.54,58.48) -- (253.73,56.49) ;
	\draw   (274.49,71.99) -- (274.04,80.03) -- (267.88,74.85) ;
	%Straight Lines [id:da6397015603204814] 
	\draw    (214.82,247.67) -- (214.82,212) ;
	\draw [shift={(214.82,209)}, rotate = 450] [fill={rgb, 255:red, 0; green, 0; blue, 0 }  ][line width=0.08]  [draw opacity=0] (10.72,-5.15) -- (0,0) -- (10.72,5.15) -- (7.12,0) -- cycle    ;
	%Straight Lines [id:da43730851134438886] 
	\draw  [dash pattern={on 0.84pt off 2.51pt}]  (214.82,375.67) -- (214.82,247.67) ;
	%Straight Lines [id:da7501150799676759] 
	\draw    (345.49,238.33) -- (345.49,202.67) ;
	\draw [shift={(345.49,199.67)}, rotate = 450] [fill={rgb, 255:red, 0; green, 0; blue, 0 }  ][line width=0.08]  [draw opacity=0] (10.72,-5.15) -- (0,0) -- (10.72,5.15) -- (7.12,0) -- cycle    ;
	%Straight Lines [id:da8491408991065308] 
	\draw    (235.49,312.33) -- (235.49,288) ;
	\draw [shift={(235.49,285)}, rotate = 450] [fill={rgb, 255:red, 0; green, 0; blue, 0 }  ][line width=0.08]  [draw opacity=0] (10.72,-5.15) -- (0,0) -- (10.72,5.15) -- (7.12,0) -- cycle    ;
	%Straight Lines [id:da6832994824042593] 
	\draw    (305.64,327) -- (305.64,306.5) ;
	\draw [shift={(305.64,306.5)}, rotate = 450] [color={rgb, 255:red, 0; green, 0; blue, 0 }  ][line width=0.75]    (0,5.59) -- (0,-5.59)   ;
	\draw [shift={(305.64,327)}, rotate = 450] [color={rgb, 255:red, 0; green, 0; blue, 0 }  ][line width=0.75]    (0,5.59) -- (0,-5.59)   ;
	%Curve Lines [id:da3018158507425728] 
	\draw    (153,327.67) .. controls (164.34,333.46) and (181.75,336.28) .. (198.71,336.91) ;
	\draw [shift={(201.67,337)}, rotate = 181.28] [fill={rgb, 255:red, 0; green, 0; blue, 0 }  ][line width=0.08]  [draw opacity=0] (10.72,-5.15) -- (0,0) -- (10.72,5.15) -- (7.12,0) -- cycle    ;
	%Straight Lines [id:da28600001034549716] 
	\draw    (307,100.05) -- (351.2,100.05) ;
	\draw [shift={(354.2,100.05)}, rotate = 180] [fill={rgb, 255:red, 0; green, 0; blue, 0 }  ][line width=0.08]  [draw opacity=0] (10.72,-5.15) -- (0,0) -- (10.72,5.15) -- (7.12,0) -- cycle    ;
	%Straight Lines [id:da46505190487507364] 
	\draw    (307,100.05) -- (307,28.4) ;
	\draw [shift={(307,25.4)}, rotate = 450] [fill={rgb, 255:red, 0; green, 0; blue, 0 }  ][line width=0.08]  [draw opacity=0] (10.72,-5.15) -- (0,0) -- (10.72,5.15) -- (7.12,0) -- cycle    ;
	%Shape: Ellipse [id:dp9031140305915208] 
	\draw  [fill={rgb, 255:red, 0; green, 0; blue, 0 }  ,fill opacity=1 ] (213.2,100.05) .. controls (213.2,99.34) and (213.78,98.75) .. (214.5,98.75) .. controls (215.22,98.75) and (215.8,99.34) .. (215.8,100.05) .. controls (215.8,100.77) and (215.22,101.35) .. (214.5,101.35) .. controls (213.78,101.35) and (213.2,100.77) .. (213.2,100.05) -- cycle ;

	% Text Node
	\draw (230,208.83) node    {$\vec{M}$};
	% Text Node
	\draw (362.67,226.17) node    {$\vec{u}_{z}$};
	% Text Node
	\draw (252,300.83) node    {$d\vec{m}$};
	% Text Node
	\draw (324,314.83) node    {$dz$};
	% Text Node
	\draw (175.6,357.63) node    {$di_{m}$};
	% Text Node
	\draw (199,95.63) node    {$\vec{M}$};
	% Text Node
	\draw (363.87,96.57) node    {$\vec{n}$};
	% Text Node
	\draw (325.87,37.37) node    {$\vec{j}_{m}$};


	\end{tikzpicture}
\end{figure}
\FloatBarrier
Guardando la superficie, notiamo che ci sono atomi confinanti con il vuoto. Mentre quindi all'interno sui lati adiacenti le correnti si compensano, sulla superficie appare un inviluppo di tante correnti microscopiche atomiche che circolano allo stesso modo. Questo ci dà una corrente superficiale lungo la superficie del cilindro.

Consideriamo una sezione del cilindro di altezza $dz$.

Notiamo che deve essere dotato di un momento di dipolo magnetico totale somma di tutti i momenti di dipolo magnetico presenti al suo interno. Possiamo calcolare $d\vec{m}$ in due modi:
\begin{itemize}
	\item Dalla definizione di magnetizzazione $ d\vec{m} =\vec{M} d\tau =\vec{M} S\,dz = M\,S\,dz\,\vec{u}_z $
	\item Per il principio di equivalenza di Ampere $ d\vec{m} = di_m\,S\,\vec{u}_z$
\end{itemize}
Troviamo allora che
\[
	M\,S\,dz\,\vec{u}_z=di_m\,S\,\vec{u}_z \implies di_m = M\,dz \implies  M=\frac{di_m}{dz}= j_{sm}
\]
Chiamiamo $ j_{sm}  $ \textbf{densità di corrente di magnetizzazione superficiale}. Se introduciamo il versore normale alla superficie, notiamo che $\vec{j}_{sm} $, $\vec{n}$ e $\vec{M}$ costituiscono una terna ortogonale. Deduciamo che il vettore $\vec{j}_{sm} $ è allora pari a
\[
	\boxed{\vec{j}_{sm} = \vec{M} \times \vec{n}}
\]
Quando la magnetizzazione non è uniforme, ci può essere una corrente di magnetizzazione anche dentro il materiale. Considerata una superficie $\Sigma$ chiusa, immaginiamo di introdurre un campo vettoriale $\vec{v}(P)$ ovunque in questa regione. Si dimostra che
\[
	\int_{\tau}(\text{rot}\vec{v} )d\tau = \int_{\Sigma} \vec{n} \times \vec{v} \,dS
\]
Mentre in un conduttore qualunque si ha: $ \int_{\tau}J\,d\tau \neq 0  $.
Le correnti di magnetizzazione sono chiuse a livello atomico. Sommando tali correnti otterremo zero. Si deve allora avere:
\begin{align*}
	\int_{\tau}\vec{J}_md\tau &+\int_{\Sigma}\vec{j}_{sm}dS = 0 \\
	\int_{\tau}\vec{J}_md\tau &+\int_{\Sigma}\vec{M} \times \vec{n} \,dS = 0 \\
	\int_{\tau}\vec{J}_md\tau &= - \int_{\Sigma}\vec{M} \times \vec{n} \,dS = \int_{\Sigma}\vec{n} \times \vec{M} \,dS = \int_{\tau}\text{rot}\vec{M} \,d\tau
\end{align*}
Da cui
\[
	\boxed{\vec{J}_m = \text{rot}\vec{M}}
\]
Questo risultato stabilisce nel caso più generale la relazione tra il
vettore magnetizzazione e le correnti amperiane, che sono l'aspetto macroscopico delle correnti atomiche originate nel mezzo dalla presenza di un campo magnetico esterno. Se ne deduce tra l'altro che gli effetti magnetici di un mezzo magnetizzato si possono calcolare a partire da una distribuzione superficiale di corrente $\vec{j}_{sm}$ e da una distribuzione spaziale di corrente con densità $\vec{J}_m$.
\begin{figure}[htpb]
	\centering
	

	\tikzset{every picture/.style={line width=0.75pt}} %set default line width to 0.75pt        

	\begin{tikzpicture}[x=0.75pt,y=0.75pt,yscale=-1,xscale=1]
	%uncomment if require: \path (0,300); %set diagram left start at 0, and has height of 300

	%Straight Lines [id:da14089481344117294] 
	\draw    (208.49,184) -- (208.49,52) ;
	\draw [shift={(208.49,49)}, rotate = 450] [fill={rgb, 255:red, 0; green, 0; blue, 0 }  ][line width=0.08]  [draw opacity=0] (10.72,-5.15) -- (0,0) -- (10.72,5.15) -- (7.12,0) -- cycle    ;
	%Straight Lines [id:da04861389550601669] 
	\draw    (202.49,168) -- (321.95,242.41) ;
	\draw [shift={(324.5,244)}, rotate = 211.92000000000002] [fill={rgb, 255:red, 0; green, 0; blue, 0 }  ][line width=0.08]  [draw opacity=0] (10.72,-5.15) -- (0,0) -- (10.72,5.15) -- (7.12,0) -- cycle    ;
	%Straight Lines [id:da28624936293878034] 
	\draw    (214.5,168) -- (95.03,242.41) ;
	\draw [shift={(92.49,244)}, rotate = 328.08000000000004] [fill={rgb, 255:red, 0; green, 0; blue, 0 }  ][line width=0.08]  [draw opacity=0] (10.72,-5.15) -- (0,0) -- (10.72,5.15) -- (7.12,0) -- cycle    ;
	%Straight Lines [id:da8527048498119507] 
	\draw    (208.5,172) -- (403.51,153.29) ;
	\draw [shift={(406.5,153)}, rotate = 534.52] [fill={rgb, 255:red, 0; green, 0; blue, 0 }  ][line width=0.08]  [draw opacity=0] (10.72,-5.15) -- (0,0) -- (10.72,5.15) -- (7.12,0) -- cycle    ;
	%Shape: Can [id:dp4456189893883906] 
	\draw   (388.39,72.98) -- (288.26,94.65) .. controls (285.14,95.32) and (280.79,87.45) .. (278.54,77.06) .. controls (276.29,66.66) and (277,57.69) .. (280.11,57.02) -- (380.24,35.35) .. controls (383.36,34.68) and (387.71,42.55) .. (389.96,52.94) .. controls (392.21,63.34) and (391.5,72.31) .. (388.39,72.98) .. controls (385.27,73.66) and (380.92,65.78) .. (378.67,55.39) .. controls (376.42,45) and (377.13,36.03) .. (380.24,35.35) ;
	%Straight Lines [id:da936996720892316] 
	\draw    (208.5,172) -- (301.4,77.14) ;
	\draw [shift={(303.5,75)}, rotate = 494.4] [fill={rgb, 255:red, 0; green, 0; blue, 0 }  ][line width=0.08]  [draw opacity=0] (10.72,-5.15) -- (0,0) -- (10.72,5.15) -- (7.12,0) -- cycle    ;
	%Straight Lines [id:da6084536348070848] 
	\draw    (303.5,75) -- (404.11,151.19) ;
	\draw [shift={(406.5,153)}, rotate = 217.14] [fill={rgb, 255:red, 0; green, 0; blue, 0 }  ][line width=0.08]  [draw opacity=0] (10.72,-5.15) -- (0,0) -- (10.72,5.15) -- (7.12,0) -- cycle    ;
	%Straight Lines [id:da08362950852228868] 
	\draw    (306.26,67.65) -- (352.53,57.63) ;
	\draw [shift={(355.46,57)}, rotate = 527.79] [fill={rgb, 255:red, 0; green, 0; blue, 0 }  ][line width=0.08]  [draw opacity=0] (10.72,-5.15) -- (0,0) -- (10.72,5.15) -- (7.12,0) -- cycle    ;

	% Text Node
	\draw (319,176) node    {$\vec{r}$};
	% Text Node
	\draw (417,153) node    {$P$};
	% Text Node
	\draw (365,59) node    {$\vec{J}$};
	% Text Node
	\draw (290,64) node    {$\tau $};
	% Text Node
	\draw (82,244) node    {$x$};
	% Text Node
	\draw (339,241) node    {$y$};
	% Text Node
	\draw (195,47) node    {$z$};


	\end{tikzpicture}
\end{figure}
\FloatBarrier
Abbiamo visto che il potenziale vettore $\vec{A}$ è pari a:
\[
	\nabla^2 \vec{A} = -\mu_0 \vec{J}  \implies  \vec{A} (P) = \int_{\tau}\frac{\mu_0 \vec{J} (\vec{r'} )}{4\pi |\vec{r} -\vec{r'} |}d\tau
\]
Possiamo sfruttare questo risultato nel caso di un materiale magnetizzato. Per calcolare il campo magnetico prodotto da questo materiale, possiamo partire dalle correnti di magnetizzazione.
\begin{gather*}
	\vec{J}_m=\vec{M} \times \vec{n} =\text{rot}\vec{M} \\
	\vec{A} (P) = \int_{\tau}\frac{\mu_0 \vec{J}_m  (\vec{r'} )}{4\pi |\vec{r} -\vec{r'} |}d\tau + \int_{\Sigma}\frac{\mu_0 \vec{j}_{sm}  (\vec{r'} )}{4\pi |\vec{r} -\vec{r'} |}dS \qquad \vec{B} =\text{rot}\vec{A}
\end{gather*}







































\section{Vettore campo magnetico H, legge di Ampere per H}

Le equazioni generali della magnetostatica nel vuoto devono essere in parte modificate quando sono presenti mezzi magnetizzati. Resta invariata la proprietà di $\vec{B}$ di essere solenoidale mentre cambiano le equazioni in cui compiono le sorgenti, a seguito dell'introduzione delle correnti amperiane. Supponiamo di voler studiare un problema di magnetostatica di carattere generale in cui vi sia un circuito percorso da corrente immerso in un materiale magnetizzato. Vogliamo calcolare il campo induzione magnetica $\vec{B}$ generato da questa configurazione. Consideriamo una linea chiusa concatenata con il filo percorso da corrente. In questa situazione la corrente i non è l'unica concatenata con $\gamma$. Se in prossimità della linea $\gamma$ infatti c'è un atomo, avremo non solo la corrente i ma anche quella microscopica dovuta all'elettrone orbitante. Avremo:
\[
	\oint_{\gamma} \vec{B} \cdot d\vec{l} = \mu_0 I^{\text{conc. a}\gamma}  = \mu_0 (I_{\text{conduzione}}^{\gamma} + I_{\text{magnetizzazione}}^{\gamma}    )
\]
Il campo magnetico generatore dalla corrente di conduzione magnetizza il materiale portando a un cambiamento della sorgente di magnetizzazione, che a sua volta diventa poi sorgente di campo magnetico e a sua volta influenza la magnetizzazione del materiale. Per risolvere il problema ci ricordiamo che:
\begin{gather*}
	\vec{J}_m=\text{rot}\vec{M} \\
	\text{rot}\vec{B} = \mu_0 \vec{J}_{\text{tot}} = \mu_0 (\vec{J} +\vec{J}_m ) = \mu_0 \vec{J} + \mu_0 \,\text{rot}\vec{M}
\end{gather*}
Allora
\begin{align*}
	\text{rot}\frac{\vec{B}}{\mu_0}-\text{rot}\vec{M} &= \vec{J} \\
	\text{rot} \underbrace{\left( \frac{\vec{B}}{\mu_0}-\vec{M} \right)}_{\vec{H}}   &= \vec{J} \implies  \text{rot}\vec{H} = \vec{J}
\end{align*}
Il rotore dipende solo dalle correnti di conduzione che a questo punto sono note. Definiamo questa nuova quantità $\vec{H}$ campo magnetico.
\[
	\boxed{\vec{H} = \frac{\vec{B}}{\mu_0} - \vec{M}} \qquad [H]=[M] \quad \left( \frac{A}{m} \right)
\]
Per il teorema di Stokes a questo punto possiamo passare ad una legge di Ampere per il campo $\vec{H}$. Se consideriamo una linea chiusa $\gamma$ e applichiamo il teorema di Stokes alla superficie $\Sigma$ su cui $\gamma$ si poggia, dimostriamo che la circuitazione lungo $\gamma$ di $\vec{H}$ è uguale alla corrente di circuitazione
\[
	\text{rot}\vec{H} =\vec{J} \implies \oint_{\gamma} \vec{H} \cdot d\vec{l} = I_{\text{conduzione}}
\]
Formalmente la dipendenza dalle correnti amperiane è sparita. In realtà il problema è solo spostato: il loro contributo, nascosto con la riformulazione delle equazioni generali, ricompare nel legame fra $\vec{B}$ ed $ \vec{H}  $. C'è dunque bisogno dell'equazione di stato del mezzo magnetizzato, cioè della relazione specifica fra $\vec{B}$ ed $ \vec{M}  $ o $\vec{B}$ ed $ \vec{H}  $. Nei materiali dia e paramagentici, si verifica che il legame fra il campo $ \vec{H}  $ e la magnetizzazione del materiale è
\[
	\boxed{\vec{M} =\chi_m\vec{H} =(\mu_r -1)\vec{H}}
\]
Possiamo sfruttare questa relazione per determinare il legame fra $\vec{B}$ e $ \vec{H}  $.
\[
	\vec{H} =\frac{\vec{B}}{\mu_0}-\vec{M} \implies \vec{B} =\mu_0 (\vec{H} +\vec{M} )=\mu_0 (\vec{H} + \underbrace{(\mu_r -1)\vec{H}}_{\vec{M}}  ) = \mu_0 (\vec{H} +\mu_r \vec{H} -\vec{H})
\]
\[
	\boxed{\vec{B} =\mu_0 \mu_r \vec{H}}
\]
Nel vuoto, $ \mu_r =1 $.

Avevamo introdotto il solenoide e calcolato il campo di induzione magnetica. Possiamo a questo punto calcolare il campo magnetico come
\[
	\vec{B}_0 = \mu_0 \,n\,I\,\vec{u}_z \qquad \vec{H} =n\,I\,\vec{u}_z
\]
Se riempio il solenoide di materiale dia e paramagnetico $\vec{H}$ non cambia perché dipende solo dalle correnti di conduzione, sul solenoide
\begin{figure}[htpb]
	\centering
	

	\tikzset{every picture/.style={line width=0.75pt}} %set default line width to 0.75pt        

	\begin{tikzpicture}[x=0.75pt,y=0.75pt,yscale=-0.9,xscale=0.9]
	%uncomment if require: \path (0,431); %set diagram left start at 0, and has height of 431

	%Shape: Rectangle [id:dp6427071080625504] 
	\draw  [draw opacity=0][fill={rgb, 255:red, 222; green, 222; blue, 222 }  ,fill opacity=1 ] (96,68.75) -- (425.5,68.75) -- (425.5,158.75) -- (96,158.75) -- cycle ;
	%Straight Lines [id:da02536138059326687] 
	\draw    (96,68.75) -- (425.5,68.75) ;
	%Shape: Circle [id:dp1826177540774201] 
	\draw  [fill={rgb, 255:red, 0; green, 0; blue, 0 }  ,fill opacity=1 ] (106.5,58.75) .. controls (106.5,58.06) and (107.06,57.5) .. (107.75,57.5) .. controls (108.44,57.5) and (109,58.06) .. (109,58.75) .. controls (109,59.44) and (108.44,60) .. (107.75,60) .. controls (107.06,60) and (106.5,59.44) .. (106.5,58.75) -- cycle ;
	%Straight Lines [id:da7226876699060629] 
	\draw    (243,99.75) -- (326.5,99.75) ;
	\draw [shift={(329.5,99.75)}, rotate = 180] [fill={rgb, 255:red, 0; green, 0; blue, 0 }  ][line width=0.08]  [draw opacity=0] (10.72,-5.15) -- (0,0) -- (10.72,5.15) -- (7.12,0) -- cycle    ;
	%Shape: Circle [id:dp135714257827928] 
	\draw  [fill={rgb, 255:red, 0; green, 0; blue, 0 }  ,fill opacity=1 ] (126.5,58.75) .. controls (126.5,58.06) and (127.06,57.5) .. (127.75,57.5) .. controls (128.44,57.5) and (129,58.06) .. (129,58.75) .. controls (129,59.44) and (128.44,60) .. (127.75,60) .. controls (127.06,60) and (126.5,59.44) .. (126.5,58.75) -- cycle ;
	%Shape: Circle [id:dp8414141297589808] 
	\draw  [fill={rgb, 255:red, 0; green, 0; blue, 0 }  ,fill opacity=1 ] (146.5,58.75) .. controls (146.5,58.06) and (147.06,57.5) .. (147.75,57.5) .. controls (148.44,57.5) and (149,58.06) .. (149,58.75) .. controls (149,59.44) and (148.44,60) .. (147.75,60) .. controls (147.06,60) and (146.5,59.44) .. (146.5,58.75) -- cycle ;
	%Shape: Circle [id:dp06458340832650489] 
	\draw  [fill={rgb, 255:red, 0; green, 0; blue, 0 }  ,fill opacity=1 ] (166.5,58.75) .. controls (166.5,58.06) and (167.06,57.5) .. (167.75,57.5) .. controls (168.44,57.5) and (169,58.06) .. (169,58.75) .. controls (169,59.44) and (168.44,60) .. (167.75,60) .. controls (167.06,60) and (166.5,59.44) .. (166.5,58.75) -- cycle ;
	%Shape: Circle [id:dp9955754546391007] 
	\draw  [fill={rgb, 255:red, 0; green, 0; blue, 0 }  ,fill opacity=1 ] (186.5,58.75) .. controls (186.5,58.06) and (187.06,57.5) .. (187.75,57.5) .. controls (188.44,57.5) and (189,58.06) .. (189,58.75) .. controls (189,59.44) and (188.44,60) .. (187.75,60) .. controls (187.06,60) and (186.5,59.44) .. (186.5,58.75) -- cycle ;
	%Shape: Circle [id:dp6562179670055472] 
	\draw  [fill={rgb, 255:red, 0; green, 0; blue, 0 }  ,fill opacity=1 ] (206.5,58.75) .. controls (206.5,58.06) and (207.06,57.5) .. (207.75,57.5) .. controls (208.44,57.5) and (209,58.06) .. (209,58.75) .. controls (209,59.44) and (208.44,60) .. (207.75,60) .. controls (207.06,60) and (206.5,59.44) .. (206.5,58.75) -- cycle ;
	%Shape: Circle [id:dp30873013575036934] 
	\draw  [fill={rgb, 255:red, 0; green, 0; blue, 0 }  ,fill opacity=1 ] (226.5,58.75) .. controls (226.5,58.06) and (227.06,57.5) .. (227.75,57.5) .. controls (228.44,57.5) and (229,58.06) .. (229,58.75) .. controls (229,59.44) and (228.44,60) .. (227.75,60) .. controls (227.06,60) and (226.5,59.44) .. (226.5,58.75) -- cycle ;
	%Shape: Circle [id:dp41591900705823415] 
	\draw  [fill={rgb, 255:red, 0; green, 0; blue, 0 }  ,fill opacity=1 ] (246.5,58.75) .. controls (246.5,58.06) and (247.06,57.5) .. (247.75,57.5) .. controls (248.44,57.5) and (249,58.06) .. (249,58.75) .. controls (249,59.44) and (248.44,60) .. (247.75,60) .. controls (247.06,60) and (246.5,59.44) .. (246.5,58.75) -- cycle ;
	%Shape: Circle [id:dp9266837575895301] 
	\draw  [fill={rgb, 255:red, 0; green, 0; blue, 0 }  ,fill opacity=1 ] (266.5,58.75) .. controls (266.5,58.06) and (267.06,57.5) .. (267.75,57.5) .. controls (268.44,57.5) and (269,58.06) .. (269,58.75) .. controls (269,59.44) and (268.44,60) .. (267.75,60) .. controls (267.06,60) and (266.5,59.44) .. (266.5,58.75) -- cycle ;
	%Shape: Circle [id:dp28704927034916183] 
	\draw  [fill={rgb, 255:red, 0; green, 0; blue, 0 }  ,fill opacity=1 ] (286.5,58.75) .. controls (286.5,58.06) and (287.06,57.5) .. (287.75,57.5) .. controls (288.44,57.5) and (289,58.06) .. (289,58.75) .. controls (289,59.44) and (288.44,60) .. (287.75,60) .. controls (287.06,60) and (286.5,59.44) .. (286.5,58.75) -- cycle ;
	%Shape: Circle [id:dp6496748933975611] 
	\draw  [fill={rgb, 255:red, 0; green, 0; blue, 0 }  ,fill opacity=1 ] (306.5,58.75) .. controls (306.5,58.06) and (307.06,57.5) .. (307.75,57.5) .. controls (308.44,57.5) and (309,58.06) .. (309,58.75) .. controls (309,59.44) and (308.44,60) .. (307.75,60) .. controls (307.06,60) and (306.5,59.44) .. (306.5,58.75) -- cycle ;
	%Shape: Circle [id:dp7774682231925598] 
	\draw  [fill={rgb, 255:red, 0; green, 0; blue, 0 }  ,fill opacity=1 ] (326.5,58.75) .. controls (326.5,58.06) and (327.06,57.5) .. (327.75,57.5) .. controls (328.44,57.5) and (329,58.06) .. (329,58.75) .. controls (329,59.44) and (328.44,60) .. (327.75,60) .. controls (327.06,60) and (326.5,59.44) .. (326.5,58.75) -- cycle ;
	%Shape: Circle [id:dp5127696851811765] 
	\draw  [fill={rgb, 255:red, 0; green, 0; blue, 0 }  ,fill opacity=1 ] (346.5,58.75) .. controls (346.5,58.06) and (347.06,57.5) .. (347.75,57.5) .. controls (348.44,57.5) and (349,58.06) .. (349,58.75) .. controls (349,59.44) and (348.44,60) .. (347.75,60) .. controls (347.06,60) and (346.5,59.44) .. (346.5,58.75) -- cycle ;
	%Shape: Circle [id:dp5825232050499962] 
	\draw  [fill={rgb, 255:red, 0; green, 0; blue, 0 }  ,fill opacity=1 ] (366.5,58.75) .. controls (366.5,58.06) and (367.06,57.5) .. (367.75,57.5) .. controls (368.44,57.5) and (369,58.06) .. (369,58.75) .. controls (369,59.44) and (368.44,60) .. (367.75,60) .. controls (367.06,60) and (366.5,59.44) .. (366.5,58.75) -- cycle ;
	%Shape: Circle [id:dp8202403521275767] 
	\draw  [fill={rgb, 255:red, 0; green, 0; blue, 0 }  ,fill opacity=1 ] (386.5,58.75) .. controls (386.5,58.06) and (387.06,57.5) .. (387.75,57.5) .. controls (388.44,57.5) and (389,58.06) .. (389,58.75) .. controls (389,59.44) and (388.44,60) .. (387.75,60) .. controls (387.06,60) and (386.5,59.44) .. (386.5,58.75) -- cycle ;
	%Shape: Circle [id:dp7808710947800921] 
	\draw  [fill={rgb, 255:red, 0; green, 0; blue, 0 }  ,fill opacity=1 ] (406.5,58.75) .. controls (406.5,58.06) and (407.06,57.5) .. (407.75,57.5) .. controls (408.44,57.5) and (409,58.06) .. (409,58.75) .. controls (409,59.44) and (408.44,60) .. (407.75,60) .. controls (407.06,60) and (406.5,59.44) .. (406.5,58.75) -- cycle ;
	\draw   (104.62,164.88) -- (111.38,171.63)(111.38,164.88) -- (104.63,171.63) ;
	\draw   (124.62,164.88) -- (131.38,171.63)(131.38,164.88) -- (124.63,171.63) ;
	\draw   (144.63,164.88) -- (151.38,171.63)(151.38,164.88) -- (144.63,171.63) ;
	\draw   (164.63,164.88) -- (171.38,171.63)(171.38,164.88) -- (164.63,171.63) ;
	\draw   (184.63,164.88) -- (191.38,171.63)(191.38,164.88) -- (184.63,171.63) ;
	\draw   (204.63,164.88) -- (211.38,171.63)(211.38,164.88) -- (204.63,171.63) ;
	\draw   (224.63,164.88) -- (231.38,171.63)(231.38,164.88) -- (224.63,171.63) ;
	\draw   (244.63,164.88) -- (251.38,171.63)(251.38,164.88) -- (244.63,171.63) ;
	\draw   (264.63,164.88) -- (271.38,171.63)(271.38,164.88) -- (264.63,171.63) ;
	\draw   (284.63,164.88) -- (291.38,171.63)(291.38,164.88) -- (284.63,171.63) ;
	\draw   (304.63,164.88) -- (311.38,171.63)(311.38,164.88) -- (304.63,171.63) ;
	\draw   (324.63,164.88) -- (331.38,171.63)(331.38,164.88) -- (324.63,171.63) ;
	\draw   (344.63,164.88) -- (351.38,171.63)(351.38,164.88) -- (344.63,171.63) ;
	\draw   (364.63,164.88) -- (371.38,171.63)(371.38,164.88) -- (364.63,171.63) ;
	\draw   (384.63,164.88) -- (391.38,171.63)(391.38,164.88) -- (384.63,171.63) ;
	\draw   (404.63,164.88) -- (411.38,171.63)(411.38,164.88) -- (404.63,171.63) ;
	%Straight Lines [id:da5515374168158742] 
	\draw    (96,158.75) -- (425.5,158.75) ;
	%Straight Lines [id:da6001340342286905] 
	\draw    (263,129.75) -- (346.5,129.75) ;
	\draw [shift={(349.5,129.75)}, rotate = 180] [fill={rgb, 255:red, 0; green, 0; blue, 0 }  ][line width=0.08]  [draw opacity=0] (10.72,-5.15) -- (0,0) -- (10.72,5.15) -- (7.12,0) -- cycle    ;
	%Straight Lines [id:da688135203676403] 
	\draw    (113,129.75) -- (196.5,129.75) ;
	\draw [shift={(199.5,129.75)}, rotate = 180] [fill={rgb, 255:red, 0; green, 0; blue, 0 }  ][line width=0.08]  [draw opacity=0] (10.72,-5.15) -- (0,0) -- (10.72,5.15) -- (7.12,0) -- cycle    ;
	%Shape: Rectangle [id:dp4520732820043767] 
	\draw  [draw opacity=0][fill={rgb, 255:red, 222; green, 222; blue, 222 }  ,fill opacity=1 ] (96,228.75) -- (425.5,228.75) -- (425.5,318.75) -- (96,318.75) -- cycle ;
	%Straight Lines [id:da31695156333379804] 
	\draw    (96,228.75) -- (425.5,228.75) ;
	%Shape: Circle [id:dp058683178208204234] 
	\draw  [fill={rgb, 255:red, 0; green, 0; blue, 0 }  ,fill opacity=1 ] (106.5,218.75) .. controls (106.5,218.06) and (107.06,217.5) .. (107.75,217.5) .. controls (108.44,217.5) and (109,218.06) .. (109,218.75) .. controls (109,219.44) and (108.44,220) .. (107.75,220) .. controls (107.06,220) and (106.5,219.44) .. (106.5,218.75) -- cycle ;
	%Straight Lines [id:da13910800112872845] 
	\draw    (243,259.75) -- (326.5,259.75) ;
	\draw [shift={(329.5,259.75)}, rotate = 180] [fill={rgb, 255:red, 0; green, 0; blue, 0 }  ][line width=0.08]  [draw opacity=0] (10.72,-5.15) -- (0,0) -- (10.72,5.15) -- (7.12,0) -- cycle    ;
	%Shape: Circle [id:dp26864700409227416] 
	\draw  [fill={rgb, 255:red, 0; green, 0; blue, 0 }  ,fill opacity=1 ] (126.5,218.75) .. controls (126.5,218.06) and (127.06,217.5) .. (127.75,217.5) .. controls (128.44,217.5) and (129,218.06) .. (129,218.75) .. controls (129,219.44) and (128.44,220) .. (127.75,220) .. controls (127.06,220) and (126.5,219.44) .. (126.5,218.75) -- cycle ;
	%Shape: Circle [id:dp23855630787270687] 
	\draw  [fill={rgb, 255:red, 0; green, 0; blue, 0 }  ,fill opacity=1 ] (146.5,218.75) .. controls (146.5,218.06) and (147.06,217.5) .. (147.75,217.5) .. controls (148.44,217.5) and (149,218.06) .. (149,218.75) .. controls (149,219.44) and (148.44,220) .. (147.75,220) .. controls (147.06,220) and (146.5,219.44) .. (146.5,218.75) -- cycle ;
	%Shape: Circle [id:dp9682794279453872] 
	\draw  [fill={rgb, 255:red, 0; green, 0; blue, 0 }  ,fill opacity=1 ] (166.5,218.75) .. controls (166.5,218.06) and (167.06,217.5) .. (167.75,217.5) .. controls (168.44,217.5) and (169,218.06) .. (169,218.75) .. controls (169,219.44) and (168.44,220) .. (167.75,220) .. controls (167.06,220) and (166.5,219.44) .. (166.5,218.75) -- cycle ;
	%Shape: Circle [id:dp9013656584078027] 
	\draw  [fill={rgb, 255:red, 0; green, 0; blue, 0 }  ,fill opacity=1 ] (186.5,218.75) .. controls (186.5,218.06) and (187.06,217.5) .. (187.75,217.5) .. controls (188.44,217.5) and (189,218.06) .. (189,218.75) .. controls (189,219.44) and (188.44,220) .. (187.75,220) .. controls (187.06,220) and (186.5,219.44) .. (186.5,218.75) -- cycle ;
	%Shape: Circle [id:dp30744533236881044] 
	\draw  [fill={rgb, 255:red, 0; green, 0; blue, 0 }  ,fill opacity=1 ] (206.5,218.75) .. controls (206.5,218.06) and (207.06,217.5) .. (207.75,217.5) .. controls (208.44,217.5) and (209,218.06) .. (209,218.75) .. controls (209,219.44) and (208.44,220) .. (207.75,220) .. controls (207.06,220) and (206.5,219.44) .. (206.5,218.75) -- cycle ;
	%Shape: Circle [id:dp8567039311674027] 
	\draw  [fill={rgb, 255:red, 0; green, 0; blue, 0 }  ,fill opacity=1 ] (226.5,218.75) .. controls (226.5,218.06) and (227.06,217.5) .. (227.75,217.5) .. controls (228.44,217.5) and (229,218.06) .. (229,218.75) .. controls (229,219.44) and (228.44,220) .. (227.75,220) .. controls (227.06,220) and (226.5,219.44) .. (226.5,218.75) -- cycle ;
	%Shape: Circle [id:dp9957234165699325] 
	\draw  [fill={rgb, 255:red, 0; green, 0; blue, 0 }  ,fill opacity=1 ] (246.5,218.75) .. controls (246.5,218.06) and (247.06,217.5) .. (247.75,217.5) .. controls (248.44,217.5) and (249,218.06) .. (249,218.75) .. controls (249,219.44) and (248.44,220) .. (247.75,220) .. controls (247.06,220) and (246.5,219.44) .. (246.5,218.75) -- cycle ;
	%Shape: Circle [id:dp34454506519841854] 
	\draw  [fill={rgb, 255:red, 0; green, 0; blue, 0 }  ,fill opacity=1 ] (266.5,218.75) .. controls (266.5,218.06) and (267.06,217.5) .. (267.75,217.5) .. controls (268.44,217.5) and (269,218.06) .. (269,218.75) .. controls (269,219.44) and (268.44,220) .. (267.75,220) .. controls (267.06,220) and (266.5,219.44) .. (266.5,218.75) -- cycle ;
	%Shape: Circle [id:dp7997299928263821] 
	\draw  [fill={rgb, 255:red, 0; green, 0; blue, 0 }  ,fill opacity=1 ] (286.5,218.75) .. controls (286.5,218.06) and (287.06,217.5) .. (287.75,217.5) .. controls (288.44,217.5) and (289,218.06) .. (289,218.75) .. controls (289,219.44) and (288.44,220) .. (287.75,220) .. controls (287.06,220) and (286.5,219.44) .. (286.5,218.75) -- cycle ;
	%Shape: Circle [id:dp6688245831598758] 
	\draw  [fill={rgb, 255:red, 0; green, 0; blue, 0 }  ,fill opacity=1 ] (306.5,218.75) .. controls (306.5,218.06) and (307.06,217.5) .. (307.75,217.5) .. controls (308.44,217.5) and (309,218.06) .. (309,218.75) .. controls (309,219.44) and (308.44,220) .. (307.75,220) .. controls (307.06,220) and (306.5,219.44) .. (306.5,218.75) -- cycle ;
	%Shape: Circle [id:dp4250502639100042] 
	\draw  [fill={rgb, 255:red, 0; green, 0; blue, 0 }  ,fill opacity=1 ] (326.5,218.75) .. controls (326.5,218.06) and (327.06,217.5) .. (327.75,217.5) .. controls (328.44,217.5) and (329,218.06) .. (329,218.75) .. controls (329,219.44) and (328.44,220) .. (327.75,220) .. controls (327.06,220) and (326.5,219.44) .. (326.5,218.75) -- cycle ;
	%Shape: Circle [id:dp8969147365794026] 
	\draw  [fill={rgb, 255:red, 0; green, 0; blue, 0 }  ,fill opacity=1 ] (346.5,218.75) .. controls (346.5,218.06) and (347.06,217.5) .. (347.75,217.5) .. controls (348.44,217.5) and (349,218.06) .. (349,218.75) .. controls (349,219.44) and (348.44,220) .. (347.75,220) .. controls (347.06,220) and (346.5,219.44) .. (346.5,218.75) -- cycle ;
	%Shape: Circle [id:dp2579939119887864] 
	\draw  [fill={rgb, 255:red, 0; green, 0; blue, 0 }  ,fill opacity=1 ] (366.5,218.75) .. controls (366.5,218.06) and (367.06,217.5) .. (367.75,217.5) .. controls (368.44,217.5) and (369,218.06) .. (369,218.75) .. controls (369,219.44) and (368.44,220) .. (367.75,220) .. controls (367.06,220) and (366.5,219.44) .. (366.5,218.75) -- cycle ;
	%Shape: Circle [id:dp9467759531843045] 
	\draw  [fill={rgb, 255:red, 0; green, 0; blue, 0 }  ,fill opacity=1 ] (386.5,218.75) .. controls (386.5,218.06) and (387.06,217.5) .. (387.75,217.5) .. controls (388.44,217.5) and (389,218.06) .. (389,218.75) .. controls (389,219.44) and (388.44,220) .. (387.75,220) .. controls (387.06,220) and (386.5,219.44) .. (386.5,218.75) -- cycle ;
	%Shape: Circle [id:dp15470905220593067] 
	\draw  [fill={rgb, 255:red, 0; green, 0; blue, 0 }  ,fill opacity=1 ] (406.5,218.75) .. controls (406.5,218.06) and (407.06,217.5) .. (407.75,217.5) .. controls (408.44,217.5) and (409,218.06) .. (409,218.75) .. controls (409,219.44) and (408.44,220) .. (407.75,220) .. controls (407.06,220) and (406.5,219.44) .. (406.5,218.75) -- cycle ;
	\draw   (104.62,324.88) -- (111.38,331.63)(111.38,324.88) -- (104.63,331.63) ;
	\draw   (124.62,324.88) -- (131.38,331.63)(131.38,324.88) -- (124.63,331.63) ;
	\draw   (144.63,324.88) -- (151.38,331.63)(151.38,324.88) -- (144.63,331.63) ;
	\draw   (164.63,324.88) -- (171.38,331.63)(171.38,324.88) -- (164.63,331.63) ;
	\draw   (184.63,324.88) -- (191.38,331.63)(191.38,324.88) -- (184.63,331.63) ;
	\draw   (204.63,324.88) -- (211.38,331.63)(211.38,324.88) -- (204.63,331.63) ;
	\draw   (224.63,324.88) -- (231.38,331.63)(231.38,324.88) -- (224.63,331.63) ;
	\draw   (244.63,324.88) -- (251.38,331.63)(251.38,324.88) -- (244.63,331.63) ;
	\draw   (264.63,324.88) -- (271.38,331.63)(271.38,324.88) -- (264.63,331.63) ;
	\draw   (284.63,324.88) -- (291.38,331.63)(291.38,324.88) -- (284.63,331.63) ;
	\draw   (304.63,324.88) -- (311.38,331.63)(311.38,324.88) -- (304.63,331.63) ;
	\draw   (324.63,324.88) -- (331.38,331.63)(331.38,324.88) -- (324.63,331.63) ;
	\draw   (344.63,324.88) -- (351.38,331.63)(351.38,324.88) -- (344.63,331.63) ;
	\draw   (364.63,324.88) -- (371.38,331.63)(371.38,324.88) -- (364.63,331.63) ;
	\draw   (384.63,324.88) -- (391.38,331.63)(391.38,324.88) -- (384.63,331.63) ;
	\draw   (404.63,324.88) -- (411.38,331.63)(411.38,324.88) -- (404.63,331.63) ;
	%Straight Lines [id:da8339849913610828] 
	\draw    (96,318.75) -- (425.5,318.75) ;
	%Straight Lines [id:da14389562494957642] 
	\draw    (263,289.75) -- (346.5,289.75) ;
	\draw [shift={(349.5,289.75)}, rotate = 180] [fill={rgb, 255:red, 0; green, 0; blue, 0 }  ][line width=0.08]  [draw opacity=0] (10.72,-5.15) -- (0,0) -- (10.72,5.15) -- (7.12,0) -- cycle    ;
	%Straight Lines [id:da29226931720777194] 
	\draw    (146,289.75) -- (229.5,289.75) ;
	\draw [shift={(143,289.75)}, rotate = 0] [fill={rgb, 255:red, 0; green, 0; blue, 0 }  ][line width=0.08]  [draw opacity=0] (10.72,-5.15) -- (0,0) -- (10.72,5.15) -- (7.12,0) -- cycle    ;

	% Text Node
	\draw (348,96.75) node    {$\vec{B}$};
	% Text Node
	\draw (368,126.75) node    {$\vec{H}$};
	% Text Node
	\draw (218,126.75) node    {$\vec{M}$};
	% Text Node
	\draw (348,256.75) node    {$\vec{B}$};
	% Text Node
	\draw (368,286.75) node    {$\vec{H}$};
	% Text Node
	\draw (128,286.75) node    {$\vec{M}$};


	\end{tikzpicture}
\end{figure}
\FloatBarrier
In particolare, se il materiale è diamagnetico $\xi_m $è negativo e quindi $\vec{H}$ sarà diretto in verso opposto a $\vec{M}$.
Nei materiali diamagnetici l'unico effetto di magnetizzazione dominante è quello del processo di Larmor. Il momento magnetico di Larmor è diretto in verso opposto rispetto a $\vec{M}$. Ecco perché in questi materiali $\vec{H}$ è diretto in verso opposto.
Nei materiali paramagnetici il processo dominante è quello di magnetizzazione per orientamento. I dipoli si orientano nel verso del campo $\vec{B}$ e quindi anche $\vec{H}$ sarà diretto come $\vec{B}$.







































\section{Ciclo di isteresi}

Nei materiali ferromagnetici la situazione è molto più complessa. D'altra parte è nelle sostanze ferromagnetiche che gli effetti sono notevoli e rivestono importantissimi aspetti tecnologici, per cui il loro studio è molto utile. In essi $\xi_m$ è una funzione non univoca di $\vec{H}$ e non si può parlare di equazione di stato in termini semplici. La relazione tra $\vec{M}$ ed $\vec{H}$ o $\vec{B}$ ed $\vec{H}$ è espressa tramite il \textbf{ciclo di isteresi}. Le proprietà magnetiche delle leghe variano notevolmente con la composizione chimica e dipendono anche dai trattamenti termici subiti. Infine per alcune leghe speciali le proprietà magnetiche possono cambiare radicalmente sotto sollecitazioni meccaniche esterne. Tutti questi fatti inducono a ritenere che i fenomeni atomici che stanno alla base del ferromagnetismo dipendano fortemente dalla struttura cristallina e dalle sue modificazioni causate da agenti termici o meccanici. Supponiamo che inizialmente il materiale si trovi nello stato vergine, cioè non sia mai stato sottoposto a magnetizzazione, e che siano nulli tutti i campi (ci troviamo nel punto $0$ del grafico sottostante).
Facendo crescere $\vec{H}$ i valori di $\vec{B}$ e di $\vec{M}$ si dispongono lungo la curva a, detta curva di prima magnetizzazione. Quando $\vec{H}$ supera un certo valore $\vec{H}_m$ la magnetizzazione resta costante al valore $\vec{M}_{\text{sat}}$ e il campo magnetico cresce linearmente con $\vec{H}$, molto più lentamente di prima $(\vec{B} =\mu_0 (\vec{H} +\vec{M}_{\text{sat}}))$. Si dice che per $\vec{H}>\vec{H}_m$ il materiale ha raggiunto la saturazione e il valore $\vec{M}_{\text{sat}}$ si chiama magnetizzazione di saturazione. Non essendo a una retta, i valori $\mu_r$, $\mu_0$ non sono costanti, ma funzioni di $\vec{H}$. Se dopo aver raggiunto il valore $\vec{H}_m$ si fa decrescere $\vec{H}$, i valori di $\vec{B}$ e di $\vec{M}$ si dispongono lungo una nuova curva b che si mantiene al di sopra della curva di prima magnetizzazione e interseca l'asse delle ordinate ($\vec{H}=0$) con il valore $\vec{M}_r$, tale che $B_r=\mu_0 M_r$. Si parla di magnetizzazione residua e di campo magnetico residuo, a significare il fatto fondamentale che il materiale è magnetizzato anche in assenza di corrente; è diventato cioè un magnete permanente. Per annullare la magnetizzazione bisogna invertire il senso della corrente e far diminuire $\vec{H}$ fino ad $\vec{H}_c$, detto campo coercitivo, in corrispondenza del quale $\vec{M}=0$ e $B=\mu_0 H_c$.
Facendo ulteriormente decrescere $\vec{H}$ si osserva che oltre il valore $-\vec{H}_m$ la curva è rettilinea, come lo era oltre $\vec{H}_m$, con stessa pendenza. Il materiale ha raggiunto la magnetizzazione di saturazione, ma con verso opposto.
Infine, se si riporta $\vec{H}$ al valore $\vec{H}_m$ si percorre la curva $c$ fino al ricongiungimento con la curva $a$. La curva completa prende il nome di ciclo di isteresi del materiale.
\begin{figure}[htpb]
	\centering
	

	\tikzset{every picture/.style={line width=0.75pt}} %set default line width to 0.75pt        

	\begin{tikzpicture}[x=0.75pt,y=0.75pt,yscale=-1,xscale=1]
	%uncomment if require: \path (0,300); %set diagram left start at 0, and has height of 300

	%Shape: Axis 2D [id:dp7150192778192135] 
	\draw  (86,158) -- (481.5,158)(271.5,37) -- (271.5,279) (474.5,153) -- (481.5,158) -- (474.5,163) (266.5,44) -- (271.5,37) -- (276.5,44)  ;
	%Straight Lines [id:da1542418627500033] 
	\draw    (88.5,68) -- (454.5,68) ;
	%Straight Lines [id:da4097895049464837] 
	\draw    (88.5,248) -- (454.5,248) ;
	%Curve Lines [id:da521997541674877] 
	\draw [line width=1.5]    (191.5,248) .. controls (272.5,248) and (317.5,91) .. (351.5,68) ;
	\draw [shift={(285.24,172.92)}, rotate = 482.5] [fill={rgb, 255:red, 0; green, 0; blue, 0 }  ][line width=0.08]  [draw opacity=0] (13.4,-6.43) -- (0,0) -- (13.4,6.44) -- (8.9,0) -- cycle    ;
	%Curve Lines [id:da8933862632419602] 
	\draw [line width=1.5]    (191.5,248) .. controls (225.5,223) and (272.5,68) .. (351.5,68) ;
	\draw [shift={(258.55,142.37)}, rotate = 302.19] [fill={rgb, 255:red, 0; green, 0; blue, 0 }  ][line width=0.08]  [draw opacity=0] (13.4,-6.43) -- (0,0) -- (13.4,6.44) -- (8.9,0) -- cycle    ;
	%Curve Lines [id:da9894547116068164] 
	\draw [line width=1.5]    (271.5,158) .. controls (289.5,137) and (314.5,84) .. (351.5,68) ;
	\draw [shift={(306.76,107.92)}, rotate = 486.48] [fill={rgb, 255:red, 0; green, 0; blue, 0 }  ][line width=0.08]  [draw opacity=0] (13.4,-6.43) -- (0,0) -- (13.4,6.44) -- (8.9,0) -- cycle    ;
	%Straight Lines [id:da5888682740936091] 
	\draw    (265.5,123) -- (277,123) ;
	%Straight Lines [id:da5547878679057141] 
	\draw  [dash pattern={on 0.84pt off 2.51pt}]  (351.5,68) -- (351.5,157.75) ;
	%Straight Lines [id:da8719290944587592] 
	\draw  [dash pattern={on 0.84pt off 2.51pt}]  (191.5,158.25) -- (191.5,248) ;
	%Straight Lines [id:da28884191057946706] 
	\draw    (265.5,193) -- (277,193) ;

	% Text Node
	\draw (497,159) node    {$H$};
	% Text Node
	\draw (259.5,25) node    {$M$};
	% Text Node
	\draw (251,117) node    {$M_{r}$};
	% Text Node
	\draw (291,116) node    {$a$};
	% Text Node
	\draw (288,86) node    {$b$};
	% Text Node
	\draw (295.5,167) node    {$c$};
	% Text Node
	\draw (353.5,169) node    {$H_{m}$};
	% Text Node
	\draw (249,57) node    {$M_{sat}$};
	% Text Node
	\draw (247,257) node    {$-M_{sat}$};
	% Text Node
	\draw (195,145) node    {$-H_{m}$};
	% Text Node
	\draw (298,192) node    {$-M_{r}$};


	\end{tikzpicture}
\end{figure}
\FloatBarrier
Finché $\vec{H}$ varia nell'intervallo $-\vec{H}_m$, $\vec{H}_m$ o maggiore si ottiene sempre lo stesso ciclo; se si riduce l'intervallo di variabilità si ottengono cicli più stretti, con i vertici sulla curva di prima magnetizzazione. È questo il metodo che si utilizza in pratica per smagnetizzare un materiale. Uno stato $(H,M)$ può coincidere con un punto del ciclo solo se viene seguita la procedura descritta. Operando in modo opportuno tutti i punti interni al ciclo sono in realtà raggiungibili e il ciclo delimita quindi una regione luogo dei possibili stati del sistema. Ad un dato valore di $\vec{H}$ possono corrispondere infiniti valori di $\vec{B}$ compresi tra le curve $b$ e $c$, situazione che viene riassunta dicendo che la magnetizzazione di una sostanza ferromagnetica dipende dalla storia della sostanza, oltre che dal valore di $\vec{H}$.

La forma del ciclo di isteresi dipende fortemente dalla composizione della sostanza. Vi sono i materiali duri, il cui ciclo è piuttosto largo. Essi sono adatti per la costruzione di magneti permanenti, dato che è difficile smagnetizzarli. Nella situazione opposta vi sono i materiali cosiddetti dolci, che hanno un ciclo di isteresi molto stretto: è facile magnetizzarli e smagnetizzarli.







































\section{Linee di flusso per il campo H}

Vediamo cosa possiamo dire circa la divergenza di $\vec{H}$.
\[
	\text{div}\vec{H} =\text{div}\left( \frac{\vec{B}}{\mu_0}-\vec{M}  \right) = \frac{1}{\mu_0} \underbrace{\text{div}\vec{B}}_{=0}-\text{div}\vec{M} \implies \text{div}\vec{H} = - \text{div}\vec{M}
\]
La divergenza di $\vec{H}$ può essere diversa da zero. $\vec{H}$ può avere linee aperte come il campo elettrico. Consideriamo un magnete permanente. Le due facce del magente sono sorgenti delle linee di flusso per $\vec{M}$. Le linee di flusso di $\vec{H}$ si comporteranno come in figura.
\begin{figure}[htpb]
	\centering
	

	\tikzset{every picture/.style={line width=0.75pt}} %set default line width to 0.75pt        

	\begin{tikzpicture}[x=0.75pt,y=0.75pt,yscale=-1,xscale=1]
	%uncomment if require: \path (0,300); %set diagram left start at 0, and has height of 300

	%Shape: Rectangle [id:dp8343549760267048] 
	\draw  [draw opacity=0][fill={rgb, 255:red, 222; green, 222; blue, 222 }  ,fill opacity=1 ] (117,59) -- (446.5,59) -- (446.5,149) -- (117,149) -- cycle ;
	%Straight Lines [id:da47270082054582674] 
	\draw    (117,129) -- (446.5,129) ;
	\draw [shift={(281.75,129)}, rotate = 180] [fill={rgb, 255:red, 0; green, 0; blue, 0 }  ][line width=0.08]  [draw opacity=0] (10.72,-5.15) -- (0,0) -- (10.72,5.15) -- (7.12,0) -- cycle    ;
	%Straight Lines [id:da4095153772938622] 
	\draw    (117,109) -- (446.5,109) ;
	\draw [shift={(281.75,109)}, rotate = 180] [fill={rgb, 255:red, 0; green, 0; blue, 0 }  ][line width=0.08]  [draw opacity=0] (10.72,-5.15) -- (0,0) -- (10.72,5.15) -- (7.12,0) -- cycle    ;
	%Straight Lines [id:da7123076971216014] 
	\draw    (117,89) -- (446.5,89) ;
	\draw [shift={(281.75,89)}, rotate = 180] [fill={rgb, 255:red, 0; green, 0; blue, 0 }  ][line width=0.08]  [draw opacity=0] (10.72,-5.15) -- (0,0) -- (10.72,5.15) -- (7.12,0) -- cycle    ;
	%Shape: Rectangle [id:dp8459906698264861] 
	\draw  [draw opacity=0][fill={rgb, 255:red, 222; green, 222; blue, 222 }  ,fill opacity=1 ] (117,179) -- (446.5,179) -- (446.5,269) -- (117,269) -- cycle ;
	%Straight Lines [id:da7887881287067753] 
	\draw    (117,249) -- (446.5,249) ;
	\draw [shift={(281.75,249)}, rotate = 0] [fill={rgb, 255:red, 0; green, 0; blue, 0 }  ][line width=0.08]  [draw opacity=0] (10.72,-5.15) -- (0,0) -- (10.72,5.15) -- (7.12,0) -- cycle    ;
	%Straight Lines [id:da4215765570127459] 
	\draw    (117,229) -- (446.5,229) ;
	\draw [shift={(281.75,229)}, rotate = 0] [fill={rgb, 255:red, 0; green, 0; blue, 0 }  ][line width=0.08]  [draw opacity=0] (10.72,-5.15) -- (0,0) -- (10.72,5.15) -- (7.12,0) -- cycle    ;
	%Straight Lines [id:da4679148828786821] 
	\draw    (117,209) -- (446.5,209) ;
	\draw [shift={(281.75,209)}, rotate = 0] [fill={rgb, 255:red, 0; green, 0; blue, 0 }  ][line width=0.08]  [draw opacity=0] (10.72,-5.15) -- (0,0) -- (10.72,5.15) -- (7.12,0) -- cycle    ;

	% Text Node
	\draw (475,108) node    {$\vec{M}$};
	% Text Node
	\draw (475,228) node    {$\vec{H}$};


	\end{tikzpicture}
\end{figure}
\FloatBarrier
Fuori dal magnete $\vec{H}$ e $\vec{B}$ vanno nello stesso modo.
Possiamo quindi disegnare le linee di flusso di $\vec{H}$ come segue.
$\vec{H}$ ha linee aperte perché partono sulla sorgente e si richiudono sull'altra sorgente. Non possiamo percorrerle con continuità nello stesso verso.
\[
	\text{rot}\vec{H} =0 \qquad \text{div}\vec{H} =-\text{div}\vec{M}
\]
\begin{figure}[htpb]
	\centering
	

	\tikzset{every picture/.style={line width=0.75pt}} %set default line width to 0.75pt        

	\begin{tikzpicture}[x=0.75pt,y=0.75pt,yscale=-1,xscale=1]
	%uncomment if require: \path (0,300); %set diagram left start at 0, and has height of 300

	%Shape: Rectangle [id:dp5907061259874049] 
	\draw  [draw opacity=0][fill={rgb, 255:red, 222; green, 222; blue, 222 }  ,fill opacity=1 ] (141,93) -- (470.5,93) -- (470.5,177) -- (141,177) -- cycle ;
	%Straight Lines [id:da1873021714837242] 
	\draw    (141,157) -- (470.5,157) ;
	\draw [shift={(305.75,157)}, rotate = 0] [fill={rgb, 255:red, 0; green, 0; blue, 0 }  ][line width=0.08]  [draw opacity=0] (10.72,-5.15) -- (0,0) -- (10.72,5.15) -- (7.12,0) -- cycle    ;
	%Straight Lines [id:da1093458697246743] 
	\draw    (141,137) -- (470.5,137) ;
	\draw [shift={(305.75,137)}, rotate = 0] [fill={rgb, 255:red, 0; green, 0; blue, 0 }  ][line width=0.08]  [draw opacity=0] (10.72,-5.15) -- (0,0) -- (10.72,5.15) -- (7.12,0) -- cycle    ;
	%Straight Lines [id:da18072749408234423] 
	\draw    (141,117) -- (470.5,117) ;
	\draw [shift={(305.75,117)}, rotate = 0] [fill={rgb, 255:red, 0; green, 0; blue, 0 }  ][line width=0.08]  [draw opacity=0] (10.72,-5.15) -- (0,0) -- (10.72,5.15) -- (7.12,0) -- cycle    ;
	%Curve Lines [id:da45159147701622504] 
	\draw    (141,157) .. controls (61.25,221) and (184.13,250.5) .. (306.69,249.25) .. controls (429.25,248) and (551.5,216) .. (470.5,157) ;
	\draw [shift={(179.37,237.49)}, rotate = 13.92] [fill={rgb, 255:red, 0; green, 0; blue, 0 }  ][line width=0.08]  [draw opacity=0] (10.72,-5.15) -- (0,0) -- (10.72,5.15) -- (7.12,0) -- cycle    ;
	\draw [shift={(433.74,235.07)}, rotate = 346.32] [fill={rgb, 255:red, 0; green, 0; blue, 0 }  ][line width=0.08]  [draw opacity=0] (10.72,-5.15) -- (0,0) -- (10.72,5.15) -- (7.12,0) -- cycle    ;

	% Text Node
	\draw (497,126) node    {$\vec{H}$};


	\end{tikzpicture}
\end{figure}
\FloatBarrier
Possiamo applicare le tecniche conosciute per il campo elettrico al calcolo dei campi magnetici, perché le equazioni di partenza sono identiche.







































\section{Condizioni al contorno per B e H}

Consideriamo il caso di una superficie di separazione $\Sigma$ fra materiali dia o para magnetici. Supporremo che sulla superficie possa scorrere una corrente superficiale diretta in modo da uscire dal piano del foglio. Supporremo che le costanti $\mu_r$ siano definite.
\begin{figure}[htpb]
	\centering
	

	\tikzset{every picture/.style={line width=0.75pt}} %set default line width to 0.75pt        

	\begin{tikzpicture}[x=0.75pt,y=0.75pt,yscale=-0.8,xscale=0.8]
	%uncomment if require: \path (0,300); %set diagram left start at 0, and has height of 300

	%Shape: Rectangle [id:dp640128671583652] 
	\draw  [draw opacity=0][fill={rgb, 255:red, 222; green, 222; blue, 222 }  ,fill opacity=1 ] (127.5,38) -- (471.33,38) -- (471.33,155) -- (127.5,155) -- cycle ;
	%Shape: Rectangle [id:dp7767382783137311] 
	\draw  [draw opacity=0][fill={rgb, 255:red, 196; green, 196; blue, 196 }  ,fill opacity=1 ] (127.5,155) -- (471.33,155) -- (471.33,272) -- (127.5,272) -- cycle ;
	%Straight Lines [id:da3379231931271103] 
	\draw    (448.33,163.7) -- (448.33,134.5) ;
	\draw [shift={(448.33,134.5)}, rotate = 450] [color={rgb, 255:red, 0; green, 0; blue, 0 }  ][line width=0.75]    (0,5.59) -- (0,-5.59)   ;
	\draw [shift={(448.33,163.7)}, rotate = 450] [color={rgb, 255:red, 0; green, 0; blue, 0 }  ][line width=0.75]    (0,5.59) -- (0,-5.59)   ;
	%Straight Lines [id:da5256110536070768] 
	\draw    (127.5,155) -- (471.5,155) ;
	%Straight Lines [id:da2917764907431699] 
	\draw    (304.99,155.11) -- (338.19,74.61) ;
	\draw [shift={(339.33,71.83)}, rotate = 472.41] [fill={rgb, 255:red, 0; green, 0; blue, 0 }  ][line width=0.08]  [draw opacity=0] (10.72,-5.15) -- (0,0) -- (10.72,5.15) -- (7.12,0) -- cycle    ;
	%Straight Lines [id:da3617226558964304] 
	\draw    (241.33,223.17) -- (302.94,157.3) ;
	\draw [shift={(304.99,155.11)}, rotate = 493.09] [fill={rgb, 255:red, 0; green, 0; blue, 0 }  ][line width=0.08]  [draw opacity=0] (10.72,-5.15) -- (0,0) -- (10.72,5.15) -- (7.12,0) -- cycle    ;
	%Straight Lines [id:da48067503289628477] 
	\draw    (415.66,108.87) -- (455.67,108.87) ;
	\draw [shift={(458.67,108.87)}, rotate = 180] [fill={rgb, 255:red, 0; green, 0; blue, 0 }  ][line width=0.08]  [draw opacity=0] (10.72,-5.15) -- (0,0) -- (10.72,5.15) -- (7.12,0) -- cycle    ;
	%Straight Lines [id:da489443582876224] 
	\draw    (304.99,155.11) -- (304.99,75.5) ;
	\draw [shift={(304.99,72.5)}, rotate = 450] [fill={rgb, 255:red, 0; green, 0; blue, 0 }  ][line width=0.08]  [draw opacity=0] (10.72,-5.15) -- (0,0) -- (10.72,5.15) -- (7.12,0) -- cycle    ;
	%Straight Lines [id:da7541095946452916] 
	\draw  [dash pattern={on 0.84pt off 2.51pt}]  (304.99,71.83) -- (339.33,71.83) ;
	%Straight Lines [id:da29810868951967096] 
	\draw  [dash pattern={on 0.84pt off 2.51pt}]  (241.33,223.17) -- (304.99,223.17) ;
	%Shape: Can [id:dp9154320986569371] 
	\draw   (391,135.27) -- (391,164.2) .. controls (391,167.62) and (354.36,170.4) .. (309.17,170.4) .. controls (263.97,170.4) and (227.33,167.62) .. (227.33,164.2) -- (227.33,135.27) .. controls (227.33,131.84) and (263.97,129.07) .. (309.17,129.07) .. controls (354.36,129.07) and (391,131.84) .. (391,135.27) .. controls (391,138.69) and (354.36,141.47) .. (309.17,141.47) .. controls (263.97,141.47) and (227.33,138.69) .. (227.33,135.27) ;
	%Straight Lines [id:da3186356193706956] 
	\draw    (362.33,134.77) -- (362.33,89.33) ;
	\draw [shift={(362.33,86.33)}, rotate = 450] [fill={rgb, 255:red, 0; green, 0; blue, 0 }  ][line width=0.08]  [draw opacity=0] (10.72,-5.15) -- (0,0) -- (10.72,5.15) -- (7.12,0) -- cycle    ;
	%Straight Lines [id:da7404993802099071] 
	\draw    (362.33,215.77) -- (362.33,168.53) ;
	\draw [shift={(362.33,218.77)}, rotate = 270] [fill={rgb, 255:red, 0; green, 0; blue, 0 }  ][line width=0.08]  [draw opacity=0] (10.72,-5.15) -- (0,0) -- (10.72,5.15) -- (7.12,0) -- cycle    ;
	%Straight Lines [id:da23941741198591937] 
	\draw  [dash pattern={on 0.84pt off 2.51pt}]  (362.33,168.53) -- (362.33,159.53) ;
	%Shape: Circle [id:dp9226117747947826] 
	\draw  [fill={rgb, 255:red, 0; green, 0; blue, 0 }  ,fill opacity=1 ] (155.5,154.75) .. controls (155.5,154.06) and (156.06,153.5) .. (156.75,153.5) .. controls (157.44,153.5) and (158,154.06) .. (158,154.75) .. controls (158,155.44) and (157.44,156) .. (156.75,156) .. controls (156.06,156) and (155.5,155.44) .. (155.5,154.75) -- cycle ;
	%Shape: Circle [id:dp49321895896161183] 
	\draw  [fill={rgb, 255:red, 0; green, 0; blue, 0 }  ,fill opacity=1 ] (175.5,154.75) .. controls (175.5,154.06) and (176.06,153.5) .. (176.75,153.5) .. controls (177.44,153.5) and (178,154.06) .. (178,154.75) .. controls (178,155.44) and (177.44,156) .. (176.75,156) .. controls (176.06,156) and (175.5,155.44) .. (175.5,154.75) -- cycle ;
	%Shape: Circle [id:dp05606420159460934] 
	\draw  [fill={rgb, 255:red, 0; green, 0; blue, 0 }  ,fill opacity=1 ] (195.5,154.75) .. controls (195.5,154.06) and (196.06,153.5) .. (196.75,153.5) .. controls (197.44,153.5) and (198,154.06) .. (198,154.75) .. controls (198,155.44) and (197.44,156) .. (196.75,156) .. controls (196.06,156) and (195.5,155.44) .. (195.5,154.75) -- cycle ;
	%Shape: Circle [id:dp15915469483255507] 
	\draw  [fill={rgb, 255:red, 0; green, 0; blue, 0 }  ,fill opacity=1 ] (215.5,154.75) .. controls (215.5,154.06) and (216.06,153.5) .. (216.75,153.5) .. controls (217.44,153.5) and (218,154.06) .. (218,154.75) .. controls (218,155.44) and (217.44,156) .. (216.75,156) .. controls (216.06,156) and (215.5,155.44) .. (215.5,154.75) -- cycle ;
	%Shape: Circle [id:dp981414568315562] 
	\draw  [fill={rgb, 255:red, 0; green, 0; blue, 0 }  ,fill opacity=1 ] (235.5,154.75) .. controls (235.5,154.06) and (236.06,153.5) .. (236.75,153.5) .. controls (237.44,153.5) and (238,154.06) .. (238,154.75) .. controls (238,155.44) and (237.44,156) .. (236.75,156) .. controls (236.06,156) and (235.5,155.44) .. (235.5,154.75) -- cycle ;
	%Shape: Circle [id:dp8020916372170195] 
	\draw  [fill={rgb, 255:red, 0; green, 0; blue, 0 }  ,fill opacity=1 ] (255.5,154.75) .. controls (255.5,154.06) and (256.06,153.5) .. (256.75,153.5) .. controls (257.44,153.5) and (258,154.06) .. (258,154.75) .. controls (258,155.44) and (257.44,156) .. (256.75,156) .. controls (256.06,156) and (255.5,155.44) .. (255.5,154.75) -- cycle ;
	%Shape: Circle [id:dp9837915663115668] 
	\draw  [fill={rgb, 255:red, 0; green, 0; blue, 0 }  ,fill opacity=1 ] (275.5,154.75) .. controls (275.5,154.06) and (276.06,153.5) .. (276.75,153.5) .. controls (277.44,153.5) and (278,154.06) .. (278,154.75) .. controls (278,155.44) and (277.44,156) .. (276.75,156) .. controls (276.06,156) and (275.5,155.44) .. (275.5,154.75) -- cycle ;
	%Shape: Circle [id:dp7422574202222258] 
	\draw  [fill={rgb, 255:red, 0; green, 0; blue, 0 }  ,fill opacity=1 ] (295.5,154.75) .. controls (295.5,154.06) and (296.06,153.5) .. (296.75,153.5) .. controls (297.44,153.5) and (298,154.06) .. (298,154.75) .. controls (298,155.44) and (297.44,156) .. (296.75,156) .. controls (296.06,156) and (295.5,155.44) .. (295.5,154.75) -- cycle ;
	%Shape: Circle [id:dp962308728843092] 
	\draw  [fill={rgb, 255:red, 0; green, 0; blue, 0 }  ,fill opacity=1 ] (315.5,154.75) .. controls (315.5,154.06) and (316.06,153.5) .. (316.75,153.5) .. controls (317.44,153.5) and (318,154.06) .. (318,154.75) .. controls (318,155.44) and (317.44,156) .. (316.75,156) .. controls (316.06,156) and (315.5,155.44) .. (315.5,154.75) -- cycle ;
	%Shape: Circle [id:dp217945754335507] 
	\draw  [fill={rgb, 255:red, 0; green, 0; blue, 0 }  ,fill opacity=1 ] (335.5,154.75) .. controls (335.5,154.06) and (336.06,153.5) .. (336.75,153.5) .. controls (337.44,153.5) and (338,154.06) .. (338,154.75) .. controls (338,155.44) and (337.44,156) .. (336.75,156) .. controls (336.06,156) and (335.5,155.44) .. (335.5,154.75) -- cycle ;
	%Shape: Circle [id:dp5568851600454661] 
	\draw  [fill={rgb, 255:red, 0; green, 0; blue, 0 }  ,fill opacity=1 ] (355.5,154.75) .. controls (355.5,154.06) and (356.06,153.5) .. (356.75,153.5) .. controls (357.44,153.5) and (358,154.06) .. (358,154.75) .. controls (358,155.44) and (357.44,156) .. (356.75,156) .. controls (356.06,156) and (355.5,155.44) .. (355.5,154.75) -- cycle ;
	%Shape: Circle [id:dp9213331130002607] 
	\draw  [fill={rgb, 255:red, 0; green, 0; blue, 0 }  ,fill opacity=1 ] (375.5,154.75) .. controls (375.5,154.06) and (376.06,153.5) .. (376.75,153.5) .. controls (377.44,153.5) and (378,154.06) .. (378,154.75) .. controls (378,155.44) and (377.44,156) .. (376.75,156) .. controls (376.06,156) and (375.5,155.44) .. (375.5,154.75) -- cycle ;
	%Shape: Circle [id:dp764392096031741] 
	\draw  [fill={rgb, 255:red, 0; green, 0; blue, 0 }  ,fill opacity=1 ] (395.5,154.75) .. controls (395.5,154.06) and (396.06,153.5) .. (396.75,153.5) .. controls (397.44,153.5) and (398,154.06) .. (398,154.75) .. controls (398,155.44) and (397.44,156) .. (396.75,156) .. controls (396.06,156) and (395.5,155.44) .. (395.5,154.75) -- cycle ;
	%Shape: Circle [id:dp053231376142844455] 
	\draw  [fill={rgb, 255:red, 0; green, 0; blue, 0 }  ,fill opacity=1 ] (415.5,154.75) .. controls (415.5,154.06) and (416.06,153.5) .. (416.75,153.5) .. controls (417.44,153.5) and (418,154.06) .. (418,154.75) .. controls (418,155.44) and (417.44,156) .. (416.75,156) .. controls (416.06,156) and (415.5,155.44) .. (415.5,154.75) -- cycle ;
	%Shape: Circle [id:dp5881828482322411] 
	\draw  [fill={rgb, 255:red, 0; green, 0; blue, 0 }  ,fill opacity=1 ] (435.5,154.75) .. controls (435.5,154.06) and (436.06,153.5) .. (436.75,153.5) .. controls (437.44,153.5) and (438,154.06) .. (438,154.75) .. controls (438,155.44) and (437.44,156) .. (436.75,156) .. controls (436.06,156) and (435.5,155.44) .. (435.5,154.75) -- cycle ;
	%Shape: Circle [id:dp029196096719917852] 
	\draw  [fill={rgb, 255:red, 0; green, 0; blue, 0 }  ,fill opacity=1 ] (455.5,154.75) .. controls (455.5,154.06) and (456.06,153.5) .. (456.75,153.5) .. controls (457.44,153.5) and (458,154.06) .. (458,154.75) .. controls (458,155.44) and (457.44,156) .. (456.75,156) .. controls (456.06,156) and (455.5,155.44) .. (455.5,154.75) -- cycle ;
	%Straight Lines [id:da9185749469986559] 
	\draw    (304.99,158.11) -- (304.99,223.17) ;
	\draw [shift={(304.99,155.11)}, rotate = 90] [fill={rgb, 255:red, 0; green, 0; blue, 0 }  ][line width=0.08]  [draw opacity=0] (10.72,-5.15) -- (0,0) -- (10.72,5.15) -- (7.12,0) -- cycle    ;

	% Text Node
	\draw (449.33,174.07) node    {$dh$};
	% Text Node
	\draw (403.67,170.67) node    {$\Sigma _{g}$};
	% Text Node
	\draw (112,125) node    {$1$};
	% Text Node
	\draw (112,186) node    {$2$};
	% Text Node
	\draw (341,103.33) node    {$\vec{B}_{1}$};
	% Text Node
	\draw (242.33,196) node    {$\vec{B}_{2}$};
	% Text Node
	\draw (486.67,155) node    {$\Sigma $};
	% Text Node
	\draw (437,94.77) node    {$\vec{u}_{t}$};
	% Text Node
	\draw (146.5,125) node    {$\mu _{r1}$};
	% Text Node
	\draw (146.5,173.33) node    {$\mu _{r2}$};
	% Text Node
	\draw (286.47,63.07) node    {$\vec{B}_{n1}$};
	% Text Node
	\draw (328.67,207.4) node    {$\vec{B}_{n2}$};
	% Text Node
	\draw (377.93,86.87) node    {$\vec{n}_{1}$};
	% Text Node
	\draw (380.33,213.33) node    {$\vec{n}_{2}$};
	% Text Node
	\draw (248,115.33) node    {$dS$};
	% Text Node
	\draw (189.5,124.33) node    {$\vec{j}_{s,cond.}$};


	\end{tikzpicture}
\end{figure}
\FloatBarrier
Consideriamo una superficie cilindrica a cavallo della superficie e applichiamo il teorema di Gauss ad essa.
\begin{gather*}
	\text{div}\vec{B} =0 \implies \Phi_{\Sigma_g}(\vec{B} )=0 \qquad \sqrt{dS} \gg dh \\
	0=\Phi_{\Sigma_g}(\vec{B} ) \sim \vec{B}_1 \cdot \vec{n}_1\,dS + \vec{B}_2 \cdot \vec{n}_2\,dS = \vec{B}_1 \cdot \vec{n}_1\,dS - \vec{B}_2 \cdot \vec{n}_1\,dS
\end{gather*}
Possiamo chiamare
\begin{gather*}
	B_{1n}=\vec{B}_1\cdot \vec{n}_1\\
	B_{2n}=\vec{B}_2\cdot \vec{n}_1
\end{gather*}
Allora avremo
\[
	B_{1n}-B_{2n}=0 \implies \boxed{B_{1n}=B_{2n}}
\]
Anche in questo caso la componente normale di $\vec{B}$ rispetto alla superficie di separazione si conserva. Vediamo che cosa accade ad $\vec{H}$. Conosciamo la legge che lega $\vec{B}$ ed $\vec{H}$.
\[
	\mu_0 \mu_{r1} H_{1n} = \mu_0 \mu_{r2} H_{2n} \implies \boxed{\mu_{r1} H_{1n} = \mu_{r2} H_{2n}}
\]
Consideriamo a questo punto un percorso chiuso $\gamma$ rettangolare a cavallo delle superfici.
\begin{figure}[htpb]
	\centering
	

	\tikzset{every picture/.style={line width=0.75pt}} %set default line width to 0.75pt        

	\begin{tikzpicture}[x=0.75pt,y=0.75pt,yscale=-0.8,xscale=0.8]
	%uncomment if require: \path (0,300); %set diagram left start at 0, and has height of 300

	%Shape: Rectangle [id:dp8889811844496274] 
	\draw  [draw opacity=0][fill={rgb, 255:red, 196; green, 196; blue, 196 }  ,fill opacity=1 ] (154,161) -- (483.33,161) -- (483.33,278) -- (154,278) -- cycle ;
	%Shape: Rectangle [id:dp8154040016170578] 
	\draw  [draw opacity=0][fill={rgb, 255:red, 222; green, 222; blue, 222 }  ,fill opacity=1 ] (154,44) -- (483.33,44) -- (483.33,161) -- (154,161) -- cycle ;
	%Straight Lines [id:da9217572029779877] 
	\draw    (154,161) -- (483.5,161) ;
	%Straight Lines [id:da25473814943791107] 
	\draw    (336.99,161.11) -- (370.19,80.61) ;
	\draw [shift={(371.33,77.83)}, rotate = 472.41] [fill={rgb, 255:red, 0; green, 0; blue, 0 }  ][line width=0.08]  [draw opacity=0] (10.72,-5.15) -- (0,0) -- (10.72,5.15) -- (7.12,0) -- cycle    ;
	%Straight Lines [id:da49271486183733937] 
	\draw    (273.33,229.17) -- (334.94,163.3) ;
	\draw [shift={(336.99,161.11)}, rotate = 493.09] [fill={rgb, 255:red, 0; green, 0; blue, 0 }  ][line width=0.08]  [draw opacity=0] (10.72,-5.15) -- (0,0) -- (10.72,5.15) -- (7.12,0) -- cycle    ;
	%Straight Lines [id:da5567500542194554] 
	\draw    (300.33,145.33) -- (440.33,145.33) ;
	\draw [shift={(370.33,145.33)}, rotate = 180] [fill={rgb, 255:red, 0; green, 0; blue, 0 }  ][line width=0.08]  [draw opacity=0] (10.72,-5.15) -- (0,0) -- (10.72,5.15) -- (7.12,0) -- cycle    ;
	%Straight Lines [id:da5436907378732236] 
	\draw    (260.33,174) -- (440.33,174) ;
	\draw [shift={(350.33,174)}, rotate = 0] [fill={rgb, 255:red, 0; green, 0; blue, 0 }  ][line width=0.08]  [draw opacity=0] (10.72,-5.15) -- (0,0) -- (10.72,5.15) -- (7.12,0) -- cycle    ;
	%Straight Lines [id:da10143937001629277] 
	\draw    (440.33,145.33) -- (440.33,174) ;
	%Straight Lines [id:da08569580649940645] 
	\draw    (260.33,145.33) -- (260.33,174) ;
	%Straight Lines [id:da7138367568727535] 
	\draw    (402.66,119.77) -- (442.67,119.77) ;
	\draw [shift={(445.67,119.77)}, rotate = 180] [fill={rgb, 255:red, 0; green, 0; blue, 0 }  ][line width=0.08]  [draw opacity=0] (10.72,-5.15) -- (0,0) -- (10.72,5.15) -- (7.12,0) -- cycle    ;
	%Straight Lines [id:da05949564858271583] 
	\draw  [dash pattern={on 0.84pt off 2.51pt}]  (336.99,229.17) -- (336.99,78.5) ;
	%Straight Lines [id:da5732763015102671] 
	\draw    (336.99,77.83) -- (368.33,77.83) ;
	\draw [shift={(371.33,77.83)}, rotate = 180] [fill={rgb, 255:red, 0; green, 0; blue, 0 }  ][line width=0.08]  [draw opacity=0] (10.72,-5.15) -- (0,0) -- (10.72,5.15) -- (7.12,0) -- cycle    ;
	%Straight Lines [id:da4618494416402683] 
	\draw    (273.33,229.17) -- (333.99,229.17) ;
	\draw [shift={(336.99,229.17)}, rotate = 180] [fill={rgb, 255:red, 0; green, 0; blue, 0 }  ][line width=0.08]  [draw opacity=0] (10.72,-5.15) -- (0,0) -- (10.72,5.15) -- (7.12,0) -- cycle    ;
	%Shape: Circle [id:dp46729674948235456] 
	\draw  [fill={rgb, 255:red, 0; green, 0; blue, 0 }  ,fill opacity=1 ] (166.5,160.75) .. controls (166.5,160.06) and (167.06,159.5) .. (167.75,159.5) .. controls (168.44,159.5) and (169,160.06) .. (169,160.75) .. controls (169,161.44) and (168.44,162) .. (167.75,162) .. controls (167.06,162) and (166.5,161.44) .. (166.5,160.75) -- cycle ;
	%Shape: Circle [id:dp8132634527969875] 
	\draw  [fill={rgb, 255:red, 0; green, 0; blue, 0 }  ,fill opacity=1 ] (186.5,160.75) .. controls (186.5,160.06) and (187.06,159.5) .. (187.75,159.5) .. controls (188.44,159.5) and (189,160.06) .. (189,160.75) .. controls (189,161.44) and (188.44,162) .. (187.75,162) .. controls (187.06,162) and (186.5,161.44) .. (186.5,160.75) -- cycle ;
	%Shape: Circle [id:dp8025783798209374] 
	\draw  [fill={rgb, 255:red, 0; green, 0; blue, 0 }  ,fill opacity=1 ] (206.5,160.75) .. controls (206.5,160.06) and (207.06,159.5) .. (207.75,159.5) .. controls (208.44,159.5) and (209,160.06) .. (209,160.75) .. controls (209,161.44) and (208.44,162) .. (207.75,162) .. controls (207.06,162) and (206.5,161.44) .. (206.5,160.75) -- cycle ;
	%Shape: Circle [id:dp25029115254588685] 
	\draw  [fill={rgb, 255:red, 0; green, 0; blue, 0 }  ,fill opacity=1 ] (226.5,160.75) .. controls (226.5,160.06) and (227.06,159.5) .. (227.75,159.5) .. controls (228.44,159.5) and (229,160.06) .. (229,160.75) .. controls (229,161.44) and (228.44,162) .. (227.75,162) .. controls (227.06,162) and (226.5,161.44) .. (226.5,160.75) -- cycle ;
	%Shape: Circle [id:dp5547302977144766] 
	\draw  [fill={rgb, 255:red, 0; green, 0; blue, 0 }  ,fill opacity=1 ] (246.5,160.75) .. controls (246.5,160.06) and (247.06,159.5) .. (247.75,159.5) .. controls (248.44,159.5) and (249,160.06) .. (249,160.75) .. controls (249,161.44) and (248.44,162) .. (247.75,162) .. controls (247.06,162) and (246.5,161.44) .. (246.5,160.75) -- cycle ;
	%Shape: Circle [id:dp33969588247409366] 
	\draw  [fill={rgb, 255:red, 0; green, 0; blue, 0 }  ,fill opacity=1 ] (266.5,160.75) .. controls (266.5,160.06) and (267.06,159.5) .. (267.75,159.5) .. controls (268.44,159.5) and (269,160.06) .. (269,160.75) .. controls (269,161.44) and (268.44,162) .. (267.75,162) .. controls (267.06,162) and (266.5,161.44) .. (266.5,160.75) -- cycle ;
	%Shape: Circle [id:dp6564194490201183] 
	\draw  [fill={rgb, 255:red, 0; green, 0; blue, 0 }  ,fill opacity=1 ] (286.5,160.75) .. controls (286.5,160.06) and (287.06,159.5) .. (287.75,159.5) .. controls (288.44,159.5) and (289,160.06) .. (289,160.75) .. controls (289,161.44) and (288.44,162) .. (287.75,162) .. controls (287.06,162) and (286.5,161.44) .. (286.5,160.75) -- cycle ;
	%Shape: Circle [id:dp037619467526890604] 
	\draw  [fill={rgb, 255:red, 0; green, 0; blue, 0 }  ,fill opacity=1 ] (306.5,160.75) .. controls (306.5,160.06) and (307.06,159.5) .. (307.75,159.5) .. controls (308.44,159.5) and (309,160.06) .. (309,160.75) .. controls (309,161.44) and (308.44,162) .. (307.75,162) .. controls (307.06,162) and (306.5,161.44) .. (306.5,160.75) -- cycle ;
	%Shape: Circle [id:dp9895523027397202] 
	\draw  [fill={rgb, 255:red, 0; green, 0; blue, 0 }  ,fill opacity=1 ] (326.5,160.75) .. controls (326.5,160.06) and (327.06,159.5) .. (327.75,159.5) .. controls (328.44,159.5) and (329,160.06) .. (329,160.75) .. controls (329,161.44) and (328.44,162) .. (327.75,162) .. controls (327.06,162) and (326.5,161.44) .. (326.5,160.75) -- cycle ;
	%Shape: Circle [id:dp5800186845267619] 
	\draw  [fill={rgb, 255:red, 0; green, 0; blue, 0 }  ,fill opacity=1 ] (346.5,160.75) .. controls (346.5,160.06) and (347.06,159.5) .. (347.75,159.5) .. controls (348.44,159.5) and (349,160.06) .. (349,160.75) .. controls (349,161.44) and (348.44,162) .. (347.75,162) .. controls (347.06,162) and (346.5,161.44) .. (346.5,160.75) -- cycle ;
	%Shape: Circle [id:dp6910511077291763] 
	\draw  [fill={rgb, 255:red, 0; green, 0; blue, 0 }  ,fill opacity=1 ] (366.5,160.75) .. controls (366.5,160.06) and (367.06,159.5) .. (367.75,159.5) .. controls (368.44,159.5) and (369,160.06) .. (369,160.75) .. controls (369,161.44) and (368.44,162) .. (367.75,162) .. controls (367.06,162) and (366.5,161.44) .. (366.5,160.75) -- cycle ;
	%Shape: Circle [id:dp9964405128303655] 
	\draw  [fill={rgb, 255:red, 0; green, 0; blue, 0 }  ,fill opacity=1 ] (386.5,160.75) .. controls (386.5,160.06) and (387.06,159.5) .. (387.75,159.5) .. controls (388.44,159.5) and (389,160.06) .. (389,160.75) .. controls (389,161.44) and (388.44,162) .. (387.75,162) .. controls (387.06,162) and (386.5,161.44) .. (386.5,160.75) -- cycle ;
	%Shape: Circle [id:dp31251253187467376] 
	\draw  [fill={rgb, 255:red, 0; green, 0; blue, 0 }  ,fill opacity=1 ] (406.5,160.75) .. controls (406.5,160.06) and (407.06,159.5) .. (407.75,159.5) .. controls (408.44,159.5) and (409,160.06) .. (409,160.75) .. controls (409,161.44) and (408.44,162) .. (407.75,162) .. controls (407.06,162) and (406.5,161.44) .. (406.5,160.75) -- cycle ;
	%Shape: Circle [id:dp8429011669329816] 
	\draw  [fill={rgb, 255:red, 0; green, 0; blue, 0 }  ,fill opacity=1 ] (426.5,160.75) .. controls (426.5,160.06) and (427.06,159.5) .. (427.75,159.5) .. controls (428.44,159.5) and (429,160.06) .. (429,160.75) .. controls (429,161.44) and (428.44,162) .. (427.75,162) .. controls (427.06,162) and (426.5,161.44) .. (426.5,160.75) -- cycle ;
	%Shape: Circle [id:dp575471172390928] 
	\draw  [fill={rgb, 255:red, 0; green, 0; blue, 0 }  ,fill opacity=1 ] (446.5,160.75) .. controls (446.5,160.06) and (447.06,159.5) .. (447.75,159.5) .. controls (448.44,159.5) and (449,160.06) .. (449,160.75) .. controls (449,161.44) and (448.44,162) .. (447.75,162) .. controls (447.06,162) and (446.5,161.44) .. (446.5,160.75) -- cycle ;
	%Shape: Circle [id:dp6087226715450742] 
	\draw  [fill={rgb, 255:red, 0; green, 0; blue, 0 }  ,fill opacity=1 ] (466.5,160.75) .. controls (466.5,160.06) and (467.06,159.5) .. (467.75,159.5) .. controls (468.44,159.5) and (469,160.06) .. (469,160.75) .. controls (469,161.44) and (468.44,162) .. (467.75,162) .. controls (467.06,162) and (466.5,161.44) .. (466.5,160.75) -- cycle ;
	%Straight Lines [id:da9848989975392397] 
	\draw    (260.33,145.33) -- (300.33,145.33) ;

	% Text Node
	\draw (134,131) node    {$1$};
	% Text Node
	\draw (134,192) node    {$2$};
	% Text Node
	\draw (373,109.33) node    {$\vec{B}_{1}$};
	% Text Node
	\draw (274.33,202) node    {$\vec{B}_{2}$};
	% Text Node
	\draw (498.67,161) node    {$\Sigma $};
	% Text Node
	\draw (424,105.67) node    {$\vec{u}_{t}$};
	% Text Node
	\draw (284,130) node    {$\gamma $};
	% Text Node
	\draw (224.5,129.33) node    {$\vec{j}_{s,cond.}$};
	% Text Node
	\draw (177.5,131) node    {$\mu _{r1}$};
	% Text Node
	\draw (177.5,179.33) node    {$\mu _{r2}$};
	% Text Node
	\draw (349.67,60.67) node    {$\vec{B}_{t1}$};
	% Text Node
	\draw (308.67,245) node    {$\vec{B}_{t2}$};


	\end{tikzpicture}
\end{figure}
\FloatBarrier
Siamo interessati alla componente tangente quindi dobbiamo applicare la legge di Ampere:
\begin{gather*}
	\oint_{\gamma} \vec{H} \cdot d\vec{l} = I_{\text{conduzione}}^{\text{conc. a}\gamma} \qquad dh \ll dl \\
	\vec{H}_1\cdot dl\,\vec{u}_t - \vec{H}_2\cdot dl\,\vec{u}_t =j^{\text{cond.}} dl \implies \boxed{H_{1t}-H_{2t}=j^{\text{cond.}}}
\end{gather*}
Possiamo ricavare cosa succede alla componente tangente di $\vec{B}$:
\[
	\frac{B_{1t}}{\mu_0 \mu_{r1}} - \frac{B_{2t}}{\mu_0 \mu_{r2}} = j^{\text{cond.}}  \implies \boxed{\frac{B_{1t}}{\mu_{r1}} - \frac{B_{2t}}{\mu_{r2}} = \mu_0 \,j^{\text{cond.}}}
\]
Le correnti di magnetizzazione concorrono ai campi magnetici. Avendo due mezzi diversi ci saranno correnti di magnetizzazione superficiali




























































\chapter{Campi elettrici e magnetici lentamente variabili}

Il ritardo di propagazione è breve perché la velocità della luce è altissima. Il fenomeno è lentamente variabile se la scala temporale su cui si presenta la corrente è molto più lunga del ritardo temporale di propagazione dell'informazione. In regime lentamente variabile facciamo finta che $\vec{B}$ giunga istantaneamente nel punto in cui stiamo calcolando il campo.
\[
	\vec{B} (P)=\vec{B} (t-\tau) \simeq \vec{B} (t)=\frac{\mu_0 I(t)}{2\pi d}\vec{u}_{\varphi}
\]
I ritardi diventano essenziali quando si affronta la propagazione di onde elettromagnetiche.
Le proprietà locali dei campi elettrici e magnetici costanti nel tempo sono stabilite nel vuoto dalle quattro equazioni di Maxwell studiate. La condizione di stazionarietà è evidenziata dal fatto che nelle equazioni compaiono solo le derivate rispetto alle coordinate spaziali e non quelle rispetto al tempo. Apparte il fatto che il campo magnetico statico $\vec{B}$ è generato da cariche elettriche in movimento, non esiste in un dato sistema di riferimento inerziale nessun'altra connessione tra i fenomeni elettrici e magnetici statici e le relative coppie di equazioni possono essere risolte separatamente. Esperimenti condotti da Faraday e da Henry separatamente misero in evidenza una diversa connessione tra elettricità e magnetismo: un campo magnetico variabile nel tempo genera un campo elettrico non conservativo che in opportuni dispositivi può dar luogo a una forza elettromotrice e ad una corrente in un circuito chiuso. Maxwell arrivò poi a una forma più generale delle equazioni che regolano i fenomeni elettrici e magnetici visibili, che contengono le quattro equazioni già studiate per il caso stazionario come caso limite. Caratteristica fondamentale è che un campo elettrico e un campo magnetico variabili non possono esistere separatamente, ma vanno riuniti sotto il concetto più generale di campo elettromagnetico.







































\section{Induzione elettromagnetica}

Dal momento che abbiamo studiato il fenomeno dell'induzione elettrostatica, potremmo chiederci se esiste un fenomeno analogo anche nel caso del magnetismo.
Se si avvicina un magnete ad una spira $A$ connessa con un galvanometro, il suo indice si sposta in una certa direzione, se lo si allontana, la direzione dello spostamento sarà quella opposta. Quando il magnete è fermo rispetto alla spira non si osserva nessuno spostamento dell'indice dello strumento. Già da questo fatto si intuisce che in una situazione in quiete non può essere indotta una corrente su un filo. Gli effetti sono uguali se si tiene fermo il magnete e si avvicina o si allontana la spira. Se sostituiamo il magnete con una spira $\vec{B}$ in cui è inserito un generatore che fa circolare corrente e muoviamo $\vec{B}$ rispetto ad $A$ o $A$ rispetto a $\vec{B}$, si ottiene lo stesso effetto. Da tali osservazioni si può concludere che in una spira compare una corrente, che chiamiamo indotta, ogni qual volta c'è un moto relativo tra la spira e un campo magnetico $\vec{B}$, generato da un magnete permanente o da un'altra spira percorsa da corrente.
Dall'esame quantitativo dei casi descritti e di tutte le altre situazioni in cui si manifesta il fenomeno dell'induzione, Faraday dedusse che \emph{ogni qual volta il flusso del campo magnetico $\Phi (\vec{B})$ concatenato con un circuito varia nel tempo, si ha nel circuito una forza elettromotrice indotta data dall'opposto della derivata del flusso rispetto al tempo}:
\[
	\boxed{f_i = - \frac{d\Phi_{\Sigma}(\vec{B} )}{dt}}
\]
Tale risultato è detto \textbf{legge di Faraday-Neumann}.
La forza elettromotrice è definita come integrale del campo elettrico $\vec{E}$ lungo una linea chiusa, cioè come la circuitazione di $ \vec{E}  $
\[
	f_i=\oint_{\gamma} \vec{E}_e  \cdot d\vec{l} \neq 0
\]
e un suo valore non nullo implica che il campo elettrico non è conservativo.
\begin{figure}[htpb]
	\centering
	

	\tikzset{every picture/.style={line width=0.75pt}} %set default line width to 0.75pt        

	\begin{tikzpicture}[x=0.75pt,y=0.75pt,yscale=-1,xscale=1]
	%uncomment if require: \path (0,300); %set diagram left start at 0, and has height of 300

	%Curve Lines [id:da9896045425165532] 
	\draw    (210,211) .. controls (202.5,271) and (425.5,263) .. (418.5,214) ;
	\draw [shift={(314.06,253.38)}, rotate = 539.9200000000001] [fill={rgb, 255:red, 0; green, 0; blue, 0 }  ][line width=0.08]  [draw opacity=0] (10.72,-5.15) -- (0,0) -- (10.72,5.15) -- (7.12,0) -- cycle    ;
	%Curve Lines [id:da47209199424878445] 
	\draw  [dash pattern={on 0.84pt off 2.51pt}]  (210,211) .. controls (211.5,163) and (416.5,174) .. (418.5,214) ;
	%Curve Lines [id:da13680608874761813] 
	\draw    (210,211) .. controls (211.5,40) and (409.5,48) .. (418.5,214) ;
	%Straight Lines [id:da8930151161079218] 
	\draw    (248.67,109) -- (219.78,69.91) ;
	\draw [shift={(218,67.5)}, rotate = 413.53999999999996] [fill={rgb, 255:red, 0; green, 0; blue, 0 }  ][line width=0.08]  [draw opacity=0] (10.72,-5.15) -- (0,0) -- (10.72,5.15) -- (7.12,0) -- cycle    ;
	%Curve Lines [id:da39004395489350374] 
	\draw    (346.5,275) .. controls (343.5,207) and (367.5,109) .. (429.5,79) ;
	\draw [shift={(362.83,164.96)}, rotate = 468.8] [fill={rgb, 255:red, 0; green, 0; blue, 0 }  ][line width=0.08]  [draw opacity=0] (10.72,-5.15) -- (0,0) -- (10.72,5.15) -- (7.12,0) -- cycle    ;
	%Curve Lines [id:da616401806299091] 
	\draw    (276.5,271) .. controls (284.5,202) and (314.5,113) .. (364.5,42) ;
	\draw [shift={(307,151.29)}, rotate = 470.75] [fill={rgb, 255:red, 0; green, 0; blue, 0 }  ][line width=0.08]  [draw opacity=0] (10.72,-5.15) -- (0,0) -- (10.72,5.15) -- (7.12,0) -- cycle    ;

	% Text Node
	\draw (439.67,148) node    {$\Sigma _{\text{aperta}}$};
	% Text Node
	\draw (203.67,71) node    {$\vec{n}$};
	% Text Node
	\draw (384.67,36) node    {$\vec{B}( t)$};
	% Text Node
	\draw (315.67,230) node    {$\vec{E}_{e}$};


	\end{tikzpicture}
\end{figure}
\FloatBarrier

\emph{Esempio.}
Immaginiamo di considerare un circuito $ \gamma  $ a cui diamo un verso arbitrario di percorrenza. Individuiamo una superficie $\Sigma$ aperta che abbia $\gamma$ come orlo. Associamo con la regola del cavatappi a $\Sigma$ una normale. Supponiamo che in questa regione dello spazio sia presente un campo magnetico variabile nel tempo. Chiamiamo il flusso concatenato al circuito flusso di $\vec{B}$ attraverso la superficie $\Sigma$ (non dipende dalla superficie). Se la variazione di campo magnetico produce una forza elettromotrice nel circuito, possiamo immaginare che vi sia un campo elettromotore $\vec{E}_e $. Possiamo calcolare la \textbf{forza elettromotrice indotta}:
\[
	f_i = - \frac{d\Phi_{\Sigma}(\vec{B} )}{dt}
\]
La legge afferma che tale $fem$ è pari all'opposto derivata rispetto al tempo del flusso concatenato al circuito.
Tutte le casistiche viste separatamente si riuniscono sotto quest'unica legge. Il segno meno può essere interpretato in forma pratica. La forza elettromotrice si oppone alla causa che la sta generando. La corrente gira in modo da generare un campo magnetico che si oppone all'aumento di flusso. Viceversa se $\vec{B}$ sta diminuendo (e quindi anche il flusso) la corrente indotta cerca di rafforzare il flusso. Questo particolare enunciato è noto come \textbf{legge di Lenz}.

Esaminiamo ora come si realizza una variazione di flusso nel tempo, elencando prima le varie possibilità:
\begin{itemize}
	\item si ha sempre una variazione del flusso quando un circuito indeformabile si muove (trasla, ruota) in un campo magnetico. L'unica eccezione è il caso di moto traslatorio in un campo uniforme.
	\item Deformazione del circuito. Si ha il moto di una sua singola parte o anche soltanto di qualche parte. Il flusso concatenato con il circuito in generale cambia nel tempo. Il fenomeno avviene sia con un campo uniforme che con un campo non uniforme.
	\item Si mantiene il circuito fisso e si sposta la sorgente con del campo magnetico.
	\item Con circuito fisso e sorgenti di $\vec{B}$ fisse il flusso concatenato cambia nel tempo se si muove un mezzo ferromagnetico magnetizzato.
	\item In assenza di qualsiasi moto relativo tra circuito e campo magnetico e di variazioni locali di permeabilità magnetica, si ha una variazione di flusso attraverso il circuito se il campo magnetico, uniforme o no, varia nel tempo a causa della variazione nel tempo dell'intensità della corrente che lo genera.
\end{itemize}
Supponiamo che ci sia una sorgente di campo magnetico non uniforme. Poniamo che in tale regione dello spazio sia presente un circuito a cui diamo un verso di percorrenza e una normale alla superficie $\Sigma$ avente $\gamma$ come contorno. Il circuito in quiete viene spostato di moto di pura traslazione nella regione in cui è presente campo magnetico con velocità $\vec{v}$. Consideriamo una delle varie cariche presenti nel circuito.
\begin{figure}[htpb]
	\centering
	

	% Pattern Info
	 
	\tikzset{
	pattern size/.store in=\mcSize, 
	pattern size = 5pt,
	pattern thickness/.store in=\mcThickness, 
	pattern thickness = 0.3pt,
	pattern radius/.store in=\mcRadius, 
	pattern radius = 1pt}
	\makeatletter
	\pgfutil@ifundefined{pgf@pattern@name@_rlc9ed9hs}{
	\pgfdeclarepatternformonly[\mcThickness,\mcSize]{_rlc9ed9hs}
	{\pgfqpoint{0pt}{0pt}}
	{\pgfpoint{\mcSize+\mcThickness}{\mcSize+\mcThickness}}
	{\pgfpoint{\mcSize}{\mcSize}}
	{
	\pgfsetcolor{\tikz@pattern@color}
	\pgfsetlinewidth{\mcThickness}
	\pgfpathmoveto{\pgfqpoint{0pt}{0pt}}
	\pgfpathlineto{\pgfpoint{\mcSize+\mcThickness}{\mcSize+\mcThickness}}
	\pgfusepath{stroke}
	}}
	\makeatother
	\tikzset{every picture/.style={line width=0.75pt}} %set default line width to 0.75pt        

	\begin{tikzpicture}[x=0.75pt,y=0.75pt,yscale=-0.7,xscale=0.7]
	%uncomment if require: \path (0,300); %set diagram left start at 0, and has height of 300

	%Shape: Rectangle [id:dp19540465549659736] 
	\draw   (50,135) -- (132,135) -- (132,164) -- (50,164) -- cycle ;
	%Shape: Rectangle [id:dp9635511785626469] 
	\draw  [fill={rgb, 255:red, 222; green, 222; blue, 222 }  ,fill opacity=1 ] (132,135) -- (161,135) -- (161,164) -- (132,164) -- cycle ;
	%Shape: Rectangle [id:dp43083401531840404] 
	\draw  [fill={rgb, 255:red, 222; green, 222; blue, 222 }  ,fill opacity=1 ] (21,135) -- (50,135) -- (50,164) -- (21,164) -- cycle ;
	%Curve Lines [id:da9230471581894757] 
	\draw    (161,140) .. controls (217,120.5) and (246,105.5) .. (294,51.5) ;
	\draw [shift={(234.74,105.91)}, rotate = 506.83] [fill={rgb, 255:red, 0; green, 0; blue, 0 }  ][line width=0.08]  [draw opacity=0] (10.72,-5.15) -- (0,0) -- (10.72,5.15) -- (7.12,0) -- cycle    ;
	%Curve Lines [id:da7402333729566795] 
	\draw    (161,157.5) .. controls (217,177) and (246,192) .. (294,246) ;
	\draw [shift={(234.74,191.59)}, rotate = 213.17000000000002] [fill={rgb, 255:red, 0; green, 0; blue, 0 }  ][line width=0.08]  [draw opacity=0] (10.72,-5.15) -- (0,0) -- (10.72,5.15) -- (7.12,0) -- cycle    ;
	%Curve Lines [id:da756664137603634] 
	\draw    (161,145.35) .. controls (217,137.78) and (246,131.96) .. (294,111) ;
	\draw [shift={(229.28,133.5)}, rotate = 526.77] [fill={rgb, 255:red, 0; green, 0; blue, 0 }  ][line width=0.08]  [draw opacity=0] (10.72,-5.15) -- (0,0) -- (10.72,5.15) -- (7.12,0) -- cycle    ;
	%Curve Lines [id:da016613918082758694] 
	\draw    (161,152.15) .. controls (217,159.72) and (246,165.54) .. (294,186.5) ;
	\draw [shift={(229.28,164)}, rotate = 193.23] [fill={rgb, 255:red, 0; green, 0; blue, 0 }  ][line width=0.08]  [draw opacity=0] (10.72,-5.15) -- (0,0) -- (10.72,5.15) -- (7.12,0) -- cycle    ;
	%Shape: Ellipse [id:dp3344917595196768] 
	\draw   (357,149) .. controls (357,106.2) and (369.54,71.5) .. (385,71.5) .. controls (400.46,71.5) and (413,106.2) .. (413,149) .. controls (413,191.8) and (400.46,226.5) .. (385,226.5) .. controls (369.54,226.5) and (357,191.8) .. (357,149) -- cycle ;
	%Shape: Ellipse [id:dp6750410970361425] 
	\draw   (477,149) .. controls (477,106.2) and (489.54,71.5) .. (505,71.5) .. controls (520.46,71.5) and (533,106.2) .. (533,149) .. controls (533,191.8) and (520.46,226.5) .. (505,226.5) .. controls (489.54,226.5) and (477,191.8) .. (477,149) -- cycle ;
	%Straight Lines [id:da14833334442314672] 
	\draw    (505,199) -- (571,199) ;
	\draw [shift={(574,199)}, rotate = 180] [fill={rgb, 255:red, 0; green, 0; blue, 0 }  ][line width=0.08]  [draw opacity=0] (10.72,-5.15) -- (0,0) -- (10.72,5.15) -- (7.12,0) -- cycle    ;
	%Straight Lines [id:da5115442410135242] 
	\draw    (505,71.5) -- (601,71.5) ;
	\draw [shift={(604,71.5)}, rotate = 180] [fill={rgb, 255:red, 0; green, 0; blue, 0 }  ][line width=0.08]  [draw opacity=0] (10.72,-5.15) -- (0,0) -- (10.72,5.15) -- (7.12,0) -- cycle    ;
	%Shape: Polygon Curved [id:ds12846041972473188] 
	\draw  [pattern=_rlc9ed9hs,pattern size=6pt,pattern thickness=0.75pt,pattern radius=0pt, pattern color={rgb, 255:red, 222; green, 222; blue, 222}] (358,129) .. controls (394.2,129.2) and (433,128.8) .. (478,129) .. controls (476.2,150.8) and (476.6,161.6) .. (479,179) .. controls (439.8,179.2) and (399.4,178.8) .. (359,179) .. controls (357.8,156.4) and (356.2,151.2) .. (358,129) -- cycle ;
	%Straight Lines [id:da8464502942319136] 
	\draw    (385,226.5) -- (505,226.5) ;
	\draw [shift={(445,226.5)}, rotate = 180] [fill={rgb, 255:red, 0; green, 0; blue, 0 }  ][line width=0.08]  [draw opacity=0] (10.72,-5.15) -- (0,0) -- (10.72,5.15) -- (7.12,0) -- cycle    ;
	%Straight Lines [id:da9007770361830605] 
	\draw    (379,179) -- (479,179) ;
	\draw [shift={(429,179)}, rotate = 180] [fill={rgb, 255:red, 0; green, 0; blue, 0 }  ][line width=0.08]  [draw opacity=0] (10.72,-5.15) -- (0,0) -- (10.72,5.15) -- (7.12,0) -- cycle    ;
	%Straight Lines [id:da1210164770651343] 
	\draw    (358,129) -- (356.71,165.4) ;
	\draw [shift={(356.6,168.4)}, rotate = 272.04] [fill={rgb, 255:red, 0; green, 0; blue, 0 }  ][line width=0.08]  [draw opacity=0] (10.72,-5.15) -- (0,0) -- (10.72,5.15) -- (7.12,0) -- cycle    ;

	% Text Node
	\draw (35.5,149.5) node    {$S$};
	% Text Node
	\draw (146.5,149.5) node    {$N$};
	% Text Node
	\draw (587.5,197.5) node    {$\vec{n}$};
	% Text Node
	\draw (617.5,69.5) node    {$\vec{v}$};
	% Text Node
	\draw (384.5,91.5) node    {$S_{i}$};
	% Text Node
	\draw (506.5,92.5) node    {$S_{f}$};
	% Text Node
	\draw (450.1,160.5) node    {$dS$};
	% Text Node
	\draw (433.9,191.5) node    {$d\vec{r}$};
	% Text Node
	\draw (342.7,144.3) node    {$d\vec{l}$};


	\end{tikzpicture}
\end{figure}
\FloatBarrier
Tale carica si muove all'interno di un campo magnetico, su di essa agirà una forza di Lorentz pari a:
\[
	\vec{F}_L = q\,\vec{v} \times \vec{B}
\]
Introduciamo un campo detto elettromotore, Ee, definito come
\[
	\vec{E}_e = \frac{\vec{F}_L}{q}= \vec{v} \times \vec{B}
\]
Proviamo a calcolare la forza elettromotrice come
\begin{equation*}
	\begin{aligned}
		f_i &= \oint_{\gamma} \vec{E}_e\cdot d\vec{l} \\
		&= \oint_{\gamma} (\vec{v} \times \vec{B} )\cdot d\vec{l} \\
		&= \oint_{\gamma} (d\vec{l} \times \vec{v} ) \cdot \vec{B}  \\
		&= \oint_{\gamma} \left(d\vec{l} \times \frac{d\vec{r}}{dt}\right)\cdot \vec{B} \\
		&= \oint_{\gamma} \frac{d\vec{l} \times d\vec{r}}{dt}\cdot \vec{B}
	\end{aligned}
\end{equation*}
$ d\vec{l} \times d\vec{r} $ ha modulo pari all'area $dS$ del parallelogramma di lati $dl$ e $dr$, descritto da $dl$ nel tempo $dt$.
Introducendo una normale a questo rettangolo, possiamo scrivere
\[
	d\vec{r} \times d\vec{l} =dS\,\vec{n}
\]
Sostituendo questo risultato nella espressione della forza elettromotrice:
\[
	f_i = \frac{\oint_{\gamma} dS\,\vec{n} \cdot \vec{B}}{dt} = \frac{\oint_{\gamma} \vec{B} \cdot \vec{n} \,dS}{dt} =\frac{d\Phi_{d\Sigma}(\vec{B} )}{dt}
\]
Il termine al denominatore rappresenta proprio il flusso di $\vec{B}$ attraverso la superficie laterale $ d\Sigma  $. A tale flusso si da normalmente il nome di \textbf{flusso tagliato}, in quanto corrisponde al flusso lungo le linee del campo magnetico attraversate (tagliate) dalla spira nel suo spostamento.
\begin{figure}[htpb]
	\centering
	

	\tikzset{every picture/.style={line width=0.75pt}} %set default line width to 0.75pt        

	\begin{tikzpicture}[x=0.75pt,y=0.75pt,yscale=-1,xscale=1]
	%uncomment if require: \path (0,300); %set diagram left start at 0, and has height of 300

	%Curve Lines [id:da7344319235731953] 
	\draw    (212.5,118.96) .. controls (299.03,108.63) and (343.83,109.35) .. (418,80.75) ;
	\draw [shift={(317.06,107.12)}, rotate = 531.99] [fill={rgb, 255:red, 0; green, 0; blue, 0 }  ][line width=0.08]  [draw opacity=0] (10.72,-5.15) -- (0,0) -- (10.72,5.15) -- (7.12,0) -- cycle    ;
	%Curve Lines [id:da6735212516716587] 
	\draw    (212.5,145.54) .. controls (299.03,155.87) and (343.83,155.15) .. (418,183.75) ;
	\draw [shift={(317.06,157.38)}, rotate = 188.01] [fill={rgb, 255:red, 0; green, 0; blue, 0 }  ][line width=0.08]  [draw opacity=0] (10.72,-5.15) -- (0,0) -- (10.72,5.15) -- (7.12,0) -- cycle    ;
	%Curve Lines [id:da38088005125071667] 
	\draw    (212.5,130.45) .. controls (299.03,126.44) and (343.83,123.36) .. (418,112.26) ;
	\draw [shift={(316.12,124.33)}, rotate = 535.4] [fill={rgb, 255:red, 0; green, 0; blue, 0 }  ][line width=0.08]  [draw opacity=0] (10.72,-5.15) -- (0,0) -- (10.72,5.15) -- (7.12,0) -- cycle    ;
	%Curve Lines [id:da8905412173196354] 
	\draw    (212.5,134.05) .. controls (299.03,138.06) and (343.83,141.14) .. (418,152.24) ;
	\draw [shift={(316.12,140.17)}, rotate = 184.6] [fill={rgb, 255:red, 0; green, 0; blue, 0 }  ][line width=0.08]  [draw opacity=0] (10.72,-5.15) -- (0,0) -- (10.72,5.15) -- (7.12,0) -- cycle    ;
	%Straight Lines [id:da138108608465088] 
	\draw    (377,198) -- (413,198) ;
	\draw [shift={(416,198)}, rotate = 180] [fill={rgb, 255:red, 0; green, 0; blue, 0 }  ][line width=0.08]  [draw opacity=0] (10.72,-5.15) -- (0,0) -- (10.72,5.15) -- (7.12,0) -- cycle    ;
	%Shape: Can [id:dp8550992483460775] 
	\draw   (375.25,225.5) -- (253.75,225.5) .. controls (240.91,225.5) and (230.5,190.8) .. (230.5,148) .. controls (230.5,105.2) and (240.91,70.5) .. (253.75,70.5) -- (375.25,70.5) .. controls (388.09,70.5) and (398.5,105.2) .. (398.5,148) .. controls (398.5,190.8) and (388.09,225.5) .. (375.25,225.5) .. controls (362.41,225.5) and (352,190.8) .. (352,148) .. controls (352,105.2) and (362.41,70.5) .. (375.25,70.5) ;
	%Curve Lines [id:da5252519670188107] 
	\draw  [dash pattern={on 0.84pt off 2.51pt}]  (253.75,70.5) .. controls (286,70.25) and (281.5,227.75) .. (253.75,225.5) ;
	%Straight Lines [id:da7264894374138775] 
	\draw    (254,198) -- (290,198) ;
	\draw [shift={(293,198)}, rotate = 180] [fill={rgb, 255:red, 0; green, 0; blue, 0 }  ][line width=0.08]  [draw opacity=0] (10.72,-5.15) -- (0,0) -- (10.72,5.15) -- (7.12,0) -- cycle    ;

	% Text Node
	\draw (431,196.5) node    {$\vec{n}_{f}$};
	% Text Node
	\draw (427,76) node    {$\vec{B}$};
	% Text Node
	\draw (257,56.5) node    {$i$};
	% Text Node
	\draw (377,56.5) node    {$f$};
	% Text Node
	\draw (306,196.5) node    {$\vec{n}_{i}$};
	% Text Node
	\draw (316.5,236) node    {$d\Sigma $};


	\end{tikzpicture}
\end{figure}
\FloatBarrier
Se consideriamo la superficie chiusa $\Sigma$ data dall'unione di $ \Sigma_i  $ e $ \Sigma_f  $ e $ d\Sigma  $ e calcoliamo il flusso del campo attraverso tale superficie, dal momento che $ \text{div}\vec{B} =0 $:
\[
	\Phi_{\Sigma}(\vec{B}) = 0 = d\Phi_{d\Sigma} + \Phi_f - \Phi_i
\]
$\Phi_{\Sigma}(\vec{B})$ variazione di flusso magnetico concatenato al circuito.\\
$\Phi_f$ flusso concatenato al circuito nella posizione finale.\\
$\Phi_i$ flusso concatenato al circuito nella posizione iniziale.\\\\
La variazione di flusso attraverso la spira vale dunque:
\begin{gather*}
	d\Phi = \Phi_f-\Phi_i= -d\Phi_{d\Sigma} \\
	f_i = \frac{d\Phi_{d\Sigma}}{dt} = - \frac{d\Phi}{dt}
\end{gather*}
Se $ \Phi_f = \Phi_i $ non si ha variazione di flusso e non c'è quindi $fem$. È il caso di un campo magnetico uniforme.
Quando un circuito si muove rispetto alla sorgente di campo magnetico, compare la forza elettromotrice causata dalla forza di Lorentz. L'interpretazione è semplice: le forze di Lorentz mettono in moto le cariche.

Soffermiamoci sul caso di una sorgente di campo che si sposta rispetto al circuito fermo. Anche in questo caso le cariche saranno messe in moto: sperimentalmente si osserva infatti una corrente indotta. Non possiamo più parlare però di forze di Lorentz perché all'inizio le cariche sono ferme. Ci si può salvare grazie al principio di relatività di Galileo. Se ci poniamo in un sistema di riferimento che viaggia con il magnete, dato che nei due sistemi di riferimento dobbiamo osservare gli stessi effetti che avremmo se il magnete fosse fermo ed il circuito in moto, nel sistema solidale con il magente dobbiamo osservare una forza di Lorentz. Le due situazioni in figura diventano allora equivalenti.
\begin{figure}[htpb]
	\centering
	

	\tikzset{every picture/.style={line width=0.75pt}} %set default line width to 0.75pt        

	\begin{tikzpicture}[x=0.75pt,y=0.75pt,yscale=-1,xscale=1]
	%uncomment if require: \path (0,300); %set diagram left start at 0, and has height of 300

	%Shape: Rectangle [id:dp4576580186476005] 
	\draw   (158.62,73.56) -- (219.74,73.56) -- (219.74,95.18) -- (158.62,95.18) -- cycle ;
	%Shape: Rectangle [id:dp884221839198301] 
	\draw  [fill={rgb, 255:red, 222; green, 222; blue, 222 }  ,fill opacity=1 ] (219.74,73.56) -- (241.36,73.56) -- (241.36,95.18) -- (219.74,95.18) -- cycle ;
	%Shape: Rectangle [id:dp6940229120054398] 
	\draw  [fill={rgb, 255:red, 222; green, 222; blue, 222 }  ,fill opacity=1 ] (137,73.56) -- (158.62,73.56) -- (158.62,95.18) -- (137,95.18) -- cycle ;
	%Curve Lines [id:da7170417180295956] 
	\draw    (241.36,77.48) .. controls (283.1,70.04) and (304.72,67.32) .. (340.5,46.73) ;
	\draw [shift={(292.71,67.16)}, rotate = 525.38] [fill={rgb, 255:red, 0; green, 0; blue, 0 }  ][line width=0.08]  [draw opacity=0] (10.72,-5.15) -- (0,0) -- (10.72,5.15) -- (7.12,0) -- cycle    ;
	%Curve Lines [id:da2354299102413986] 
	\draw    (241.36,90.15) .. controls (283.1,97.59) and (304.72,100.31) .. (340.5,120.9) ;
	\draw [shift={(292.71,100.47)}, rotate = 194.62] [fill={rgb, 255:red, 0; green, 0; blue, 0 }  ][line width=0.08]  [draw opacity=0] (10.72,-5.15) -- (0,0) -- (10.72,5.15) -- (7.12,0) -- cycle    ;
	%Curve Lines [id:da6237417554819333] 
	\draw    (241.36,82.52) .. controls (283.1,79.63) and (304.72,77.41) .. (340.5,69.42) ;
	\draw [shift={(291.35,78.11)}, rotate = 533.3299999999999] [fill={rgb, 255:red, 0; green, 0; blue, 0 }  ][line width=0.08]  [draw opacity=0] (10.72,-5.15) -- (0,0) -- (10.72,5.15) -- (7.12,0) -- cycle    ;
	%Curve Lines [id:da42834637232751094] 
	\draw    (241.36,85.11) .. controls (283.1,88) and (304.72,90.22) .. (340.5,98.21) ;
	\draw [shift={(291.35,89.52)}, rotate = 186.67] [fill={rgb, 255:red, 0; green, 0; blue, 0 }  ][line width=0.08]  [draw opacity=0] (10.72,-5.15) -- (0,0) -- (10.72,5.15) -- (7.12,0) -- cycle    ;
	%Shape: Ellipse [id:dp569553660133667] 
	\draw   (372.55,84) .. controls (372.55,52.09) and (381.9,26.23) .. (393.42,26.23) .. controls (404.95,26.23) and (414.29,52.09) .. (414.29,84) .. controls (414.29,115.91) and (404.95,141.77) .. (393.42,141.77) .. controls (381.9,141.77) and (372.55,115.91) .. (372.55,84) -- cycle ;
	%Straight Lines [id:da6756468793205106] 
	\draw    (393.42,86.61) -- (464.22,86.61) ;
	\draw [shift={(467.22,86.61)}, rotate = 180] [fill={rgb, 255:red, 0; green, 0; blue, 0 }  ][line width=0.08]  [draw opacity=0] (10.72,-5.15) -- (0,0) -- (10.72,5.15) -- (7.12,0) -- cycle    ;
	%Shape: Rectangle [id:dp33213827151315334] 
	\draw   (158.62,203.56) -- (219.74,203.56) -- (219.74,225.18) -- (158.62,225.18) -- cycle ;
	%Shape: Rectangle [id:dp6552487363357307] 
	\draw  [fill={rgb, 255:red, 222; green, 222; blue, 222 }  ,fill opacity=1 ] (219.74,203.56) -- (241.36,203.56) -- (241.36,225.18) -- (219.74,225.18) -- cycle ;
	%Shape: Rectangle [id:dp8652540665212969] 
	\draw  [fill={rgb, 255:red, 222; green, 222; blue, 222 }  ,fill opacity=1 ] (137,203.56) -- (158.62,203.56) -- (158.62,225.18) -- (137,225.18) -- cycle ;
	%Curve Lines [id:da928852271271053] 
	\draw    (241.36,207.48) .. controls (283.1,200.04) and (304.72,197.32) .. (340.5,176.73) ;
	\draw [shift={(292.71,197.16)}, rotate = 525.38] [fill={rgb, 255:red, 0; green, 0; blue, 0 }  ][line width=0.08]  [draw opacity=0] (10.72,-5.15) -- (0,0) -- (10.72,5.15) -- (7.12,0) -- cycle    ;
	%Curve Lines [id:da07363356769513385] 
	\draw    (241.36,220.15) .. controls (283.1,227.59) and (304.72,230.31) .. (340.5,250.9) ;
	\draw [shift={(292.71,230.47)}, rotate = 194.62] [fill={rgb, 255:red, 0; green, 0; blue, 0 }  ][line width=0.08]  [draw opacity=0] (10.72,-5.15) -- (0,0) -- (10.72,5.15) -- (7.12,0) -- cycle    ;
	%Curve Lines [id:da3642152110945718] 
	\draw    (241.36,212.52) .. controls (283.1,209.63) and (304.72,207.41) .. (340.5,199.42) ;
	\draw [shift={(291.35,208.11)}, rotate = 533.3299999999999] [fill={rgb, 255:red, 0; green, 0; blue, 0 }  ][line width=0.08]  [draw opacity=0] (10.72,-5.15) -- (0,0) -- (10.72,5.15) -- (7.12,0) -- cycle    ;
	%Curve Lines [id:da628470651353888] 
	\draw    (241.36,215.11) .. controls (283.1,218) and (304.72,220.22) .. (340.5,228.21) ;
	\draw [shift={(291.35,219.52)}, rotate = 186.67] [fill={rgb, 255:red, 0; green, 0; blue, 0 }  ][line width=0.08]  [draw opacity=0] (10.72,-5.15) -- (0,0) -- (10.72,5.15) -- (7.12,0) -- cycle    ;
	%Shape: Ellipse [id:dp0488916567302462] 
	\draw   (372.55,214) .. controls (372.55,182.09) and (381.9,156.23) .. (393.42,156.23) .. controls (404.95,156.23) and (414.29,182.09) .. (414.29,214) .. controls (414.29,245.91) and (404.95,271.77) .. (393.42,271.77) .. controls (381.9,271.77) and (372.55,245.91) .. (372.55,214) -- cycle ;
	%Straight Lines [id:da9571130691953209] 
	\draw    (61.42,214.61) -- (132.22,214.61) ;
	\draw [shift={(58.42,214.61)}, rotate = 0] [fill={rgb, 255:red, 0; green, 0; blue, 0 }  ][line width=0.08]  [draw opacity=0] (10.72,-5.15) -- (0,0) -- (10.72,5.15) -- (7.12,0) -- cycle    ;

	% Text Node
	\draw (147.81,84.37) node    {$S$};
	% Text Node
	\draw (230.55,84.37) node    {$N$};
	% Text Node
	\draw (477.28,85.12) node    {$\vec{v}$};
	% Text Node
	\draw (147.81,214.37) node    {$S$};
	% Text Node
	\draw (230.55,214.37) node    {$N$};
	% Text Node
	\draw (50.28,211.12) node    {$\vec{v}$};


	\end{tikzpicture}
\end{figure}
\FloatBarrier
Se vogliamo rimanere sullo stesso SI (quello solidale al circuito), nel secondo caso non possiamo parlare di forza di Lorentz. Possiamo solo interpretare la situazione dicendo che a causa della variazione di flusso di campo magnetico nel circuito si introduce un campo elettrico.

Abbiamo visto una legge di tipo integrale. Vediamo se esiste un equivalente locale della \textbf{legge di Faraday-Neumann-Lenz}. Immaginiamo di avere la solita situazione con $\gamma$, $\Sigma$ ed $\vec{n}$ in cui è presente un campo magnetico variabile nel tempo. Secondo questa legge ci aspettiamo che lungo il circuito si indica una $fem$ pari a:
\[
	-\frac{d\Phi_{\Sigma}(\vec{B} )}{dt}
\]
Possiamo portare la derivata dentro all'integrale e scrivere, usando anche il teorema di Stokes
\begin{equation*}
	\begin{aligned}
		\oint_{\gamma} \vec{E} \cdot d\vec{l} &= - \frac{d}{dt} \underbrace{\int_{\Sigma}\vec{B} \cdot \vec{n}\,dS}_{\Phi_{\Sigma}(\vec{B} )} = - \int_{\Sigma}\left( \frac{\partial \vec{B}}{\partial t}  \right) \cdot \vec{n} \,dS \\
		\oint_{\gamma} \vec{E} \cdot d\vec{l} &= \int_{\Sigma}(\text{rot}\vec{E} ) \cdot \vec{n} \,dS \qquad \implies \boxed{\text{rot}\vec{E} = - \frac{\partial \vec{B}}{\partial t}}
	\end{aligned}
\end{equation*}
Se i campi magnetici non variano nel tempo, torniamo alla situazione dei campi elettrostatici conservativi. Se c'è un campo magnetico variabile nel tempo, il campo elettrico diventa non conservativo, perché deve essere in grado di porre delle cariche in moto in un circuito.
\[
	\vec{\nabla} \cdot [\vec{\nabla} \times \vec{E} ] = 0 = \vec{\nabla} \cdot \left[ -\frac{\partial \vec{B}}{\partial t}  \right] = -\frac{\partial}{\partial t} \vec{\nabla} \cdot \vec{B}  \implies  \vec{\nabla} \cdot \vec{B} \quad \text{costante}
\]
Inoltre, in tal caso, la derivata è diversa da zero, quindi il rotore del campo è non nullo e automaticamente esiste un campo elettrico, anche in assenza di cariche. C'è un legame fortissimo fra campi elettrici e campi magnetici

Identità operatoriale: L'espressione Wb= Vs si può ricavare dalla legge di Faraday
\[
	[f_i] = \frac{[\Phi ]}{[T]} \qquad (V)=\frac{(Wb)}{(S)}
\]
Supponiamo di trovarci in una regione dello spazio in cui sono presenti delle cariche elettriche e dei circuiti con correnti variabili nel tempo. Una parte del campo elettrico sarà generato da Q, un'altra generata dai campi magnetici variabili nel tempo. In questo caso il campo elettrico potrebbe avere una espressione più complessa.
Nel caso stazionario vale la legge: $ \vec{E} = -\vec{\nabla} V $.
Avevamo poi espresso il campo $\vec{B}$ come rotore del potenziale vettore $\vec{A}$.
\begin{equation*}
	\begin{aligned}
		\vec{B} =\text{rot}\vec{A} \qquad \text{rot}\vec{E} &= - \frac{\partial \vec{B}}{\partial t} \\
		\text{rot}\vec{E} &= - \frac{\partial \text{rot}\vec{A}}{\partial t} = \text{rot}\left(-\frac{\partial \vec{A}}{\partial t} \right)
	\end{aligned}
\end{equation*}
Quindi $ \vec{E} = - \frac{\partial \vec{A}}{\partial t}  $
L'espressione più generale che lega i campi elettrici è allora questa:
\[
	\boxed{\vec{E} = - \vec{\nabla} V - \frac{\partial \vec{A}}{\partial t}}
\]
Usiamo il potenziale vettore per descrivere la parte di campo elettrico legata ai campi magnetici variabili







































\section{Autoinduzione}

Abbiamo messo in evidenza che il campo magnetico generato dalla corrente che percorre un circuito dia luogo ad un flusso magnetico attraverso il circuito stesso, il cosiddetto autoflusso, che risulta proporzionale alla corrente i tramite il coefficiente di autoinduzione $L$, che dipende dalla forma geometrica del circuito e dalla permeabilità magnetica del mezzo in cui il circuito è immerso.
\begin{figure}[htpb]
	\centering
	

	% Pattern Info
	 
	\tikzset{
	pattern size/.store in=\mcSize, 
	pattern size = 5pt,
	pattern thickness/.store in=\mcThickness, 
	pattern thickness = 0.3pt,
	pattern radius/.store in=\mcRadius, 
	pattern radius = 1pt}
	\makeatletter
	\pgfutil@ifundefined{pgf@pattern@name@_jmy2ayvgh}{
	\pgfdeclarepatternformonly[\mcThickness,\mcSize]{_jmy2ayvgh}
	{\pgfqpoint{0pt}{0pt}}
	{\pgfpoint{\mcSize+\mcThickness}{\mcSize+\mcThickness}}
	{\pgfpoint{\mcSize}{\mcSize}}
	{
	\pgfsetcolor{\tikz@pattern@color}
	\pgfsetlinewidth{\mcThickness}
	\pgfpathmoveto{\pgfqpoint{0pt}{0pt}}
	\pgfpathlineto{\pgfpoint{\mcSize+\mcThickness}{\mcSize+\mcThickness}}
	\pgfusepath{stroke}
	}}
	\makeatother
	\tikzset{every picture/.style={line width=0.75pt}} %set default line width to 0.75pt        

	\begin{tikzpicture}[x=0.75pt,y=0.75pt,yscale=-1,xscale=1]
	%uncomment if require: \path (0,300); %set diagram left start at 0, and has height of 300

	%Shape: Ellipse [id:dp24672730295208045] 
	\draw  [pattern=_jmy2ayvgh,pattern size=6pt,pattern thickness=0.75pt,pattern radius=0pt, pattern color={rgb, 255:red, 222; green, 222; blue, 222}] (192.63,146.83) .. controls (192.71,111.3) and (238.4,82.6) .. (294.68,82.74) .. controls (350.95,82.87) and (396.5,111.79) .. (396.41,147.33) .. controls (396.33,182.87) and (350.64,211.56) .. (294.36,211.43) .. controls (238.09,211.29) and (192.54,182.37) .. (192.63,146.83) -- cycle ;
	\draw   (291.47,205.66) -- (302.39,211.6) -- (291.09,216.78) ;
	%Straight Lines [id:da9427327099690423] 
	\draw    (294.52,147.08) -- (294.78,40.26) ;
	\draw [shift={(294.79,37.26)}, rotate = 450.14] [fill={rgb, 255:red, 0; green, 0; blue, 0 }  ][line width=0.08]  [draw opacity=0] (10.72,-5.15) -- (0,0) -- (10.72,5.15) -- (7.12,0) -- cycle    ;
	%Curve Lines [id:da49715380719001945] 
	\draw    (200,244) .. controls (250.5,163) and (263.5,86) .. (219.5,33) ;
	\draw [shift={(243.62,138.9)}, rotate = 460.89] [fill={rgb, 255:red, 0; green, 0; blue, 0 }  ][line width=0.08]  [draw opacity=0] (10.72,-5.15) -- (0,0) -- (10.72,5.15) -- (7.12,0) -- cycle    ;
	%Curve Lines [id:da30757913025138395] 
	\draw    (348,256) .. controls (316.5,163) and (313.5,87) .. (384.5,44) ;
	\draw [shift={(327.05,140.01)}, rotate = 452.21] [fill={rgb, 255:red, 0; green, 0; blue, 0 }  ][line width=0.08]  [draw opacity=0] (10.72,-5.15) -- (0,0) -- (10.72,5.15) -- (7.12,0) -- cycle    ;

	% Text Node
	\draw (406,160.5) node    {$\gamma $};
	% Text Node
	\draw (313,41) node    {$\vec{n}$};
	% Text Node
	\draw (374,140.5) node    {$\Sigma $};
	% Text Node
	\draw (291.5,229) node    {$I( t)$};
	% Text Node
	\draw (399.3,38.08) node    {$\vec{B}$};


	\end{tikzpicture}
\end{figure}
\FloatBarrier
La nozione di autoflusso acquista particolare importanza quando la corrente nel circuito non è costante nel tempo (oppure quando viene variata la sua forma).
\[
	\Phi_{\Sigma}(\vec{B}) = \int_{\Sigma}\vec{B} \cdot \vec{n} \,dS >0  \qquad \Phi_{\Sigma}(\vec{B}) = L\cdot I(t)
\]
Il flusso concatenato con il circuito cambia nel tempo e nel circuito compare una $fem$ indotta che con i suoi effetti tende ad opporsi alla variazione che l'ha generata. Tale $fem$ si calcola come:
\[
	f_i = - \frac{d\Phi_{\Sigma}(\vec{B} )}{dt} = -\frac{d}{dt}(L\,I) = -L\,\frac{d}{dt}I
\]
In caso di materiali dia e para magnetici, $L$ può essere portato fuori dalla derivata.
Tale risultato fornisce operativamente la definizione del coefficiente $L$ di un qualsiasi circuito, purché rigido, immerso in un mezzo qualunque. Si deve supporre però che le variazioni di corrente non siano così rapide da avvenire in un tempo paragonabile a quello impiegato dalla luce a percorrere la dimensione tipica del circuito, perché in tal caso la corrente non avrebbe più lo stesso valore su diverse sezioni del circuito. Il coefficiente di autoinduzione di un circuito viene comunemente indicato con il termine \emph{induttanza}. Un circuito con induttanza non nulla si dice induttivo.
Molto spesso si rappresenta l'effetto di induzione elettromagnetica sul circuito immaginando che questo effetto sia legato ad un oggetto chiamato \emph{induttore}, ad esempio perché il filo conduttore è avvolto così da formare un solenoide.

Ad ogni modo, la presenza di un induttore in un circuito impedisce alla corrente di aumentare o diminuire istantaneamente in quanto la variazione genera una $fem$ che si oppone alla variazione stessa. Esaminiamo il caso di un circuito $RLC$ serie, costituito da un generatore di $fem$ e resistenza interna trascurabile, da un induttore con induttanza $L$ e da un resistore di resistenza $R$. Le variazioni di corrente sono causate inizialmente dall'apertura e dalla chiusura dell'interruttore $T$.
\begin{figure}[htpb]
	\centering
	

	\tikzset{every picture/.style={line width=0.75pt}} %set default line width to 0.75pt        

	\begin{tikzpicture}[x=0.75pt,y=0.75pt,yscale=-1,xscale=1]
	%uncomment if require: \path (0,300); %set diagram left start at 0, and has height of 300

	%Shape: Inductor (Air Core) [id:dp1811299889194351] 
	\draw   (220,77) -- (234.4,77) .. controls (234.61,66.49) and (236.61,57.51) .. (239.43,54.37) .. controls (242.26,51.23) and (245.35,54.57) .. (247.2,62.79) .. controls (248.63,69.2) and (249.21,77.48) .. (248.8,85.53) .. controls (248.8,88.67) and (248.08,91.21) .. (247.2,91.21) .. controls (246.32,91.21) and (245.6,88.67) .. (245.6,85.53) .. controls (245.19,77.48) and (245.77,69.2) .. (247.2,62.79) .. controls (248.86,55.96) and (251.18,52.09) .. (253.6,52.09) .. controls (256.02,52.09) and (258.34,55.96) .. (260,62.79) .. controls (261.43,69.2) and (262.01,77.48) .. (261.6,85.53) .. controls (261.6,88.67) and (260.88,91.21) .. (260,91.21) .. controls (259.12,91.21) and (258.4,88.67) .. (258.4,85.53) .. controls (257.99,77.48) and (258.57,69.2) .. (260,62.79) .. controls (261.66,55.96) and (263.98,52.09) .. (266.4,52.09) .. controls (268.82,52.09) and (271.14,55.96) .. (272.8,62.79) .. controls (274.23,69.2) and (274.81,77.48) .. (274.4,85.53) .. controls (274.4,88.67) and (273.68,91.21) .. (272.8,91.21) .. controls (271.92,91.21) and (271.2,88.67) .. (271.2,85.53) .. controls (270.79,77.48) and (271.37,69.2) .. (272.8,62.79) .. controls (274.65,54.57) and (277.74,51.23) .. (280.57,54.37) .. controls (283.39,57.51) and (285.39,66.49) .. (285.6,77) -- (300,77) ;
	%Shape: Resistor [id:dp6080934326111651] 
	\draw   (353,77) -- (353,91.4) -- (373,94.6) -- (333,101) -- (373,107.4) -- (333,113.8) -- (373,120.2) -- (333,126.6) -- (373,133) -- (333,139.4) -- (353,142.6) -- (353,157) ;
	%Shape: Battery [id:dp6120838199192846] 
	\draw  [fill={rgb, 255:red, 0; green, 0; blue, 0 }  ,fill opacity=1 ] (167,157) -- (167,121) (149.33,113) -- (184.67,113) (167,113) -- (167,77) (158.17,124.2) -- (158.17,121) -- (175.83,121) -- (175.83,124.2) -- (158.17,124.2) -- cycle ;
	%Shape: Simple Switch [id:dp7474022825810089] 
	\draw   (167,157) -- (167,173) (167,221) -- (167,237) (169.11,179.4) -- (200.68,217.8) (167,214.6) .. controls (170.49,214.6) and (173.32,216.03) .. (173.32,217.8) .. controls (173.32,219.57) and (170.49,221) .. (167,221) .. controls (163.51,221) and (160.68,219.57) .. (160.68,217.8) .. controls (160.68,216.03) and (163.51,214.6) .. (167,214.6) -- cycle (167,173) .. controls (170.49,173) and (173.32,174.43) .. (173.32,176.2) .. controls (173.32,177.97) and (170.49,179.4) .. (167,179.4) .. controls (163.51,179.4) and (160.68,177.97) .. (160.68,176.2) .. controls (160.68,174.43) and (163.51,173) .. (167,173) -- cycle ;
	%Straight Lines [id:da3068766871707447] 
	\draw    (167,77) -- (220,77) ;
	%Curve Lines [id:da6227103172248754] 
	\draw    (197.58,184.12) .. controls (191.81,198.19) and (174.26,206.48) .. (157.48,201.92) ;
	\draw [shift={(154.84,201.09)}, rotate = 379.61] [fill={rgb, 255:red, 0; green, 0; blue, 0 }  ][line width=0.08]  [draw opacity=0] (10.72,-5.15) -- (0,0) -- (10.72,5.15) -- (7.12,0) -- cycle    ;
	%Straight Lines [id:da4133366237660214] 
	\draw    (300,77) -- (353,77) ;
	%Straight Lines [id:da9569159529670197] 
	\draw    (167,237) -- (353.33,237) ;
	%Straight Lines [id:da2964493224854299] 
	\draw    (353,237) -- (353,157) ;

	% Text Node
	\draw (260.33,34.67) node    {$L$};
	% Text Node
	\draw (389,113.33) node    {$R$};
	% Text Node
	\draw (131.67,116) node    {$\Delta V$};
	% Text Node
	\draw (223.33,181.67) node    {$t=0$};


	\end{tikzpicture}
\end{figure}
\FloatBarrier
Chiediamoci cosa succede quando in $T=0$ chiudiamo il circuito.
In assenza di fenomeni di induzione elettromagnetica, ci aspetteremmo che alla chiusura abbiamo un corrente istantanea calcolabile con la legge di Ohm. Questo nella pratica non accade. Nel momento in cui chiudiamo un circuito, esso comincia a produrre un campo magnetico. Apparirà una forza elettromotrice indotta che si oppone all'aumento di $\vec{B}$. Possiamo dire che:
\begin{gather*}
	\Delta V + f_i = RI \\
	\Delta V \underbrace{-L \frac{dI}{dt}}_{f_i} = RI \\
	- L \frac{dI}{dt} = RI - \Delta V \\
	- \underbrace{\frac{L}{R}}_{\tau} \frac{dI}{dt} = I - \underbrace{\frac{\Delta V}{R}}_{I_{\infty}} \\
	-\tau \frac{dI}{dt} = I - I_{\infty} \\
	- \frac{\tau}{dt} = \frac{I-I_{\infty}}{dI} \\
	- \int \frac{dt}{\tau} = \int \frac{dI}{I-I_{\infty}} \\
	-\frac{t}{\tau} = [\log (I-I_{\infty})]_0^{I(t)} = \log (I(t)-I_{\infty})-\log (-I_{\infty} ) \\
	-\frac{t}{\tau} = \log \frac{I(t)-I_{\infty}}{-I_{\infty}} \\
	e^{-t/\tau} = \frac{I(t)-I_{\infty}}{-I_{\infty}} \\
	I_{\infty}-I_{\infty}e^{-t/\tau} = I(t) \implies \boxed{I(t)=I_{\infty}(1-e^{-t/\tau} )}
\end{gather*}
All'istante iniziale compare subito una $fem$ che si oppone alla causa che la sta generando. All'inizio tale forza è talmente intensa da dare corrente iniziale pari a zero. Più avanti vince il generatore e la corrente sale lentamente.
\begin{figure}[htpb]
	\centering
	

	\tikzset{every picture/.style={line width=0.75pt}} %set default line width to 0.75pt        

	\begin{tikzpicture}[x=0.75pt,y=0.75pt,yscale=-0.9,xscale=0.9]
	%uncomment if require: \path (0,355); %set diagram left start at 0, and has height of 355

	%Straight Lines [id:da45293825265519105] 
	\draw [color={rgb, 255:red, 155; green, 155; blue, 155 }  ,draw opacity=1 ]   (232.35,243.2) -- (281.33,141.33) ;
	%Shape: Axis 2D [id:dp5077172548790807] 
	\draw  (203.15,242.6) -- (496.65,242.6)(232.5,54.5) -- (232.5,263.5) (489.65,237.6) -- (496.65,242.6) -- (489.65,247.6) (227.5,61.5) -- (232.5,54.5) -- (237.5,61.5)  ;
	%Straight Lines [id:da4075929645086587] 
	\draw  [dash pattern={on 0.84pt off 2.51pt}]  (232.5,142) -- (473,142) ;
	% Plotting does not support converting to Tikz
	%Curve Lines [id:da11473619874977481] 
	\draw [line width=1.5]    (232.35,243.2) .. controls (280.67,139.33) and (351.33,152) .. (470,144.67) ;
	%Straight Lines [id:da9117039057161145] 
	\draw    (232.5,271.33) -- (281.33,271.33) ;
	\draw [shift={(281.33,271.33)}, rotate = 180] [color={rgb, 255:red, 0; green, 0; blue, 0 }  ][line width=0.75]    (0,5.59) -- (0,-5.59)   ;
	\draw [shift={(232.5,271.33)}, rotate = 180] [color={rgb, 255:red, 0; green, 0; blue, 0 }  ][line width=0.75]    (0,5.59) -- (0,-5.59)   ;

	% Text Node
	\draw (256,57) node    {$I( t)$};
	% Text Node
	\draw (511,245) node    {$t$};
	% Text Node
	\draw (194,139) node    {$I_{\infty } =\frac{\Delta V}{R}$};
	% Text Node
	\draw (258.33,294.33) node    {$\tau =\frac{L}{R}$};


	\end{tikzpicture}
\end{figure}
\FloatBarrier
\textbf{Osservazione.} Notiamo che $ I(t) $ tende asintoticamente al valore di regime $ I_{\infty}=\frac{\Delta V}{R}  $ corrispondente alla legge di Ohm per le correnti costanti. Il raggiungimento di tale valore è legato alla costante di tempo.







































\section{Energia magnetica}

La presenza di una $fem$ in un circuito implica, per definizione, un lavoro sulle cariche che costituiscono la corrente, positivo o negativo a seconda del segno della $fem$. Il lavoro $dt$ compiuto dal generatore per sostenere il moto delle cariche quando la corrente ha valore $I$ è dato da:
\[
	\underbrace{\Delta V = RI + L\frac{dI}{dt}}_{\text{moltiplico per}I\,dt} \implies \Delta V\,I\,dt = R\,I^2 dt + L\,I\,dI
\]
Questa espressione esprime il bilancio energetico del circuito. Il primo membro è il lavoro compiuto dal generatore per far circolare corrente nel circuito e trasformato in calore (effetto Joule). Il termine $L\,I\,dI$ è il lavoro speso contro la $fem$ di autoinduzione per far aumentare la corrente da $I$ a $I+dI$. Nell'intervallo di tempo in cui, a seguito della chiusura del circuito, la corrente passa da zero al valore $ I_{\infty}  $, il generatore oltre che al lavoro corrispondente all'effetto Joule deve spendere contro la $fem$ di autoinduzione il lavoro:
\[
	\mathcal{L}_i = \int d\mathcal{L}_i = \int_0^{I_{\infty}} L\,I\,dI = \left[ \frac{1}{2} L\,I^2  \right]_0^{I_{\infty}} = \frac{1}{2} L\,I_{\infty}^2
\]
Che non dipende dal modo in cui avviene la variazione di corrente, ma solo dal valore iniziale e finale. Possiamo interpretare questo lavoro come una energia magnetica che il circuito possiede. Avevamo parlato di una forma di energia analoga nel caso del campo elettrostatico del condensatore e avevamo definito una densità di energia per unità di volume. Ci si può chiedere se anche quest'altra energia magnetica sia visibile come un energia distribuita in tutto lo spazio in cui è presente il campo magnetico generato dal circuito. Dopotutto, l'espressione dell'energia magnetica suggerisce che essa sia legata al campo $\vec{B}$ e localizzabile nello spazio in cui tale campo esiste. Consideriamo un solenoide vuoto rettilineo infinito e su di esso un pezzo di lunghezza finita $L$.
\begin{figure}[htpb]
	\centering
	

	\tikzset{every picture/.style={line width=0.75pt}} %set default line width to 0.75pt        

	\begin{tikzpicture}[x=0.75pt,y=0.75pt,yscale=-1,xscale=1]
	%uncomment if require: \path (0,300); %set diagram left start at 0, and has height of 300

	%Shape: Can [id:dp7741966840400853] 
	\draw  [fill={rgb, 255:red, 222; green, 222; blue, 222 }  ,fill opacity=1 ] (406.81,187.5) -- (148.73,187.5) .. controls (141.97,187.5) and (136.5,169.26) .. (136.5,146.75) .. controls (136.5,124.24) and (141.97,106) .. (148.73,106) -- (406.81,106) .. controls (413.56,106) and (419.03,124.24) .. (419.03,146.75) .. controls (419.03,169.26) and (413.56,187.5) .. (406.81,187.5) .. controls (400.06,187.5) and (394.58,169.26) .. (394.58,146.75) .. controls (394.58,124.24) and (400.06,106) .. (406.81,106) ;
	%Curve Lines [id:da210960011768905] 
	\draw    (235.8,68.4) .. controls (191.8,83.6) and (193,187.6) .. (209.81,187.5) ;
	%Curve Lines [id:da8398771468545265] 
	\draw    (229.81,106) .. controls (213,105.2) and (213,187.6) .. (229.81,187.5) ;
	%Curve Lines [id:da4560936194642726] 
	\draw    (249.81,106) .. controls (233,105.2) and (233,187.6) .. (249.81,187.5) ;
	%Curve Lines [id:da5759943950482904] 
	\draw    (269.81,106) .. controls (253,105.2) and (253,187.6) .. (269.81,187.5) ;
	%Curve Lines [id:da7046738061237652] 
	\draw    (289.81,106) .. controls (273,105.2) and (273,187.6) .. (289.81,187.5) ;
	%Curve Lines [id:da5602030346503439] 
	\draw    (309.81,106) .. controls (293,105.2) and (293,187.6) .. (309.81,187.5) ;
	%Curve Lines [id:da16002715564423253] 
	\draw    (329.81,106) .. controls (313,105.2) and (313,187.6) .. (329.81,187.5) ;
	%Curve Lines [id:da19557435324446426] 
	\draw    (349.81,106) .. controls (333,105.2) and (325.4,194.4) .. (371.8,214.8) ;
	%Straight Lines [id:da3714557523081883] 
	\draw    (360.5,136.75) -- (496.5,136.75) ;
	\draw [shift={(499.5,136.75)}, rotate = 180] [fill={rgb, 255:red, 0; green, 0; blue, 0 }  ][line width=0.08]  [draw opacity=0] (10.72,-5.15) -- (0,0) -- (10.72,5.15) -- (7.12,0) -- cycle    ;
	%Straight Lines [id:da5351098831815313] 
	\draw    (360.5,156.75) -- (448.5,156.75) ;
	\draw [shift={(451.5,156.75)}, rotate = 180] [fill={rgb, 255:red, 0; green, 0; blue, 0 }  ][line width=0.08]  [draw opacity=0] (10.72,-5.15) -- (0,0) -- (10.72,5.15) -- (7.12,0) -- cycle    ;

	% Text Node
	\draw (510.3,134.08) node    {$\vec{B}$};
	% Text Node
	\draw (465.3,155.08) node    {$\vec{H}$};


	\end{tikzpicture}
\end{figure}
\FloatBarrier
Conosciamo l'area di sezione del solenoide $S$. Si produce a causa della corrente un campo magnetico al suo interno che sappiamo calcolare come:
\[
	B = \mu_0 n_s\,I \qquad H = n_s\,I
\]
Dobbiamo moltiplicare il flusso per il numero di spire.
\begin{align*}
	&L = \frac{\Phi_{\text{autoconcatenato}}(\vec{B})}{I} = \frac{B\,S\,N}{I} = \frac{\mu_0 n_s I\cdot S \cdot n_s\,l}{I} = \mu_0 n_s^2 S\,l \\
	&\boxed{U_m = \frac{1}{2} L\,I^2} = \frac{1}{2} (\mu_0 n_s^2 S\,l)\,I^2
\end{align*}
Notiamo che $S\,l$ è il volume contenuto all'interno di questa porzione di solenoide. Questa energia magnetica in effetti è proporzionale al suo volume. Ha senso definire una densità di energia magnetica per unità di volume come:
\[
	u_m = \frac{U_m}{\text{volume}} = \frac{U_m}{S\,l} = \frac{\frac{1}{2} (\mu_0 n_s^2 S\,l)\,I^2}{S\,l} = \frac{1}{2} \mu_0 n_s^2\,I^2
\]
Data la struttura di questa espressione, che dipende solo dal valore del campo magnetico e dalla permeabilità del mezzo (che è il vuoto) siamo portati a concludere che il risultato trovato valga sempre, nel vuoto, qualunque sia l'andamento del campo magnetico e la sorgente che lo genera. Possiamo riscrivere la densità di energia per unità di volume come:
\[
	u_m = \underbrace{\frac{1}{2} \mu_0 n_s^2\,I^2}_{\text{moltiplico e divido per}\mu_0} = \frac{1}{2} \frac{\mu_0^2 n_s^2\,I^2}{\mu_0} = \boxed{\frac{1}{2} \frac{B^2}{\mu_0} = u_m}
\]
Notiamo l'analogia con la struttura della densità di energia elettrostatica:
\[
	u_e = \frac{1}{2} \varepsilon_0 E^2
\]
L'energia magnetica potrà allora essere calcolata come:
\[
	U_m = \int_{\text{spazio}} u_m d\tau
\]
Vediamo cosa accade in presenza di materiali magnetici. Dobbiamo distinguere anche in questo caso i materiali dia e para magnetici da quelli ferromagnetici.
Immaginiamo che dentro il solenoide ci sia un materiale con una certa permeabilità magnetica $\mu_r$.
\begin{figure}[htpb]
	\centering
	

	\tikzset{every picture/.style={line width=0.75pt}} %set default line width to 0.75pt        

	\begin{tikzpicture}[x=0.75pt,y=0.75pt,yscale=-0.8,xscale=0.8]
	%uncomment if require: \path (0,300); %set diagram left start at 0, and has height of 300

	%Shape: Can [id:dp3503983742611452] 
	\draw  [fill={rgb, 255:red, 222; green, 222; blue, 222 }  ,fill opacity=1 ] (326.81,197.5) -- (68.72,197.5) .. controls (61.97,197.5) and (56.5,179.26) .. (56.5,156.75) .. controls (56.5,134.24) and (61.97,116) .. (68.72,116) -- (326.81,116) .. controls (333.56,116) and (339.03,134.24) .. (339.03,156.75) .. controls (339.03,179.26) and (333.56,197.5) .. (326.81,197.5) .. controls (320.06,197.5) and (314.58,179.26) .. (314.58,156.75) .. controls (314.58,134.24) and (320.06,116) .. (326.81,116) ;
	%Curve Lines [id:da6125100454529431] 
	\draw    (155.8,78.4) .. controls (111.8,93.6) and (113,197.6) .. (129.81,197.5) ;
	%Curve Lines [id:da9453283806790904] 
	\draw    (149.81,116) .. controls (133,115.2) and (133,197.6) .. (149.81,197.5) ;
	%Curve Lines [id:da05294172617481241] 
	\draw    (169.81,116) .. controls (153,115.2) and (153,197.6) .. (169.81,197.5) ;
	%Curve Lines [id:da5691564596361309] 
	\draw    (189.81,116) .. controls (173,115.2) and (173,197.6) .. (189.81,197.5) ;
	%Curve Lines [id:da5246721629101647] 
	\draw    (209.81,116) .. controls (193,115.2) and (193,197.6) .. (209.81,197.5) ;
	%Curve Lines [id:da5701541332841926] 
	\draw    (229.81,116) .. controls (213,115.2) and (213,197.6) .. (229.81,197.5) ;
	%Curve Lines [id:da3391028947263257] 
	\draw    (249.81,116) .. controls (233,115.2) and (233,197.6) .. (249.81,197.5) ;
	%Curve Lines [id:da06860635740055199] 
	\draw    (269.81,116) .. controls (253,115.2) and (245.4,204.4) .. (291.8,224.8) ;
	%Straight Lines [id:da5282854737648179] 
	\draw    (270.5,156.75) -- (396.5,156.75) ;
	\draw [shift={(399.5,156.75)}, rotate = 180] [fill={rgb, 255:red, 0; green, 0; blue, 0 }  ][line width=0.08]  [draw opacity=0] (10.72,-5.15) -- (0,0) -- (10.72,5.15) -- (7.12,0) -- cycle    ;
	%Straight Lines [id:da26118252633211325] 
	\draw    (270.5,176.75) -- (358.5,176.75) ;
	\draw [shift={(361.5,176.75)}, rotate = 180] [fill={rgb, 255:red, 0; green, 0; blue, 0 }  ][line width=0.08]  [draw opacity=0] (10.72,-5.15) -- (0,0) -- (10.72,5.15) -- (7.12,0) -- cycle    ;
	%Shape: Battery [id:dp7697106590049856] 
	\draw  [fill={rgb, 255:red, 0; green, 0; blue, 0 }  ,fill opacity=1 ] (502,168) -- (502,132) (472,124) -- (532,124) (502,124) -- (502,88) (487,135.2) -- (487,132) -- (517,132) -- (517,135.2) -- (487,135.2) -- cycle ;
	%Shape: Resistor [id:dp3604906038927598] 
	\draw   (502,153.6) -- (502,168) -- (522,171.2) -- (482,177.6) -- (522,184) -- (482,190.4) -- (522,196.8) -- (482,203.2) -- (522,209.6) -- (482,216) -- (502,219.2) -- (502,233.6) ;
	%Curve Lines [id:da19416694906660026] 
	\draw    (502,88) .. controls (504,38.5) and (219,41.5) .. (155.8,78.4) ;
	%Curve Lines [id:da028991734151780246] 
	\draw    (502,233.6) .. controls (504,283.1) and (362,252.5) .. (291.8,224.8) ;

	% Text Node
	\draw (410.3,154.08) node    {$\vec{B}$};
	% Text Node
	\draw (375.3,175.08) node    {$\vec{H}$};
	% Text Node
	\draw (328.3,140.08) node    {$S$};
	% Text Node
	\draw (554.3,127.08) node    {$\Delta V$};
	% Text Node
	\draw (539.3,191.08) node    {$R$};


	\end{tikzpicture}
\end{figure}
\FloatBarrier
Colleghiamo un generatore reale al solenoide. Facendo aumentare la corrente da $0$ a $I$ il flusso concatenato alle spire cambia e nascerà una $fem$. La legge del circuito all'istante $t$ è:
\begin{equation*}
	\begin{aligned}
		\Delta V + f_i &= RI \\
		\Delta V - \frac{d\Phi (\vec{B})}{dt} &= RI \\
		\Delta V &= RI + \frac{d\Phi (\vec{B})}{dt} \\
	\end{aligned}
\end{equation*}
E il lavoro speso dal generatore è:
\[
	\underbrace{\Delta V\,I\,dt}_{\text{generatore}} = \underbrace{R\,I^2 dt}_{\text{effetto Joule}} + \underbrace{I\,d\Phi}_{\text{campo} \vec{B}}
\]
A sinistra abbiamo un termine legato al lavoro compiuto dal generatore. Il secondo è il lavoro speso per vincere l'effetto Joule. Il terzo è associato al lavoro speso per creare il campo magnetico all'interno del materiale. La variazione di flusso è:
\[
	\Phi (\vec{B}) = B\,S\,n_s\,l  \implies  d\Phi (\vec{B} ) = n_s\,S\,l\,dB
\]
Essa è legata alla variazione del campo $\vec{B}$. Pertanto il lavoro speso contro la $fem$ di autoinduzione è dato da
\[
	\mathcal{L}_i = \int d\mathcal{L}_i =\int I\,d\Phi = \int \underbrace{I n_s}_H\,dB\,\underbrace{S\,l}_{\tau} = \tau \int_0^{B_{\text{finale}}} H(B) \;dB
\]
E quindi l'energia magnetica per unità di volume è data da:\[
	\boxed{u_m = \int_0^{B_{\text{finale}}} H(B) \;dB}
\]
Possiamo scrivere $ \vec{H}  $ come
\[
	H = \frac{B}{\mu_0 \mu_r} \implies u_m = \int_0^{B_{\text{finale}}} \frac{B}{\mu_0 \mu_r} \;dB = \frac{1}{2} \frac{B^2_{\text{finale}}}{\mu_r \mu_0} = \frac{1}{2} \vec{B} \cdot \vec{H}
\]
Tale legge ha validità generale nei materiali diamagnetici e paramagnetici, mentre nei ferromagnetici si possono applicare solo in quelle situazioni in cui la permeabilità si può ritenere costante. Avevamo anche in questo caso trovato un espressione simile parlando di energia elettrostatica in presenza di dielettrici.

L'energia magnetica corrisponde al lavoro speso dal generatore per produrre il campo magnetico, nel vuoto e nei materiali, in aggiunta al lavoro speso per far circolare la corrente. Nei mezzi in cui vi è proporzionalità diretta fra $\vec{B}$ ed $\vec{H}$, i processi di magnetizzazione e smagnetizzazione avvengono reversibilmente. (Ci muoviamo lungo la stessa retta).
\begin{figure}[htpb]
	\centering
	

	\tikzset{every picture/.style={line width=0.75pt}} %set default line width to 0.75pt        

	\begin{tikzpicture}[x=0.75pt,y=0.75pt,yscale=-1,xscale=1]
	%uncomment if require: \path (0,300); %set diagram left start at 0, and has height of 300

	%Shape: Axis 2D [id:dp5096057532745446] 
	\draw  (167,142) -- (363.5,142)(265.5,43) -- (265.5,239.5) (356.5,137) -- (363.5,142) -- (356.5,147) (260.5,50) -- (265.5,43) -- (270.5,50)  ;
	%Straight Lines [id:da5479025621943316] 
	\draw    (177.5,230) -- (364.5,43) ;
	%Straight Lines [id:da5756808453246931] 
	\draw    (290.62,116.88) -- (333.38,74.12) ;
	\draw [shift={(335.5,72)}, rotate = 495] [fill={rgb, 255:red, 0; green, 0; blue, 0 }  ][line width=0.08]  [draw opacity=0] (10.72,-5.15) -- (0,0) -- (10.72,5.15) -- (7.12,0) -- cycle    ;
	\draw [shift={(288.5,119)}, rotate = 315] [fill={rgb, 255:red, 0; green, 0; blue, 0 }  ][line width=0.08]  [draw opacity=0] (10.72,-5.15) -- (0,0) -- (10.72,5.15) -- (7.12,0) -- cycle    ;

	% Text Node
	\draw (378,142) node    {$H$};
	% Text Node
	\draw (246,42) node    {$B$};


	\end{tikzpicture}
\end{figure}
\FloatBarrier
Consideriamo il caso di un materiale ferromagnetico. Il legame fra $\vec{B}$ ed $\vec{H}$ è quello che richiama il ciclo di isteresi. Il lavoro lungo il ciclo sarà dato da:
\[
	\mathcal{L}_i = \tau \oint H(B)\,dB
\]
Si tratta dell'area racchiusa dal ciclo. Sicuramente questo integrale è diverso da zero. Tale lavoro finisce in surriscaldamento del materiale ferromagnetico (si tratta di un ciclo irreversibile). Se si usa materiali ferromagnetici con ciclo moto larghi, ad ogni ciclo c'è una gran quantità di energia rilasciata e il materiale si scalda tanto. Se un materiale ferromagnetico deve essere sottoposto ad un campo magnetico variabile, come avviene ad esempio nei motori elettrici e nei trasformatori, conviene in generale che il ciclo di isteresi sia stretto, in modo da ridurre la perdita di energia.

Nel caso del solenoide possiamo adattare quello che abbiamo visto per la forza agente fra le armature di un condensatore collegato a generatore. Possiamo usare $U_m$ per determinare la pressione magnetica che agisce sul filo del solenoide.

Pressione magnetica e forza sui corpi magnetizzati. Consideriamo un tratto $l$ di un solenoide rettilineo. La forza infinitesima che agisce sul filo è data da:
\[
	d\vec{F} = I\,d\vec{l} \times \vec{B}
\]
diretta verso l'esterno che tende a far espande il solenoide. Possiamo introdurre dal centro del solenoide una coordinata radiale. Poniamo che il suo raggio sia $r$ (variabile) e proviamo a calcolarci l'energia magnetica $U_m$ contenuta al suo interno.
\begin{figure}[htpb]
	\centering
	

	\tikzset{every picture/.style={line width=0.75pt}} %set default line width to 0.75pt        

	\begin{tikzpicture}[x=0.75pt,y=0.75pt,yscale=-1,xscale=1]
	%uncomment if require: \path (0,300); %set diagram left start at 0, and has height of 300

	%Shape: Circle [id:dp7647812567427628] 
	\draw   (202,140.75) .. controls (202,96.15) and (238.15,60) .. (282.75,60) .. controls (327.35,60) and (363.5,96.15) .. (363.5,140.75) .. controls (363.5,185.35) and (327.35,221.5) .. (282.75,221.5) .. controls (238.15,221.5) and (202,185.35) .. (202,140.75) -- cycle ;
	%Shape: Circle [id:dp8735148170225524] 
	\draw   (212,140.75) .. controls (212,101.68) and (243.68,70) .. (282.75,70) .. controls (321.82,70) and (353.5,101.68) .. (353.5,140.75) .. controls (353.5,179.82) and (321.82,211.5) .. (282.75,211.5) .. controls (243.68,211.5) and (212,179.82) .. (212,140.75) -- cycle ;
	%Straight Lines [id:da05302453801430729] 
	\draw    (339.85,197.85) -- (374.5,232.5) ;
	\draw [shift={(376.62,234.62)}, rotate = 225] [fill={rgb, 255:red, 0; green, 0; blue, 0 }  ][line width=0.08]  [draw opacity=0] (10.72,-5.15) -- (0,0) -- (10.72,5.15) -- (7.12,0) -- cycle    ;
	%Straight Lines [id:da5595293977987936] 
	\draw    (225.65,197.85) -- (191,232.5) ;
	\draw [shift={(188.88,234.62)}, rotate = 315] [fill={rgb, 255:red, 0; green, 0; blue, 0 }  ][line width=0.08]  [draw opacity=0] (10.72,-5.15) -- (0,0) -- (10.72,5.15) -- (7.12,0) -- cycle    ;
	%Straight Lines [id:da29253742485468104] 
	\draw    (191,49) -- (225.65,83.65) ;
	\draw [shift={(188.88,46.88)}, rotate = 45] [fill={rgb, 255:red, 0; green, 0; blue, 0 }  ][line width=0.08]  [draw opacity=0] (10.72,-5.15) -- (0,0) -- (10.72,5.15) -- (7.12,0) -- cycle    ;
	%Straight Lines [id:da6465499849778398] 
	\draw    (374.5,49) -- (339.85,83.65) ;
	\draw [shift={(376.62,46.88)}, rotate = 135] [fill={rgb, 255:red, 0; green, 0; blue, 0 }  ][line width=0.08]  [draw opacity=0] (10.72,-5.15) -- (0,0) -- (10.72,5.15) -- (7.12,0) -- cycle    ;


	% Text Node
	\draw (367,79) node    {$r$};


	\end{tikzpicture}
\end{figure}
\FloatBarrier
\[
	U_m = \frac{1}{2} \frac{B^2}{\mu_0}\cdot \pi r^2 l = \frac{1}{2} \frac{\mu_0^2 n_s^2 I^2}{\mu_0}\cdot \pi r^2 l = \frac{\mu_0 n_s^2 I^2 \pi r^2 l}{2}
\]
La forza vale allora
\[
	F = \frac{dU_m}{dr} = \frac{\mu_0 n_s^2 I^2 \pi l}{2} \cdot 2r = \frac{\mu_0 n_s^2 I^2}{2} \underbrace{2\pi r\,l}_{\Sigma \text{ laterale}}
\]
Vediamo quindi che sulla superficie laterale del tratto di solenoide lungo $l$ agisce la pressione magnetica:
\[
	p_m = \frac{F}{2\pi r\,l} = \frac{\mu_0 n_s^2 I^2}{2} \frac{\mu_0}{\mu_0} = \frac{B^2}{2\mu_0}= u_m
\]
Nella progettazione meccanica di un solenoide bisogna tener conto della pressione magnetica. Ad esempio, in un solenoide capace di produrre un campo $B=1\,T$ la pressione è circa quattro volte quella atmosferica.
\[
	p_m \sim \frac{1}{2\cdot 4\pi \cdot 10^{-7}} \sim \frac{10^6}{2} Pa
\]
L'induttore deve essere in grado di sostenere un campo di questo valore, che tenderebbe a farlo esplodere







































\section{Circuiti in mutua induzione}

L'induzione tra due circuiti diventa fondamentale quando si hanno variazioni di corrente e quindi di flusso concatenato tra un circuito e l'altro o anche quando c'è movimento relativo. Due circuiti per cui il coefficiente di induzione mutua non è nullo si dicono accoppiati. Essi sono caratterizzati completamente dalla loro resistenza, dalla loro induttanza e dall'induttanza mutua.
\begin{figure}[htpb]
	\centering
	

	\tikzset{every picture/.style={line width=0.75pt}} %set default line width to 0.75pt        

	\begin{tikzpicture}[x=0.75pt,y=0.75pt,yscale=-1,xscale=1]
	%uncomment if require: \path (0,300); %set diagram left start at 0, and has height of 300

	%Shape: Ellipse [id:dp022353169953952978] 
	\draw   (192.67,223.17) .. controls (192.67,206.32) and (227.48,192.67) .. (270.42,192.67) .. controls (313.36,192.67) and (348.17,206.32) .. (348.17,223.17) .. controls (348.17,240.01) and (313.36,253.67) .. (270.42,253.67) .. controls (227.48,253.67) and (192.67,240.01) .. (192.67,223.17) -- cycle ;
	\draw   (261.07,248.27) -- (271.87,253.67) -- (261.07,259.07) ;
	%Straight Lines [id:da8413101360434543] 
	\draw    (221.47,221.87) -- (221.47,174.47) ;
	\draw [shift={(221.47,171.47)}, rotate = 450] [fill={rgb, 255:red, 0; green, 0; blue, 0 }  ][line width=0.08]  [draw opacity=0] (10.72,-5.15) -- (0,0) -- (10.72,5.15) -- (7.12,0) -- cycle    ;
	%Shape: Ellipse [id:dp9546111924139196] 
	\draw   (192.67,93.17) .. controls (192.67,76.32) and (227.48,62.67) .. (270.42,62.67) .. controls (313.36,62.67) and (348.17,76.32) .. (348.17,93.17) .. controls (348.17,110.01) and (313.36,123.67) .. (270.42,123.67) .. controls (227.48,123.67) and (192.67,110.01) .. (192.67,93.17) -- cycle ;
	\draw   (261.07,118.27) -- (271.87,123.67) -- (261.07,129.07) ;
	%Straight Lines [id:da19597076868122398] 
	\draw    (221.47,91.87) -- (221.47,44.47) ;
	\draw [shift={(221.47,41.47)}, rotate = 450] [fill={rgb, 255:red, 0; green, 0; blue, 0 }  ][line width=0.08]  [draw opacity=0] (10.72,-5.15) -- (0,0) -- (10.72,5.15) -- (7.12,0) -- cycle    ;

	% Text Node
	\draw (279.07,266.27) node    {$I_{2}$};
	% Text Node
	\draw (323.47,219.87) node    {$\Sigma _{2}$};
	% Text Node
	\draw (238.67,169.87) node    {$\vec{n}_{2}$};
	% Text Node
	\draw (279.07,136.27) node    {$I_{1}$};
	% Text Node
	\draw (323.47,89.87) node    {$\Sigma _{1}$};
	% Text Node
	\draw (238.67,39.87) node    {$\vec{n}_{1}$};


	\end{tikzpicture}
\end{figure}
\FloatBarrier
Il circuito $2$ ($1$) genererà un campo magnetico che provocherà un flusso $\Phi_{21}$ ($\Phi_{12}$) attraverso il circuito $1$ ($2$) dato da:
\[
	\Phi_{21} = \int_{\Sigma_1}  \vec{B}_2 \cdot \vec{n}_1 \, dS = M_{21}\,I_2 \qquad \Phi_{12} = \int_{\Sigma_2}  \vec{B}_1 \cdot \vec{n}_2 \, dS = M_{12}\,I_1
\]
Possiamo immaginare di schematizzare i due circuiti con il simbolismo di sotto
\begin{gather*}
	\Delta V_1 + f_{21} = R_1 I_1 \implies f_{21} = - L_1  \frac{dI_1}{dt} - M \frac{dI_2}{dt} \\
	\Delta V_2 + f_{12} = R_2 I_2 \implies f_{12} = - L_2  \frac{dI_2}{dt} - M \frac{dI_1}{dt}
\end{gather*}
Notiamo che le $f_{21} $ e $f_{12} $ sono date dalla somma della $fem$ di autoinduzione e di quella indotta in un circuito dalla variazione di corrente nell'altro.
\begin{figure}[htpb]
	\centering
	

	\tikzset{every picture/.style={line width=0.75pt}} %set default line width to 0.75pt        

	\begin{tikzpicture}[x=0.75pt,y=0.75pt,yscale=-0.9,xscale=0.9]
	%uncomment if require: \path (0,446); %set diagram left start at 0, and has height of 446

	%Straight Lines [id:da5450385787974983] 
	\draw    (229,76) -- (300,76) ;
	%Shape: Resistor [id:dp8836135854518683] 
	\draw   (300,76) -- (314.4,76) -- (317.6,56) -- (324,96) -- (330.4,56) -- (336.8,96) -- (343.2,56) -- (349.6,96) -- (356,56) -- (362.4,96) -- (365.6,76) -- (380,76) ;
	%Shape: Inductor (Air Core) [id:dp5572793678529586] 
	\draw   (300,178.5) -- (314.4,178.5) .. controls (314.61,167.99) and (316.61,159.01) .. (319.43,155.87) .. controls (322.26,152.73) and (325.35,156.07) .. (327.2,164.29) .. controls (328.63,170.7) and (329.21,178.98) .. (328.8,187.03) .. controls (328.8,190.17) and (328.08,192.71) .. (327.2,192.71) .. controls (326.32,192.71) and (325.6,190.17) .. (325.6,187.03) .. controls (325.19,178.98) and (325.77,170.7) .. (327.2,164.29) .. controls (328.86,157.46) and (331.18,153.59) .. (333.6,153.59) .. controls (336.02,153.59) and (338.34,157.46) .. (340,164.29) .. controls (341.43,170.7) and (342.01,178.98) .. (341.6,187.03) .. controls (341.6,190.17) and (340.88,192.71) .. (340,192.71) .. controls (339.12,192.71) and (338.4,190.17) .. (338.4,187.03) .. controls (337.99,178.98) and (338.57,170.7) .. (340,164.29) .. controls (341.66,157.46) and (343.98,153.59) .. (346.4,153.59) .. controls (348.82,153.59) and (351.14,157.46) .. (352.8,164.29) .. controls (354.23,170.7) and (354.81,178.98) .. (354.4,187.03) .. controls (354.4,190.17) and (353.68,192.71) .. (352.8,192.71) .. controls (351.92,192.71) and (351.2,190.17) .. (351.2,187.03) .. controls (350.79,178.98) and (351.37,170.7) .. (352.8,164.29) .. controls (354.65,156.07) and (357.74,152.73) .. (360.57,155.87) .. controls (363.39,159.01) and (365.39,167.99) .. (365.6,178.5) -- (380,178.5) ;
	%Shape: Battery [id:dp46637736818803077] 
	\draw  [fill={rgb, 255:red, 0; green, 0; blue, 0 }  ,fill opacity=1 ] (229,178.5) -- (229,132.38) (208,122.13) -- (250,122.13) (229,122.13) -- (229,76) (218.5,136.48) -- (218.5,132.38) -- (239.5,132.38) -- (239.5,136.48) -- (218.5,136.48) -- cycle ;
	%Straight Lines [id:da5138357380761005] 
	\draw    (380,76) -- (451,76) ;
	%Straight Lines [id:da11798535931656229] 
	\draw    (451,76) -- (451,178.5) ;
	\draw [shift={(451,127.25)}, rotate = 270] [fill={rgb, 255:red, 0; green, 0; blue, 0 }  ][line width=0.08]  [draw opacity=0] (10.72,-5.15) -- (0,0) -- (10.72,5.15) -- (7.12,0) -- cycle    ;
	%Straight Lines [id:da7247473825711936] 
	\draw    (380,178.5) -- (451,178.5) ;
	%Straight Lines [id:da6229977282962058] 
	\draw    (229,178.5) -- (300,178.5) ;
	%Straight Lines [id:da5438168256387526] 
	\draw    (229,322.71) -- (300,322.71) ;
	%Shape: Resistor [id:dp03383264680160458] 
	\draw   (300,322.71) -- (314.4,322.71) -- (317.6,342.71) -- (324,302.71) -- (330.4,342.71) -- (336.8,302.71) -- (343.2,342.71) -- (349.6,302.71) -- (356,342.71) -- (362.4,302.71) -- (365.6,322.71) -- (380,322.71) ;
	%Shape: Inductor (Air Core) [id:dp2169359568934608] 
	\draw   (300,220.21) -- (314.4,220.21) .. controls (314.61,230.72) and (316.61,239.7) .. (319.43,242.85) .. controls (322.26,245.99) and (325.35,242.65) .. (327.2,234.43) .. controls (328.63,228.02) and (329.21,219.73) .. (328.8,211.69) .. controls (328.8,208.55) and (328.08,206) .. (327.2,206) .. controls (326.32,206) and (325.6,208.55) .. (325.6,211.69) .. controls (325.19,219.73) and (325.77,228.02) .. (327.2,234.43) .. controls (328.86,241.26) and (331.18,245.13) .. (333.6,245.13) .. controls (336.02,245.13) and (338.34,241.26) .. (340,234.43) .. controls (341.43,228.02) and (342.01,219.73) .. (341.6,211.69) .. controls (341.6,208.55) and (340.88,206) .. (340,206) .. controls (339.12,206) and (338.4,208.55) .. (338.4,211.69) .. controls (337.99,219.73) and (338.57,228.02) .. (340,234.43) .. controls (341.66,241.26) and (343.98,245.13) .. (346.4,245.13) .. controls (348.82,245.13) and (351.14,241.26) .. (352.8,234.43) .. controls (354.23,228.02) and (354.81,219.73) .. (354.4,211.69) .. controls (354.4,208.55) and (353.68,206) .. (352.8,206) .. controls (351.92,206) and (351.2,208.55) .. (351.2,211.69) .. controls (350.79,219.73) and (351.37,228.02) .. (352.8,234.43) .. controls (354.65,242.65) and (357.74,245.99) .. (360.57,242.85) .. controls (363.39,239.7) and (365.39,230.72) .. (365.6,220.21) -- (380,220.21) ;
	%Shape: Battery [id:dp7599691141751248] 
	\draw  [fill={rgb, 255:red, 0; green, 0; blue, 0 }  ,fill opacity=1 ] (229,220.21) -- (229,266.34) (208,276.59) -- (250,276.59) (229,276.59) -- (229,322.71) (218.5,262.24) -- (218.5,266.34) -- (239.5,266.34) -- (239.5,262.24) -- (218.5,262.24) -- cycle ;
	%Straight Lines [id:da47980589411006647] 
	\draw    (380,322.71) -- (451,322.71) ;
	%Straight Lines [id:da14423806812067075] 
	\draw    (451,322.71) -- (451,220.21) ;
	\draw [shift={(451,271.46)}, rotate = 450] [fill={rgb, 255:red, 0; green, 0; blue, 0 }  ][line width=0.08]  [draw opacity=0] (10.72,-5.15) -- (0,0) -- (10.72,5.15) -- (7.12,0) -- cycle    ;
	%Straight Lines [id:da2656525495579487] 
	\draw    (380,220.21) -- (451,220.21) ;
	%Straight Lines [id:da07229214274510931] 
	\draw    (229,220.21) -- (300,220.21) ;

	% Text Node
	\draw (305,94) node    {$R_{1}$};
	% Text Node
	\draw (305,144) node    {$L_{1}$};
	% Text Node
	\draw (470,121) node    {$\vec{J}_{1}$};
	% Text Node
	\draw (187,127) node    {$\Delta V_{1}$};
	% Text Node
	\draw (305,244) node    {$L_{2}$};
	% Text Node
	\draw (305,301) node    {$R_{2}$};
	% Text Node
	\draw (470,271) node    {$\vec{J}_{1}$};
	% Text Node
	\draw (187,269) node    {$\Delta V_{2}$};


	\end{tikzpicture}
\end{figure}
\FloatBarrier
Otteniamo quindi il seguente sistema di equazione differenziali accoppiate:
\[
	\Delta V_1 - L_1  \frac{dI_1}{dt} - M \frac{dI_2}{dt} = R_1I_1  \qquad  \Delta V_2 - L_2  \frac{dI_2}{dt} - M \frac{dI_1}{dt} = R_2I_2
\]
Dove il termine di accoppiamento in ciascuna è quello contenente $M$.







































\section{Energia magnetica di circuiti accoppiati}

Facciamo un analisi del bilancio energetico del sistema visto nel paragrafo precedente.
\begin{gather*}
	I_1\,dt\,\Delta V_1 = I_1\,dt \left[ R_1I_1+L_1\frac{dI_1}{dt}+M\frac{dI_2}{dt}    \right] \\
	I_2\,dt\,\Delta V_2 = I_2\,dt \left[ R_2I_2+L_2\frac{dI_2}{dt}+M\frac{dI_1}{dt}    \right]
\end{gather*}
Sommando il tutto, dato che siamo interessati all'energia legata al sistema nel complesso si ottiene il lavoro che i due generatori compiono nell'istante di tempo.
\begin{align*}
	\Delta V_1I_1dt + \Delta V_2I_2dt &= (R_1I_1^2 +R_2I_2^2 )dt + \\
	&\quad + (L_1I_1dI_1+L_2I_2dI_2) + \tag*{$d\mathcal{L}_{ai} $}\\
	&\quad + (MI_1dI_2 + MI_2dI_1) \tag*{$d\mathcal{L}_{mi} $}
\end{align*}
L'ultimo termine è il lavoro di mutua induzione. Quello al centro è il lavoro di autoinduzione.
\begin{align*}
	\mathcal{L} = U_m &= \int d\mathcal{L}_{ai} + \int d\mathcal{L}_{mi} \\
	&= \int_0^{I_{1\infty}} L_1I_1dI_1 + \int_0^{I_{2\infty}} L_2I_2dI_2 + \iint_0^{I_{1\infty}I_{2\infty}} M d(I_1I_2) \\		&= \frac{1}{2} L_1I_{1\infty}^2 + \frac{1}{2} L_2I_{2\infty}^2 + MI_{1\infty}I_{2\infty}
\end{align*}
L'ultimo termine è chiamata \textbf{energia di accoppiamento}. L'energia non è solo quella che serve per creare il campo separatamente, ma c'è un termine di interazione che deve essere preso in considerazione.
Per estendere il calcolo dell'energia magnetica ad un sistema di $n$ circuiti accoppiati riscriviamo l'ultima espressione trovata per Um in modo che sia suscettibile di generalizzazione. Posto $L_1=M_{11}$, $L_2=M_{22}$ si ha, per $n$ circuiti:
\begin{align*}
	U_m &= \frac{1}{2} M_{11} I_1^2 + \frac{1}{2} M_{22}I_2^2 + \frac{1}{2} M_{12}I_1I_2 + \frac{1}{2} M_{21}I_2I_1 = \frac{1}{2}  \sum_{k,i=1}^n M_{ik}I_iI_k   \\
	U_m &= \frac{1}{2} L_1I_{1\infty}^2 + \frac{1}{2} M_{21} I_{1\infty}I_{2\infty} + \frac{1}{2} L_2I_{2\infty}^2  + \frac{1}{2}  M_{12}I_{2\infty}I_{1\infty} \\
	&= \frac{I_{1\infty}}{2} \underbrace{(L_1I_{1\infty} +M_{21}I_{2\infty})}_{\Phi_1} +\frac{I_{2\infty}}{2} \underbrace{(L_2I_{2\infty} +M_{12}I_{I\infty})}_{\Phi_2} \\
	&= \frac{1}{2} \sum_{i=1}^n I_i\Phi_i
\end{align*}
Questa espressione contiene tutti i flussi di tutti i campi magnetici che passano attraverso il circuito $i$-esimo.







































\section{Corrente di spostamento, legge di Ampere - Maxwell}

In regime stazionario la corrente ha gli stessi valori in tutti i punti di un filo conduttore e quindi, qualunque superficie io scelga, la corrente concatenata è sempre la stessa. Nella legge di Ampere quindi, la superficie $ \Sigma  $ attraverso cui si calcola il flusso della densità di corrente è una qualsiasi superficie avente come contorno la linea che concatena $I$ e lungo cui si calcola la circuitazione di $\vec{B}$. Applicando l'operatore divergenza alla quarta equazione di Maxwell abbiamo constatato che essa è in accordo con la conservazione della carica elettrica solo in regime stazionario. Infatti calcolando la divergenza di entrambi i membri dell'equazione si trova:
\[
	\vec{\nabla} (\vec{\nabla} \times \vec{B}) = 0 = \mu_0 \vec{\nabla} \vec{J}_c \implies \vec{\nabla} \vec{J}_c = 0
\]
Che è appunto la forma differenziale della legge di conservazione della carica elettrica nei processi non dipendenti dal tempo. Ma nel caso generale in cui la densità di carica varia nel tempo, la densità di corrente non ha divergenza nulla. La non validità della legge di Ampere in condizioni non stazionarie si può constatare nel processo di carica di un condensatore.
In regime variabile nel tempo, circolerà una corrente $I(t)$. Sulle armature di questo condensatore si accumula o si scarica della quantità di carica in funzione del tempo. Supponiamo di voler determinare il campo magnetico nella regione all'interno del condensatore. Consideriamo una linea chiusa $\lambda$ e calcoliamo la circuitazione di $\vec{B}$ lungo $\lambda$. Scegliamo due superfici $\Sigma_1$ e $\Sigma_2$ come in figura. Attraverso una superficie chiusa che racchiude le armature abbiamo un flusso netto di carica che è nullo, come se ci fosse continuità nel circuito.
\begin{figure}[htpb]
	\centering
	

	\tikzset{every picture/.style={line width=0.75pt}} %set default line width to 0.75pt        

	\begin{tikzpicture}[x=0.75pt,y=0.75pt,yscale=-1,xscale=1]
	%uncomment if require: \path (0,300); %set diagram left start at 0, and has height of 300

	%Straight Lines [id:da26715603099101415] 
	\draw    (313,64.5) -- (313,126.67) ;
	%Shape: Contact [id:dp9021730769009411] 
	\draw   (313,126.67) -- (313,161.27) (313,242) -- (313,207.4) (406,161.27) -- (220,161.27) (406,207.4) -- (220,207.4) ;
	%Curve Lines [id:da17311446532911834] 
	\draw    (238.97,127.71) .. controls (231.22,65.58) and (374.04,57.44) .. (385.65,127.63) ;
	%Curve Lines [id:da04270615963409807] 
	\draw    (238.97,127.71) .. controls (229.55,146.77) and (192.04,162.87) .. (185.38,173.94) .. controls (178.72,185.01) and (202.92,191.04) .. (316.92,189.97) .. controls (430.91,188.9) and (450.73,182.47) .. (440.73,171.72) .. controls (430.72,160.98) and (390.91,145.93) .. (385.65,127.63) ;
	%Shape: Resistor [id:dp06847943880021123] 
	\draw   (88,64.5) -- (88,78.9) -- (108,82.1) -- (68,88.5) -- (108,94.9) -- (68,101.3) -- (108,107.7) -- (68,114.1) -- (108,120.5) -- (68,126.9) -- (88,130.1) -- (88,144.5) ;
	%Straight Lines [id:da6677276818736742] 
	\draw    (88,64.5) -- (313,64.5) ;
	%Straight Lines [id:da11229733873192393] 
	\draw    (88,144.5) -- (88,242) ;
	%Straight Lines [id:da9496446795307356] 
	\draw    (88,242) -- (313,242) ;
	%Shape: Circle [id:dp5729624003302884] 
	\draw  [fill={rgb, 255:red, 222; green, 222; blue, 222 }  ,fill opacity=1 ] (69.33,182.95) .. controls (69.33,172.45) and (77.84,163.95) .. (88.33,163.95) .. controls (98.83,163.95) and (107.33,172.45) .. (107.33,182.95) .. controls (107.33,193.44) and (98.83,201.95) .. (88.33,201.95) .. controls (77.84,201.95) and (69.33,193.44) .. (69.33,182.95) -- cycle ;
	%Shape: Sine Wave Form [id:dp12969099670743334] 
	\draw   (73.64,182.95) .. controls (79.62,172.02) and (82.44,171.97) .. (88.33,182.95) .. controls (94.23,193.94) and (96.99,194.01) .. (103.03,182.95) ;

	%Curve Lines [id:da23731231610916503] 
	\draw  [dash pattern={on 0.84pt off 2.51pt}]  (238.97,127.71) .. controls (238.33,108.17) and (385.33,108.17) .. (385.65,127.63) ;
	%Curve Lines [id:da42849339813073684] 
	\draw    (238.97,127.55) .. controls (238.33,147.09) and (385.33,147.09) .. (385.65,127.63) ;

	% Text Node
	\draw (397,120.67) node    {$\gamma $};
	% Text Node
	\draw (377.67,84) node    {$\Sigma _{1}$};
	% Text Node
	\draw (448.67,158) node    {$\Sigma _{2}$};


	\end{tikzpicture}
\end{figure}
\FloatBarrier
Se consideriamo una superficie chiusa che racchiude una sola armatura il flusso di $\vec{J}$ non è nullo perché c'è una carica entrante o uscente, ma nello spazio tra le armature del condensatore non c'è nessun passaggio di carica. Il problema non è ben posto: il concetto di concatenazione funziona in regime stazionario. In tal caso non ci sarebbero correnti concatenate alle due superfici perché la carica sul condensatore è pari a zero. Maxwell propose di estendere il significato della densità di corrente. Il principio di conservazione della carica si può esprimere con il principio di continuità della corrente:
\[
	\left. \begin{array}{r}
	 	\vec{\nabla} \cdot \vec{J}_c = -\frac{\partial \rho_{\text{lib}}}{\partial t} \\
		\vec{\nabla} \cdot \vec{D} = \rho_{\text{lib}}
	\end{array} \right\} \implies \vec{\nabla} \cdot \vec{J}_c = - \frac{\partial}{\partial t} [\vec{\nabla} \cdot \vec{D} ]
\]
Quindi
\[
	\vec{\nabla} \cdot \left[ \vec{J}_c + \underbrace{\frac{\partial \vec{D}}{\partial t}}_{\vec{J}_s} \right] = 0
\]
La quantità $ \vec{J}_s = \frac{\partial \vec{D}}{\partial t} $ viene chiamata \textbf{densità di corrente di spostamento}. Dentro la quadra abbiamo la somma di due densità di corrente, che chiamiamo $\vec{J}_{\text{tot}}$, densità di corrente totale generalizzata $ \vec{J}_{\text{tot}} = \vec{J}_c + \vec{J}_s $. Abbiamo trovato una nuova densità di corrente che è sempre solenoidale in ogni regime. Nel circuito $RC$, sulla superficie $ \Sigma_2  $ è nulla la densità di corrente di conduzione, ma è diverso da zero il campo elettrico generato dalle cariche che stanno sulle armature. D'altra parte su $ \Sigma_1  $ possiamo pensare nullo $\vec{E}$
\[
	\vec{\nabla} \cdot \vec{J}_{\text{tot}} = 0 \implies \int_{\Sigma}\vec{J}_{\text{tot}}\cdot \vec{n} \,dS = 0
\]
Che possiamo pensare come somma dei flussi
\begin{equation*}
	\begin{aligned}
		\int_{\Sigma}\vec{J}_{\text{tot}}\cdot \vec{n} \,dS &= 0 \\
		\underbrace{\int_{\Sigma_1}\vec{J}_{\text{tot}}\cdot (-\vec{n}_1)  \,dS}_{-I_{\text{tot}1}} + \underbrace{\int_{\Sigma_2}\vec{J}_{\text{tot}}\cdot \vec{n}_2 \,dS}_{I_{\text{tot}2}} &= 0 \implies \boxed{I_{\text{tot}1}=I_{\text{tot}2}}
	\end{aligned}
\end{equation*}
$\vec{J}_{\text{tot}} $ ha lo stesso valore in tutto il circuito: esso coincide con la corrente di conduzione nei cavi di collegamento e con la corrente di spostamento all'interno del condensatore.

Nei conduttori di solito $ \vec{J}_s  $ è trascurabile rispetto a $ \vec{J}_c $.
\begin{equation*}
	\begin{aligned}
		I_{\text{tot}1} &= \int_{\Sigma_1} \vec{J}_{\text{tot}} \cdot \vec{n}_1 \,dS = \int_{\Sigma_1} [\vec{J}_c+\vec{J}_s ] \cdot \vec{n}_1 \,dS \simeq \int_{\Sigma_1} \vec{J}_c \cdot \vec{n}_1 \,dS = I_c \\
		I_{\text{tot}2} &= \int_{\Sigma_2} \vec{J}_{\text{tot}} \cdot \vec{n}_2 \,dS = \int_{\Sigma_2} [\vec{J}_c+\vec{J}_s ] \cdot \vec{n}_2 \,dS \simeq \int_{\Sigma_2} \vec{J}_s \cdot \vec{n}_1 \,dS = I_s \\
	\end{aligned}
\end{equation*}
Inoltre
\[
	I_s = \int_{\Sigma_2} \vec{J}_s \cdot \vec{n}_1 \,dS = \int_{\Sigma_2} \frac{\partial \vec{D}}{\partial t} \cdot \vec{n}_1 \,dS = \frac{\partial}{\partial t} \int_{\Sigma_2} \vec{D} \cdot \vec{n}_1 = \frac{\partial \Phi_{\Sigma_2} \vec{D}}{\partial t} \,dS
\]
Abbiamo quindi un nuovo modo per definire in modo corretto il problema di concatenazione grazie a questa nuova densità di corrente generalizzata. Possiamo finalmente calcolare la circuitazione del campo magnetico lungo $ \lambda $:
\[
	\boxed{\oint_{\gamma} \vec{B} \cdot d\vec{l} = \mu_0 \,I_{\text{tot}}}
\]
Questo risultato è chiamato legge di Ampere-Maxwell: essa attribuisce gli stessi effetti magnetici di una corrente di conduzione alle variazioni temporali del campo elettrico. Il termine corrente di spostamento non deve trarre in inganno: alla densità non è collegato nessun moto di carica. La legge ci porta ad un risultato simile alla legge di Faraday. Nel primo caso si parla della circuitazione di $\vec{B}$, nel secondo caso di quella di $\vec{E}$.
\begin{equation*}
	\begin{aligned}
		\oint_{\gamma} \vec{B} \cdot d\vec{l} &= \mu_0 I_{\text{tot}} \\
		\int_{\Sigma} \text{rot}\vec{B} \cdot \vec{n} \, dS &= \mu_0 \int_{\Sigma}\vec{J}_{\text{tot}}\cdot \vec{n} \, dS \\
		\int_{\Sigma} \text{rot}\vec{B} \cdot \vec{n} \, dS &= \int_{\Sigma}\mu_0 \left[ \vec{J}_c + \frac{\partial \vec{D}}{\partial t} \right] \cdot \vec{n} \, dS
	\end{aligned}
\end{equation*}
Queste due espressioni sono sempre uguali qualunque sia la scelta di $\Sigma$ perché l'uguaglianza deriva dal teorema di Stokes e da una legge generalizzata da Maxwell. Gli integrali sono calcolati sulla stessa superficie, quindi devono essere uguali
\[
	\boxed{\text{rot}\vec{B} = \mu_0 \left(\vec{J}_c + \frac{\partial \vec{D}}{\partial t} \right)}
\]
Potremmo trovarci in una situazione più ampia in cui oltre alle correnti di conduzione può esserci anche un mezzo magnetizzato. Le correnti di magnetizzazione danno un ulteriore contributo al campo e non alterano la solenoidalità di $\vec{J}_{\text{tot}} $
\begin{equation*}
	\begin{aligned}
		\text{rot}\vec{B} &= \mu_0 \left[ \vec{J}_c+\frac{\partial \vec{D}}{\partial t} + \vec{J}_m \right] \\
		\frac{\text{rot}\vec{B}}{\mu_0} &= \left[ \vec{J}_c+\frac{\partial \vec{D}}{\partial t} + \text{rot}\vec{M}  \right] \\
		\text{rot} \left[ \frac{\vec{B}}{\mu_0} - \vec{M}  \right] &= \vec{J}_c + \frac{\partial \vec{D}}{\partial t} \\
	\end{aligned}
\end{equation*}
\[
	\boxed{\text{rot}\vec{H} = \vec{J}_c + \frac{\partial \vec{D}}{\partial t}}
\]
\textbf{Riassunto Equazioni di Maxwell.}
\begin{align}
	\text{div}\vec{D} &= \rho_{\text{lib}} (t) \\
	\text{rot}\vec{E} &= -\frac{\partial \vec{B}}{\partial t} \\
	\text{div}\vec{B} &= 0 \\
	\text{rot}\vec{H} &= \vec{J}_c + \frac{\partial \vec{D}}{\partial t}
\end{align}
A queste bisogna aggiungere le relazioni esplicite tra $ \vec{E}  $ e $ \vec{D}  $ e tra $\vec{B}$ e $ \vec{H}  $. Nei mezzi isotropi e omogeni si ha $ \vec{D} = \varepsilon_0 \varepsilon_r \vec{E}  $. Nei materiali diamagnetici e paramagnetici si ha $ \vec{B} = \mu_0 \mu_r \vec{H} $
\[
	\text{rot}\vec{B} = \mu_0 \left[ \vec{J}_c + \vec{J}_m + \varepsilon_0 \frac{\partial \vec{E}}{\partial t} + \frac{\partial \vec{P}}{\partial t} \right]
\]
Possiamo interpretare la derivata di $\vec{P}$ rispetto al tempo come qualcosa di legato all'oscillazione di dipoli causati da campi magnetici variabili nel tempo. Una situazione particolarmente significativa si ha nello spazio vuoto privo di correnti e di cariche. In tal caso si ha
\[
	\text{rot}\vec{E} = \frac{\partial \vec{B}}{\partial t} \qquad \text{rot}\vec{B} = \mu_0 \varepsilon_0 \frac{\partial \vec{E}}{\partial t}
\]
Ogni campo vettoriale può essere scomposto su tre assi coordinati. In tutto abbiamo sei componenti scalari, sei incognite. Le equazioni a disposizione però sono in numero maggiore:
\begin{itemize}
	\item Le due equazioni sui rotori sono vettoriali
	\item Le equazioni sulle divergenze sono di per sé scalari
\end{itemize}
Abbiamo otto equazioni scalari. Il sistema delle equazioni di Maxwell appare sovradimensionato. \emph{In effetti si può notare che le equazioni che contengono veramente tutte le informazioni sono solo le due equazioni sui rotori}. Le altre due equazioni sono già in accordo con esse. Se uno applica a rotore di $\vec{E}$ o di $\vec{B}$ la divergenza da entrambe le parti si ritrovano le altre equazioni.
\emph{Le due equazioni sulle divergenze sono già in accordo con le altre due. Esse sono qualcosa in più e ci dicono qualcosa sulla struttura dei due campi.} Avremo bisogno di loro per capire come sono fatti i campi studiando la propagazione delle onde elettromagnetiche.
Dalle due equazioni notiamo che se consideriamo il caso nel vuoto, un campo magnetico variabile nel tempo genera un campo elettrico. Ma vale anche il caso contrario. I due aspetti sono sempre assieme, ecco perché si parla di campo elettromagnetico. La caratteristica si può vedere anche quando parliamo di campi in regime stazionario.

\textbf{Esempi.}
Consideriamo una regione in cui è presente un campo magnetico $\vec{B}$ uniforme e costante nel tempo. Poniamo che in questa regione dello spazio l'osservatore veda transitare un carica $q$ con una certa velocità $v$. In questo istante non ci sono campi elettrici. Sulla carica agirà la forza di Lorentz pari a:
\[
	\vec{F} = q\,\vec{v} \times \vec{B}
\]
Supponiamo che da queste parti stia transitando un secondo osservatore che si muove per puro caso con la stessa velocità $v$ della particella. L'osservatore $O'$ vede la particella ferma rispetto a sé.
\[
	\vec{v'}=0
\]
Tuttavia nella meccanica classica quando due osservatori sono su un sistema di riferimento inerziale, misurano le stesse forze. La forza misurata da $O'$ non può essere una forza di Lorentz perché la carica è in quiete. L'osservatore $O$ vedrebbe carica doppia se raddoppiassimo la carica, quindi tale forza deve essere proporzionale alla carica. L'unica spiegazione deve essere che sulla carica agisce un campo elettrico $\vec{E}'$ pari a:
\[
	\vec{E'} = \frac{\vec{F'}}{q} = \frac{\vec{F}}{q} = \vec{v} \times \vec{B}
\]
\emph{Campi elettrici e magnetici non sono invarianti passando da un sistema di riferimento all'altro}. Ci si aspettava che siccome le forze non cambiano, nemmeno i due campi sarebbero dovuti cambiare.

Immaginiamo il seguente sistema. Supponiamo che ci sia un filo uniformemente carico caratterizzato da una certa densità di carica lineare $\lambda$. Vi è inoltre un osservatore $O'$ che si muove parallelamente al filo, esso ha l'impressione che le cariche si spostino in direzione opposta a sé. Ma se c'è una corrente, per la legge di Biot-Savart c'è un campo magnetico
\begin{align*}
	\vec{E} = \frac{\lambda}{2\pi \varepsilon_0 r} \vec{u}_n \qquad \vec{B'} &= \frac{\mu_0 I'}{2\pi r} \vec{u}_{\varphi} = \frac{\mu_0 \lambda v}{2\pi r} \vec{u}_{\varphi} = \varepsilon_0 \mu_0 \frac{\lambda}{2\pi \varepsilon_0 r} v \vec{u}_{\varphi} \\
	&= \mu_0 \varepsilon_0 E \, v \, \vec{u}_{\varphi} = \frac{E\,v}{c^2}\vec{u}_{\varphi} = \vec{E} \times \frac{\vec{v}}{c^2} \\
	&= - \frac{\vec{v}}{c^2} \times \vec{E}
\end{align*}
\begin{figure}[htpb]
	\centering
	

	\tikzset{every picture/.style={line width=0.75pt}} %set default line width to 0.75pt        

	\begin{tikzpicture}[x=0.75pt,y=0.75pt,yscale=-1,xscale=1]
	%uncomment if require: \path (0,366); %set diagram left start at 0, and has height of 366

	%Straight Lines [id:da6831299920859881] 
	\draw    (164,226.41) -- (200.38,226.41) ;
	\draw [shift={(203.38,226.41)}, rotate = 180] [fill={rgb, 255:red, 0; green, 0; blue, 0 }  ][line width=0.08]  [draw opacity=0] (10.72,-5.15) -- (0,0) -- (10.72,5.15) -- (7.12,0) -- cycle    ;
	%Straight Lines [id:da5079842846749154] 
	\draw    (164,226.41) -- (164,190.04) ;
	\draw [shift={(164,187.04)}, rotate = 450] [fill={rgb, 255:red, 0; green, 0; blue, 0 }  ][line width=0.08]  [draw opacity=0] (10.72,-5.15) -- (0,0) -- (10.72,5.15) -- (7.12,0) -- cycle    ;
	%Straight Lines [id:da46686264765372854] 
	\draw    (164,226.41) -- (138.28,252.13) ;
	\draw [shift={(136.16,254.25)}, rotate = 315] [fill={rgb, 255:red, 0; green, 0; blue, 0 }  ][line width=0.08]  [draw opacity=0] (10.72,-5.15) -- (0,0) -- (10.72,5.15) -- (7.12,0) -- cycle    ;
	%Straight Lines [id:da35801584179398116] 
	\draw    (129,162) -- (517,162) ;
	%Straight Lines [id:da7987795836241427] 
	\draw    (375,162) -- (498,162) ;
	\draw [shift={(501,162)}, rotate = 180] [fill={rgb, 255:red, 0; green, 0; blue, 0 }  ][line width=0.08]  [draw opacity=0] (10.72,-5.15) -- (0,0) -- (10.72,5.15) -- (7.12,0) -- cycle    ;
	%Straight Lines [id:da4216029542897226] 
	\draw    (394,86.41) -- (430.38,86.41) ;
	\draw [shift={(433.38,86.41)}, rotate = 180] [fill={rgb, 255:red, 0; green, 0; blue, 0 }  ][line width=0.08]  [draw opacity=0] (10.72,-5.15) -- (0,0) -- (10.72,5.15) -- (7.12,0) -- cycle    ;
	%Straight Lines [id:da6488797391063266] 
	\draw    (394,86.41) -- (394,50.04) ;
	\draw [shift={(394,47.04)}, rotate = 450] [fill={rgb, 255:red, 0; green, 0; blue, 0 }  ][line width=0.08]  [draw opacity=0] (10.72,-5.15) -- (0,0) -- (10.72,5.15) -- (7.12,0) -- cycle    ;
	%Straight Lines [id:da3740424598573331] 
	\draw    (394,86.41) -- (368.28,112.13) ;
	\draw [shift={(366.16,114.25)}, rotate = 315] [fill={rgb, 255:red, 0; green, 0; blue, 0 }  ][line width=0.08]  [draw opacity=0] (10.72,-5.15) -- (0,0) -- (10.72,5.15) -- (7.12,0) -- cycle    ;

	% Text Node
	\draw (167.8,235.79) node    {$O$};
	% Text Node
	\draw (143,149) node    {$+$};
	% Text Node
	\draw (163,149) node    {$+$};
	% Text Node
	\draw (183,149) node    {$+$};
	% Text Node
	\draw (203,149) node    {$+$};
	% Text Node
	\draw (223,149) node    {$+$};
	% Text Node
	\draw (243,149) node    {$+$};
	% Text Node
	\draw (263,149) node    {$+$};
	% Text Node
	\draw (283,149) node    {$+$};
	% Text Node
	\draw (303,149) node    {$+$};
	% Text Node
	\draw (323,149) node    {$+$};
	% Text Node
	\draw (343,149) node    {$+$};
	% Text Node
	\draw (363,149) node    {$+$};
	% Text Node
	\draw (383,149) node    {$+$};
	% Text Node
	\draw (403,149) node    {$+$};
	% Text Node
	\draw (423,149) node    {$+$};
	% Text Node
	\draw (443,149) node    {$+$};
	% Text Node
	\draw (463,149) node    {$+$};
	% Text Node
	\draw (483,149) node    {$+$};
	% Text Node
	\draw (127.24,253.01) node    {$x$};
	% Text Node
	\draw (214.09,224.67) node    {$y$};
	% Text Node
	\draw (155.03,181.99) node    {$z$};
	% Text Node
	\draw (499,177.33) node    {$I'$};
	% Text Node
	\draw (117,159.33) node    {$\lambda $};
	% Text Node
	\draw (400.8,97.79) node    {$O'$};
	% Text Node
	\draw (356.24,114.01) node    {$x'$};
	% Text Node
	\draw (444.09,84.67) node    {$y'$};
	% Text Node
	\draw (384.03,41.99) node    {$z'$};


	\end{tikzpicture}
\end{figure}
\FloatBarrier
A seconda del sistema in cui ci troviamo si manifestano i due campi. È l'idea alla base della relatività ristretta di Einstein. Egli riconobbe che i due campi fanno parte dello stesso oggetto. Siccome i due campi vettoriali hanno tre componenti ciascuno, per rappresentarli in un unico oggetto usiamo una matrice, il tensore elettromagnetico. Nello spazio tempo quando passo da un sistema di riferimento inerziale a un altro è come se stessi facendo una rotazione di coordinare spazio-temporali e con un nuovo sistema di coordinate le componenti dei vettori cambiano ma il tensore è lo stesso. Campo elettrico e campo magnetico sono le componenti di questo tensore. Cambiando sistema di riferimento facciamo una rotazione delle coordinate: cambiano la componente elettrica e magnetica. Ecco perché passando da un SI ad un altro compaiono i campi, sono solo due facce di un unica medaglia da vedere nel suo complesso. È interessante il fatto che l'elettromagnetismo ha portato alla crisi della fisica classica con lo sviluppo della teoria della relatività.




























































\chapter{Onde elettromagnetiche}

In generale si definisce come onda una qualsiasi perturbazione, impulsiva o periodica, rispetto ai valori di equilibrio di un campo che descrive una proprietà di un sistema fisico, la quale si propaghi a velocità ben determinata.

Ricordiamo che con campo intendiamo una grandezza fisica che può essere definita in ogni istante in ciascun punto dello spazio e si può trattare di una grandezza vettoriale o scalare. La temperatura, la pressione, la densità di un fluido, pensato come un mezzo continuo, sono esempi già incontrati di campi. Essi sono scalari perché basta una sola funzione per definirli completamente: il valore del campo in un punto $(x,y,z)$ in un certo istante $t$ è quindi un numero. Il campo elettrico e il campo magnetico nel vuoto o in mezzi materiali sono esempi di campi vettoriali, per definirli in un dato sistema cartesiano occorrono tre funzioni. Un'onda elastica in un oggetto (una sbarra, ad esempio) è una deformazione locale, rappresentabile con uno spostamento dalla posizione di equilibrio di una sezione della sbarra e con una variazione di pressione, che si propaga lungo la sbarra stessa. Un'onda sonora è una perturbazione dello stato di equilibrio della pressione e della densità che si propaga in un gas. Le onde elettromagnetiche sono la perturbazione di un campo elettrico o di un campo magnetico, prodotta da cariche in moto, che si propagano nello spazio circostante. Esse vengono descritte in forma vettoriale ma il campo elettromagnetico ha una struttura più complessa, di carattere tensoriale.

La perturbazione di un campo che, prodotta da una sorgente, si propaga nello spazio, viene rappresentata con la funzione $\xi(x,y,z,t)$, detta funzione d'onda. Una situazione particolare è costituita dalle cosiddette onde piane, descritte dalla funzione $\xi(x,t)$, spazialmente unidimensionale, cioè dipendenti dalla sola coordinata spaziale $x$ oltre che dal tempo. Il nome di onda piana deriva dal fatto che la perturbazione in un certo istante $t_0$ assume lo stesso valore $\xi(x_0,t_0)$ in tutti i punti del piano di equazione $x=x_0$, ortogonale all'asse di propagazione $x$.
\[
	f=(\vec{r},t) = f(x,t)
\]
Supporremo che la forma della perturbazione (profilo dell'onda) rimanga inalterata durante la propagazione.
Si verifica che sono possibili solo due forme generali di propagazione:
\begin{itemize}
	\item \textbf{onda progressiva}, descritta da una $f(x,t)$ data da:
	\[
		f_p(x,t)=\xi(x-vt)
	\]
	\item \textbf{onda regressiva}, descritta da $f(x,t)$ data da:
	\[
		f_r(x,t)=\xi(x+vt)
	\]
\end{itemize}
\begin{figure}[htpb]
	\centering
	

	\tikzset{every picture/.style={line width=0.75pt}} %set default line width to 0.75pt        

	\begin{tikzpicture}[x=0.75pt,y=0.75pt,yscale=-0.8,xscale=0.8]
	%uncomment if require: \path (0,348); %set diagram left start at 0, and has height of 348

	%Shape: Axis 2D [id:dp43615178825650913] 
	\draw  (66.5,242) -- (290,242)(80.73,77) -- (80.73,255.5) (283,237) -- (290,242) -- (283,247) (75.73,84) -- (80.73,77) -- (85.73,84)  ;
	%Curve Lines [id:da6912254946371756] 
	\draw    (89,242) .. controls (118.46,232.67) and (104.33,126.67) .. (128.77,126) .. controls (153.21,125.33) and (143.5,233.33) .. (166.33,242) ;
	%Curve Lines [id:da055272540593983566] 
	\draw    (189,242) .. controls (218.46,232.67) and (204.33,126.67) .. (228.77,126) .. controls (253.21,125.33) and (243.5,233.33) .. (266.33,242) ;
	%Straight Lines [id:da676398524434098] 
	\draw  [dash pattern={on 0.84pt off 2.51pt}]  (148.33,181.25) -- (148.33,242) ;
	%Straight Lines [id:da3877990784406371] 
	\draw  [dash pattern={on 0.84pt off 2.51pt}]  (248.33,181.25) -- (248.33,242) ;
	%Straight Lines [id:da3894091384014513] 
	\draw    (165,181) -- (195,181) ;
	\draw [shift={(198,181)}, rotate = 180] [fill={rgb, 255:red, 0; green, 0; blue, 0 }  ][line width=0.08]  [draw opacity=0] (10.72,-5.15) -- (0,0) -- (10.72,5.15) -- (7.12,0) -- cycle    ;
	%Shape: Axis 2D [id:dp9772982264856409] 
	\draw  (336.5,242) -- (560,242)(350.73,77) -- (350.73,255.5) (553,237) -- (560,242) -- (553,247) (345.73,84) -- (350.73,77) -- (355.73,84)  ;
	%Curve Lines [id:da9272160229851911] 
	\draw    (359,242) .. controls (388.46,232.67) and (374.33,126.67) .. (398.77,126) .. controls (423.21,125.33) and (413.5,233.33) .. (436.33,242) ;
	%Curve Lines [id:da8269961796909089] 
	\draw    (459,242) .. controls (488.46,232.67) and (474.33,126.67) .. (498.77,126) .. controls (523.21,125.33) and (513.5,233.33) .. (536.33,242) ;
	%Straight Lines [id:da5013690882799777] 
	\draw  [dash pattern={on 0.84pt off 2.51pt}]  (418.33,181.25) -- (418.33,242) ;
	%Straight Lines [id:da3290243185281636] 
	\draw  [dash pattern={on 0.84pt off 2.51pt}]  (518.33,181.25) -- (518.33,242) ;
	%Straight Lines [id:da27607590836085505] 
	\draw    (438,181) -- (468,181) ;
	\draw [shift={(435,181)}, rotate = 0] [fill={rgb, 255:red, 0; green, 0; blue, 0 }  ][line width=0.08]  [draw opacity=0] (10.72,-5.15) -- (0,0) -- (10.72,5.15) -- (7.12,0) -- cycle    ;

	% Text Node
	\draw (115.5,79) node    {$f_{p}( x,t)$};
	% Text Node
	\draw (200.5,79) node    {$t=t_{0}$};
	% Text Node
	\draw (230.17,113.17) node    {$\xi ( x_{0} ,t_{0})$};
	% Text Node
	\draw (149,252.5) node    {$x_{0}$};
	% Text Node
	\draw (249,252.5) node    {$x_{0} +\Delta x$};
	% Text Node
	\draw (301,239) node    {$x$};
	% Text Node
	\draw (385.5,79) node    {$f_{r}( x,t)$};
	% Text Node
	\draw (470.5,79) node    {$t=t_{0}$};
	% Text Node
	\draw (500.17,113.17) node    {$\xi ( x_{0} ,t_{0})$};
	% Text Node
	\draw (419,252.5) node    {$x_{0} +\Delta x$};
	% Text Node
	\draw (519,252.5) node    {$x_{0}$};
	% Text Node
	\draw (571,239) node    {$x$};
	% Text Node
	\draw (191,284) node   [align=left] {onda progressiva};
	% Text Node
	\draw (461,284) node   [align=left] {onda regressiva};


	\end{tikzpicture}
\end{figure}
\FloatBarrier
L'argomento di $\xi$ deve quindi contenere le variabili $x$ e $t$ sotto forma di combinazione lineare. Si tratta di una condizione sull'argomento. La funzione può poi assumere qualsiasi forma.

\textbf{Caso dell'onda progressiva}
Attendiamo un certo $ \Delta t $ e facciamo propagare l'onda. Dobbiamo vedere qual è il legame fra $ \Delta x $ e $ \Delta t $. Sfruttiamo il fatto che il valore assunto dalla funzione all'istante $t_0 $ nella posizione $x_0$ si ritrova in qualsiasi istante successivo che soddisfi la condizione:
\begin{gather*}
	f_p(x_0,t_0) = f_p(x_0+\Delta x,t_0+\Delta t) \\
	\xi(x_0-v\,t_0) = \xi(x_0+\Delta x-v\,t_0-v\Delta t) \\
	\Delta x - v\Delta t = 0 \\
	v = \frac{\Delta x}{\Delta t}
\end{gather*}
Si ottiene tale relazione che indica un moto rettilineo lungo l'asse $x$ con velocità $v$.

\textbf{Caso dell'onda regressiva}
Possiamo fare un discorso analogo per le onde regressive e scrivere:
\begin{gather*}
	f_r(x_0,t_0) = f_r(x_0+\Delta x,t_0+\Delta t) \\
	\xi(x_0+v\,t_0) = \xi(x_0+\Delta x + v\,t_0 + v\Delta t) \\
	\Delta x + v\Delta t = 0 \\
	v = - \frac{\Delta x}{\Delta t}
\end{gather*}
Analogamente ritroviamo che la funzione rappresenta una grandezza che si propaga con velocità $v$ lungo il verso negativo dell'asse $x$.
In entrambi i casi si tratta di una traslazione rigida, nel senso che la funzione non cambia mai forma.
Vogliamo trovare un equazione delle onde che descriva qualunque fenomeno ondulatorio. Una qualsiasi equazione ondulatoria deve obbedire all'\textbf{equazione delle onde} o \textbf{di d'Alembert} che nel caso di onda piana assume la seguente forma semplificata:
\[
	\frac{\partial^2 \xi}{\partial x^2}  - \frac{1}{v^2} \frac{\partial^2 \xi}{\partial t^2} = 0
\]
è semplice verificare che $ \xi(x\pm vt) $ è soluzione dell'equazione di d'Alembert. A tal proposito, occorre introdurre la variabile ausiliaria $ \eta = x \pm vt $
\begin{gather*}
	\frac{\partial \xi}{\partial x} = \frac{\partial \xi}{\partial \eta} \cdot \frac{\partial \eta}{\partial \xi} = \frac{\partial \xi}{\partial \eta}\\
	\frac{\partial^2 \xi}{\partial x^2} = \frac{\partial}{\partial x} \left[ \frac{\partial \xi}{\partial \eta}  \right] = \frac{\partial}{\partial \eta} \left[ \frac{\partial \xi}{\partial \eta}  \right] \cdot \frac{\partial \eta}{\partial x} = \frac{\partial^2 \xi}{\partial \eta^2}
\end{gather*}
E poi
\begin{gather*}
	\frac{\partial \xi}{\partial t} = \frac{\partial \xi}{\partial \eta} \cdot \frac{\partial \eta}{\partial t} = \frac{\partial \xi}{\partial \eta} (\pm v)\\
	\frac{\partial^2 \xi}{\partial t^2} = \frac{\partial}{\partial t} \left[ (\pm v)\frac{\partial \xi}{\partial \eta}  \right] = \frac{\partial}{\partial \eta} \left[ (\pm )\frac{\partial \xi}{\partial \eta}  \right] \cdot \frac{\partial \eta}{\partial t} = \frac{\partial^2 \xi}{\partial \eta^2} (\pm v)^2
\end{gather*}
Da cui
\[
	\frac{\partial^2 \xi}{\partial t^2} = v^2 \frac{\partial^2 \xi}{\partial x^2} \implies \boxed{\frac{\partial^2 \xi}{\partial x^2} -\frac{1}{v^2}\frac{\partial^2 \xi}{\partial t^2} = 0}
\]







































\section{Onde piane armoniche (o sinusoidali)}

Un tipo particolare di onda piana è l'onda sinusoidale, o armonica. Le onde sinusoidali sono quelle per cui la funzione è un seno o un coseno; vengono percepite dall'occhio umano con un colore ben preciso e prendono per questo il nome di onde monocromatiche. La loro funzione d'onda si scrive come
\[
	\xi(x,t) = \xi_0 \sin [k(x-vt)] \qquad \xi(x,t) = \xi_0 \cos [k(x-vt)]
\]
$k$ prende il nome di \textbf{numero d'onda} e viene inserita per ragioni dimensionali in quanto l'argomento di un seno o di un coseno deve essere espresso in radianti. Possiamo riscrivere l'equazione come:
\[
	\xi(x,t) = \xi_0 \sin [kx - \underbrace{kv}_{\omega} t] = \xi_0 \sin [kx - \omega t]
\]
$kv$ viene chiamato pulsazione dell'onda, $\omega$. Se fissiamo un certo istante temporale $t_0 $ e facciamo un grafico dell'andamento della funzione, vedremo un andamento sinusoidale. La distanza lungo l'asse $x$ fra due punti che assumono lo stesso valore prende il nome di \textbf{lunghezza d'onda} $\lambda$ e ci dà la sua periodicità spaziale. Possiamo legare $\lambda$ con $k$ con la relazione:
\[
	\lambda = \frac{2\pi}{k}
\]
Si deduce che $k$ è uguale al numero di lunghezze d'onda che stanno su una distanza pari a $ 2\pi  $ metri. Se invece fissiamo una certa posizione, la $ \xi(x_0,t) $ dà la variazione nel tempo della funzione d'onda. Parleremo di \textbf{periodo} invece che di lunghezza d'onda. Si tratta dell'intervallo di tempo che separa due istanti successivi in cui la funzione assume lo stesso valore.
\[
	T = \frac{2\pi}{\omega}
\]
Definiamo anche la \textbf{frequenza} come l'inverso del periodo: $ \nu = \frac{1}{T} $. Si trova quindi che i due periodi, spaziale e temporale, sono legati dalla relazione
\[
	\nu = \frac{\omega}{2\pi} = \frac{kv}{2\pi} = \frac{v}{\lambda} \implies v = \lambda\nu \quad \lambda = vT
\]
L'onda in un periodo $T$ alla velocità $v$ percorre una distanza pari alla lunghezza d'onda.
Nel seguito dovremo spesso considerare onde armoniche del tipo:
\[
	\xi = \xi_0 \sin [\underbrace{kx-\omega t + \varphi}_{\text{fase dell'onda}}]
\]
Tutto l'argomento del seno è detto fase dell'onda sinusoidale. La fase ci consente di trovare un modo per visualizzare com'è fatta un onda. Infatti, fissato l'istante $t$, la superficie in cui la fase è costante prende il nome di \textbf{fronte d'onda} o fronte di fase.
\[
	\Phi (x_0, t_0) = \text{costante}
\]
Per un'onda piana il fronte d'onda è un piano o quanto meno una porzione di piano. Esso si sposta con la velocità $v$ di propagazione dell'onda.
\begin{figure}[htpb]
	\centering
	

	\tikzset{every picture/.style={line width=0.75pt}} %set default line width to 0.75pt        

	\begin{tikzpicture}[x=0.75pt,y=0.75pt,yscale=-1,xscale=1]
	%uncomment if require: \path (0,300); %set diagram left start at 0, and has height of 300

	%Straight Lines [id:da4035621045003759] 
	\draw    (276.9,162.92) -- (371.44,220.93) ;
	\draw [shift={(374,222.5)}, rotate = 211.53] [fill={rgb, 255:red, 0; green, 0; blue, 0 }  ][line width=0.08]  [draw opacity=0] (10.72,-5.15) -- (0,0) -- (10.72,5.15) -- (7.12,0) -- cycle    ;
	%Straight Lines [id:da4284318663584288] 
	\draw    (276.9,162.92) -- (276.9,15.5) ;
	\draw [shift={(276.9,12.5)}, rotate = 450] [fill={rgb, 255:red, 0; green, 0; blue, 0 }  ][line width=0.08]  [draw opacity=0] (10.72,-5.15) -- (0,0) -- (10.72,5.15) -- (7.12,0) -- cycle    ;
	%Straight Lines [id:da09535241990565257] 
	\draw    (276.9,162.92) -- (182.36,220.93) ;
	\draw [shift={(179.8,222.5)}, rotate = 328.47] [fill={rgb, 255:red, 0; green, 0; blue, 0 }  ][line width=0.08]  [draw opacity=0] (10.72,-5.15) -- (0,0) -- (10.72,5.15) -- (7.12,0) -- cycle    ;
	%Shape: Rectangle [id:dp0706669936577331] 
	\draw  [fill={rgb, 255:red, 222; green, 222; blue, 222 }  ,fill opacity=1 ] (305.9,46.89) -- (305.9,190.92) -- (227.15,236.39) -- (227.15,92.36) -- cycle ;
	%Shape: Rectangle [id:dp28618755049087685] 
	\draw  [fill={rgb, 255:red, 222; green, 222; blue, 222 }  ,fill opacity=1 ] (315.9,56.89) -- (315.9,200.92) -- (237.15,246.39) -- (237.15,102.36) -- cycle ;
	%Shape: Rectangle [id:dp8603680037835846] 
	\draw  [fill={rgb, 255:red, 222; green, 222; blue, 222 }  ,fill opacity=1 ] (325.9,66.89) -- (325.9,210.92) -- (247.15,256.39) -- (247.15,112.36) -- cycle ;
	%Straight Lines [id:da3406983498207887] 
	\draw    (217.15,246.39) -- (227.15,256.39) ;
	\draw [shift={(227.15,256.39)}, rotate = 225] [color={rgb, 255:red, 0; green, 0; blue, 0 }  ][line width=0.75]    (0,5.59) -- (0,-5.59)   ;
	\draw [shift={(217.15,246.39)}, rotate = 225] [color={rgb, 255:red, 0; green, 0; blue, 0 }  ][line width=0.75]    (0,5.59) -- (0,-5.59)   ;

	% Text Node
	\draw (172.87,226.84) node    {$x$};
	% Text Node
	\draw (385.47,221.75) node    {$y$};
	% Text Node
	\draw (264.29,12.16) node    {$z$};
	% Text Node
	\draw (401,124) node   [align=left] {fronti d'onda};
	% Text Node
	\draw (209.37,259.34) node    {$\lambda $};


	\end{tikzpicture}
\end{figure}
\FloatBarrier
Può infine convenire descrivere le funzioni d'onda come le formule di Eulero.
\begin{equation*}
	\begin{aligned}
		\xi(x,t) &= \xi_0\cos [kx-\omega t+\varphi] = \xi_0\cos [\Phi (x,t)] \\
		&= \frac{\xi_0}{2} [e^{i\Phi (x,t)} + e^{-i\Phi (x,t)}] \\
		&= \frac{\xi_0}{2} [e^{i\Phi (x,t)} + \text{complesso congiugato} ]
	\end{aligned}
\end{equation*}







































\section{Onde longitudinali, onde trasversali, polarizzazione}

Un'onda piana è caratterizzata da un'unica direzione di propagazione, che per semplicità abbiamo indicato con l'asse $x$. Se tutte le grandezze significative relative
alla perturbazione che si propaga hanno direzione di
variazione che coincide con l'asse $x$, l'onda si dice longitudinale. L'onda acustica ne è un esempio. La pressione infatti non cambia in direzione perpendicolare a quella dell'onda ma nella direzione stessa di propagazione.
Invece in una corda tesa la direzione di propagazione è quella della corda ma il campo varia in direzione ortogonale ad essa. Quando la direzione di variazione dl campo è ortogonale alla direzione di variazione dell'onda, parleremo di onda trasversale. In questo caso non possiamo più rappresentare l'onda con una funzione scalare perché dobbiamo specificare la direzione in cui varia il campo. Dunque dobbiamo proiettare il vettore nel piano ortogonale alla direzione di propagazione. In una qualunque onda trasversale, fissati arbitrariamente gli assi $y$ e $z$ in un piano ortogonale all'asse $x$, la funzione d'onda è rappresentabile come un vettore $\xi$ le cui componenti sono in direzione $y$ e $z$. Visto che c'è questo ulteriore grado di libertà si possono verificare diverse situazioni. La direzione di variazione, ad esempio, può essere una qualsiasi delle infinite perpendicolari alla direzione di propagazione.
Potrebbe accadere che il vettore $\xi$ cambi casualmente nel tempo. Parleremo di \textbf{onda trasversale non polarizzata}. Se esiste una legge che ci dice come varia il versore nel tempo, parleremo di \textbf{onda polarizzata}. Le onde armoniche sono sempre polarizzate.

\paragraph{Onde piane sinusoidali trasversali.} Esse si rappresentano come:
\begin{gather*}
	\vec{\xi} (x,t) = \xi_z (x,t) \vec{u}_z + \xi_y (x,t) \vec{u}_y \\
	\xi_z = \xi_{0z}\sin (kx-\omega t) \qquad \xi_y = \xi_{0y}\sin (kx-\omega t + \delta )
\end{gather*}
In cui $\delta$ rappresenta la differenza di fase tra le due onde componenti. Le ampiezze $\xi_{0z}$, $\xi_{0y}$ insieme a $\delta$ permettono di conoscere ovunque e in ogni istante la funzione d'onda $\xi$. Esaminiamo alcuni casi in cui la differenza di fase ha valore costante.

Se $ \delta = 0 $ in ogni punto dell'asse $x$ e in ogni istante il vettore d'onda $\xi$ ha direzione fissa, formante con l'asse $y$ l'angolo $ \vartheta  $. Se $ \delta = \pi $ (componenti dell'onda in opposizione di fase), la situazione è la stessa, solo che la direzione di $\xi$ è opposta.
\begin{figure}[htpb]
	\centering
	

	\tikzset{every picture/.style={line width=0.75pt}} %set default line width to 0.75pt        

	\begin{tikzpicture}[x=0.75pt,y=0.75pt,yscale=-1,xscale=1]
	%uncomment if require: \path (0,300); %set diagram left start at 0, and has height of 300

	%Shape: Axis 2D [id:dp1578987810426533] 
	\draw  (398.5,153) -- (208.5,153)(303.71,67) -- (303.71,241) (215.5,148) -- (208.5,153) -- (215.5,158) (308.71,74) -- (303.71,67) -- (298.71,74)  ;
	%Straight Lines [id:da9166949085494358] 
	\draw    (236.71,86) -- (376.71,226) ;
	%Straight Lines [id:da1792511694869101] 
	\draw    (260.83,110.12) -- (376.71,226) ;
	\draw [shift={(258.71,108)}, rotate = 45] [fill={rgb, 255:red, 0; green, 0; blue, 0 }  ][line width=0.08]  [draw opacity=0] (10.72,-5.15) -- (0,0) -- (10.72,5.15) -- (7.12,0) -- cycle    ;

	% Text Node
	\draw (227,117) node    {$\vec{\xi }( x_{0} ,t)$};
	% Text Node
	\draw (320,68) node    {$y$};
	% Text Node
	\draw (203,162) node    {$z$};
	% Text Node
	\draw (372,108) node    {$x=x_{0}$};


	\end{tikzpicture}
\end{figure}
\FloatBarrier
Quando il vettore $\xi$ ha direzione fissa si dice che l'onda piana è \textbf{polarizzata linearmente} o rettilinearmente. La direzione fissa di $\xi$ è chiamata \textbf{direzione di polarizzazione} e il piano fisso su cui essa giace è detto \textbf{piano di polarizzazione}. Poniamo ora $ \delta =\frac{\pi}{2} $. Si avrà:
\[
	\xi_z = \xi_{0z} \cos (kx-\omega t) \qquad \xi_y = \xi_{0y}\sin (kx-\omega t)
\]
Si tratta di due sinusoidi in quadratura di fase. Se ci poniamo in un punto $x=x_0$, siccome le due componenti oscillano in modo sfasato, il vettore avrà andamento ellittico in funzione del tempo. Le componenti dell'onda infatti soddisfano l'equazione:
\[
	\frac{\xi_y^2}{\xi_{0y}^2} + \frac{\xi_z^2}{\xi_{0z}^2} = 1
\]
Al passare del tempo si vede la punta di $\xi$ descrivere un'ellisse.
Per $ \frac{3}{2} \pi  $ cambia il verso in cui l'elica gira, da orario a antiorario. Si parla di onda \textbf{polarizzata ellitticamente}.
\begin{figure}[htpb]
	\centering
	

	\tikzset{every picture/.style={line width=0.75pt}} %set default line width to 0.75pt        

	\begin{tikzpicture}[x=0.75pt,y=0.75pt,yscale=-1,xscale=1]
	%uncomment if require: \path (0,300); %set diagram left start at 0, and has height of 300

	%Shape: Axis 2D [id:dp4786260302496421] 
	\draw  (435,172.83) -- (212,172.83)(323.74,71.89) -- (323.74,276.11) (219,167.83) -- (212,172.83) -- (219,177.83) (328.74,78.89) -- (323.74,71.89) -- (318.74,78.89)  ;
	%Straight Lines [id:da6430509685027268] 
	\draw    (298.62,134.02) -- (323.71,173) ;
	\draw [shift={(297,131.5)}, rotate = 57.24] [fill={rgb, 255:red, 0; green, 0; blue, 0 }  ][line width=0.08]  [draw opacity=0] (10.72,-5.15) -- (0,0) -- (10.72,5.15) -- (7.12,0) -- cycle    ;
	%Shape: Ellipse [id:dp03331606762220707] 
	\draw   (285.41,173) .. controls (285.41,140.69) and (302.56,114.5) .. (323.71,114.5) .. controls (344.86,114.5) and (362,140.69) .. (362,173) .. controls (362,205.31) and (344.86,231.5) .. (323.71,231.5) .. controls (302.56,231.5) and (285.41,205.31) .. (285.41,173) -- cycle ;

	% Text Node
	\draw (269,118) node    {$\vec{\xi }( x_{0} ,t)$};
	% Text Node
	\draw (338.67,67.33) node    {$y$};
	% Text Node
	\draw (223,182) node    {$z$};
	% Text Node
	\draw (443.67,111) node    {$x=x_{0}$};
	% Text Node
	\draw (340.33,102) node    {$\xi _{oy}$};
	% Text Node
	\draw (343.33,240) node    {$-\xi _{oy}$};
	% Text Node
	\draw (273,183.33) node    {$\xi _{oz}$};


	\end{tikzpicture}
\end{figure}
\FloatBarrier
Se le due ampiezze sono uguali l'ellisse diventa un cerchio. Parleremo di \textbf{polarizzazione circolare}, caso particolare della polarizzazione ellittica
\[
	\xi_{0y} = \xi_{0z}
\]
Se $\delta$ non è uno dei quattro angoli citati ($0$, $ \pi  $, $ \frac{\pi}{2} $, $ \frac{3\pi}{2} $) si parla di \textbf{onda polarizzata ellitticamente ma con assi ruotati} rispetto a $y$ e $z$.







































\section{Onde in più dimensioni}

Vediamo quale sarebbe la forma più generale, se la direzione di propagazione fosse generica. Chiameremo $x'$ l'asse in cui l'onda propaga. Vogliamo esprimere la funzione che rappresenta l'andamento sinusoidale in un sistema di coordinate differente. Sia $\vec{r}$ il raggio-vettore che individua un punto $P$ di un certo fronte d'onda piano.
\begin{figure}[htpb]
	\centering
	

	\tikzset{every picture/.style={line width=0.75pt}} %set default line width to 0.75pt        

	\begin{tikzpicture}[x=0.75pt,y=0.75pt,yscale=-1,xscale=1]
	%uncomment if require: \path (0,300); %set diagram left start at 0, and has height of 300

	%Straight Lines [id:da9506068607557134] 
	\draw    (296.9,182.92) -- (441.05,209.95) ;
	\draw [shift={(444,210.5)}, rotate = 190.62] [fill={rgb, 255:red, 0; green, 0; blue, 0 }  ][line width=0.08]  [draw opacity=0] (10.72,-5.15) -- (0,0) -- (10.72,5.15) -- (7.12,0) -- cycle    ;
	%Straight Lines [id:da8959417063927446] 
	\draw    (296.9,182.92) -- (391.44,240.93) ;
	\draw [shift={(394,242.5)}, rotate = 211.53] [fill={rgb, 255:red, 0; green, 0; blue, 0 }  ][line width=0.08]  [draw opacity=0] (10.72,-5.15) -- (0,0) -- (10.72,5.15) -- (7.12,0) -- cycle    ;
	%Straight Lines [id:da9787666861777233] 
	\draw    (296.9,182.92) -- (296.9,35.5) ;
	\draw [shift={(296.9,32.5)}, rotate = 450] [fill={rgb, 255:red, 0; green, 0; blue, 0 }  ][line width=0.08]  [draw opacity=0] (10.72,-5.15) -- (0,0) -- (10.72,5.15) -- (7.12,0) -- cycle    ;
	%Straight Lines [id:da969195712272505] 
	\draw    (296.9,182.92) -- (202.36,240.93) ;
	\draw [shift={(199.8,242.5)}, rotate = 328.47] [fill={rgb, 255:red, 0; green, 0; blue, 0 }  ][line width=0.08]  [draw opacity=0] (10.72,-5.15) -- (0,0) -- (10.72,5.15) -- (7.12,0) -- cycle    ;
	%Shape: Rectangle [id:dp43868189433161064] 
	\draw  [fill={rgb, 255:red, 222; green, 222; blue, 222 }  ,fill opacity=1 ] (303.85,71.45) -- (355.21,206.01) -- (305.2,295.83) -- (253.84,161.27) -- cycle ;
	%Straight Lines [id:da4888413142571806] 
	\draw    (304.53,183.64) -- (389.54,152.7) ;
	\draw [shift={(392.36,151.67)}, rotate = 520] [fill={rgb, 255:red, 0; green, 0; blue, 0 }  ][line width=0.08]  [draw opacity=0] (10.72,-5.15) -- (0,0) -- (10.72,5.15) -- (7.12,0) -- cycle    ;
	%Straight Lines [id:da6230223655892035] 
	\draw    (342.61,143.5) -- (369.54,133.7) ;
	\draw [shift={(372.36,132.67)}, rotate = 520] [fill={rgb, 255:red, 0; green, 0; blue, 0 }  ][line width=0.08]  [draw opacity=0] (10.72,-5.15) -- (0,0) -- (10.72,5.15) -- (7.12,0) -- cycle    ;

	% Text Node
	\draw (192.87,246.84) node    {$x$};
	% Text Node
	\draw (405.47,241.75) node    {$y$};
	% Text Node
	\draw (284.29,32.16) node    {$z$};
	% Text Node
	\draw (402.87,147.84) node    {$x'$};
	% Text Node
	\draw (388.87,123.84) node    {$\vec{u}_{x'}$};
	% Text Node
	\draw (456.47,209.75) node    {$P$};


	\end{tikzpicture}
\end{figure}
\FloatBarrier
\begin{gather*}
	\xi(x',t) = \xi_0 \sin [kx'-\omega t] \\
	\vec{r} = x'\vec{u}_{x'}+y'\vec{u}_{y'}+z'\vec{u}_{z'} = x\vec{u}_x+y\vec{u}_y+z\vec{u}_z \\
\end{gather*}
Possiamo ottenere $ x' $ in funzione delle coordinate $ (x,y,z)  $ imponendo
\[
	x'=\vec{r} \cdot \vec{u}_{x'} = (x\vec{u}_x+y\vec{u}_y+z\vec{u}_z) \cdot  \vec{u}_{x'} = x\cos \alpha + y\cos \beta + z \cos \gamma
\]
Se moltiplichiamo per $k$ otteniamo l'espressione che sta nell'argomento del seno.
\[
	kx'=k\;\vec{r} \cdot \vec{u}_{x'} = \underbrace{k\,\vec{u}_{x'}}_{\vec{k}} \cdot \,\vec{r} = \vec{k} \cdot \vec{r}
\]
Abbiamo ottenuto un vettore che contiene due informazioni fondamentali riguardo l'onda:
\begin{itemize}
	\item Il numero d'onda
	\item Il verso di propagazione
\end{itemize}
Si tratta del vettore $ \vec{k}  $ che prende il nome di \textbf{vettore d'onda}. Possiamo esprimere in modo più generale un onda piana quando si propaga in una direzione generica dello spazio come:
\[
	\xi(x,y,z,t) = \xi_0\sin [\vec{k} \cdot \vec{r} -\omega t]
\]
L'invarianza del prodotto scalare assicura che questa espressione generale di onda armonica piana è indipendente dal sistema di coordinate prescelto per la sua descrizione analitica. Si dimostra che in questo caso l'onda di propagazione obbedisce alla seguente equazione di d'Alembert generale:
\[
	\frac{\partial^2 \xi}{\partial x^2} + \frac{\partial^2 \xi}{\partial y^2} + \frac{\partial^2 \xi}{\partial z^2} = \frac{1}{v^2} \frac{\partial^2 \xi}{\partial t^2} \implies \nabla^2 \xi = \frac{1}{v^2} \frac{\partial^2 \xi}{\partial t^2}
\]







































\section{Onde elettromagnetiche}

Abbiamo visto che, nel caso di onde piane sinusoidali, l'espressione di $\xi$ è darà da:
\[
	\xi(\vec{r},t) = \xi_0\sin [\vec{k} \cdot \vec{r} -\omega t]
\]
Vogliamo adesso dimostrare come nelle equazioni di Maxwell siano in effetti contenuti fenomeni ondulatori e definire le onde elettromagnetiche nel più semplice caso di onda piana.

Consideriamo di trovarci in uno spazio vuoto nel quale non ci sono cariche libere e correnti di
conduzione. In queste ipotesi le equazioni di Maxwell hanno la forma
\begin{align}
	&\text{div}\vec{E} = 0 \\
	&\text{div}\vec{B} = 0 \\
	&\text{rot}\vec{E} = \frac{\partial \vec{B}}{\partial t} \\
	&\text{rot}\vec{B} = \mu_0 \varepsilon_0 \frac{\partial \vec{E}}{\partial t}
\end{align}
Proviamo ad applicare il rotore ad entrambe le parti della terza equazione
\begin{equation}
	\begin{aligned}
		\vec{\nabla} \times \left( \vec{\nabla} \times \vec{E} \right)   &= - \vec{\nabla} \times \left( \frac{\partial \vec{B}}{\partial t}  \right) \\
		\vec{\nabla} \underbrace{\left( \vec{\nabla} \cdot \vec{E} \right)}_{=0} - \nabla^2 \vec{E} &= - \frac{\partial}{\partial t} \underbrace{\left[\vec{\nabla} \times \vec{B} \right]}_{= \mu_0 \varepsilon_0 \frac{\partial \vec{E}}{\partial t}} \\
		\Aboxed{\nabla^2 \vec{E} &= \mu_0 \varepsilon_0 \frac{\partial^2 \vec{E}}{\partial t^2}}
	\end{aligned}
	\label{eq:maxwellElettrico}
\end{equation}
Analogamente, moltiplicando per il rotore entrambe le parti della quarta equazione:
\begin{equation}
	\begin{aligned}
		\vec{\nabla} \times \left( \vec{\nabla} \times \vec{B} \right)   &= \vec{\nabla} \times \left( \mu_0 \varepsilon_0 \frac{\partial \vec{E}}{\partial t}  \right) \\
		\vec{\nabla} \underbrace{\left( \vec{\nabla} \cdot \vec{B} \right)}_{=0} - \nabla^2 \vec{B} &= \mu_0 \varepsilon_0 \frac{\partial}{\partial t} \underbrace{\left[\vec{\nabla} \times \vec{E} \right]}_{= - \frac{\partial \vec{B}}{\partial t}} \\
		\Aboxed{\nabla^2 \vec{B} &= \mu_0 \varepsilon_0 \frac{\partial^2 \vec{B}}{\partial t^2}}
	\end{aligned}
	\label{eq:maxwellMagnetico}
\end{equation}
Le equazioni di Maxwell prevedono che i campi elettrici e magnetici possano propagarsi come delle onde; i risultati ottenuti infatti hanno forma analoga all'equazione di d'Alembert. Il risultato all'epoca risultò in contraddizione con le teorie esistenti perché prevedeva che le onde potessero propagarsi anche nel vuoto. Si pensava invece che la propagazione di esse dovesse sempre avvenire attraverso un mezzo, tant'è che furono condotti numerosi studi al fine di provare l'esistenza dell'\emph{etere luminifero}, ipotetico mezzo di trasmissione delle onde elettromagnetiche. Ne è un esempio l'esperimento (senza successo) di rilevare il movimento della terra attraverso l'etere di Michelson-Morley. È possibile calcolare la velocità di propagazione dell'onda come:
\[
	\frac{1}{c^2} = \mu_0 \varepsilon_0 \implies c = \frac{1}{\sqrt{\mu_0 \varepsilon_0}} \simeq 2,9979\ldots  \times 10^8 \,m/s
\]
Maxwell trovò questo valore della velocità $c$ (dal latino \emph{celeritas}) di propagazione delle onde elettromagnetiche, numero che era già stato individuato dall'apparato di Fizeau-Foucault (1849), un dispositivo per il calcolo della velocità della luce. Fu poi in seguito, grazie alle esperienze di Hertz, che fu definitivamente dimostrata la natura elettromagnetica dei fenomeni luminosi.
Sembrerebbe che fra i due campi non ci sia relazione, che possano esistere onde magnetiche o elettriche. Tuttavia se osserviamo le equazioni di Maxwell elencate all'inizio ci rendiamo conto di come la presenza di uno comporti la presenza dell'altro. Tale apparenza si ha perché nel momento in cui calcoliamo la derivata di una funzione (applicando il rotore) perdiamo informazioni sulle costanti.\\
\textbf{Osservazione}. Sostanzialmente le equazioni \eqref{eq:maxwellElettrico} e \eqref{eq:maxwellMagnetico} ci danno soluzioni per le onde elettromagnetiche, ma può capitare che alcune di esse non siano soluzioni delle equazioni di Maxwell, che vadano scartate. Abbiamo infatti perso informazioni sulla solenoidaleità dei campi.

In questo contesto si introduce l'\textbf{operatore d'alembertiano}, che viene indicato con un quadrato. Così come per il rotore viene utilizzato un triangolo, riferimento alle tre derivate spaziali, per tale operatore si hanno quattro lati che corrispondono alle coordinate $(x,y,z,t)$. Si ha una quarta coordinata spaziale, quella del tempo. Nella teoria della relatività si lavora con quattro dimensioni.
\[
	\boxed{\Box = \nabla^2 - \frac{1}{c^2} \frac{\partial^2}{\partial t^2}  \implies  \Box \vec{E} = 0 \quad \Box \vec{B} = 0}
\]
\textbf{Osservazione.} Se $\xi_1$ e $\xi_2$ sono soluzioni indipendenti dell'equazione di d'Alembert, per la sua linearità, anche la loro combinazione lineare lo è.
\[
	\xi = \xi_1a_1 + \xi_2a_2
\]
Supponiamo ora di avere un'onda piana che si propaga lungo l'asse $x$.
\[
	\vec{E} = E(x,t) \qquad \vec{B} = B(x,t)
\]
Questo comporta che la derivata rispetto a $y$ e $z$ di qualsiasi componente di questi vettori $\vec{B}$ ed $\vec{E}$ debba essere nulla. Applichiamo questa osservazione alle prime due equazioni di Maxwell:
\begin{equation*}
	\begin{aligned}
		\text{div}\vec{E} &= \frac{\partial E_x}{\partial x} + \frac{\partial E_y}{\partial y} + \frac{\partial E_z}{\partial z}  = 0 \implies \frac{\partial E_x}{\partial x} = 0 \\
		\text{div}\vec{B} &= \frac{\partial B_x}{\partial x} + \frac{\partial B_y}{\partial y} + \frac{\partial B_z}{\partial z}  = 0 \implies \frac{\partial B_x}{\partial x} = 0
	\end{aligned}
\end{equation*}
Consideriamo il rotore
\begin{equation*}
	\begin{aligned}
		\text{rot}\vec{E} &= \text{det} \begin{pmatrix}
		\vec{u}_x & \vec{u}_y & \vec{u}_z \\
		\frac{\partial}{\partial x}  &  \frac{\partial}{\partial y} & \frac{\partial}{\partial z} \\
		E_x & E_y & E_z
	\end{pmatrix} \\
	&= \vec{u}_x\left( \underbrace{\frac{\partial E_z}{\partial y}}_{=0} - \underbrace{\frac{\partial E_y}{\partial z}}_{=0}  \right) + \vec{u}_y \left( \underbrace{\frac{\partial E_x}{\partial z}}_{=0} - \frac{\partial E_z}{\partial x}  \right) + \vec{u}_z \left( \underbrace{\frac{\partial E_x}{\partial y}}_{=0} - \frac{\partial E_y}{\partial x}  \right)  \\
	&=-\frac{\partial \vec{B}}{\partial t} = - \vec{u}_x\frac{\partial B_x}{\partial t} - \vec{u}_y\frac{\partial B_y}{\partial t} - \vec{u}_z\frac{\partial B_z}{\partial t}
	\end{aligned}
\end{equation*}
Si ottiene
\begin{equation}
	\boxed{-\frac{\partial B_x}{\partial t} = 0 \qquad - \frac{\partial E_z}{\partial x} = - \frac{\partial B_y}{\partial t} \qquad \frac{\partial E_y}{\partial x} = - \frac{\partial B_z}{\partial t}}
	\label{eq:componenti1}
\end{equation}
Analogamente per il rotore di $\vec{B}$
\begin{equation*}
	\begin{aligned}
		\text{rot}\vec{B} &= \text{det} \begin{pmatrix}
		\vec{u}_x & \vec{u}_y & \vec{u}_z \\
		\frac{\partial}{\partial x}  &  \frac{\partial}{\partial y} & \frac{\partial}{\partial z} \\
		B_x & B_y & B_z
	\end{pmatrix} \\
	&= \vec{u}_x\left( \underbrace{\frac{\partial B_z}{\partial y}}_{=0} - \underbrace{\frac{\partial B_y}{\partial z}}_{=0}  \right) + \vec{u}_y \left( \underbrace{\frac{\partial B_x}{\partial z}}_{=0} - \frac{\partial B_z}{\partial x}  \right) + \vec{u}_z \left( \underbrace{\frac{\partial B_x}{\partial y}}_{=0} - \frac{\partial B_y}{\partial x}  \right)  \\
	&= \frac{1}{c^2}\frac{\partial \vec{E}}{\partial t} = \frac{1}{c^2} \vec{u}_x\frac{\partial E_x}{\partial t} + \frac{1}{c^2} \vec{u}_y\frac{\partial E_y}{\partial t} + \frac{1}{c^2} \vec{u}_z\frac{\partial E_z}{\partial t}
	\end{aligned}
\end{equation*}
Si ottiene
\begin{equation}
	\boxed{\frac{1}{c^2} \frac{\partial E_x}{\partial t} = 0 \qquad \frac{1}{c^2}\frac{\partial E_y}{\partial t} = - \frac{\partial B_z}{\partial x} \qquad \frac{1}{c^2} \frac{\partial E_z}{\partial t} = \frac{\partial B_y}{\partial x}}
	\label{eq:componenti2}
\end{equation}
Diamo una interpretazione ai risultati ottenuti.
\[
	\left. \begin{array}{r}
	 	\frac{\partial E_x}{\partial t} = 0 \\
		\frac{\partial E_x}{\partial t} = 0
	\end{array} \right\} \implies E_x = \text{costante} = 0
\]
\[
	\left. \begin{array}{r}
	 	\frac{\partial B_x}{\partial t} = 0 \\
		\frac{\partial B_x}{\partial t} = 0
	\end{array} \right\} \implies B_x = \text{costante} = 0
\]
Tali relazioni permettono di dire che le componenti $ E_x $ e $ B_x $ sono costanti. Un tale caso potrebbe essere prodotto da distribuzioni di cariche stazionarie (per $ \vec{E}  $) o da correnti stazionarie (per $ \vec{B}  $). Siccome escludiamo l'esistenza di sorgenti di questo tipo in quanto interessati ai campi variabili nel tempo, concludiamo che tali componenti sono nulle. Se la componente dei due campi lungo l'asse di propagazione è nullo significa che le onde elettromagnetiche sono trasversali. Osservando le altre 4 equazioni, si osserva un accoppiamento fra le componenti dei due campi fra loro ortogonali. Ci possiamo concentrare su una delle due modalità di oscillazione.

Ipotizziamo $ E_z = 0 $
\[
	\frac{\partial B_y}{\partial t} = 0 \quad \frac{\partial B_y}{\partial x} = 0 \implies B_y = \text{costante} = 0
\]
Derivando da entrambi i lati l'equazione ricavata precedentemente si ha:
\begin{equation*}
	\begin{aligned}
		\frac{\partial E_y}{\partial x} &= - \frac{\partial B_z}{\partial t} \\
		\frac{\partial}{\partial x} \left( \frac{\partial E_y}{\partial x}  \right) &= - \frac{\partial}{\partial x} \left( \frac{\partial B_z}{\partial t}  \right) \\
		\frac{\partial^2 E_y}{\partial x^2} &= - \frac{\partial^2 B_z}{\partial x \partial t}
	\end{aligned}
\end{equation*}
Poi
\begin{equation*}
	\begin{aligned}
		\frac{1}{c^2} \frac{\partial E_y}{\partial t} &= - \frac{\partial B_z}{\partial x} \\
		\frac{\partial}{\partial t} \left( \frac{1}{c^2} \frac{\partial E_y}{\partial t} \right) &= - \frac{\partial}{\partial t} \left( \frac{\partial B_z}{\partial x} \right) \\
		\frac{1}{c^2} \frac{\partial^2 E_y}{\partial t^2} &= - \frac{\partial^2 B_z}{\partial t\partial x}
	\end{aligned}
\end{equation*}
Uguagliando i due membri
\[
	\boxed{\frac{\partial^2 E_y}{\partial x^2} = \frac{1}{c^2} \frac{\partial^2 E_y}{\partial t^2}}
\]
Con conti analoghi, per $ \vec{B}$ si trova:
\[
	\boxed{\frac{\partial^2 B_z}{\partial x^2} = \frac{1}{c^2} \frac{\partial^2 B_z}{\partial t^2}}
\]
Con $ E_y = E_y(x-ct) $ e $ B_z  = B_z (x-ct) $
Vediamo come le due equazioni sono legate. Introduciamo la variabile ausiliaria $ \eta = x-ct $ e ricordiamo la terza equazione di \eqref{eq:componenti1}:
\begin{gather*}
	\frac{\partial E_y}{\partial x} = \frac{\partial E_y}{\partial \eta} \frac{\partial \eta}{\partial x} = \frac{\partial E_y}{\partial \eta} \qquad \frac{\partial B_z}{\partial t} = \frac{\partial B_z}{\partial \eta} \frac{\partial \eta}{\partial t} = (-c) \frac{\partial B_z}{\partial \eta} \\
	\frac{\partial E_y}{\partial \eta} = c \frac{\partial B_z}{\partial \eta} \implies E_y = c\,B_z + \text{costante}
\end{gather*}
Ponendo anche in questo caso a zero la costante, che descrive una situazione stazionaria alla quale non siamo interessati, ricaviamo
\[
	\boxed{E_y = c\,B_z}
\]
Ricaviamo la relazione vettoriale
\[
	\left. \begin{array}{r}
	 	E_y = E_y(x-ct) \\
		B_z = \frac{E_y}{c}(x-ct)
	\end{array} \right. \implies
	\left. \begin{array}{l}
	 	\vec{E} = E_y\vec{u}_y  \\
		\vec{B} = \frac{E_y}{c} \vec{u}_z
	\end{array} \right.
\]
Vista la relazione fra le componenti, possiamo scrivere direttamente $\vec{E}$ come:
\[
	\boxed{\vec{E} = \vec{B} \times \vec{v}}
\]
\begin{figure}[htpb]
	\centering
	

	\tikzset{every picture/.style={line width=0.75pt}} %set default line width to 0.75pt        

	\begin{tikzpicture}[x=0.75pt,y=0.75pt,yscale=-1,xscale=1]
	%uncomment if require: \path (0,300); %set diagram left start at 0, and has height of 300

	%Straight Lines [id:da9625958687428804] 
	\draw    (172,133) -- (172,29) ;
	\draw [shift={(172,26)}, rotate = 450] [fill={rgb, 255:red, 0; green, 0; blue, 0 }  ][line width=0.08]  [draw opacity=0] (10.72,-5.15) -- (0,0) -- (10.72,5.15) -- (7.12,0) -- cycle    ;
	%Straight Lines [id:da22532612058365742] 
	\draw    (172,133) -- (68,133) ;
	\draw [shift={(65,133)}, rotate = 360] [fill={rgb, 255:red, 0; green, 0; blue, 0 }  ][line width=0.08]  [draw opacity=0] (10.72,-5.15) -- (0,0) -- (10.72,5.15) -- (7.12,0) -- cycle    ;
	%Straight Lines [id:da060490375769127525] 
	\draw    (172,133) -- (400.78,239.73) ;
	\draw [shift={(403.5,241)}, rotate = 205.01] [fill={rgb, 255:red, 0; green, 0; blue, 0 }  ][line width=0.08]  [draw opacity=0] (10.72,-5.15) -- (0,0) -- (10.72,5.15) -- (7.12,0) -- cycle    ;
	%Straight Lines [id:da48744842392750254] 
	\draw    (237.16,145.61) -- (237.16,75.89) ;
	\draw [shift={(237.16,72.89)}, rotate = 450] [fill={rgb, 255:red, 0; green, 0; blue, 0 }  ][line width=0.08]  [draw opacity=0] (10.72,-5.15) -- (0,0) -- (10.72,5.15) -- (7.12,0) -- cycle    ;
	%Straight Lines [id:da8915174532151426] 
	\draw    (224.16,166.61) -- (154.44,166.61) ;
	\draw [shift={(151.44,166.61)}, rotate = 360] [fill={rgb, 255:red, 0; green, 0; blue, 0 }  ][line width=0.08]  [draw opacity=0] (10.72,-5.15) -- (0,0) -- (10.72,5.15) -- (7.12,0) -- cycle    ;
	%Straight Lines [id:da46065770700614905] 
	\draw    (257.16,158.61) -- (383.88,217.73) ;
	\draw [shift={(386.6,219)}, rotate = 205.01] [fill={rgb, 255:red, 0; green, 0; blue, 0 }  ][line width=0.08]  [draw opacity=0] (10.72,-5.15) -- (0,0) -- (10.72,5.15) -- (7.12,0) -- cycle    ;

	% Text Node
	\draw (55,130) node    {$z$};
	% Text Node
	\draw (189,23) node    {$y$};
	% Text Node
	\draw (417,243) node    {$x$};
	% Text Node
	\draw (363,177) node    {$\vec{v} =c\vec{u}_{x}$};
	% Text Node
	\draw (284,93) node    {$\vec{E} =E_{y}\vec{u}_{y}$};
	% Text Node
	\draw (182,193) node    {$\vec{B} =\frac{E_{y}}{c}\vec{u}_{z}$};


	\end{tikzpicture}
\end{figure}
\FloatBarrier
Lo stesso ragionamento può essere attuato ipotizzando $E_y=0$. Si troveranno allora le seguenti espressioni per il campo elettrico e magnetico (qui in particolare nel caso di onde progressive):
\[
	\left\{ \begin{array}{l}
	 	\vec{E} = E_y(x-ct)\vec{u}_y + E_z(x-ct) \vec{u}_z \\
		\vec{B} = \frac{E_y}{c} \vec{u}_z -\frac{E_z}{c}\vec{u}_y
	\end{array} \right.
\]
Si verifica facilmente che i due campi sono ortogonali calcolando il prodotto scalare:
\[
	\vec{E} \cdot \vec{B} = E_yB_y + E_zB_z = E_y\left( -\frac{E_z}{c} \right) +E_z \left( \frac{E_y}{c} \right)   = 0
\]
Si verifica che sono a loro volta ortogonali alla direzione di propagazione dell'onda:
\begin{equation*}
	\begin{aligned}
		\vec{E} \times \vec{B} &= \begin{vmatrix}
		\vec{u}_x & \vec{u}_y & \vec{u}_z \\
		E_x & E_y & E_z \\
		B_x & B_y & B_z \\
		\end{vmatrix}
		= \begin{vmatrix}
		\vec{u}_x & \vec{u}_y & \vec{u}_z \\
		0 & E_y & E_z \\
		0 & -\frac{E_z}{c} & \frac{E_y}{c} \\
		\end{vmatrix} \\
		&= \vec{u}_x\left( \frac{E_y^2}{c} + \frac{E_z^2}{c} \right) = \frac{|\vec{E}|^2}{c} \vec{u}_x
	\end{aligned}
\end{equation*}
L'onda elettromagnetica non è di carattere vettoriale ma \emph{tensoriale}. Per convenzione il concetto di polarizzazione si applica alla componente elettrica dell'onda elettromagnetica. La relazione fra i due è dopotutto chiara. Abbiamo la polarizzazione lineare in cui il campo elettrico oscilla soltanto in una posizione. La luce naturale emessa dal sole è un onda non polarizzata, è costituita dalla luce emessa da tantissimi emettitori senza alcuna correlazione.
\begin{figure}[htpb]
	\centering
	\colorlet{myblue}{black!40!blue}
	\colorlet{myred}{black!40!red}




	% Electromagnetic wave - black
	\begin{tikzpicture}[x=(-15:1.2), y=(90:1.0), z=(-150:1.0),
	                    line cap=round, line join=round,
	                    axis/.style={black, thick,->},
	                    vector/.style={>=stealth,->}]
	  \large
	  \def\A{1.5}
	  \def\nNodes{5} % use even number
	  \def\nVectorsPerNode{8}
	  \def\N{\nNodes*40}
	  \def\xmax{\nNodes*pi/2*1.01}
	  \pgfmathsetmacro\nVectors{(\nVectorsPerNode+1)*\nNodes}
	 
	  \def\vE{\mathbf{E}}
	  \def\vB{\mathbf{B}}
	  \def\vk{\mathbf{\hat{k}}}
	 
	  % main axes
	  \draw[axis] (0,0,0) -- ++(\xmax*1.1,0,0) node[right] {$x$};
	  \draw[axis] (0,-\A*1.4,0) -- (0,\A*1.4,0) node[right] {$y$};
	  \draw[axis] (0,0,-\A*1.4) -- (0,0,\A*1.4) node[above left] {$z$};
	 
	  % small axes
	  \def\xOffset{{(\nNodes-2)*pi/2}}
	  \def\yOffset{\A*1.2}
	  \def\zOffset{\A*1.2}
	  \draw[axis] (\xOffset,\yOffset,-\zOffset) -- ++(\A*0.6,0,0) node[right] {$\vec{k}$};
	  \draw[axis] (\xOffset,\yOffset,-\zOffset) -- ++(0,\A*0.6,0) node[right] {$\vec{E}$};
	  \draw[axis] (\xOffset,\yOffset,-\zOffset) -- ++(0,0,\A*0.6) node[above left] {$\vec{B}$};
	 
	  % equation
	  %\node[above right] at (\xOffset,-0.5*\yOffset,4*\zOffset)
	  %  {$\begin{aligned}
	  %    \vE &= \mathbf{E_0}\sin(\vk\cdot\mathbf{x}-c_0t)\\
	  %    \vB &= \mathbf{B_0}\sin(\vk\cdot\mathbf{x}-c_0t)\\
	  %    \end{aligned}$};
	  %\node[below right] at (\xOffset,-0.5*\yOffset,4*\zOffset)
	  %  {$\vE\cdot\vk = 0,\;\; \vB\cdot\vk = 0,\;\; \vB = \frac{1}{c_0}\vk\times\vE$};
	 
	  % waves
	  \draw[very thick,variable=\t,domain=0:\nNodes*pi/2*1.01,samples=\N]
	    plot (\t,{\A*sin(\t*360/pi)},0);
	  \draw[very thick,variable=\t,domain=0:\nNodes*pi/2*1.01,samples=\N]
	    plot (\t,0,{\A*sin(\t*360/pi)});
	 
	  % draw vectors
	  \foreach \k [evaluate={\t=\k*pi/2/(\nVectorsPerNode+1);
	                         \angle=\k*90/(\nVectorsPerNode+1);
	                         \c=(mod(\angle,90)!=0);}]
	              in {1,...,\nVectors}{
	    \if\c1
	      \draw[vector] (\t,0,0) -- ++(0,{\A*sin(2*\angle)},0);
	      \draw[vector] (\t,0,0) -- ++(0,0,{\A*sin(2*\angle)});
	    \fi
	  }
	 
	\end{tikzpicture}
\end{figure}
\FloatBarrier







































\section{Energia di un'onda elettromagnetica piana e Vettore di Poynting}

Al campo elettrico e al campo magnetico sono associate delle densità di energia per unità di volume.
\[
	u_e = \frac{1}{2} \varepsilon_0 E^2 \qquad u_m=\frac{1}{2} \frac{B^2}{\mu_0}
\]
Questo fatto si può generalizzare nel caso di campo magnetico ed elettrico variabili nel tempo. La loro presenza comporta l'esistenza di una certa quantità di \textbf{energia distribuita nello spazio} con densità $u$. Per un'onda elettromagnetica avremo una densità di energia per unità di volume pari a:
\begin{align*}
	u = u_e + u_m &= \frac{1}{2} \varepsilon_0 E^2 + \frac{1}{2} \frac{B^2}{\mu_0} \\
	&= \frac{1}{2} \varepsilon_0 B^2 c^2 + \frac{1}{2} \frac{B^2}{\mu_0} \\
	&= \frac{1}{2} \varepsilon_0 B^2 \frac{1}{\mu_0 \varepsilon_0} + \frac{1}{2} \frac{B^2}{\mu_0} \\
	&= \frac{1}{2} \frac{B^2}{\mu_0} + \frac{1}{2} \frac{B^2}{\mu_0} \\
	&= 2u_m = 2u_e \\
	\Aboxed{u = u_e + u_m &= 2u_m = 2u_e}
\end{align*}
Abbiamo scritto una densità di energia istantanea per unità di volume, perché i campi variano nel tempo.
Avendo verificato che un'onda elettromagnetica trasporta dell'energia associata ai due campi, facciamo qualche ragionamento in più sui bilanci energetici dell'interazione tra un'onda elettromagnetica e un sistema. Consideriamo un sistema racchiuso da una superficie $\Sigma$ che interagisca con delle onde elettromagnetiche. Ci possono essere al suo interno delle correnti che eventualmente sono messe in moto da generatori di $fem$, dei campi elettromagnetici e quindi delle onde elettromagnetiche che si propagano e a cui è associata una certa quantità di energia elettromagnetica complessiva:
\[
	\boxed{U = \int_{\tau}(u_e+u_m)d\tau = \int_{\tau}\left[ \frac{1}{2} \vec{D} \cdot \vec{E} + \frac{1}{2} \vec{B} \cdot \vec{H} \right] d\tau}
\]
Si tratta di un'espressione più generale che tiene conto della presenza di eventuali materiali dielettrici e materiali magnetizzati. Tale energia potrà variare nel tempo per vari motivi. I generatori infatti mettono in moto le cariche, ci può essere un flusso di energia causato dalle onde elettromagnetiche che entrano o escono dal sistema trasportando energia o ancora ci può essere dissipazione per effetto Joule. Se andiamo a studiare come tale energia varia, otteniamo:
\begin{equation*}
	\begin{aligned}
		\frac{dU}{dt} &= \int_{\tau}\frac{\partial}{\partial t} \left[ \frac{1}{2} \vec{D} \cdot \vec{E} + \frac{1}{2} \vec{B} \cdot \vec{H} \right] d\tau \\
		&= \int_{\tau} \frac{1}{2} \left[ \frac{\partial \vec{D}}{\partial t} \cdot \vec{E} +\vec{D} \cdot \frac{\partial \vec{E}}{\partial t} + \frac{\partial \vec{B}}{\partial t} \cdot \vec{H} + \vec{B} \cdot \frac{\partial \vec{H}}{\partial t}  \right] d\tau
	\end{aligned}
\end{equation*}
Supponendo che i dielettrici siano omogenei ed isotropi e che i materiali magnetizzati siano di tipo diamagnetico o paramagnetico
\begin{align*}
	\frac{dU}{dt} &= \int_{\tau} \frac{1}{2} \left[ \frac{\partial \vec{D}}{\partial t} \cdot \vec{E} + \varepsilon_0 \varepsilon_r \vec{E} \cdot \frac{\partial \vec{E}}{\partial t} + \frac{\partial \vec{B}}{\partial t} \cdot \vec{H} + \mu_0 \mu_r \vec{H} \cdot \frac{\partial \vec{H}}{\partial t}  \right] d\tau \\
	&= \int_{\tau} \frac{1}{2} \left[ \frac{\partial \vec{D}}{\partial t} \cdot \vec{E} + \vec{E} \cdot \frac{\partial \vec{D}}{\partial t} + \frac{\partial \vec{B}}{\partial t} \cdot \vec{H} + \vec{H} \cdot \frac{\partial \vec{B}}{\partial t}  \right] d\tau \\
	&= \int_{\tau} \left[ \vec{E} \cdot \frac{\partial \vec{D}}{\partial t} + \vec{H} \cdot \frac{\partial \vec{B}}{\partial t}  \right] d\tau \\
	&= \int_{\tau} \left[ \vec{E} \cdot \left( \vec{\nabla} \times \vec{H} -\vec{J}  \right)   + \vec{H} \cdot \left( -\vec{\nabla} \times \vec{E}  \right)    \right] d\tau \tag*{III e IV di Maxwell}\\
	&= \int_{\tau}\left[ \vec{E}\cdot  \vec{\nabla} \times \vec{H} -\vec{H} \cdot \vec{\nabla} \times \vec{E} -\vec{E} \cdot \vec{J}  \right] d\tau
\end{align*}
Si può dimostrare che
\[
	\vec{\nabla} \cdot (\vec{a} \times \vec{b} ) = \vec{b} (\vec{\nabla} \times \vec{a} ) - \vec{a} (\vec{\nabla} \times \vec{b} )
\]
Pertanto:
\begin{equation*}
	\begin{aligned}
		\frac{dU}{dt} &= \int_{\tau} \left[ \vec{\nabla} \cdot \left( \vec{H} \times \vec{E}  \right) - \vec{E} \cdot \vec{J} \right] d\tau \\
		&= \int_{\tau} \left[ - \vec{\nabla} \cdot \underbrace{\left( \vec{E} \times \vec{H}  \right)}_{\vec{S}} - \vec{E} \cdot \vec{J} \right] d\tau
	\end{aligned}
\end{equation*}
Introduciamo una nuova quantità, chiamiamo \textbf{Vettore di Poynting.}
\[
	\boxed{\vec{S} = \vec{E} \times \vec{H}}
\]
Quindi
\begin{equation*}
	\begin{aligned}
		\frac{dU}{dt} &= \int_{\tau} \left[ - \vec{\nabla} \cdot \vec{S} - \vec{E} \cdot \vec{J} \right] d\tau \\
		&= -\int_{\Sigma}\vec{S} \cdot \vec{n} \,dS - \int_{\tau}  \vec{E} \cdot \vec{J} d\tau \\
	\end{aligned}
\end{equation*}
Nel caso di conduttori ohmici possiamo attuare una seconda sostituzione:
\[
	\vec{J} = \sigma (\vec{E} + \vec{E}_m) \implies  \vec{E} = \frac{\vec{J}}{\sigma} - \vec{E}_m = \rho \vec{J} - \vec{E}_m
\]
L'espressione dell'energia diventa quindi:
\[
	\frac{dU}{dt} = -\int_{\Sigma}\vec{S} \cdot \vec{n} \,dS - \int_{\tau}  \left( \rho \vec{J} -\vec{E}_m  \right)   \cdot \vec{J} \, d\tau
\]
\[
	\boxed{\frac{dU}{dt} = -\int_{\Sigma}\vec{S} \cdot \vec{n} \,dS - \int_{\tau} \rho J^2 \, d\tau - \int_{\tau} \vec{J} \cdot \vec{E}_m \, d\tau}
\]
Notiamo che l'ultimo termine ha le dimensioni di una potenza, quindi, trattandosi di una somma, anche gli altri due termini avranno le stesse dimensioni. Possiamo attribuire un significato a questo punto ai contributi rilevati.
\begin{itemize}
	\item Il primo contributo rappresenta la \emph{potenza irradiata verso l'esterno assieme alle onde elettromagnetiche}
	\item Il secondo è l'\emph{energia dissipata per effetto Joule}
	\item Il terzo la \emph{potenza che i generatori danno al sistema} per mantenere le cariche in moto
\end{itemize}
Vediamo come quindi parte dell'energia venga trasformata in calore. A causa dell'ultimo termine tuttavia essa può anche aumentare.
Il bilancio energetico può essere scritto in modo compatto come:
\[
	\frac{dU}{dt} = - W_{oem} - W_J + W_g
\]
Concentriamoci sul primo dei tre contributi di questa sommatoria, che abbiamo detto essere la potenza dissipata durante la propagazione dell'onda. Esso è pari a:
\[
	W_{oem} = \int_{\Sigma} \vec{S} \cdot \vec{n} \,dS
\]
Tale potenza è quindi data dal flusso del vettore di Poynting attraverso la superficie. Il modulo di $\vec{S}$ rappresenta allora l'energia elettromagnetica che per unità di tempo passa attraverso l'unità di superficie ortogonale alla direzione di propagazione. Ecco perché quindi esso prende il nome di \textbf{intensità dell'onda elettromagnetica}.
Nella pratica è importante non tanto calcolare il flusso istantaneo di energia quanto il flusso medio. Il motivo è che la pulsazione delle onde elettromagnetiche è in generale molto elevata $ 10^{15}\,rad/s $ e gli strumenti di misura riescono a determinare soltanto il valor medio dell'energia che li colpisce, non potendo essere sensibili a variazioni così importanti.

Applichiamo questi risultati, validi per una qualsiasi onda elettromagnetica, ad \emph{un'onda piana armonica polarizzata rettilinearmente}. Proviamo a calcolarne il vettore di Poynting.
\begin{align*}
		\vec{H} = \frac{\vec{B}}{\mu_0 \mu_r} \implies \vec{S} &= \vec{E} \times \vec{H} \\
		&=\frac{\vec{E} \times \vec{B}}{\mu_0 \mu_r} \\
		&=\frac{EB\,\vec{u}_z}{\mu_0 \mu_r} \\
		&=\frac{B^2 \,v\,\vec{u}_z}{\mu_0 \mu_r} \\
		&=\frac{B^2}{\mu_0 \mu_r}\vec{v} \\
		&=u\,\vec{v} \\
	\Aboxed{\vec{S} = u\,\vec{v} = \frac{B^2}{\mu_0 \mu_r}\vec{v} &= \varepsilon_0 \varepsilon_r E^2 \vec{v}}
\end{align*}
\begin{figure}[htpb]
	\centering
	

	\tikzset{every picture/.style={line width=0.75pt}} %set default line width to 0.75pt        

	\begin{tikzpicture}[x=0.75pt,y=0.75pt,yscale=-1,xscale=1]
	%uncomment if require: \path (0,300); %set diagram left start at 0, and has height of 300

	%Straight Lines [id:da7364847646617674] 
	\draw    (192,153) -- (192,69.5) ;
	\draw [shift={(192,66.5)}, rotate = 450] [fill={rgb, 255:red, 0; green, 0; blue, 0 }  ][line width=0.08]  [draw opacity=0] (10.72,-5.15) -- (0,0) -- (10.72,5.15) -- (7.12,0) -- cycle    ;
	%Straight Lines [id:da760698141490767] 
	\draw    (192,153) -- (108.5,153) ;
	\draw [shift={(105.5,153)}, rotate = 360] [fill={rgb, 255:red, 0; green, 0; blue, 0 }  ][line width=0.08]  [draw opacity=0] (10.72,-5.15) -- (0,0) -- (10.72,5.15) -- (7.12,0) -- cycle    ;
	%Straight Lines [id:da3443805156824993] 
	\draw    (192,153) -- (340.4,222.23) ;
	\draw [shift={(343.12,223.5)}, rotate = 205.01] [fill={rgb, 255:red, 0; green, 0; blue, 0 }  ][line width=0.08]  [draw opacity=0] (10.72,-5.15) -- (0,0) -- (10.72,5.15) -- (7.12,0) -- cycle    ;
	%Shape: Boxed Line [id:dp3966841975472175] 
	\draw    (218,135) -- (334.25,189.23) ;
	\draw [shift={(336.97,190.5)}, rotate = 205.01] [fill={rgb, 255:red, 0; green, 0; blue, 0 }  ][line width=0.08]  [draw opacity=0] (10.72,-5.15) -- (0,0) -- (10.72,5.15) -- (7.12,0) -- cycle    ;
	%Straight Lines [id:da8784689277109099] 
	\draw    (243,210.66) -- (304.25,239.23) ;
	\draw [shift={(306.97,240.5)}, rotate = 205.01] [fill={rgb, 255:red, 0; green, 0; blue, 0 }  ][line width=0.08]  [draw opacity=0] (10.72,-5.15) -- (0,0) -- (10.72,5.15) -- (7.12,0) -- cycle    ;

	% Text Node
	\draw (357,225) node    {$\vec{v}$};
	% Text Node
	\draw (210,68) node    {$\vec{E}$};
	% Text Node
	\draw (90,151) node    {$\vec{B}$};
	% Text Node
	\draw (276,246) node    {$\vec{u}_{z}$};
	% Text Node
	\draw (321,163) node    {$\vec{S}$};


	\end{tikzpicture}
\end{figure}
\FloatBarrier
Poiché il campo elettrico di un'onda piana polarizzata può essere pensato come la somma vettoriale di due campi elettrici sfasati, ortogonali tra loro e alla direzione di propagazione, ciascuno trasportante energia elettromagnetica, possiamo calcolar l'intensità media dell'onda per ognuna delle due componenti:
\begin{gather*}
	\vec{E}_y = \vec{E}_{0y}\sin (kx-\omega t)\vec{u}_y \\
	|\vec{S} (t)| = I(t) = \varepsilon_0 \varepsilon_r vE^2 (t) = \varepsilon_0 \varepsilon_r v\,E_{0y}^2 \sin^2 (kx-\omega t) \\
	I_{\text{media}}= I_m = \frac{1}{T}\int_0^T I(t) dt = \varepsilon_0 \varepsilon_r v\,E_{0y}^2 \underbrace{\frac{1}{T} \int_0^T \sin^2 (kx-\omega t)dt}_{= \frac{1}{2}}
\end{gather*}
Quindi
\[
	I_m = \frac{1}{2} \varepsilon_0 \varepsilon_r v\,E_{0y}^2
\]
Per un generico stato di polarizzazione
\[
	\boxed{I_m = \frac{1}{2} \varepsilon_0 \varepsilon_r v \left( E_{0y}^2 + E_{0z}^2 \right)}
\]







































\section{Quantità di moto di un'onda elettromagnetica piana}

Per dimostrare che le onde elettromagnetiche trasportano quantità di moto oltre che energia, consideriamo un volume $ \tau  $ in una regione in cui si sta propagando un'onda. Facciamo l'ipotesi che in questo volume siano contenute delle cariche. Sappiamo che la forza agente su una singola carica è data dalla somma di una componente elettrica e di una componente magnetica:
\[
	\vec{F}_{\text{singola carica}} =q\vec{E} +q\vec{v} \times \vec{B}
\]
Se consideriamo un'insieme di cariche, ciascuna di esse si muove di moto casuale, legato all'agitazione termica. Ciò che conta come risultante delle forze è quella parte delle forze di Lorentz che riguarda la componente coordinata del moto delle cariche, e quindi quella velocità di deriva introdotta parlando delle correnti.
A questo punto consideriamo una porzione di materiale. L'onda potrebbe anche non arrestarsi alla superficie, ma attraversarla. Consideriamone un cubetto $ d\tau  $ al cui interno vi sono un numero di cariche $dN$. La forza agente sul cubetto è data da
\[
	d\vec{F} =dN\,q(\vec{E} +\vec{v}_d\times \vec{B}  )
\]
Conviene a questo punto ragionare in termini di forza per unità di volume:
\[
	\vec{f} = \frac{d\vec{F}}{d\tau}=\frac{dN}{d\tau}q(\vec{E} +\vec{v}_d\times \vec{B}  ) = n\,q(\vec{E} +\vec{v}_d\times \vec{B}  )
\]
Possiamo calcolare la potenza dissipata dalla forza all'interno del materiale.
\[
	dW=d\vec{F} \cdot \vec{v}_d
\]
Quella dissipata per unità di volume è data da:
\[
	w=\frac{dW}{d\tau}=\frac{d\vec{F} \cdot \vec{v}_d}{d\tau} = \underbrace{\frac{d\vec{F}}{d\tau}}_{\vec{f}}   \vec{v}_d  = n\,q(\vec{E} +\vec{v}_{d\times \vec{B}}  )\cdot \vec{v}_d = n\,q\,\vec{E} =\vec{J} \cdot \vec{E}
\]
Per i conduttori ohmici abbiamo $w = \vec{J} \cdot \vec{E} =\sigma E^2$.

Si tratta di una potenza istantanea, perché il campo elettrico è oscillante. Se vogliamo lavorare su valori misurabili, si lavora con la potenza media. Notiamo che un onda elettromagnetica interagente con un materiale, può cedere ad esso dell'energia, c'è un assorbimento.
\[
	w_m=\sigma (E^2)_m=\frac{\sigma E_0^2}{2}
\]
Questo non è l'unico effetto che incontriamo. La forza di lorentz è stata scartata perché non compie lavoro, ma non è trascurabile. Il campo $\vec{E}$ pone in moto le cariche e il campo magnetico, vedendole in movimento, applica una forza di Lorentz, diretta nella stessa direzione di propagazione dell'onda. L'onda sta spingendo il materiale. Per comprendere più nel dettaglio cosa accade, ricorriamo al \emph{teorema dell'impulso}.

Grazie ad esso possiamo calcolare la variazione della quantità di moto legata all'azione di una forza variabile nel tempo, in un intervallo T come
\[
	m\vec{v}_f - m\vec{v}_i=\int_{t_i}^{t_f} \vec{F} (t)dt = \vec{I} (t_i,t_f  )
\]
Questo integrale si chiama impulso della forza $\vec{F}$ in $(t_i, t_f)$. Dal momento che siamo interessati all'interazione fra la parte magnetica dell'onda e il materiale, consideriamo solo la componente magnetica della forza per unità di volume:
\[
	\vec{f}_m = n\,q\,\vec{v}_d\times \vec{B} = \vec{J} \times \vec{B} = |\vec{J}|\,|\vec{B}|\vec{u}_x
\]
Possiamo sfruttare la relazione fra campo elettrico e magnetico studiata in precedenza e riscrivere l'espressione in un altro modo:
\[
	\vec{f}_m = |\vec{J}|\,|\vec{B}|\vec{u}_x = \underbrace{\sigma |\vec{E} |}_{|\vec{J}|} |\vec{B} |\vec{u}_x = \sigma E \underbrace{\frac{E}{v}}_B \vec{u}_x = \frac{\sigma E^2}{v}\vec{u}_x
\]
La forza che agisce nella direzione di propagazione dell'onda sarà pari a:
\[
	d\vec{F} = \frac{\sigma E^2}{v}\vec{u}_x \, d\tau
\]
Ci aspettiamo che la forza trasferisca all'oggetto un quantità di moto. Possiamo calcolare la sua variazione su un ciclo $T$ di oscillazione dell'onda con il teorema dell'impulso.
\[
	\frac{\Delta \vec{q}}{T} = \frac{1}{T} \int_0^T d\vec{F} \,dt = d\vec{F}_{\text{media}} =  \frac{\sigma E^2}{v}\vec{u}_x \, d\tau = \frac{w_m\,d\tau}{v}\vec{u}_x
\]
Abbiamo trovato la quantità di moto media ceduta al materiale per unità di tempo e per unità di volume
\[
	\vec{q}_m = \frac{\Delta \vec{q}}{T\,d\tau} = \frac{w_m}{v}\vec{u}_x = \frac{w_m}{v^2}\vec{v}
\]
Abbiamo appreso che quando l'onda elettromagnetica interagisce con il materiale ci sono due effetti:
\begin{itemize}
	\item Dissipazione di energia: l'onda cede energia al materiale perché mette in moto le cariche e compie lavoro.
	\item Trasferimento di quantità di moto: componente magnetica spinge il materiale e le trasferisce una quantità di moto per unità di volume e di tempo direttamente proporzionale alla potenza assorbita per unità di volume.
\end{itemize}
Questo ragionamento attuato in ambito classico, ritorna anche quando si descrive il campo elettromagnetico in termini quantistici. Il campo infatti si può vedere come un'insieme di fotoni, quanti che trasportano energia. In base a quanto visto un fotone deve avere anche una quantità di moto che viene ceduta al materiale, secondo la formula appena trovata. I fotoni quindi pur essendo particelle di massa nulla, hanno anche una quantità i moto.

\textbf{Caso di assorbimento completo dell'onda alla superficie. Processo di radiazione.} Abbiamo ragionato in termini di assorbimento nel volume di un materiale. Ci sono tuttavia delle situazioni in cui l'onda viene assorbita direttamente sulla superficie, senza attraversarla. In tal caso non possiamo ragionare in termini di energia per unità di volume, ma di energia per unità di superficie.
La potenza ceduta alla superficie per unità di superficie è data dal vettore di Poynting. L'intensità media trasferita sarà proprio il modulo di tale vettore.
\[
	I_m = |\vec{S}_m |
\]
La quantità di moto ceduta per unità di tempo unità di superficie sarà pari a: $ \vec{p}_m = \frac{|\vec{S}_m |}{v}\vec{u}_x = \frac{\vec{S}_m}{v}  $. Il vettore di Poynting è diretto secondo la direzione di propagazione dell'onda.

Le dimensioni della quantità di moto saranno date da:
\[
	[\vec{p}_m ] = \frac{[S_m]}{\frac{[L]}{[T]}} = \frac{[S_m][T]}{[L]} = \frac{[W][T]}{[L^2][L]} = \frac{[F][L][T]}{[T][L^2][L]} = \frac{[F]}{[L^2 ]}
\]
Essa ha le dimensioni di una pressione. Viene per questo chiamata pressione di radiazione. Quando un'onda elettromagnetica colpisce una superficie e viene assorbita, la spinge esercitando una pressione. L'effetto è piccolo ma può essere rilevato. Ci sono esperienze di laboratorio in cui si costruiscono dei mulinelli con superfici assorbenti evsi pongono all'interno di una sfera, che così si illumina. C'è chi propone di usare la pressione di radiazione per propulsione spaziale. Costruendo una navicella con una gigantesca vela solare, la pressione di radiazione la spingerebbe senza dover consumare carburante.







































\section{Propagazione di onde elettromagnetiche all'interno di un dielettrico}

Siamo interessati a capire come si propagano le onde elettromagnetiche in un materiale dielettrico. Supporremo che esso sia omogeneo e isotropo. Qualunque materiale ha delle proprietà magnetiche, supporremo che si tratti di un materiale o dia o paramagnetico. Valgono allora le seguenti relazioni:
\[
	\vec{D} =\varepsilon_0 \varepsilon_r \vec{E} \qquad \vec{B} =\mu_0 \mu_r \vec{H}
\]
Cominceremo dalle equazioni di Maxwell nella loro forma più generale. Le considereremo sempre in assenza di cariche e correnti nella regione considerata. Sotto queste ipotesi esse sono molto
semplici:
\begin{equation*}
	\begin{aligned}
		&\text{div}\vec{D} =0 \\
		&\text{div}\vec{B} =0 \\
		&\text{rot}\vec{E} = -\frac{\partial \vec{B}}{\partial t} \\
		&\text{rot}\vec{H} = \frac{\partial \vec{D}}{\partial t}
	\end{aligned}
\end{equation*}
Applicando il rotore da entrambe le parti della terza equazione:
\begin{equation*}
	\begin{aligned}
		\vec{\nabla} \times (\vec{\nabla} \times \vec{E} ) &= \vec{\nabla} \times \left( -\frac{\partial \vec{B}}{\partial t}  \right) \\
		\vec{\nabla} ( \underbrace{\vec{\nabla} \cdot \vec{E}}_{=0} )-\nabla^2 \vec{E}  &= - \frac{\partial}{\partial t} ( \vec{\nabla} \times \underbrace{\mu_0 \mu_r \vec{H}}_{\vec{B}} )
	\end{aligned}
\end{equation*}
In quanto $ \text{div}\vec{D} = \text{div}(\varepsilon_0 \varepsilon_r \vec{E} )=\varepsilon_0 \varepsilon_r \,\text{div}\vec{E} = 0 $. Allora continuando i calcoli sulla terza equazione e sostituendo la quarta possiamo scrivere:
\begin{equation*}
	\begin{aligned}
		\nabla^2 \vec{E}  &= \mu_0 \mu_r \frac{\partial}{\partial t} ( \vec{\nabla} \times \vec{H}) \\
		\nabla^2 \vec{E}  &= \mu_0 \mu_r \frac{\partial}{\partial t} \left(\frac{\partial \vec{D}}{\partial t} \right) \\
		\nabla^2 \vec{E}  &= \mu_0 \mu_r \frac{\partial}{\partial t} \left(\frac{\varepsilon_0 \varepsilon_r \partial \vec{E}}{\partial t} \right) \\
		\nabla^2 \vec{E}  &= \mu_0 \mu_r \varepsilon_0 \varepsilon_r \frac{\partial^2 \vec{E}}{\partial t^2} \implies \nabla^2 \vec{E}  - \mu_0 \mu_r \varepsilon_0 \varepsilon_r \frac{\partial^2 \vec{E}}{\partial t^2} = 0\\
	\end{aligned}
\end{equation*}
Abbiamo trovato l'equazione dell'onda, nella forma
\[
	\nabla^2 \xi - \frac{1}{v^2} \frac{\partial^2 \xi}{\partial t^2} =0
\]
Quando abbiamo studiato le onde elettromagnetiche nel vuoto, avevamo detto che $ \frac{1}{c^2}=\mu_0 \varepsilon_0  $. Quello che possiamo dire è che
\[
	\boxed{\frac{1}{v^2}= \mu_0 \varepsilon_0 \mu_r \varepsilon_r =\frac{\mu_r \varepsilon_r}{c^2}}
\]
La velocità delle onde elettromagnetiche nel materiale è diversa
\[
	\boxed{v = \frac{c}{\sqrt{\mu_r \varepsilon_r}}}
\]
In particolare è inferiore alla velocità di propagazione nel vuoto.
Si noti che nei materiali dia e paramagnetici il valore di $\mu_r$ è prossimo all'unità, per cui spesso viene omesso nella espressione di $v$.
Consideriamo solo la parte elettrica in un'onda elettromagnetica piana. Essa si scrive come:
\[
	\vec{E} (x,t)=\vec{E}_0\sin (kx-\omega t) \implies \omega = kc
\]
\begin{figure}[htpb]
	\centering
	

	% Pattern Info
	 
	\tikzset{
	pattern size/.store in=\mcSize, 
	pattern size = 5pt,
	pattern thickness/.store in=\mcThickness, 
	pattern thickness = 0.3pt,
	pattern radius/.store in=\mcRadius, 
	pattern radius = 1pt}
	\makeatletter
	\pgfutil@ifundefined{pgf@pattern@name@_5m8r8uhee}{
	\pgfdeclarepatternformonly[\mcThickness,\mcSize]{_5m8r8uhee}
	{\pgfqpoint{0pt}{0pt}}
	{\pgfpoint{\mcSize+\mcThickness}{\mcSize+\mcThickness}}
	{\pgfpoint{\mcSize}{\mcSize}}
	{
	\pgfsetcolor{\tikz@pattern@color}
	\pgfsetlinewidth{\mcThickness}
	\pgfpathmoveto{\pgfqpoint{0pt}{0pt}}
	\pgfpathlineto{\pgfpoint{\mcSize+\mcThickness}{\mcSize+\mcThickness}}
	\pgfusepath{stroke}
	}}
	\makeatother
	\tikzset{every picture/.style={line width=0.75pt}} %set default line width to 0.75pt        

	\begin{tikzpicture}[x=0.75pt,y=0.75pt,yscale=-1,xscale=1]
	%uncomment if require: \path (0,300); %set diagram left start at 0, and has height of 300

	%Curve Lines [id:da06465267321472212] 
	\draw [line width=1.5]    (124.5,111.75) .. controls (134.71,111.5) and (134.14,189.5) .. (144.5,189.75) .. controls (154.86,190) and (155.86,112.64) .. (165,112.75) .. controls (174.14,112.86) and (174.43,189.5) .. (185.5,189.25) .. controls (196.57,189) and (195.86,111.21) .. (206,111.25) .. controls (216.14,111.29) and (216.43,189.21) .. (226.5,189.25) .. controls (236.57,189.29) and (236.14,112.07) .. (246.5,112.25) .. controls (256.86,112.43) and (256.43,189.5) .. (267,189.75) .. controls (277.57,190) and (277.29,112.64) .. (287.5,112.75) ;
	%Shape: Rectangle [id:dp2521469075297418] 
	\draw  [draw opacity=0][pattern=_5m8r8uhee,pattern size=6pt,pattern thickness=0.75pt,pattern radius=0pt, pattern color={rgb, 255:red, 222; green, 222; blue, 222}] (337.5,92) -- (427,92) -- (427,207) -- (337.5,207) -- cycle ;
	%Straight Lines [id:da9287130946355424] 
	\draw    (337.5,92) -- (337.5,207) ;
	%Straight Lines [id:da6685712932328518] 
	\draw    (257.5,94) -- (183.5,94) ;
	\draw [shift={(260.5,94)}, rotate = 180] [fill={rgb, 255:red, 0; green, 0; blue, 0 }  ][line width=0.08]  [draw opacity=0] (10.72,-5.15) -- (0,0) -- (10.72,5.15) -- (7.12,0) -- cycle    ;
	%Straight Lines [id:da7983240786429857] 
	\draw    (226,209.25) -- (185.5,209.25) ;
	\draw [shift={(185.5,209.25)}, rotate = 360] [color={rgb, 255:red, 0; green, 0; blue, 0 }  ][line width=0.75]    (0,5.59) -- (0,-5.59)   ;
	\draw [shift={(226,209.25)}, rotate = 360] [color={rgb, 255:red, 0; green, 0; blue, 0 }  ][line width=0.75]    (0,5.59) -- (0,-5.59)   ;

	% Text Node
	\draw (206,219) node    {$\lambda $};
	% Text Node
	\draw (217,56) node   [align=left] {vuoto};
	% Text Node
	\draw (387,56) node   [align=left] {materiale};


	\end{tikzpicture}
\end{figure}
\FloatBarrier
Sappiamo che nel caso di propagazione nel vuoto, la pulsazione $\omega$ è pari a $kc$. In presenza di un dielettrico, sarà pari all'indice $k$ moltiplicato per la velocità di propagazione sopra rilevata. Tuttavia sappiamo che quando un'onda attraversa il materiale, la sua pulsazione non può cambiare perché è legata alla frequenza con cui oscilla nel tempo. Ma il campo elettromagnetico può essere visto come un insieme di fotoni, ciascuno trasportante un'energia $E=hf$. Dal momento che, attraversando un materiale, l'energia di un fotone non varia, anche la frequenza deve rimanere la stessa. Ma se nel passaggio dal vuoto al materiale $\omega$ non può cambiare, deve cambiare $k$. Esso ci da una informazione sulla lunghezza d'onda: $ \lambda=2\pi /k $. Se la velocità di propagazione dal vuoto al materiale diminuisce, essendo $k$ inversamente proporzionale ad essa, $k$ aumenta. Ma se $k$ aumenta, la lunghezza d'onda, per come le due grandezze sono fra di loro legate, diventa più piccola
\[
	\omega = k\,v= k \frac{c}{\sqrt{\mu_r \varepsilon_r}} \implies \boxed{k=\frac{\omega}{c}\sqrt{\mu_r \varepsilon_r}}
\]
Quindi considerando la lunghezza d'onda
\[
	\lambda_{\text{materiale}} = \frac{2\pi}{k_{\text{materiale}}} = \frac{2\pi}{\frac{\omega}{c}\sqrt{\mu_r \varepsilon_r}} = \frac{2\pi c}{\omega \sqrt{\mu_r \varepsilon_r}}
\]
Possiamo a questo punto introdurre una nuova quantità a cui diamo il nome di indice di rifrazione. Esso sarà dato dal rapporto fra la velocità di un'onda elettromagnetica nel vuoto e in un mezzo in cui le onde possono propagarsi, cioè trasparente alle onde elettromagnetiche.
La velocità dell'onda nel materiale si scrive allora come:
\[
	v=\frac{c}{n} \implies \boxed{k=\frac{n\omega}{c}}
\]
E l'indice di rifrazione è dato da:
\[
	\boxed{n=\sqrt{\mu_r \varepsilon_r}}
\]
Per i mezzi trasparenti alle onde elettromagnetiche, $ \mu_r  $ assume valori molto prossimi a $1$:
\[
	\mu_r \simeq 1 \implies n \simeq  \sqrt{\varepsilon_r}
\]







































\section{Propagazione di un'onda elettromagnetica in un mezzo dielettrico. Dispersione}

In realtà il problema è molto più complesso di ciò che appare. Il valore di $ \varepsilon_r  $ che compare nella espressione dell'indice di rifrazione non è quello che si misura in regime stazionario (ovvero regime elettrostatico); al contrario esso dipende dalla frequenza $\omega$ dell'onda elettromagnetica (se si considerano onde sinusoidali). Dunque in generale avremo che la velocità di propagazione di un'onda elettromagnetica sinusoidale in un materiale dipende dalla sua frequenza:
\[
	v(\omega)=\frac{c}{n(\omega)}
\]
Questo fenomeno è detto dispersione perché un'onda elettromagnetica composta dalla sovrapposizione di onde di varia frequenza subirà un cambiamento nel corso della propagazione in quanto le sue componenti viaggeranno a differenti velocità. La dispersione è molto importante nel campo delle comunicazioni in fibra ottica.

Si noti che questo non è l'unico effetto che possiamo osservare in un dielettrico; in alcuni casi un'onda elettromagnetica che si propaga subirà una attenuazione; parleremo in tal caso di fenomeno di assorbimento, in quanto l'onda cederà parte della sua energia al materiale. L'intensità di un'onda elettromagnetica diminuisce all'aumentare della distanza di propagazione del materiale.
\begin{figure}[htpb]
	\centering
	

	\tikzset{every picture/.style={line width=0.75pt}} %set default line width to 0.75pt        

	\begin{tikzpicture}[x=0.75pt,y=0.75pt,yscale=-0.7,xscale=0.7]
	%uncomment if require: \path (0,300); %set diagram left start at 0, and has height of 300

	%Shape: Circle [id:dp1413380251931049] 
	\draw   (198.5,168.25) .. controls (198.5,110.12) and (245.62,63) .. (303.75,63) .. controls (361.88,63) and (409,110.12) .. (409,168.25) .. controls (409,226.38) and (361.88,273.5) .. (303.75,273.5) .. controls (245.62,273.5) and (198.5,226.38) .. (198.5,168.25) -- cycle ;
	%Curve Lines [id:da46707183707009703] 
	\draw [line width=1.5]    (14.5,131.75) .. controls (24.71,131.5) and (24.14,209.5) .. (34.5,209.75) .. controls (44.86,210) and (45.86,132.64) .. (55,132.75) .. controls (64.14,132.86) and (64.43,209.5) .. (75.5,209.25) .. controls (86.57,209) and (85.86,131.21) .. (96,131.25) .. controls (106.14,131.29) and (106.43,209.21) .. (116.5,209.25) .. controls (126.57,209.29) and (126.14,132.07) .. (136.5,132.25) .. controls (146.86,132.43) and (146.43,209.5) .. (157,209.75) .. controls (167.57,210) and (167.29,132.64) .. (177.5,132.75) ;
	%Straight Lines [id:da19574424723643502] 
	\draw    (147.5,114) -- (73.5,114) ;
	\draw [shift={(150.5,114)}, rotate = 180] [fill={rgb, 255:red, 0; green, 0; blue, 0 }  ][line width=0.08]  [draw opacity=0] (10.72,-5.15) -- (0,0) -- (10.72,5.15) -- (7.12,0) -- cycle    ;
	%Shape: Circle [id:dp5081045279397389] 
	\draw   (418.5,168.25) .. controls (418.5,110.12) and (465.62,63) .. (523.75,63) .. controls (581.88,63) and (629,110.12) .. (629,168.25) .. controls (629,226.38) and (581.88,273.5) .. (523.75,273.5) .. controls (465.62,273.5) and (418.5,226.38) .. (418.5,168.25) -- cycle ;

	% Text Node
	\draw  [fill={rgb, 255:red, 222; green, 222; blue, 222 }  ,fill opacity=1 ]  (303.75, 168.25) circle [x radius= 31.06, y radius= 31.06]   ;
	\draw (303.75,168.25) node  [font=\Huge]  {$+$};
	% Text Node
	\draw  [fill={rgb, 255:red, 222; green, 222; blue, 222 }  ,fill opacity=1 ]  (268.31, 101.69) circle [x radius= 13.6, y radius= 13.6]   ;
	\draw (268.31,101.69) node    {$-$};
	% Text Node
	\draw  [fill={rgb, 255:red, 222; green, 222; blue, 222 }  ,fill opacity=1 ]  (234.59, 139.72) circle [x radius= 13.6, y radius= 13.6]   ;
	\draw (234.59,139.72) node    {$-$};
	% Text Node
	\draw  [fill={rgb, 255:red, 222; green, 222; blue, 222 }  ,fill opacity=1 ]  (365.27, 156.94) circle [x radius= 13.6, y radius= 13.6]   ;
	\draw (365.27,156.94) node    {$-$};
	% Text Node
	\draw  [fill={rgb, 255:red, 222; green, 222; blue, 222 }  ,fill opacity=1 ]  (236.32, 189.86) circle [x radius= 13.6, y radius= 13.6]   ;
	\draw (236.32,189.86) node    {$-$};
	% Text Node
	\draw  [fill={rgb, 255:red, 222; green, 222; blue, 222 }  ,fill opacity=1 ]  (275.71, 232.49) circle [x radius= 13.6, y radius= 13.6]   ;
	\draw (275.71,232.49) node    {$-$};
	% Text Node
	\draw  [fill={rgb, 255:red, 222; green, 222; blue, 222 }  ,fill opacity=1 ]  (350.44, 109.73) circle [x radius= 13.6, y radius= 13.6]   ;
	\draw (350.44,109.73) node    {$-$};
	% Text Node
	\draw  [fill={rgb, 255:red, 222; green, 222; blue, 222 }  ,fill opacity=1 ]  (347.43, 227.59) circle [x radius= 13.6, y radius= 13.6]   ;
	\draw (347.43,227.59) node    {$-$};
	% Text Node
	\draw  [fill={rgb, 255:red, 222; green, 222; blue, 222 }  ,fill opacity=1 ]  (523.75, 228.25) circle [x radius= 31.06, y radius= 31.06]   ;
	\draw (523.75,228.25) node  [font=\Huge]  {$+$};
	% Text Node
	\draw  [fill={rgb, 255:red, 222; green, 222; blue, 222 }  ,fill opacity=1 ]  (488.31, 101.69) circle [x radius= 13.6, y radius= 13.6]   ;
	\draw (488.31,101.69) node    {$-$};
	% Text Node
	\draw  [fill={rgb, 255:red, 222; green, 222; blue, 222 }  ,fill opacity=1 ]  (454.59, 139.72) circle [x radius= 13.6, y radius= 13.6]   ;
	\draw (454.59,139.72) node    {$-$};
	% Text Node
	\draw  [fill={rgb, 255:red, 222; green, 222; blue, 222 }  ,fill opacity=1 ]  (599.27, 156.94) circle [x radius= 13.6, y radius= 13.6]   ;
	\draw (599.27,156.94) node    {$-$};
	% Text Node
	\draw  [fill={rgb, 255:red, 222; green, 222; blue, 222 }  ,fill opacity=1 ]  (500.32, 149.86) circle [x radius= 13.6, y radius= 13.6]   ;
	\draw (500.32,149.86) node    {$-$};
	% Text Node
	\draw  [fill={rgb, 255:red, 222; green, 222; blue, 222 }  ,fill opacity=1 ]  (529.71, 108.49) circle [x radius= 13.6, y radius= 13.6]   ;
	\draw (529.71,108.49) node    {$-$};
	% Text Node
	\draw  [fill={rgb, 255:red, 222; green, 222; blue, 222 }  ,fill opacity=1 ]  (570.44, 109.73) circle [x radius= 13.6, y radius= 13.6]   ;
	\draw (570.44,109.73) node    {$-$};
	% Text Node
	\draw  [fill={rgb, 255:red, 222; green, 222; blue, 222 }  ,fill opacity=1 ]  (553.43, 156.59) circle [x radius= 13.6, y radius= 13.6]   ;
	\draw (553.43,156.59) node    {$-$};


	\end{tikzpicture}
\end{figure}
\FloatBarrier
Per comprendere l'origine di questi due fenomeni è necessario ricorrere ad un modello semplificato della interazione fra un atomo di un dielettrico ed il campo elettromagnetico. Tenendo conto che gli elettroni sono molto più leggeri dei nuclei atomici, possiamo immaginare che siano i primi ad essere posti in moto da un'onda elettromagnetica incidente sull'atomo, mentre potremo assumere i nuclei sostanzialmente fissi. È utile a questo punto introdurre un modello meccanico dell'atomo, che vediamo come una nube di elettroni intorno ad un nucleo, in cui le forze agenti sull'elettrone sono:
\begin{itemize}
	\item \textbf{una forza di richiamo} (forza di Coulomb) che tende a riportare l'elettrone verso il nucleo quando se ne allontana. L'onda elettromagnetica è composta da campo elettrico e magnetico oscillante che incidono sull'atomo. Il campo elettrico tende a spostare gli elettroni in una direzione e il nucleo in una direzione opposta. Tale forza è in prima approssimazione elastica e la modellizzeremo come:
	\[
		\vec{F}_r= -k\,\vec{r}_e \qquad \vec{r}_e \quad \text{posizione dell'elettrone}
	\]
	\begin{figure}[htpb]
		\centering
		

		\tikzset{every picture/.style={line width=0.75pt}} %set default line width to 0.75pt        

		\begin{tikzpicture}[x=0.75pt,y=0.75pt,yscale=-1,xscale=1]
		%uncomment if require: \path (0,300); %set diagram left start at 0, and has height of 300

		%Straight Lines [id:da19276666803521425] 
		\draw    (156,116) -- (156,180) ;
		%Shape: Spring [id:dp8089892542429886] 
		\draw   (219,148) .. controls (220,142.5) and (223.5,137) .. (230.5,137) .. controls (244.5,137) and (244.5,159) .. (238.5,159) .. controls (232.5,159) and (232.5,137) .. (246.5,137) .. controls (260.5,137) and (260.5,159) .. (254.5,159) .. controls (248.5,159) and (248.5,137) .. (262.5,137) .. controls (276.5,137) and (276.5,159) .. (270.5,159) .. controls (264.5,159) and (264.5,137) .. (278.5,137) .. controls (292.5,137) and (292.5,159) .. (286.5,159) .. controls (280.5,159) and (280.5,137) .. (294.5,137) .. controls (299.49,137) and (302.71,139.8) .. (304.5,143.4) ;
		%Straight Lines [id:da10038793636155963] 
		\draw    (220,148) -- (156,148) ;
		%Straight Lines [id:da6277949140980068] 
		\draw    (368,144) -- (304,144) ;
		%Straight Lines [id:da9775807978954678] 
		\draw    (368,124) -- (307,124) ;
		\draw [shift={(304,124)}, rotate = 360] [fill={rgb, 255:red, 0; green, 0; blue, 0 }  ][line width=0.08]  [draw opacity=0] (10.72,-5.15) -- (0,0) -- (10.72,5.15) -- (7.12,0) -- cycle    ;

		% Text Node
		\draw  [fill={rgb, 255:red, 222; green, 222; blue, 222 }  ,fill opacity=1 ]  (380.59, 144.72) circle [x radius= 13.6, y radius= 13.6]   ;
		\draw (380.59,144.72) node    {$-$};
		% Text Node
		\draw (336,105) node    {$\vec{F}_{r}$};


		\end{tikzpicture}
	\end{figure}
	\FloatBarrier
	\item \textbf{una forza dissipativa} che tenga conto della perdita di energia per urti con atomi adiacenti e per emissione di radiazione elettromagnetica, la modellizziamo come una forza di attrito viscoso:
	\[
		\vec{F}_{av} = -b\,\vec{v}_e \qquad \vec{v}_e=\frac{d\vec{r}_e}{dt} \quad \text{velocità dell'elettrone}
	\]
	\item \textbf{una forza che descrive l'interazione tra l'elettrone e l'onda elettromagnetica incidente}, trascureremo gli effetti della componente magnetica (forza di Lorentz) e quindi porremo:
	\[
		\vec{F}_{em} = -e\,\vec{E} (t)
	\]
\end{itemize}
Dove abbiamo tralasciato la dipendenza spaziale di $\vec{E}$ in quanto l'atomo è in una posizione fissa e supporremo che l'elettrone si sposti di poco rispetto alla lunghezza d'onda della radiazione.
A questo punto possiamo applicare il secondo principio della dinamica. L'equazione del moto dell'elettrone sarà:
\begin{equation*}
	\begin{aligned}
		\vec{R} = \vec{F}_r + \vec{F}_{av} + \vec{F}_{em} &= m_e \, \frac{d^2 \vec{r}_e}{dt^2} \\
		-k\,\vec{r}_e -b\,\vec{v}_e -e\,\vec{E} (t) &= m_e \, \frac{d^2 \vec{r}_e}{dt^2} \\
		-\frac{k\,\vec{r}_e}{m_e} - \frac{b\,\vec{v}_e}{m_e} -\frac{e\,\vec{E} (t)}{m_e} &= \frac{d^2 \vec{r}_e}{dt^2} \implies \frac{d^2 \vec{r}_e}{dt^2} + \frac{k\,\vec{r}_e}{m_e} + \frac{b\,\vec{v}_e}{m_e} + \frac{e\,\vec{E} (t)}{m_e} = 0 \\
	\end{aligned}
\end{equation*}
Chiamiamo coefficiente di attrito viscoso, la quantità: $ \gamma = \frac{b}{m_e} $. Sia inoltre $ \omega_0^2 = \frac{k}{m_e}  $. Otteniamo la seguente equazione del moto
\begin{equation}
	\boxed{\frac{\partial^2 \vec{r}_e}{\partial t^2} -\omega_0^2 \,\vec{r}_e + \gamma \frac{\partial \vec{r}_e}{\partial t} + \frac{e}{m_e}\vec{E} (t) = 0}
	\label{eq:oscillatoreAtomico}
\end{equation}
Notiamo che se non ci fosse attrito viscoso otterremmo proprio l'equazione dell'oscillatore armonico
\[
	\frac{\partial^2 \vec{r}_e}{\partial t^2} + \underbrace{\frac{k}{m}}_{\omega_0^2} \vec{r}_e = 0
\]
Grazie a questa osservazione, capiamo che la situazione generale descritta dalla \eqref{eq:oscillatoreAtomico} è quella di un oscillatore atomico di pulsazione propria $ \omega_0  $, la cui oscillazione è smorzata da una forza di attrito viscoso che riassume l'effetto dell'irraggiamento e di una eventuale interazione con gli atomi circostanti.

Risolveremo la \eqref{eq:oscillatoreAtomico} assumendo che l'onda elettromagnetica sia sinusoidale con pulsazione generica $\omega$, dunque ponendo
\[
	\vec{E} (t) = \vec{E}_0 \cos (\omega t)= \frac{\vec{E}_0}{2}[e^{-i\omega t}+e^{i\omega t}  ]  = \frac{\vec{E}_0}{2}e^{-i\omega t} +\frac{\vec{E}_0}{2}e^{i\omega t}
\]
Essendo l'elettrone forzato ad oscillare con la stessa pulsazione di $E(t)$, poniamo analogamente:
\[
	\vec{r}_e = \vec{r}_0 \cos (\omega t) = \frac{\vec{r}_0}{2}[e^{-i\omega t}+e^{i\omega t}  ]  = \frac{\vec{r}_0}{2}e^{-i\omega t} +\frac{\vec{r}_0}{2}e^{i\omega t}
\]
Derivandolo due volte
\begin{equation*}
	\begin{aligned}
		\frac{d\vec{r}_e}{dt} &= -i\omega \frac{\vec{r}_0}{2}e^{-i\omega t} + c.c. \\
		\frac{d^2 \vec{r}_e}{dt^2} &= (-i\omega)^2 \frac{\vec{r}_0}{2}e^{-i\omega t} + c.c. = - \frac{\omega^2\vec{r}_0}{2}e^{-i\omega t} + c.c.
	\end{aligned}
\end{equation*}
Sostituendo questa espressione nell'equazione \eqref{eq:oscillatoreAtomico}
\begin{gather*}
	\left[ -\frac{\omega^2 \vec{r}_0}{2} +\frac{\omega^2 \vec{r}_0}{2} - \frac{i\omega \gamma \vec{r}_0}{2} + \frac{e}{m_e}\frac{\vec{E}_0}{2} \right] e^{-i\omega t} + \\
	+ \left[ -\frac{\omega^2 \vec{r}_0}{2} +\frac{\omega^2 \vec{r}_0}{2} - \frac{i\omega \gamma \vec{r}_0}{2} + \frac{e}{m_e}\frac{\vec{E}_0}{2} \right] e^{i\omega t} =0
\end{gather*}
Perché l'equazione sia sodisfatta, è necessario che il termine nelle parentesi quadre si annulli.
\begin{equation*}
	\begin{aligned}
		-\omega^2 \vec{r}_0 + \omega_0 ^2 \vec{r}_0 - i\gamma \omega \vec{r}_0 + \frac{e}{m_e}\vec{E}_0 &=0 \\
		\vec{r}_0[\omega_0^2 -\omega^2  - i\gamma \omega] + \frac{e}{m_e} \vec{E}_0 &=0 \\
		\vec{r}_0 = -\frac{e\vec{E}_0}{[\omega_0^2 -\omega^2  - i\gamma \omega]m_e} \\
		\vec{r}_e(t) = \frac{\vec{r}_0}{2} (e^{-i\omega t}+e^{i\omega t}  ) &= -\frac{e\vec{E}_0}{2[\omega_0^2 -\omega^2  - i\gamma \omega]m_e} (e^{-i\omega t}+e^{i\omega t}  )
	\end{aligned}
\end{equation*}
Abbiamo così trovato la soluzione della \eqref{eq:oscillatoreAtomico}, che rappresenta l'equazione del moto dell'elettrone oscillante in un atomo, quando eccitato da un'onda elettromagnetica. Poiché l'elettrone oscilla a causa del campo $\vec{E}$, l'atomo presenterà un momento di dipolo oscillante pari a: $ \vec{p} = -e \vec{r}_e  $. In un materiale con $N$ atomi per unità di volume, definiamo il vettore polarizzazione come:
\begin{equation*}
	\begin{aligned}
		\vec{P} = N\vec{p} =-Ne\,\vec{r}_e &= -N e \left( -\frac{\vec{r}_0}{2}(e^{-i\omega t}+e^{i\omega t}) \right) \\
		&= \frac{N\,e^2 \vec{E}_0}{2[\omega_0^2-\omega^2 -i\gamma \omega]m_e}(e^{-i\omega t}+e^{i\omega t}) \\
		&= \frac{\vec{P}_0}{2} (e^{-i\omega t}+e^{i\omega t})
	\end{aligned}
\end{equation*}
Sarà questo il vettore polarizzazione $\vec{P}$ in un mezzo dielettrico composto da atomi identici con densità per unità di volume $N$.
Ricordiamo la relazione $ \vec{P} = \varepsilon_0 (\varepsilon_r -1)\vec{E}  $
Essa era stata introdotta per i campi elettrostatici. Estendendola anche ai campi (e di conseguenza alla polarizzazione) oscillanti nel tempo, avremo:
\begin{gather*}
	\vec{P}_0 = \frac{Ne^2 \,\vec{E}_0}{[\omega_0^2-\omega^2 -i\gamma \omega]m_e} = \varepsilon_0 \left[ \varepsilon_r (\omega) -1 \right] \vec{E}_0 \\
	\boxed{\varepsilon_r (\omega) = 1 + \frac{Ne^2}{m_e\,\varepsilon_0\,[\omega_0^2-\omega^2 -i\gamma \omega]}}
\end{gather*}
Moltiplicando nominatore e denominatore per il complesso coniugato:
\[
	\varepsilon_r (\omega) = 1 + \frac{Ne^2 \left[ (\omega_0^2 -\omega^2) + i\gamma\omega  \right]}{m_e\,\varepsilon_0\,\left[ (\omega_0^2 -\omega^2)^2 +\gamma^2 \omega^2 \right]}
\]
Possiamo definire un indice di rifrazione complesso che sarà la radice quadrata della quantità appena ricavata.
\[
	n = \sqrt{\varepsilon_r (\omega)}
\]
Per calcolarla possiamo ricorrere a delle approssimazioni basata sull'idea che il dielettrico considerato sia rarefatto. In tal caso:
\[
	n=\sqrt{\varepsilon_r (\omega)} = \sqrt{1 + \underbrace{\frac{Ne^2 \left[ (\omega_0^2 -\omega^2) + i\gamma\omega  \right]}{m_e\,\varepsilon_0\,\left[ (\omega_0^2 -\omega^2)^2 +\gamma^2 \omega^2 \right]}}_{\to 0}}
\]
In una situazione di questo tipo, in cui abbiamo una funzione della forma
\[
	f(x) = \sqrt{1+x} \qquad x\to 0
\]
Possiamo attuare la seguente approssimazione:
\[
	f(x) = 1 + \frac{x}{2} + o(x)
\]
L'indice di rifrazione sarà allora dato da:
\[
	n \simeq 1 + \frac{Ne^2 \left[ (\omega_0^2 -\omega^2) + i\gamma\omega  \right]}{2\,m_e\,\varepsilon_0\,\left[ (\omega_0^2 -\omega^2)^2 +\gamma^2 \omega^2 \right]} = n_R + i\,n_I
\]
Con
\begin{align*}
	n_R &= 1 + \frac{Ne^2}{2m_e\,\varepsilon_0}\frac{\omega_0^2 -\omega^2}{\left[ (\omega_0^2 -\omega^2)^2 +\gamma^2 \omega^2 \right]} \\
	n_I &= 1 + \frac{Ne^2}{2m_e\,\varepsilon_0}\frac{\gamma \omega}{\left[ (\omega_0^2 -\omega^2)^2 +\gamma^2 \omega^2 \right]}
\end{align*}
Per comprendere il significato di quanto trovato, rianalizziamo l'equazione delle onde, che aveva comportato nel caso di onde piane la soluzione:
\begin{equation*}
	\begin{aligned}
		\vec{E} (x,t) &= \frac{\vec{E}_0}{2}e^{i(kx-\omega t)} + \frac{\vec{E}_0}{2}e^{-i(kx-\omega t)} \qquad k = \frac{n\omega}{c} \\
		&=\frac{\vec{E}_0}{2}e^{i\left( \frac{\omega}{c}n_Rx + \frac{\omega}{c}n_Ix - \omega t   \right)} + c.c.
	\end{aligned}
\end{equation*}
La sostituzione operata su $k$ proviene dalle seguenti considerazioni:
\[
	\omega = kv \qquad k = \frac{\omega}{v} = \frac{\omega}{c/n} = \frac{n\omega}{c} = \frac{\omega}{c} [n_R + i\,n_I]
\]
\emph{Nota}: per semplicità di scrittura indichiamo con $c.c.$ il complesso coniugato del termine appena precedente.
Sarà possibile scrivere l'onda nella forma:
\begin{equation*}
	\begin{aligned}
		\vec{E} (x,t) &= \frac{\vec{E}_0}{2}e^{i\left[ \frac{\omega}{c}n_R x-\omega t \right]} e^{-\frac{\omega}{c}n_I x} + c.c. \\
		&= \frac{\vec{E}_0}{2} e^{i[k_R(\omega) x - \omega t ]} e^{-\alpha(\omega)\,x} + c.c.
	\end{aligned}
\end{equation*}
abbiamo posto:
\begin{gather*}
	k_R(\omega) = \frac{\omega}{c} n_R(\omega) \\
	\alpha (\omega) = \frac{\omega}{c} n_I(\omega)
\end{gather*}
l'espressione del campo elettrico può allora essere scritta come:
\[
	\boxed{\vec{E} (x,t) = \vec{E}_0 e^{-\alpha x} \cos [k_R(\omega)x-\omega t]}
\]
Dove compare un'espressione più generale del numero d'onda $k$ ed un termine esponenziale decrescente
\[
	\vec{E}_0\,e^{-\alpha x}
\]
Che dunque comporta un'alterazione dell'onda durante la propagazione. Si vede che la sua intensità diminuisce esponenzialmente con la distanza, indicando un assorbimento di energia da parte del dielettrico. Anche l'intensità media, proporzionale al quadrato dell'ampiezza, diminuisce progressivamente con legge esponenziale:
\[
	I(x) = I(0)\,e^{-2\alpha x} = I(0)\,e^{-\beta x}
\]
\begin{figure}[htpb]
	\centering
	

	\tikzset{every picture/.style={line width=0.75pt}} %set default line width to 0.75pt        

	\begin{tikzpicture}[x=0.75pt,y=0.75pt,yscale=-1,xscale=1]
	%uncomment if require: \path (0,300); %set diagram left start at 0, and has height of 300

	%Shape: Axis 2D [id:dp8573669816086147] 
	\draw  (174,225) -- (481.5,225)(194.5,51) -- (194.5,241) (474.5,220) -- (481.5,225) -- (474.5,230) (189.5,58) -- (194.5,51) -- (199.5,58)  ;
	%Curve Lines [id:da0876937084718834] 
	\draw [line width=1.5]    (194.5,74) .. controls (205.5,177) and (208.5,218) .. (274.5,221) .. controls (340.5,224) and (366.5,221) .. (407.5,222) ;

	% Text Node
	\draw (168,53) node    {$I( x)$};
	% Text Node
	\draw (495,223) node    {$x$};


	\end{tikzpicture}
\end{figure}
\FloatBarrier
$\beta$ prende il nome di \textbf{coefficiente di assorbimento}. Il suo inverso, $\frac{1}{\beta}$, è la \textbf{lunghezza di assorbimento} e ha il solito significato: dopo una distanza $\frac{1}{\beta}$ nel dielettrico l'intensità si è ridotta al valore $I(0)e^{-1}$.
In conclusione, la dispersione durante la propagazione dell'onda nel mezzo è descritta completamente dalla parte reale $n_R$ dell'indice di rifrazione complesso, mentre la parte immaginaria $n_I$ determina il coefficiente di assorbimento ed è quindi legata alla diminuzione dell'intensità dell'onda.
\begin{figure}[htpb]
	\centering
	

	\tikzset{every picture/.style={line width=0.75pt}} %set default line width to 0.75pt        

	\begin{tikzpicture}[x=0.75pt,y=0.75pt,yscale=-1,xscale=1]
	%uncomment if require: \path (0,300); %set diagram left start at 0, and has height of 300

	%Shape: Axis 2D [id:dp7607986096735857] 
	\draw  (173.5,244.41) -- (457,244.41)(192.4,64) -- (192.4,261) (450,239.41) -- (457,244.41) -- (450,249.41) (187.4,71) -- (192.4,64) -- (197.4,71)  ;
	%Curve Lines [id:da5227641534353398] 
	\draw [line width=1.5]    (192.67,105.17) .. controls (220.33,102.33) and (248.5,80.75) .. (257,80.75) .. controls (274,81.25) and (279,156.75) .. (305,155.75) .. controls (321.67,155.75) and (352,137.75) .. (408,136.75) ;
	%Curve Lines [id:da08841698827363254] 
	\draw [line width=1.5]    (192.57,244.41) .. controls (266.5,244.75) and (264,174.75) .. (279,175.25) .. controls (295,175.25) and (293,244.75) .. (409.33,243.83) ;
	%Straight Lines [id:da9091818157433591] 
	\draw  [dash pattern={on 0.84pt off 2.51pt}]  (279,122.75) -- (279,244.75) ;
	%Straight Lines [id:da9890017011677661] 
	\draw  [dash pattern={on 0.84pt off 2.51pt}]  (408,122.75) -- (193,122.75) ;

	% Text Node
	\draw (468,243) node    {$\omega $};
	% Text Node
	\draw (181,119.5) node    {$1$};
	% Text Node
	\draw (278,73) node    {$n_{R}$};
	% Text Node
	\draw (278.5,253) node    {$\omega _{0}$};
	% Text Node
	\draw (337,216) node    {$n_{I}$};


	\end{tikzpicture}
\end{figure}
\FloatBarrier
In corrispondenza del massimo assorbimento c'è anche un cambiamento di velocità.







































\section{Equazioni di Maxwell in presenza di sorgenti. Potenziali elettrodinamici}

Finora ci siamo occupati delle equazioni di Maxwell e della loro risoluzione in assenza di sorgenti. Vogliamo quindi ora occuparci del caso più generale in cui sono presenti sorgenti variabili nel tempo. Questa situazione si può riscrivere tramite i potenziali elettrodinamici. Ipotizziamo di essere nel vuoto. Le equazioni saranno date da:
\begin{align}
	\text{div}\vec{E} &=\frac{\rho (t)}{\varepsilon_0} \label{eq:max1}\\
	\text{rot}\vec{E} &= -\frac{\partial \vec{B}}{\partial t} \label{eq:max2} \\
	\text{div}\vec{B} &= 0 \label{eq:max3}\\
	\text{rot}\vec{B} &= \mu_0 \left[ \vec{J} (t) + \varepsilon_0 \frac{\partial \vec{E}}{\partial t}  \right] \label{eq:max4}
\end{align}
Si dimostra dall'equazione \eqref{eq:max3} che possiamo sempre scrivere il campo magnetico $\vec{B}$ come il rotore di un certo potenziale vettore $\vec{A}$, $ \vec{B} = \text{rot}\vec{A} $.
Applicando poi l'operatore divergenza e si trova:
\[
	\text{div}\vec{B} = \text{div}(\text{rot}\vec{A} ) = \vec{\nabla} \cdot (\vec{\nabla} \times \vec{A} ) = 0
\]
Sostituendo l'espressione di $\vec{B}$ nella seconda equazione:
\[
	\vec{\nabla} \times \vec{E} = -\frac{\partial}{\partial t}(\underbrace{\vec{\nabla} \times \vec{A}}_{\vec{B}} ) = \vec{\nabla} \times \left( -\frac{\partial \vec{A}}{\partial t}  \right) \implies \vec{\nabla} \times \left( \vec{E} +\frac{\partial \vec{A}}{\partial t}  \right) =0
\]
L'ultima espressione afferma che il vettore fra parentesi è conservativo e dunque è possibile introdurre un potenziale scalare $V$ tale che
\[
	\vec{E} +\frac{\partial \vec{A}}{\partial t} = - \vec{\nabla} V
\]
Quindi abbiamo
\[
	\boxed{\vec{B} = \vec{\nabla} \times \vec{A} \qquad \vec{E} =-\vec{\nabla} V-\frac{\partial \vec{A}}{\partial t}}
\]
Questi potenziali li abbiamo definiti partendo da due equazioni che non dipendono dalle sorgenti, quindi valgono in qualsiasi situazione. Vogliamo sostituire questi due potenziale nelle equazioni rimanenti. Nella \eqref{eq:max1}:
\begin{equation*}
	\begin{aligned}
		\vec{\nabla} \cdot \vec{E} =\vec{\nabla} \cdot \left( -\vec{\nabla} V-\frac{\partial \vec{A}}{\partial t}  \right) &= \frac{\rho (t)}{\varepsilon_0} \\
		- \nabla^2 V - \vec{\nabla} \cdot \left( \frac{\partial \vec{A}}{\partial t}  \right) &= \frac{\rho (t)}{\varepsilon_0} \implies
		\nabla^2 V + \vec{\nabla} \cdot \frac{\partial \vec{A}}{\partial t} &= -\frac{\rho (t)}{\varepsilon_0}
	\end{aligned}
\end{equation*}
Se fossimo al regime stazionario, ritorneremmo all'equazione di Poisson già incontrata. A questo punto dall'equazione \eqref{eq:max4} ricaviamo:
\begin{equation*}
	\begin{aligned}
		\vec{\nabla} \times \underbrace{\vec{\nabla} \times \vec{A}}_{\vec{B}}  &= \mu_0 \left[ \vec{J} (t) + \varepsilon_0 \frac{\partial}{\partial t} \overbrace{\left(-\vec{\nabla} V - \frac{\partial \vec{A}}{\partial t}\right)}^{\vec{E}}   \right] \\
		\vec{\nabla} (\vec{\nabla} \cdot \vec{A} ) - \nabla^2 \vec{A} &= \mu_0 \vec{J} (t) + \underbrace{\mu_0 \varepsilon_0}_{1/c^2} \left( -\frac{\partial \vec{\nabla} V}{\partial t} -\frac{\partial^2 \vec{A}}{\partial t^2}  \right)  \\
		\nabla^2 \vec{A} -\frac{1}{c^2}\frac{\partial^2 \vec{A}}{\partial t^2} -\frac{1}{c^2}\frac{\partial \vec{\nabla} V}{\partial t} - \vec{\nabla} (\vec{\nabla} \cdot \vec{A} ) &= -\mu_0 \vec{J} (t) \\
		\nabla^2 \vec{A} -\frac{1}{c^2}\frac{\partial^2 \vec{A}}{\partial t^2} - \vec{\nabla} \left(\frac{1}{c^2}\frac{\partial V}{\partial t} + \vec{\nabla} \cdot \vec{A} \right) &= -\mu_0 \vec{J} (t) \\
	\end{aligned}
\end{equation*}
Abbiamo così ottenuto due equazioni differenziali nei soli potenziali $\vec{A}$ e $V$ che dipendono dalle sorgenti note. Si osservi che il numero di equazioni scalari e di funzioni incognite è il medesimo. Tuttavia le equazioni non sono disaccoppiate, per disaccoppiarle si sfrutta una libertà nella scelta delle funzioni incognite $\vec{A}$ e $V$ detta \textbf{invarianza di gauge}, che pur trasformando $\vec{A}$ e $V$ in nuove funzioni $\vec{A'}$ e $V'$, mantiene inalterati i valori di $\vec{E}$ e di $\vec{B}$ a cui siamo in realtà interessati.

Infatti, scegliendo una funzione $f$ qualsiasi purché derivabile, poniamo di ottenere nuovi potenziali a partire dai precedenti con le trasformazioni:
\[
	\boxed{\vec{A} \to \vec{A}' = \vec{A} +\vec{\nabla} \vec{f} \qquad V\to V' = V-\frac{\partial f}{\partial t}}
\]
Pertanto le trasformazioni proposte non alternano i valori di $\vec{E}$ e di $ \vec{B}  $, ma possono consentire di riscrivere le equazioni dei potenziali in forma opportuna. Sceglieremo allora una opportuna trasformazione che comporti il seguente legame:
\[
	\boxed{\frac{1}{c^2}\frac{\partial V'}{\partial t} + \vec{\nabla} \cdot \vec{A'} = 0}
\]
Questa condizione è anche nota come \textbf{condizione di Lorentz}. Pur non dimostrandone la fattibilità, osserviamo che esiste una opportuna funzione f che consente di scrivere tale relazione. Dalla condizione di Lorentz, le due equazioni nei potenziali diventano:
\begin{equation}
	\boxed{\nabla^2 \vec{A} -\frac{1}{c^2}\frac{\partial^2 \vec{A}}{\partial t^2} = -\mu_0 \vec{J} (t)}
\end{equation}
Ricordiamo ora che:
\[
	\frac{1}{c^2}\frac{\partial V'}{\partial t} + \vec{\nabla} \cdot \vec{A'} = 0 \implies  \vec{\nabla} \cdot \vec{A} = - \frac{1}{c^2}\frac{\partial V}{\partial t}
\]
Quindi:
\begin{equation*}
	\begin{aligned}
		\nabla^2 V + \vec{\nabla} \cdot \frac{\partial \vec{A}}{\partial t} &= -\frac{\rho (t)}{\varepsilon_0} \\
		\nabla^2 V + \frac{\partial}{\partial t} (\vec{\nabla} \cdot \vec{A}) &= -\frac{\rho (t)}{\varepsilon_0} \\
		\nabla^2 V + \frac{\partial}{\partial t} \left( - \frac{1}{c^2}\frac{\partial V}{\partial t} \right)   &= -\frac{\rho (t)}{\varepsilon_0} \\
	\end{aligned}
\end{equation*}
\begin{equation}
	\boxed{\nabla^2 V  - \frac{1}{c^2}\frac{\partial^2 V}{\partial t^2} = -\frac{\rho (t)}{\varepsilon_0}}
\end{equation}
In tal modo le due equazioni si disaccoppiano, perché non compaiono più i due potenziali nella stessa equazione. Se fossimo in regime stazionario le equazioni diventerebbero semplicemente:
\[
	\nabla^2 V  = -\frac{\rho (t)}{\varepsilon_0} \qquad \nabla^2 \vec{A} = -\mu_0 \vec{J} (t)
\]
La prima è l'equazione di Poisson e la seconda è stata ricavata in magnetostatica. Quindi le due equazioni comprendono tutti i casi e si possono estendere anche a quello stazionario. Avendo introdotto l'operatore d'alembertiano, possiamo riscrivere le due equazioni semplicemente come:
\[
	\boxed{\left\{ \begin{array}{l}
		 	\Box \vec{A} = -\mu_0 \vec{J} (t) \\
		 	\Box \vec{V} = -\frac{\rho (t)}{\varepsilon_0}
		\end{array} \right.}
\]
La soluzione delle due equazioni è anche essa simile a quanto trovato nel caso stazionario, ma tiene conto della variazione temporale delle sorgenti.
Supponiamo di avere un certo volume $\tau$. In ogni cubetto $d\tau$ vi sono cariche e correnti che variano nel tempo.
\begin{figure}[htpb]
	\centering
	

	\tikzset{every picture/.style={line width=0.75pt}} %set default line width to 0.75pt        

	\begin{tikzpicture}[x=0.75pt,y=0.75pt,yscale=-1,xscale=1]
	%uncomment if require: \path (0,300); %set diagram left start at 0, and has height of 300

	%Straight Lines [id:da6797463639029115] 
	\draw    (216.5,185.67) -- (216.5,105.67) ;
	\draw [shift={(216.5,102.67)}, rotate = 450] [fill={rgb, 255:red, 0; green, 0; blue, 0 }  ][line width=0.08]  [draw opacity=0] (10.72,-5.15) -- (0,0) -- (10.72,5.15) -- (7.12,0) -- cycle    ;
	%Straight Lines [id:da15998144999629993] 
	\draw    (216.5,185.67) -- (276.93,222.12) ;
	\draw [shift={(279.5,223.67)}, rotate = 211.1] [fill={rgb, 255:red, 0; green, 0; blue, 0 }  ][line width=0.08]  [draw opacity=0] (10.72,-5.15) -- (0,0) -- (10.72,5.15) -- (7.12,0) -- cycle    ;
	%Straight Lines [id:da31848190328869985] 
	\draw    (216.5,185.67) -- (157.06,222.1) ;
	\draw [shift={(154.5,223.67)}, rotate = 328.5] [fill={rgb, 255:red, 0; green, 0; blue, 0 }  ][line width=0.08]  [draw opacity=0] (10.72,-5.15) -- (0,0) -- (10.72,5.15) -- (7.12,0) -- cycle    ;
	%Shape: Circle [id:dp1661340389317938] 
	\draw  [fill={rgb, 255:red, 0; green, 0; blue, 0 }  ,fill opacity=1 ] (401.17,224.42) .. controls (401.17,223.17) and (402.17,222.17) .. (403.42,222.17) .. controls (404.66,222.17) and (405.67,223.17) .. (405.67,224.42) .. controls (405.67,225.66) and (404.66,226.67) .. (403.42,226.67) .. controls (402.17,226.67) and (401.17,225.66) .. (401.17,224.42) -- cycle ;
	%Straight Lines [id:da9318394165074981] 
	\draw    (216.5,185.67) -- (420.51,112.02) ;
	\draw [shift={(423.33,111)}, rotate = 520.15] [fill={rgb, 255:red, 0; green, 0; blue, 0 }  ][line width=0.08]  [draw opacity=0] (10.72,-5.15) -- (0,0) -- (10.72,5.15) -- (7.12,0) -- cycle    ;
	%Straight Lines [id:da5083602993568179] 
	\draw    (216.5,185.67) -- (398.23,223.8) ;
	\draw [shift={(401.17,224.42)}, rotate = 191.85] [fill={rgb, 255:red, 0; green, 0; blue, 0 }  ][line width=0.08]  [draw opacity=0] (10.72,-5.15) -- (0,0) -- (10.72,5.15) -- (7.12,0) -- cycle    ;
	%Straight Lines [id:da6962376677301336] 
	\draw    (436,122.33) -- (404.35,219.31) ;
	\draw [shift={(403.42,222.17)}, rotate = 288.08] [fill={rgb, 255:red, 0; green, 0; blue, 0 }  ][line width=0.08]  [draw opacity=0] (10.72,-5.15) -- (0,0) -- (10.72,5.15) -- (7.12,0) -- cycle    ;
	%Shape: Circle [id:dp25731616548756575] 
	\draw   (376.33,119.5) .. controls (376.33,79.83) and (408.49,47.67) .. (448.17,47.67) .. controls (487.84,47.67) and (520,79.83) .. (520,119.5) .. controls (520,159.17) and (487.84,191.33) .. (448.17,191.33) .. controls (408.49,191.33) and (376.33,159.17) .. (376.33,119.5) -- cycle ;
	%Shape: Cube [id:dp047324299810165726] 
	\draw   (428.17,102.83) -- (433.17,97.83) -- (453.17,97.83) -- (453.17,114.5) -- (448.17,119.5) -- (428.17,119.5) -- cycle ; \draw   (453.17,97.83) -- (448.17,102.83) -- (428.17,102.83) ; \draw   (448.17,102.83) -- (448.17,119.5) ;

	% Text Node
	\draw (154.33,233.67) node    {$x$};
	% Text Node
	\draw (286.5,232.17) node    {$y$};
	% Text Node
	\draw (203,99.67) node    {$z$};
	% Text Node
	\draw (430.47,203.4) node    {$\vec{J}( t)$};
	% Text Node
	\draw (339.67,192.83) node    {$\vec{r}$};
	% Text Node
	\draw (310,132.17) node    {$\vec{r}_{i}$};
	% Text Node
	\draw (419.83,241.33) node    {$P( x ,y,z)$};
	% Text Node
	\draw (439.97,85.27) node    {$d\tau $};
	% Text Node
	\draw (480.17,170.07) node    {$\tau $};
	% Text Node
	\draw (491.17,115.67) node    {$\rho ( t)$};
	% Text Node
	\draw (553.17,218.87) node    {$\vec{r} -\vec{r}_{i} =\Delta r$};


	\end{tikzpicture}
\end{figure}
\FloatBarrier
Si dimostra che in tale situazione le soluzioni alle due equazioni sono date da:
\begin{gather*}
	\boxed{V(\vec{r},t) = \int_{\tau}\frac{\rho (\vec{r'}, t - \frac{\Delta r}{c})}{4\pi \varepsilon_0 \, \Delta r}d\tau} \\
	\boxed{\vec{A} (\vec{r},t) = \int_{\tau}\frac{\mu_0 \vec{J} (\vec{r'}, t - \frac{\Delta r}{c})}{4\pi \, \Delta r}d\tau} \\
\end{gather*}
I potenziali così ricavati mantengono la stessa struttura del caso stazionario, ma le sorgenti vanno valutate non nello stesso istante t in cui valutiamo i potenziali nel punto $P(r)$, bensì in un istante precedente $ t'=t-\frac{\Delta \vec{r}}{c} $, per tener conto del ritardo di propagazione delle informazioni dalla sorgente al punto $P$. Per questo motivo tali quantità sono anche dette \emph{potenziali ritardati}, si noti che quanto esposto rappresenta solo un integrale particolare delle equazioni per i potenziali; l'integrale generale si ottiene come combinazione lineare di quello particolare con la soluzione delle rispettive equazioni omogenee, che corrispondono alla propagazione di onde elettromagnetiche in assenza di sorgenti.







































\section{Campo elettromagentico prodotto da un bipolo elettrico oscillante}

Consideriamo il caso di un dipolo che oscilla sinusoidalmente con pulsazione $\omega$. Vorremmo determinare, sfruttando le equazioni dei potenziali ritardati, l'andamento della radiazione elettromagnetica emessa dal dipolo.
\begin{figure}[htpb]
	\centering
	

	\tikzset{every picture/.style={line width=0.75pt}} %set default line width to 0.75pt        

	\begin{tikzpicture}[x=0.75pt,y=0.75pt,yscale=-1,xscale=1]
	%uncomment if require: \path (0,300); %set diagram left start at 0, and has height of 300

	%Straight Lines [id:da05818592277158685] 
	\draw    (240.5,257) -- (240.5,54.67) ;
	\draw [shift={(240.5,51.67)}, rotate = 450] [fill={rgb, 255:red, 0; green, 0; blue, 0 }  ][line width=0.08]  [draw opacity=0] (10.72,-5.15) -- (0,0) -- (10.72,5.15) -- (7.12,0) -- cycle    ;
	%Straight Lines [id:da3710797089764373] 
	\draw    (244.5,221.33) -- (455.78,91.24) ;
	\draw [shift={(458.33,89.67)}, rotate = 508.38] [fill={rgb, 255:red, 0; green, 0; blue, 0 }  ][line width=0.08]  [draw opacity=0] (10.72,-5.15) -- (0,0) -- (10.72,5.15) -- (7.12,0) -- cycle    ;
	%Shape: Can [id:dp6034508492628499] 
	\draw   (256.33,96.42) -- (256.33,216.58) .. controls (256.33,219.21) and (249.24,221.33) .. (240.5,221.33) .. controls (231.76,221.33) and (224.67,219.21) .. (224.67,216.58) -- (224.67,96.42) .. controls (224.67,93.79) and (231.76,91.67) .. (240.5,91.67) .. controls (249.24,91.67) and (256.33,93.79) .. (256.33,96.42) .. controls (256.33,99.04) and (249.24,101.17) .. (240.5,101.17) .. controls (231.76,101.17) and (224.67,99.04) .. (224.67,96.42) ;
	%Straight Lines [id:da3866371667239994] 
	\draw    (244.5,221.33) -- (244.5,154.33) ;
	\draw [shift={(244.5,151.33)}, rotate = 450] [fill={rgb, 255:red, 0; green, 0; blue, 0 }  ][line width=0.08]  [draw opacity=0] (10.72,-5.15) -- (0,0) -- (10.72,5.15) -- (7.12,0) -- cycle    ;
	%Straight Lines [id:da20551356813850896] 
	\draw    (244.5,151.33) -- (244.5,84.33) ;
	\draw [shift={(244.5,81.33)}, rotate = 450] [fill={rgb, 255:red, 0; green, 0; blue, 0 }  ][line width=0.08]  [draw opacity=0] (10.72,-5.15) -- (0,0) -- (10.72,5.15) -- (7.12,0) -- cycle    ;
	%Straight Lines [id:da2652861406077569] 
	\draw    (244.5,151.33) -- (454.47,86.55) ;
	\draw [shift={(457.33,85.67)}, rotate = 522.85] [fill={rgb, 255:red, 0; green, 0; blue, 0 }  ][line width=0.08]  [draw opacity=0] (10.72,-5.15) -- (0,0) -- (10.72,5.15) -- (7.12,0) -- cycle    ;
	%Straight Lines [id:da35178094374328306] 
	\draw    (405,80.25) -- (439.46,69.92) ;
	\draw [shift={(442.33,69.06)}, rotate = 523.3199999999999] [fill={rgb, 255:red, 0; green, 0; blue, 0 }  ][line width=0.08]  [draw opacity=0] (10.72,-5.15) -- (0,0) -- (10.72,5.15) -- (7.12,0) -- cycle    ;
	%Shape: Circle [id:dp8287369725898297] 
	\draw  [fill={rgb, 255:red, 0; green, 0; blue, 0 }  ,fill opacity=1 ] (457.95,86.67) .. controls (457.95,85.35) and (459.02,84.29) .. (460.33,84.29) .. controls (461.65,84.29) and (462.71,85.35) .. (462.71,86.67) .. controls (462.71,87.98) and (461.65,89.05) .. (460.33,89.05) .. controls (459.02,89.05) and (457.95,87.98) .. (457.95,86.67) -- cycle ;
	%Straight Lines [id:da5074134243029063] 
	\draw    (460.33,86.67) -- (470.66,121.13) ;
	\draw [shift={(471.52,124)}, rotate = 253.32] [fill={rgb, 255:red, 0; green, 0; blue, 0 }  ][line width=0.08]  [draw opacity=0] (10.72,-5.15) -- (0,0) -- (10.72,5.15) -- (7.12,0) -- cycle    ;
	%Straight Lines [id:da272551315978391] 
	\draw    (202.5,86) -- (202.5,55.52) ;
	\draw [shift={(202.5,52.52)}, rotate = 450] [fill={rgb, 255:red, 0; green, 0; blue, 0 }  ][line width=0.08]  [draw opacity=0] (10.72,-5.15) -- (0,0) -- (10.72,5.15) -- (7.12,0) -- cycle    ;

	% Text Node
	\draw (227,48.67) node    {$z$};
	% Text Node
	\draw  [fill={rgb, 255:red, 222; green, 222; blue, 222 }  ,fill opacity=1 ]  (127.51, 214.13) circle [x radius= 12.04, y radius= 12.04]   ;
	\draw (127.51,214.13) node  [font=\footnotesize]  {$-$};
	% Text Node
	\draw (251,232) node    {$O$};
	% Text Node
	\draw (211,222.67) node    {$dS$};
	% Text Node
	\draw (211,172.67) node    {$dl$};
	% Text Node
	\draw (272.67,118.67) node    {$\vec{P}( t)$};
	% Text Node
	\draw (266.67,171.67) node    {$\overrightarrow{r'}$};
	% Text Node
	\draw (348.67,172.33) node    {$\vec{r}$};
	% Text Node
	\draw (352,96.33) node    {$\vec{r} -\overrightarrow{r'} \simeq \vec{r}$};
	% Text Node
	\draw (418,60.33) node    {$\vec{u}_{r}$};
	% Text Node
	\draw (473.14,76.05) node    {$P$};
	% Text Node
	\draw (544,58.33) node    {$\times \vec{u}_{\varphi }$};
	% Text Node
	\draw (487.71,113.76) node    {$\vec{u}_{\vartheta }$};
	% Text Node
	\draw (329.57,229.81) node    {$d\tau =dS\ dl$};
	% Text Node
	\draw (188.14,69.81) node    {$\vec{u}_{z}$};
	% Text Node
	\draw  [fill={rgb, 255:red, 222; green, 222; blue, 222 }  ,fill opacity=1 ]  (127.51, 99.13) circle [x radius= 12.04, y radius= 12.04]   ;
	\draw (127.51,99.13) node  [font=\footnotesize]  {$+$};
	% Text Node
	\draw (97.86,97.52) node    {$+q$};
	% Text Node
	\draw (97.86,212.52) node    {$-q$};
	% Text Node
	\draw (115.86,159.52) node    {$\vec{d}$};
	% Connection
	\draw    (127.51,202.09) -- (127.51,114.17) ;
	\draw [shift={(127.51,111.17)}, rotate = 450] [fill={rgb, 255:red, 0; green, 0; blue, 0 }  ][line width=0.08]  [draw opacity=0] (10.72,-5.15) -- (0,0) -- (10.72,5.15) -- (7.12,0) -- cycle    ;

	\end{tikzpicture}
\end{figure}
\FloatBarrier
Considereremo a tal fine il dipolo come un elemento di volume $d\tau$ al cui interno il vettore polarizzazione sia diretto lungo l'asse $z$ e vari nel tempo secondo la relazione:
\[
	\vec{P} (t) = P_0\sin (\omega t)\vec{u}_z
\]
Possiamo a tal punto determinare la corrente di polarizzazione che scorre nell'elemento di volume come:
\[
	\vec{J}_p = \frac{\partial \vec{P}}{\partial t} = P_0\omega \cos (\omega t) \vec{u}_z
\]
Inseriremo questa corrente nell'espressione del potenziale ritardato $\vec{A}(\vec{r},t)$, dove $\vec{r}$ individua la posizione del punto $P$ in cui valutare tale vettore. Otterremo:
\[
	\vec{A}(\vec{r}, t) = \int_{\tau}\frac{\mu_0 \vec{J}_p(\vec{r'},t-\frac{\Delta r}{c})}{4\pi \Delta r} d\tau
\]
Con $ \Delta \vec{r} = \vec{r} - \vec{r'} $ e $ \vec{r'}$ la posizione dell'elemento di corrente. Ponendo l'origine del sistema di coordinate in corrispondenza del dipolo stesso, avremo:
\[
	\vec{r'} \simeq 0 \implies  \Delta \vec{r} = \vec{r} - \vec{r'} \simeq \vec{r}
\]
Inoltre risulta che:
\[
	\vec{J}_pd\tau = \frac{\partial \vec{P} \, d\tau}{\partial t} = \frac{\partial \vec{p}}{\partial t}
\]
Pertanto risulterà che:
\[
	\vec{A} (\vec{r},t) = \frac{\mu_0 \dot{\vec{p}}(t-\frac{r}{c})}{4\pi r} = \frac{\mu_0 \dot{p}(t-\frac{r}{c})}{4\pi r} \vec{u}_z \qquad \left( \dot{\vec{p}} = \frac{\partial \vec{p}}{\partial t}  \right)
\]
Dato che:
\[
	\vec{p} (t) = p_0 \sin (\omega t) \vec{u}_z \implies \dot{\vec{p}} = p_0\omega \cos (\omega t) \vec{u}_z
\]
Si ha:
\[
	\boxed{\vec{A} (\vec{r},t) = \frac{\mu_0}{4\pi} \frac{p_0 \omega}{r} \cos \left[ \omega\left( t-\frac{r}{c} \right)   \right]  \vec{u}_z}
\]
$\vec{A}$ è in tutti i punti dello spazio diretto come l'asse $z$.
Potremo poi determinare il potenziale scalare ritardato $V$ impiegando la condizione di Lorentz:
\[
	\vec{\nabla} \cdot \vec{A} + \frac{1}{c^2}\frac{\partial V}{\partial t} =0 \implies  \frac{\partial V}{\partial t} = - c^2 \vec{\nabla} \cdot \vec{A} = - c^2 \frac{\partial A_z}{\partial z}
\]
Essendo le altre componenti di $\vec{A}$ nulle. Partendo da questo punto si può dimostrare che:
\begin{gather*}
	V(t)=\int \frac{\partial V}{\partial t} \partial + \text{costante}\\
	\Downarrow\\
	V(\vec{r},t) = \frac{z}{4\pi \varepsilon_0 r}\left[ \frac{p(t-\frac{r}{c})}{r^2} + \frac{\dot{p}(t-\frac{r}{c})}{cr}\right]
\end{gather*}
Che nel nostro caso specifico diverrà:
\[
	\boxed{V(\vec{r},t) = \frac{z}{4\pi \varepsilon_0 r}\left[ \frac{p_0\sin [\omega(t-\frac{r}{c})]}{r^2} + \frac{p_0\omega\cos [\omega(t-\frac{r}{c})]}{cr}\right]}
\]
Nel caso di un dipolo che non cambia forma, avevamo trovato che in regime stazionario il potenziale assume questa forma:
\[
	V(r) = \frac{\vec{p} \cdot \vec{u}_r}{4\pi \varepsilon_0 r^2} = \frac{p_0\vec{u}_r\cdot r\,\vec{u}_r}{4\pi \varepsilon_0 r^3} = \frac{p_0\vec{u}_r\cdot \vec{r}}{4\pi \varepsilon_0 r^3} = \frac{p_0z}{4\pi \varepsilon_0 r^3}
\]
Notiamo quindi che un dipolo statico con momento di dipolo $ p_0  $ darebbe vita ad un potenziale elettrostatico coincidente solo con la prima parte dell'espressione di $V(\vec{r}, t)$ pur di sostituire il momento di dipolo dipolo $p(t)$ con il momento di dipolo $p_0 $. Dunque nel caso tempo variante la struttura del potenziale scalare cambia notevolmente. Mentre il primo pezzo ha al denominatore $r^2$, il secondo ha $r$. Le due parti non hanno lo stesso andamento nel tempo, ma l'ultima decresce più lentamente.

A partire dalle espressioni trovate per $\vec{A}$ e per $V$ è possibile determinare i campi $\vec{E}$ e $\vec{B}$ delle onde elettromagnetiche prodotte, mediante le relazioni:
\[
	\vec{B} =\text{rot}\vec{A} \qquad \vec{E} =-\text{grad}V - \frac{\partial \vec{A}}{\partial t}
\]
Tenendo conto solo dei termini che decrescono più lentamente con la distanza, si troverà allora che, a grande distanza dal dipolo:
\begin{gather*}
	\vec{B} \sim \frac{\mu_0}{4\pi} \frac{\sin \vartheta \ddot{p}(t-\Delta r/t)}{rc}\vec{u}_{\varphi}\\
	\vec{E} \sim \frac{1}{4\pi \varepsilon_0}\frac{\sin \vartheta \ddot{p}(t-r/t)}{rc^2} \vec{u}_\vartheta
\end{gather*}
Nel caso specifico:
\begin{gather*}
	p=p_0\sin (\omega t)\\
	\ddot{p} = - \omega^2 p_0\sin (\omega t)\\
	\ddot{p} (t-r/c) = -\omega^2 p_0\sin \left[ \omega\left( t-\frac{r}{c} \right)   \right] = \omega^2 p_0\sin \left( \frac{\omega}{c}r-\omega t \right)
\end{gather*}
Ottenendo allora:
\begin{gather*}
	\vec{B} = \frac{B_0\sin \vartheta \sin (kr-\omega t)}{r}\vec{u}_{\varphi}\\
	\vec{E} = \frac{B_0c\sin \vartheta \sin (kr-\omega t)}{r}\vec{u}_\vartheta
\end{gather*}
Caratteristico di un'\textbf{onda sferica}. Si tratta di un onda che si propaga in tutte le direzioni, non su un asse. L'ampiezza dell'onda non è costante ma allontanandosi dal dipolo diminuisce sempre di più.
è utile osservare che il vettore di Poynting è dato da:
\[
	\vec{S} =\vec{E} \times \vec{H} = \frac{\vec{E} \times \vec{B}}{\mu_0} = \frac{p_0^2 \omega^4 \sin^2 \vartheta \sin^2 [\omega(t-r/c)]}{16\pi^2 \varepsilon_0 c^3  r^2} \vec{u}_r
\]
E notando che il valor medio su un ciclo ottico del termine tempovariante è pari a $\frac{1}{2} $ si otterrà alla fine:
\[
	\vec{S}_m = \frac{P_0^2 \omega^4 \sin^2 \vartheta}{32\pi^2 r^2 c^3 \varepsilon_0} \vec{u}_r = I_m\vec{u}_r
\]
Dove per $I_m $ intendiamo l'intensità media in funzione della distanza dal dipolo e dell'angolo $\vartheta $ che $ \vec{u}_r $ forma con $p$.
Se studiamo il bipolo come se fosse un \textbf{antenna}, le onde elettromagnetiche non hanno intensità costante in tutte le direzioni: sono più intense in direzione perpendicolare a quella dell'oscillazione. Nella direzione parallela al dipolo, non c'è campo magnetico.
\begin{figure}[htpb]
	\centering
	

	\tikzset{every picture/.style={line width=0.75pt}} %set default line width to 0.75pt        

	\begin{tikzpicture}[x=0.75pt,y=0.75pt,yscale=-1,xscale=1]
	%uncomment if require: \path (0,300); %set diagram left start at 0, and has height of 300

	%Straight Lines [id:da42412084017493457] 
	\draw    (260.5,189.69) -- (260.5,50.98) ;
	\draw [shift={(260.5,47.98)}, rotate = 450] [fill={rgb, 255:red, 0; green, 0; blue, 0 }  ][line width=0.08]  [draw opacity=0] (10.72,-5.15) -- (0,0) -- (10.72,5.15) -- (7.12,0) -- cycle    ;
	\draw [shift={(260.5,192.69)}, rotate = 270] [fill={rgb, 255:red, 0; green, 0; blue, 0 }  ][line width=0.08]  [draw opacity=0] (10.72,-5.15) -- (0,0) -- (10.72,5.15) -- (7.12,0) -- cycle    ;
	%Shape: Polygon Curved [id:ds9861847736168488] 
	\draw  [line width=1.5]  (370.17,120.33) .. controls (370.33,170.67) and (299.67,159.33) .. (260.5,120.33) .. controls (221.33,81.33) and (151,70.67) .. (150.83,120.33) .. controls (150.66,170) and (220.67,160) .. (260.5,120.33) .. controls (300.33,80.67) and (370.01,70) .. (370.17,120.33) -- cycle ;




	\end{tikzpicture}
\end{figure}
\FloatBarrier
Il calcolo della potenza $W$ irradiata dall'onda generata dal dipolo oscillante attraverso una superficie sferica $\Sigma$ centrata nel dipolo ed avente raggio $r$ è pari a:
\begin{align*}
	W_{oem} &= \int_{\Sigma} \vec{S}_m\cdot \vec{n} dS = \int_{\Sigma}\vec{S}_m\cdot \vec{u}_rdS\\
	&= \int_{\Sigma} \frac{p_0^2 \omega^4 \sin^2 \vartheta}{32\pi^2 r^2 \varepsilon_0 c^3} \cdot 2\pi r\sin \vartheta rd\vartheta  \\
	&= \frac{p_0^2 \omega^4}{16\pi \varepsilon_0 c^3} \underbrace{\int_0^{\pi} \sin^3 \vartheta d\vartheta}_{4/3} \\
	&= \frac{p_0^2 \omega^4}{12\pi \varepsilon_0 c^3}
\end{align*}
Che a parità di frequenza ed ampiezza del dipolo non dipende dal raggio della superficie prescelto. Pertanto, al fine di conservare la potenza irradiata, i campi $\vec{E}$ e $\vec{B}$ devono decrescere come $1/r$ per compensare l'aumento dell'area con $r$.
Come ultima osservazione notiamo che dalla formula generale di $\vec{E}$ e $\vec{B}$ a grandi distanze otteniamo
\[
	\vec{S} = \frac{\sin^2 \vartheta \left[ \ddot{p}\left( t-\frac{r}{c} \right)   \right]^2}{16\pi^2 \varepsilon_0 c^3 r^2} \vec{u}_r
\]
Qualora il dipolo oscillante fosse composto da una carica $-q$ in quiete ed una $+q$ in moto accelerato, otterremo:
\[
	\vec{p} = q\vec{d} \implies \ddot{\vec{p}}=\frac{d^2 \vec{p}}{dt^2} = q\frac{d^2 \vec{d}}{dt^2} = q\vec{a}
\]
Dove $a$ è l'accelerazione della carica positiva. Pertanto possiamo dalla formula precedente ottenere l'espressione dell'intensità di radiazione emessa da una carica $q$ in moto accelerato:
\[
	\vec{S} = \frac{\sin^2 \vartheta\,  q^2 a^2 \left( t-\frac{r}{c} \right) ^2}{16\pi^2 \varepsilon_0 c^3 r^2} \vec{u}_r
\]
Una carica in moto accelerato emette onde elettromagnetiche e la potenza di questa emissione va con il quadrato della distanza. Se $ a $ è costante:
\[
	W_{oem} = \frac{q^2 a^2}{6\pi \varepsilon_0 c^3} \qquad \text{Formula di Larmor}
\]




























































\chapter{Ottica}

\section{Propagazione di onde elettromagnetiche all'interfaccia fra due dielettrici}

Riepiloghiamo quello che abbiamo visto nel caso stazionario sulle condizioni al contorno per campi $\vec{B}$ ed $\vec{H}$ sulla superficie di separazione fra i due mezzi.
Se ipotizziamo che non ci siano cariche e correnti sulla superficie otteniamo le seguenti condizioni:
\[
	E_{t1}=E_{t2} \quad H_{t1}=H_{t2} \quad D_{n1}=D_{n2} \quad B_{n1}=B_{n2}
\]
Si dimostra che tali condizioni continuano a valere anche per campi variabili nel tempo. Consideriamo una superficie di suddivisione fra due materiali e un percorso chiuso a cavallo della superficie. Ricordando l'equazione di Maxwell:
\[
	\text{rot}\vec{E} = - \frac{\partial \vec{B}}{\partial t}
\]
grazie al teorema di Stokes possiamo ricavare il flusso del rotore di $\vec{E}$ attraverso la superficie delimitata da $\gamma$:
\begin{align*}
	\int_{\Sigma}\text{rot}\vec{E} \cdot \vec{n} dS &= -\frac{\partial}{\partial t} \int_{\Sigma} \vec{B} \cdot \vec{n} dS \\
	\vec{E}_1\cdot \vec{u}_t \,dl - \vec{E}_2\cdot \vec{u}_t\,dl &= - \frac{d\Phi_{\Sigma} (\vec{B})}{dt} \simeq 0
\end{align*}
Questa derivata è trascurabile perché l'area racchiusa da $\gamma$ tende a zero. Pertanto si ottiene ancora:
\[
	E_{t1}=E_{t2}
\]
La dimostrazione è analoga nel caso degli altri vettori che interessano le condizioni al contorno. Ricordiamo inoltre che per un onda piana valgono le seguenti relazioni:
\[
	\vec{E} =E_0\sin (\vec{k} \cdot \vec{r} -\omega t + \varphi) \quad \vec{E} =\vec{B} \times \vec{v}  \quad k = \frac{\omega}{v}=\frac{\omega}{\frac{c}{n(\omega)}} = \frac{\omega\,n(\omega)}{c}
\]
Consideriamo quindi una superficie di separazione fra due materiali di indice di rifrazione $n_1$ e $n_2$. Supponiamo che su tale superficie dal lato superiore incida un'onda piana.
\begin{figure}[htpb]
	\centering
	

	\tikzset{every picture/.style={line width=0.75pt}} %set default line width to 0.75pt        

	\begin{tikzpicture}[x=0.75pt,y=0.75pt,yscale=-1,xscale=1]
	%uncomment if require: \path (0,300); %set diagram left start at 0, and has height of 300

	%Straight Lines [id:da6183005433500193] 
	\draw    (164,167) -- (478,167) ;
	%Straight Lines [id:da22946316926380184] 
	\draw    (205.4,98.93) -- (284.3,165.14) ;
	\draw [shift={(286.6,167.07)}, rotate = 220] [fill={rgb, 255:red, 0; green, 0; blue, 0 }  ][line width=0.08]  [draw opacity=0] (10.72,-5.15) -- (0,0) -- (10.72,5.15) -- (7.12,0) -- cycle    ;
	%Straight Lines [id:da5066543042959124] 
	\draw    (275.27,133.9) -- (251.97,161.67) ;
	%Straight Lines [id:da00843700446911022] 
	\draw    (259.95,121.04) -- (236.65,148.81) ;
	%Straight Lines [id:da4219182653439604] 
	\draw    (244.63,108.19) -- (221.33,135.96) ;
	%Straight Lines [id:da7662075357732401] 
	\draw    (229.31,95.33) -- (206.01,123.1) ;

	%Straight Lines [id:da8746531724314954] 
	\draw    (285.93,166.6) -- (352.14,87.7) ;
	\draw [shift={(354.07,85.4)}, rotate = 490] [fill={rgb, 255:red, 0; green, 0; blue, 0 }  ][line width=0.08]  [draw opacity=0] (10.72,-5.15) -- (0,0) -- (10.72,5.15) -- (7.12,0) -- cycle    ;
	%Straight Lines [id:da9060241990807703] 
	\draw    (320.9,96.73) -- (348.67,120.03) ;
	%Straight Lines [id:da19712746629119682] 
	\draw    (308.04,112.05) -- (335.81,135.35) ;
	%Straight Lines [id:da7849526542504699] 
	\draw    (295.19,127.37) -- (322.96,150.67) ;
	%Straight Lines [id:da9233816968746384] 
	\draw    (282.33,142.69) -- (310.1,165.99) ;

	%Straight Lines [id:da5011592586326945] 
	\draw    (285.5,166.1) -- (337,255.3) ;
	\draw [shift={(338.5,257.9)}, rotate = 240] [fill={rgb, 255:red, 0; green, 0; blue, 0 }  ][line width=0.08]  [draw opacity=0] (10.72,-5.15) -- (0,0) -- (10.72,5.15) -- (7.12,0) -- cycle    ;
	%Straight Lines [id:da27638160120125854] 
	\draw    (339.2,222.86) -- (307.8,240.98) ;
	%Straight Lines [id:da6707584174541343] 
	\draw    (329.2,205.54) -- (297.8,223.66) ;
	%Straight Lines [id:da0469208813543982] 
	\draw    (319.2,188.22) -- (287.8,206.34) ;
	%Straight Lines [id:da38473496110325756] 
	\draw    (309.2,170.89) -- (277.8,189.02) ;

	%Straight Lines [id:da20096840221456458] 
	\draw [line width=1.5]    (164,167) -- (274,167) ;
	\draw [shift={(278,167)}, rotate = 180] [fill={rgb, 255:red, 0; green, 0; blue, 0 }  ][line width=0.08]  [draw opacity=0] (13.4,-6.43) -- (0,0) -- (13.4,6.44) -- (8.9,0) -- cycle    ;

	% Text Node
	\draw (270,186) node    {$P$};
	% Text Node
	\draw (204,154) node    {$\vec{r}$};
	% Text Node
	\draw (437.5,151) node    {$1$};
	% Text Node
	\draw (437.5,179) node    {$2$};
	% Text Node
	\draw (437,104.5) node    {$n_{1}$};
	% Text Node
	\draw (190,75.5) node    {$\vec{k}_{i} ,\omega _{i} ,\varphi _{i}$};
	% Text Node
	\draw (370,55.5) node    {$\vec{k}_{r} ,\omega _{r} ,\varphi _{r}$};
	% Text Node
	\draw (380.5,251.5) node    {$\vec{k}_{t} ,\omega _{t} ,\varphi _{t}$};


	\end{tikzpicture}
\end{figure}
\FloatBarrier
Nel momento in cui ciò accade, risulterà emergere un'\textbf{onda riflessa}. Essa sarà piana se la superficie è piana ma potrebbe avere una fase e un'ampiezza da determinare. Generalmente una porzione dell'onda incidente verrà anche trasmessa attraverso la superficie di separazione e propagherà. Quest'onda viene anche detta \textbf{onda trasmessa} o rifratta. Vogliamo determinare il legame fra queste tre onde.

Possiamo utilizzare le condizioni al contorno prima citate.
È necessario, perché questo siano rispettate, che gli argomenti dei seni delle sinusoidi siano gli stessi. Altrimenti non potremmo avere uguaglianza fra le componenti tangenti e normali dei vari campi. Si avrà allora che:
\[
	\vec{k}_i\cdot \vec{r} -\omega_it+\varphi_i = \vec{k}_r\cdot \vec{r} -\omega_rt+\varphi_r = \vec{k}_t\cdot \vec{r} -\omega_tt+\varphi_t
\]
Osserviamo che se poniamo $t=0$ e $r=0$ (scegliamo un certo istante e una certa posizione particolare), la relazione si semplifica:
\[
	\varphi_i = \varphi_r = \varphi_t
\]
Quindi di sicuro le tre fasi si eguagliano. Possiamo porre $\varphi=0$, cambiando così l'origine dei tempi, e portare tutte e tre le fasi a zero.
Se poniamo solo $r=0$, abbiamo:
\[
	-\omega_it = -\omega_rt = -\omega_tt
\]
Quindi anche le onde trasmesse e riflesse hanno la stessa frequenza. Parleremo d'ora in poi di $\omega$. A questo punto, rimane la relazione:
\begin{equation}
	\vec{k}_i\cdot \vec{r} = \vec{k}_r\cdot \vec{r} = \vec{k}_t\cdot \vec{r}
	\label{eq:uguaglianzaVetOnda}
\end{equation}
Scegliamo le coordinate in modo che la superficie di ha separazione coincida con il piano $xy$, su cui pertanto si trovano l'origine e il punto $P$ di incidenza. Inoltre l'asse $x$ è ortogonale al piano di incidenza individuato da $\vec{k}_i$ e da $\vec{k}_r$. Quindi $\vec{k}$ non ha componenti rispetto a tale variabile.
\begin{figure}[htpb]
	\centering
	

	\tikzset{every picture/.style={line width=0.75pt}} %set default line width to 0.75pt        

	\begin{tikzpicture}[x=0.75pt,y=0.75pt,yscale=-1,xscale=1]
	%uncomment if require: \path (0,300); %set diagram left start at 0, and has height of 300

	%Shape: Axis 2D [id:dp43043438928604094] 
	\draw  (143,155) -- (539.5,155)(177.5,44) -- (177.5,262) (532.5,150) -- (539.5,155) -- (532.5,160) (172.5,51) -- (177.5,44) -- (182.5,51)  ;
	%Straight Lines [id:da6171096901926401] 
	\draw [line width=1.5]    (177.5,155) -- (329.5,155) ;
	\draw [shift={(333.5,155)}, rotate = 180] [fill={rgb, 255:red, 0; green, 0; blue, 0 }  ][line width=0.08]  [draw opacity=0] (13.4,-6.43) -- (0,0) -- (13.4,6.44) -- (8.9,0) -- cycle    ;
	%Straight Lines [id:da7950107605009162] 
	\draw [line width=1.5]    (277.5,63) -- (331.42,151.58) ;
	\draw [shift={(333.5,155)}, rotate = 238.67000000000002] [fill={rgb, 255:red, 0; green, 0; blue, 0 }  ][line width=0.08]  [draw opacity=0] (13.4,-6.43) -- (0,0) -- (13.4,6.44) -- (8.9,0) -- cycle    ;
	%Straight Lines [id:da4358193570786466] 
	\draw [line width=1.5]    (387.42,66.42) -- (333.5,155) ;
	\draw [shift={(389.5,63)}, rotate = 121.33] [fill={rgb, 255:red, 0; green, 0; blue, 0 }  ][line width=0.08]  [draw opacity=0] (13.4,-6.43) -- (0,0) -- (13.4,6.44) -- (8.9,0) -- cycle    ;
	%Straight Lines [id:da06331109301956106] 
	\draw [line width=1.5]    (333.5,155) -- (414.47,219.8) ;
	\draw [shift={(417.59,222.3)}, rotate = 218.67000000000002] [fill={rgb, 255:red, 0; green, 0; blue, 0 }  ][line width=0.08]  [draw opacity=0] (13.4,-6.43) -- (0,0) -- (13.4,6.44) -- (8.9,0) -- cycle    ;
	%Straight Lines [id:da2398676595359941] 
	\draw [line width=0.75]  [dash pattern={on 0.84pt off 2.51pt}]  (333.5,242) -- (333.5,60) ;
	%Shape: Circle [id:dp75582930196853] 
	\draw  [fill={rgb, 255:red, 0; green, 0; blue, 0 }  ,fill opacity=1 ] (173.33,155) .. controls (173.33,152.7) and (175.2,150.83) .. (177.5,150.83) .. controls (179.8,150.83) and (181.67,152.7) .. (181.67,155) .. controls (181.67,157.3) and (179.8,159.17) .. (177.5,159.17) .. controls (175.2,159.17) and (173.33,157.3) .. (173.33,155) -- cycle ;
	%Shape: Arc [id:dp8841166469092541] 
	\draw  [draw opacity=0] (306.43,111.18) .. controls (313.88,106.57) and (322.59,103.81) .. (331.92,103.52) -- (333.5,155) -- cycle ; \draw   (306.43,111.18) .. controls (313.88,106.57) and (322.59,103.81) .. (331.92,103.52) ;
	%Shape: Arc [id:dp5001740494755782] 
	\draw  [draw opacity=0] (335.13,103.53) .. controls (344.19,103.81) and (352.67,106.43) .. (359.97,110.81) -- (333.5,155) -- cycle ; \draw   (335.13,103.53) .. controls (344.19,103.81) and (352.67,106.43) .. (359.97,110.81) ;
	%Shape: Arc [id:dp2155363034717026] 
	\draw  [draw opacity=0] (373.98,186.84) .. controls (364.6,198.75) and (350.08,206.42) .. (333.77,206.5) -- (333.5,155) -- cycle ; \draw   (373.98,186.84) .. controls (364.6,198.75) and (350.08,206.42) .. (333.77,206.5) ;

	% Text Node
	\draw (122,89) node    {$n_{1}$};
	% Text Node
	\draw (122,209) node    {$n_{2}$};
	% Text Node
	\draw (166,165) node    {$O$};
	% Text Node
	\draw (166,40) node    {$z$};
	% Text Node
	\draw (247.33,139.67) node    {$\vec{r}$};
	% Text Node
	\draw (323.33,171.67) node    {$P$};
	% Text Node
	\draw (265.33,64.33) node    {$\vec{k}_{i}$};
	% Text Node
	\draw (404,61.67) node    {$\vec{k}_{r}$};
	% Text Node
	\draw (431.33,217.67) node    {$\vec{k}_{t}$};
	% Text Node
	\draw (166,141) node    {$x$};
	% Text Node
	\draw (552,154.33) node    {$y$};
	% Text Node
	\draw (488,140.33) node    {$\Sigma $};
	% Text Node
	\draw (316.83,90.67) node    {$\vartheta _{i}$};
	% Text Node
	\draw (349.83,90.67) node    {$\vartheta _{r}$};
	% Text Node
	\draw (360.33,211.67) node    {$\vartheta _{t}$};


	\end{tikzpicture}
\end{figure}
\FloatBarrier
\begin{align*}
	\vec{k}_i &= \underbrace{k_{ix}}_{=0} \vec{u}_x + k_{iy} \vec{u}_y + k_{iz} \vec{u}_z \\
	\vec{k}_i\cdot \vec{r}  &= (k_{iy} \vec{u}_y + k_{iz} \vec{u}_z)\cdot \vec{r} \tag*{moltiplico per $\vec{r}$}\\
	\vec{k}_i\cdot \vec{r}  &= (k_{iy} \vec{u}_y + k_{iz} \vec{u}_z)\cdot (x\vec{u}_x+y\vec{u}_y) \tag*{$\vec{r}$ sta su $xy$}\\
	\vec{k}_i\cdot \vec{r}  &= k_{iy} \,y \\
	\vec{k}_r\cdot \vec{r} = \vec{k}_t\cdot \vec{r}  &= k_{iy} \,y
\end{align*}
Queste uguaglianza deve valere per qualsiasi valore di $x$ e di $y$, cioè per qualsiasi posizione di $P$ sul piano, quindi deve essere:
\[
	k_{rx} = 0 \qquad k_{tx} = 0 \qquad k_{rz} = 0 \qquad k_{tz} = 0
\]
Dal momento che anche $k_t$ e $k_r$ hanno la componente in $x$ nulla, si deduce che giacciono sul piano di incidenza. Si enuncia pertanto la \textbf{prima legge della riflessione e della rifrazione}:\\
\emph{Le direzioni di propagazione dell'onda incidente, riflessa e rifratta giacciono nel piano di incidenza, individuato dalla direzione di incidenza e dalla normale alla superficie di separazione nel punto di incidenza.}
\begin{itemize}
	\item Chiameremo $\vartheta_i$ l'angolo che il vettore dell'onda incidente forma con la normale alla superficie. Esso prende il nome di \textbf{angolo di incidenza}.
	\item Chiamiamo \textbf{angolo di riflessione} $\vartheta_r$ quello che $\vec{k}_r$ forma con la normale.
	\item Analogamente chiameremo \textbf{angolo di trasmissione} o di rifrazione $\vartheta_t$ quello che $k_t$ forma con la normale.
\end{itemize}

La componente lungo l'asse $y$ di $\vec{k}_i$ è pari a:
\[
	k_{iy} = |\vec{k}_i |\sin \vartheta_i = \frac{\omega n_1}{c}\sin \vartheta_i
\]
A questo punto è semplice scrivere anche le altre componenti:
\begin{align*}
	k_{ry} &= |\vec{k}_r |\sin \vartheta_r = \frac{\omega n_1}{c}\sin \vartheta_r \\
	k_{ty} &= |\vec{k}_t |\sin \vartheta_t = \frac{\omega n_2}{c}\sin \vartheta_t \\
	k_{iy}&=k_{ry} \implies \frac{\omega n_1}{c}\sin \vartheta_i = \frac{\omega n_1}{c}\sin \vartheta_r \implies \boxed{\vartheta_i=\vartheta_r}\\
	k_{iy}&=k_{ty} \implies \frac{\omega n_1}{c}\sin \vartheta_i = \frac{\omega n_2}{c}\sin \vartheta_t \implies \boxed{n_1\sin  \vartheta_i=n_2\sin  \vartheta_r}\\
\end{align*}
Questo risultato è noto come \textbf{legge di Snell}.
Detto $n_2/n_1$ \emph{indice di rifrazione relativo del secondo mezzo rispetto al primo}, la legge di Snell viene enunciata dicendo:\\
\emph{Il rapporto tra il seno dell'angaolo di incidenza e il seno dell'angolo di rifrazione è costante ed eguale all'indice di rifrazione relativo fra i due mezzi.}\\
La legge di Snell può anche essere utilizzata come definizione operativa dell'indice di rifrazione di una sostanza trasparente relativo ad un mezzo campione, ovvero dell'indice assoluto rispetto al vuoto. Per eseguire la msiura, deve però essere possibile definire una superficie di separazione tra il mezzo in esame e il mezzo esterno con indice di rifrazione noto. Il metodo quindi non si applica ai gas.

Nella legge $\vartheta_i=\vartheta_r  $ non compare invece alcuna caratteristica del mezzo in cui si propaga l'onda incidente e l'onda riflessa. L'angolo di riflessione è sempre lo stesso, eguale all'angolo di incidenza, qualunque sia la lunghezza d'onda della luce incidente e non c'è dispersione.

La legge di Snell ci permette di studiare due casi in particolare.
\begin{itemize}
	\item Caso $n_1<n_2$. Qualunque sia l'angolo di incidenza, c'è sempre soluzione per la legge di Snell, quindi si trova sempre un corrispondente angolo di trasmissione. Infatti:
	\[
		\forall \vartheta_i\qquad \sin \vartheta_t= \frac{n_1}{n_2}\sin \vartheta_i \le 1
	\]
	\begin{figure}[htpb]
		\centering
		

		\tikzset{every picture/.style={line width=0.75pt}} %set default line width to 0.75pt        

		\begin{tikzpicture}[x=0.75pt,y=0.75pt,yscale=-1,xscale=1]
		%uncomment if require: \path (0,300); %set diagram left start at 0, and has height of 300

		%Straight Lines [id:da5050294782052775] 
		\draw [line width=0.75]    (167.08,154) -- (457.92,154) ;
		%Straight Lines [id:da8251861314221638] 
		\draw [line width=1.5]    (256.5,62) -- (310.42,150.58) ;
		\draw [shift={(312.5,154)}, rotate = 238.67000000000002] [fill={rgb, 255:red, 0; green, 0; blue, 0 }  ][line width=0.08]  [draw opacity=0] (13.4,-6.43) -- (0,0) -- (13.4,6.44) -- (8.9,0) -- cycle    ;
		%Straight Lines [id:da4739059636586811] 
		\draw [line width=1.5]    (366.42,65.42) -- (312.5,154) ;
		\draw [shift={(368.5,62)}, rotate = 121.33] [fill={rgb, 255:red, 0; green, 0; blue, 0 }  ][line width=0.08]  [draw opacity=0] (13.4,-6.43) -- (0,0) -- (13.4,6.44) -- (8.9,0) -- cycle    ;
		%Straight Lines [id:da44077768680661067] 
		\draw [line width=1.5]    (312.5,154) -- (332.87,255.68) ;
		\draw [shift={(333.66,259.6)}, rotate = 258.67] [fill={rgb, 255:red, 0; green, 0; blue, 0 }  ][line width=0.08]  [draw opacity=0] (13.4,-6.43) -- (0,0) -- (13.4,6.44) -- (8.9,0) -- cycle    ;
		%Straight Lines [id:da7486337079341854] 
		\draw [line width=0.75]  [dash pattern={on 0.84pt off 2.51pt}]  (312.5,255.25) -- (312.5,59) ;
		%Shape: Arc [id:dp3100987369844985] 
		\draw  [draw opacity=0] (285.43,110.18) .. controls (292.88,105.57) and (301.59,102.81) .. (310.92,102.52) -- (312.5,154) -- cycle ; \draw   (285.43,110.18) .. controls (292.88,105.57) and (301.59,102.81) .. (310.92,102.52) ;
		%Shape: Arc [id:dp6634686778117325] 
		\draw  [draw opacity=0] (314.13,102.53) .. controls (323.19,102.81) and (331.67,105.43) .. (338.97,109.81) -- (312.5,154) -- cycle ; \draw   (314.13,102.53) .. controls (323.19,102.81) and (331.67,105.43) .. (338.97,109.81) ;
		%Shape: Arc [id:dp36248656848360494] 
		\draw  [draw opacity=0] (322.77,204.48) .. controls (319.54,205.13) and (316.19,205.48) .. (312.77,205.5) -- (312.5,154) -- cycle ; \draw   (322.77,204.48) .. controls (319.54,205.13) and (316.19,205.48) .. (312.77,205.5) ;

		% Text Node
		\draw (201,88) node    {$n_{1}\text{ (aria)}$};
		% Text Node
		\draw (205,208) node    {$n_{2} \ \text{(vetro)}$};
		% Text Node
		\draw (442.33,141.33) node    {$\Sigma $};
		% Text Node
		\draw (295.83,89.67) node    {$\vartheta _{i}$};
		% Text Node
		\draw (328.83,89.67) node    {$\vartheta _{r}$};
		% Text Node
		\draw (336.83,205.17) node    {$\vartheta _{t}$};
		% Text Node
		\draw (418.83,205.17) node    {$\vartheta _{t} < \vartheta _{i}$};


		\end{tikzpicture}
	\end{figure}
	\FloatBarrier
	\item Caso $n_2<n_1$. All'aumentare dell'angolo di incidenza, l'onda trasmessa arriverà a propagare lungo la superficie di separazione per poi essere totalmente riflessa. Il rapporto $n_1/n_2$ infatti è maggiore di $1$, quindi il seno dell'angolo trasmesso non potrà essere definito per qualsiasi ampiezza di $ \vartheta_i  $.
	\[
		\sin \vartheta_t = \frac{n_1}{n_2}\sin \vartheta_i
	\]
	Quell'angolo per il quale $\sin \vartheta_t $ è pari a $1$ è detto \textbf{angolo limite}. L'onda viene subito deflessa. Per trovarlo basta invertire l'equazione, trovando:
	\[
		\vartheta_i = \arcsin \left( \frac{n_2}{n_1} \right)
	\]
	\begin{figure}[htpb]
		\centering
		

		\tikzset{every picture/.style={line width=0.75pt}} %set default line width to 0.75pt        

		\begin{tikzpicture}[x=0.75pt,y=0.75pt,yscale=-1,xscale=1]
		%uncomment if require: \path (0,300); %set diagram left start at 0, and has height of 300

		%Straight Lines [id:da965119002350822] 
		\draw [line width=0.75]    (167.08,154) -- (457.92,154) ;
		%Straight Lines [id:da41123154558130426] 
		\draw [line width=1.5]    (256.5,62) -- (310.42,150.58) ;
		\draw [shift={(312.5,154)}, rotate = 238.67000000000002] [fill={rgb, 255:red, 0; green, 0; blue, 0 }  ][line width=0.08]  [draw opacity=0] (13.4,-6.43) -- (0,0) -- (13.4,6.44) -- (8.9,0) -- cycle    ;
		%Straight Lines [id:da2239839616290593] 
		\draw [line width=1.5]    (366.42,65.42) -- (312.5,154) ;
		\draw [shift={(368.5,62)}, rotate = 121.33] [fill={rgb, 255:red, 0; green, 0; blue, 0 }  ][line width=0.08]  [draw opacity=0] (13.4,-6.43) -- (0,0) -- (13.4,6.44) -- (8.9,0) -- cycle    ;
		%Straight Lines [id:da8708148668464029] 
		\draw [line width=1.5]    (312.5,154) -- (415.02,169.63) ;
		\draw [shift={(418.97,170.24)}, rotate = 188.67] [fill={rgb, 255:red, 0; green, 0; blue, 0 }  ][line width=0.08]  [draw opacity=0] (13.4,-6.43) -- (0,0) -- (13.4,6.44) -- (8.9,0) -- cycle    ;
		%Straight Lines [id:da37791866391045437] 
		\draw [line width=0.75]  [dash pattern={on 0.84pt off 2.51pt}]  (312.5,255.25) -- (312.5,59) ;
		%Shape: Arc [id:dp2694839214350737] 
		\draw  [draw opacity=0] (285.43,110.18) .. controls (292.88,105.57) and (301.59,102.81) .. (310.92,102.52) -- (312.5,154) -- cycle ; \draw   (285.43,110.18) .. controls (292.88,105.57) and (301.59,102.81) .. (310.92,102.52) ;
		%Shape: Arc [id:dp3463452444005113] 
		\draw  [draw opacity=0] (314.13,102.53) .. controls (323.19,102.81) and (331.67,105.43) .. (338.97,109.81) -- (312.5,154) -- cycle ; \draw   (314.13,102.53) .. controls (323.19,102.81) and (331.67,105.43) .. (338.97,109.81) ;
		%Shape: Arc [id:dp5103197837937474] 
		\draw  [draw opacity=0] (363.23,162.92) .. controls (359.02,187.02) and (338.05,205.37) .. (312.77,205.5) -- (312.5,154) -- cycle ; \draw   (363.23,162.92) .. controls (359.02,187.02) and (338.05,205.37) .. (312.77,205.5) ;

		% Text Node
		\draw (201,88) node    {$n_{1}\text{ (vetro)}$};
		% Text Node
		\draw (197,208) node    {$n_{2} \ \text{(aria)}$};
		% Text Node
		\draw (442.33,141.33) node    {$\Sigma $};
		% Text Node
		\draw (295.83,89.67) node    {$\vartheta _{i}$};
		% Text Node
		\draw (328.83,89.67) node    {$\vartheta _{r}$};
		% Text Node
		\draw (356.83,201.17) node    {$\vartheta _{t}$};
		% Text Node
		\draw (438.83,202.17) node    {$\vartheta _{t}  >\vartheta _{i}$};


		\end{tikzpicture}
	\end{figure}
	\FloatBarrier
\end{itemize}
Aumentando l'angolo l'incidenza oltre l'angolo limite, non ci sarà più trasmissione. Si parla di un fenomeno noto come \textbf{riflessione totale}, tutta l'onda incidente viene riflessa. Questa proprietà viene altamente sfruttata nelle \emph{fibre ottiche}. Una fibra ottica è un cilindro di vetro di indice di rifrazione maggiore rispetto al mezzo in cui è immerso. La luce che penetra nel cilindro attraverso una base incide sulle pareti laterali formando un angolo superiore all'angolo limite e viene riflessa totalmente molte volte, senza apprezzabili perdite, fino ad uscire dall'altra base. Questa tecnica è usata in medicina per l'osservazione di organi interni (endoscopia): con un fascio di fibre si convoglia la luce di una sorgente esterna sulla parte da osservare e con un altro fascio si riceve la luce diffusa dalla parte illuminata. Un'altra applicazione importante e in rapido sviluppo si ha nel campo delle telecomunicazioni.







































\section{Dispersione della luce in un mezzo trasparente}

Quando nel fascio incidente sono contenute più lunghezze d'onda, ad un dato angolo di incidenza corrispondono più angoli di rifrazione. L'indice di rifrazione infatti dipende dalla frequenza e diminuisce al crescere della lunghezza d'onda. Se ci riferiamo a onde elettromagnetiche dello spettro visibile, in tutti i materiali si trova che $n_{\text{rosso}} < n_{\text{verde}} < n_{\text{blu}}$.
Se ad esempio un sottile fascio di luce bianca, che contiene tutte le lunghezze d'onda visibili, incide su una lastra di vetro, la luce riflessa è ancora bianca mentre il fascio trasmesso nel vetro è composto da una serie di raggi di colore diversi, ognuno con diverso angolo di rifrazione. È questo aspetto del fenomeno che ha dato origine al termine \textbf{dispersione della luce}.
La dispersione della luce spiega anche il fenomeno degli arcobaleni. La luce del sole viene trasmessa dalle gocce d'acqua, e i colori appaiono quindi separati.







































\section{Intensità delle onde elettromagnetiche riflesse e rifratte. Formule di Fresnel}

Le relazioni tra le ampiezze si ricavano dalle equazioni di Maxwell, precisamente dalle condizioni di continuità dei campi.




















\subsection{Caso trasversale magnetico (TM)}

In figura è indicata la situazione in cui il campo elettrico incidente è polarizzato n rettilinearmente nel piano $\pi$ di incidenza. Il campo $\vec{B}$ inoltre è perpendicolare al piano del foglio ed è continuo nel passaggio attraverso la superficie in quanto assumiamo trascurabili le proprietà magnetiche dei dielettrici.
\begin{figure}[htpb]
	\centering
	

	\tikzset{every picture/.style={line width=0.75pt}} %set default line width to 0.75pt        

	\begin{tikzpicture}[x=0.75pt,y=0.75pt,yscale=-1,xscale=1]
	%uncomment if require: \path (0,300); %set diagram left start at 0, and has height of 300

	%Straight Lines [id:da2090760346759728] 
	\draw [line width=0.75]    (175.08,159) -- (465.92,159) ;
	%Straight Lines [id:da019521017012416708] 
	\draw [line width=1.5]    (237.76,91.25) -- (264.28,117.77) ;
	\draw [shift={(267.11,120.59)}, rotate = 225] [fill={rgb, 255:red, 0; green, 0; blue, 0 }  ][line width=0.08]  [draw opacity=0] (13.4,-6.43) -- (0,0) -- (13.4,6.44) -- (8.9,0) -- cycle    ;
	%Straight Lines [id:da8473806581412822] 
	\draw [line width=1.5]    (237.76,91.25) -- (264.28,64.73) ;
	\draw [shift={(267.11,61.91)}, rotate = 495] [fill={rgb, 255:red, 0; green, 0; blue, 0 }  ][line width=0.08]  [draw opacity=0] (13.4,-6.43) -- (0,0) -- (13.4,6.44) -- (8.9,0) -- cycle    ;
	%Shape: Circle [id:dp2638401536274628] 
	\draw  [fill={rgb, 255:red, 0; green, 0; blue, 0 }  ,fill opacity=1 ] (235.41,88.89) .. controls (236.71,87.59) and (238.82,87.59) .. (240.12,88.89) .. controls (241.42,90.19) and (241.42,92.31) .. (240.12,93.61) .. controls (238.82,94.91) and (236.71,94.91) .. (235.41,93.61) .. controls (234.1,92.31) and (234.1,90.19) .. (235.41,88.89) -- cycle ;

	%Straight Lines [id:da7803022354972504] 
	\draw [line width=0.75]  [dash pattern={on 0.84pt off 2.51pt}]  (320.5,260.25) -- (320.5,48) ;
	%Straight Lines [id:da7670188340524602] 
	\draw [line width=1.5]    (384.08,118.24) -- (410.6,91.72) ;
	\draw [shift={(413.43,88.89)}, rotate = 495] [fill={rgb, 255:red, 0; green, 0; blue, 0 }  ][line width=0.08]  [draw opacity=0] (13.4,-6.43) -- (0,0) -- (13.4,6.44) -- (8.9,0) -- cycle    ;
	%Straight Lines [id:da8684392720931324] 
	\draw [line width=1.5]    (384.08,118.24) -- (357.57,91.72) ;
	\draw [shift={(354.74,88.89)}, rotate = 405] [fill={rgb, 255:red, 0; green, 0; blue, 0 }  ][line width=0.08]  [draw opacity=0] (13.4,-6.43) -- (0,0) -- (13.4,6.44) -- (8.9,0) -- cycle    ;
	%Shape: Circle [id:dp6351258422342285] 
	\draw  [fill={rgb, 255:red, 0; green, 0; blue, 0 }  ,fill opacity=1 ] (381.73,120.59) .. controls (380.42,119.29) and (380.42,117.18) .. (381.73,115.88) .. controls (383.03,114.58) and (385.14,114.58) .. (386.44,115.88) .. controls (387.74,117.18) and (387.74,119.29) .. (386.44,120.59) .. controls (385.14,121.9) and (383.03,121.9) .. (381.73,120.59) -- cycle ;

	%Straight Lines [id:da860631649047952] 
	\draw [line width=1.5]    (360.21,209.51) -- (391.12,230.74) ;
	\draw [shift={(394.42,233.01)}, rotate = 214.49] [fill={rgb, 255:red, 0; green, 0; blue, 0 }  ][line width=0.08]  [draw opacity=0] (13.4,-6.43) -- (0,0) -- (13.4,6.44) -- (8.9,0) -- cycle    ;
	%Straight Lines [id:da735762826157579] 
	\draw [line width=1.5]    (360.21,209.51) -- (381.45,178.6) ;
	\draw [shift={(383.71,175.3)}, rotate = 484.49] [fill={rgb, 255:red, 0; green, 0; blue, 0 }  ][line width=0.08]  [draw opacity=0] (13.4,-6.43) -- (0,0) -- (13.4,6.44) -- (8.9,0) -- cycle    ;
	%Shape: Circle [id:dp8501408964699348] 
	\draw  [fill={rgb, 255:red, 0; green, 0; blue, 0 }  ,fill opacity=1 ] (357.47,207.62) .. controls (358.51,206.1) and (360.59,205.72) .. (362.1,206.76) .. controls (363.62,207.8) and (364,209.88) .. (362.96,211.39) .. controls (361.92,212.91) and (359.84,213.3) .. (358.33,212.25) .. controls (356.81,211.21) and (356.42,209.14) .. (357.47,207.62) -- cycle ;


	% Text Node
	\draw (192.33,144.33) node    {$n_{1}$};
	% Text Node
	\draw (194.33,171) node    {$n_{2} \ $};
	% Text Node
	\draw (450.33,146.33) node    {$\Sigma $};
	% Text Node
	\draw (279.67,125.67) node    {$\vec{k}_{i}$};
	% Text Node
	\draw (279.67,59.67) node    {$\vec{E}_{i}$};
	% Text Node
	\draw (223,93.67) node    {$\vec{B}_{i}$};
	% Text Node
	\draw (425,77) node    {$\vec{k}_{r}$};
	% Text Node
	\draw (358.33,72.33) node    {$\vec{E}_{r}$};
	% Text Node
	\draw (399,129.67) node    {$\vec{B}_{r}$};
	% Text Node
	\draw (345,215) node    {$\vec{B}_{t}$};
	% Text Node
	\draw (396.33,179.67) node    {$\vec{E}_{t}$};
	% Text Node
	\draw (406.33,234.33) node    {$\vec{k}_{t}$};


	\end{tikzpicture}
\end{figure}
\FloatBarrier
\[
	\vec{B}_i + \vec{B}_r = \vec{B}_t
\]
Affinché questa uguaglianza sia soddisfatta, $\vec{B}_r$ e $\vec{B}_t$ devono anche essi essere ortogonali al piano di incidenza come $\vec{B}_i$. Quindi il campo elettrico riflesso e rifratto sono anche essi polarizzati nel piano di incidenza. Sapendo che:
\[
	B=\frac{E}{v}=\frac{nE}{c}
\]
Possiamo scrivere:
\[
	\frac{n_1E_i}{c} + \frac{n_1E_r}{c} = \frac{n_2E_t}{c}
\]
Inoltre, dal momento che la componente normale del campo elettrico si conserva (condizioni al contorno):
\[
	E_i\cos \vartheta_i - E_r\cos \vartheta_r = E_t\cos \vartheta_t
\]
Abbiamo così un sistema di due equazioni che ci permette di arrivare a due risultati:
\begin{gather*}
	\left\{ \begin{array}{l}
	 	\frac{n_1E_i}{c} + \frac{n_1E_r}{c} = \frac{n_2E_t}{c} \\
		E_i\cos \vartheta_i - E_r\cos \vartheta_r = E_t\cos \vartheta_t
	\end{array} \right.\\\\
	\boxed{\frac{E_r}{E_i} = \frac{n_2\cos \vartheta_i-n_1\cos \vartheta_t}{n_2\cos \vartheta_i + n_1\cos \vartheta_t}}\qquad
	\boxed{\frac{E_t}{E_i} = \frac{2n_1\cos \vartheta_i}{n_1\cos \vartheta_t + n_2\cos \vartheta_i}}
\end{gather*}
Queste sono note come \textbf{formule di Fresnel} nel piano di incidenza $\pi$. Esse permettono di calcolare a partire dall'ampiezza incidente le ampiezze riflesse e rifratte: dipendono soltanto dall'angolo di incidenza e dall'angolo di rifrazione.




















\subsection{Caso trasversale elettrico (TE)}

Si tratta di un caso complementare in cui si invertono i ruoli dei due campi. $\vec{E}$ è polarizzato rettilinearmente ed è ortogonale al piano di incidenza $\pi$. Anche in questo caso si dimostra che i campi elettrici riflesso e rifratto mantengono la polarizzazione del campo elettrico incidente, per cui i campi magnetici stanno tutti su $\pi$.
\begin{figure}[htpb]
	\centering
	

	\tikzset{every picture/.style={line width=0.75pt}} %set default line width to 0.75pt        

	\begin{tikzpicture}[x=0.75pt,y=0.75pt,yscale=-1,xscale=1]
	%uncomment if require: \path (0,300); %set diagram left start at 0, and has height of 300

	%Straight Lines [id:da146620167581893] 
	\draw [line width=0.75]    (188.08,170) -- (478.92,170) ;
	%Straight Lines [id:da48568077110951213] 
	\draw [line width=1.5]    (254.75,97.26) -- (228.23,123.78) ;
	\draw [shift={(225.41,126.61)}, rotate = 315] [fill={rgb, 255:red, 0; green, 0; blue, 0 }  ][line width=0.08]  [draw opacity=0] (13.4,-6.43) -- (0,0) -- (13.4,6.44) -- (8.9,0) -- cycle    ;
	%Straight Lines [id:da48687204867088085] 
	\draw [line width=1.5]    (254.75,97.26) -- (281.27,123.78) ;
	\draw [shift={(284.09,126.61)}, rotate = 225] [fill={rgb, 255:red, 0; green, 0; blue, 0 }  ][line width=0.08]  [draw opacity=0] (13.4,-6.43) -- (0,0) -- (13.4,6.44) -- (8.9,0) -- cycle    ;
	%Shape: Circle [id:dp25299218306132554] 
	\draw  [fill={rgb, 255:red, 0; green, 0; blue, 0 }  ,fill opacity=1 ] (257.11,94.91) .. controls (258.41,96.21) and (258.41,98.32) .. (257.11,99.62) .. controls (255.81,100.92) and (253.69,100.92) .. (252.39,99.62) .. controls (251.09,98.32) and (251.09,96.21) .. (252.39,94.91) .. controls (253.69,93.6) and (255.81,93.6) .. (257.11,94.91) -- cycle ;

	%Straight Lines [id:da8386406799955135] 
	\draw [line width=0.75]  [dash pattern={on 0.84pt off 2.51pt}]  (333.5,271.25) -- (333.5,59) ;
	%Straight Lines [id:da70976718398795] 
	\draw [line width=1.5]    (382.1,115.25) -- (408.61,141.77) ;
	\draw [shift={(411.44,144.59)}, rotate = 225] [fill={rgb, 255:red, 0; green, 0; blue, 0 }  ][line width=0.08]  [draw opacity=0] (13.4,-6.43) -- (0,0) -- (13.4,6.44) -- (8.9,0) -- cycle    ;
	%Straight Lines [id:da10798554149673545] 
	\draw [line width=1.5]    (382.1,115.25) -- (408.61,88.73) ;
	\draw [shift={(411.44,85.91)}, rotate = 495] [fill={rgb, 255:red, 0; green, 0; blue, 0 }  ][line width=0.08]  [draw opacity=0] (13.4,-6.43) -- (0,0) -- (13.4,6.44) -- (8.9,0) -- cycle    ;
	%Shape: Circle [id:dp29239224121168195] 
	\draw  [fill={rgb, 255:red, 0; green, 0; blue, 0 }  ,fill opacity=1 ] (379.74,112.89) .. controls (381.04,111.59) and (383.15,111.59) .. (384.45,112.89) .. controls (385.75,114.19) and (385.75,116.31) .. (384.45,117.61) .. controls (383.15,118.91) and (381.04,118.91) .. (379.74,117.61) .. controls (378.44,116.31) and (378.44,114.19) .. (379.74,112.89) -- cycle ;

	%Straight Lines [id:da25353547697892953] 
	\draw [line width=1.5]    (397.2,210.38) -- (365.11,229.79) ;
	\draw [shift={(361.69,231.86)}, rotate = 328.83000000000004] [fill={rgb, 255:red, 0; green, 0; blue, 0 }  ][line width=0.08]  [draw opacity=0] (13.4,-6.43) -- (0,0) -- (13.4,6.44) -- (8.9,0) -- cycle    ;
	%Straight Lines [id:da08542631169290593] 
	\draw [line width=1.5]    (397.2,210.38) -- (416.61,242.46) ;
	\draw [shift={(418.68,245.89)}, rotate = 238.82999999999998] [fill={rgb, 255:red, 0; green, 0; blue, 0 }  ][line width=0.08]  [draw opacity=0] (13.4,-6.43) -- (0,0) -- (13.4,6.44) -- (8.9,0) -- cycle    ;
	%Shape: Circle [id:dp18706324259343288] 
	\draw  [fill={rgb, 255:red, 0; green, 0; blue, 0 }  ,fill opacity=1 ] (400.05,208.65) .. controls (401.01,210.23) and (400.5,212.28) .. (398.93,213.23) .. controls (397.35,214.18) and (395.3,213.68) .. (394.35,212.1) .. controls (393.4,210.53) and (393.9,208.48) .. (395.48,207.53) .. controls (397.05,206.57) and (399.1,207.08) .. (400.05,208.65) -- cycle ;


	% Text Node
	\draw (205.33,155.33) node    {$n_{1}$};
	% Text Node
	\draw (207.33,182) node    {$n_{2} \ $};
	% Text Node
	\draw (463.33,157.33) node    {$\Sigma $};
	% Text Node
	\draw (296.17,135.17) node    {$\vec{k}_{i}$};
	% Text Node
	\draw (258.67,77.67) node    {$\vec{E}_{i}$};
	% Text Node
	\draw (214,122.67) node    {$\vec{B}_{i}$};
	% Text Node
	\draw (423.5,80.5) node    {$\vec{k}_{r}$};
	% Text Node
	\draw (366.33,116.83) node    {$\vec{E}_{r}$};
	% Text Node
	\draw (425,145.17) node    {$\vec{B}_{r}$};
	% Text Node
	\draw (355.5,238) node    {$\vec{B}_{t}$};
	% Text Node
	\draw (409.33,192.67) node    {$\vec{E}_{t}$};
	% Text Node
	\draw (432.33,247.83) node    {$\vec{k}_{t}$};


	\end{tikzpicture}
\end{figure}
\FloatBarrier
Procediamo in modo analogo al caso precedente:
\[
	\vec{B} =\mu_0 \vec{H} \implies \vec{H} = \frac{\vec{B}}{\mu_0} \qquad B = \frac{nE}{c}
\]
Per le condizioni al contorno:
\[
	\left\{ \begin{array}{l}
	 	-\frac{B_i}{\mu_0}\cos \vartheta_i + \frac{B_r}{\mu_0}\sin \vartheta_r = - \frac{B_t}{\mu_0}\cos \vartheta_t\\
	 	\vec{E}_i + \vec{E}_r = \vec{E}_t
	\end{array} \right.
\]
si trovano le equazioni di Fresnel nel piano ortogonale a quello di incidenza:
\[
	\boxed{\frac{E_r}{E_i} = \frac{n_1\cos \vartheta_i-n_2\cos \vartheta_t}{n_1\cos \vartheta_i + n_2\cos \vartheta_t}}\qquad
	\boxed{\frac{E_t}{E_i} = \frac{2n_1\cos \vartheta_i}{n_1\cos \vartheta_i + n_2\cos \vartheta_t}}
\]
Un aspetto interessante è che l'ampiezza dell'onda riflessa è diversa nei vari casi. Generalmente accade che se consideriamo una superficie di separazione fra l'aria e un mezzo di indice di rifrazione maggiore (ad esempio l'acqua) e consideriamo la luce del sole che incide su tale superficie, essa di solito è non polarizzata, può essere vista come la somma di più onde polarizzate. Quando esse incidono sulla superficie si riflettono con differenti percentuali di riflessione. Ciò che accade è che la percentuale di riflessione della componente polarizzata orizzontalmente è maggiore. A causa di questa differenza fra coefficienti di riflessione, quando osserviamo luce naturale riflessa da una superficie, questa apparirà parzialmente o anche totalmente polarizzata in direzione orizzontale. Per eliminare questo effetto si utilizzano occhiali polarizzati. Il polarizzatore è un oggetto che assorbe la luce polarizzata in una certa direzione ma non quella in un'altra.
\begin{figure}[htpb]
	\centering
	

	\tikzset{every picture/.style={line width=0.75pt}} %set default line width to 0.75pt        

	\begin{tikzpicture}[x=0.75pt,y=0.75pt,yscale=-1,xscale=1]
	%uncomment if require: \path (0,300); %set diagram left start at 0, and has height of 300

	%Straight Lines [id:da10584314933739614] 
	\draw [line width=0.75]    (186.08,178) -- (476.92,178) ;
	%Straight Lines [id:da6776483318154041] 
	\draw [line width=1.5]    (212.5,77.67) -- (253.27,118.45) ;
	\draw [shift={(256.09,121.27)}, rotate = 225] [fill={rgb, 255:red, 0; green, 0; blue, 0 }  ][line width=0.08]  [draw opacity=0] (13.4,-6.43) -- (0,0) -- (13.4,6.44) -- (8.9,0) -- cycle    ;
	\draw [shift={(209.67,74.85)}, rotate = 45] [fill={rgb, 255:red, 0; green, 0; blue, 0 }  ][line width=0.08]  [draw opacity=0] (13.4,-6.43) -- (0,0) -- (13.4,6.44) -- (8.9,0) -- cycle    ;
	%Shape: Circle [id:dp5620783808094743] 
	\draw  [fill={rgb, 255:red, 0; green, 0; blue, 0 }  ,fill opacity=1 ] (400.75,121.93) .. controls (400.75,120.09) and (402.24,118.6) .. (404.08,118.6) .. controls (405.92,118.6) and (407.42,120.09) .. (407.42,121.93) .. controls (407.42,123.77) and (405.92,125.26) .. (404.08,125.26) .. controls (402.24,125.26) and (400.75,123.77) .. (400.75,121.93) -- cycle ;
	%Straight Lines [id:da5967449817586901] 
	\draw [line width=0.75]  [dash pattern={on 0.84pt off 2.51pt}]  (331.5,279.25) -- (331.5,67) ;
	%Straight Lines [id:da2020356289457561] 
	\draw [line width=1.5]    (232.88,69.23) -- (232.88,126.89) ;
	\draw [shift={(232.88,130.89)}, rotate = 270] [fill={rgb, 255:red, 0; green, 0; blue, 0 }  ][line width=0.08]  [draw opacity=0] (13.4,-6.43) -- (0,0) -- (13.4,6.44) -- (8.9,0) -- cycle    ;
	\draw [shift={(232.88,65.23)}, rotate = 90] [fill={rgb, 255:red, 0; green, 0; blue, 0 }  ][line width=0.08]  [draw opacity=0] (13.4,-6.43) -- (0,0) -- (13.4,6.44) -- (8.9,0) -- cycle    ;
	%Straight Lines [id:da3215170377627532] 
	\draw [line width=1.5]    (253.27,77.67) -- (212.5,118.45) ;
	\draw [shift={(209.67,121.27)}, rotate = 315] [fill={rgb, 255:red, 0; green, 0; blue, 0 }  ][line width=0.08]  [draw opacity=0] (13.4,-6.43) -- (0,0) -- (13.4,6.44) -- (8.9,0) -- cycle    ;
	\draw [shift={(256.09,74.85)}, rotate = 135] [fill={rgb, 255:red, 0; green, 0; blue, 0 }  ][line width=0.08]  [draw opacity=0] (13.4,-6.43) -- (0,0) -- (13.4,6.44) -- (8.9,0) -- cycle    ;
	%Straight Lines [id:da6609842584002097] 
	\draw [line width=1.5]    (261.71,98.06) -- (204.05,98.06) ;
	\draw [shift={(200.05,98.06)}, rotate = 360] [fill={rgb, 255:red, 0; green, 0; blue, 0 }  ][line width=0.08]  [draw opacity=0] (13.4,-6.43) -- (0,0) -- (13.4,6.44) -- (8.9,0) -- cycle    ;
	\draw [shift={(265.71,98.06)}, rotate = 180] [fill={rgb, 255:red, 0; green, 0; blue, 0 }  ][line width=0.08]  [draw opacity=0] (13.4,-6.43) -- (0,0) -- (13.4,6.44) -- (8.9,0) -- cycle    ;
	%Straight Lines [id:da020335654979837248] 
	\draw [line width=0.75]    (232.88,98.06) -- (331.5,178) ;
	\draw [shift={(282.19,138.03)}, rotate = 219.03] [fill={rgb, 255:red, 0; green, 0; blue, 0 }  ][line width=0.08]  [draw opacity=0] (10.72,-5.15) -- (0,0) -- (10.72,5.15) -- (7.12,0) -- cycle    ;
	%Straight Lines [id:da601886995433131] 
	\draw [line width=1.5]    (419.83,211.01) -- (460.6,251.78) ;
	\draw [shift={(463.43,254.61)}, rotate = 225] [fill={rgb, 255:red, 0; green, 0; blue, 0 }  ][line width=0.08]  [draw opacity=0] (13.4,-6.43) -- (0,0) -- (13.4,6.44) -- (8.9,0) -- cycle    ;
	\draw [shift={(417,208.18)}, rotate = 45] [fill={rgb, 255:red, 0; green, 0; blue, 0 }  ][line width=0.08]  [draw opacity=0] (13.4,-6.43) -- (0,0) -- (13.4,6.44) -- (8.9,0) -- cycle    ;
	%Straight Lines [id:da3974384306150227] 
	\draw [line width=1.5]    (440.21,202.56) -- (440.21,260.22) ;
	\draw [shift={(440.21,264.22)}, rotate = 270] [fill={rgb, 255:red, 0; green, 0; blue, 0 }  ][line width=0.08]  [draw opacity=0] (13.4,-6.43) -- (0,0) -- (13.4,6.44) -- (8.9,0) -- cycle    ;
	\draw [shift={(440.21,198.56)}, rotate = 90] [fill={rgb, 255:red, 0; green, 0; blue, 0 }  ][line width=0.08]  [draw opacity=0] (13.4,-6.43) -- (0,0) -- (13.4,6.44) -- (8.9,0) -- cycle    ;
	%Straight Lines [id:da2923488269969061] 
	\draw [line width=1.5]    (460.6,211.01) -- (419.83,251.78) ;
	\draw [shift={(417,254.61)}, rotate = 315] [fill={rgb, 255:red, 0; green, 0; blue, 0 }  ][line width=0.08]  [draw opacity=0] (13.4,-6.43) -- (0,0) -- (13.4,6.44) -- (8.9,0) -- cycle    ;
	\draw [shift={(463.43,208.18)}, rotate = 135] [fill={rgb, 255:red, 0; green, 0; blue, 0 }  ][line width=0.08]  [draw opacity=0] (13.4,-6.43) -- (0,0) -- (13.4,6.44) -- (8.9,0) -- cycle    ;
	%Straight Lines [id:da0769012257333288] 
	\draw [line width=1.5]    (469.04,231.39) -- (411.38,231.39) ;
	\draw [shift={(407.38,231.39)}, rotate = 360] [fill={rgb, 255:red, 0; green, 0; blue, 0 }  ][line width=0.08]  [draw opacity=0] (13.4,-6.43) -- (0,0) -- (13.4,6.44) -- (8.9,0) -- cycle    ;
	\draw [shift={(473.04,231.39)}, rotate = 180] [fill={rgb, 255:red, 0; green, 0; blue, 0 }  ][line width=0.08]  [draw opacity=0] (13.4,-6.43) -- (0,0) -- (13.4,6.44) -- (8.9,0) -- cycle    ;
	%Straight Lines [id:da16469269400468023] 
	\draw [line width=0.75]    (331.5,178) -- (440.21,231.39) ;
	\draw [shift={(385.86,204.7)}, rotate = 206.16] [fill={rgb, 255:red, 0; green, 0; blue, 0 }  ][line width=0.08]  [draw opacity=0] (10.72,-5.15) -- (0,0) -- (10.72,5.15) -- (7.12,0) -- cycle    ;
	%Straight Lines [id:da0759372633547084] 
	\draw [line width=0.75]    (331.5,178) -- (453.67,82.67) ;
	\draw [shift={(392.58,130.33)}, rotate = 502.03] [fill={rgb, 255:red, 0; green, 0; blue, 0 }  ][line width=0.08]  [draw opacity=0] (10.72,-5.15) -- (0,0) -- (10.72,5.15) -- (7.12,0) -- cycle    ;
	%Shape: Arc [id:dp652910950328399] 
	\draw  [draw opacity=0] (308,159.35) .. controls (313.5,152.43) and (321.98,148) .. (331.5,148) .. controls (341.27,148) and (349.96,152.68) .. (355.44,159.91) -- (331.5,178) -- cycle ; \draw   (308,159.35) .. controls (313.5,152.43) and (321.98,148) .. (331.5,148) .. controls (341.27,148) and (349.96,152.68) .. (355.44,159.91) ;

	% Text Node
	\draw (203.33,163.33) node    {$n_{1}$};
	% Text Node
	\draw (205.33,190) node    {$n_{2} \ $};
	% Text Node
	\draw (461.33,165.33) node    {$\Sigma $};
	% Text Node
	\draw (388.33,104.17) node    {$\vec{E}_{r}$};
	% Text Node
	\draw (314.67,139.33) node    {$\vartheta $};
	% Text Node
	\draw (347.33,138.67) node    {$\vartheta $};
	% Text Node
	\draw (481.33,117.33) node   [align=left] {polarizzazione\\orizzontale};


	\end{tikzpicture}
\end{figure}
\FloatBarrier







































\section{Incidenza normale alla superficie di separazione}

Quando l'angolo di incidenza è nullo la direzione di incidenza coincide con la normale alla superficie di separazione e la nozione di piano di incidenza perde significato. I campi elettrici dell'onda incidente, dell'onda riflessa e dell'onda trasmessa sono paralleli tra loro e alla superficie e le condizioni al contorno si riducono all'unica condizione:
\[
	\vec{E}_i+\vec{E}_r = \vec{E}_t
\]
Le formule di Fresnel saranno:
\[
	\boxed{r = \frac{E_r}{E_i} = \frac{n_1-n_2}{n_1+n_2} \qquad t = \frac{E_t}{E_i}= \frac{2n_1}{n_1+n_2}}
\]
Mentre $t$ è sempre positivo, $r$ è negativo se $n_1<n_2$ e positivo in caso contrario. Nel primo caso (es aria-vetro) il campo elettrico riflesso è opposto al campo elettrico incidente (sfasamento di $\pi$). Nel secondo caso campo elettrico riflesso e incidente sono concordi (in fase).
\begin{figure}[htpb]
	\centering
	

	\tikzset{every picture/.style={line width=0.75pt}} %set default line width to 0.75pt        

	\begin{tikzpicture}[x=0.75pt,y=0.75pt,yscale=-1,xscale=1]
	%uncomment if require: \path (0,300); %set diagram left start at 0, and has height of 300

	%Straight Lines [id:da8343733937814433] 
	\draw [line width=0.75]    (185.42,166.67) -- (476.25,166.67) ;
	%Straight Lines [id:da9542436852727807] 
	\draw [line width=1.5]    (326.83,62.67) -- (326.83,162.67) ;
	\draw [shift={(326.83,166.67)}, rotate = 270] [fill={rgb, 255:red, 0; green, 0; blue, 0 }  ][line width=0.08]  [draw opacity=0] (13.4,-6.43) -- (0,0) -- (13.4,6.44) -- (8.9,0) -- cycle    ;
	%Straight Lines [id:da124426243237312] 
	\draw [line width=1.5]    (334.83,66.67) -- (334.83,166.67) ;
	\draw [shift={(334.83,62.67)}, rotate = 90] [fill={rgb, 255:red, 0; green, 0; blue, 0 }  ][line width=0.08]  [draw opacity=0] (13.4,-6.43) -- (0,0) -- (13.4,6.44) -- (8.9,0) -- cycle    ;
	%Straight Lines [id:da1865350143004918] 
	\draw [line width=1.5]    (330.83,166.67) -- (330.83,266.67) ;
	\draw [shift={(330.83,270.67)}, rotate = 270] [fill={rgb, 255:red, 0; green, 0; blue, 0 }  ][line width=0.08]  [draw opacity=0] (13.4,-6.43) -- (0,0) -- (13.4,6.44) -- (8.9,0) -- cycle    ;
	%Straight Lines [id:da2872402626087014] 
	\draw [line width=1.5]    (330.83,166.67) -- (401,166.67) ;
	\draw [shift={(405,166.67)}, rotate = 180] [fill={rgb, 255:red, 0; green, 0; blue, 0 }  ][line width=0.08]  [draw opacity=0] (13.4,-6.43) -- (0,0) -- (13.4,6.44) -- (8.9,0) -- cycle    ;
	%Straight Lines [id:da8246612781024789] 
	\draw [line width=1.5]    (260.67,166.67) -- (330.83,166.67) ;
	\draw [shift={(256.67,166.67)}, rotate = 0] [fill={rgb, 255:red, 0; green, 0; blue, 0 }  ][line width=0.08]  [draw opacity=0] (13.4,-6.43) -- (0,0) -- (13.4,6.44) -- (8.9,0) -- cycle    ;

	% Text Node
	\draw (202.67,152) node    {$n_{1}$};
	% Text Node
	\draw (204.67,178.67) node    {$n_{2} \ $};
	% Text Node
	\draw (344.67,221.33) node    {$\vec{k}_{t}$};
	% Text Node
	\draw (349,98) node    {$\vec{k}_{r}$};
	% Text Node
	\draw (315,98) node    {$\vec{k}_{i}$};
	% Text Node
	\draw (378.33,149.67) node    {$\vec{E}_{r}$};
	% Text Node
	\draw (286.33,149.67) node    {$\vec{E}_{i}$};


	\end{tikzpicture}
\end{figure}
\FloatBarrier







































\section{Riflessione su una superficie metallica}

Le equazioni di Fresnel si possono usare anche quando il materiale assorbe (quindi quando l'indice di rifrazione è complesso). Il caso più interessante è quello in cui tale materiale è un conduttore. Ivi non c'è la forza di richiamo che avevamo introdotto nel caso della propagazione delle onde elettromagnetiche nel dielettrico: gli elettroni di conduzione infatti sono liberi. Possiamo quindi usare la formula dell'indice di rifrazione complesso ponendo $\omega_0=0$. Se non ci sono effetti dissipativi (conducibilità infinita) anche il coefficiente di attrito viscoso può essere posto pari a $0$. A questo punto l'indice di rifrazione di un metallo perfetto si può scrivere nella forma:
\begin{align*}
	n(\omega) &= \sqrt{1+\frac{Ne^2 (\omega_0^2 -\omega^2)-i\gamma \omega}{m_e\varepsilon_0 (\omega_0^2 -\omega^2)^2 + \gamma^2 \omega^2}} \tag*{$\gamma=0,\omega_0=0$}\\
	&= \sqrt{1+\frac{Ne^2}{m_e\varepsilon_0}\frac{(-\omega^2)}{\omega^4}}\\
	&= \sqrt{1-\frac{Ne^2}{m_e\varepsilon_0 \omega^2}}
\end{align*}
In questa situazione si introduce una quantità detta \textbf{pulsazione di plasma} del metallo. Essa è data dal rapporto:
\[
	\omega_p = \sqrt{\frac{Ne^2}{m_e\varepsilon_0}} \qquad \text{pulsazione di plasma}
\]
L'indice di rifrazione complesso si scrive allora come:
\[
	n(\omega) = \sqrt{1-\frac{\omega_p^2}{\omega^2}}
\]
Studiamo il segno della quantità sotto radice.
\begin{itemize}
	\item Per $\omega<\omega_p\implies n=i\,n_I(\omega)$. In tal caso la quantità sotto radice è negativa e l'indice di rifrazione è solo immaginario, il che comporta che non ci può essere propagazione nel metallo, ma solo assorbimento di energia.
	\item Per $\omega>\omega_p\implies n=i\,n_R(\omega)$. In tal caso la quantità sotto radice è positiva e l'indice di rifrazione è solo reale, non c'è assorbimento e il metallo diventa trasparente alle onde.
\end{itemize}
Dal momento che le pulsazioni di plasma hanno valori dell'ordine di $10^{16}\,rad/s$, le pulsazioni visibili obbediscono alla $\omega<\omega_p$ e quindi l'indice di rifrazione ha un valore immaginario puro. Sappiamo già che in questa situazione non può esserci propagazione nel metallo. Si possono usare le equazioni di Fresnel per dimostrare che una superficie metallica riflette quasi tutta l'energia luminosa che la colpisce:
\[
	\frac{E_r}{E_i} = \frac{n_1-n_2}{n_1+n_2} = \frac{1-in_I}{1+in_I} = \frac{\rho e^{-i\varphi}}{\rho e^{i\varphi}} = e^{-2i\varphi} \implies \left|\frac{E_r}{E_i}\right|^2 = \frac{I_r}{I_i} = \left( e^{-2i\varphi}  \right)^2 = 1
\]
I metalli che hanno il maggior coefficiente di riflessione nel visibile, sono l'argento e l'alluminio. Il fenomeno di riflessione delle onde elettromagnetiche su una superficie metallica è alla base del funzionamento degli specchi. Dato che l'alluminio, se esposto all'aria per tempi molto lunghi, resta sostanzialmente inalterato a differenza dell'argento, gli specchi vengono fatti preferibilmente con una sottile pellicola di alluminio.







































\section{Cenni ai fenomeni di interferenza fra onde elettromagnetiche}

Tutti i fenomeni di tipo ondulatorio generalmente sono associati a fenomeni di interferenza, termine riferito alla sovrapposizione di due o più onde elettromagnetiche della stessa natura. Per parlare di tale argomento, dobbiamo introdurre il concetto di \textbf{coerenza} fra due onde elettromagnetiche. Immaginiamo due di queste sinusoidali e piane che nella stessa regione di spazio si sovrappongono.
\[
	\vec{E}_1(\vec{r},t) = \vec{E}_{01}\cos (\underbrace{\vec{k}_1\cdot \vec{r} -\omega_1 t + \varphi_1}_{\Phi_1}) \qquad \vec{E}_2(\vec{r},t) = \vec{E}_{02}\cos (\underbrace{\vec{k}_2\cdot \vec{r} -\omega_2 t + \varphi_2}_{\Phi_2})
\]
Diremo che due onde elettromagnetiche sono fra loro coerenti se la differenza tra le due fasi non dipende dal tempo.
\[
	\Delta \Phi = \Phi_1 (\vec{r},t) - \Phi_2 (\vec{r}, t)
\]
Per analogia, sorgenti che generano onde coerenti sono a loro volta dette coerenti.
Se vogliamo che le onde siano coerenti, dobbiamo fare in modo che $t$ non compaia nell'espressione di $ \Delta \Phi  $
\[
	\Delta \Phi = (\vec{k}_1 - \vec{k}_2)\vec{r} -(\omega_1 - \omega_2) t + (\varphi_1 - \varphi _2  )
\]
Dobbiamo quindi prima di tutto richiedere che le due frequenze siano le stesse. Inoltre, quando si estende la trattazione a onde non polarizzate, $\varphi_1$ e $\varphi_2$ potrebbero variare nel tempo. Quindi si richiede anche che la loro differenza sia costante.\\
\textbf{Osservazione.} Si utilizzano sorgenti laser per ottenere onde coerenti.
Poniamo che ci siano due onde piane sinusoidali coerenti linearmente polarizzate nella stessa direzione (ad esempio l'asse $y$).
\[
	\vec{E}_1(\vec{r},t) = E_{01} \vec{u}_y \cos (\Phi_1) \qquad \vec{E}_2(\vec{r},t) = E_{02} \vec{u}_y \cos (\Phi_2)
\]
Avremo un'onda complessiva data dalla somma delle due onde. La loro sovrapposizione sarà:
\[
	\vec{E} (\vec{r},t) = \vec{E}_1(\vec{r},t) + \vec{E}_2(\vec{r},t)
\]
Chiediamoci quanto vale l'intensità di questa onda complessiva. L'intensità istantanea è pari al modulo del vettore di Poynting.
\begin{equation*}
	\begin{aligned}
		I(\vec{r},t) = |\vec{S}| &= \varepsilon_0 \varepsilon_r v |\vec{E}|^2 \\
		&=\varepsilon_0 \varepsilon_r v \left\{ E_{01}\cos (\Phi_1) + E_{02}\cos (\Phi_2 )  \right\}^2 \\
		&= \varepsilon_0 \varepsilon_r v \left\{ E_{01}^2 \cos^2 (\Phi_1 ) + E_{02}^2 \cos^2 (\Phi_2 ) + 2 E_{01}E_{02}\cos (\Phi_1 )\cos (\Phi_2 ) \right\} \\
		&= \varepsilon_0 \varepsilon_r v \{ E_{01}^2 \cos^2 (\Phi_1 ) + E_{02}^2 \cos^2 (\Phi_2 ) + E_{01}E_{02}\cos (\Phi_1 + \Phi_2) \\
		&\qquad\qquad + E_{01}E_{02}\cos (\Phi_1 - \Phi_2) \}
	\end{aligned}
\end{equation*}
Il valor medio del terzo termine è nullo perché si tratta di una funzione oscillante intorno allo zero. Il quarto non è in realtà una funzione oscillante perché, essendo le due onde coerenti, la differenza fra le fasi non dipende dal tempo ma è una costante.
Tenendo conto che il valor medio del coseno al quadrato è pari a 1/2, l'intensità media sarà data da:
\[
	I_m = \varepsilon_0 \varepsilon_r v \left\{ \frac{E_{01}^2}{2} + \frac{E_{02}^2}{2} + E_{01}E_{02} \cos (\Delta \Phi) \right\}
\]
Osserviamo che l'intensità rispettivamente delle due onde è data da:
\[
	I_{m1} = \frac{1}{2} \varepsilon_0 \varepsilon_r v E_{01}^2 \qquad I_{m2} = \frac{1}{2} \varepsilon_0 \varepsilon_r v E_{02}^2
\]
Si può dimostrare che
\[
	\varepsilon_0 \varepsilon_r v E_{02}E_{02} = 2 \sqrt{I_{m1}I_{m2}}
\]
L'intensità media totale sarà quindi
\[
	I_m = I_{m1} + I_{m2} + 2 \sqrt{I_{m1}I_{m2}} \cos (\Delta \Phi)
\]
Quando l'intensità totale dipende dalla differenza di fase fra le due onde elettromagnetiche, si parla di fenomeno di interferenza. L'intensità complessiva dovuta alla sovrapposizione delle due onde dipende dalla differenza di fase e non si ottiene come semplice somma delle singole intensità medie.
Quando le due onde non sono coerenti, anche la differenza di fase sarà funzione del tempo. Calcolando il valor medio della parte sottolineata in blu, esso verrebbe zero perché la funzione oscillerebbe. Per onde incoerenti, l'intensità totale è la somma delle intensità delle due onde. Si parla di sorgenti incoerenti quando esse generando onde incoerenti.
Affinché si manifestino fenomeni di interferenza, è molto importante che la direzione di polarizzazione sia la stessa. Consideriamo ad esempio due onde coerenti con polarizzazione ortogonale. Il campo elettrico sarà dato da:
\[
	\vec{E} (\vec{r},t) = E_{01}\vec{u}_y\cos (\vec{k}_1\cdot \vec{r} -\omega t + \varphi_1  ) + E_{02}\vec{u}_z\cos (\vec{k}_2\cdot \vec{r} -\omega t + \varphi_2  )
\]
Quando calcoliamo l'intensità istantanea abbiamo:
\[
	I = \varepsilon_0 \varepsilon_r v |\vec{E}|^2 = \varepsilon_0 \varepsilon_r v(\vec{E} \cdot \vec{E})
\]
Calcolando il prodotto scalare avremmo in teoria quattro termini. Quelli in cui compare il prodotto scalare fra i due versori $ \vec{u}_z $ e $ \vec{u}_y $ sono pari a zero. Risulterà:
\[
	I = E_{01}^2 \cos^2 (\vec{k}_1\cdot \vec{r} -\omega t + \varphi_1) + E_{02}^2 \cos^2 (\vec{k}_2+\omega t + \varphi_2 = I_{m1}+I_{m2}
\]
Le onde devono quindi essere polarizzate lungo la stessa direzione perché possa comparire il doppio prodotto fra la loro intensità.
Consideriamo il semplice caso in cui le due onde hanno la stessa intensità $ I_{m1}=I_{m2}=I_0    $.
\[
	I=I_0+I_0+2\sqrt{I_0^2} \cos \left[ \Delta \Phi (\vec{r}) \right] = 2I_0 \left[ 1+\cos \Delta \Phi (\vec{r}) \right]
\]
Vediamo che cosa può accadere a questa quantità in due casi diversi.
\begin{itemize}
	\item \textbf{Caso 1.} La fase è un multiplo pari di $ \pi  $.
	\[
		\Delta \Phi = 2\pi m\qquad m=0,\pm 1,\pm 2\ldots  \implies \cos [\Delta \Phi ]=1 \quad I=4I_0
	\]
	Si parla in tal caso di \textbf{interferenza costruttiva}: le onde sono in fase: la differenza di cammino è un multiplo intero della lunghezza d'onda.
	\begin{figure}[htpb]
		\centering
		

		\tikzset{every picture/.style={line width=0.75pt}} %set default line width to 0.75pt        

		\begin{tikzpicture}[x=0.75pt,y=0.75pt,yscale=-1,xscale=1]
		%uncomment if require: \path (0,300); %set diagram left start at 0, and has height of 300

		%Curve Lines [id:da925891687150687] 
		\draw [line width=1.5]    (214.5,101.75) .. controls (224.71,101.5) and (224.14,179.5) .. (234.5,179.75) .. controls (244.86,180) and (245.86,102.64) .. (255,102.75) .. controls (264.14,102.86) and (264.43,179.5) .. (275.5,179.25) .. controls (286.57,179) and (285.86,101.21) .. (296,101.25) .. controls (306.14,101.29) and (306.43,179.21) .. (316.5,179.25) .. controls (326.57,179.29) and (326.14,102.07) .. (336.5,102.25) .. controls (346.86,102.43) and (346.43,179.5) .. (357,179.75) .. controls (367.57,180) and (367.29,102.64) .. (377.5,102.75) ;
		%Shape: Axis 2D [id:dp85463191663918] 
		\draw  (193,181.1) -- (404.5,181.1)(214.15,65) -- (214.15,194) (397.5,176.1) -- (404.5,181.1) -- (397.5,186.1) (209.15,72) -- (214.15,65) -- (219.15,72)  ;

		% Text Node
		\draw (424,179) node    {$\Delta \Phi $};
		% Text Node
		\draw (204,63) node    {$I$};
		% Text Node
		\draw (199,100) node    {$4I_{0}$};


		\end{tikzpicture}
	\end{figure}
	\FloatBarrier
	\item \textbf{Caso 2.} La fase è un multiplo dispari di $ \pi  $.
	\[
		\Delta \Phi = (2m+1)\pi \implies \cos [\Delta \Phi ]=-1 \quad I=0
	\]
	In tal caso si parla di \textbf{interferenza distruttiva}. Le onde sono in opposizione di fase e ciò comporta che la differenza di cammino sia un multiplo dispari di mezza lunghezza d'onda. In tutti gli altri casi In l'intensità totale varia fra questi due estremi.
\end{itemize}







































\section{Interferenza fra onde emesse da due sorgenti luminose puntiformi}

Un tipico esempio di sorgente puntiforme di onde elettromagnetiche è una stella. Immaginiamo che la potenza irradiata dalla sorgente sia $W_0$.
\begin{figure}[htpb]
	\centering
	

	\tikzset{every picture/.style={line width=0.75pt}} %set default line width to 0.75pt        

	\begin{tikzpicture}[x=0.75pt,y=0.75pt,yscale=-0.8,xscale=0.8]
	%uncomment if require: \path (0,300); %set diagram left start at 0, and has height of 300

	%Shape: Circle [id:dp7799479554093001] 
	\draw  [dash pattern={on 0.84pt off 2.51pt}] (243,165.75) .. controls (243,135.51) and (267.51,111) .. (297.75,111) .. controls (327.99,111) and (352.5,135.51) .. (352.5,165.75) .. controls (352.5,195.99) and (327.99,220.5) .. (297.75,220.5) .. controls (267.51,220.5) and (243,195.99) .. (243,165.75) -- cycle ;
	%Straight Lines [id:da746600712772753] 
	\draw  [dash pattern={on 0.84pt off 2.51pt}]  (297.75,165.75) -- (259.04,127.04) ;

	% Text Node
	\draw (291,139) node    {$r$};
	% Text Node
	\draw (215,127) node    {$P$};
	% Text Node
	\draw (314,202) node    {$S$};


	\end{tikzpicture}
\end{figure}
\FloatBarrier
Poniamo di voler calcolare l'intensità della radiazione che giunge nel punto $P$. La sorgente puntiforme irradia potenza in tutte le direzioni in modo uniforme, è sufficiente pensare che essa si distribuisce in una superficie sferica di raggio $r$, quindi l'intensità si può calcolare semplicemente come:
\[
	I=\frac{W_0}{4\pi r^2}
\]
Essa decresce con il quadrato della distanza perché si distribuisce su una superficie sempre più grande. Siccome $I$ è uguale in modulo al vettore di Poynting, che è proporzionale al modulo del campo elettrico dell'onda al quadrato, $\vec{E}$ deve decrescere come $1/r$. L'onda elettromagnetica emessa prende il nome di \textbf{onda sferica}. Le sue caratteristiche sono:
\begin{itemize}
	\item $\vec{E}$ decresce con la distanza.
	\item Mentre nelle onde piane ha senso parlare di direzione di propagazione, nel caso di un'onda sferica non c'è direzione specifica per identificarla. In questo caso useremo il modulo del vettore $\vec{k} $.
\end{itemize}
Le onde sferiche si possono identificare in questo modo:
\[
	E(\vec{r},t) = \frac{A}{r} \cos (kr-\omega t)
\]
A questo punto se abbiamo due sorgenti puntiformi e vogliamo studiare il fenomeno di interferenza nel punto $P$ procediamo calcolando le intensità delle radiazioni nel punto $P$ emesse da $S_1$ ed $S_2$ rispettivamente.
\begin{gather*}
	I_1(P) = \frac{W_0}{4\pi r_1^2} \qquad I_2(P) = \frac{W_0}{4\pi r_2^2} \\
	I(P) = I_1(P) + I_2(P) + 2\sqrt{I_1I_2} \cos \Delta \Phi \\
	\Delta \Phi = kr_1-kr_2 = k(r_1-r_2  )
\end{gather*}
Vediamo quando si ha interferenza costruttiva:
\begin{align*}
	\Delta \Phi &= 2\pi m \\
	k(r_1-r_2) &= 2\pi m \\
	\frac{2\pi}{\lambda}(r_1-r_2) &= 2\pi m \\
	\frac{1}{\lambda}(r_1-r_2) &= m \\
	\Aboxed{r_1-r_2 &= m\lambda}
\end{align*}
Si trova l'equazione di una famiglia di iperboli. A seconda del valore di m avremo differenti iperboli, i cui fuochi si trovano sempre su $S_1$ ed $S_2$. Tali curve sono il luogo dei punti in cui si ha interferenza costruttiva.
\begin{figure}[htpb]
	\centering
	

	\tikzset{every picture/.style={line width=0.75pt}} %set default line width to 0.75pt        

	\begin{tikzpicture}[x=0.75pt,y=0.75pt,yscale=-1,xscale=1]
	%uncomment if require: \path (0,300); %set diagram left start at 0, and has height of 300

	%Shape: Circle [id:dp08060543997148284] 
	\draw  [color={rgb, 255:red, 222; green, 222; blue, 222 }  ,draw opacity=1 ] (265,74.09) .. controls (276.19,74.09) and (285.25,83.16) .. (285.25,94.34) .. controls (285.25,105.53) and (276.19,114.6) .. (265,114.6) .. controls (253.81,114.6) and (244.75,105.53) .. (244.75,94.34) .. controls (244.75,83.16) and (253.81,74.09) .. (265,74.09) -- cycle ;
	%Shape: Circle [id:dp3870702516407525] 
	\draw  [color={rgb, 255:red, 222; green, 222; blue, 222 }  ,draw opacity=1 ] (265,179.4) .. controls (276.19,179.4) and (285.25,188.47) .. (285.25,199.66) .. controls (285.25,210.84) and (276.19,219.91) .. (265,219.91) .. controls (253.81,219.91) and (244.75,210.84) .. (244.75,199.66) .. controls (244.75,188.47) and (253.81,179.4) .. (265,179.4) -- cycle ;
	%Shape: Arc [id:dp7818118033781998] 
	\draw  [draw opacity=0] (233.92,219.13) .. controls (230.38,213.49) and (228.33,206.81) .. (228.33,199.66) .. controls (228.33,179.41) and (244.75,162.99) .. (265,162.99) .. controls (285.25,162.99) and (301.67,179.41) .. (301.67,199.66) .. controls (301.67,206.83) and (299.61,213.52) .. (296.04,219.18) -- (265,199.66) -- cycle ; \draw  [color={rgb, 255:red, 222; green, 222; blue, 222 }  ,draw opacity=1 ] (233.92,219.13) .. controls (230.38,213.49) and (228.33,206.81) .. (228.33,199.66) .. controls (228.33,179.41) and (244.75,162.99) .. (265,162.99) .. controls (285.25,162.99) and (301.67,179.41) .. (301.67,199.66) .. controls (301.67,206.83) and (299.61,213.52) .. (296.04,219.18) ;
	%Shape: Arc [id:dp2550254959984686] 
	\draw  [draw opacity=0] (219.96,227.58) .. controls (214.92,219.47) and (212.01,209.91) .. (212.01,199.66) .. controls (212.01,170.39) and (235.73,146.67) .. (265,146.67) .. controls (294.27,146.67) and (317.99,170.39) .. (317.99,199.66) .. controls (317.99,210.08) and (314.98,219.8) .. (309.79,227.99) -- (265,199.66) -- cycle ; \draw  [color={rgb, 255:red, 222; green, 222; blue, 222 }  ,draw opacity=1 ] (219.96,227.58) .. controls (214.92,219.47) and (212.01,209.91) .. (212.01,199.66) .. controls (212.01,170.39) and (235.73,146.67) .. (265,146.67) .. controls (294.27,146.67) and (317.99,170.39) .. (317.99,199.66) .. controls (317.99,210.08) and (314.98,219.8) .. (309.79,227.99) ;
	%Shape: Arc [id:dp06300146565821207] 
	\draw  [draw opacity=0] (206.14,236.32) .. controls (199.5,225.68) and (195.67,213.12) .. (195.67,199.66) .. controls (195.67,161.36) and (226.71,130.32) .. (265,130.32) .. controls (303.29,130.32) and (334.33,161.36) .. (334.33,199.66) .. controls (334.33,213.1) and (330.51,225.65) .. (323.88,236.28) -- (265,199.66) -- cycle ; \draw  [color={rgb, 255:red, 222; green, 222; blue, 222 }  ,draw opacity=1 ] (206.14,236.32) .. controls (199.5,225.68) and (195.67,213.12) .. (195.67,199.66) .. controls (195.67,161.36) and (226.71,130.32) .. (265,130.32) .. controls (303.29,130.32) and (334.33,161.36) .. (334.33,199.66) .. controls (334.33,213.1) and (330.51,225.65) .. (323.88,236.28) ;
	%Shape: Arc [id:dp9877380941575995] 
	\draw  [draw opacity=0] (205.6,138.39) .. controls (220.97,123.49) and (241.91,114.32) .. (265,114.32) .. controls (312.13,114.32) and (350.33,152.53) .. (350.33,199.66) .. controls (350.33,216.16) and (345.65,231.56) .. (337.54,244.62) -- (265,199.66) -- cycle ; \draw  [color={rgb, 255:red, 222; green, 222; blue, 222 }  ,draw opacity=1 ] (205.6,138.39) .. controls (220.97,123.49) and (241.91,114.32) .. (265,114.32) .. controls (312.13,114.32) and (350.33,152.53) .. (350.33,199.66) .. controls (350.33,216.16) and (345.65,231.56) .. (337.54,244.62) ;
	%Shape: Arc [id:dp5004358798621307] 
	\draw  [draw opacity=0] (192.85,125.23) .. controls (211.51,107.14) and (236.96,96) .. (265,96) .. controls (322.25,96) and (368.66,142.41) .. (368.66,199.66) .. controls (368.66,219.93) and (362.84,238.85) .. (352.77,254.82) -- (265,199.66) -- cycle ; \draw  [color={rgb, 255:red, 222; green, 222; blue, 222 }  ,draw opacity=1 ] (192.85,125.23) .. controls (211.51,107.14) and (236.96,96) .. (265,96) .. controls (322.25,96) and (368.66,142.41) .. (368.66,199.66) .. controls (368.66,219.93) and (362.84,238.85) .. (352.77,254.82) ;
	%Shape: Arc [id:dp2166898603176528] 
	\draw  [draw opacity=0] (181.01,113.02) .. controls (202.73,91.95) and (232.35,78.99) .. (265,78.99) .. controls (331.64,78.99) and (385.67,133.01) .. (385.67,199.66) .. controls (385.67,223.07) and (379,244.93) .. (367.46,263.43) -- (265,199.66) -- cycle ; \draw  [color={rgb, 255:red, 222; green, 222; blue, 222 }  ,draw opacity=1 ] (181.01,113.02) .. controls (202.73,91.95) and (232.35,78.99) .. (265,78.99) .. controls (331.64,78.99) and (385.67,133.01) .. (385.67,199.66) .. controls (385.67,223.07) and (379,244.93) .. (367.46,263.43) ;
	%Shape: Arc [id:dp4781416417159392] 
	\draw  [draw opacity=0] (169.87,101.53) .. controls (194.48,77.67) and (228.02,62.99) .. (265,62.99) .. controls (340.48,62.99) and (401.67,124.18) .. (401.67,199.66) .. controls (401.67,226.23) and (394.08,251.04) .. (380.95,272.03) -- (265,199.66) -- cycle ; \draw  [color={rgb, 255:red, 222; green, 222; blue, 222 }  ,draw opacity=1 ] (169.87,101.53) .. controls (194.48,77.67) and (228.02,62.99) .. (265,62.99) .. controls (340.48,62.99) and (401.67,124.18) .. (401.67,199.66) .. controls (401.67,226.23) and (394.08,251.04) .. (380.95,272.03) ;
	%Shape: Arc [id:dp25918293710577345] 
	\draw  [draw opacity=0] (233.16,74.35) .. controls (229.68,79.96) and (227.67,86.58) .. (227.67,93.67) .. controls (227.67,113.92) and (244.08,130.33) .. (264.33,130.33) .. controls (284.58,130.33) and (301,113.92) .. (301,93.67) .. controls (301,86.67) and (299.04,80.12) .. (295.63,74.56) -- (264.33,93.67) -- cycle ; \draw  [color={rgb, 255:red, 222; green, 222; blue, 222 }  ,draw opacity=1 ] (233.16,74.35) .. controls (229.68,79.96) and (227.67,86.58) .. (227.67,93.67) .. controls (227.67,113.92) and (244.08,130.33) .. (264.33,130.33) .. controls (284.58,130.33) and (301,113.92) .. (301,93.67) .. controls (301,86.67) and (299.04,80.12) .. (295.63,74.56) ;
	%Shape: Arc [id:dp37519196802558397] 
	\draw  [draw opacity=0] (218.97,66.26) .. controls (214.13,74.26) and (211.34,83.64) .. (211.34,93.67) .. controls (211.34,122.93) and (235.07,146.66) .. (264.33,146.66) .. controls (293.6,146.66) and (317.32,122.93) .. (317.32,93.67) .. controls (317.32,83.75) and (314.6,74.46) .. (309.85,66.52) -- (264.33,93.67) -- cycle ; \draw  [color={rgb, 255:red, 222; green, 222; blue, 222 }  ,draw opacity=1 ] (218.97,66.26) .. controls (214.13,74.26) and (211.34,83.64) .. (211.34,93.67) .. controls (211.34,122.93) and (235.07,146.66) .. (264.33,146.66) .. controls (293.6,146.66) and (317.32,122.93) .. (317.32,93.67) .. controls (317.32,83.75) and (314.6,74.46) .. (309.85,66.52) ;
	%Shape: Arc [id:dp45144955151664035] 
	\draw  [draw opacity=0] (205.62,56.78) .. controls (198.89,67.46) and (195,80.11) .. (195,93.67) .. controls (195,131.96) and (226.04,163) .. (264.33,163) .. controls (302.63,163) and (333.67,131.96) .. (333.67,93.67) .. controls (333.67,80.4) and (329.94,68.01) .. (323.48,57.48) -- (264.33,93.67) -- cycle ; \draw  [color={rgb, 255:red, 222; green, 222; blue, 222 }  ,draw opacity=1 ] (205.62,56.78) .. controls (198.89,67.46) and (195,80.11) .. (195,93.67) .. controls (195,131.96) and (226.04,163) .. (264.33,163) .. controls (302.63,163) and (333.67,131.96) .. (333.67,93.67) .. controls (333.67,80.4) and (329.94,68.01) .. (323.48,57.48) ;
	%Shape: Arc [id:dp8154313281969809] 
	\draw  [draw opacity=0] (204.94,154.94) .. controls (220.3,169.83) and (241.25,179) .. (264.33,179) .. controls (311.46,179) and (349.67,140.79) .. (349.67,93.67) .. controls (349.67,77.31) and (345.06,62.02) .. (337.08,49.04) -- (264.33,93.67) -- cycle ; \draw  [color={rgb, 255:red, 222; green, 222; blue, 222 }  ,draw opacity=1 ] (204.94,154.94) .. controls (220.3,169.83) and (241.25,179) .. (264.33,179) .. controls (311.46,179) and (349.67,140.79) .. (349.67,93.67) .. controls (349.67,77.31) and (345.06,62.02) .. (337.08,49.04) ;
	%Shape: Arc [id:dp9855388845867168] 
	\draw  [draw opacity=0] (192.18,168.09) .. controls (210.84,186.19) and (236.29,197.32) .. (264.33,197.32) .. controls (321.58,197.32) and (367.99,150.91) .. (367.99,93.67) .. controls (367.99,73.67) and (362.33,55) .. (352.52,39.16) -- (264.33,93.67) -- cycle ; \draw  [color={rgb, 255:red, 222; green, 222; blue, 222 }  ,draw opacity=1 ] (192.18,168.09) .. controls (210.84,186.19) and (236.29,197.32) .. (264.33,197.32) .. controls (321.58,197.32) and (367.99,150.91) .. (367.99,93.67) .. controls (367.99,73.67) and (362.33,55) .. (352.52,39.16) ;
	%Shape: Arc [id:dp942189796709634] 
	\draw  [draw opacity=0] (180.34,180.31) .. controls (202.07,201.37) and (231.69,214.33) .. (264.33,214.33) .. controls (330.98,214.33) and (385,160.31) .. (385,93.67) .. controls (385,70.53) and (378.49,48.92) .. (367.2,30.56) -- (264.33,93.67) -- cycle ; \draw  [color={rgb, 255:red, 222; green, 222; blue, 222 }  ,draw opacity=1 ] (180.34,180.31) .. controls (202.07,201.37) and (231.69,214.33) .. (264.33,214.33) .. controls (330.98,214.33) and (385,160.31) .. (385,93.67) .. controls (385,70.53) and (378.49,48.92) .. (367.2,30.56) ;
	%Shape: Arc [id:dp9829940753304982] 
	\draw  [draw opacity=0] (169.21,191.79) .. controls (193.81,215.65) and (227.36,230.33) .. (264.33,230.33) .. controls (339.81,230.33) and (401,169.15) .. (401,93.67) .. controls (401,67.43) and (393.61,42.92) .. (380.79,22.11) -- (264.33,93.67) -- cycle ; \draw  [color={rgb, 255:red, 222; green, 222; blue, 222 }  ,draw opacity=1 ] (169.21,191.79) .. controls (193.81,215.65) and (227.36,230.33) .. (264.33,230.33) .. controls (339.81,230.33) and (401,169.15) .. (401,93.67) .. controls (401,67.43) and (393.61,42.92) .. (380.79,22.11) ;

	%Straight Lines [id:da039950294145592036] 
	\draw    (410.16,147) -- (119.84,147) ;
	%Straight Lines [id:da687475725369987] 
	\draw    (265,94.34) -- (361.5,58.25) ;
	%Straight Lines [id:da7389326208092715] 
	\draw    (265,199.66) -- (361.5,58.25) ;
	%Curve Lines [id:da25129407023789674] 
	\draw    (120,119.25) .. controls (235.5,144.75) and (295.5,143.25) .. (411,120.25) ;
	%Curve Lines [id:da7407192064205514] 
	\draw    (120,88.25) .. controls (239.5,144.75) and (287.5,142.75) .. (411,88.25) ;
	%Curve Lines [id:da1779033021420895] 
	\draw    (120,53.25) .. controls (256.5,145.25) and (271.5,144.75) .. (411,53.25) ;
	%Curve Lines [id:da04466478827647724] 
	\draw    (120,12.25) .. controls (262.5,147.25) and (267.5,147.25) .. (411,12.25) ;
	%Curve Lines [id:da27523224407567537] 
	\draw    (120,173.94) .. controls (235.5,148.44) and (295.5,149.94) .. (411,172.94) ;
	%Curve Lines [id:da04300412010855448] 
	\draw    (120,204.94) .. controls (239.5,148.44) and (287.5,150.44) .. (411,204.94) ;
	%Curve Lines [id:da7619666524257553] 
	\draw    (120,239.94) .. controls (256.5,147.94) and (271.5,148.44) .. (411,239.94) ;
	%Curve Lines [id:da9700408206673654] 
	\draw    (120,280.94) .. controls (262.5,145.94) and (267.5,145.94) .. (411,280.94) ;
	%Straight Lines [id:da9707297020712791] 
	\draw    (420.16,8.63) -- (420.16,285.38) ;
	%Shape: Circle [id:dp3454026835092512] 
	\draw  [fill={rgb, 255:red, 0; green, 0; blue, 0 }  ,fill opacity=1 ] (361.5,56.08) .. controls (362.7,56.08) and (363.67,57.05) .. (363.67,58.25) .. controls (363.67,59.45) and (362.7,60.42) .. (361.5,60.42) .. controls (360.3,60.42) and (359.33,59.45) .. (359.33,58.25) .. controls (359.33,57.05) and (360.3,56.08) .. (361.5,56.08) -- cycle ;
	%Curve Lines [id:da7309507900387369] 
	\draw [line width=1.5]    (501.42,79.22) .. controls (501.67,87.74) and (423.67,87.26) .. (423.42,95.9) .. controls (423.17,104.53) and (500.52,105.36) .. (500.42,112.99) .. controls (500.31,120.61) and (423.67,120.85) .. (423.92,130.08) .. controls (424.17,139.31) and (501.95,138.71) .. (501.92,147.17) .. controls (501.88,155.62) and (423.95,155.86) .. (423.92,164.26) .. controls (423.88,172.65) and (501.1,172.3) .. (500.92,180.93) .. controls (500.74,189.57) and (423.67,189.21) .. (423.42,198.02) .. controls (423.17,206.83) and (500.52,206.6) .. (500.42,215.11) ;
	%Curve Lines [id:da26551712155483265] 
	\draw [line width=1.5]    (501.42,79.22) .. controls (501.67,87.74) and (423.67,87.26) .. (423.42,95.9) .. controls (423.17,104.53) and (500.52,105.36) .. (500.42,112.99) .. controls (500.31,120.61) and (423.67,120.85) .. (423.92,130.08) .. controls (424.17,139.31) and (501.95,138.71) .. (501.92,147.17) .. controls (501.88,155.62) and (423.95,155.86) .. (423.92,164.26) .. controls (423.88,172.65) and (501.1,172.3) .. (500.92,180.93) .. controls (500.74,189.57) and (423.67,189.21) .. (423.42,198.02) .. controls (423.17,206.83) and (500.52,206.6) .. (500.42,215.11) ;
	%Curve Lines [id:da3892415292835949] 
	\draw [line width=1.5]    (502.92,11.28) .. controls (503.17,20.51) and (424.95,19.97) .. (424.92,28.37) .. controls (424.88,36.76) and (502.1,36.41) .. (501.92,45.04) .. controls (501.74,53.68) and (424.67,53.32) .. (424.42,62.13) .. controls (424.17,70.95) and (501.52,70.71) .. (501.42,79.22) ;
	%Curve Lines [id:da3954813803501307] 
	\draw [line width=1.5]    (500.42,215.11) .. controls (500.67,223.63) and (422.67,223.15) .. (422.42,231.79) .. controls (422.17,240.42) and (499.52,241.25) .. (499.42,248.88) .. controls (499.31,256.5) and (422.67,256.74) .. (422.92,265.97) .. controls (423.17,275.2) and (500.95,274.6) .. (500.92,283.06) ;

	% Text Node
	\draw (259.5,87) node  [font=\scriptsize]  {$S_{1}$};
	% Text Node
	\draw (260,203.5) node  [font=\scriptsize]  {$S_{2}$};
	% Text Node
	\draw (335.5,60) node  [font=\scriptsize]  {$r_{1}$};
	% Text Node
	\draw (357.5,75) node  [font=\scriptsize]  {$r_{2}$};
	% Text Node
	\draw (354,53) node  [font=\scriptsize]  {$P$};


	\end{tikzpicture}
\end{figure}
\FloatBarrier







































\section{Esperimento di interferenza}

Quello che si fa è considerare uno schermo e valutare l'andamento dell'intensità generata dalle due onde su di esso, tale andamento prende il nome di \textbf{figura di interferenza}. Li dove c'è interferenza costruttiva avremo massima intensità.
Per trovare tale andamento si ricorre all'approssimazione di considerare lo schermo molto lontano dalle sorgenti. Fissiamo la distanza $D$ fra di esse, consideriamo un punto $P$ sullo schermo e tracciamo la distanza di $S_1$ ed $S_2$ da tale punto. Se lo schermo è molto lontano, $\vec{r}_1$ ed $\vec{r}_2$ sono quasi paralleli e l'angolo formato con la normale a $D$ è praticamente lo stesso ($\vartheta$). Possiamo esprimere in funzione di tale angolo l'andamento a cui siamo interessati. Tracciando un segmento ortogonale a $\vec{r}_2$, in prima approssimazione la proiezione di $D$ su $\vec{r}_2$ si può interpretare come la differenza $r_1-r_2$:
\begin{gather*}
	r_1-r_2 \simeq D\sin \vartheta \\
	\Delta \Phi =k (r_2-r_1) = kD\sin \vartheta \\
	I(\vartheta) = I_1+I_2+2\sqrt{I_1I_2} \cos [\Delta \Phi ]
\end{gather*}
\begin{figure}[htpb]
	\centering
	

	\tikzset{every picture/.style={line width=0.75pt}} %set default line width to 0.75pt        

	\begin{tikzpicture}[x=0.75pt,y=0.75pt,yscale=-0.9,xscale=0.9]
	%uncomment if require: \path (0,300); %set diagram left start at 0, and has height of 300

	%Shape: Circle [id:dp9911458442903904] 
	\draw  [fill={rgb, 255:red, 0; green, 0; blue, 0 }  ,fill opacity=1 ] (245,149.67) .. controls (245,148.19) and (246.19,147) .. (247.67,147) .. controls (249.14,147) and (250.33,148.19) .. (250.33,149.67) .. controls (250.33,151.14) and (249.14,152.33) .. (247.67,152.33) .. controls (246.19,152.33) and (245,151.14) .. (245,149.67) -- cycle ;
	%Shape: Circle [id:dp9902185672506356] 
	\draw  [fill={rgb, 255:red, 0; green, 0; blue, 0 }  ,fill opacity=1 ] (245,229.67) .. controls (245,228.19) and (246.19,227) .. (247.67,227) .. controls (249.14,227) and (250.33,228.19) .. (250.33,229.67) .. controls (250.33,231.14) and (249.14,232.33) .. (247.67,232.33) .. controls (246.19,232.33) and (245,231.14) .. (245,229.67) -- cycle ;
	%Straight Lines [id:da6057152889101978] 
	\draw    (509,81.5) -- (509,245.5) ;
	%Straight Lines [id:da4873729451655593] 
	\draw    (509,92.5) -- (247.67,229.67) ;
	%Straight Lines [id:da8139723340308482] 
	\draw    (509,92.5) -- (247.67,149.67) ;
	%Straight Lines [id:da30832811432688634] 
	\draw    (330.5,149.67) -- (247.67,149.67) ;
	%Straight Lines [id:da4562670545035852] 
	\draw    (431,176.67) -- (348.17,176.67) ;
	%Shape: Arc [id:dp7854871689199094] 
	\draw  [draw opacity=0] (293.52,140.1) .. controls (294.14,143.07) and (294.47,146.14) .. (294.5,149.29) -- (247.67,149.67) -- cycle ; \draw   (293.52,140.1) .. controls (294.14,143.07) and (294.47,146.14) .. (294.5,149.29) ;
	%Shape: Arc [id:dp12821468762939858] 
	\draw  [draw opacity=0] (389.71,155.02) .. controls (393.03,161.39) and (394.94,168.62) .. (395,176.29) -- (348.17,176.67) -- cycle ; \draw   (389.71,155.02) .. controls (393.03,161.39) and (394.94,168.62) .. (395,176.29) ;
	%Straight Lines [id:da6348505628449739] 
	\draw    (247.67,155.33) -- (247.67,224) ;
	\draw [shift={(247.67,227)}, rotate = 270] [fill={rgb, 255:red, 0; green, 0; blue, 0 }  ][line width=0.08]  [draw opacity=0] (10.72,-5.15) -- (0,0) -- (10.72,5.15) -- (7.12,0) -- cycle    ;
	\draw [shift={(247.67,152.33)}, rotate = 90] [fill={rgb, 255:red, 0; green, 0; blue, 0 }  ][line width=0.08]  [draw opacity=0] (10.72,-5.15) -- (0,0) -- (10.72,5.15) -- (7.12,0) -- cycle    ;
	%Straight Lines [id:da39999949135956214] 
	\draw    (280.5,212.22) -- (247.67,149.67) ;
	%Shape: Arc [id:dp47411815496431053] 
	\draw  [draw opacity=0] (262.44,177.87) .. controls (258.08,180.16) and (253.12,181.47) .. (247.86,181.5) -- (247.67,149.67) -- cycle ; \draw   (262.44,177.87) .. controls (258.08,180.16) and (253.12,181.47) .. (247.86,181.5) ;
	%Straight Lines [id:da17885638933016845] 
	\draw    (285.5,222.22) -- (252.67,239.67) ;
	\draw [shift={(252.67,239.67)}, rotate = 332.02] [color={rgb, 255:red, 0; green, 0; blue, 0 }  ][line width=0.75]    (0,5.59) -- (0,-5.59)   ;
	\draw [shift={(285.5,222.22)}, rotate = 332.02] [color={rgb, 255:red, 0; green, 0; blue, 0 }  ][line width=0.75]    (0,5.59) -- (0,-5.59)   ;

	% Text Node
	\draw (230.83,148.67) node  [font=\normalsize]  {$S_{1}$};
	% Text Node
	\draw (232.67,228.5) node  [font=\normalsize]  {$S_{2}$};
	% Text Node
	\draw (281.33,129.67) node  [font=\normalsize]  {$\vartheta $};
	% Text Node
	\draw (403.83,161.17) node  [font=\normalsize]  {$\vartheta $};
	% Text Node
	\draw (371.83,106.17) node  [font=\normalsize]  {$\vec{r}_{1}$};
	% Text Node
	\draw (449.83,140.17) node  [font=\normalsize]  {$\vec{r}_{2}$};
	% Text Node
	\draw (234.83,185.67) node  [font=\normalsize]  {$D$};
	% Text Node
	\draw (257.33,189.67) node  [font=\normalsize]  {$\vartheta $};
	% Text Node
	\draw (296.83,239.17) node  [font=\normalsize]  {$r_{1} -r_{2}$};


	\end{tikzpicture}
\end{figure}
\FloatBarrier
$I_1$ e $I_2$ in $P$ sarebbero chiaramente diverse, ma se ci concentriamo su una regione sullo schermo abbastanza prossima al centro, per valori di $\vartheta$ quindi non troppo grandi, sostanzialmente le intensità sono uguali se le sorgenti sono identiche.
\begin{align*}
	I(\vartheta) &= 2I_0 [ 1 + \cos (\Delta \Phi)] \\
	&= 2I_0 [ 1 + \cos (\overbrace{kD\sin \vartheta}^{\Delta \Phi})] \tag*{$\frac{1+\cos \alpha}{2}=\cos^2 \left( \frac{\alpha}{2} \right)  $}\\
	&= 4I_0\cos^2 \left( \frac{kD\sin \vartheta}{2} \right)
\end{align*}
I massimi di interferenza si trovano come segue:
\begin{align*}
	\Delta \Phi &= 2\pi m \\
	kD\sin \vartheta &= 2\pi m \\
	\frac{2\pi}{\lambda}D\sin \vartheta &=2\pi m \\
	\Aboxed{\sin \vartheta &=m\frac{\lambda}{D}}
\end{align*}







































\section{Inferenza in lamine sottili di materiale dielettrico}

L'interferenza dovuta alla riflessione della luce sulle due superfici di una lamina sottile di una sostanza trasparente è il caso di interferenza più facilmente osservabile nella vita comune. Supponiamo di disporre di una lamina molto sottile di indice di rifrazione $n_2$. Tale lamina è immersa in un altro dielettrico, ad esempio l'aria, di indice $n_1$. Sia $d$ lo spessore della lamina.
\begin{figure}[htpb]
	\centering
	

	\tikzset{every picture/.style={line width=0.75pt}} %set default line width to 0.75pt        

	\begin{tikzpicture}[x=0.75pt,y=0.75pt,yscale=-1,xscale=1]
	%uncomment if require: \path (0,300); %set diagram left start at 0, and has height of 300

	%Shape: Rectangle [id:dp24401556706323402] 
	\draw   (179,142) -- (503,142) -- (503,170.5) -- (179,170.5) -- cycle ;
	%Straight Lines [id:da6143374172020759] 
	\draw    (313,52.5) -- (313,139) ;
	\draw [shift={(313,142)}, rotate = 270] [fill={rgb, 255:red, 0; green, 0; blue, 0 }  ][line width=0.08]  [draw opacity=0] (10.72,-5.15) -- (0,0) -- (10.72,5.15) -- (7.12,0) -- cycle    ;
	%Straight Lines [id:da4538070003023835] 
	\draw    (333,142) -- (333,167.5) ;
	\draw [shift={(333,170.5)}, rotate = 270] [fill={rgb, 255:red, 0; green, 0; blue, 0 }  ][line width=0.08]  [draw opacity=0] (10.72,-5.15) -- (0,0) -- (10.72,5.15) -- (7.12,0) -- cycle    ;
	%Straight Lines [id:da5117695983840902] 
	\draw    (341,145) -- (341,170.5) ;
	\draw [shift={(341,142)}, rotate = 90] [fill={rgb, 255:red, 0; green, 0; blue, 0 }  ][line width=0.08]  [draw opacity=0] (10.72,-5.15) -- (0,0) -- (10.72,5.15) -- (7.12,0) -- cycle    ;
	%Straight Lines [id:da8396691056997594] 
	\draw    (341,55.5) -- (341,142) ;
	\draw [shift={(341,52.5)}, rotate = 90] [fill={rgb, 255:red, 0; green, 0; blue, 0 }  ][line width=0.08]  [draw opacity=0] (10.72,-5.15) -- (0,0) -- (10.72,5.15) -- (7.12,0) -- cycle    ;
	%Straight Lines [id:da16556866549700122] 
	\draw    (324,55.5) -- (324,142) ;
	\draw [shift={(324,52.5)}, rotate = 90] [fill={rgb, 255:red, 0; green, 0; blue, 0 }  ][line width=0.08]  [draw opacity=0] (10.72,-5.15) -- (0,0) -- (10.72,5.15) -- (7.12,0) -- cycle    ;
	%Straight Lines [id:da5274788703770195] 
	\draw    (323,170.5) -- (323,215.67) ;
	\draw [shift={(323,218.67)}, rotate = 270] [fill={rgb, 255:red, 0; green, 0; blue, 0 }  ][line width=0.08]  [draw opacity=0] (10.72,-5.15) -- (0,0) -- (10.72,5.15) -- (7.12,0) -- cycle    ;

	% Text Node
	\draw (324.67,38.33) node    {$1$};
	% Text Node
	\draw (341.33,38.33) node    {$2$};
	% Text Node
	\draw (200.67,127.33) node    {$n_{1}$};
	% Text Node
	\draw (219.67,154.33) node    {$n_{2}  >n_{1}$};


	\end{tikzpicture}
\end{figure}
\FloatBarrier
Vi è un onda che incide perpendicolarmente. Una parte della luce incidente sulla lamina è riflessa dalla superficie superiore (onda $2$); l'onda trasmessa si propaga nella lamina ed è parzialmente riflessa dalla superficie inferiore. La parte riflessa (onda $1$) riattraversa la lamina e riemerge nel primo mezzo con direzione parallela a quella del primo raggio riflesso. Le due onde giungono all'occhio sfasate: l'onda $2$ per la differenza di cammino ottico $2d$, l'onda $1$ per lo sfasamento di $\pi$ subito alla prima riflessione (dato che $n_1<n_2$). Le due fasi saranno date da:
\[
	\Phi_2 = \pi \qquad \Phi_1 = k\cdot 2d
\]
Le due onde sono sfasate di $\pi$ per quanto detto sull'incidenza normale alla superficie di separazione.
Utilizzando le forme di Fresnel troviamo che:
\[
	\frac{E_r}{E_i} = \frac{n_1-n_2}{n_1+n_2}<0 \qquad \frac{E_r'}{E_i'} = \frac{n_2-n_1}{n_1+n_2}>0
\]
Le due onde sono coerenti, quindi è possibile osservare interferenza fra queste due onde riflesse.
\[
	\Phi_1 = \frac{2\pi n_2}{\lambda}\cdot 2d=\frac{4\pi n_2d}{\lambda}
\]
Calcolando la differenza di fase fra le due onde:
\[
	\Delta \Phi =\Phi_1-\Phi_2 =\frac{4\pi n_2d}{\lambda}  - \pi
\]
Cerchiamo quali sono le condizioni per avere interferenza costruttiva.
\begin{align*}
	\Delta \Phi &= 2\pi m\\
	\frac{4\pi n_2d}{\lambda}  - \pi  &= 2\pi m\\
	\frac{4 n_2d}{\lambda}  - 1  &= 2 m\\
	\frac{4n_2d}{\lambda} &= 2m+1 \\
	\Aboxed{\lambda &= \frac{4n_2d}{2m+1}}
\end{align*}
Questa relazione ci permette di vedere a quale lunghezza d'onda abbiamo interferenza costruttiva. Supponiamo di illuminare la lamina con della luce bianca. La formula ci dice che vedremo la riflessione solo di quelle lunghezze d'onda che obbediscono all'interferenza costruttiva. I colori che si vedono ad esempio riflessi sulle bolle di sapone corrispondono all'interferenza costruttiva della loro superficie. Ci saranno poi altre lunghezze d'onda che daranno interferenza distruttiva e che quindi non vedremo perché si cancellano.







































\section{Cenni ai fenomeni diffrattivi}

La diffrazione è una particolare forma di interferenza che appare quando un'onda elettromagnetica incide su un ostacolo o su uno schermo in cui è stata praticata un apertura. Qualitativamente, nello spazio oltre l'ostacolo o l'apertura, le onde si propagano anche lungo direzioni diverse da quella di incidenza e hanno origine differenze di cammino tra onde che si sovrappongono in un dato punto; ecco perché possono avvenire fenomeni di interferenza con conseguente ridistribuzione dell'energia nei punti dello spazio. Gli effetti della diffrazione sono di norma tanto più vistosi quanto più le dimensioni dell'apertura o dell'ostacolo sono vicine al valore della lunghezza d'onda delle onde incidenti. I fenomeni di diffrazione si verificano con tutti i tipi di onde; essi si osservano facilmente nel caso di onde sulla superficie di un liquido e delle onde sonore, aventi lunghezze d'onda prossime alle dimensioni di oggetti molto comuni. Più difficile è l'osservazione nel caso di onde luminose, proprio a causa della piccola lunghezza d'onda. Per capire come si propaghino oltre ad un ostacolo o un foro, sarebbe possibile partire dalle equazioni di Maxwell e risolvere il problema con le condizioni al contorno. Vista la complessità di questo metodo, si utilizza però un approccio più semplice, basato sul \textbf{principio di Huygens-Fresnel}. Esso, che era stato introdotto in forma semi empirica, durante il 19esimo secolo è stato dimostrato in forma rigorosa tramite il teorema di Kirchhoff. Supponiamo di avere un'onda elettromagnetica sinusoidale. Consideriamo un fronte d'onda di essa (la fase è costante in tutti i suoi punti) e suddividiamolo in aree infinitesime. Ciascuna di queste può essere considerata come una sorgente puntiforme di onde sferiche che propagano nello spazio. Il principio afferma che il valore dell'onda in un punto $P$ si può calcolare come sovrapposizione dei campi elettrici prodotti dalle sorgenti puntiformi associate alle aree. Dobbiamo determinare come esprimere ciascuna di queste onde sferiche.
\begin{figure}[htpb]
	\centering
	

	\tikzset{every picture/.style={line width=0.75pt}} %set default line width to 0.75pt        

	\begin{tikzpicture}[x=0.75pt,y=0.75pt,yscale=-1,xscale=1]
	%uncomment if require: \path (0,300); %set diagram left start at 0, and has height of 300

	%Straight Lines [id:da4386397553414383] 
	\draw    (199,65.5) -- (199,139.5) ;
	%Straight Lines [id:da4550684617529315] 
	\draw    (199,85.5) -- (553,85.5) ;
	%Straight Lines [id:da5825124480043755] 
	\draw    (199,85.5) -- (553,125.5) ;
	\draw [shift={(376,105.5)}, rotate = 186.45] [fill={rgb, 255:red, 0; green, 0; blue, 0 }  ][line width=0.08]  [draw opacity=0] (10.72,-5.15) -- (0,0) -- (10.72,5.15) -- (7.12,0) -- cycle    ;
	%Shape: Arc [id:dp1439929142239722] 
	\draw  [draw opacity=0] (403,85.32) .. controls (403,85.38) and (403,85.44) .. (403,85.5) .. controls (403,93.41) and (402.55,101.21) .. (401.67,108.88) -- (199,85.5) -- cycle ; \draw   (403,85.32) .. controls (403,85.38) and (403,85.44) .. (403,85.5) .. controls (403,93.41) and (402.55,101.21) .. (401.67,108.88) ;
	%Shape: Circle [id:dp40378381740495084] 
	\draw  [fill={rgb, 255:red, 0; green, 0; blue, 0 }  ,fill opacity=1 ] (551,125.5) .. controls (551,124.4) and (551.9,123.5) .. (553,123.5) .. controls (554.1,123.5) and (555,124.4) .. (555,125.5) .. controls (555,126.6) and (554.1,127.5) .. (553,127.5) .. controls (551.9,127.5) and (551,126.6) .. (551,125.5) -- cycle ;
	%Shape: Circle [id:dp7859178245765386] 
	\draw  [fill={rgb, 255:red, 0; green, 0; blue, 0 }  ,fill opacity=1 ] (197,85.5) .. controls (197,84.4) and (197.9,83.5) .. (199,83.5) .. controls (200.1,83.5) and (201,84.4) .. (201,85.5) .. controls (201,86.6) and (200.1,87.5) .. (199,87.5) .. controls (197.9,87.5) and (197,86.6) .. (197,85.5) -- cycle ;

	% Text Node
	\draw (411.33,97) node    {$\vartheta $};
	% Text Node
	\draw (568,125) node    {$P$};
	% Text Node
	\draw (186,83.67) node    {$Q$};
	% Text Node
	\draw (200.67,55) node   [align=left] {fronte d'onda};


	\end{tikzpicture}
\end{figure}
\FloatBarrier
Consideriamo un punto $Q$ del fronte d'onda e tracciamo la congiungente tra il punto $Q$ ed il punto $P$. Introduciamo un versore normale a tale fronte e chiamiamo $\vartheta$ l'angolo che la congiungente forma con esso. Secondo tale principio, l'ampiezza del campo nel punto $P$ prodotto dalla sorgente puntiforme fittizia in $Q$ si ottiene come:
\[
	dE(P)=\frac{E(Q)}{r\lambda}f(\vartheta) dS \qquad f(\vartheta)=\frac{1+\cos \vartheta}{2}
\]
Questa funzione è detta \textbf{funzione di obliquità}. Il principio afferma anche che l'onda secondaria (cioè quella emessa da $Q$) è in anticipo di fase di $\pi/2$ rispetto all'onda principale (ossia l'onda di cui stiamo studiando la propagazione). Calcolando il campo nel punto $P$, sommando le varie onde:
\begin{align*}
	\vec{E} (P)&=\int_{\Sigma}dE(P)\sin \left( kr-\omega t - \frac{\pi}{2} \right)\\
	&=\int_{\Sigma}  \frac{E(Q)}{r\lambda}f(\vartheta) \sin \left( kr-\omega t - \frac{\pi}{2} \right) dS
\end{align*}
Notiamo che:
\begin{itemize}
	\item Per $\vartheta=\pi$ stiamo considerando le onde che si propagano all'indietro, $f(\vartheta)=0$ minimo di $f$.
	\item Per $\vartheta=0$, $f(\vartheta)=1$ massimo di $f$.
\end{itemize}
Possiamo capire come l'integrale sopra calcolato dia maggiore importanza ai contributi che si propagano in avanti ed esclude quelli che tornano indietro.
\textbf{Osservazione.} Avendo a che fare con onde sferiche, non si utilizza l'espressione vettoriale per i campi. La teoria della diffrazione prende il nome anche di teoria scalare.







































\section{Diffrazione ad una fenditura rettilinea}

Vediamo cosa accade nel caso di interazione con un ostacolo. Consideriamo l'interazione con una fenditura praticata in uno schermo opaco, di forma rettangolare di altezza $h$, larghezza $d$. Su tale schermo incide un'onda piana elettromagnetica e sinusoidale.
\begin{figure}[htpb]
	\centering
	

	\tikzset{every picture/.style={line width=0.75pt}} %set default line width to 0.75pt        

	\begin{tikzpicture}[x=0.75pt,y=0.75pt,yscale=-1,xscale=1]
	%uncomment if require: \path (0,300); %set diagram left start at 0, and has height of 300

	%Curve Lines [id:da5914380383052462] 
	\draw    (382.5,36.25) .. controls (382,61.25) and (411.5,51.25) .. (411.5,65.25) .. controls (411.5,79.25) and (378.62,84.15) .. (379.5,93.25) .. controls (380.38,102.35) and (446.5,103.75) .. (446.5,114.75) .. controls (446.5,125.75) and (375.5,129.75) .. (376,138.25) .. controls (376.5,146.75) and (474.5,143.25) .. (474.5,156) ;
	%Curve Lines [id:da04832983911044875] 
	\draw    (382.5,275.75) .. controls (382,250.75) and (411.5,260.75) .. (411.5,246.75) .. controls (411.5,232.75) and (378.62,227.85) .. (379.5,218.75) .. controls (380.38,209.65) and (446.5,208.25) .. (446.5,197.25) .. controls (446.5,186.25) and (375.5,182.25) .. (376,173.75) .. controls (376.5,165.25) and (474.5,168.75) .. (474.5,156) ;
	%Straight Lines [id:da546076973690863] 
	\draw    (370.5,36.25) -- (370.5,275.75) ;
	%Straight Lines [id:da7212551859058829] 
	\draw [line width=2.25]    (220.5,36.25) -- (220.5,140) ;
	%Straight Lines [id:da466265345750674] 
	\draw [line width=2.25]    (220.5,172) -- (220.5,275.75) ;
	%Straight Lines [id:da9852288833275191] 
	\draw [line width=1.5]    (130.5,86.25) -- (130.5,225.75) ;
	%Straight Lines [id:da08948640210316205] 
	\draw [line width=1.5]    (160.5,86.25) -- (160.5,225.75) ;
	%Straight Lines [id:da9783729066556608] 
	\draw [line width=1.5]    (100.5,86.25) -- (100.5,225.75) ;
	%Straight Lines [id:da9867791784292883] 
	\draw [line width=0.75]    (187.25,156) -- (78.33,156) ;
	\draw [shift={(190.25,156)}, rotate = 180] [fill={rgb, 255:red, 0; green, 0; blue, 0 }  ][line width=0.08]  [draw opacity=0] (10.72,-5.15) -- (0,0) -- (10.72,5.15) -- (7.12,0) -- cycle    ;
	%Shape: Arc [id:dp5890001897873898] 
	\draw  [draw opacity=0][line width=1.5]  (230.5,127.71) .. controls (242.15,131.83) and (250.5,142.94) .. (250.5,156) .. controls (250.5,169.28) and (241.87,180.55) .. (229.9,184.5) -- (220.5,156) -- cycle ; \draw  [line width=1.5]  (230.5,127.71) .. controls (242.15,131.83) and (250.5,142.94) .. (250.5,156) .. controls (250.5,169.28) and (241.87,180.55) .. (229.9,184.5) ;
	%Shape: Arc [id:dp3857484668049591] 
	\draw  [draw opacity=0][line width=1.5]  (241.84,95.64) .. controls (266.69,104.43) and (284.5,128.14) .. (284.5,156) .. controls (284.5,184.34) and (266.08,208.38) .. (240.56,216.79) -- (220.5,156) -- cycle ; \draw  [line width=1.5]  (241.84,95.64) .. controls (266.69,104.43) and (284.5,128.14) .. (284.5,156) .. controls (284.5,184.34) and (266.08,208.38) .. (240.56,216.79) ;
	%Shape: Arc [id:dp6477878378582584] 
	\draw  [draw opacity=0][line width=1.5]  (254.45,59.96) .. controls (294,73.95) and (322.33,111.66) .. (322.33,156) .. controls (322.33,201.09) and (293.02,239.34) .. (252.42,252.73) -- (220.5,156) -- cycle ; \draw  [line width=1.5]  (254.45,59.96) .. controls (294,73.95) and (322.33,111.66) .. (322.33,156) .. controls (322.33,201.09) and (293.02,239.34) .. (252.42,252.73) ;




	\end{tikzpicture}
\end{figure}
\FloatBarrier
Assumeremo $h\gg d$.
Nei punti in cui non c'è fenditura l'onda viene assorbita. Possiamo applicare il principio di Huygens-Fresnel per calcolare l'andamento dell'onda oltre la fenditura.
Consideriamo un certo punto $P$ dello schermo. Introduciamo un asse ortogonale ad esso. Individuiamo la posizione di $P$ con un angolo formato dalla congiungente $QP$ con la normale. Suddividiamo la fenditura in tante aree. Essendo essa rettangolare, possiamo immaginare che esse siano strisce di lunghezza $h$ e ampiezza $dx$.
\begin{figure}[htpb]
	\centering
	

	\tikzset{every picture/.style={line width=0.75pt}} %set default line width to 0.75pt        

	\begin{tikzpicture}[x=0.75pt,y=0.75pt,yscale=-1,xscale=1]
	%uncomment if require: \path (0,300); %set diagram left start at 0, and has height of 300

	%Straight Lines [id:da7994316663783338] 
	\draw    (470.5,46.25) -- (470.5,245.75) ;
	%Straight Lines [id:da3631398435488242] 
	\draw [line width=2.25]    (200.5,46.25) -- (200.5,110) ;
	%Straight Lines [id:da3478792651437992] 
	\draw [line width=2.25]    (200.5,182) -- (200.5,245.75) ;
	%Straight Lines [id:da4931368342341502] 
	\draw [line width=1.5]    (120.5,76.25) -- (120.5,215.75) ;
	%Straight Lines [id:da6396195075323743] 
	\draw [line width=1.5]    (150.5,76.25) -- (150.5,215.75) ;
	%Straight Lines [id:da46727844054075396] 
	\draw [line width=1.5]    (90.5,76.25) -- (90.5,215.75) ;
	%Straight Lines [id:da6344169929086054] 
	\draw [line width=0.75]    (177.25,146) -- (68.33,146) ;
	\draw [shift={(180.25,146)}, rotate = 180] [fill={rgb, 255:red, 0; green, 0; blue, 0 }  ][line width=0.08]  [draw opacity=0] (10.72,-5.15) -- (0,0) -- (10.72,5.15) -- (7.12,0) -- cycle    ;
	%Straight Lines [id:da9436036410898272] 
	\draw    (210.5,81.25) -- (210.5,182) ;
	\draw [shift={(210.5,78.25)}, rotate = 90] [fill={rgb, 255:red, 0; green, 0; blue, 0 }  ][line width=0.08]  [draw opacity=0] (10.72,-5.15) -- (0,0) -- (10.72,5.15) -- (7.12,0) -- cycle    ;
	%Straight Lines [id:da8833480033622181] 
	\draw    (470.33,146) -- (210.25,146) ;
	%Straight Lines [id:da8508207041439235] 
	\draw    (463.33,236) -- (209.25,236) ;
	\draw [shift={(206.25,236)}, rotate = 360] [fill={rgb, 255:red, 0; green, 0; blue, 0 }  ][line width=0.08]  [draw opacity=0] (10.72,-5.15) -- (0,0) -- (10.72,5.15) -- (7.12,0) -- cycle    ;
	\draw [shift={(466.33,236)}, rotate = 180] [fill={rgb, 255:red, 0; green, 0; blue, 0 }  ][line width=0.08]  [draw opacity=0] (10.72,-5.15) -- (0,0) -- (10.72,5.15) -- (7.12,0) -- cycle    ;
	%Straight Lines [id:da6080543816878032] 
	\draw    (216,110) -- (205.5,110) ;
	%Straight Lines [id:da2800689018761018] 
	\draw    (470.33,226) -- (210.25,146) ;
	%Straight Lines [id:da21794098535866158] 
	\draw    (470.33,226) -- (210.5,182) ;
	%Shape: Arc [id:dp7291768055910737] 
	\draw  [draw opacity=0] (294,146.25) .. controls (293.97,154.8) and (292.67,163.06) .. (290.26,170.83) -- (210.25,146) -- cycle ; \draw   (294,146.25) .. controls (293.97,154.8) and (292.67,163.06) .. (290.26,170.83) ;
	%Shape: Circle [id:dp5488133986890094] 
	\draw  [fill={rgb, 255:red, 0; green, 0; blue, 0 }  ,fill opacity=1 ] (468.67,226) .. controls (468.67,225.08) and (469.41,224.33) .. (470.33,224.33) .. controls (471.25,224.33) and (472,225.08) .. (472,226) .. controls (472,226.92) and (471.25,227.67) .. (470.33,227.67) .. controls (469.41,227.67) and (468.67,226.92) .. (468.67,226) -- cycle ;
	%Shape: Square [id:dp46177419755755844] 
	\draw   (461,136.67) -- (470.33,136.67) -- (470.33,146) -- (461,146) -- cycle ;
	%Straight Lines [id:da9124192741007744] 
	\draw    (210.5,182) -- (220.6,149.16) ;
	%Straight Lines [id:da9027137350461123] 
	\draw    (224.78,138.1) -- (214.05,134.8) ;
	\draw [shift={(214.05,134.8)}, rotate = 377.1] [color={rgb, 255:red, 0; green, 0; blue, 0 }  ][line width=0.75]    (0,5.59) -- (0,-5.59)   ;
	\draw [shift={(224.78,138.1)}, rotate = 377.1] [color={rgb, 255:red, 0; green, 0; blue, 0 }  ][line width=0.75]    (0,5.59) -- (0,-5.59)   ;
	%Shape: Arc [id:dp5116988879360505] 
	\draw  [draw opacity=0] (210.56,165.3) .. controls (212.23,165.31) and (213.83,165.56) .. (215.35,166.01) -- (210.5,182) -- cycle ; \draw   (210.56,165.3) .. controls (212.23,165.31) and (213.83,165.56) .. (215.35,166.01) ;

	% Text Node
	\draw (210.67,193.33) node    {$O$};
	% Text Node
	\draw (212,67.33) node    {$x$};
	% Text Node
	\draw (200,36) node   [align=left] {schermo opaco};
	% Text Node
	\draw (460,36) node   [align=left] {schermo osservatore};
	% Text Node
	\draw (226,107.33) node    {$d$};
	% Text Node
	\draw (339.33,248.67) node    {$L\gg d$};
	% Text Node
	\draw (255.5,195.83) node    {$r_{0}$};
	% Text Node
	\draw (378.3,179.63) node    {$r\sim r_{0}$};
	% Text Node
	\draw (121.5,234.33) node    {$\lambda ,\omega ,E_{0}$};
	% Text Node
	\draw (300.5,157.73) node    {$\vartheta $};
	% Text Node
	\draw (479.5,225.33) node    {$P$};
	% Text Node
	\draw (254.6,129.33) node    {$x\sin \vartheta $};
	% Text Node
	\draw (221.5,166.33) node    {$\vartheta $};


	\end{tikzpicture}
\end{figure}
\FloatBarrier
\[
	dS=h\,dx \qquad E(P)=\int_{\text{fenditura}}  \frac{E_0}{r\lambda}f(\vartheta) \sin \left( kr-\omega t - \frac{\pi}{2} \right) h\,dx
\]
Se assumiamo lo schermo molto lontano, sostanzialmente $r$ può essere considerata costante per tutti i punti sulla fenditura.
Possiamo porre poi $\vartheta$ circa pari a zero. Con questa approssimazione il campo in $P$ sarà dato da:
\[
	E(P) \simeq \frac{E_0h}{r_0 \lambda} \int_{\text{fenditura}} \sin \left( kr-\omega t - \frac{\pi}{2} \right) dx
\]
L'onda emessa avrà andamento:
\[
	\sin (kr_0-\omega t -\pi /2 )
\]
Essendo lo schermo molto lontano, $\vec{r}$ e $\vec{r}_0$ possono considerarsi circa paralleli e l'angolo mostrato in figura è di conseguenza circa pari a $\vartheta$. Grazie a questa approssimazione, possiamo esprimere la lunghezza del tratto indicato in figura come: $x\sin \vartheta$ e l'onda emessa dal generico punto $Q$ può essere scritta come:
\[
	\sin (kr_0+kx\sin \vartheta -\pi /2 - \omega t)
\]
Ed $E(P)$ sarà dato da:
\[
	E(P) \simeq \frac{E_0h}{r_0 \lambda} \int_0^d  \sin \left( kr_0 +kx\sin \vartheta  -\omega t - \frac{\pi}{2} \right) dx
\]
Chiamiamo $\varphi_0 $ la quantità:
\[
	\varphi_0 = kr_0  -\omega t - \frac{\pi}{2}
\]
E proseguiamo col calcolo:
\begin{align*}
	E(P) &\simeq \frac{E_0h}{r_0 \lambda} \int_0^d  \sin (kx\sin \vartheta  +\varphi_0) dx \\
	&\simeq \frac{E_0h}{r_0 \lambda} \left[ -\frac{\cos (\varphi_0+kx\sin \vartheta)}{k\sin \vartheta} \right]_0^d \\
	&\simeq \frac{E_0h}{r_0\lambda k\sin \vartheta} [-\cos (kd\sin \vartheta +\varphi_0 ) + \cos \varphi_0 ] \tag*{(1)}\\
	&\simeq \frac{E_0h}{r_0\lambda k \sin \vartheta} \cdot 2 \sin \left( \frac{2\varphi_0+kd\sin \vartheta}{2} \right)  \sin \left( -\frac{kd\sin \vartheta}{2} \right)\\
	&\simeq \frac{E_0h}{r_0\lambda \underbrace{\frac{2\pi}{\lambda}}_k \sin \vartheta} \cdot 2 \sin \left( \frac{2\varphi_0+kd\sin \vartheta}{2} \right)  \sin \left( -\frac{kd\sin \vartheta}{2} \right)\\
	&\simeq \frac{E_0h}{\pi r_0\sin \vartheta}\sin \left( kr_0-\omega t - \frac{\pi}{2} + \frac{kd\sin \vartheta}{2}  \right) \sin \left( -\frac{kd\sin \vartheta}{2} \right)
\end{align*}
Dove nel passaggio $(1)$ abbiamo usato l'identità:
\[
	\cos \alpha -\cos \beta =2\sin \left( \frac{\alpha +\beta}{2} \right) \sin \left( \frac{\alpha -\beta}{2} \right)
\]
Siamo a questo punto interessanti al calcolo dell'intensità media dell'onda:
\begin{align*}
	I_m(P) &= \varepsilon_0 \varepsilon_r vE^2 (P) \\
	&= \varepsilon_0 c E^2 (P) \\
	&= \varepsilon_0 c(E^2)_m \\
	&= \varepsilon_0 c \frac{E_0^2 h^2}{\pi^2 r_0^2 \sin^2 \vartheta} \cdot \frac{1}{2} \cdot \sin^2 \left( \frac{kd\sin \vartheta}{2} \right) \\
	&= \varepsilon_0 c \frac{E_0^2 h^2}{\pi^2 r_0^2 2} \cdot  \frac{\sin^2 \left( \frac{kd\sin \vartheta}{2} \right)}{\frac{k^2 d^2 \sin^2 \vartheta}{4}} \cdot \frac{k^2 d^2}{4} \tag*{$\alpha=\frac{kd\sin \vartheta}{2}$} \\
	&= \frac{E_0^2 h^2 k^2 d^2 \varepsilon_0 c}{8\pi^2 r_0^2} \left( \frac{\sin^2 \alpha}{\alpha^2} \right) \\
	\Aboxed{I_m(P) &= I_0\frac{\sin^2 \alpha}{\alpha^2}}
\end{align*}
Capiamo come sullo schermo osservatore non vedremo l'ombra netta della fenditura ma una striscia centrale più intensa (detta \emph{immagine della fenditura}) e delle frangette laterali che via via decrescono.
\begin{figure}[htpb]
	\centering
	

	\tikzset{every picture/.style={line width=0.75pt}} %set default line width to 0.75pt        

	\begin{tikzpicture}[x=0.75pt,y=0.75pt,yscale=-1,xscale=1]
	%uncomment if require: \path (0,300); %set diagram left start at 0, and has height of 300

	%Shape: Rectangle [id:dp4129016214176424] 
	\draw  [draw opacity=0][fill={rgb, 255:red, 128; green, 128; blue, 128 }  ,fill opacity=1 ] (174.33,50.5) -- (507.67,50.5) -- (507.67,236.5) -- (174.33,236.5) -- cycle ;
	%Rounded Rect [id:dp030416714934673017] 
	\draw  [draw opacity=0][fill={rgb, 255:red, 222; green, 222; blue, 222 }  ,fill opacity=1 ] (328,74.5) .. controls (328,67.32) and (333.82,61.5) .. (341,61.5) -- (341,61.5) .. controls (348.18,61.5) and (354,67.32) .. (354,74.5) -- (354,212.5) .. controls (354,219.68) and (348.18,225.5) .. (341,225.5) -- (341,225.5) .. controls (333.82,225.5) and (328,219.68) .. (328,212.5) -- cycle ;
	%Rounded Rect [id:dp3423247111627783] 
	\draw  [draw opacity=0][fill={rgb, 255:red, 222; green, 222; blue, 222 }  ,fill opacity=1 ] (374.33,70.17) .. controls (374.33,65.38) and (378.21,61.5) .. (383,61.5) -- (383,61.5) .. controls (387.79,61.5) and (391.67,65.38) .. (391.67,70.17) -- (391.67,216.83) .. controls (391.67,221.62) and (387.79,225.5) .. (383,225.5) -- (383,225.5) .. controls (378.21,225.5) and (374.33,221.62) .. (374.33,216.83) -- cycle ;
	%Rounded Rect [id:dp3403280424120414] 
	\draw  [draw opacity=0][fill={rgb, 255:red, 222; green, 222; blue, 222 }  ,fill opacity=1 ] (415,67.5) .. controls (415,64.19) and (417.69,61.5) .. (421,61.5) -- (421,61.5) .. controls (424.31,61.5) and (427,64.19) .. (427,67.5) -- (427,219.5) .. controls (427,222.81) and (424.31,225.5) .. (421,225.5) -- (421,225.5) .. controls (417.69,225.5) and (415,222.81) .. (415,219.5) -- cycle ;
	%Rounded Rect [id:dp13891059068839184] 
	\draw  [draw opacity=0][fill={rgb, 255:red, 222; green, 222; blue, 222 }  ,fill opacity=1 ] (452.67,64.83) .. controls (452.67,62.99) and (454.16,61.5) .. (456,61.5) -- (456,61.5) .. controls (457.84,61.5) and (459.33,62.99) .. (459.33,64.83) -- (459.33,222.17) .. controls (459.33,224.01) and (457.84,225.5) .. (456,225.5) -- (456,225.5) .. controls (454.16,225.5) and (452.67,224.01) .. (452.67,222.17) -- cycle ;
	%Rounded Rect [id:dp5743528555322956] 
	\draw  [draw opacity=0][fill={rgb, 255:red, 222; green, 222; blue, 222 }  ,fill opacity=1 ] (308,216.83) .. controls (308,221.62) and (304.12,225.5) .. (299.33,225.5) -- (299.33,225.5) .. controls (294.55,225.5) and (290.67,221.62) .. (290.67,216.83) -- (290.67,70.17) .. controls (290.67,65.38) and (294.55,61.5) .. (299.33,61.5) -- (299.33,61.5) .. controls (304.12,61.5) and (308,65.38) .. (308,70.17) -- cycle ;
	%Rounded Rect [id:dp07505878796940846] 
	\draw  [draw opacity=0][fill={rgb, 255:red, 222; green, 222; blue, 222 }  ,fill opacity=1 ] (267.33,219.5) .. controls (267.33,222.81) and (264.65,225.5) .. (261.33,225.5) -- (261.33,225.5) .. controls (258.02,225.5) and (255.33,222.81) .. (255.33,219.5) -- (255.33,67.5) .. controls (255.33,64.19) and (258.02,61.5) .. (261.33,61.5) -- (261.33,61.5) .. controls (264.65,61.5) and (267.33,64.19) .. (267.33,67.5) -- cycle ;
	%Rounded Rect [id:dp9042984730324282] 
	\draw  [draw opacity=0][fill={rgb, 255:red, 222; green, 222; blue, 222 }  ,fill opacity=1 ] (229.67,222.17) .. controls (229.67,224.01) and (228.17,225.5) .. (226.33,225.5) -- (226.33,225.5) .. controls (224.49,225.5) and (223,224.01) .. (223,222.17) -- (223,64.83) .. controls (223,62.99) and (224.49,61.5) .. (226.33,61.5) -- (226.33,61.5) .. controls (228.17,61.5) and (229.67,62.99) .. (229.67,64.83) -- cycle ;




	\end{tikzpicture}
\end{figure}
\FloatBarrier
I minimi di diffrazione si hanno per $\sin \alpha =0$, in corrispondenza di essi l'intensità dell'onda si annulla.
\[
	\sin \alpha =0 \to \alpha =m\pi \qquad m=0,\pm 1,\pm 2,\ldots
\]
Tra due minimi d'intensità esiste un massimo (detto secondario, a eccezione di quello al centro).
\begin{align*}
	\alpha =\frac{kd\sin \vartheta}{2}= \frac{2\pi d\sin \vartheta}{\lambda 2} = \frac{\pi d\sin \vartheta}{\lambda} &= m \pi \\
	\sin \vartheta &= m\frac{\lambda}{d}
\end{align*}
L'unico caso particolare è quello per $m=0$ invece di avere un minimo abbiamo un massimo.







































\section{Diffrazione ad un foro circolare}

Un caso molto interessante è quello della fenditura circolare. La figura di diffrazione, per ragioni di simmetria, consta di un disco luminoso centrale circondato da una serie di corone circolari alternativamente scure e chiare. Queste frange presentano molte analogie con quanto visto nel caso dell'apertura rettilinea, anche se lo studio analitico è più complesso. Se facciamo un grafico dell'intensità della figura di diffrazione in funzione di $\vartheta$, anche in questo caso si ottiene qualcosa di analogo.
\begin{figure}[htpb]
	\centering
	

	\tikzset{every picture/.style={line width=0.75pt}} %set default line width to 0.75pt        

	\begin{tikzpicture}[x=0.75pt,y=0.75pt,yscale=-1,xscale=1]
	%uncomment if require: \path (0,300); %set diagram left start at 0, and has height of 300

	%Shape: Rectangle [id:dp5667083945200566] 
	\draw  [draw opacity=0][fill={rgb, 255:red, 128; green, 128; blue, 128 }  ,fill opacity=1 ] (147,68) -- (473,68) -- (473,243) -- (147,243) -- cycle ;
	%Shape: Circle [id:dp6476931658861269] 
	\draw  [draw opacity=0][fill={rgb, 255:red, 222; green, 222; blue, 222 }  ,fill opacity=1 ] (292.33,155.5) .. controls (292.33,145.74) and (300.24,137.83) .. (310,137.83) .. controls (319.76,137.83) and (327.67,145.74) .. (327.67,155.5) .. controls (327.67,165.26) and (319.76,173.17) .. (310,173.17) .. controls (300.24,173.17) and (292.33,165.26) .. (292.33,155.5) -- cycle ;
	%Shape: Circle [id:dp23108609558462168] 
	\draw  [color={rgb, 255:red, 222; green, 222; blue, 222 }  ,draw opacity=1 ][line width=6]  (273,155.5) .. controls (273,135.07) and (289.57,118.5) .. (310,118.5) .. controls (330.43,118.5) and (347,135.07) .. (347,155.5) .. controls (347,175.93) and (330.43,192.5) .. (310,192.5) .. controls (289.57,192.5) and (273,175.93) .. (273,155.5) -- cycle ;
	%Shape: Circle [id:dp08821188808781044] 
	\draw  [color={rgb, 255:red, 222; green, 222; blue, 222 }  ,draw opacity=1 ][line width=4.5]  (252.33,155.5) .. controls (252.33,123.65) and (278.15,97.83) .. (310,97.83) .. controls (341.85,97.83) and (367.67,123.65) .. (367.67,155.5) .. controls (367.67,187.35) and (341.85,213.17) .. (310,213.17) .. controls (278.15,213.17) and (252.33,187.35) .. (252.33,155.5) -- cycle ;
	%Shape: Circle [id:dp6714301698293956] 
	\draw  [color={rgb, 255:red, 222; green, 222; blue, 222 }  ,draw opacity=1 ][line width=2.25]  (234.33,155.5) .. controls (234.33,113.71) and (268.21,79.83) .. (310,79.83) .. controls (351.79,79.83) and (385.67,113.71) .. (385.67,155.5) .. controls (385.67,197.29) and (351.79,231.17) .. (310,231.17) .. controls (268.21,231.17) and (234.33,197.29) .. (234.33,155.5) -- cycle ;




	\end{tikzpicture}
\end{figure}
\FloatBarrier
Si trova in particolare che l'angolo a cui cade il primo minimo di intensità, corrispondente al bordo del disco centrale della figura di diffrazione, è dell'ordine di $1.22\,\lambda/d$. Studiare il caso della fenditura circolare è importante perché tutti gli strumenti ottici (telescopi, microscopi, macchine fotografiche, telecamere ecc...) sono dei sistemi che hanno una limitazione nell'apertura dell'obbiettivo. Quando un onda elettromagnetica incide su uno strumento ottico subisce diffrazione. È ciò che accade se ad esempio osserviamo un oggetto puntiforme con un telescopio. Se il diametro è piccolo, dato che:
\[
	\boxed{\alpha_{\text{min}} = 1.22\,\frac{\lambda}{d}}
\]
la figura sarà molto grande e quindi il telescopio avrà una scarsa risoluzione. Più grande è $d$ più questa figura si stringe.







































\section{Limite di risoluzione delle lenti}

Le osservazioni fatte finora sono importanti se vogliamo distinguere due oggetti molto vicini fra di loro e quindi è richiesta una capacità di risoluzione molto elevata.
\begin{figure}[htpb]
	\centering
	

	\tikzset{every picture/.style={line width=0.75pt}} %set default line width to 0.75pt        

	\begin{tikzpicture}[x=0.75pt,y=0.75pt,yscale=-0.9,xscale=0.9]
	%uncomment if require: \path (0,300); %set diagram left start at 0, and has height of 300

	%Curve Lines [id:da38620041188966714] 
	\draw [color={rgb, 255:red, 155; green, 155; blue, 155 }  ,draw opacity=1 ]   (353.5,44.92) .. controls (354.38,57.84) and (390.5,59.82) .. (390.5,75.44) .. controls (390.5,91.06) and (349.5,96.73) .. (350,108.8) .. controls (350.5,120.87) and (448.5,115.9) .. (448.5,134) ;
	%Curve Lines [id:da38887358508281] 
	\draw [color={rgb, 255:red, 155; green, 155; blue, 155 }  ,draw opacity=1 ]   (365.5,262.83) .. controls (365.5,242.96) and (352.62,236) .. (353.5,223.08) .. controls (354.38,210.16) and (390.5,208.18) .. (390.5,192.56) .. controls (390.5,176.94) and (349.5,171.27) .. (350,159.2) .. controls (350.5,147.13) and (448.5,152.1) .. (448.5,134) ;
	%Straight Lines [id:da49895926631976306] 
	\draw  [dash pattern={on 0.84pt off 2.51pt}]  (136.33,170.5) -- (335.5,134) ;
	%Straight Lines [id:da39084005186322335] 
	\draw    (344.5,34) -- (344.5,259) ;
	%Curve Lines [id:da44789670749394483] 
	\draw    (365.5,30.17) .. controls (365.5,50.04) and (352.62,57) .. (353.5,69.92) .. controls (354.38,82.84) and (390.5,84.82) .. (390.5,100.44) .. controls (390.5,116.06) and (349.5,121.73) .. (350,133.8) .. controls (350.5,145.87) and (448.5,140.9) .. (448.5,159) ;
	%Curve Lines [id:da9112796141758652] 
	\draw    (353.5,248.08) .. controls (354.38,235.16) and (390.5,233.18) .. (390.5,217.56) .. controls (390.5,201.94) and (349.5,196.27) .. (350,184.2) .. controls (350.5,172.13) and (448.5,177.1) .. (448.5,159) ;
	%Straight Lines [id:da2237085438594002] 
	\draw [line width=2.25]    (266.33,150.5) -- (266.33,194.67) ;
	%Straight Lines [id:da604548386071253] 
	\draw [line width=2.25]    (266.33,98.33) -- (266.33,142.5) ;
	%Straight Lines [id:da1980329329861159] 
	\draw  [dash pattern={on 0.84pt off 2.51pt}]  (136.33,122.5) -- (335.5,159) ;
	%Shape: Circle [id:dp616090689309877] 
	\draw  [fill={rgb, 255:red, 0; green, 0; blue, 0 }  ,fill opacity=1 ] (134.33,122.5) .. controls (134.33,121.4) and (135.23,120.5) .. (136.33,120.5) .. controls (137.44,120.5) and (138.33,121.4) .. (138.33,122.5) .. controls (138.33,123.6) and (137.44,124.5) .. (136.33,124.5) .. controls (135.23,124.5) and (134.33,123.6) .. (134.33,122.5) -- cycle ;
	%Shape: Circle [id:dp11252939339196733] 
	\draw  [fill={rgb, 255:red, 0; green, 0; blue, 0 }  ,fill opacity=1 ] (134.33,170.5) .. controls (134.33,169.4) and (135.23,168.5) .. (136.33,168.5) .. controls (137.44,168.5) and (138.33,169.4) .. (138.33,170.5) .. controls (138.33,171.6) and (137.44,172.5) .. (136.33,172.5) .. controls (135.23,172.5) and (134.33,171.6) .. (134.33,170.5) -- cycle ;
	%Shape: Arc [id:dp8493492791386488] 
	\draw  [draw opacity=0] (229.43,154.09) .. controls (228.93,151.64) and (228.67,149.1) .. (228.67,146.5) .. controls (228.67,144.05) and (228.9,141.65) .. (229.35,139.32) -- (266.33,146.5) -- cycle ; \draw   (229.43,154.09) .. controls (228.93,151.64) and (228.67,149.1) .. (228.67,146.5) .. controls (228.67,144.05) and (228.9,141.65) .. (229.35,139.32) ;

	% Text Node
	\draw (122.67,121) node    {$S_{1}$};
	% Text Node
	\draw (123.33,169) node    {$S_{2}$};
	% Text Node
	\draw (257.33,156.33) node    {$d$};
	% Text Node
	\draw (204.33,143.83) node    {$\alpha _{min}$};


	\end{tikzpicture}
\end{figure}
\FloatBarrier
Supponiamo di avere due stelle che vogliamo osservare con il telescopio. Esse emettono delle onde piane che arrivano fino a noi. Immaginiamo di rappresentare l'apertura dello strumento con una fenditura circolare. Poniamo che le direzioni di propagazione formino un certo angolo $\alpha$. Sullo schermo appariranno due figure differenti e finché $\alpha \gg\alpha_{min} $, le distinguiamo nettamente. In caso contrario esse si sovrappongono. Esiste un criterio per stabilire qual è la minima distanza angolare alla quale i due oggetti si devono trovare che prende il nome di \textbf{criterio di Rayleigh}. Tale criterio afferma che \emph{siamo in grado distinguere due figure di diffrazione fintanto che la distanza fra di esse è maggiore o uguale a una distanza minima che corrisponde al caso in cui il picco di una delle due figure coincide con il primo minimo dell'altra figura di diffrazione}. Quando la distanza a cui le due figure sono poste è proprio pari a questa distanza minima, si dice che le onde sono \emph{appena risolte}.
\[
	\alpha_{\text{min}} = 1.22\,\frac{\lambda}{d}
\]
Maggiore è il diametro del telescopio, più piccolo sarà l'angolo $\alpha$-minimo a cui riesco a distinguere con risolutezza le stelle. Lavorando nel campo dell'astronomia ottica, quindi osservando la luce, è sufficiente che i telescopi siano del diametro di alcuni metri. Lavorando nel campo della radio astronomia la situazione è differente. Le onde radio hanno $\lambda$ molto maggiore di quella della luce. Per una buona capacità di risoluzione abbiamo bisogno di diametri enormi. I radio telescopi infatti sono parabole gigantesche.







































\section{Reticolo di diffrazione}

Invece di avere la singola fenditura potremmo avere fenditure multiple. Il caso più semplice è quello in cui abbiamo due fenditure rettangolari uguali e parallele.
\begin{figure}[htpb]
	\centering
	

	\tikzset{every picture/.style={line width=0.75pt}} %set default line width to 0.75pt        

	\begin{tikzpicture}[x=0.75pt,y=0.75pt,yscale=-1,xscale=1]
	%uncomment if require: \path (0,300); %set diagram left start at 0, and has height of 300

	%Straight Lines [id:da9982051154678055] 
	\draw [line width=2.25]    (286.33,170.5) -- (286.33,204.67) ;
	%Straight Lines [id:da30716518689549455] 
	\draw [line width=2.25]    (286.33,98.33) -- (286.33,160.5) ;
	%Straight Lines [id:da921180189443459] 
	\draw [line width=2.25]    (286.33,55.33) -- (286.33,89.5) ;
	%Straight Lines [id:da7811689716718062] 
	\draw [line width=0.75]    (276.33,173.5) -- (276.33,194.75) ;
	\draw [shift={(276.33,170.5)}, rotate = 90] [fill={rgb, 255:red, 0; green, 0; blue, 0 }  ][line width=0.08]  [draw opacity=0] (10.72,-5.15) -- (0,0) -- (10.72,5.15) -- (7.12,0) -- cycle    ;
	%Straight Lines [id:da48287849255629745] 
	\draw [line width=0.75]    (276.33,138.25) -- (276.33,159.5) ;
	\draw [shift={(276.33,162.5)}, rotate = 270] [fill={rgb, 255:red, 0; green, 0; blue, 0 }  ][line width=0.08]  [draw opacity=0] (10.72,-5.15) -- (0,0) -- (10.72,5.15) -- (7.12,0) -- cycle    ;
	%Straight Lines [id:da02614628480081649] 
	\draw [line width=0.75]    (276.33,101.5) -- (276.33,122.75) ;
	\draw [shift={(276.33,98.5)}, rotate = 90] [fill={rgb, 255:red, 0; green, 0; blue, 0 }  ][line width=0.08]  [draw opacity=0] (10.72,-5.15) -- (0,0) -- (10.72,5.15) -- (7.12,0) -- cycle    ;
	%Straight Lines [id:da15297787750244418] 
	\draw [line width=0.75]    (276.33,66.25) -- (276.33,87.5) ;
	\draw [shift={(276.33,90.5)}, rotate = 270] [fill={rgb, 255:red, 0; green, 0; blue, 0 }  ][line width=0.08]  [draw opacity=0] (10.72,-5.15) -- (0,0) -- (10.72,5.15) -- (7.12,0) -- cycle    ;
	%Straight Lines [id:da6444856828113377] 
	\draw [line width=0.75]    (306.33,101.33) -- (306.33,157.5) ;
	\draw [shift={(306.33,160.5)}, rotate = 270] [fill={rgb, 255:red, 0; green, 0; blue, 0 }  ][line width=0.08]  [draw opacity=0] (10.72,-5.15) -- (0,0) -- (10.72,5.15) -- (7.12,0) -- cycle    ;
	\draw [shift={(306.33,98.33)}, rotate = 90] [fill={rgb, 255:red, 0; green, 0; blue, 0 }  ][line width=0.08]  [draw opacity=0] (10.72,-5.15) -- (0,0) -- (10.72,5.15) -- (7.12,0) -- cycle    ;

	% Text Node
	\draw (262.83,163.83) node    {$d$};
	% Text Node
	\draw (262.83,91.83) node    {$d$};
	% Text Node
	\draw (316.83,127.83) node    {$D$};


	\end{tikzpicture}
\end{figure}
\FloatBarrier
Si avranno due figure di diffrazione che provengono dalla stessa onda incidente. Siccome si tratta di un onda monocromatica piana, possiamo interpretare la situazione come se avessimo due sorgenti coerenti di onde elettromagnetiche. Abbiamo visto che in presenza di sorgenti coerenti otteniamo un'interferenza. Si ha dunque una combinazione dei due fenomeni.
L'andamento che osserveremo sarà una funzione di interferenza contenuta nell'inviluppo dato dal fenomeno di diffrazione.
\begin{figure}[htpb]
	\centering
	

	\tikzset{every picture/.style={line width=0.75pt}} %set default line width to 0.75pt        

	\begin{tikzpicture}[x=0.75pt,y=0.75pt,yscale=-1,xscale=1]
	%uncomment if require: \path (0,300); %set diagram left start at 0, and has height of 300

	%Curve Lines [id:da4430232758282888] 
	\draw    (162.75,196) .. controls (187.75,196.5) and (177.75,167) .. (191.75,167) .. controls (205.75,167) and (210.65,199.88) .. (219.75,199) .. controls (228.85,198.12) and (230.25,114) .. (241.25,114) .. controls (252.25,114) and (256.25,203) .. (264.75,202.5) .. controls (273.25,202) and (269.75,84) .. (282.5,84) ;
	%Curve Lines [id:da3958706293037202] 
	\draw    (402.25,196) .. controls (377.25,196.5) and (387.25,167) .. (373.25,167) .. controls (359.25,167) and (354.35,199.88) .. (345.25,199) .. controls (336.15,198.12) and (334.75,114) .. (323.75,114) .. controls (312.75,114) and (308.75,203) .. (300.25,202.5) .. controls (291.75,202) and (295.25,84) .. (282.5,84) ;
	%Straight Lines [id:da5381577750567956] 
	\draw    (112.75,208) -- (452.25,208) ;
	%Curve Lines [id:da9555907596763293] 
	\draw  [dash pattern={on 0.84pt off 2.51pt}]  (112.75,196) .. controls (219.67,170.67) and (236.67,80) .. (282.5,80) .. controls (328.33,80) and (345.67,170.67) .. (452.25,196) ;




	\end{tikzpicture}
\end{figure}
\FloatBarrier
In termini quantitativi, se introduciamo le quantità $\alpha$ (vista in precedenza) e $\beta$ come:
\[
	\alpha = \frac{kd\sin \vartheta}{2} \qquad \beta =\frac{kD\sin \vartheta}{2}
\]
La figura di interferenza la possiamo scrivere come:
\[
	I = 4I_0\frac{\sin^2 \alpha}{\alpha^2}\cos^2 \beta
\]
Invece di considerare solo due fenditure si può immaginare di considerarne un numero molto grande. Uno schermo in cui ho praticato un numero $N$ molto elevato di fenditure tutte uguali ed equi spaziate prende il nome di \textbf{reticolo di diffrazione}. Osservando quello che accade all'intensità a grande distanza otterremo una funzione con inviluppo dato dalla figura di diffrazione.

Questa volta però avremo dei picchi di interferenza molto più stretti separati da delle oscillazioni molto più piccole.
\begin{figure}[htpb]
	\centering
	

	\tikzset{every picture/.style={line width=0.75pt}} %set default line width to 0.75pt        

	\begin{tikzpicture}[x=0.75pt,y=0.75pt,yscale=-1,xscale=1]
	%uncomment if require: \path (0,300); %set diagram left start at 0, and has height of 300

	%Curve Lines [id:da2879964292305275] 
	\draw    (461,225.2) .. controls (448.5,225.45) and (450.78,209.41) .. (446.2,209.2) .. controls (441.63,208.99) and (443.8,225.4) .. (436.6,225.4) .. controls (429.4,225.4) and (413.8,224.6) .. (395,225) .. controls (376.2,225.4) and (385.2,153) .. (374.2,153) .. controls (363.2,153) and (384.6,225.6) .. (340.2,225.2) .. controls (295.8,224.8) and (315.25,81) .. (302.5,81) ;
	%Straight Lines [id:da5771587807011713] 
	\draw    (132.75,228) -- (472.25,228) ;
	%Curve Lines [id:da6999062790272035] 
	\draw  [dash pattern={on 0.84pt off 2.51pt}]  (132.75,214) .. controls (239.67,188.67) and (256.67,78) .. (302.5,78) .. controls (348.33,78) and (365.67,188.67) .. (472.25,214) ;
	%Curve Lines [id:da02489589483366328] 
	\draw    (144,225.2) .. controls (156.5,225.45) and (154.22,209.41) .. (158.8,209.2) .. controls (163.38,208.99) and (161.2,225.4) .. (168.4,225.4) .. controls (175.6,225.4) and (191.2,224.6) .. (210,225) .. controls (228.8,225.4) and (219.8,153) .. (230.8,153) .. controls (241.8,153) and (220.4,225.6) .. (264.8,225.2) .. controls (309.2,224.8) and (289.75,81) .. (302.5,81) ;




	\end{tikzpicture}
\end{figure}
\FloatBarrier
Invece di osservare un andamento di tipo sinusoidale osserveremo picchi di intensità molto elevata molto separati fra di loro. In tal caso si può dimostrare che l'intensità è data da:
\[
	\boxed{I=I_0\,\frac{\sin^2 \alpha}{\alpha^2}\,\frac{\sin^2 (N\beta)}{\sin^2 \beta}}
\]
Si verifica che per $\vartheta=0 \implies I=N^2 I_0 $, dove $I_0$ è l'intensità massima della figura di diffrazione della singola fenditura.
Se illuminiamo il reticolo di diffrazione con della luce policromatica, ad esempio la luce bianca, per ciascun colore la posizione del picco cambierà. Quest'oggetto funziona come un'analizzatore di schermi, ci consente di separare fra di loro i vari colori perché li vediamo apparire in posizioni diverse.

Il reticolo di diffrazione è la novità che ha consentito di cominciare a studiare in modo dettagliato le proprietà della materia permettendone un'\emph{analisi spettroscopica}. Esso analizza i colori presenti in un fascio di luce monocromatica. Un esempio di reticolo di diffrazione è il \emph{compact disc} (anche se funziona in modo leggermente differente). La sua superficie è infatti costituita da tanti solchi che quando vengono illuminati dall'onda elettromagnetica si comportano un po' come queste fenditure.


























% APPENDICE
\appendix

\chapter{Costanti ed unità di misura}

\section*{Costanti fisiche}

\begin{equation*}
	\begin{array}{ l l}
		\text{Carica elementare} & e=1.602\times 10^{-19} \ C\\
		\text{Massa dell'elettrone} & m_e =9.109\times 10^{-31} \ kg\\
		\text{Costante dielettrica del vuoto} & \varepsilon_0 =8.854\times 10^{-12} \ C^2 /\left( N\ m^2\right)\\
		 \text{Permeabilita magnetica del vuoto} & \mu_0 =4\pi \times 10^{-7} \ N/A^2\\
		\text{Velocita della luce nel vuoto} & c=2.998\times 10^8 \ m/s\\
		\text{Raggio di Bohr} & a_0 =5.292\times 10^{-11} \ m\\
		\text{Magnetone di Bohr} & \mu_B =9.274\times 10^{-24} \ J/T
	\end{array}
\end{equation*}







































\section*{Prefissi per le potenze di dieci}

\begin{center}

	\begin{table}[!h]
	\centering
	\begin{tabular}{ccc|ccc}
		\textbf{Potenza} & \textbf{Prefisso} & \textbf{Abbreviazione} & \textbf{Potenza} & \textbf{Prefisso} & \textbf{Abbreviazione} \\
		\hline
		$\displaystyle 10^{-18}$ & atto & a & $\displaystyle 10^1$ & deca & da \\
		$\displaystyle 10^{-15}$ & femto & f & $\displaystyle 10^2$ & etto & h \\
		$\displaystyle 10^{-12}$ & pico & p & $\displaystyle 10^3$ & chilo & k \\
		$\displaystyle 10^{-9}$ & nano & n & $\displaystyle 10^6$ & mega & M \\
		$\displaystyle 10^{-6}$ & micro & $\displaystyle \mu $ & $\displaystyle 10^9$ & giga & G \\
		$\displaystyle 10^{-3}$ & milli & m & $\displaystyle 10^{12}$ & tera & T \\
		$\displaystyle 10^{-2}$ & centi & c & $\displaystyle 10^{15}$ & peta & P \\
		$\displaystyle 10^{-1}$ & deci & d & $\displaystyle 10^{18}$ & exa & E \\
	\end{tabular}
	\end{table}
\end{center}







































\newpage\section*{Grandezze ed unità di misura impiegate nel testo}


\begin{center}
	\textbf{Grandezze fondamentali nel Sistema Internazionale}
\end{center}

\begin{table}[!h]
	\centering
	\begin{tabular}{l|l|l}
		\hline
		\textbf{Grandezza} & \textbf{Unità di misura} & \textbf{Simbolo} \\
		\hline
		Lunghezza & metro & $m$ \\
		Massa & chilogrammo & $kg$ \\
		Intervalli di tempo & secondo & $s$ \\
		Temperatura assoluta & kelvin & $K$ \\
		Quantità di sostanza & mole & $mol$ \\
		Angolo & radiante & $rad$ \\
	\end{tabular}
\end{table}

\begin{center}
	\textbf{Grandezze derivate}
\end{center}

\begin{table}[!h]
	\centering
	\begin{tabular}{l|c|c|c}
		\hline
		\textbf{Grandezza} & \textbf{Unità} & \textbf{Simbolo} & \textbf{Espressioni equivalenti} \\
		\hline
		Velocità $\displaystyle \vec{v}$ & - & $\displaystyle m/s$ & - \\
		Velocità angolare $\displaystyle \vec{\omega}$ & - & $\displaystyle rad/s$ & - \\
		Frequenza $\displaystyle \nu $ & hertz & $\displaystyle Hz$ & $\displaystyle 1/s$ \\
		Accelerazione $\displaystyle \vec{a}$ & - & $\displaystyle m/s^2$ & - \\
		Accelerazione angolare $\displaystyle \vec{\alpha}$ & - & $\displaystyle rad/s^2$ & - \\
		Forza $\displaystyle \vec{F}$ & newton & $\displaystyle N$ & $\displaystyle kg\ m/s^2 ;\ J/m$ \\
		Momento meccanico $\displaystyle \vec{M}$ & - & $\displaystyle N\ m$ & $\displaystyle kg\ m^2 /s^2$ \\
		Momento angolare $\displaystyle \vec{L}$ & - & $\displaystyle kg\ m^2 /s$ & - \\
		Quantità di moto $\displaystyle \vec{p}$ & - & $\displaystyle kg\ m /s$ & - \\
		Momento d'inerzia $\displaystyle I$ & - & $\displaystyle kg\ m^2$ & - \\
		Energia $\displaystyle E,U$; lavoro $\displaystyle L$; calore $\displaystyle Q$ & joule & $\displaystyle J$ & $\displaystyle N\ m;\ kg\ m^2 /s^2$ \\
		Potenza $\displaystyle P$ & watt & $\displaystyle W$ & $\displaystyle J/s;\ kg\ m^2 /s^3$ \\
		Pressione $\displaystyle p$ & pascal & $\displaystyle Pa$ & $\displaystyle N/m^2 ;\ kg/m\ s^2$ \\
		Densità per unità di volume $\displaystyle \rho $ & - & $\displaystyle kg/m^3$ & - \\
		Entropia $\displaystyle S$ & - & $\displaystyle J/K$ & - \\
	\end{tabular}
\end{table}

\newpage

\begin{center}
	\textbf{Altre unità di misura}
\end{center}

\begin{table}[!h]
	\centering
	\begin{tabular}{l|l|l}
		\hline
		\textbf{Grandezza} & \textbf{Unità di misura} & \textbf{Equivalenza nel S.I.} \\
		\hline
		Velocità & chilometro-ora ($\displaystyle km/h$) & $\displaystyle 0.2778\ m/s$ \\
		Forza & chilogrammo-forza ($\displaystyle kg_f$) & $\displaystyle 9.81\ N$ \\
		Energia & chilowatt-ora ($\displaystyle kWh$) & $\displaystyle 3.6\times 10^6 \ J$ \\
		Volume & litro ($\displaystyle L$) & $\displaystyle 10^{-3} \ m^3$ \\
		Pressione & $\displaystyle bar$ & $\displaystyle 10^5 \ Pa$ \\
		'' & $\displaystyle mm\ Hg$ & $\displaystyle 133.3\ Pa$ \\
		'' & atmosfera ($\displaystyle atm$) & $\displaystyle 1.013\times 10^5 \ Pa$ \\
		Calore & caloria ($\displaystyle cal$) & $\displaystyle 4.18\ J$ \\
	\end{tabular}
\end{table}

\begin{center}
	\textbf{Equivalenze utili}
\end{center}

Detta $\displaystyle t$ la temperatura di un oggetto misurata nella scala Celsius e $\displaystyle T$ la corrispondente temperatura assoluta, vale la relazione:
\begin{equation*}
	t( \degree \text{C}) =T(\text{K}) +273.15
\end{equation*}















\chapter{Richiami di trigonometria}

\section*{Funzioni trigonometriche}

Dato il triangolo rettangolo mostrato in figura di cateti $\displaystyle a$ e $\displaystyle b$, ipotenusa $\displaystyle c$ ed angolo $\displaystyle \vartheta $ opposto al cateto $\displaystyle a$, si definiscono:

\begin{equation*}
	\begin{array}{ l l}
		\sin \vartheta =\frac{a}{c} & \cos \vartheta =\frac{b}{c}\\
		[4mm]
		\tan \vartheta =\frac{a}{b} =\frac{\sin \vartheta}{\cos \vartheta} \ \ \ \  & \cot \vartheta =\frac{b}{a} =\frac{\cos \vartheta}{\sin \vartheta}
	\end{array}
\end{equation*}


\begin{figure}[htpb]
	\centering
	\tikzset{every picture/.style={line width=0.75pt}} %set default line width to 0.75pt

	\begin{tikzpicture}[x=0.75pt,y=0.75pt,yscale=-1,xscale=1]
	%uncomment if require: \path (0,300); %set diagram left start at 0, and has height of 300

	%Shape: Right Triangle [id:dp9340776108358415]
	\draw  [line width=1.5]  (162,57) -- (444.5,211.44) -- (162,211.44) -- cycle ;
	%Shape: Arc [id:dp8374391809216872]
	\draw  [draw opacity=0] (394.5,211.86) .. controls (394.5,211.72) and (394.5,211.58) .. (394.5,211.44) .. controls (394.5,202.5) and (396.84,194.12) .. (400.95,186.86) -- (444.5,211.44) -- cycle ; \draw   (394.5,211.86) .. controls (394.5,211.72) and (394.5,211.58) .. (394.5,211.44) .. controls (394.5,202.5) and (396.84,194.12) .. (400.95,186.86) ;

	% Text Node
	\draw (384.8,196) node    {$\vartheta $};
	% Text Node
	\draw (149.6,134.8) node    {$a$};
	% Text Node
	\draw (288,224.8) node    {$b$};
	% Text Node
	\draw (306,118) node    {$c$};

	\end{tikzpicture}
\end{figure}







































\section*{Identità trigonometriche}

Risultano le seguenti relazioni

\begin{equation*}
	\begin{array}{ c c}
		\sin \vartheta =\cos\left(\frac{\pi}{2} -\vartheta \right) & \cos \vartheta =\sin\left(\frac{\pi}{2} -\vartheta \right)\\
		[4mm]
		\cot \vartheta =\tan\left(\frac{\pi}{2} -\vartheta \right) & \sin^2 \vartheta +\cos^2 \vartheta =1\\
		[4mm]
		\cos^2 \vartheta =\frac{1}{1+\tan^2 \vartheta} & \sin^2 \vartheta =\frac{1}{1+\cot^2 \vartheta}\\
		[4mm]
		\cos 2\vartheta =\cos^2 \vartheta -\sin^2 \vartheta  & \sin 2\vartheta =2\sin \vartheta \cos \vartheta \\
		[4mm]
		\cos^2\frac{\vartheta}{2} =\frac{1+\cos \vartheta}{2} & \sin^2\frac{\vartheta}{2} =\frac{1-\cos \vartheta}{2}\\
		[4mm]
		\tan 2\vartheta =\frac{2\tan \vartheta}{1-\tan^2 \vartheta} & \tan\frac{\vartheta}{2} =\sqrt{\frac{1-\cos \vartheta}{1+\cos \vartheta}}
	\end{array}
\end{equation*}

Inoltre valgono le seguenti regole di somma:

\begin{equation*}
	\begin{array}{ c}
		\sin( \alpha \pm \beta ) =\sin \alpha \cos \beta \pm \cos \alpha \sin \beta \\
		[4mm]
		\cos( \alpha \pm \beta ) =\cos \alpha \cos \beta \mp \sin \alpha \sin \beta \\
		[4mm]
		\tan( \alpha \pm \beta ) =\frac{\tan \alpha \pm \tan \beta}{1\mp \tan \alpha \tan \beta}\\
		[4mm]
		\sin \alpha \pm \sin \beta =2\sin\left[\frac{1}{2}( \alpha \pm \beta )\right]\cos\left[\frac{1}{2}( \alpha \mp \beta )\right]\\
		[4mm]
		\cos \alpha +\cos \beta =2\cos\left[\frac{1}{2}( \alpha +\beta )\right]\cos\left[\frac{1}{2}( \alpha -\beta )\right]\\
		[4mm]
		\cos \alpha -\cos \beta =2\sin\left[\frac{1}{2}( \alpha +\beta )\right]\sin\left[\frac{1}{2}( \alpha -\beta )\right]\\
		[4mm]
		\sin \alpha \cos \beta =\frac{1}{2}[\sin( \alpha +\beta ) +\sin( \alpha -\beta )]\\
		[4mm]
		\cos \alpha \cos \beta =\frac{1}{2}[\cos( \alpha +\beta ) +\cos( \alpha -\beta )]\\
		[4mm]
		\sin^2 \alpha -\sin^2 \beta =\sin( \alpha +\beta )\sin( \alpha -\beta )\\
		[4mm]
		\cos^2 \alpha -\cos^2 \beta =\sin( \alpha +\beta )\sin( \beta -\alpha )
	\end{array}
\end{equation*}

Nel campo complesso valgono le seguenti relazioni

\begin{equation*}
	\begin{array}{ c}
		\sin z=\frac{1}{2i}\left( e^{iz} -e^{-iz}\right)\\
		[4mm]
		\cos z=\frac{1}{2}\left( e^{iz} +e^{-iz}\right)
	\end{array}
\end{equation*}

dette formule di Eulero; inoltre

\begin{equation*}
	\begin{array}{ c}
		e^{iz} =\cos z+i\sin z\\
		[4mm]
		e^{x+iy} =e^x(\cos y+i\sin y)
	\end{array}
\end{equation*}







































\section*{Formule notevoli per un triangolo}

Dato il triangolo mostrato in figura di lati $\displaystyle a$, $\displaystyle b$ e $\displaystyle c$ ed angoli $\displaystyle \alpha$, $\displaystyle \beta $ e $\displaystyle \gamma$, valgono le seguenti relazioni:

\begin{equation*}
	\begin{array}{ c}
		\alpha +\beta +\gamma =\pi \\
		[4mm]
		\frac{a}{\sin \alpha} =\frac{b}{\sin \beta} =\frac{c}{\sin \gamma}\\
		[4mm]
		a^2 =b^2 +c^2 -2bc\cos \alpha
	\end{array}
\end{equation*}




\begin{figure}[htpb]
	\centering
	\tikzset{every picture/.style={line width=0.75pt}} %set default line width to 0.75pt

	\begin{tikzpicture}[x=0.75pt,y=0.75pt,yscale=-1,xscale=1]
	%uncomment if require: \path (0,300); %set diagram left start at 0, and has height of 300

	%Shape: Boxed Polygon [id:dp5667907928913438]
	\draw  [line width=1.5]  (459.41,171.13) -- (162.59,172.5) -- (343.14,53.5) -- cycle ;
	%Shape: Arc [id:dp1523693032409319]
	\draw  [draw opacity=0] (187.57,155.88) .. controls (190.73,160.62) and (192.58,166.31) .. (192.59,172.43) -- (162.59,172.5) -- cycle ; \draw   (187.57,155.88) .. controls (190.73,160.62) and (192.58,166.31) .. (192.59,172.43) ;
	%Shape: Arc [id:dp8340291029684717]
	\draw  [draw opacity=0] (364.27,74.8) .. controls (358.85,80.18) and (351.38,83.5) .. (343.14,83.5) .. controls (332.66,83.5) and (323.44,78.13) .. (318.07,69.98) -- (343.14,53.5) -- cycle ; \draw   (364.27,74.8) .. controls (358.85,80.18) and (351.38,83.5) .. (343.14,83.5) .. controls (332.66,83.5) and (323.44,78.13) .. (318.07,69.98) ;
	%Shape: Arc [id:dp5702801543421352]
	\draw  [draw opacity=0] (429.41,171.18) .. controls (429.41,171.17) and (429.41,171.15) .. (429.41,171.13) .. controls (429.41,162.73) and (432.86,155.14) .. (438.42,149.7) -- (459.41,171.13) -- cycle ; \draw   (429.41,171.18) .. controls (429.41,171.17) and (429.41,171.15) .. (429.41,171.13) .. controls (429.41,162.73) and (432.86,155.14) .. (438.42,149.7) ;

	% Text Node
	\draw (209.6,155.2) node    {$\gamma $};
	% Text Node
	\draw (338.8,92.8) node    {$\beta $};
	% Text Node
	\draw (420.4,153.6) node    {$\alpha $};
	% Text Node
	\draw (247.2,97.6) node    {$a$};
	% Text Node
	\draw (413.2,100.8) node    {$c$};
	% Text Node
	\draw (316.4,182.4) node    {$b$};

	\end{tikzpicture}
\end{figure}















\chapter{Calcolo differenziale ed integrale}

\section{Regole di derivazione di una funzione}

\subsection*{Regole generali}

Sia $\displaystyle k$ una costante ed $\displaystyle f( x)$, $\displaystyle g( x)$ due funzioni continue e derivabili (per la nozione di derivabilità si rimanda ad un testo di Analisi Matematica). Indicando con $\displaystyle f'( x) =\frac{df}{dx}$ e $\displaystyle g'( x) =\frac{dg}{dx}$ le derivate prime di $\displaystyle f( x)$ e $\displaystyle g( x)$, si hanno le seguenti relazioni
\begin{equation*}
	\begin{array}{ c}
		\frac{d}{dx}[ k\ f( x)] =k\ f'( x)\\
		[4mm]
		\frac{d}{dx}[ g( x) +f( x)] =g'( x) +f'( x)\\
		[4mm]
		\frac{d}{dx}[ g( x) \cdot f( x)] =g'( x) f( x) +g( x) f( x) '\\
		[4mm]
		\frac{d}{dx}\left[\frac{g( x)}{f( x)}\right] =\frac{g'( x) f( x) -g( x) f( x) '}{f^2( x)}
	\end{array}
\end{equation*}
Se $\displaystyle g=g( z)$ e $\displaystyle z=f( x)$, la funzione $\displaystyle g=g[ f( x)]$ si dice composita. In tal caso la derivata di $\displaystyle g( x)$ risulta
\begin{equation*}
	\frac{d}{dx} g( z) =\frac{dg}{dz} \cdot \frac{dz}{dx} =\frac{dg}{dz} \cdot \frac{df}{dx}
\end{equation*}



\subsection*{Tabella delle principali derivate}

\begin{equation*}
	\begin{array}{ l l}
		\frac{d}{dx} k=0 & \frac{d}{dx} x^n =n\ x^{n-1}\\
		[4mm]
		\frac{d}{dx} e^{kx} =k\ e^{kx} & \frac{d}{dx} a^{kx} =k\ a^{kx}\ln a\\
		[4mm]
		\frac{d}{dx}\sin x=\cos x & \frac{d}{dx}\cos x=-\sin x\\
		[4mm]
		\frac{d}{dx}\tan x=\frac{1}{\cos^2 x} \ \ \ \  & \frac{d}{dx}\cot x=-\frac{1}{\sin^2 x}\\
		[4mm]
		\frac{d}{dx}\ln x=\frac{1}{x} & \frac{d}{dx}\ln[ f( x)] =\frac{f'( x)}{f( x)}
	\end{array}
\end{equation*}







































\section{Regole di integrazione di una funzione}

\subsection*{Regole generali}

Date due funzioni $\displaystyle f( x)$ ed $\displaystyle F( x)$, diremo che $\displaystyle F$ è una primitiva di $\displaystyle f$ se:
\begin{equation*}
	\frac{d}{dx} F(x) =f(x)
\end{equation*}
La primitiva di una funzione viene anche detta integrale indefinito ed indicata con la notazione:
\begin{equation*}
	F(x) =\int f(x) \ dx
\end{equation*}
In virtù delle regole di derivazione, la primitiva è sempre definita a meno di una costante arbitraria. Per il teorema del calcolo integrale, l'integrale definito di una funzione $\displaystyle f( x)$ valutato fra gli estremi $\displaystyle a$ e $\displaystyle b$ risulta pari alla differenza dei valori assunti dalla primitiva nei due estremi:
\begin{equation*}
	\int^b_a f( x) \ dx\ =[ F( x)]^b_a =F( b) -F( a)
\end{equation*}
Vi sono due metodi importanti di risoluzione di un integrale definito:
\begin{enumerate}
	\item \textit{Integrazione per sostituzione}. Posto $\displaystyle x=y( t)$ e detta $\displaystyle t=g( x)$ la sua funzione inversa, risulta:
	\[
		\int^b_a f( x) \ dx=\int^{g( b)}_{g( a)} f[ y( t)]\frac{dy}{dt} dt
	\]
	\item \textit{Integrazione per parti}. Se in un integrale appare il prodotto di una funzione $\displaystyle f( x)$ per la derivata di una funzione $\displaystyle g( x)$, possiamo allora porre:
	\[
		\int^b_a f( x)\frac{dg}{dx} dx=[ f( x) \ g( x)]^b_a -\int^b_a\frac{df}{dx} g( x) \ dx
	\]
\end{enumerate}



\subsection*{Tabella dei principali integrali indefiniti}

Si noti che tutti gli integrali qui riportati sono sempre assegnati a meno di una costante arbitraria $\displaystyle C$.
\begin{equation*}
	\begin{array}{ l l}
		\int k\ dx=kx & \int x^n \ dx=\frac{x^{n+1}}{n+1}\\
		[4mm]
		\int e^{kx} \ dx=\frac{e^{kx}}{k} & \int a^{kx} \ dx=\frac{a^{kx}}{k\ln a}\\
		[4mm]
		\int \sin x\ dx=-\cos x & \int \cos x\ dx=\sin x\\
		[4mm]
		\int \tan x\ dx=-\ln(\cos x) \ \ \ \  & \int \cot x\ dx=\ln(\sin x)\\
		[4mm]
		\int \frac{1}{x} \ dx=\ln x & \int \ln x\ dx=x\ln x-x
	\end{array}
\end{equation*}







































\section{Espansione in serie di una funzione}

Si consideri una funzione $\displaystyle f( x)$ continua assieme alle sue derivate in un intorno del punto $\displaystyle x_0 ;\ f( x)$ può essere rappresentata in tale intorno mediante una serie di potenze di $\displaystyle x$:
\begin{equation*}
	f( x) =\sum^{\infty}_{n=0}\frac{1}{n!}\frac{d^n f}{dx^n}( x_0)( x-x_0)^n
\end{equation*}
dove $\displaystyle n!=n\cdot ( n-1) \cdot ( n-2) \dots 3\cdot 2\cdot 1;\ 0!=1$ e le derivate di ordine $\displaystyle n$-esimo della funzione $\displaystyle f( x)$ sono tutte valutate nel punto $\displaystyle x_0$. Se la precedente sommatoria viene arrestata ai primi addendi, si ottiene una espressione che approssima la funzione $\displaystyle f( x)$. Di seguito si riportano le approssimazioni di alcune funzioni in un intorno di $\displaystyle x_0 =0$.







\subsection*{Approssimazione di $\displaystyle f( x)$ nell'intorno di $\displaystyle x_0 =0$}


\begin{equation*}
	\begin{array}{ l l}
		\sin x\approx x & \cos x\approx 1-\frac{x^2}{2}\\
		[4mm]
		e^x \approx 1+x & \tan x\approx x\\
		[4mm]
		\frac{1}{1+x} \approx 1-x & \frac{1}{1-x} \approx 1+x\\
		[4mm]
		( 1+x)^2 \approx 1+2x\ \ \ \  & ( 1-x)^2 \approx 1-2x\\
		[4mm]
		\sqrt{1+x} \approx 1+\frac{x}{2} & \sqrt{1-x} \approx 1-\frac{x}{2}\\
		[4mm]
		\ln( 1+x) \approx x &
	\end{array}
\end{equation*}















\chapter{Identità vettoriali}

Si indicherà col simbolo $\mathbf{A}$ il vettore $\vec{A}$ per non appesantire la notazione. Si utilizzerà il simbolo $\nabla$ per indicare la divergenza ($\nabla\cdot\mathbf{A}$), il rotore ($\nabla\times\mathbf{A}$) di un campo vettoriale $\mathbf{A}$ e il gradiente ($\nabla f$) di un campo scalare $f$.

\section{Identità vettoriali generiche}

\textbf{Triplo prodotto}
\begin{align*}
	&\mathbf{A} \times (\mathbf{B} \times \mathbf{C}) = \mathbf{B} (\mathbf{A} \cdot \mathbf{C}) - \mathbf{C} (\mathbf{A} \cdot \mathbf{B}) \\
	&\mathbf{A} \cdot (\mathbf{B} \times \mathbf{C}) = \mathbf{B} \cdot (\mathbf{C} \times \mathbf{A}) = \mathbf{C} \cdot (\mathbf{A} \times \mathbf{B})
\end{align*}
da cui si ha
\begin{align*}
	(\mathbf{A} \times \mathbf{B}) \cdot (\mathbf{C} \times \mathbf{D}) = (\mathbf{A} \cdot \mathbf{C}) (\mathbf{B} \cdot \mathbf{D}) - (\mathbf{A} \cdot \mathbf{D}) (\mathbf{B} \cdot \mathbf{C})
\end{align*}
ed in particolare
\begin{align*}
	|\mathbf{A} \times \mathbf{B}|^2 = |\mathbf{A}|^2 |\mathbf{B}|^2 - (\mathbf{A} \cdot \mathbf{B})^2
\end{align*}







































\section{Proprietà degli operatori vettoriali}

\textbf{Proprietà distributiva}
\begin{align*}
	& \nabla (f+g) = \nabla f + \nabla g \\
	& \nabla \cdot ( \mathbf{A} + \mathbf{B} ) = \nabla \cdot \mathbf{A} + \nabla \cdot \mathbf{B} \\
	& \nabla \times ( \mathbf{A} + \mathbf{B} ) = \nabla \times \mathbf{A} + \nabla \times \mathbf{B}
\end{align*}
\textbf{Proprietà del prodotto scalare}
\begin{align*}
	&\nabla(\mathbf{A} \cdot \mathbf{B}) = (\mathbf{A} \cdot \nabla)\mathbf{B} + (\mathbf{B} \cdot \nabla)\mathbf{A} + \mathbf{A} \times (\nabla \times \mathbf{B}) + \mathbf{B} \times (\nabla \times \mathbf{A})
\end{align*}
\textbf{Proprietà del prodotto vettoriale}
\begin{align*}
	&\nabla \cdot (\mathbf{A} \times \mathbf{B}) = \mathbf{B} \cdot \nabla \times \mathbf{A} - \mathbf{A} \cdot \nabla \times \mathbf{B} \\
	&\nabla \times (\mathbf{A} \times \mathbf{B}) = \mathbf{A} (\nabla \cdot \mathbf{B}) -\mathbf{B}  (\nabla \cdot \mathbf{A})+( \mathbf{B}\cdot \nabla)\mathbf{A}-( \mathbf{A}\cdot \nabla)\mathbf{B}
\end{align*}
\textbf{Prodotto tra scalari e vettori}
\begin{align*}
	&\nabla (fg) = f \nabla g + g \nabla f \\
	&\nabla \cdot (f\mathbf{A}) = \nabla f \cdot \mathbf{A} + f \nabla \cdot \mathbf{A}\\
	&\nabla \times (f \mathbf{A}) = \nabla f \times \mathbf{A} + f \nabla \times \mathbf{A}
\end{align*}







































\section{Combinazione di operatori vettoriali}

\textbf{Divergenza del gradiente}
\begin{align*}
	\nabla \cdot \nabla f = \nabla^2 f   = \sum_{i=1}^n \frac {\partial^2f}{\partial x^2_i}
\end{align*}
L'operatore $ \nabla^2$ viene detto operatore di Laplace (o laplaciano) e viene anche indicato con $ \Delta $. \\\\
\textbf{Rotore del gradiente}
\begin{align*}
	\nabla \times \nabla f = 0
\end{align*}
\textbf{Divergenza del rotore}
\begin{align*}
	\nabla \cdot \nabla \times \mathbf{A} = 0
\end{align*}
\textbf{Rotore del rotore}
\begin{align*}
	\nabla \times \left( \mathbf{\nabla \times F} \right) = \mathbf{\nabla} (\mathbf{\nabla \cdot F}) - \nabla^2 \mathbf{F}
\end{align*}
\textbf{Altre identità}
\begin{align*}
	\frac{1}{2} \nabla \mathbf{A}^2 = \mathbf{A} \times (\nabla \times \mathbf{A}) + (\mathbf{A} \cdot \nabla) \mathbf{A}
\end{align*}




\end{document}
