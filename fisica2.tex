\documentclass[10pt,a4paper]{book}
\usepackage{amsfonts}
\usepackage[none]{hyphenat} % PER NON FAR ANDARE A CAPO LE PAROLE CON IL TRATTINO
\usepackage[nottoc,notlot,notlof]{tocbibind} % boh
\usepackage{graphicx}
\usepackage{float}
\usepackage{centernot} % serve per il 'A NON implica B' nel comando \centernot
\usepackage{wrapfig}
\pdfsuppresswarningpagegroup=1

% FIGURE MATHCHA.IO
\usepackage{amsmath}
\usepackage{tikz}
\usepackage{mathdots}
\usepackage{yhmath}
\usepackage{cancel}
\usepackage{color}
\usepackage{siunitx}
\usepackage{array}
\usepackage{multirow}
\usepackage{amssymb}
\usepackage{gensymb}
\usepackage{tabularx}
\usepackage{booktabs}
\usepackage{caption}\captionsetup{belowskip=12pt,aboveskip=4pt}
\usetikzlibrary{fadings}
\usetikzlibrary{patterns}
\usetikzlibrary{shadows.blur}
\usepackage{placeins} % The placeins package gives the command \FloatBarrier, which will make sure any floats will be put in before this point
\usepackage{flafter}  % The flafter package ensures that floats don't appear until after they appear in the code.

% PER INCLUDERE FIGURE
\usepackage{import}
\usepackage{pdfpages}
\usepackage{transparent}
\usepackage{xcolor}

% QUESTO NUOVO COMANDO SERVE PER INCLDERE LE FIGURE FATTE IN INKSCAPE, CHE SI DEVONO TROVARE NELLA STESSA DIRECTORY DENTRO LA CARTELLA figures
\newcommand{\incfig}[2][1]{%
	\def\svgwidth{#1\columnwidth}
	\import{./figures/}{#2.pdf_tex}
}

% RISCRITTURA DI COMANDI
%\renewcommand{\epsilon}{\varepsilon} % NON USATO
%\renewcommand{\theta}{\vartheta} % NON USATO
%\renewcommand{\rho}{\varrho}
%\renewcommand{\phi}{\varphi} % NON USATO
\renewcommand{\degree}{^\circ\text{C}} % SIMBOLO GRADI
\newcommand{\notimplies}{\mathrel{{\ooalign{\hidewidth$\not\phantom{=}$\hidewidth\cr$\implies$}}}}

\usepackage{mathtools} % SERVE PER I DUE COMANDI DOPO
\DeclarePairedDelimiter{\abs}{\lvert}{\rvert} % CREA UN COMANDO abs()
\DeclarePairedDelimiter{\norma}{\lVert}{\rVert} % CREA UN COMANDO norma()

% INDICE
\setcounter{secnumdepth}{3} % DI DEFAULT LE SUBSUBSECTION NON SONO NUMERATE, COSÌ SÌ
\setcounter{tocdepth}{1} % FISSA LA PROFONDITÀ DELLE COSE MOSTRATE NELL'INDICE
\usepackage{tocstyle}
\usetocstyle{standard}
\usepackage[hidelinks]{hyperref} % RENDE L'INDICE INTERATTIVO E hidelinks NASCONDE IL BORDO ROSSO DAI RIFERIMENTI

% APPENDICI
\usepackage[toc,page]{appendix}
\newcommand{\nocontentsline}[3]{} % QUESTO COMANDO E QUELLO DOPO SERVONO PER AVERE IL COMANDO \tocless DA METTERE PRIMA DI UNA SEZIONE CHE NON VOGLIO FAR APPARIRE NELL'INDICE
\newcommand{\tocless}[2]{\bgroup\let\addcontentsline=\nocontentsline#1{#2}\egroup}

% FONT
\usepackage[T1]{fontenc}
\usepackage[utf8]{inputenc}
\usepackage[italian]{babel}
\usepackage{latexsym}
\usepackage{textcomp}
\usepackage{siunitx}

% SCRITTA LaTeX
\newcommand{\latex}{\LaTeX\xspace}
\newcommand{\tex}{\TeX\xspace}

% PARAGRAFI, INTERLINEA E MARGINE
\emergencystretch 3em % PER EVITARE CHE IL TESTO VADA OLTRE I MARGINI
\parindent 0ex % TOGLIE INTENDAMENTO PARAGRAFI
% \setlength{\parindent}{4em} % CAMBIA INDENTAMENTO PARAGRAFI
\setlength{\parskip}{\baselineskip} % CAMBIA SPAZIO TRA PARAGRAFI (POSSO METTERE ANCHE 1em)
% \renewcommand{\baselinestretch}{1.5} % CAMBIA INTERLINEA
% \usepackage[margin=1in]{geometry} % CAMBIA MARGINE DEL DOCUMENTO

% HEADER
\usepackage{fancyhdr}
\pagestyle{fancy}
\fancyhead{} % PULISCI HEADER
\fancyfoot{} % PULISCI FOOTER

% TOGLI I PROSISMI DUE COMMENTI PER CAMBIARE DA 'Capitolo X. Blabla' A 'X. Blabla'
%\renewcommand{\chaptermark}[1]{\markboth{#1}{}}
\fancyhead[RE]{\nouppercase{\leftmark}} %\fancyhead[RE]{\thechapter.  \leftmark}
\fancyhead[LO]{\nouppercase{\rightmark}}
\fancyhead[LE,RO]{\thepage}

%\fancyhead[L]{\slshape \MakeUppercase{FISICA SPERIMENTALE 2}}
%\fancyhead[R]{\slshape A cura di Teo Bucci e Giulia Di Giusto}
%\fancyfoot[C]{\thepage}
%\renewcommand{\headrulewidth}{0pt} % CANCELLA LINEA ORIZZONTALE HEAD

% CENTRARE VERTICALMENTE IL TITOLO DELLA PAGINA PRINCIPALE
\usepackage{titling}
\renewcommand\maketitlehooka{\null\mbox{}\vfill}
\renewcommand\maketitlehookd{\vfill\null}

% BOX DI VARIO TIPO

% TEOREMA TIPO 1
\usepackage{mdframed}
\newmdtheoremenv{mdtheo}{Teorema}

% BOX DEFINIZIONI/FORMULE IMPORANTI/TEOREMA TIPO 2
\usepackage{tcolorbox}
\usepackage{blindtext}
\usepackage{tikz,tkz-tab,amsmath}
\tcbuselibrary{theorems}

% FORMULE
%\newtcolorbox{formula}{
%	colback=red!5!white,
%	colframe=red!75!black
%	}
\newtcolorbox{formula}{
	colback=black!5!white,
	colframe=black
	}

% TEOREMA TIPO 2
\newtcbtheorem
	[number within=section]
	{theo}
	{Teorema}
	{
		colback=green!5,
		colframe=green!35!black,
		fonttitle=\bfseries,
		separator sign none % toglie i :
	}
	{th}

% DEFINIZIONI
\newtcbtheorem
	[number within=section] % init options
	{definition} % nome da scrivere in LaTeX
	{Definizione}% Titolo che verrà visualizzazione
	{
		colback=blue!5,
		colframe=blue!45!black,
		fonttitle=\bfseries,
	} % options
	{def} % PREFISSO/LABEL PER LE REF

% CHECKLIST
%% perchèé èé ' "
% $1$ $ 1$ $1 $ $ 1 $
%% pedici fin, in con \text{} tot int a Sigma sup lat
%% \[ \]
%% spazio prima di ,:;
%% spazio equation spazio
%$$ spazio section spazio
%% spazio a inizio riga
%% spazio a fine riga
%% spazio }
%% itemize
%%_{}^{} togliere graffe se possibile
%% spazio prima di |
%% i-esima $i$-esima
%% \vec{ spazio}
%%^\circ\text{C} gradi \degree o^o
%% minuscole dopo il .
%% Terra/terra
%% ne né nè
% n,. ?????
%% . dopo paragraph
%% f.e.m.
%% finora, NON fin ora, NON fin'ora

\numberwithin{equation}{subsection}
\counterwithout{equation}{section}

%%%%%%%%%%%%%%%%%%%%%%%%%%%%%%%%%%%%%%%%%%%%%%%
%%%%%%%%%%%%%%%%%%%%%%%%%%%%%%%%%%%%%%%%%%%%%%%

\begin{document}

%%%%%%%%%%%%%%%%%%%%%%%%%%%%%%%%%%%%%%%%%%%%%%%
%%%%%%%%%%%%%%%%%%%%%%%%%%%%%%%%%%%%%%%%%%%%%%%

% INFORMAZIONI PER LA PAGINA INIZIALE
%\title{Fisica Sperimentale 2}
%\author{Teo Bucci, Giulia Di Giusto}
%\date{\today}

% PAGINA INIZIALE
%\begin{titlingpage} % PER AVERLO CENTRATO
%	\maketitle
%	\thispagestyle{empty} % TOGLI STILE HEADER E FOOTER
%	\clearpage % PASSA ALLA PAGINA DOPO
%\end{titlingpage}

%%%%%%%%%%%%%%%%%%%%%%%%%%%%%%%%%%%%%%%%%%%%%%%
%%%%%%%%%%%%%%%%%%%%%%%%%%%%%%%%%%%%%%%%%%%%%%%

% COPERTINA

\pagestyle{empty} % SWITCHA PER NON AVERE NUMERO PAGINA
\vspace*{\fill}
Teo Bucci \\
Giulia Di Giusto \\

\begin{center}
	{\large \textsc{Appunti di}}\\
	\vspace*{0.4cm}
	{\Huge \textsc{Fisica Sperimentale 2}}

	%{\Huge \textsc{Appunti di Fisica 2}} \\
	%\vspace*{0.4cm}
	%{\large \textsc{Dalle lezioni del prof. Salvatore Stagira}}
\end{center}
\vspace*{\fill}
\newpage

%%%%%%%%%%%%%%%%%%%%%%%%%%%%%%%%%%%%%%%%%%%%%%%
%%%%%%%%%%%%%%%%%%%%%%%%%%%%%%%%%%%%%%%%%%%%%%%

% SECONDA PAGINA

% {\Large \textit{Appunti di Fisica Sperimentale 1}}

\vspace*{\fill}

\textcopyright \ Gli autori, tutti i diritti riservati

Sono proibite tutte le riproduzioni senza autorizzazione scritta degli autori.

Revisione del 20 marzo 2020%\today

Developed by\\
Teo Bucci - \texttt{teobucci8@gmail.com}\\
Giulia Di Giusto - \texttt{digiusto99@gmail.com}

Compiled with \ensuremath\heartsuit \\
%Foto di copertina by Matt Pett (\emph{@mattpunsplash, unsplash.com})

Per segnalare eventuali errori o suggerimenti potete contattare gli autori.

\newpage

%%%%%%%%%%%%%%%%%%%%%%%%%%%%%%%%%%%%%%%%%%%%%%%
%%%%%%%%%%%%%%%%%%%%%%%%%%%%%%%%%%%%%%%%%%%%%%%

% PREFAZIONE

{\Huge \textbf{Prefazione}}

Questo libro raccoglie gli appunti del corso di \emph{Fisica Sperimentale 2} per alunni ingegneri matematici, tenuto al Politecnico di Milano nell'anno accademico 2019-2020.% dal prof. Salvatore Stagira.

Seguendo l'esperienza di altri ragazzi abbiamo deciso di realizzare questo manuale per venire incontro alle esigenze degli studenti, che talvolta non riescono a seguire perfettamente il docente e non trovano una corrispondenza appropriata sul libro di testo, il quale è spesso sovrabbondante di argomenti.

Scrivendo questo libro abbiamo avuto modo di capire meglio alcuni aspetti della materia, cercando di renderli più chiari possibile con gli strumenti che l'ambiente \latex ci ha messo a disposizione. Speriamo che questo stimoli lo studio e l'interesse nei lettori.

Il manuale si articola in vari capitoli che seguono il programma del corso e si conclude con una serie di appendici con alcuni riferimenti utili durante lo studio dei vari argomenti (derivazione, integrazione, costanti fisiche, trigonometria, ecc.).

Ringraziamo tutti coloro che ci hanno supportato.\\
Ringraziamo in special modo tutti i forum e gli utenti che inizialmente ci hanno aiutato a farci strada nei meandri di questo nuovo ambiente di scrittura. È stato faticoso e abbiamo sbattuto la testa davanti a molti ostacoli, ma è stato anche estremamente istruttivo e formativo.\\
Ringraziamo infine il docente per i suoi insegnamenti e la chiarezza con cui ha portato avanti uno dei corsi più interessanti e fondamentali per ogni ingegnere.

\begin{flushright}
\emph{Gli autori} \hspace*{2cm}
\end{flushright}

\newpage

%%%%%%%%%%%%%%%%%%%%%%%%%%%%%%%%%%%%%%%%%%%%%%%
%%%%%%%%%%%%%%%%%%%%%%%%%%%%%%%%%%%%%%%%%%%%%%%

% INDICE

\addtocontents{toc}{\protect\thispagestyle{empty}} % MI ASSICURA CHE NON CI SIA STILE NELL'INDICE, COME ULTERIORE METODO DI SICUREZZA
\tableofcontents % PRODUCE L'INDICE
\newpage % PER TERMINARE SUBITO LA PAGINA DOPO L'INDICE

% FACOLTATIVA PAGINA VUOTA
$ $\\
\newpage

\pagestyle{fancy} % RISWITCHA PER RIAVERE IL NUMERO PAGINA
\setcounter{page}{1} % FA RIPARTIRE IL CONTATORE PAGINA DA 1

%%%%%%%%%%%%%%%%%%%%%%%%%%%%%%%%%%%%%%%%%%%%%%%
%%%%%%%%%%%%%%%%%%%%%%%%%%%%%%%%%%%%%%%%%%%%%%%
%%%%%%%%%%%%%%%%%%%%%%%%%%%%%%%%%%%%%%%%%%%%%%%
%%%%%%%%%%%%%%%%%%%%%%%%%%%%%%%%%%%%%%%%%%%%%%%













% APPENDICE
\appendix



\end{document}
